% TODO : double entrie in media music room index
% mettre aussi Varèse aux côté de Cage dasn le préambule en ce qui concerne l'art des sons organisés
% remplacer appendices par annexes
% tout le jury en entête
% un mot sur Derrida pour la grammatologie

%% LLT: Turn off some annoying warnings...
\RequirePackage{silence}
\WarningFilter{titlesec}{Non standard sectioning command}
\WarningFilter{scrreprt}{Usage of package}
\WarningFilter{scrreprt}{Activating an ugly workaround}

% !TEX TS–program = pdflatexmk

% **************************************************
% Document Class Definition
% **************************************************
\documentclass[%
	paper=A4,					% paper size --> A4 is default in Germany
	twoside=true,				% onesite or twoside printing
	openright,					% doublepage cleaning ends up right side
	parskip=full,				% spacing value / method for paragraphs, set to half for printing
	chapterprefix=true,			% prefix for chapter marks
	11pt,						% font size
	headings=normal,			% size of headings
	bibliography=totoc,			% include bib in toc
	listof=totoc,				% include listof entries in toc
	titlepage=on,				% own page for each title page
	captions=tableabove,		% display table captions above the float env
	draft=false,				% value for draft version
]{scrreprt}%

% \usepackage{draftwatermark}
% \SetWatermarkAngle{55}
% \SetWatermarkLightness{0.9}
% \SetWatermarkScale{1.6}

\usepackage{imakeidx}
\makeindex[name=people,title=Artistes et œuvres cités, intoc]
\indexsetup{level=document}
%\makeindex[name=works,title=Index des œuvres citées]


% for highlight
\usepackage{soulutf8}

% utile pour inclure des figures pdf 1.7, par défaut =5 (pour pdf 1.5)
\pdfminorversion=7

% for text warping around figures
\usepackage{wrapfig}

% for placing figure "exactly here"
\usepackage{float}

\usepackage{breakurl}

% **************************************************
% Debug LaTeX Information
% **************************************************
%\listfiles

% **************************************************
% Load and Configure Packages
% **************************************************
\usepackage[utf8]{inputenc}		% defines file's character encoding
\usepackage[french, english]{babel} % babel system, adjust the language of the content
\usepackage[					% clean thesis style
	figuresep=colon,%
	sansserif=false,%
	hangfigurecaption=false,%
	hangsection=true,%
	hangsubsection=true,%
	colorize=full,%
	colortheme=bluemagenta,%
% LLT: Use biber if using UTF8 encoding
% 	bibsys=bibtex,%
	bibsys=biber,%
	bibfile=./thesis.bib,%
	bibstyle=alphabetic,%
]{cleanthesis}

\bibliography{thesis.bib}

%\usepackage[yyyymmdd,hhmmss]{datetime}

\hypersetup{					% setup the hyperref-package options
	%pdftitle={\thesisTitle},	% 	- title (PDF meta)   (VG : ne fonctionne pas)
	%pdfsubject={\thesisSubject},% 	- subject (PDF meta) (VG : ne fonctionne pas)
	%pdfauthor={\thesisName},	% 	- author (PDF meta) (VG : ne fonctionne pas)
	plainpages=false,			% 	-
	colorlinks=false,			% 	- colorize links?
	pdfborder={0 0 0},			% 	-
	breaklinks=true,			% 	- allow line break inside links
	bookmarksnumbered=true,		%
	bookmarksopen=true			%
}
%\def\UrlBreaks{\do\/\do-}

% for multiple figures
\usepackage{subcaption}

% always break URLs anywhere!
\setcounter{biburllcpenalty}{9000} % lower letters
\setcounter{biburlucpenalty}{9000} % upper letters

% shortcut for siècle
\def\siecle#1{\textsc{\romannumeral #1}\textsuperscript{e}}

% big line in tables
\newcommand\hkrule[1]{\noalign{\hrule height #1}}
\usepackage{tabularx}
\usepackage{diagbox} 

\newlength{\conditionwd}
\newenvironment{conditions}[1][où:]
  {%
   #1\tabularx{\textwidth-\widthof{#1}}[t]{
     >{$}l<{$} @{${}={}$} X@{}
   }%
  }
  {\endtabularx\\[\belowdisplayskip]}

% **************************************************
% tcolorbox to make nice framed texts
% **************************************************

\definecolor{akindofblue}{RGB}{0,101,152} % same color as titles

\usepackage{tcolorbox}% http://ctan.org/pkg/tcolorbox

\newtcolorbox{textbox}{colback=red!5!white,colframe=red!75!black}
\newtcolorbox{titlebox}[1]{colback=akindofblue!5!white,colframe=akindofblue,fonttitle=\bfseries,title=#1}
\newtcolorbox{notebox}[1]{colback=gray!5!white,colframe=gray!75!black, boxrule=1pt,leftrule=10pt,arc=0pt,auto outer arc}

\newtcolorbox{reviewbox}[1]{colback=red!5!white,colframe=red,fonttitle=\bfseries,title=#1}

% todo box
\newtcolorbox{todobox}[1]{colback=red!5!white,colframe=red!75!black,fonttitle=\bfseries,title=#1, left=0pt,right=0pt,top=0pt,bottom=0pt, boxrule=.5pt}
\newcommand\todo[1]{
	\marginpar{
		\begin{todobox}{\flushleft TODO}
			\flushleft\scriptsize\textcolor{black}{#1}
		\end{todobox}
	}
}

\newcommand\Pierre[1]{
		\begin{reviewbox}{Pierre}
			\flushleft\scriptsize\textcolor{black}{#1}
		\end{reviewbox}
}

\newcommand\Hugues[1]{
		\begin{reviewbox}{Hugues}
			\flushleft\scriptsize\textcolor{black}{#1}
		\end{reviewbox}
}


\newcommand\extra[1]{
		\begin{notebox}
			\flushleft\scriptsize\textcolor{black}{#1}
		\end{notebox}
}

%\newcommand{\hl}[1]{{\colorbox{yellow}{\parbox{\textidth}{10cm}}{#1}}}
    %\deffootnote[<width of mark>]
    %   {<indent of footnote text>}
    %   {<paragraph indent in the footnote text>}
    %   {<definition of mark>}

% **************************************************
% Footnotes indentation
% **************************************************

\newcommand*\footnotemarkspace{\normalparindent} % set distance of the footnote text from the margin
\deffootnote{\footnotemarkspace}% use distance from above
  {\parindent}% paragraph indent in footnotes (footnotes should never have paragraphs!)
  {\makebox[\footnotemarkspace][l]{\footfont\phantom{99}\llap{\thefootnotemark}.}} % footfont with period for footnote marks in footnote
\newlength{\normalparindent}
\AtBeginDocument{\setlength{\normalparindent}{\parindent}}


% **************************************************
% custom wrapfig
% **************************************************

\newenvironment{vgwrapfig}
 {%
%  \setlength{\intextsep}{0pt}% <--- Wrong!
  \setlength{\columnsep}{15pt}%
  \wrapfloat{figure}%
 }
 {\endwrapfloat}


% **************************************************
% Glossary of acronyms
% **************************************************
\usepackage[acronym, nogroupskip]{glossaries}

%Alternative glossary for top-level domains
%\newglossary[tlg]{domain}{tld}{tdn}{Top-Level Domains}

\setlength{\glsdescwidth}{0.5\textwidth}
\makenoidxglossaries
\loadglsentries{content/glossary}

% define glossary for Noms propres
%\loadglsentries{content/glossaryNames}
\glstoctrue


%% modsuper glossary display style (pour ne pas avoir de marge à gauche)
%% cf. : https://tex.stackexchange.com/questions/415275/flush-glossaries-super-style-to-left
%\newglossarystyle{modsuper}{%
%	\glossarystyle{super}%
%	\renewenvironment{theglossary}%
%		{\tablehead{}\tabletail{}%
%    	 \begin{supertabular}{@{}lp{\glsdescwidth}}}%<----no margin
%    	{\end{supertabular}}%
%		\renewcommand{\glsgroupskip}{}%
%		\renewcommand*{\glossaryentryfield}[5]{%
%			\glsentryitem{##1}\glstarget{##1}{##2} & ##3\glspostdescription\space ##5\\[2pt]}%nogroupskip
%}
\definecolor{oxfordblue}{rgb}{0.0, 0.13, 0.28}
\definecolor{mediumpersianblue}{rgb}{0.0, 0.3, 0.5}
\definecolor{royalblue}{rgb}{0.0, 0.14, 0.4}
\definecolor{prussianblue}{rgb}{0.0, 0.2, 0.4}
\definecolor{cerulean}{rgb}{0.0, 0.48, 0.65}
\definecolor{myBlue}{rgb}{0.0, 0.3, 0.5}
%\definecolor{glscolor}{rgb}{0.1, 0.4, 0.6} %kinda darkcerulean
\definecolor{glscolor}{rgb}{0.4, 0., 0.} %kinda darkcerulean

\renewcommand*{\glstextformat}[1]{\textcolor{glscolor}{#1}}



% **************************************************
% Annexes
% **************************************************
\usepackage[toc,page]{appendix}

% **************************************************
% TODOs
% **************************************************
\usepackage{marginnote}
\newcommand\note[1]{\textcolor{blue}{\scriptsize#1}}
\newcommand\unsure[1]{\marginpar{\scriptsize\textcolor{magenta}{#1}}}
\newcommand\info[1]{\marginpar{\scriptsize\textcolor{ForestGreen}{#1}}}
%\newcommand\todo[1]{\marginpar{\flushleft\scriptsize\textcolor{red}{TODO:\\#1}}}

%\newcommand\test[1]{\noindent\fbox{\parbox{\textwidth}{#1}}}

% **************************************************
% Inline quotes style
% **************************************************
\DeclareQuoteStyle{inlinequote}
    {\itshape\textquotedblleft}
    [\textquotedblleft]
    {\textquotedblright}
    	[0.05em]
    {\textquoteleft}
    {\textquoteright}

\newcommand{\iquote}[1]{{\setquotestyle{inlinequote}\enquote{#1}}}
 
% **************************************************
% Information and Commands for Reuse
% **************************************************
\newcommand{\thesisTitle}{Représentation et contrôle\\dans le design interactif des\\instruments de musique numériques}
\newcommand{\thesisTitleCaps}{REPRÉSENTATION ET CONTRÔLE\\DANS LE DESIGN INTERACTIF DES\\INSTRUMENTS DE MUSIQUE NUMÉRIQUES}
\newcommand{\thesisName}{Vincent Goudard}
\newcommand{\thesisSubject}{Ecole doctorale SMAER (ED 391)}
\newcommand{\thesisDate}{2019}
\newcommand{\thesisVersion}{\today}
%\newcommand{\thesisVersion}{\today\ — \currenttime}

\newcommand{\thesisFirstReviewer}{Myriam Desainte-Catherine}
\newcommand{\thesisFirstReviewerPosition}{professeure}
\newcommand{\thesisFirstReviewerUniversity}{\protect{Bordeaux INP}}
\newcommand{\thesisFirstReviewerDepartment}{Laboratoire Bordelais de Recherche en Informatique (LaBRI)}
\newcommand{\thesisFirstReviewerJury}{rapporteuse}

\newcommand{\thesisSecondReviewer}{Laurent Pottier}
\newcommand{\thesisSecondReviewerPosition}{professeur}
\newcommand{\thesisSecondReviewerUniversity}{\protect{Université Jean Monnet, Saint-Étienne}}
\newcommand{\thesisSecondReviewerDepartment}{Centre Interdisciplinaire d'Etudes et de Recherches sur l'Expression Contemporaine (CIEREC)}
\newcommand{\thesisSecondReviewerJury}{rapporteur}

\newcommand{\thesisThirdReviewer}{Marcelo M. Wanderley}
\newcommand{\thesisThirdReviewerPosition}{professeur}
\newcommand{\thesisThirdReviewerUniversity}{\protect{Université McGill, Montréal}}
\newcommand{\thesisThirdReviewerJury}{examinateur}

\newcommand{\thesisFourthReviewer}{Brigitte d'Andréa-Novel}
\newcommand{\thesisFourthReviewerPosition}{professeure}
\newcommand{\thesisFourthReviewerUniversity}{Sorbonne Université}
\newcommand{\thesisFourthReviewerJury}{examinatrice, présidente du jury}

\newcommand{\thesisFirstSupervisor}{Pierre Couprie}
\newcommand{\thesisFirstSupervisorPosition}{maître de conférences}
\newcommand{\thesisFirstSupervisorUniversity}{Sorbonne Université}

\newcommand{\thesisSecondSupervisor}{Jean-Dominique Polack}
\newcommand{\thesisSecondSupervisorPosition}{professeur}
\newcommand{\thesisSecondSupervisorUniversity}{Sorbonne Université}

\newcommand{\thesisFirstAdvisor}{Hugues Genevois}
\newcommand{\thesisFirstAdvisorPosition}{ingénieur de recherche}
\newcommand{\thesisFirstAdvisorUniversity}{Sorbonne Université}

\newcommand{\thesisUniversity}{\protect{Sorbonne Université}}
\newcommand{\thesisUniversityDepartment}{Équipe Lutherie Acoustique Musique - Institut $\partial$'Alembert}
\newcommand{\thesisUniversityInstitute}{Institut de Recherche en Musicologie (IReMus)}
\newcommand{\thesisUniversityCollegium}{Collegium Musicæ}

%\newcommand{\thesisUniversityGroup}{Collegium Musicæ}
\newcommand{\thesisUniversityCity}{PARIS}
\newcommand{\thesisUniversityStreetAddress}{4, place Jussieu}
\newcommand{\thesisUniversityPostalCode}{75252}


\graphicspath{{./gfx/}} %Where the figures folder is located

% do not indent each first paragraph after a \section
% cf. https://latex.org/forum/viewtopic.php?t=23250
% \titlespacing{\command}{left}{before-sep}{after-sep}[right-sep]
%\titlespacing*{\section}{0em}{.75em}{.2em}[0pt]
%\titlespacing*{\subsection}{0em}{.75em}{.2em}[0pt]
%\titlespacing*{\subsubsection}{-1.5em}{0em}{-1.5em}[0pt]

\usepackage[tight]{shorttoc}

% **************************************************
% Document CONTENT
% **************************************************
\begin{document}

% (un)comment this to switch from/to dark mode
%\pagecolor{gray!40}

% --------------------------
% rename document parts
% --------------------------
%\renewcaptionname{ngerman}{\figurename}{Abb.}	
%\renewcaptionname{ngerman}{\tablename}{Tab.}
\renewcaptionname{english}{\figurename}{Fig.}
\renewcaptionname{english}{\tablename}{Tab.}

% --------------------------
% Front matter
% --------------------------
\pagenumbering{roman}			% roman page numbing (invisible for empty page style)
\pagestyle{empty}				% no header or footers

% !TEX root = ../thesis-example.tex
%
% ------------------------------------  --> cover title page
\begin{titlepage}
	\pdfbookmark[0]{Couverture}{Couverture}
	\flushright
	\hfill
	\vfill
	{\LARGE\thesisTitle \par}
	\rule[5pt]{\textwidth}{.4pt} \par
	{\Large\thesisName}
	\vfill
	\textit{\large\thesisDate} \\
	%Version: \thesisVersion
\end{titlepage}


% ------------------------------------  --> main title page
\begin{titlepage}
	\pdfbookmark[0]{Page de titre}{Page de titre}
	\tgherosfont
	\centering

	{\Large \thesisUniversity} \\[4mm]
	\includegraphics[width=6cm]{gfx/SU-logo.png} \\[2mm]
	\textsf{\thesisUniversityDepartment} \\
	\textsf{\thesisUniversityInstitute} \\
	%\textsf{\thesisUniversityGroup} \\

	\vfill
	{\large \thesisSubject} \\[5mm]
	{\LARGE \color{ctcolortitle}\textbf{\thesisTitle} \\[10mm]}
	{\Large \thesisName} \\

	\vfill
	\begin{minipage}[t]{.27\textwidth}
		\raggedleft
		\textit{1. Rapporteur}
	\end{minipage}
	\hspace*{15pt}
	\begin{minipage}[t]{.65\textwidth}
		{\Large \thesisFirstReviewer} \\
	  	{\small \thesisFirstReviewerDepartment} \\[-1mm]
		{\small \thesisFirstReviewerUniversity}
	\end{minipage} \\[5mm]
	\begin{minipage}[t]{.27\textwidth}
		\raggedleft
		\textit{2. Rapporteur}
	\end{minipage}
	\hspace*{15pt}
	\begin{minipage}[t]{.65\textwidth}
		{\Large \thesisSecondReviewer} \\
	  	{\small \thesisSecondReviewerDepartment} \\[-1mm]
		{\small \thesisSecondReviewerUniversity}
	\end{minipage} \\[10mm]
	
	% supervisors
	\begin{minipage}[t]{.27\textwidth}
		\raggedleft
		\textit{Direction}
	\end{minipage}
	\hspace*{15pt}
	\begin{minipage}[t]{.65\textwidth}
		\thesisFirstSupervisor\ et \thesisSecondSupervisor
	\end{minipage} \\[0mm]
	
	\begin{minipage}[t]{.27\textwidth}
		\raggedleft
		\textit{Co-encadrement}
	\end{minipage}
	\hspace*{15pt}
	\begin{minipage}[t]{.65\textwidth}
		\thesisFirstAdvisor
	\end{minipage} \\[10mm]

	\thesisDate \\

\end{titlepage}


% ------------------------------------  --> lower title back for single page layout
\hfill
\vfill
{
	\small
	\noindent\textbf{\thesisName} \\
	\textit{\thesisTitle} \\
	\thesisSubject, \thesisDate \\
	Rapporteur: \thesisFirstReviewer\ et \thesisSecondReviewer \\
	Rapporteur: \thesisFirstSupervisor\ et \thesisSecondSupervisor \\
	Co-encadrant: \thesisFirstAdvisor \\[1.5em]
	\textbf{\thesisUniversity} \\
	%\textit{\thesisUniversityGroup} \\
	\thesisUniversityInstitute \\
	\thesisUniversityDepartment \\
	\thesisUniversityStreetAddress \\
	\thesisUniversityPostalCode\ \thesisUniversityCity
}
		% INCLUDE: all titlepages

%\cleardoublepage

\pagestyle{plain}				% display just page numbers
% !TEX root = ../thesis-example.tex
%
\pdfbookmark[0]{Resumé}{Abstract}
\chapter*{Resumé}
\label{sec:abstract}
\vspace*{-10mm}

\noindent Les instruments de musique numériques se présentent comme des objets complexes, qui se situent à la fois dans une continuité historique avec l'histoire de la lutherie tout en étant marqués par une rupture forte provoquée par le numérique et ses conséquences en terme de possibilités sonores, de relations entre le geste et le son, de situations d'écoute, de re-configurabilité des instruments, etc. Ce travail de doctorat propose une analyse des caractéristiques émanant de l'intégration du numérique dans les instruments de musique, en s'appuyant notamment sur une réflexion musicologique, sur des développements logiciels et matériels et sur pratique musicale, ainsi que sur des échanges avec d'autres musiciens, luthiers, compositeurs et chercheurs.

\vspace*{20mm}

{\usekomafont{chapter}Abstract}\label{sec:abstract-diff} \\

\noindent Digital musical instruments come as complex objects, which are both in historical continuity with the history of lutherie while being marked by a strong rupture caused by digital technology and its consequences in terms of sound possibilities, relations between gesture and sound, listening situations, re-configurability of instruments, etc. This thesis proposes an analysis of the characteristics resulting from the integration of digital technology into musical instruments, supported by musicological considerations, software and hardware developments and musical practice, as well as exchanges with other musicians, luthiers, composers and researchers.		% INCLUDE: the abstracts (english and french)

%\clearpage %

% !TEX root = ../thesis-example.tex
%
\chapter*{Remerciements}
\label{sec:acknowledgement}
\vspace*{-10mm}
\pdfbookmark[0]{Remerciements}{Remerciements}


% Je ne pourrai malheureusement pas remercier ici toutes les personnes qui m'ont aidé à réaliser ce travail, dont les origines remontent trop loin pour qu'elles soient vraiment 


Je tiens à remercier chaleureusement toutes les personnes qui m'ont permis de réaliser ce travail. Elles sont malheureusement bien trop nombreuses pour être toutes citées ici, et je me restreindrai aussi à celles et ceux qui ont plus directement soutenu cette recherche.

Je remercie en premier lieu mes directeurs de thèse Jean-Dominique Polack, Pierre Couprie et tout particulièrement mon ``co-encadrant'' Hugues Genevois --~bien étrange titre pour quelqu'un qui se plait tant à faire sortir la pensée hors des cadres établis. Merci à tous les trois de m'avoir laissé errer dans des directions diverses, avec une méthodologie parfois plus proche de l'improvisation que de la partition, tout en restant bienveillant à mon égard.

% Merci Hugues, ta confiance tout autant que ton ouverture d'esprit ont été des soutiens 

Également, je remercie chaleureusement le Collegium Musicæ, notamment Cécile Davy-Rigaux, Benoit Fabre et plus particulièrement Agnès Puissilieux, dont j'ai eu le plaisir de partager le bureau durant mes présences au LAM. Au delà du soutien financier que m'a apporté le Collegium en m'accordant sa confiance pour ce travail, l'implication aux séminaires rassemblant des chercheu.rs.ses de ses différentes composantes ont été, chaque fois, des moments de rencontre très enrichissants dans cette \ul{poursuite} d'inter-disciplinarité. 

C'est bien souvent au détour d'une conversation, si anecdotique parût-elle, que des connexions s'opèrent que l'on cherchait vainement ailleurs, ou que des théories apparemment bien ancrées trouvent leur failles et laissent passer un peu de lumière. C'est un plaisir, en tant que chercheur, d'avoir pu travailler dans un tel contexte où l'inter-disciplinarité est promue et encouragée.

Cette inter-disciplinarité n'est pas qu'un terme à la mode, mais se construit à travers des rencontres que le Collegium Musicæ suscite.

% Catherine Pélachaud pour sa bienveillance lors du suivi de cette thèse.

Mes remerciements vont également à mes collègues de ONE, pour le plaisir à jouer avec eux autant que pour ce qu'ils ont pu apporter aux réflexions et à l'élaboration de certains outils comme ``John'' : Pierre et Hugues qui outre le fait d'encadrer ce travail de recherche ont été des partenaires de jeu, Laurence Bouckaert, Jean Haury, György Kurtág Jr. dont la rencontre il y a maintenant plus de quinze ans a été décisive dans le fait de poursuivre certaines intuitions, et Serge de Laubier auprès de qui les quelques années passées à Puce Muse m'ont permis

 d'apprécier son talent à transposer les lutheries numériques, les musiques expérimentales et les questions théoriques qu'elles posent, sur des terrains très concrèts, d'un spectacle de rue un soir de décembre sous la pluie, ou d'une groupe d'enfants dont l'impatience tourne à fréquence plus élevée que le CPU des ordinateurs high-tech.

Je remercie également chaleureusement ceux qui m'ont généreusement offert leur temps et leurs idées durant les entretiens : Nicolas Bernier, Nicolas Collins, Serge De Laubier, François Dumeaux, Adrien Mamou-Mani, Luca Turchet, Bruno Zamborlin, Patrick Saint-Denis et José-Miguel Fernandez.
Merci également à celles et ceux avec qui les discussions informelles ont nourri ce travail, même si ces discussions n'y figurent pas explicitement : Marcelo Wanderley, Caroline Traube, Bernard Sève, Pascale Criton, ainsi qu'aux membres du LAM, en particulier le groupe nouvelles lutheries: Gabriela Patiño-Lakatos, Christophe d'Alessandro, Boris Doval, Michèle Castellengo, Claudia Fritz, Jean-Loïc Le Carrou, Louise Condi, Xiao Xiao, Grégoire Locqueville.

les membres du groupe nouvelles-lutheries au LAM : 

Pour l'accès à toutes les ressources auxquelles il est parfois difficile d'accéder dans un monde académique scindés en spécialités : Dušan Barok, Alexandra Elbakyan.


% Je remercie également les chercheurs et chercheuses du LAM, mon laboratoire d'accueil,  qui m'a accueilli durant mes venues occasionelles, 
, Cécile Babiole

Je tiens à remercier chaleureusement les ami.e.s qui m'ont accueillis durant mes nombreux aller-retours à Paris : Ingrid, Guillaume, Karolina, Hugues, Guillaume, Hilla, Gaëlle, Ignazio, Fabrizia, Jérome, Agathe. 

Bertrand Gibert, pour le volume de Leroi-Gourhan qu'il m'a offert et que j'ai bien usé!

% Hugues Genevois, Pierre Couprie, Jean-Dominique Polack, Catherine Pélachaud, LAM

% Agnès Puissilieux et CM

% Interviewés (Nic Collins, Adrien Mamou-Mani, Bruno Zamborlin, François Dumeaux, Serge De Laubier, György Kurtág Jr., Luca Turchet, Nicolas Bernier, Patrick Saint Denis, Marcello Wanderley, Caroline Traube,  Bernard Sève, José-Miguel Fernandez, Rémy Muller, Thor Magnusson, ...)

% Tous ceux rencontrés et collaborations (sans citer explicitement).

% Les ami.e.s qui m'ont hébergés durant mes passages à Paris durant toutes ces dernières années, en particulier Guillaume avec qui les discussions ont nourri

% G, A, O.

% et pour l'accès à toutes les ressources auxquelles il est parfois difficile d'accéder dans un monde académique scindés en spécialités : Dušan Barok, Alexandra Elbakyan.

A mes parents, pour leur amour et soutien inconditionnel depuis le premier jour.
À Olga et Anatole, pour leur patience autant que pour leur impatience et leur curiosité stimulante.
A Gladys Brégeon.
 % INCLUDE: acknowledgement

\cleardoublepage
\selectlanguage{french}

%\shorttoc{Sommaire}{1}

\setcounter{tocdepth}{2}		% define depth of toc
\pdfbookmark{Table des matières}{contents} % put toc in pdf bookmarks (but not in TOC)
\tableofcontents				% display table of contents

%\cleardoublepage%

% --------------------------
% Body matter
% --------------------------
\pagenumbering{arabic}			% arabic page numbering
\setcounter{page}{1}			% set page counter
\pagestyle{maincontentstyle} 	% fancy header and footer
%\clearpage


%\part{Partie I} 
% !TEX root = ../thesis-example.tex
%
\chapter{Introduction}
\label{ch:introduction}
%
\cleanchapterquote{L'instrument est un compromis instable entre des qualités non-convergentes.}{Bernard Sève}{(L'instrument de musique: une étude philosophique \cite{seve_instrument_2013})}

\Pierre{ l'introduction doit présenter le titre : représentation, contrôle, desgin interactif, instrument de musique et instrument de musique numérique (pas forçément dans cet ordre)}

\section{Préambule}

\Pierre{ je pense que tu devrais ici repartir du début : tes motivations pour ce sujet et quels sont les notions que tu dois absolument présenter pour que l'on comprenne ta problématique. Typiquement, une intro de thèse = contexte -> problématique/hypothèses -> annonce du plan}

J'ai commencé à étudier un ``vrai'' instrument de musique—le saxophone—en école de musique à l'âge de 12 ans, mais j'ai réalisé des années plus tard que le premier instrument de musique que j'avais pratiqué avait cinq touches et deux contrôleurs continus: REC, PLAY, STOP, RWD et FWD, un contrôle de volume et un sélecteur de fréquence FM. Né l'année de l'invention du walkman, j'ai eu rapidement entre les mains cet objet que peu de monde appelerait ``instrument de musique'', mais qui permettaient pourtant de jouer des sons, des sons venant d'ailleurs ou des sons personnels, enregistrés à la main.\\
J'ai créé durant ces 15 dernières années diverses sortes d'applications, d'instruments, d'outils dans différents contextes : spectacle vivant, installations multimédia, ateliers pédagogiques, expositions muséographiques, émissions de radio, projets de recherche, etc. Il est sûrement vain de vouloir discriminer dans ces objets lesquels constituent des instruments de musique et lesquels n'en sont pas, mais il me semble intéressant de constater que ces développements posent à chaque fois, sous différents angles, la question du rapport de musicalité avec la machine.
C'est ce constat qui m'a amené à réfléchir sur la notion d'instrument numérique pour essayer d'en cerner les contours du moins d'en percevoir les points de fuite.

Poser la question de la représentation des instruments de musique, à une époque où l’objet a volé en éclat ainsi que les traditions musicales qu’il soutend, pose imanquablement la question des motivations pour lesquelles nous construisons des instruments, des raisons pour lesquelle nous inventons, pratiquons, écoutons la musique. De cela découlent les multiples manières dont nous jouons avec le réel, avec les objets, avec les sons pour produire cet étrange —et pourtant si familier— phénomène de musique.

La notion d’instrument de musique numérique embrasse des problématiques complexes, sur le plan technique, mais également sur le plan esthétique et sociologique. La façon dont nous créons la musique et la manière dont nous l’écoutons a tellement changé en un siècle qu’il semble désuet de tenter de l’aborder sur un plan purement technique, tant celle-ci semble promise à bouleverser encore davantage nos usages dans l’avenir.

Pourquoi jouons nous de la musique ?
Même s’il semble impossible d’apporter une réponse simple à cette question, il faut donc prendre en compte celle-ci 


Ce travail de recherche a commencé après plus de 15 ans passés à concevoir, fabriquer, programmer, pratiquer et écouter des instruments de musique à l'aide d'ordinateurs. 


\section{Une thèse interdisciplinaire}

En se situant entre les domaines relativement distincts des sciences et de l'ingénierie d'une part et de la musicologie d'autre part, cette thèse est la tentative d'une étude des instruments de musique numériques prenant ces deux dimensions en compte.
Usages, formats et conférences relativement différentes et séparées dans ces domaines. 
Evolution vers une interdisciplinarité nécessaire à la compréhension mutuelle.

+ ajouter un mot sur le Collegium Musicæ.

\section{Problématique}

\Pierre{ Je ne vois pas de problématique présentée. La problématique est la question que tu poses dans ta thèse ainsi que les questions annexes ou sous-questions.}

Les instruments de musique ont la particularité de pouvoir s'envisager sous ces deux aspects et bien qu'il soit possible de ne s'intéresser qu'à l'un des deux, nous croyons fortement qu'une étude des motivations qui poussent à leur design ne saurait faire abstraction des conditions particulières de leur inscription dans le domaine socio-culturel.

\Pierre{ "domaine socio-culturel" : es-tu sûr de vouloir te placer sur un niveau sociologique ? }

Les instruments de musique sont des instrument pour faire de la musique (ou pour musiquer, dirait Christopher Small \cite{small_musicking:_1998}). Cette apparente évidence est nécessaire pour signaler qu'il ne s'agit pas simplement de faire des notes, ou même du son. La musique implique également notre \textit{mémoire du son}, l'imagination que nous en avons, les aspects visuels qui se rattachent à la notion de musicalité, et d'autres dimensions esthétiques et culturelles. \todo{être plus précis}

La performance musicale a cela de particulier qu'elle ne possède pas de cahier des charges préalables (la partition ne saurait être considérée comme telle!) et que loin de se plier à la nécessité d'exécuter une tâche précise, comme il pourrait être le cas dans le design d'autres interfaces homme-machine, les instruments sont des objets techniques dont les musiciens abusent (plus qu'ils en usent), dont les artefacts peuvent être appréciables et souhaitables, dont la compréhension n'est pas un préalable requis pour leur utilisation, pas davantage que leur fiabilité n'est garante d'une performance musicale intéressante.
%
Le design des DMI, ainsi que le design des outils-mêmes du luthier numérique, doivent être informés de ces particularités propres à la création artistique si l'on souhaite qu'ils se prêtent à la création de musiques nouvelles et à l'exploration de territoires sonores inexplorés.

Nécessité de prendre en compte la part expérientielle de la performance musicale, notamment dans sa dimension subversive.

\section{Enjeux et hypothèses}

Enjeu de trouver des caractéristiques transversales dans les lutheries numériques malgré l'absence de tradition, de répertoire, de notation, de méthode d'apprentissage, etc.
Enjeu de confronter au réel des réalisations instrumentales et logicielles à travers une pratique musicale.

Ce travail de recherche s'offre donc comme une présentation "en coupe" d'un travail de lutherie, dans ce qu'il comporte de réflexions, de choix de matériaux, d'assemblages, de programmation, de notations, de pratiques et comment ces différents aspects interfèrent dans le cas particulier des instruments intégrant le numérique dans le design de leur interaction.

\subsubsection*{Hypothèse 1 : l'instrument atomisé, recomposé}

L'instrument est atomisé et se retrouve configuré comme un agencement modulaire évolutif qui se cristallise ponctuellement dans des instances contextuelles. \\
Pas d'instrument standard qui sorte du lot, si ce n'est pour imiter l'existant (le clavier, la guitare, etc.), mais des protocoles et modules qui deviennent standards et interconnectables.

\subsubsection*{Hypothèse 2 : l'instrument est subversif}

\Pierre{ le subversif est une très bonne idée mais c'est un terme tellement chargé de sens que tu dois en dire un peu plus et notamment dans quel sens tu le prends.}

La relation instrumentale n'est pas de même nature que la relation des HCI.
Les instrumentistes ne sont pas des \textit{interface users} mais plutôt des \textit{interface abusers}.
La relation entre le geste et le son n'est pas nécessairement faite pour être lisible et comprise du public, le musicien est un magicien.
L'œil augmente l'écoute (et la subvertit).


\subsubsection*{Hypothèse 3 : Le continu et le discret}

Si la continuité de la vibration physique semble être une donnée consitutive des instruments acoustique, qui recrééent artificiellement des espaces discrets (tels que les échelles harmoniques et rythmiques), le domaine du numérique part d'une certaine manière dans la direction opposée. Les instruments numériques sont caractérisé par la nature discrète (et même binaire) de l'encodage symbolique sous-jacents aux données traitées. Il s'agit donc davantage de pouvoir retrouver une continuité dans ce monde discret.\\
Cette bipolarité du continu et du discret traverse ainsi, à des degrés variés, les questions de design qui se présentent dans la conception des instruments numériques, que cela soit au niveau de l'encodage du geste capté ou de celui de la synthèse audio. Les développements présentés dans ce travail sont orientés par les possibilités de passage fluide du continu au discret et inversement, animés par la conviction qu'une partie du jeu musical se joue dans cette ambivalence.


\section{Interviews}

Une caractéristique notable des lutheries numériques est leur diversité et le foisonnement d'approches, de propositions, de positions prises par ceux qui les inventent et les pratiquent. Un certain nombre d'entretiens ont été menées durant ce travail de thèse afin d'élargir le champ de la réflexion à différentes approches sur les instruments de musique numérique. Ces entretiens sont reproduits intégralement en annexe, accompagnés d'une brève biographie présentant les personnes ayant acceptés de présenter leur travail et leurs réflexions.

Ces interviews ont pris la forme de discussions libres, orientées par un certain nombre de questions, dont la première était invariablement : "quelle a été la motivation originale qui vous a poussé à concevoir et utiliser des DMI ?". La suite de la discussion dépendait ensuite de l'interlocuteur, leurs projets étant relativement différents entre ceux d'entrepreneurs et ceux d'artistes. %On y trouve cependant quelques idées transversales qui ont contribuées à nourrir ma propre réflexion.

\todo{reproduire le guide d'interview en annexe}

Liste des personnes interviwées (et liens vers annexes) :

\vspace{-1em}
\begin{itemize}[noitemsep]
\item \textbf{\hyperref[appendix:bernier]{Nicolas Bernier}}, artiste canadien créant des installations et performances audio-visuelles, et enseigne la "musique numérique" à l'Université de Montréal;
\item \textbf{\hyperref[appendix:collins]{Nicolas Collins}}, compositeur, artiste sonore, professor au département son  à la School of the Art Institute de Chicago 1999 et auteur notable du livre "Handmade Electronic Music –The Art of Hardware Hacking";
\item \textbf{\hyperref[appendix:dumeaux]{François Dumeaux}}, musicien et compositeur de musiques électro-acoustiques;
\item \textbf{\hyperref[appendix:delaubier]{Serge De Laubier}}, musicien, inventeur du Méta-Instrument, directeur artistique de Puce Muse;
\item \textbf{\hyperref[appendix:fernandez]{Jose-Miguel Fernandez}}, compositeur 
%\item \textbf{\hyperref[appendix:kurtag]{György Kurtag Jr.}}, musicien improvisateur, compositeur, pédagogue...
\item \textbf{\hyperref[appendix:mamou-mani]{Adrien Mamou-Mani}}, chercheur et co-fondateur de HyVibes, startup créant des instruments augmentés tels la \textit{SmartGuitar};
\item \textbf{\hyperref[appendix:saint-denis]{Patrick Saint-Denis}}, compositeur, luthier numérique, enseigne à l'Université de Montréal.
\item \textbf{\hyperref[appendix:turchet]{Lucas Turchet}} (b. 1982), designer sonore, musicien, compositeur et écrivain, co-fondateur de Mind Music Labs, startup créant des instruments augmentés.
\item \textbf{\hyperref[appendix:zamborlin]{Bruno Zamborlin}} (b. 1984), fondateur et CEO de Mogees et HyperSurfaces. 
\end{itemize}



\section{Contributions de cette thèse}

Cette thèse propose plusieurs contributions théoriques dans le domaine de la recherche sur les \glspl{DMI}, ainsi que plusieurs contributions pratiques sous la forme de \glspl{LogicielLibre} et disponibles sur le web.

Les contributions théoriques concernent :
\vspace{-1em}
\begin{itemize}[noitemsep]
\item des perspectives sur la nature des \gls{DMI}, leur cycle de vie, la notion d'assemblage éphémère et ses conséquences sur leur design;
\item la caractérisation du geste musicale, en particulier sa part subversive et son inscription dans le design de l'instrument;
\end{itemize}

Les contributions pratiques sont les suivantes :
\vspace{-1em}
\begin{itemize}[noitemsep]
%\setlength\itemsep{-1.5em}
\item \textbf{LAM-lib} : un package pour le logiciel Max proposant une collection d'algorithmes utiles pour la lutherie numérique;
\item \textbf{MP} : un protocole de communication pour le contrôle de la synthèse, venant palier un certain nombre de limitations rencontrées dans le protocole MIDI, ainsi qu'un package Max rassemblant un certain nombre d'objets supportant ce protocole;
\item \textbf{mp.TUI} : un package pour Max permettant la création d'interfaces graphique tangibles personnalisables et polyphoniques, basées sur le protocole MP;
\item \textbf{sagrada} : un package Max de synthèse granulaire modulaire contrôlé par signal;
\item \textbf{John} : un logiciel pour la (semi-) composition et conduite d'improvisation électroacoustique, éditable collectivement;
\end{itemize}


\section{Structure de la thèse}
\label{sec:preamble:structure}

\textbf{Chapitre \ref{ch:introduction}} \\[0.2em]
Vous êtes ici.

\textbf{Chapitre \ref{ch:ephemeral}} \\[0.2em]
Le chapitre \ref{ch:ephemeral} présente un certain nombre de considérations sur les instruments de musique numériques et le contexte de leur utilisation. En particulier, la nature éphémère des assemblages modulaires, souvent ignorée, est ici considérée comme une des caractéristiques essentielles venant influencer leur design. Un distribution entre répertoire, musicien et contexte permet de re-définir la façon dont s'articulent ces différents pôles ainsi à l'œuvre dans la création et l'évolution des DMIs.

\textbf{Chapitre \ref{ch:transparency}} \\[0.2em]
Le chapitre \ref{ch:transparency} vient questionner la notion de geste musical dans le cas de la pratique avec des DMIs. En particulier, la lisibilité du geste et de sa relation à la synthèse sonore, souvent considérée comme un critère de design souhaitable, y est remise en question en prenant en compte les fins subversives de l'art musical. 

L'étude des artefacts qui en résultent et viennent bouleverser la perception de continuité(s) permet d'introduire la notion de morpho-dynamisme des DMIs, son intérêt dans la création et la pratique musicale et son intégration dans le corps de l'instrument.

\textbf{Chapitre \ref{ch:interfaces}} \\[0.2em]
Le chapitre \ref{ch:interfaces} présente une exemple particulier d'interface instrumentale et retrace l'histoire de son évolution à travers plusieurs générations, partant d'une interface standard et disponible dans le commerce (la tablette graphique) et évoluant vers une personnalisation et un enrichissement du dispositif. 
Seront discutées les raisons motivant l'ajout de capteurs, l'organisation de l'espace de jeu, la polyphonie des sources sonores, etc.

\textbf{Chapitre \ref{ch:algorithms}} \\[0.2em]
Le chapitre \ref{ch:algorithms} présente des développement réalisés pour la conception du "mapping" de l'instrument en tentant notamment de répondre aux problématiques soulevées dans les chapitres \ref{ch:ephemeral} et \ref{ch:transparency}. Sont présentés dans ce chapitre les concepts de modèle intermédiaire, ainsi qu'un protocole de contrôle expressif polyphonique, nommé "MP", permettant la communication entre interfaces, modules de transformation et de synthèse. Une extension des idées de MP dans le domaine du signal et appliqué à la synthèse granulaire, nommé Sagrada, est également présenté.

\textbf{Chapitre \ref{ch:visual_representation}} \\[0.2em]
Le chapitre \ref{ch:visual_representation} présente un système d'interface graphique tangible (TUI) basée sur le protocole présenté au chapitre  \ref{ch:algorithms}, afin de permettre notamment une reconfiguration dynamique de l'interface de jeu et une représentation graphique des processus utilisés pour la performance musicale. Ces interfaces graphiques permettent également d'intégrer des éléments de représentation musicale (forme d'onde, échelle, motifs rythmiques, etc.) ou non-musicale comme composants interactifs pour le contrôle expressif.

\textbf{Chapitre \ref{ch:notation}} \\[0.2em]
Le chapitre \ref{ch:notation} présente des travaux portant sur la notation musicale dans le domaine de la performance électroacoustique utilisant des DMI. Les questions de composition collective, d'édition collaborative et d'écologie de l'attention sont abordées et sont mises en œuvre dans "John, the semi-conductor", un logiciel permettant la génération automatique et l'édition collective de partitions minimales, utilisé dans l'ensemble d'improvisation électroacoustique ONE. 

\textbf{Chapitre \ref{ch:conclusion}} \\[0.2em]
Pistes de recherches à suivre...


\section*{extra material}
 % INCLUDE: preamble
% !TEX root = ../thesis-example.tex
%
\chapter{Instances éphémères d'agencements modulaires}
\label{ch:ephemeral}

% \cleanchapterquote{Paradoxalement, la musique du futur est écrite sur du sable!}{Michel Chion}{La musique du futur a-t-elle un avenir?, 1977}

%\cleanchapterquote{Le vieux Paris n'est plus (la forme d'une ville\\
%Change plus vite, hélas ! que le coeur d'un mortel)}{Charles Baudelaire}{Le Cygne, 1861}

\cleanchapterquote{La musique (...) est trop en deça du monde et du désignable pour figurer autre chose que des épures de l'Être, son flux et son reflux, sa croissance, ses éclatements, ses tourbillons.}{Maurice Merleau-Pontry}{L'Œil et l'esprit, 1964} %\cite{merleau-ponty_loeil_1964}

\cleanchapterquote{Dufourt suggests that contemporary music highlights what was rejected in the Greek world : it rather captures the evanescent, the ephemeral, the ambivalent, the Erebus, it favors the endless metamorphosis of qualities and forms; as Nietzsche proclaimed, western music tends toward the liberation of the dyonisiac dimension and the acceptance of the inacceptable part of myths.}{Jean-Claude Risset}{Discours invité à la conférence ICMC, 2014}%\cite{risset_sound_2014} % Remettre cette citation dans le corps du texte.


%\test{blabla}
\section{Paysage des DMIs}
\label{sec:ephemerality:landscape}

%\todo{traduire ce chapitre!}

%--------------------------------------------------------------
\subsection{Les origines}

\subsubsection{Pré-histoire}

Les instruments acoustiques bénéficient d'une histoire vieille de plus de 40000 ans et la finesse de leur fabrication a atteint une excellence qui fait de certains instruments de véritables pièces d'orfèvres. Les instruments numériques sont beaucoup plus récents et ne peuvent rivaliser avec ce degré de raffinement. Pour autant, il ne sont pas totalement déshérités de la tradition et du savoir-faire des instruments acoustiques et par ailleurs, le développement hautement collaboratif à l'œuvre dans le domaine de la programmation informatique fait qu'ils sont, malgré leur jeunesse, des outils extrêmement complexes et les compétences requises pour leur fabrication dépassent souvent de loin ce qu'il serait possible pour un individu seul de concevoir.

Découplage geste et énergie de production dans l'orgue.
Risset dans \cite{genevois_les_1999} ``L'orgue marque le rôle croissant de la technologie dans l'instrument de musique: il introduit le premier interrupteur, le premier clavier, et dès le XVè siècle la première synthèse additive (qui ne sera justifiée mathématiquement par Fouier qu'au XIXè siècle). L'orgue est aussi la première machine informationnelle: l'information donnée par le geste du musicien y est décuplée de l'énergie sonore''\\
Introduction de la mécanique, déport du doigté (Boehm) — révolution industrielle\\
Introduction de l'électricité (théremin, Martenot, patching de la téléphonie)\\
Introduction de l'enregistrement (phonographie, bande, Schaeffer et l'écoute réduite)\\

\subsubsection{L'arrivée du numérique}

1957 : MusicN (Music I en 1957)
1979 : Casio VL-1
1981 : IRCAM 4X
1982 : E-mu Emulator
1983 : MIDI
1983 : DX7
1985 : The Patcher
1989 : FTS/ISPW

%--------------------------------------------------------------
\subsection{Instruments augmentés}
notion d'instrument augmenté discutable (quid traverso=>Boehm ?, quid guitare électrique?) cf. \href{{sec:ephemeral:longevity_stability}}\\
smart instruments (HyVibe, MIND Music Labs)

%--------------------------------------------------------------
\subsection{Interfaces commerciales}
1ère génération MIDI : clavier, sax, guitare MIDI\\
2ème génération MPE :  LinnStrument, Seabord, SoundPlane
Synthés hybrides (Berhinger, Arturia, teenage engineering)
%--------------------------------------------------------------
\subsection{Instruments collectifs}
Laptop orchestras (Plork, Slork, L2Ork, etc.), Méta-Orchestre
Liste de langages : \url{https://github.com/toplap/awesome-livecoding#languages}

%--------------------------------------------------------------
\subsection{DIY DMIs}
Le \gls{DIY}, instruments en kit, arduino, bela, modular, ethersense, etc.

%--------------------------------------------------------------
\subsection{Live Coding}
Instrument = soft, Interface = keyboard
CucK, Tidal, TopLap, \gls{ICLC}

%--------------------------------------------------------------
\subsection{Installations sonores et instruments à la frontière}
installations sur le web (e.g. tentative d'épuisement du bruit blanc)
sonification de données
apps musicales pour smartphones

%--------------------------------------------------------------
\subsection{Musical organics}
Un mot sur la classification de Magnusson.
le terme se traduit difficilement. Organics renvoit à l'idée de l'organologie autant qu'à l'idée d'organicité, i.e. l'organisation d'un être vivant, en particulier à son organisation et sa prolifération rhizomatique.

%%%%%%%%%%%%%%%%%%%%%%%%%%%%%%%%%%%%%%%%%%%%%%%%%%%%%%%%%%%%%%%
\section{Une critique de la longévité}
\label{sec:ephemerality:critique}

\subsection{DMI will survive}

La longévité des \glspl{DMI} est une question complexe qui a été soulevée à plusieurs reprises dans la littérature des \gls{NIME} (et d'autres domaines connexes) et a fait l'objet d'un débat croissant au cours de la dernière décennie \cite{baguyos_contemporary_2014} \cite{morreale_design_2017} \cite{bonardi_preservation_2008}. Les auteurs qui se sont intéressés à cette question ont identifié un certain nombre de causes de cette situation, qu'elles soient techniques, méthodologiques ou sociologiques et ont apporté réflexions et propositions pour y remédier, telles que de nouveaux environnements pour concevoir et évaluer les instruments \cite{jorda_digital_2004} \cite{morreale_design_2017}, une meilleure documentation, de nouvelles méthodes pédagogiques et la création de communautés ainsi qu'un travail visant à établir une notation musicale et un répertoire pour ces nouveaux instruments \cite{mamedes_composing_2014}\cite{mays_notation_2014}. Cependant, dans la majorité de ces articles, le manque de longévité des \glspl{DMI} est essentiellement considéré comme un défaut, ou du moins un problème à résoudre.\\
\indent Dès 1975, des compositeurs de musique électroacoustique au \gls{GRM} réfléchissaient aux questions de préservation soulevées par une musique \iquote{écrite sur du sable}\footnote{Michel Chion utilise cette formule dans \cite{chion_musique_1977}, en faisant référence aux particules ferro-magnétiques des bandes audio, vouées à une dégradation prochaine} : certains compositeurs disaient qu'ils s'en moquaient et faisaient leur musique pour le présent, tandis que d'autres voyaient dans l'ère numérique naissante la possibilité de préserver leurs œuvres dans le futur. Comme nous le savons aujourd'hui, troquer le sable contre le silicium (ou le nuage, maintenant) n'a pas totalement résolu le problème.\\
\indent Les \glspl{DMI} ayant largement intégré la partie compositionnelle des œuvres musicales, parfois même confondue avec l'instrument, le désir de préserver les œuvres musicales s'est trouvé partiellement transposé dans la question de la conservation des instruments et des outils utilisés pour leur production. Mais quelles sont les raisons de cette quête de longévité ? Et qu’est ce qui légitime à ce point la longévité d’un instrument pour qu’elle soit d’emblée vue comme une qualité ? 

\indent Le désir de longévité est lié ontologiquement à une réaction profondément enracinée dans notre condition de simples mortels, qui consiste à chercher un moyen d'assurer notre survie, notamment par la transmission des connaissances et la création de traditions. Le paléoanthroplogue André Leroi-Gourhan a analysé le phénomène des traditions comme un moyen d'extérioriser et de transmettre notre mémoire à travers la création de systèmes techniques et de ``chaînes opératoires'' \cite{leroi-gourhan_geste_1964}. Plus récemment, Bernard Stiegler s'est appuyé sur cette idée (todo : faut il citer aussi Simondon et Auroux ici?) pour définir le concept de ``grammatisation'', comme processus par lequel le continuum temporel des comportements humains est transformé en un spatial discret, ce qui permet de les intégrer dans des outils \cite{stiegler_for_2010}.\\
\indent Les humains ont ainsi développé des méthodes et des outils tels que la psalmodie de textes (surtout religieux), ou l'écriture comme moyens à la fois d'enregistrer des informations pour un usage ultérieur et de transmettre des connaissances à ceux qui y survivent. L'écriture a partiellement libéré l'homme du besoin de tradition orale en transférant ces connaissances sur un support physique, ce qui lui a également permis de capitaliser et de spéculer sur ses connaissances.\\
\indent Ainsi, la notion de longévité traverse le champ des arts et des sciences, aux frontières desquels se trouvent les instruments de musique. Dans l'histoire de l'art, il reste principalement les œuvres durables, ``gravées dans la marbre'' dont sont faites les sculptures. De même, la science aspire à trouver des lois durables pour décrire le monde observable, et les formuler dans le langage pérenne des mathématiques. Mais si la longévité évidente d'une œuvre constitue souvent un atout pour sa propre légitimation, lorsqu'il s'agit d'un instrument numérique, et plus encore lorsqu'il est conçu comme un moyen interactif de créer une expérience musicale par essence éphémère, la question ne semble pas pouvoir se régler dans les mêmes termes.\\

	
\subsection{Longevité, adoption, succès}
Deux aspects semblent être souvent confondus : la longévité d'un instrument d'une part et son ``succès'' d'autre part. De plus, la notion de succès, éminemment sujette à la perspective adoptée, semble être souvent considérée comme le taux d'adoption par une communauté d'instrumentistes, au delà des aspects financier d'un succès commercial.\\
\indent Ces trois aspects, longévité, succès et adoption, sont cependant relativement différents, en partie indépendants et même parfois contradictoires. Il existe des exemples notoires de ce décalage: The Hands de Michel Waisvisz \cite{torre_hands:_2016} (Figure \ref{fig:ephemeral:Waisvisz_TheHands}) ou encore le Méta-Instrument de Serge De Laubier \cite{couprie_meta-instrument:_2018} (Figure \ref{fig:ephemeral:DeLaubier_MI4}) sont deux instruments ayant eu une longévité remarquable\footnote{Plus de 20 ans pour The Hands —jusqu'au décès de Michel Waisvisz, et plus de 30 ans pour le Méta-Instrument dont l'actuelle 4ème version a été finalisée en 2019.}, soutenue par une pratique régulière de leur inventeurs, sans toutefois avoir été adopté par une large communauté d'instrumentistes. Inversement, l'éphémérité d'un outil ne mène pas systématiquement à une absence de popularité \footnote{Considérons ici tous les gadgets éphémères qui, sous l'influence d'une mode et/ou d'une puissante campagne publicitaire, envahissent le marché, ou encore tous les appareils qui deviennent obsolètes lorsqu'un nouvel appareil les remplace, tel que le smartphone qui, outre le remplacement de nos anciens téléphones, a également balayé d'un coup les lecteurs mp3, les GPS, les consoles de jeux portables, les lampes de poche, les montres, etc.} et encore moins à un manque d'intérêt musical pour les performances réalisées avec ces instruments.

%------------------ Figure : Waisvisz — De Laubier ---------------------
\begin{figure}[!htbp]
	\makebox[\linewidth][c]{%
		\begin{subfigure}[b]{.5\textwidth}
			\centering
			\includegraphics[width=.95\textwidth]{gfx/02_ephemeral/Waisvisz_TheHands.jpg}
			\caption{Michel Waisvisz, The Hands v2\\ photographie: Carla van Thijn}
			\label{fig:ephemeral:Waisvisz_TheHands}
		\end{subfigure}%
		\begin{subfigure}[b]{.5\textwidth}
			\centering
			\includegraphics[width=.95\textwidth]{gfx/02_ephemeral/DeLaubier-MI4.jpg}
			\caption{Serge de Laubier, Méta-Instrument 4\\ photographie: Puce Muse}
			\label{fig:ephemeral:DeLaubier_MI4}
		\end{subfigure}%
	}
	\caption{The Hands v2 et le Méta-Instrument 4: durabilité n'est pas synonyme d'adoption}
\end{figure}

\indent La notion de succès dépend ainsi de la perspective adoptée, selon qu'elle soit celle des luthiers qui créent des instruments pour d'autres ou de ceux qui créent des instruments pour eux-mêmes. Dans ce dernier cas, l'adaptation de l'instrument aux besoins ou à l'esthétique propres de l'instrumentiste peut s'avérer telle qu'il soit difficile pour les autres de l'adopter.\\
\indent Egalement, les évolutions techniques ainsi que les modes peuvent amener à la réapparition d'instruments tombés dans l'oubli. On appréciera ici la perspicacité de François-Alexandre Garsault, cité par Malou Haine dans \cite{haine_les_2018}, qui dans sa D``ivision des instruments selon leurs différentes utilisations'' (1761) classait une série d'instruments, dont la harpe et la ``guitare'' (sic), dans la catégorie des \iquote{Instruments hors d'usage, mais qui peuvent revenir.}


\subsection{Longevité versus stabilité}
\label{sec:ephemeral:longevity_stability}
La question de la durabilité d'un instrument soulève implicitement la question de sa stabilité historique. Ainsi, l'histoire organologique des instruments de musique européens révèle de nombreux facteurs qui conduisent à l'apparition, à l'évolution ou à la disparition des instruments de musique. A cet égard, les nombreuses innovations technologiques de la révolution industrielle s'avèrent intéressantes car cette période bien documentée illustre les débuts des grandes révolutions qui allaient se produire au XXe siècle, tout en soulevant la question même de la stabilité de la forme des instruments. Ainsi, lorsque le traverso fut équipée du système de clétage inventé par Théobald Boehm en 1832 et devint une flûte traversière, s'agissait-il d'un nouvel instrument ? À quel moment décidons-nous qu'un instrument qui subit des changements n'est plus le même ?

\subsection{Éphémérité dans le contexte musical}
\label{sec:ephemeral:ephemerality_in_musical_context}

\subsubsection{Impermanence du phénomène sonore}
\noindent Rappelons tout d'abord une évidence : la musique elle-même est intrinsèquement intangible, évanescente et nécessite une énergie entretenue pour exister : le phénomène sonore est en éphémèrité permanente. La musique, dans sa forme sensible, n'existe que pendant le temps de son performance. Bien que les instruments utilisés pour la produire puissent être durables, leur convocation et le son lui-même sont toujours temporaires\footnote{à tel point que les pièces qui mettent au défi cette éphémérité, telles que les Vexations d'Erik Satie ou encore Organ²/ASLSP de John Cage sont des exceptions notoires.}.

\subsubsection{De la performance}

\noindent Même lorsqu'elle est notée sur une partition, la musique en tant que qu'art vivant est en constante réinterprétation. Cette interprétation permet de transformer une partition notée sous forme symbolique en une expression sensible sujette à variations. On peut objecter que cette interprétation n'existe que lorsque la musique est notée de manière symbolique, laissant aux interprètes la possibilité de la jouer à leur façon dans le contexte de l'interprétation. Mais est-ce toujours le cas lorsque la musique est `intégralement notée' jusqu'au son lui-même, comme c'est le cas sur un disque audio ? Cela signifie-t-il que l'interprétation disparaît ? Les performances de spatialisation en direct de la musique électroacoustique par des musiciens professionnels ou les différentes pratiques de remixes que l'on retrouve dans le hip-hop tendent à prouver le contraire. Toute performance musicale, même la simple écoute d'un disque, convoque inévitablement un nouveau contexte d'écoute, car elle se produit nécessairement dans un moment présent unique. Entre le son enregistré et son écoute, on retrouve la même \textit{différance}\footnote{La \textit{Différance} est un concept proposé par Derrida \cite{derrida_lecriture_2014} pour désigner à la fois l'ajournement (le fait de différer) et la différenciation qui se créé entre un texte et sa signification.} qu'entre une partition et ses interprétations.

\subsubsection{Une esthétique musicale mûe par le mouvement}

Par ailleurs, la musique contemporaine occidentale poursuit une quête de la nouveauté et de territoires sonores inexplorés, comme le soulignait Jean-Claude Risset dans son discours à Athènes en 2014 \cite{risset_sound_2014}: \iquote{Dufourt suggests that contemporary music highlights what was rejected in the Greek world : it rather captures the evanescent, the ephemeral, the ambivalent, the Erebus, it favors the endless metamorphosis of qualities and forms; as Nietzsche proclaimed, western music tends toward the liberation of the dyonisiac dimension and the acceptance of the inacceptable part of myths.}

\subsubsection{Partitions dynamiques, ouvertes, ad-hoc}

\noindent Les partitions musicales sont en partie intégrées dans les \glspl{DMI}, pour lesquels Norbert Schnell et Marc Battier ont proposé le terme `d'instruments composés' dans \cite{schnell_introducing_2002}\footnote{L'idée d'instruments composés est toutefois plus ancienne, voir Harry Partch ou Gordon Mumma (1967) : \iquote{I consider that my designing and building circuits is really ‘composing’ y my ‘instruments’ are inseparable from the compositions themselves.}\cite{mumma_creative_1967}}. La partition elle-même a fait l'objet d'une reconfiguration plus ouverte depuis le milieu du XXe siècle et les compositeurs ont progressivement intégré les possibilités algorithmiques dans leurs processus de création : des systèmes dynamiques et interactifs mettent en mouvement la stabilité des figures de notes. Plusieurs compositeurs\footnote{Parmi ceux qui ont écrit et analysé des partitions dynamiques, voir les œuvres de Hajdu \cite{hajdu_disdisposable_2016}, Bhagwati \cite{bhagwati_vexations_2017} ou Freeman \cite{freeman_extreme_2008}} questionnent ainsi la stabilité de la partition en utilisant l'ordinateur pour créer des instances \textit{ad hoc}, soit à l'aide d'algorithmes génératifs, soit en introduisant des parties improvisées dans des formes hybrides pour lesquelles Richard Dudas propose le terme de ``comprovisation'' \cite{dudas_comprovisation:_2010}. En est-il ainsi, que la technologie numérique offre ce support idéal qui permettrait à la fois la préservation des œuvres musicales en même temps que leur mutation ?

\subsubsection{Obsolescence de la technologie}

The materials used for acoustic instruments seem to age relatively well. Electronic hardware ages poorly in comparison, and the copper of its circuits is more fragile than that of brass instruments. Moreover, the extreme miniaturization of microprocessors often makes them impossible to repair; they need to be replaced and there is great chance that the substitutes will be new, different versions. Computer code, in its compiled form, is just as cryptic as the microprocessor: an unreadable block that embodies the paradox of computer-based notation as compared to traditional paper—we are writing things which we can no longer read. And when the operating system is updated, chances are it will no longer be able to read them either.\\
In an article where he compares the ontological differences between hardware and software, Nicolas Collins \cite{collins_semiconducting_2013} summarizes their relation to time with the formula: \iquote{hardware is yesterday, software is now}, what could be translated as the fact that software is under permanent update while hardware is ever outdated. Neither seems to be able to offer a reliable continuity between the past and the future.


	
\subsubsection{Economie de la nouveauté}

In addition to the obsolescence of technology, DMIs are confronted with the effects of consumer society. For more than a century, the industry has increasingly promoted a disposable paradigm by encouraging consumers \iquote{to trade for style, not just for technological improvements} \cite{slade_made_2006} and while organizing planned obsolescence.\\
\indent This economic model also affected that of the performing arts, which promotes creations much more than the revival of a show to such an extent that, as Georg Hajdu recalls in \cite{hajdu_disposable_2016}: \iquote{Pieces rarely see more than a single performance} Artist residencies are likewise targeted towards new creations and rarely propose the continuation of previous works.\\
\indent This economy of obsolescence (planned or not) does not favour attachment to an instrument and, as far as commercial MIDI controllers are concerned, the cheap plastic they are most often made of degrades the value that can be attributed to a traditional acoustic instrument. The attachment and commitment to a virtual instrument is also challenged by its virtual nature. Most commercial software is now moving towards a rental—rather than purchase—economy, since the purchase no longer guarantees the sustainability of the property.
	
\subsubsection{L'instrument comme compromis instable}

The musical instrument is also, as Bernard Sève points out in \cite{seve_instrument_2013}: \iquote{an unstable compromise between non-convergent qualities}. For acoustic instruments, this compromise between gestural ergonomics and acoustic performance, imposed by the physicality of the materials, is generally fixed in a fitted and glued assembly. This bonding acts as a stabilizing factor compared to a digital environment in which the absence of physical constraint leaves the instrument open-hearted, ready to be modified at any time.\\
\indent Bill Buxton pointed out the difference between standard, military and artistic specifications to underline the higher requirement of the latter \cite{buxton_artists_1997}. Art-driven design require great finesse indeed. Tuning the sonic and ergodynamic\footnote{Magnusson proposed this term in \cite{magnusson_ergodynamics_2019} to name the \iquote{expressive power and depth of an instrument}.} qualities of a musical instrument is a quest for an inframince\footnote{Marcel Duchamp \cite{duchamp_notes_2008} coined the term inframince in a series of examples depicting a difference so small that it can only be imagined.} for which there is no agreed specifications. But another particularity of the technologies used for live performance is that they are \iquote{devoted to an experience, not a sound track; unavailable for reshuffle or back-up or exchange or duplication}, as Nicolas Collins notes in \cite{collins_semiconducting_2013}.\\
\indent Thus, the sustainability of the instrument outside the very duration of performance is not an essential criterion and it is not uncommon for digital musicians\footnote{Andrew Hugill defines a \textit{digital musician} in \cite{hugill_digital_2019} as \iquote{one who has embraced the possibilities opened up by news technologies, in particular the potential of the computer for exploring, storing, manipulating and processing sound, and the development of numerous other digital tools and devices which enable musical invention and discovery} underlying the fact that they are \iquote{not defined by their use of technology alone}, but also have \iquote{a certain curiosity, a questioning and critical engagement that goes with the territory} .} to modify their instrument minutes before the beginning of a concert, just for the needs of the present moment.
	
\subsubsection{Esthétique du dysfonctionnement}

Indeed, the risk of dysfunction is not a major obstacle to many musical performances. Bugs and artefacts caused by malfunctions are proving to be fertile sources of musical materials and subverting the cryptic functioning of processors reveals an invisible aspect of them, bringing to the surface their very nature, beyond the purpose for which they were designed\footnote{Among significant examples, Yasuano Tone's works on “wounded CDs”, Nicolas Collins's works on dead circuitry or Carsten Nicolai's sonification of raw data exemplifies this approach.}. David Zicarelli, quoted by Cascone in \cite{cascone_aesthetics_2000}, sums it up in these terms: \iquote{I would only observe that in most high-profile gigs, failure tends to be far more interesting to the audience than success}.

\subsubsection{Plus besoin de tradition?}

The appearance of musical notation made performance no longer necessary for the only purpose of transmission; audio recording made performance no longer necessary for the only purpose of listening; computers and sound banks made the learning of a particular instrument no longer necessary to produce the sound of that instrument\footnote{See for example, the rendition of Stravinsky's Rite of Spring by Jay Bacal with VSL. https://youtu.be/PB3njyDW8SY.}; and now, artificial intelligence makes the very act of composing no longer necessary for music to be composed\footnote{See for example the outcomes of the FlowMachines project by François Pachet et al. in \cite{hadjeres_deepbach:_2016}: “Daddy's car” (\url{https://youtu.be/LSHZ_b05W7o}) or “DeepBach” (\url{https://youtu.be/QiBM7-5hA6o}).}.\\
\indent In 1964, Leroi-Gourhan, who saw in the computing machine the unprecedented possibility of outsourcing memory, was wondering what would happen if the machines became capable of \iquote{writing perfect plays, creating inimitable paintings} \cite{leroi-gourhan_geste_1964}. In 1992, John Cage seemed to answer him a radical way, when saying: \iquote{We don't have to have traditions if we free ourselves from memory} \cite{sebestik_ecoute_1992}. \\
\indent However, if it is possible to evolve, as Buci-Glucksmann describes it in \cite{buci-glucksmann_esthetique_2003}, from a culture of the object to a culture of flows, she remarks that in a country like Japan that values impermanence positively, the ephemeral has a central place while being deeply rooted in tradition.	\\
\indent The resolution of this apparent antagonism between Cage's position and Buci-Glucksman's seems to lie in the displacement of objects (or flows, for that matter) supported by tradition, in the reformulation of the motivations for sustainability and ephemerality.


\section{Articulation du pérenne et de l'éphémère}

\subsection{Les DMI comme agencements instables et sauvages}

The very term DMI, which has gradually invaded the NIME literature, may lead us to believe that it is a well-defined category when it is in fact a hodgepodge of objects only sharing their use of digital computation. A resulting bias in the assessment of the failure of DMIs to reach maturity stems from the fact that a musical instrument is often still considered as a coherent whole, requiring longevity, similar to the acoustic instruments taken as role models.\\
\indent Yet, the modularity induced by electronics and digital technology has atomised the integrity of the instrument. This atomisation can both be understood in the sense of “destructed” but also in the sense of “fragmented into atomic bits”. On stage, we can further observe that DMIs are often fragile (cf. Figure \ref{fig:ephemeral:Gordeff}), prototypical assemblies, full of cables ready to be interchanged minutes before the concert, or even during it. So why should we consider DMIs as durable monoliths rather than ephemeral assemblages\footnote{Following Deleuze and Guattari's concept proposed in \cite{deleuze_mille_1980} voir aussi ``musical instruments as assemblage'' de Paul Theberge.} that they most often are?\\
\todo{évoquer la notion de behavioural objects de \cite{bown_understanding_2009}}
\indent From this point of view, the academic format of a conference such as NIME makes it difficult to present DMIs in their chaotic form and their selection is biased by the fact that their authors often belong to the academic world. This favours a demonstration of duly considered technical criteria rather than the presentation of a chaos of hectically connected algorithms whose functioning is not really understood, except for the fact that the musician who plays them does wonders.\\
\indent By confronting an ephemeral instrumental setup, the instrumentalist, however virtuosic they may be, necessarily finds themselves in tension with a wild instrument to tame. This calls for an intense gestural and auditory attention and the research of resonance with the instrument. (Otherwise, one might as well compose comfortably at home and provide the listener with an audio record to be played with a single button). Maybe more important than longevity, here is an interesting design criterion for digital lutherie: the possibility that the instrument spins out of control.

%-------------------------- Figure : Gordeff ----------------------------------
\begin{figure}[!htbp]
	\includegraphics[width=\textwidth]{gfx/02_ephemeral/PierreGordeff.jpg}
	\caption{Détail d'un instrument de Pierre Gordeff.}
	\label{fig:ephemeral:Gordeff}
\end{figure}

\todo{rajouter une réf et un mot sur Fragile Instruments \cite{haddad_fragile_2017}}

\subsection{Cuisiner des instruments à la volée}

Another reason that contributes to the stability of acoustic instruments is related to their physicality and manufacture, which requires a considerable amount of work compared to the virtual arrangement of software blocks—it takes more than two months to build a cello for a luthier who knows his job! Conversely, Bowers and al. promoted the use of readymade objects as half-made infra-instruments \cite{bowers_not_2005} and “pin-and-play” ad-hoc instruments \cite{bowers_creating_2006}, while rethinking the life-cycle of an instrument with such kind of quickly-built ephemeral assemblages.\\
\indent More generally, as one create a DMI with an audio programming environment, the software not only provides basic functions but comes with libraries, ready-made examples, supplemented by countless online resources, ready to be downloaded, copied and pasted.\\
\indent This means that the building of a DMI can be a much faster subtractive process: rather than starting from a blank page, it is possible to search for a version close to what one wants to achieve and modify it from there. Nicolas Collins compared this simplicity to cooking, emphasising its democratisation: \iquote{What it means is that if you are doing live performance, if you do need specialized instruments, it's almost more like cooking than it is building musical instruments. Everybody cooks, you don't need to go to ‘chef schools’!} [Collins, personal communication TODO = annexe].\\
\indent Recent evolutions in audio-programming languages tend to address the issue of sustainability by creating Domain Specific Languages that can be exported to various targets. Hardware platforms like Bela or The Owl\footnote{Bela: \url{https://bela.io}; The Owl: \url{https://www.rebeltech.org}} and languages like FAUST \cite{orlarey_faust_2008} or the announced SOUL language by Roli\footnote{Announced at the Audio Developer Conference 2018. \url{https://youtu.be/-GhleKNaPdk}} all reflect this trend. It is worth noting that FAUST, which was designed with preservation in mind\footnote{FAUST was a key component of ASTREE, a project focusing on the preservation of musical work with electronics.}, actually helps building ephemeral instances by offering both an online compiler and just-in-time compilation.
	
\subsection{Une relation tri-partite : répertoire / musicien / contexte}

If we therefore stop considering the ephemerality of instruments as a problem, we can consider how longevity and ephemerality can be articulated in the agency of DMI practices. It can be conceived as a tripartite coupling between materials, musician and context. Each component of this triad may then has a different degree of stability.

\subsubsection{Le grand répertoire}

The components of a DMI can be considered as belonging to a large repertoire of both material and immaterial heritage. The material repertoire includes any physical material that can be used in the construction of acoustic instruments: raw materials as well as manufactured, machined or mechanical parts. \\
\indent The immaterial repertoire is all the theoretical knowledge and cultural heritage one can rely on during the making of an instrument\footnote{Obviously, practical knowledge is also essential to the making of an instrument, although it does not really belong to the shareable heritage to which I refer here.}: music theory, scientific and technical knowledge, established playing techniques, musical repertoire, etc. This knowledge helps to shape the materials and imprint musical markers to create the instrument: the placement of frets, the tuning of strings, the layout of keys, etc.\\
\indent In the case of DMIs however, the repertoire of physical materials is considerably expanded by reified knowledge available as digital materials, either in the form of computer code or datasets (e.g. audio samples, impulse responses, scores, etc.) that enables the musical qualities of an instrument to to be shaped beyond what is possible with physical materials.
This set, as heterogeneous as it may seem, constitutes a shareable repertoire from which digital luthiers can draw the necessary ingredients for the design of their instrument.

\subsubsection{Le musicien in-progress/in-process}

The second element of the assemblage is the musician\footnote{Here the generic term “musician” mainly represent the instrumentalist the blurring roles of instrumentalists, composers and luthiers. The instrumentalist is not necessarily present during the performance—the instrument can be autonomous, but their presence is then implicit as a luthier or composer, the boundaries between these roles have been largely blurred since the 20th century.}. Musicians are alive and subject to change: their knowledge, feelings and desires, musical skills and awareness, projects and physical capacities, all evolve during their existence. This evolution is reflected in the instrumental device, by the addition or removal of features, or the development of new instruments related to a new musical project. Just like you may learn to ride with a bicycle equipped with side wheels and later remove them, DMI can offer evolutive assistance for progressive learning. A co-dynamic relation with one's instrument can help improve the intimacy between the musician and the technical object becoming an instrument.

\subsubsection{Le \textit{hic et nunc} de la performance}

Eventually, the DMI can be adapted to the context of performance, which is generally more ephemeral than the two aspects mentioned above.\\
\indent Expanding their own musical repertoire by drawing on the grand repertoire mentioned above and on their own experience, musicians selects a subset of elements in the perspective of a particular performance, for a singular artistic proposition, and to meet the spatial and temporal conditions of the performance, as well as the audience. As an example, the costless duplication of code offers possibilities to rescale DMIs from soloist to collective instruments by distributing control over multiple interfaces. New projects can imply starting from scratch, but existing ones often only involve contextual adjustments rather than a thorough reprogramming of one's system. Kiefer and Magnusson coined the term “pre-gramming” \cite{kiefer_live_2019} to describe that particular kind of preparation.
	

\section{Jouer d'un DMI éphémère}

As we can see, creating a DMI can be a very fast process as it can be done by simply assembling already pre-built elements. But once the assemblage is done, how do we learn to play it?

\subsection{Composition, conception, apprentissage et jeu en parallèle}

\indent Traditional acoustic instruments are supported by methods and repertoire that can rely in turn on the instrument's stability. But for a new—possibly unique, possibly ephemeral—DMI, such resources are scarcely available. Software comes at best with manuals, but manuals usually explain how to make the software run, not how to play music with it.\\
\indent From there, the learning process can follow two seemingly opposite directions: finding the right moves to play desired sounds and finding the right sounds for chosen gestures. A consequence is that often, learning a new DMI starts right from its conception and is a co-dynamic process that accompanies its development up to the \textit{pre-gramming} of the instrument, with back and forth between moments of play and moments of adjustment.

\subsection{Entrer dans l'avenir à reculons}

DMIs and their practices integrate the know-how inherited from electroacoustic music since the middle of the 20th century. The pedagogy of electroacoustic music essentially developed a musical theory of listening \cite{schaeffer_traite_1966} and metaphors for composing \cite{bayle_musique_1993} but conceived at a time when electroacoustic music could only be composed, before real-time audio allowed live practices. As a result, these theories were more oriented towards musical composition than performance as such.\\
\indent In the absence of established musical notation for sound, experimental electronic music is largely oriented towards free improvisation. This implies a letting go enabling the instrument to express its potentialities and a practice of “aurality”\footnote{Described by Alain Savouret as a music theory for the audible.} to “enter the future backwards” \cite{savouret_introduction_2010} and react to what is coming out of the instrument rather than completely controlling it.
	
	
\subsection{Trouver les résonances}

The learning of an instrument (beyond the learning of the heritage idioms of this instrument) thus requires a search for resonance. We can experience this resonance at an acoustic level, but more generally as an empathetic resonance, which consists in immersing ourselves in the instrument to find the spaces that will (re)sound satisfactorily, to find the “sweet spots” where what we hear meets what we were seeking—sometimes unknowingly. Since mathematical linearity is rarely satisfactory at a perceptual level, this exploration involving the coordination between play and critical listening is essential to adjust the mapping functions that will define the instrument's behaviour.
	
\subsection{Entomologie musicale (bestiaire)}
The musical exploration of a DMI brings out unknown musical forms, like ephemeral butterflies. Learning a DMI therefore often involve an entomologist-like task of pinning these sonic creatures and giving them a name. This naming will allow to come back to them later on (by saving them in presets for example) as well as to discuss with other musicians about a performance which, in the absence of established musical idioms on which to rely—like scales or time signature— can be cruelly lacking references. Such a task was led in the development of “John, the semi-conductor”, an open score system described in \cite{goudard_john_2018} (TODO : renvoyer au chapitre notation).\todo{fusionner avec la section \ref{sec:ephemeral:vessels}?}



%-------------------------- Figure : Entomologie ----------------------------------
\begin{figure}[!htbp]
	\includegraphics[width=\textwidth]{gfx/02_ephemeral/Bestiaire.png}
	\caption{Entomologie musicale}
	\label{fig:ephemeral:entomologie}
\end{figure}

\subsection{Pratique modulaire de la stabilité}

While a DMI can be an unstable assemblage, its individual components may provide more stable grips. For example, if the performance is based on a written score, the instrumentalist can learn the sequence of appropriate gestures necessary for its realisation\footnote{A interesting and critical example is the piece Aphasia by Mark Applebaum (https://youtu.be/wWt1qh67EnA), where the performance relies on gestures and a soundtrack which are totally notated, yet to be performed.}, such as a pilot in a cockpit \cite{vertegaal_towards_1996}.\\
\indent As far as the behaviour of the DMI is concerned, one can partly transfer one's knowledge of other DMIs to a new instance one is trying to learn. For example, the integration of an FM synthesis into a DMI can help a person familiar with this type of synthesis to navigate its timbre space (bells, siren, brassy, wiggly, etc.), independently of the control interface plugged onto it, relying on their own knowledge and representation of FM synthesis parameter space. The timbre space of various audio syntheses can also be remapped on a common and more stable perceptive space (e.g. pitch, loudness, brilliance, etc.) that abstracts control from their differing parameter spaces, such as presented in \cite{wessel_timbre_1979}, \cite{arfib_strategies_2002}, \cite{schwarz_sound_2012} or \cite{tubb_divergent_2014}.\\
\indent Likewise, an expertise can be acquired on a gestural interface, which calls for specific gestures and moves\footnote{For instance, consider the “launchpad” scene, characterized by the publication of battle-videos of rhythmic virtuosity.}; this expertise relies on an embodied spatial memory that—to some extent— remains partly independent from the audio syntheses or effects controlled with the interface. The behavioural stability of the instrument can also be of virtual nature, for example when using dynamic intermediate models \cite{goudard_dynamic_2011}, which can act as a stable reference taking place between various changing syntheses and interfaces.\\
\indent Overall, this transposed and “modular” knowledge can only provide broad outlines of what is necessary for the subtle practice of an instrument. The devil's is obviously to be found in the details.

\subsection{Les DMI comme vaisseau pour la mémoire}
\label{sec:ephemeral:vessels}
DMIs are heterogeneous vessels loaded with memories of our performing, composing or instrument-making experiences. The sounds that we collect, the synthesis algorithms that we develop (or download), the parameters that we adjust, the kitchen recipes and mapping functions that we carefully craft, all contribute to the evolution of a personal repertoire where ephemeral instances crystallize. Magnusson proposed the term epistemic tool to describe a musical instrument as \iquote{a designed tool with such a high degree of symbolic relevance that it becomes a system of knowledge and thinking in its own terms} \cite{magnusson_epistemic_2009}.\\
\indent Thus, DMIs tend to be evolving assemblages of these stored memories and often involve activities that are not generally associated with instrumental practice, such as file management, bookmarking online resources or organizing sound banks, in order to be able to convene these resources as quickly as possible during the performance.\\
\indent It is remarkable that the possibilities of duplication and dissemination offered by digital media and the Internet have not led to the standardisation of instrumentariums; digital musical instruments are often very personal and singular.

\section{Conclusion}

This article has presented how the ephemerality of DMIs should not only be considered as a problem, but as an intrinsic modality of their ontology. Rather than opposing longevity, it actually informs the technical design of the environments conducive to their development and sustainability.\\
\indent Ephemerality of the tools does not prevent great music to be produced, nor great musical performances to happen. On the contrary, it can both help to adapt musical assemblages to contexts that are in essence ephemeral and to challenge human's ability to respond to a fleeting, untamed musical setup. In the end, great musical works seem to find their way, sustained by the care and the work of those who recognize these works as master pieces. These works may stand the test of time being dispersed, distributed, transformed, recomposed, reinterpreted or even renamed, by all those who will attach importance to them. This loving care probably belongs to the part of our memory that we cannot outsource in a tool and that redefines tradition and preservation outside the technological frame.\\
\indent In the sand that was evoked by Michel Chion, we could see another interesting metaphor for digital musical instruments, as sand in a previous life was a rock that got atomized, fragmented into tiny bits. The same thing seem to have happened to musical instruments, that have been atomized into tiny modules. We can play with sand as a fluid material, or givin it a shape, or adding cement and make something more concrete.
\indent If digital technologies reach maturity one day, we may be able to rely on stable and sustainable instruments. In this case, following Garsault's premonition, it should not be forgotten to classify all the ephemeral instruments that preceded them in the category of \iquote{instruments out of use, but that could come back}.


\section{extra material}

Thor Magnusson in Sonic Writing (p12) : Anyone who plays a musical instrument will be familiar with the special moment when a new instrument is picked and its ergodynamics studied through play. (footnote : we often change our instruments during performance : we retune string instruments, change effect settings in electronics, and the \textit{whole point} of live coding is to create and redefine the instrument during play). This experience of ergodynamics recognises that an instrument is an object that never rests, or enter a period of stasis: that every time we pick it up there are new things to discover, new patterns our fingers know, because we have changed, the instrument has changed, and so has the whole world itself — the general performance context.

Risset dans \cite{genevois_les_1999} : ``La disponibilité `'d'accès'' gestuels tout à fait différents de accès instrumentaux risque de rester lettre morte, dans la mesure où il est improbable que des interprètes réalisent l'investissement considérable que représente l'apprenttissage d'un instrument complètement nouveaux s'ils n'ont pas l'assurance que cet instrument va durer et qu'un répertoire va se développer pour lui''

 % INCLUDE: introduction
% !TEX root = ../thesis-example.tex
%
\chapter{L'instrument et le geste}
\label{ch:gesture}

\cleanchapterquote{La percussion de ce pseudo-gong est illlusoire :\\
rien ne tape sur rien dans l’ordinateur.\\
Schumann qualifiait le legato au piano\\
de ``trompe-l’oreille'' : la musique est aussi un art\\
du mirage, de l’illusion.}{Jean-Claude Risset}{Discours invité aux JIM 2010 \cite{risset_propos_2010}}

Proprioception et schema corporels (cf. Miranda unconventional computing)

%L'expression musicale a été contrainte (et non ``prisonnière'', car la contrainte peut être fertile) par la relation causale entre le geste d'excitation et le son produit par l'instrument dans les lutheries acoustiques.

\noindent Poursuivant des évolutions organologiques latentes, telles que présentées dans la section \ref{sec:ephemerality:landscape}, les \glspl{DMI} finissent d'opérer un découplage énergétique, une ``dislocation du contrôle'' (\textit{control dislocation} \cite{miranda_new_2006}), rompant avec plus de 35.000 ans de tradition musicale \cite{conard_new_2009} et avec l'unité de temps, de lieu et d'action —définie en règle par le théâtre classique, mais qui reflète la relation qui existait entre l'auditeur et le phénomène sonore, comme le rappelle Hugues Genevois dans \cite{cance_what_2012}.\\
\indent L'introduction de la mémoire et de la computation numérique permet une re-programmation complète de l'interaction, rendant leur fonctionnement à la fois complexe et cryptique. Cet aspect peut se révéler être un inconvénient, dans la mesure où il prive le public d’une lecture possible de la performance musicale. Cela peut cependant s’avérer être un avantage, si l'on considère que la performance musicale comporte une part scénographique dans laquelle l’illusion a toute sa place.\\
\indent Nous allons voir dans ce chapitre comment le geste instrumental s'en trouve affecté, par un examen critique des catégories gestuelles déjà proposées, la proposition de nouvelles catégories prenant en compte l'aspect subversif de l'art musical et les conséquences que nous pouvons en tirer en termes de conceptions des \glspl{DMI}.


\section{Introduction : L'étude du geste en musique}

\noindent Si l'étude du rythme et plus encore, des hauteurs, a été particulièrement importante dans la culture classique occidentale, l'étude du geste a longtemps été négligée voire méprisée, comme le rapporte Jean-Marc Warszawski\footnote{Conférence La musique et le geste : \url{https://www.musicologie.org/18/la_musique_et_le_geste.html}.}: \iquote{La tradition savante occidentale, grâce à l'écriture, dématérialise le geste créatif musicien, et impose une ligne de démarcation ente musique écrite et non écrite, en quelque sorte une frontière entre le primitif et le civilisé.}\\
\indent Peut-être faut-il également y voir une autre raison : le geste ne se laisse pas aussi facilement mesurer, encore moins définir, que la hauteur. Si cette dernière peut se  réduire en première approximation à une grandeur physique mesurable --~sa fréquence fondamentale\footnote{La perception de la hauteur est évidemment plus complexe que l'évalutation de la fondamentale et un des sujets d'étude de la psycho-acoustique, voir notamment les travaux de Michèle Castellengo \cite{castellengo_ecoute_2015}. L'écriture musicale classique s'est toutefois largement construite sur cette réduction, que Robert Francès nomme ``abstraction notale'' \cite{frances_perception_1984}.}, le geste ne se laisse pas aussi facilement réduire à une mesure. Il entraine depuis sa production jusqu'à sa réception toute la complexité du vivant : sa multidimensionalité, son instabilité, sa relativité, sa combinatoire, et l'ensemble de ses aspects culturels... (TODO : développer et/ou enlever les pointillés)\\
\indent L'étude du geste se développe dans le courant du \siecle{19}~siècle, sous l'effet de la révolution industrielle, de l'étude mécanique du mouvement et la création de conservatoires qui développent des techniques d'apprentissage où le geste est pris en compte\footnote{Voir en particulier la collection rassemblée sur le site de la Bibliothèque Nationale de France: \url{https://gallica.bnf.fr/html/und/partitions/oeuvres-theoriques-et-pedagogiques}. L'intérêt pour le geste à cette période de développement industriel donne lieu par ailleurs à l'invention d'étonnants systèmes mécaniques servant à guider, contraindre et fortifier les gestes, en particulier pour le piano, tels que le ``chiroplaste'' ou le ``dactylion'', mais qui s'avèrent être davantage des instruments de torture, endommageant parfois les mains de manière irréversible.}. L'étude du geste a progressivement gagné en importance, d'une part avec l'émergence de l'anthropologie au début du \siecle{20}~siècle, et d'autre part suite à l'explosion des technologies de télécommunication, lorsque sa compréhension et sa modélisation sont devenues nécessaires pour le développement des \gls{IHM} dans la seconde moitié du siècle, jusqu'à devenir un domaine d'étude à part entière, les \textit{gesture studies}, soutenue notamment par l'\gls{ISGS} créée en 2002.\\
\indent Dans le domaine de la musique, c'est l'arrivée du ``temps-réel'' et l'émergence des \glspl{DMI} dans les années 1980, qui entraine la parution croissante d'articles traitant de la question du geste instrumental. Au tournant du siècle (du millénaire), le geste devient un objet d'étude majeur dans le domaine de l'informatique musicale, se traduisant notamment par la parution d'ouvrages collectifs dédiés\footnote{Voir en particulier : \cite{genevois_les_1999}, \cite{wanderley_trends_2000} et \cite{godoy_musical_2010}}, ainsi par que l'apparition de la conférence \gls{NIME} en 2001, qui lui accorde une place importante.\\
\indent Plusieurs projets de recherche interdisciplinaires ont également été menés, comme le projet ConGAS\footnote{``Gesture Controlled Audio Systems'', projet financé entre 2003 et 2007 dans le cadre de la Coopération Européenne en Sciences et Technologies (COST Action 287)}, ou plus récemment le projet Gemme\footnote{``Geste musical : modèles et expériences'', projet de recherche financé par l'ANR de 2012 à 2016, dont un carnet de recherche en ligne est disponible : \url{https://geste.hypotheses.org/gemme}} et la chaire thématique pluridisciplinaire GeAcMus\footnote{La chaire``Geste - Acoustique - Musique'' a été créée en 2015 à Sorbonne Université \url{http://www.sorbonne-universites.fr/actions/recherche/chaires-thematiques/geacmus.html}} créée à Sorbonne-Université.

\extra{Le terme Gesture est utilisé dans 62\% des articles publiés à NIME (cf. Jensenius paper : To Gesture or Not? An Analysis of Terminology in NIME Proceedings 2001–2013) <= update this}


\section{Geste instrumental et geste musical}

\subsection{La musique et ses instruments}

\noindent La définition du geste musical pose le double problème de définir ce qu'on entend par ``geste'' et par ``musique''. Pour ce qui est de la musique, j'adopterai ici la définition proposée par Christopher Small, non pas du \textit{nom} ``musique'', mais du \textit{verbe} qu'il invente : ``musiquer'' \cite{small_musicking:_1998}:
\iquote{Musiquer, c'est participer, à quelque titre que ce soit, à une performance musicale, que ce soit en jouant, en écoutant, en répétant ou en pratiquant, en fournissant du matériel pour la performance (ce qu'on appelle ``composer''), ou en dansant. Nous pourrions même parfois en étendre le sens à ce que font la personne qui prend les billets à l'entrée, les gros bras qui déplacent le piano et les tambours, les roadies qui installent les instruments et font les balances ou les personnes qui nettoient après que tout le monde soit parti. Ils contribuent tous, eux aussi, à la nature de l'événement qu'est une performance musicale.}\\
\indent Si j'adopte ce néologisme, ce n'est pas pour l'exhaustivité que semble conférer une définition aussi large mais essentiellement pour deux raisons: la première est le choix d'utiliser un verbe plutôt qu'un nom, c'est-à-dire d'identifier une pratique plutôt qu'un objet, ce qui dans le cas de la performance musicale semble mieux adapté. La deuxième raison est liée au contexte technico-culturel dans lequel ces pratiques s'inscrivent: la reconfiguration des modes de production et de réception de la musique\footnote{voir à ce sujet l'ouvrage de Paul Théberge \cite{theberge_any_1997}} qu'ont entrainée les évolutions technologiques du \siecle{20}~siècle ont rendu poreuses les frontières entre les différentes pratiques liées à la création musicale. Je restreindrai toutefois le champ de cette définition aux pratiques qui gravitent directement autour de l'instrument de musique, tel que présenté dans le chapitre précédent.\\
\indent Partant de cette définition et reprenant une proposition d'Hugues Genevois \cite{genevois_geste_1999}, on peut distinguer plusieurs phases au cours desquelles se manifeste un geste musical :

\vspace{-1em}
\begin{itemize}[noitemsep]
\item \textbf{la composition} : la production de structure musicales hors-temps de leur rendu sonore;
\item \textbf{la lutherie} : la réalisation de l'instrument et sa préparation pour le jeu;
\item \textbf{la performance} : le jeu instrumental qui produit, modifie, mélange, dans le temps même de l'écoute, la matière sonore, les gestes de l'instrumentiste, du chef, de l'ingénieur du son;
\item \textbf{l'écoute} : qui construit l'intelligibilité de notre environnement sonore et se mobilise sans répit pour en garantir la cohérence;
\item \textbf{la pédagogie} : durant laquelle la complexité du geste musical se transmet progressivement, en étant guidée, soutenue, encouragée, critiquée, en se pratiquant parfois sous la forme d'exercices idiomatiques et systématiques (e.g. faire ses gammes), et à travers des allers-retours entre performance et écoute.
\end{itemize}

\noindent Dans la pratique, ces différents gestes se superposent souvent mais pourront se traduire par différentes formes de configurations instrumentales, adaptées aux spécificités de chaque situation (TODO : développer ou renvoyer à une section ultérieure qui développe).

\subsection{Définition(s) du geste}

\noindent Dans sa définition générale, le Littré, le Larousse ou le dictionnaire de l'Académie Française s'accordent à le définir comme \iquote{un mouvement du corps, principalement de la main, des bras, de la tête, porteur de sens ou non} (Larousse).

\noindent \textbf{aspects phénoménologiques} Un premier aspect du geste est sa nature de mouvement, son déploiement spatial et temporel, dont les qualités spatiales, cinématiques, cinétiques, synchroniques, fréquentielles, balistiques (...) abstraites font écho à la mémoire de tous les mouvements dont nous faisons l'expérience: battements, ondulations, chutes, éclosions, envols, éruptions, roulements, contractions, extensions, sursauts... Le geste possède une force expressive, en dehors de toute interprétation sémiotique, génératrice de sensations dont la logique ne peut être décrite que par la métaphore\footnote{C'est en particulier l'entreprise de Gaston Bachelard dans son travail de ``phénoménologie de l'imagination'' \cite{bachelard_air_1943}}.\\
\indent Ce mouvement est l'expression même du vivant et il reflète à la fois la relation du corps à son environnement externe (force de gravité et cinétique, contournement d'obstacles, frictions sur les matériaux, contraintes formelles chorégraphiques...) et l'expression des mouvements internes du corps (vigueur ou mollesse, souple ou crispé, émotion se traduisant par des gestes tels que haussement d'épaule, sursaut de surprise, grimaces diverses, etc.). La maitrise de ces mouvements est un aspect fondamental de la danse qui, comme la musique le fait avec le son, met en œuvre le corps dans des intensités, des rythmes, des trajectoires. Si les gestes de la danse ne sont pas \textit{a priori} instrumentaux, les \glspl{DMI} rendent plus que jamais possible leur utilisation à des fins d'interaction musicale, comme nous le verrons plus loin (cf. notamment \cite{bevilacqua_gesture_2011, alaoui_movement_2012, silang_maranan_designing_2014, hsueh_understanding_2019}.)

\noindent \textbf{aspects sémiotiques} Un deuxième aspect du geste est indiqué par le curieux épithète de la définition précédente : ``porteur de sens ou non''. S'il a semblé utile d'évoquer une qualité potentiellement absente, c'est précisément parce les différentes définitions qu'on donne du geste dépendent en partie de cet aspect sémiotique, qui reste soumis à une interprétation subjective et contextuelle dans un système de valeurs. Les significations d'un geste peuvent en effet être attribuées par l'auteur du geste ou bien par celui ou celle qui observe ce geste, sans qu'il n'y ait nécessairement de correspondance, ni en terme de signification, ni sur la part du mouvement considérée signifiante. La signification d'un geste peut également être attribuée par la machine, en particulier dans les systèmes d'apprentissage, comme nous le verrons plus loin.

\noindent \textbf{aspects ergotiques} Une autre définition du geste relève sa dimension manipulative : \iquote{Manière de mouvoir le corps, les membres et, en particulier, manière de mouvoir les mains dans un but de préhension, de manipulation} (Larousse). Cette définition met l'accent sur la relation possible entre le geste et un objet, un outil, un instrument et s'applique par conséquence particulièrement bien à la situation instrumentale. C'est sur cet aspect que se concentre notamment les typologies gestuelles proposées par Claude Cadoz que nous discuterons plus loin.

\noindent \textbf{aspects (dia)grammatiques} Enfin, si le geste \textit{ex-prime} (du latin \textit{ex-premo} : ``faire sortir en pressant'') des mouvements internes du corps, il \textit{im-prime} et laisse une trace dans la matière, dans la machine, dans les esprits. C'est parfois cette trace résultante qu'on nomme geste, par métonymie et analogie avec les qualités de mouvement qui sont à son origine, lorsqu'on parle par exemple du \textit{geste de composition}. Les capacités des machines à enregistrer le geste dans sa dynamique --~auparavant éphémère et seulement visible sous la forme de traces statiques: littérature, peinture, partition~-- confèrent à cet aspect \textit{diagrammatique} du geste une importance particulière à l'époque de sa reproductibilité technique\footnote{L'expression utilisée ici revoie bien évidemment à l'œuvre éponyme de Walter Benjamin\cite{benjamin_loeuvre_2013}}, qu'il nous faut souligner et qui sera analysée en dernier lieu.

%--------------------------------------------

La fonction sémiotique du geste a occupé une grande place dans son étude au \siecle{20}~siècle, notamment sous l'influence conjointe de l'anthropologie naissante, analysant les systèmes d'interaction sociale de différentes cultures \footnote{Adam Kendon, Marcel Jousse, Bernard Koechlin, André Leroi-Gourhan...} mais aussi de l'explosion de la télécommunication et des recherches scientifiques qui ont accompagné cette révolution technologique.

L'influence de la théorie de l'information de Claude Shannon (cf. figure TODO), en particulier, a contribué à envisager le geste, dans une perspective communicationnelle. Les analyses qui en résultent s'inscrivaient alors, plus ou moins explicitement, dans une perspective d'encodage efficace, en extrayant l'information qu'il contient, et par inférence, sa signification supposée.

Cette influence est particulièrement sensible dans l'école Anglo-Saxonne, centrée sur les figures de Adam Kendon et David McNeil, qui confèrent au geste une signification préalable à son existence. D'après McNeil, \iquote{(...) le geste est créé par le locuteur comme une matérialisation du sens'' \iquote{(...) a gesture does not represent at all; the gesture is created by the speaker as a materialization of meaning. }, \cite{mcneill_gesture_2005}}

A l'opposé, l'approche phénoménologique adopte un autre point de vue en conférant au geste un statut pré-sémiotique et un pouvoir créatif et expressif.
Guerino Mazzola : \iquote{Gestures are dialogical, live in presence, are circular, elastic, are presemiotic, and are as such already differentiated (being gestural can be ramified into different types).} \cite{mazzola_topos_2018}, p. 59 (852)

Le geste peut être porteur d'un sens préalable (McNeil) ou d'un sens 

D'une certaine manière, ces deux approches se placent de part et d'autre du processus de communication, du point de vue de l'auteur du geste (Gesture and thoughts, chez McNeil) et de son observateur (Phénoménologie de la perception).

Le geste \textit{exprime} (du latin \textit{ex-premo} : ``faire sortir en pressant'') quelque chose, mais n'a pas nécessairement une signification.


Pour autant, on se rend bien compte que tous les mouvements du corps ne sont pas nécessairement en permanence associés à l'idée de geste. Dans le cadre d'une activité spécifique telle que la pratique musicale, impliquant la participation active de l'individu, la plupart des études (todo:mettre plusieurs ref) sur le geste s'accordent à le définir comme l'association d'un mouvement et d'une intention.

\iquote{Le geste doit être défini comme un mouvement intentionnel plus ou moins complexe, orienté vers un but déterminé qui lui donne un sens individuel, social ou historique.} \cite{imberty_mouvement_2013}

\noindent L'intention gestuelle peut être envisagée selon différent points de vue, selon que son étude porte sur sa fonction sémiotique (le geste-signe), ergotique (le geste-action), épistémique (le geste de perception).\\
\indent La notion de geste est également utilisée pour décrire des formes indirectement liées au mouvement physique, telles que le ``geste de composition'', en transposant par analogie le mouvement et l'intention le caractérisant. On voit donc que la question de l'intention soulève des questions esthétiques voire politiques : l'inscription de la performance musicale dans la vie sociale et culturelle pose la question de la place du geste dans un système de valeurs spécifique. En particulier, l'époque postmoderne est caractérisée par un art qui attribue davantage d'importance à \textit{l'intérêt} du geste pris dans un contexte \textit{expérimental}, qu'à sa \textit{beauté}, prise dans un contexte \textit{normatif} classique.

\indent Enfin, en tant que phénomène de transfert d'énergie et d'information, le geste pourra être évaluée différemment selon qu'on se place du point de vue de sa production ou de celui de sa réception. 

La ``geste'' en musique a ainsi été utilisée pour décrire différents aspects de la relation qu'entretiennent le mouvement, l'instrument, l'intention, la morphologie et la signification du son ou encore les formes d'écritures compositionnelles. L'instrument de musique se trouvant à la croisée de ces différents chemins, la notion de ``geste instrumental'' a souvent été liée voire confondue avec celle de ``geste musical''. 

La relation entre musique et mouvement du corps dépasse le domaine de l'interaction instrumentale. Par exemple, les mouvements de la danse, s'il peuvent être en corrélation avec la musique, ne sont pas a priori des gestes instrumentaux. Cependant, si les mouvements du danseur sont captés et interagissent avec la production du son — ce qui est particulièrement rendu possible avec les nouvelles technologies, alors ils peuvent être envisagés comme relevant du geste instrumental.

Le geste instrumental parait au premier abord plus simple que le geste musical, en ce qu'il semble davantage possible de le décrire à travers une approche purement fonctionnelle de sa mécanique. 

Dès 1995, Todd Winkler propose de repenser le geste instrumental dans les \glspl{DMI} en proposant des contraintes et des idiomes libérés du modèle de l'instrument acoustique \cite{winkler_making_1995}. Sa formulation de cette problématique est intéressante en ce qu'elle envisage déjà la sonorité des gestes au-delà du modèle acoustique excitation/résonance (``le son du claquement d'une seule main''). Cependant, ses propositions reflètent son attachement au paradigme physique et à la corrélation entre l'énergie du geste et du son : \iquote{Qu'est ce que la musique des doigts ? Qu'est ce que la musique de course ? Quel \textit{est} le son d'\textit{une seule} main qui claque ? On peut répondre à ces questions en permettant à la physicalité du mouvement d'avoir un impact sur le matériel et les processus musicaux. Ces relations peuvent être établies en considérant le corps et l'espace comme des instruments de musique, libérés des relations dans les instruments acoustiques, mais avec des contraintes similaires qui peuvent donner du caractère au son par des mouvements idiomatiques.}\footnote{\iquote{What is finger music? What is running music? What \textit{is} the sound of \textit{one} hand clapping? These questions may be answered by allowing the physicality of movement to impact on musical material and processes. These relationships may be established by viewing the body and space as musical instruments, free from the associations of acoustic instruments, but with similar limitations that can lend character to sound through idiomatic movements.}}





% Jensenius in Musical Gesture : "Based on the above viewpoints, it seems straightforward to define musical gesture as an action pattern that produces music, is encoded in music, or is made in response to music. Qualifications can be added to the term musical gesture whenever needed to avoid misunderstandings. For example, one can speak about sound-producing gestures, sound-modifying gestures, sound-accompanying gestures, sonic gestures, playing gestures, and so on."


% Par ailleurs, certains auteurs font une distinction entre ``geste avec contact'' et ``geste sans contact''. Je propose d'envisager cet aspect d'une manière généralisée en parlant du retour gestuel dans le ``geste interactif''. Cette interaction existe toujours, mais avec plus ou moins d'importance, et plus ou moins de simultanéité. Ce retour peut être de nature physique, vibratoire, acoustique mais également visuel ou kinesthésique.



Guerino Mazzola : \iquote{Gestures are dialogical, live in presence, are circular, elastic, are presemiotic, and are as such already differentiated (being gestural can be ramified into different types).} \cite{mazzola_topos_2018}, p. 59 (852)

Le geste \textit{exprime} (i.e. du latin ex-premo : ``faire sortir en pressant'') quelque chose, mais n'a pas nécessairement une signification.


Le geste + intention ? => le geste ne passe pas nécessairement pas un dessein connu d'avance, il se produit aussi en réaction à la vivacité de la musique produite dans une circularité qui ne laisse pas de place à la mentalisation de l'instant (flow).

%-------------------------------------------
\subsection{Spécificités des DMIs}

\noindent Dans le contexte particulier des lutheries numériques, un certain nombre de problèmes se posent par rapport au geste instrumental sur les instruments acoustiques, liés à des spécificités du numérique que nous avons déjà en partie évoqué dans le chapitre précédent\footnote{De manière plus globale et sociologique, les révolutions technologiques électroniques ont également modifiés les modes de production et de réception de la musique à l'échelle industrielle, entrainant d'autre conséquences que celles listées ci-après, mais qui dépasse le sujet que nous traitons ici. Je renvoie pour ces questions là au travail de TODO : Bacot, Collins, Auslander, Rebecca Bennett, etc.}.

\vspace{-1em}
\begin{itemize}[noitemsep]
\item \textbf{le découplage énergétique} : ce découplage est la différence la plus saillante par rapport aux lutheries acoustiques. Certains de ses aspects sont déjà présents dans les instruments électriques et électroniques, mais l'informatique numérique ajoute un découplage de plus: les signaux (gestuels, sonores) n'y existent plus sous forme analogique et se présentent sous la forme de représentations numériques du signal, c'est-à-dire encodés dans un alphabet.
\item \textbf{la mémoire et le temps différé}: si le terme de ``temps réel'' est abondemment utilisé quand on parle des \glspl{DMI}, il ne faut pas oublier que l'informatique introduit avant tout le temps la possiblité du temps différé, de l'enregistrement exploitable dans le futur.
\item \textbf{la computation} : le traitement algorithmique permet --~ou impose~-- une reconfiguration permanente des modèles et des représentations, qui entraine une instabilité, un métamorphisme des relations entre les données.
\end{itemize}

TODO : intégrer ici le fait que les DMIs ne sont pas des IHMs ?



%%%%%%%%%%%%%%%%%%%%%%%%%%%%%%%%%%%%%%%%%%%%%%%%%%%%%%%%%%%%%%%%%%%%
\section{Limites d'une analyse des DMI comme IHM}
\label{sec:gesture:limitesIHM}

\subsection{La scène et le laboratoire}
Catégories gestuelles établies via une analyse ``de laboratoire'', qui visent à étudier les phénomènes pris isolément, hors du contexte de performance. Il en est ainsi de l'analyse Schaefferienne et de l'écoute réduite, qui bien que très utile pour la compréhension des sons autant que pour le compositeur qui veut les utiliser comme un matériau plastique, ne reflète pas une situation généralisable au public d'un concert.

Un certain nombre de critères ont ainsi été proposés pour améliorer le design des DMIs, souvent en prenant comme modèle l'instrument acoustique classique, tant en terme d'affordance de l'objet, qu'en terme de relation socio-culturelle (todo:trouver plus précis que socio-culturel) pour le cadre qu'il offre à l'instrumentiste et au public (notamment, l'instrument est sur scène, entre les mains de l'instrumentiste, la relation geste-son est lisibile, sa palette sonore est connue d'avance par le public, etc.).
Au delà d'un effet diligence\footnote{todo expliquer}, le désir de retrouver les qualités d'affordance des instruments classiques s'explique par la riche histoire instrumentale dont on souhaite tirer profit pour le design de nouveaux instruments. La tâche s'avère cependant complexe et difficile et parallèlement, les lutheries numériques se sont développées de manière empirique, ``sauvage''\footnote{cf. Max Vandervorst, John Bowers, etc.} en s'adaptant au medium tel qu'il se présente, et en inventant de nouveaux gestes et sans forcément s'inscrire dans un contexte de performance similaire à celui des instruments classiques (instrument hors scène, ou invisible, relation gestuelle et timbre mé-connus, etc.)

%-------------------------------------------
\subsection{Nouvelles IHM, nouveaux gestes}

Notons que dans leur classification, Cadoz et Wanderley \cite{todo} définissent le geste instrumental comme un geste \iquote{appliqué à un objet matériel et en interaction physique avec lui}, ajoutant que \iquote{les gestes nus (empty-handed gestures) ne sont pas des gestes instrumentaux pour la raison qu'il ne possède qu'une fonction sémiotique}.

Depuis, le nombre de capteurs permettant une interaction sans contact physique n'a cessé d'augmenter et de se démocratiser : caméras vidéos, caméras 3D (e.g. kinect, leap motion), capteurs de distance à infra-rouge ou ultra-sons, capteurs photosensibles, gyroscopes et accéléromètres... entrainant une croissance proportionnelle du nombre de DMIs recourant à des gestes sans contact physique.

On perçoit évidemment les limites de cette notion de contact physique, quand il s'agit d'instruments comme le theremin, ou quand des capteurs sans contacts tels que les capteurs de distance par ultra-son ou infra-rouge, les caméras vidéo, les kinect\footnote{interface conçue par Microsoft, qui s'apparente à une caméra fournissant, en plus d'une image vidéo classique, une carte de profondeur de l'image captée.}, leap-motion\footnote{interface conçue par Microsoft, qui s'apparente à une caméra fournissant, en plus d'une image vidéo classique, une carte de profondeur de l'image captée.} et autres radars... Le \textit{geste nu} s'apparente dans ce cas à un geste ergotique (si toutefois cette notion de geste ergotique conserve un sens dans les DMI).

Dans le cas des \glspl{DMI}, la relation entre le geste et le son se pose d'une manière précisément opposée : la relation entre le ``geste de production du son'' et le son n'y est pas causale a priori.

%-------------------------------------------
\subsection{Geste produit, capté, perçu}

cf Benford \cite{benford_performing_2010}

Dans le domaine des IHM, peu d'importance est accordée au geste en dehors de son interaction directe avec l'ordinateur, comme le rapporte Jensenius (Musical Gesture):
\iquote{A gesture is a motion of the body that contains information. Waving goodbye is a gesture. Pressing a key on a keyboard is not a gesture because the motion of a finger on its way to hitting a key is neither observed nor significant. All that matters is which key was pressed}(p. 310)\cite{kurtenbach_art_1990}

Pourtant comme le rappelle Richard Leppert dans \cite{leppert_sight_1993} (cité par \cite{iazzetta_meaning_2000}) souligne à quel point la nature intangible du son et de la musique est polarisée par l'expérience visuelle :
\begin{quotation}
Precisely because musical sound is abstract, intangible, and ethereal [...] the visual experience of its production is crucial to both musicians and audience alike for locating and communicating the place of music and musical sound within society and culture. [...] Music, despite its phenomenological sonoric ethereality, is an embodied practice, like dance and theater." 
\end{quotation}

Il faut également noter que si la réalisation d'un geste technique sur une interface est orientée vers l'efficacité pour la réalisation d'une tâche, \cite{ryan_remarks_1991}

Un geste musical ne consistera donc pas prioritairement à chercher une efficacité pour accomplir une tâche mais 



\textbf{Différence entre geste effectué et geste capté} : un son électroacoustique, par exemple une figure de ``delta'', pourra ainsi avec un même geste de balaiement de la main, et une même interface de captation (e.g. un simple slider), être déclenchée via un seuillage de l'entrée qui déclenche un sample, être produit de manière continue, via un scrub du sample, pourra avoir son intensité sonore fonction de la vitesse du geste ou pas, etc. On voit que pour un même geste, un même son (échantillonné), et un même capteur, les possibilités de relations entre le geste et le son sont multiples. Le pré-geste hors contact avec l'instrument pourra être capté ou non selon le capteur. (e.g. leap motion)



%-------------------------------------------
\subsection{Les musiciens \emph{ne sont pas} des utilisateurs d'instruments}

Un autre biais de l'approche fonctionnelle des \gls{IHM} est lié au fait que l'interaction est souvent définie par la perspective d'une tâche que l'utilisateur doit accomplir. Ce contexte écologique a contribué au développement de tout un champ d'étude dans le domaine des IHM: les \textit{user studies}, transposé dans le domaine de l'ingénierie industrielle sous le nom d'``Experience Utilisateur''\footnote{En abrégé : UX pour User eXperience; le métier de ``UX designer'' étant désormais très répandu dans le domaine du développement logiciel mais également appliqué à d'autres domaines tels que la grande consommation.}.
%---- Figure : Einarsson sculpture ---------
\begin{figure}[!htbp]
	\captionsetup{format=plain}%
	\includegraphics[width=\textwidth]{gfx/03_gesture/instrumentabusers.png}
	\caption[Les instrument ne sont pas des interfaces utilisateur]{Les instrument ne sont pas des interfaces utilisateur. De gauche à droite : Thomas Bonvallet, John Cage, Jimi Hendrix, Paul Simonon, Charlotte Moorman, Sarah Kenchington}
	\label{fig:gesture:abusers}
\end{figure}
%---- Figure : Einarsson sculpture ---------
\noindent Or dans une performance musicale vivante, cette situation n'existe pas. Le musicien ne saurait être défini comme ``l'utilisateur de son instrument''(figure \ref{fig:gesture:abusers}), pas plus que son instrument comme une ``interface utilisateur''. Les musiciens cherchent bien souvent à produire quelque chose à la limite des possiblités de leur instrument\footnote{cf. interview de Nicolas Bernier désignant ``l'infini des possibles'' comme principale motivation l'ayant amené à utiliser les instruments électro-acoustiques.}. Que cela soit le contre fa de la Reine de la Nuit, les partitions impossibles de Brian Ferneyhough, les pianos préparé de John Cage, le scratch sur les platines vinyle ou les mises en larsen de table de mixage de la scène Onkyokei, les exemples abondent dans l'histoire de la musique qui font preuve d'une démarche allant dans l'outrepassement des possibilités instrumentales lors du jeu musical. Dans une conversation durant la dernière conférence NIME, Paul Stapleton utilisait ironiquement le terme \textit{interface abusers} pour souligner l'erreur du terme \textit{interface users}. \footnote{Un exemple significatif de cette différence entre usage et abus de l'instrument est la réduction de la hauteur dans le format MIDI à un paramètres entier borné entre 0 et 127 pour contrôler la hauteur—ce qui excède déjà la tessiture du piano à 88 touches. Les fréquences audibles excèdent largement cette tessiture, et certains genres de musiques électroniques, comme l'\gls{IDM} ou le \gls{glitch}, ont précisément fait de l'utilisation de fréquences suraigues ou infra-basses une de leurs composantes esthétiques.}


nément admis comme des qualités requises pour l'interaction instrumentale.
Le problème de toute catégorisation est qu'elle peine à rendre compte des chevauchements et intersection entre ses catégories. Nommément, les gestes subversifs peuvent être de nature excitatoire tout en jouant sur le \textit{pré-geste} (sound-facilitating) pour lui faire dire le contraire.


D'un certain point de vue, on pourrait dire que cela n'a pas de sens de vouoir jouer d'un \gls{DMI} de manière \textit{transparente} dans la mesure où les gestes sont subvertis à la source même de l'interaction. C'est en quelque sort un mésusage de l'informatique qui consiste à l'utiliser comme si les \glspl{DMI} étaient des synthés analogiques (dans lesquels une certaine continuité énergétique subsiste entre les capteurs et le son).


Il a été souvent déclaré comme critère de design des DMIs qu'ils se devaient d'être très réactifs. Cependant, si cette caractéristique est éminnement présente dans les instruments acoustiques où l'énergie gestuelle est transformée et traduite de manière continue et instantanée dans le son, ce n'est pas le cas des instruments numériques. Mais au lieu de voir cela comme un défaut, considérons cette caractéristique qui en découle : le fait de ne pas constamment devoir agir pour entretenir un son sur un \gls{DMI} libère l'esprit pour s'occuper de gérer des formes à plus long terme. Le live-coding (ou plutôt, les musiques séquencées, de manière générale) est quasi uniquement dans ce mode d'interaction ou les actions sur le clavier n'ont de conséquence sur le son qu'un fois les commandes validées et que l'état du séquenceur permet la prise en considération de la modification du pattern.

%-------------------------------------------
\subsection{Se départir du modèle acoustique}

Les études du geste musical qui ont amenée à la classification ci-dessus ont essentiellement été menée sur des instruments acoustiques, pour lesquels la relation entre le geste et le son est généralement causale, immédiate, et caractérisée, dans la plupart des cas, par un transfert énergétique proportionnel.
Par ailleurs et jusqu'à récemment, la tradition musicale de l'IRCAM de composition pour instruments acoustiques classiques et la relégation quasi-systématique de la partie électronique des compositions\footnote{synchronisée—et non jouée, par des \gls{RIM} et non des musiciens} dans l'ombre de l'arrière-scène n'a guère favorisé la considération des interfaces électroniques et des nouveaux gestes qui leur étaient propres en tant qu'instrument de performance musicale à part entière.

%-------------------------------------------
\subsubsection{continuum énergétique}

Claude Cadoz accorde une grande importance à la question du continuum énergétique, et considère que l'interaction physique ``par contact'' avec l'instrument est une condition nécessaire pour pouvoir qualifier un geste d'instrumental. Non seulement le continuum énergétique doit être assuré pour que l'énergie gestuelle soit retrouvée dans le son produit, mais l'interface doit opposer un retour d'effort dynamique pour assurer la bi-directionnalité du canal gestuel.

\iquote{Nous touchons ici un point crucial, car c’est précisément cette unidirectionalité qui rend ces systèmes inaptes à assurer la fonction ergotique du canal gestuel. La présence simultanée de capteurs et d’effecteurs dans le dispositif d’interface gestuelle est en effet une condition nécessaire à cette fonction.}\cite{cadoz_musique_1999}

La description des fonctions du geste instrumental pour les instruments acoustique l'amène à poser la fonction ergotique (celle qui fournit l'énergie) comme base nécessaire pour le design des interfaces musicales. Cette condition requise de l'instrumentalité a éminemment poussé l'équipe de l'\gls{ACROE} dans des développements singuliers et intéressants à plus d'un titre. 

On ne saurait pour autant, en considérant le paysage des pratiques musicales actuelles avec des \glspl{DMI}, accepter un tel critère comme une condition de leur existence. Par ailleurs, il semble paradoxal de vouloir préserver la continuité énergétique lors de l'usage d'un dispositif numérique qui, par nature, rompt cette continuité\footnote{Ajourdh'hui encore, malgré de récents développements notamment sur la base de système Hamiltonien à ports (cf Hélie), le bilan énergétique n'est pas préservé lors de la quantification et de l'échantillonage de signaux analogiques. TODO : développer un peu ou supprimer}. 



Si l'introduction de l'électricité a entrainé un découplage énergétique, l'introduction de l'informatique a opéré un découplage de la causalité. Le son numérisé n'est pas directement le courant électrique qui va \textit{in fine} faire bouger la membre d'un haut-parler, c'est une image, une représentation de ce signal. Il en va de même pour les gestes numérisés. Il n'y a plus de continuité (même électrique!) dans l'interaction énergétique, mais une succession d'opérations discrètes qui simule une continuité. 


\subsubsection{La latence du temps-réel}

Remarquons à ce propos que si l'on a beaucoup utilisé le terme de ``temps-réel''\footnote{Comme par exemple dans la dénomination de l'équipe de l'IRCAM dévolue au développement d'objets pour le traitement du son en ``temps-réel'', nommée successivement ``Equipe Application Temps-Réel'', puis ``Interactions Musicale Temps-Réel'', afin de finalement abandonner ce terme de temps-réel pour s'appeler ``Sound Music Movement Interaction''.} lorsque sont arrivés les premiers synthétiseurs permettant un calcul du son à une fréquence plus rapide que celle de son échantillonage, il ne faut pas oublier que l'ordinateur n'a pas tant introduit le ``temps-réel'' que le ``temps différé''.
Mais il a fallu tant d'effort pour arriver à réduire cette différance en deça des seuils perceptible, et produire les premiers instruments numériques s'approchant de l'immédiateté naturelle des instruments acoustiques, qu'elle a pour ainsi dire éclipsé la disposition des machines à produire du temps différé dans une perspective instrumentale.


Large développement d'outils pour la gestion "offline" de la musique (déplacement, copié/collé,etc) et de l'ergonomie de ces outils.
Hybridation des instruments entre du controle instrumental direct ("traditionnel") et des techniques issues de la production musicale offline.


accorde à la musique le droit de \iquote{tromper l'oreille} (et la vue).

\subsubsection{lisibilité, répétabilité, fiabilité, fidélité}

Pendant longtemps (TODO : combien?), les instrumentistes utilisant des \glspl{DMI} ont été cachés derrière des machines, à la position souvent occupé par l'ingénieur du son, ne laissant rien voir ou si peu de ce qu'ils faisaient vraiment. Pire, ils se trouvaient suspectés, quand ils étaient sur scène, de faire semblant de jouer. \cite{cascone_aesthetics_2000}


Expliquer en quoi le découplage énergétique, qui a amené à "un sens de la discontinuité avec la tradition, aliénation et manque de compréhension par le public en ce qui concerne ce que l'instrumentiste ou l'instrument fait en réalité". (T Magnusson, in \cite{magnusson_sonic_2019} pp. 61) a amené à une contre-réaction faisant passer la lisibilité 


\subsubsection{transparence}

Louange de la transparence : \cite{fels_mapping_2002} 
\iquote{We consider transparency as a predictor for expressivity. (...) We identify transparency as a quality of a mapping.}

\iquote{Another basic need is that the software provide correspondences between input data and output sound that are sufficiently intuitive for both performer and audience.}\cite{dobrian_e_2006}



\subsubsection{fidélité}

La notion de ``fidélité'' est également vantée dans les dispositifs technologiques comme un gage de qualité. Fernando Iazzetta \cite{iazzetta_meaning_2000} analyse la manière dont la notion de fidélité dans l'industrie musicale, orginalement utilisée pour désigner la capacité d'un enregistrement à reproduire les qualités sonores de la performance originale, a progressivement vers une notion de fidélité non plus basée sur le son original lui-même, mais établie en fonction de la technologie d'enregistrement disponible. La situation actuelle (de la musique pop en particulier) dans laquelle l'écoute, voire la production en studio, d'un enregistrement précède l'écoute de la performance elle-même amène ainsi de nombreux musiciens à une situation paradoxale, consistant à chercher à reproduire dans leurs performances live les mêmes qualités sonores que celles de leurs disques.


de la relation geste/son comme un critère pertinent de design instrumental.


\subsubsection{incompréhension}
Parmi les aspects qui reviennent le plus souvent pour décrire la relation qui s'établit dans une performance entre le musicien et l'auditeur, il est celui de la ``compréhension'' par le public de ce que le musicien fait sur scène (e.g. \cite{schloss_using_2003}, \cite{fels_mapping_2002}) et l'incapacité à faire soi-même cette performance.

Pourtant, bien souvent, des performances musicales tout à fait obscures et incompréhensibles, dépourvues de virtosité gestuelle démonstrative, m'ont davantage intéressé que les gesticulations visibles et prévisibles d'une performance, même virtuose. 

\iquote{Si ça se trouve, cette notion que dans quelques années, ``tout sera possible avec la technologie'' fera que cela sera compliqué de créer un mystère entier et profond, parce que les gens du coup diront ``oui, j'en ai entendu parler, maintenant on peut faire ça''.
 J'ai un ami (...) qui a fait voler un espèce de morceau de tulle au dessus des gens avec des principes mécaniques, et beaucoup de gens disaient ``ah oui, c'était incroyable mais je pense que c'était un drône'', alors que pas du tout. Mais je me suis dit, c'est vrai que d'ici quelques années, un objet qui vole tout seul en silence dans l'espace, n'aura plus le même pouvoir de mystère qu'il y a quelques années.} Yann Frish dans \url{https://www.youtube.com/watch?v=5BqHXbQC36M}



%%%%%%%%%%%%%%%%%%%%%%%%%%%%%%%%%%%%%%%%%%%%%%%%%%%%%%%%%%%%%%%%%%%%
%%%%%%%%%%%%%%%%%%%%%%%%%%%%%%%%%%%%%%%%%%%%%%%%%%%%%%%%%%%%%%%%%%%%
%%%%%%%%%%%%%%%%%%%%%%%%%%%%%%%%%%%%%%%%%%%%%%%%%%%%%%%%%%%%%%%%%%%%
\section{Le modèle ergotique: héritage acoustique}


\subsection{Continuum énergétique : classification de Cadoz}

\noindent Claude Cadoz fait partie des pionniers dans l'analyse du geste instrumental, en prenant en compte les spécificités propres à cette pratique gestuelle et en les mettant en perspectives des technologies informatiques. En particulier, comme son nom l'indique, le geste instrumental n'est pas un ``geste nu'' mais se retrouve couplé à un instrument qui polarise les termes de leur interaction (sans toutefois les définir totalement). Le geste est instrumentalisé, médiatisé, et sa perception est contenue (de manière indissociable ?) dans la perception globale de l'instrument et de la réalité (sonore) qu'il engendre.\\
\indent Dès 1978, Cadoz décrit avec Jean-Loup Florens dans un article séminal\footnote{``Fondement d’une démarche de recherche informatique / musique'' \cite{cadoz_fondement_1978}} un grand nombre des  enjeux soulevés par le contrôle gestuel de la synthèse audio-numérique. Ils y explicitent notamment les caractéristiques présentées précedemment et proposent, dans la continuité des travaux de Pierre Schaeffer, la notion d'\textit{objet gestuel}\footnote{Le titre de la section : ``L'objet gestuel - champ expérimental'' laisse entendre que tout reste à y faire, et ce concept sera au final assez peu repris par Cadoz.}. Bien que la rupture du numérique et notamment \iquote{l'artifice [du continuum énergétique]} y soit exposée avec lucidité, Cadoz et Florens remarquent aussi que : \iquote{La perception des objets musicaux a ses racines, ses références, ses codages dans la pratique traditionnelle}\footnote{ibid.}. C'est probablement cette volonté d'ancrage dans la pratique traditionnelle acoustique qui amène Cadoz à définir dès 1981 \cite{cadoz_synthese_1981} des catégories gestuelles basées sur la notion de continuum énergétique, qui polariseront fortement les développements de l'\gls{ACROE}. Claude Cadoz, Annie Luciani, Jean-Loup Florens et Sylvie Gibet définiront progressivement la nomenclature suivante pour décrire les différents types de \iquote{gestes instrumentaux} :
\vspace{-1em}
	\begin{itemize}[noitemsep]
		\item \textbf{gestes d'excitation} qui fournissent l'énergie qui sera présente dans le son \textit{in fine}. Ils peuvent être de nature ``continue'', quand le son et le geste co-existent (e.g. frottement de l'archet, souffle dans un instrument à vent), ou de nature ``instantanée'', si le son commence quand le geste finit (e.g. percussion, pincement de corde) \cite{cadoz_gesture_2000};
		\item \textbf{gestes de modification} venant modifier les propriétés de l'instrument. Une distinction est apportée par la suite dans \cite{cadoz_synthese_1983} entre modifications ``paramétriques'', telles que le vibrato, et les modifications ``structurelles'' (e.g. ajout d'une sourdine sur une trompette, sélection d'un jeu d'orgues, etc.).
		\item \textbf{gestes de sélection}, ajoutés à cette nomenclature en 1984 dans \cite{luciani_modelisation_1984}, ils consistent à choisir parmi plusieurs éléments similaires d'un instrument (e.g. quelle touche de piano, quel corde de harpe, quel fût de batterie, etc.).
		\item \textbf{gestes de polarisation ou de maintien}, ajoutés en en 1999 dans \cite{cadoz_gesture_2000}, ils consistent à assurer des conditions normales de fonctionnement à l'instrument (e.g.le geste du bras qui assure un niveau de pression suffisant pour le jeu de cornemuse).
\end{itemize}

\subsection{Intérêts et limites du modèle ergotique}

\noindent Cette classification du geste instrumental ``producteur de son'' peut sembler relativement bien adaptée aux instruments acoustiques. Elle est devenue une référence sur le sujet et abondamment citée dans la littérature des \gls{NIME}. Assez paradoxalement, elle a été prise comme modèle pour le design de l'interaction des \glspl{DMI} développés à l'\gls{ACROE} mais également par de nombreux autres luthiers numériques (e.g. \cite{arfib_strategies_2002}, \cite{schwarz_sound_2012}), alors même que ses auteurs précisent dans \cite{cadoz_geste_1994, cadoz_gesture_2000} que :
\vspace{-1em}
	\begin{itemize}[noitemsep]
		\item il est nécessaire que l'instrument soit stable durant la performance;
		\item un continuum énergétique doit exister entre le geste et le phénomène perçu;
		\item le geste doit être appliqué à un objet matériel et il doit existe une interaction physique avec lui (le cas du Theremin étant considéré comme une exception rare).	
\end{itemize}
\noindent Or, ces trois aspects sont précisément mis en défaut dans le cas des \glspl{DMI}: le continuum energétique est \textit{a priori} rompu, les instruments sont sujets à des possibles reconfigurations dynamiques\footnote{qu'elles soient souhaitées ou dûes au contexte d'obsolescence (cf. \ref{sec:ephemeral:ephemerality_in_musical_context})}, et le geste est bien souvent capté en dehors de tout contact physique\footnote{Par des capteurs de distance, accéléromètres, caméras, etc. Voir le chapitre \ref{ch:interfaces}.}\\
\indent Cette relation pluri-millénaire étant cassée, l'ambition de l'\gls{ACROE} a été de tenter de la recréer artificiellement par des systèmes de capteurs, d'actioneurs et des stratégies de mapping servant la définition de cette relation. Ces recherches ont notamment abouti au système \gls{CORDIS-ANIMA}, dispositif pionnier en matière de retour d'effort et de synthèse par modèle physique, qui n'a malheureusement pas connu une grande utilisation hors du laboratoire. On se garderait donc bien de dire que l'inaptitude de cette catégorisation à décrire le geste instrumental dans le cas des \glspl{DMI} ait été un obstacle à l'avancement des travaux de l'\gls{ACROE}. Elle a au contraire été une direction idéologique\footnote{Et d'une certaine manière, on peut aussi y voir un choix esthétique, Cadoz étant aussi compositeur.} motrice pour le développement de modèles physiques et de systèmes à retour d'effort avancés.\\
\indent Pour autant, après bientôt quarante ans de pratiques musicales numériques, nous pouvons observer facilement que cette direction n'était pas la seule possible et que de nombreuses stratégies de jeu se sont développées, sans être freinées par l'absence de continuum énergétique, ni l'instabilité de l'instrument, ni l'absence de contact --~au contraire, les artistes ont embracés ces artéfacts à bras le corps. L'idée du continuum énegétique est donc insuffisante pour comprendre les termes du geste instrumental numérique et ses traductions en terme de lutherie\footnote{Il ne faudrait cependant pas déduire de cette critique que Cadoz n'est pas conscient des limites de ce modèle; même s'il leur refuse généralement le statut de ``geste instrumental'' au sens qu'il a donné à ce terme, il a contribué dans de nombreux articles à analyser de manière nuancée d'autres aspects du geste présentés ci-après.}.


\section{Expressivité et sémiologie du geste nu}

\noindent Dans son analyse des gestes de Glenn Gould au piano \cite{delalande_geste_1988}, François Delalande a établi une autre typologie de geste abondamment citée\footnote{Notons toutefois que si Delalande utilise ces différents termes, il ne les présentent pas explicitement comme des catégories gestuelles absolues, comme peut le faire Cadoz, pas même comme une liste, comme ils sont souvent présentés --~et ici encore.}, sur \iquote{au moins trois niveaux, qui vont du du purement fonctionnel au purement symbolique}. Il distingue ainsi :
\vspace{-1em}
\begin{itemize}[noitemsep]
	\item \textbf{des gestes effecteurs} responsables de la production du son (le toucher du clavier dans le cas de Gould) et correspondant, d'une certaine manière, à la notion de geste instrumentale de Cadoz définit ci-avant;
	\item \textbf{des gestes accompagnateurs}, qui engagent le corps entier et en apparence moins indispensables à la production du son;
	\item \textbf{des gestes évocateurs} perçus dans la musique par l'auditeur, tels qu'un appui de phrase ou une envolée, et qui ne semblent pas directement liés aux mouvements du corps. Delalande évoque notamment ces gestes comme traduction possible d'un imaginaire associé à la musique, tel qu'une dimension orchestrale dans le jeu pianistique de Gould.
\end{itemize}
\noindent Marcello Wanderley a mis en évidence que les gestes accompagnateurs, qu'il appelle \iquote{gestes ancillaires} ou \iquote{gestes non-évidents}\footnote{\iquote{Non-obvious gestures}. Voir \cite{wanderley_non-obvious_1999}.} avaient une influence mesurable sur le résultat sonore, par exemple en terme de projection acoustique. Il est par ailleurs assez évident que les doigts ne sont pas un simple système mécanique indépendant des mouvements du reste du corps et que la performance d'une phrase musicale requiert des inflexions \textit{facilitées} par ces mouvements du corps. Par ailleurs, la perception du résultat sonore est influencée par la perception visuelle\footnote{Un exemple connu en est l'effet McGurk, mettant en évidence l'interférence entre l'audition et la vision lors de la perception de la parole. Cf. \cite{macdonald_visual_1978}).}. Les gestes accompagnateurs, outre leur influence sur le jeu et le son, ont également une influence sur la manière dont l'auditeur les perçoit. 

\indent Dès 1995, Todd Winkler proposait de repenser le geste instrumental dans les \glspl{DMI} en proposant des contraintes et des idiomes libérés du modèle de l'instrument acoustique \cite{winkler_making_1995}. Sa formulation de cette problématique est intéressante en ce qu'elle envisage déjà la sonorité des gestes au-delà du modèle acoustique excitation/résonance (``le son du claquement d'une seule main''). Cependant, ses propositions reflètent son attachement au paradigme physique et à la corrélation entre l'énergie du geste et du son : \iquote{Qu'est ce que la musique des doigts ? Qu'est ce que la musique de course ? Quel \textit{est} le son d'\textit{une seule} main qui claque ? On peut répondre à ces questions en permettant à la physicalité du mouvement d'avoir un impact sur le matériel et les processus musicaux. Ces relations peuvent être établies en considérant le corps et l'espace comme des instruments de musique, libérés des relations dans les instruments acoustiques, mais avec des contraintes similaires qui peuvent donner du caractère au son par des mouvements idiomatiques.}\footnote{\iquote{What is finger music? What is running music? What \textit{is} the sound of \textit{one} hand clapping? These questions may be answered by allowing the physicality of movement to impact on musical material and processes. These relationships may be established by viewing the body and space as musical instruments, free from the associations of acoustic instruments, but with similar limitations that can lend character to sound through idiomatic movements.}}\\
\indent En analysant les gestes induits par du son, Rolf Inge Godøy a souligné la manifestation de deux autres types de gestes d'accompagnement : les ``gestes de tracé sonore''\footnote{\iquote{sound-tracing gestures}, cf \cite{godoy_exploring_2006}.} qui suivent le contour des morphologies sonores (e.g. le contour mélodique) et les ``gestes d'imitation des gestes de production du son''\footnote{\iquote{mimicry of sound-producing gestures}, cf. \cite{godoy_playing_2005}.} en prenant notamment en exemple les performance d'\textit{air-guitar}\footnote{Le \textit{air guitar} est une activité qui consiste à mimer le geste d’un guitariste, typiqument de guitare électrique dans un groupe de rock ou de métal, sans avoir l’instrument en main, dans une sorte de playback instrumental.}. Il est intéressant de s'arrêter ici sur ces deux types de geste. En effet, ils ne sont pas \textit{a priori} des gestes instrumentaux dans la mesure où ils ne sont pas effectués par quelqu'un en situation de jeu instrumental (mais sans aucun doute par quelqu'un qui \textit{musique}, au sens de Christopher Small). Il s'agit là d'un geste qui relève en partie d'une forme de théâtralité mais aussi, comme l'explique Godøy, d'une forme de ``geste d'écoute'' : \iquote{(...) nous pouvons donner un sens à ce que nous entendons parce que nous devinons comment les sons sont produits (...) des études récentes sur la neuro-imagerie semblent appuyer l'idée que la perception est un process actif de la cognition motrice.} \footnote{\iquote{(...) we can make sense out of what we hear because we guess how the sounds are produced. (...) recent neuro-imaging studies seem to support the idea of perception as an active process involving motor cognition.}\cite{godoy_exploring_2006}}. Godøy développe cette idée dans ce qu'il appelle des \iquote{objets gestuels-sonores}\footnote{\iquote{Gestural-sonorous objets} décrits dans \cite{godoy_gestural-sonorous_2006}}, qui étende la typologie Schaefferienne des ``objets sonores''\cite{schaeffer_traite_1966}, pour inclure l'exploration de gestes associés aux différents objets sonores.\\
\indent Cette fonctionalité gestuelle, à rapprocher des \textit{gestes évocateurs} évoqués par Delalande, présente ici l'intérêt de décrire des gestes \textit{physiques} exprimant des relations \textit{imaginaires} entre le geste et le son. La morphologie du geste y découle de l'écoute musicale, renversant ainsi la perspective de causalité entre le geste et le son, telle qu'elle existe dans les instruments acoustiques. Dans le cas des \glspl{DMI} où cette relation est \textit{a priori} dépourvue de causalité, ces catégories s'avèrent très intéressantes, en ce qu'elles nous renseignent sur des axes possibles sur lesquels cette relation peut se construire en l'absence de toute contrainte physico-énergétique. C'est sur la principe de telles correspondances qu'ont été développés des systèmes de contrôle musical par suivi de gestes\footnote{Voir les travaux menés au sein de l'équipe \textit{Interaction Son Musique Mouvement} (ISMM) de l'\gls{IRCAM}, en particulier le projet \textit{Modular Musical Objects} \cite{caramiaux_mapping_2014, francoise_motion-sound_2015}, ou ceux de Rebecca Fiebrinks, en particulier le projet Wekinator, \cite{fiebrink_wekinator:_2010} à la Queen Mary Univeristy de Londres}.\\
\indent Le suivi de geste et sa reconnaissance par apprentissage-machine sur la base d'un vocabulaire de formes gestuelles pré-définies, rend possible le design d'une interaction à mi-chemin entre ce que Cadoz appelle \textit{gestes de sélection} et \textit{gestes de modulation}. Les systèmes d'apprentissage permettent en effet de calculer non pas une catégorie mais la probabilité de présence de ces catégories dans le geste, et leur écart par rapport aux formes typiques pré-définies. Les capacités d'interpolation de la machine permettent ici de reconstruire un espace continu à partir d'un espace catégoriel, à l'inverse de ce qu'on fait souvent sur les instruments acoustiques, à savoir le striage d'un espace continu en des catégories discrètes (touches de piano, fretes de guitare, etc.)


%%%%%%%%%%%%%%%%%%%%%%%%%%%%%%%%%%%%%%%%%%%%%%%%%%%%%%%%%%%%%%%%%%%
%%%%%%%%%%%%%%%%%%%%%%%%%%%%%%%%%%%%%%%%%%%%%%%%%%%%%%%%%%%%%%%%%%%%


\section{Geste programmé, geste re-sonné}
\label{sec:gesture:instrumental_to_musical}
%-------------------------------------------
\subsection{L'outil comme externalisation de la mémoire}
\label{sec:gesture:instrumental_to_musical:externalisation}

\noindent Dans son essai ``Le geste et la parole'' paru en 1964 \cite{leroi-gourhan_geste_1964}, le paléo-anthropologue André Leroi-Gourhan met en lumière la manière dont l'invention et l'utilisation d'outils techniques contribuent à l'évolution de l'humain, par une d'externalisation progressive des processus opératoires dans les outils :\iquote{Au cours de l’évolution humaine, la main enrichit ses modes d’action dans le processus opératoire. L’action manipulatrice des Primates, dans laquelle geste et outil se confondent, est suivie avec les premiers Anthropiens par celle de la main en motricité directe où l’outil manuel est devenu séparable du geste moteur. A l’étape suivante, franchie peut-être avant le Néolithique, les machines manuelles annexent le geste et la main en motricité directe n’apporte que son impulsion motrice. Au cours des temps historiques la force motrice elle-même quitte le bras humain, la main déclenche le processus moteur dans les machines animales ou les machines automotrices comme les moulins. Enfin au dernier stade, la main déclenche un processus programmé dans les machines automatiques qui non seulement extériorisent l’outil, le geste et la motricité, mais empiètent sur la mémoire et le comportement machinal.} \cite{leroi-gourhan_geste_1964} pp 41-42.\\
\indent Il note ainsi que l’externalisation des facultés de l'humain s’est étendue à tous ses organes, jusqu'aux fonctions cérébrales de la mémoire, et prédit les opérations de computabilité rendue possible par le numérique :
\iquote{Les fichiers à perforations sont des machines à rassembler des souvenirs, elles agissent comme une mémoire cérébrale de capacité indéfinie, susceptible, au-delà des moyens de la mémoire cérébrale humaine, de mettre chaque souvenir en corrélation avec tous les autres.} \cite{leroi-gourhan_geste_1964} p 74.

TODO cf. Magnusson outil épistémique

%-------------------------------------------
\subsection{Le geste programmé}
\label{sec:gesture:instrumental_to_musical:geste_programme}

\noindent Le mouvement qui anime le son peut être réalisé explicitement par un instrumentiste humain ou bien produit de manière automatisée par la machine; on pourra alors parler de gestes ``programmés''. Leur définition peut être ``extensive'', par exemple sous la forme d'enregistrements (samples, courbes d'automation, etc. (cf. Figure \ref{fig:gesture:automation})) ou ``intensive'', c'est-à-dire définie par une règle qui permet à un processus de la générer)\footnote{Sur les notions de notation ``intensive'' et ``extensive'', voir Giavitto \cite{giavitto_du_2014}}. Si toutefois la définition du geste implique que le mouvement soit associé à une intention, on ne peut prêter une intention à la machine qu'à travers la ``programmation'' de ce mouvement machinique par le compositeur/luthier numérique.\\
%------------------ Figure : geste programmé ---------------------
\begin{figure}[!htbp]
	\captionsetup{format=plain}%
	\centering
	\begin{minipage}[t]{0.48\textwidth}
		\includegraphics[width=\linewidth]{gfx/03_gesture/AbletonLiveAutomation_72dpi.png}
		\caption{Une courbe d'automation dans le logiciel Ableton Live}
		\label{fig:gesture:automation}
	\end{minipage}
	\hspace{.02\linewidth}
	\begin{minipage}[t]{0.48\textwidth}
	  \includegraphics[width=\linewidth]{gfx/03_gesture/EnriqueThomas-TangibleScore_72dpi.jpg}
		\caption{Partition tangible d'Enrique Tomás}
		\label{fig:gesture:tangible_score}
	\end{minipage}
\end{figure}
%------------------ Figure : geste programmé ---------------------
\indent Stiegler développe le concept de ``gramme'' comme \iquote{corps organisé de signes et de symboles} et emprunte à Sylvain Auroux \cite{auroux_revolution_1994} le concept de ``grammatisation'' comme \iquote{le processus par lequel le continuum temporel des comportements humains est transformé en un spatial discret, qui permet de les intégrer dans les outils}. Ainsi en est-il de l'informatique, qui dissocie en symboles et en catégories discrètes ce qui est continu et intégré dans le geste, comme dans le son. Pour Stiegler, les objets sont des enregistrements\footnote{Stiegler utilise les termes de ``rétentions tertiaires'' pour décrire cette inscription de la mémoire dans les objets, afin de la mettre en perspective des ``rétentions primaires'' que sont la conscience du flux temporel et les ``rétentions secondaires'' que sont les souvenirs qui constituent l'expérience d'un individu.}, dans lesquels la mémoire de ce que nous faisons et de ce que nous connaissons est déposée sous la forme d'une mémoire technique.\\
\indent Le concept de Stiegler fait également écho à la notion de ``diagramme'', proposée par Gilles Deleuze dans son étude sur la peinture de Francis Bacon \cite{deleuze_francis_1981} : \iquote{Le diagramme, c'est donc l'ensemble opératoire des lignes et des zones, des traits et des taches asignifiants et non représentatifs. Et l'opération du diagramme, sa fonction, dit Bacon, c'est de suggérer. Ou, plus rigoureusement, c'est d'introduire des «~possibilités de fait~» : langage proche de celui de Wittgenstein. Les traits et les taches doivent d'autant plus rompre aevc la figuration qu'elles sont destinées à nous donner la Figure. C'est pourquoi elles ne suffisent pas elle-mêmes, elles doivent être «~utilisées~» : elles tracent des possibilités de fait, mais ne constituent pas encore un fait (le fait pictural).}\\
\indent On peut voir un parallèle frappant entre cette notion de diagramme et les fonctions assumées à la fois par l'instrument et par la partition musicale. L'instrument de musique contient ainsi l'enregistrement de la théorie musicale qui lui est propre (ses ``traits et ses taches'' que représentent son organisation des hauteurs, sa signature timbrale, son ergonomie, etc.) et que le luthier lui imprime. De même, on peut voir la partition comme un ``enregistrement'' de la pensée et du travail du compositeur, une trace de ses ``gestes sédimentés'' comme le dit Jean-Paul Olive dans \cite{olive_expression_2013}. La composition et la lutherie sont deux formes d’écriture diagrammatiques du geste et du son, qui s’inscrivaient jusqu’alors (avant le numérique) sur des médiums distincts: papier pour la composition, matériaux physiques pour la lutherie. Le numérique offre un médium commun qui permet leur recomposition mutuelle: l'instrument est ``composé'' \cite{schnell_introducing_2002}, la partition est ``instrumentalisée''\footnote{Voir à ce sujet le travail explicite de Enrique Tomás sur les ``partitions tangibles'' \cite{tomas_tangible_2014}} et les gestes de l'expression compositionnelle et de l'expression performative s'interpénètrent \cite{dobrian_e_2006}.\\
\indent Ces gestes programmés ne sont pas de simple enregistrements linéaire à repoduire tels quels, mais des modèles complexes et dynamiques\footnote{Cette idée est développée au chapitre \ref{sec:algorithms:MID}} qui invitent à l'interaction, au jeu\footnote{Cf. les propos de Stiegler dans \cite{stiegler_circuit_2004} \iquote{Quant à la duction de l'instrumentiste, elle vient retemporaliser ce qui ne peut être que spatial : le travail de la composition, ce n'est que spatial, c'est du temps spatialisé, et en cela, essentiellement en défaut d'être. C'est du virtuel pur. C'est du temps discrétisé et détemporalisé dans cette mesure. Discrétisé, il devient manipulable dans sa détemporalisation temporaire telle que la pratique le com­positeur, mais il n'est que virtuel. Il ne peut devenir actuel qu'avec l'interprète, qui doit le re-temporaliser.}}, par le musicien qui n'est pas, justement, un \textit{utilisateur} qui démarre, par exemple, la lecture d'un enregistrement audio. La performance musicale consiste justement à faire entendre ce qui n'est pas calculable comme le dit Stiegler \cite{stiegler_circuit_2004}: \iquote{Un musicien, c'est quelqu'un qui d'abord entend, c'est-à-dire qu'il est primordialement affecté par l'oreille, une oreille qui a cependant des yeux et des mains, et un corps qui les relie. Il ne se contente pas de calculer. Il peut calculer, il doit même calculer, mais s'il le fait, c'est pour donner à entendre ce qu'il a lui-même entendu comme l'incalculable même.}\\
\indent Ce jeu de la main et de l'oreille, qui vient mettre en mouvement ces \textit{gestes programmés} par un geste expressif incalculable, m'amènent à introduire l'idée de geste de ``re-sonnance'', qui se départit du modèle acoustique d'excitation/modulation, en ce qu'il nourrit une relation avec le son qui n'est pas simplement causale mais intègre l'idée d'agencement compositionnel et d'aller-retour entre la temporalité du geste re-sonnant et celle, intrinsèque au geste programmé.

%-------------------------------------------
\subsection{Le geste de re-sonnance}

\noindent Si les gestes programmés s'apparentent au résultat d'un processus de ``composition instrumentale''\footnote{Ce processus qui se décline sur différentes échelles temporelles est toutefois réalisables durant le temps même de la performance, en particulier dans le \textit{live-coding}.}, comment donc nommer ces gestes qui permettent de faire sonner un \gls{DMI} ?\\
\indent On ne peut se réduire à les nommer ``gestes d'excitation et de modulation'' (même si la métaphore qui les sous-tend peut s'appuyer sur cette idée), car de nombreux gestes n'évoquent en rien cette dimension physique (e.g. l'ouverture d'un fader de volume).\\
\indent On pourrait utiliser le terme de ``gestes effecteurs'' de Delalande, mais il faudrait les coupler d'une part à un processus dynamique, et d'autre part à une métaphore définissant leur logique. Or dans le cas de l'analyse du jeu pianistique de Gould, le piano est déjà sa propre métaphore, car l'instrument acoustique \textit{est} sa propre interface et son propre modèle intermédiaire à la fois\footnote{Ce qui n'empêche pas d'autres métaphores extra-pianistiques de se superposer à l'instrument et d'en orienter les gestes effecteurs, comme le note Delalande : \iquote{(...) les différents touchers sont pour lui [Gould, NdE] l'équivalent d'une orchestration pour différencier les parties polyphoniques; ainsi le staccato joue-t-il le rôle des \textit{pizzicati} de violon et le grand \textit{legato} de la basse, celui des violoncelles. Il n'est donc pas exclu que certains gestes puissent être dictés par cette imagination orchestrale (...)}}.\\
\indent La notion ``d'objet sonore-gestuel'' de Rolf Inge Godøy s'approche le mieux de l'idée présentée ici, en tant qu'elle intègre cette notion de métaphore intermédiaire entre le geste et le son, mais d'une part sa nature ``d'objet'' semble plutôt s'appliquer au modèle intermédiaire lui-même qu'au geste, et d'autre part Godøy semble (malheureusement) en limiter le cadre à une relation de congruence spectromorphologique entre le geste et le son (avec le dessein d'offrir, là-encore, une lisibilité de la relation énergétique).\\
%---- Figure : Guqin ---------
\begin{figure}[!htbp]
	\captionsetup{format=plain}%
	\includegraphics[width=\textwidth]{gfx/03_gesture/Guqin-hand01w.jpg}
	\caption[Le Tayin da quanji : métaphore poétique et animale dans la pédagogie du Guqin]{Un feuillet du \textit{Tayin da quanji} : les gestes instrumentaux du Guqin y sont décrits par un aphorisme poétique, et illustrés par le mouvement d'un animal.}
	\label{fig:gesture:guqin}
\end{figure}
%---- Figure : Guqin ---------
\indent On pourra élargir ce cadre métaphorique au delà de la spectromorphologie en s'inspirant d'un système métaphorique un peu plus ancien et plus ouvert: le ``Tayin da quanji''\footnote{``L’Encyclopédie des sons suprêmes'', ouvrage ayant probablement vu le jour durant la dynastie Song, mais n'ayant survécu qu'à travers diverses éditions, une datant du \siecle{16}~siècle. Une reproduction est disponible en ligne : \url{http://www.silkqin.com/02qnpu/05tydq/ty3.htm}. Voir \cite{picard_chine:_1991}.} décrivant la relation gestuelle-sonore pour l'apprentissage du Guqin dans un ensemble de feuillets comprenant une illustration de la position de main, associée à l'image d'un animal et d'un texte poétique condensant l'esprit du mouvement (cf. figure \ref{fig:gesture:guqin}), par exemple : ``A la manière d'une grue qui danse parce qu'elle est effrayée par une brise.'' Considérons maintenant un hypothétique \gls{DMI} dont le modèle intermédiaire représenterait la danse de la grue durant la dynastie Song; le geste de re-sonnance d'un tel instrument pourrait consister à jouer la brise qui effraie la grue. Ce geste pourrait alors se concrétiser de différentes manières, que cela soit en simulant les mouvements du vent par le mouvement d'un stylet sur une tablette graphique, par le mouvement des mains munies d'accéléromètres, en soufflant dans un microphone, ou encore en pinçant des cordes sur lesquelles --~s'il est possible d'y évoquer la danse de la grue au Guqin~-- il serait sûrement possible d'évoquer le mouvement du vent.

\noindent J`utiliserai donc ici le terme de ``gestes de re-sonnance'' pour désigner un geste qui consiste à faire sonner un \gls{DMI} basé sur un ``geste programmé'', c'est-à-dire un modèle dynamique préalablement enregistré, encodé dans le \gls{DMI} et définissant son comportement. Son apparente similitude avec le terme \textit{résonance} n'est pas fortuite : le geste de re-sonnance a affaire à un processus qui possède sa propre dynamique avec lequel il doit s'accorder (ou non) et se mettre ``en résonance''.  

\indent Le ``geste de re-sonnance'' s'appuie donc sur un \textit{modèle intermédiaire} projeté dans l'imaginaire par \textit{une métaphore} : modèle physique appelant des gestes d'excitation, modèle de lecture de vinyle appelant le \textit{scratching}, modèle de robinets-à-son\footnote{Je reprend ici l'expression utilisée par François Dumeaux pour décrire une partie de son interface de jeu. Cf. annexe \ref{appendix:dumeaux}.} controlés par l'ouverture de \textit{faders} sur une table de mixage, modèle cartographique (e.g. interpolation par boules) définissant un terrain à parcourir sur une tablette, modèle de grue effrayée dansant dans la brise, etc. Le geste de re-sonnance se déploie dans un espace polarisé par cette métaphore.\\
\indent Ce modèle est rendu tangible via l'interface, mais l'interface ne définit pas l'ensemble de la métaphore : ainsi un clavier peut servir au déclenchement de notes (comme sur sur un piano), mais si l'intensité des notes est contrôlé par un autre processus (e.g. une pédale d'expression, ou une foule d'agents virtuels autonomes) le clavier ne sera plus envisagé comme une surface ``de percussion'' mais comme un filtre, un crible ne laissant passer que certaines fréquences. Le métaphore du piano avec ses marteaux projetés par l'enfoncement des touches disparait pour laisser place à un autre rapport sensible à l'instrument.\\
\noindent \textbf{contextualité} A la différence du geste d'excitation/modulation/sélection de Cadoz à prétention universelle, le geste de re-sonnance est un geste \textit{contextuel}, dépendant de l'interface, du modèle intermédiaire, de la musique jouée, du contexte de performance. Cette contextualité n'empêche cependant pas qu'une typologie soit établie pour un contexte donné, en s'appuyant sur l'observation des pratiques qui s'y rattachent. C'est par exemple l'objet de la thèse de Baptiste Bacot \footnote{\cite{bacot_geste_2017}}: son analyse qu'il définit comme une \iquote{organologie située}, s'appuie sur l'observation de situations concrètes de performance musicale chez différents instrumentistes numériques, et prend en compte les spécificités de leur configurations instrumentales.\\
\indent C'est aussi le cas dans le travail mené par Nathanaëlle Raboisson et Pierre Couprie, sur l'étude du geste performatif dans l'interprétation de musiques acousmatique sur table de mixage \cite{raboisson_experience_2017}. Une analyse partant de l'observation méthodique des gestes et de leur corrélation (ou non) avec le résultat musical leur permet d'esquisser une typologie propre à cette pratique, incluant des \iquote{gestes de placement} qui participent à la (dé)construction de l'espace sonore, ou encore des \iquote{gestes d'accompagnement}\footnote{Le sens est ici tout autre que le geste d'accompagnement tel que défini par Delalande évoqué précédemment.} caractérisés par une synchronie entre le geste et la morphologie sonore.\\
\indent De même qu'il existe un répertoire gestuel associé aux instruments acoustique, les modèles intermédiaires définissent, de manière modulaire, un vocabulaire d'interaction qui leur est lié. Le geste de re-sonnance s'appuie donc sur les affordances et la structure musicale associées à certains modèles intermédiaires : un geste venant contrôler une boucle de batterie telle que le Amen Break peut s'appuyer sur l'ensemble des techniques de \textit{chopping}\footnote{Dans les musiques basées sur l'utilisation du break-beat, le chopping consiste à découper la boucle de certaines manière en segments plus petits afin de les reconfigurer dans un ordre différent.}, \textit{cutting}, \textit{filtrage} passe-bas, \textit{stuttering}, \textit{varispeed}, etc. qui lui sont historiquement associées.\\
\indent Norbert Schnell développe l'idée de ``ré-animation d'enregistrements audio''\footnote{Le terme anglais \textit{reenactment} utilisé dans son travail reflète mieux que ``ré-animation'' le lien avec la théorie de l'enaction sur laquelle s'appuie notamment son travail.} \cite{schnell_playing_2013} et propose une étude très riche sur le plan théorique comme sur le plan pratique des modalités d'engagement dans une relation gestuelle/sonore par le biais d'une action métaphorique. En particulier, plusieurs exemples s'appuient sur une pièce musicale complète (de Johann Sebastian Bach, en l'occurence) et présentent diverses stratégies possibles d'avancement dans la pièce, sur la base d'une même structure commune définie par l'œuvre de Bach.\\



\subsection{Résonance entre les gestes re-sonnants et programmés}

\noindent Les instruments acoustiques présentent des modes de résonance que l'instrumentiste apprend à connaître, à apprivoiser, pour obtenir les qualités sonores qu'il ou elle recherche. Ces modes de résonance influent sur le timbre qui sera, par exemple sur un instrument à corde, rond et ample si l'on joue la corde au milieu de sa longueur, tandis qu'il sera plus grêle et chuintant si l'on joue \textit{sul ponticello}. La résonance de l'instrument acoustique peut également affecter d'autres paramètres que le timbre, par exemple le rythme : un percussioniste peut mettre à profit le rebond de ses baguettes sur la peau tendue, lors d'un jeu de roulements, et adaptera le poids et/ou la tension qu'il applique sur ses baguettes en fonction de la zone qu'il frappe et de son élasticité.

\indent Cette adaptation dynamique du geste à la résonance de l'instrument prend d'autres dimensions encore dans les \glspl{DMI}, pouvant générer de manière autonome\footnote{L'autonomie totale n'est évidemment plus une situation instrumentale à proprement parler.} toute une phrase musicale, tout un morceau. La nature dynamique et générative des \glspl{DMI} déplace l'agentivité\footnote{La notion d'agentivité dans la performance musicale dépasse sa simple implémentation opérante dans les \gls{IHM}. Par exemple, les figures dialogiques dans la musique classique ont été également analysée à travers ce prisme, voir notamment \cite{graybill_whose_2016}} de l'interaction instrumentale sur un terrain où elle se distribue entre des processus ``qui tournent'' et qu'il s'agit ``d'attraper''\footnote{Guerino Mazzola rapporte cette phrase du mathématicien Jean Cavaillès dans \cite{mazzola_topos_2018}: \iquote{Comprendre, c'est attraper le geste et pouvoir continuer}.} pour en jouer. Un exemple populaire en est l'alignement rythmique entre deux morceaux mixés par un \gls{DJ}, ou lors de l'utilisation d'un \textit{looper}.\\
\indent Le \gls{DMI} peut se retrouver en position de mener le jeu et imposer sa cadence à l'instrumentiste. La performance musicale avec un \gls{DMI} est donc une co-performance où la distribution du contrôle de la synthèse et de la gestuelle qui la provoque, ou en découle, peut se définir de manier polymorphe. Une partie de la dynamique de jeu peut être prise en charge par la machine et une autre partie par l'instrumentiste dans une relation qui peut parfois s'apparenter à un duo\footnote{Cette redistribution dialogique devient explicite dans des dispositifs tels qu'Omax (\cite{assayag_omax_2006}) ou le Continuator (\cite{pachet_continuator:_2003}). Voir par exemple la session d'improvisation entre György Kurtág père et fils avec le Continuator : \url{https://youtu.be/pqfKGlRvddg}.}.


\indent La part d'agentivité respective de la machine et du musicien dans la production du son des \glspl{DMI} définit en fait tout une échelle de nuances, allant de la posture de la personne écoutant la musique ``malgré elle'' (dans un supermarché, typiquement) à celle du musicien engageant tout son corps (physiques, cogécoute) dans l'interaction musicale. Cette recherche de la résonance avec l'instrument, d'en comprendre l'organisation des sons et les rythmes propres, d'en attraper les gestes programmés, d'y trouver les \textit{sweet-spots} est peut-être, davantage que le medium constitutif de l'instrument, ce qui définit vraiment son instrumentalité.

Le développement de l'ésthétique minimaliste n'est pas étrangère à cette capacité de pouvoir saturer l'espace sonore (ou visuel) sans qu'aucun effort ne soit à faire.


La possibilité des \glspl{DMI} de pouvoir générer du son en continu sans qu'aucun effort ne soit nécessaire pose de manière plus radicale que jamais la question de l'engagement du musicien, de quoi jouer et quoi \textit{ne pas} jouer), de quand jouer et quand \textit{ne pas} jouer et de comment le jouer.

La différence des \glspl{DMI} est de définir non-seulement l'agentivité de l'instrumentiste comme définissant les modes de production du sonore, mais également les modes de non-production du sonore, de la part d'instruments capables de produire en continu sans intervention. L'engagement musical avec un processus de génération sonore autonome ne nécessite pas forcément tant de savoir ``quoi'' jouer, mais de savoir en premier lieu ``quand'' jouer (ou ne pas jouer).


Les gestes du musicien peuvent alors entretenir diverses relations:
\vspace{-1em}
\begin{itemize}[noitemsep]
	\item une relation d'accompagnement, cohérente (en phase) ou dissonante (opposition de phase);
	\item une relation dialogique (typiquement un jeu de question/réponse)
	\item une relation de d'alignement;
	\item une relation d'accentuation.
\end{itemize}
 
Le geste de résonance qui associe à la fois une fonction ergotique (de type modulation) et une fonction épistémique (en cela qu'on cherche le \textbf{sweet spot}). 

Niveau plus ou moins cohérent de résonance entre le geste et le résultat produit, amène à étudier le résultat perçu dans ces différents cas. (commencer à amener la notion de causalité et de lisibilité présentée dans subversive gestures.)

L'apparence de causalité est fonction de
\vspace{-1em}
\begin{itemize}[noitemsep]
	\item congruence temporelle : synchronisation temporelle, morphologie rythmique similaire
	\item congruence spatiale : colocalisation spatiale, trajectoire similaires
	\item médiation par un objet intermédiaire dont on pense connaître le fonctionnement
\end{itemize}

%%%%%%%%%%%%%%%% FIN de cette section ? %%%%%%%%

%%%%%%%%%%%%%%%%%%%%%%%%%%%%%%%%%%%%%%%%%%%%%%%%%%%%%%%%%%%%%%%%%%%%
%%%%%%%%%%%%%%%%%%%%%%%%%%%%%%%%%%%%%%%%%%%%%%%%%%%%%%%%%%%%%%%%%%%%


\section{Subversion sonore, subversion gestuelle}
\label{sec:gesture:subversion}
%------------------ Figure : geste lisible ou subversif ---------------------
\begin{figure}[!htbp]
	\captionsetup{format=plain}%
	\centering
	\begin{minipage}[t]{0.48\textwidth}
		\includegraphics[width=\linewidth]{gfx/03_gesture/ManipulationVsEffect2.pdf}
		\caption[Stratégies de design d'interfaces pour le spectateur]{Stratégies de design d'interfaces pour le spectateur, d'après \cite{reeves_designing_2005, benford_performing_2010}}
		\label{fig:gesture:Benford}
	\end{minipage}
	\hspace{.02\linewidth}
	\begin{minipage}[t]{0.48\textwidth}
	  \includegraphics[width=\linewidth]{gfx/03_gesture/CoherenceVsCausalite2.pdf}
		\caption{Lisibilité et subversivité de la relation geste/effet}
		\label{fig:gesture:lisibility_subversion}
	\end{minipage}
\end{figure}

\noindent Nous avons vu dans les sections précédentes différents aspects concernant la fonctionnalité du geste, considérée du point de vue de l'instrumentiste. Ce point de vue privilégie généralement la recherche d'une relation lisible et cohérente entre le geste et le son, se situant à cette intersection où les actions réalisés par l'instrumentiste, les actions effectivement captées par l'interface et le résultat désiré de ces actions se correspondent(cf. figure \ref{fig:benford_expected-sensed-desired}).

\indent La poursuite d'une telle transparence pour l'instrumentiste (cf. \cite{fels_mapping_2002}), comme condition nécessaire à l'expressivité d'un instrument, pourrait masquer le fait que la relation entre le musicien et son public n'est pas exclusivement motivé par la recherche de cette transparence. 
On pourra même arguer du contraire : le design d'un instrument acoustique est largement motivé par l'obtention d'une homogénéité de son timbre qui est tout à fait artificielle, autant que la discrétisation du \textit{continuum} des hauteur en notes. La musique est un art du mirage, de l'illusion --~comme le rappelle Risset dans la citation en exergue de ce chapitre~--, dans lequel la subversion joue un rôle essentiel et se substitue à la transparence pour dessiner de nouvelles relations du sensible. De même, les gestes du musicien et du compositeur crééent des continuités là où il y a discontinuité (e.g. le \textit{legato} sur des notes de piano, la modulation tonale pour passer d'un mode à un autre, la fusion des timbres dans le travail d'orchestre),  et inversement introduit des ruptures pour déjouer les attentes du public (e.g. les jeu de contre-temps rythmiques, les \textit{subito-pianos}).

Les \textit{DMI} offrent de nouvelles possibilités de subversion de la perception, qui s'appuient sur les caractéristiques propres aux \gls{DMI} que nous avons déjà évoquées. Le découplage énergétique permet des correspondance inattendues entre force du geste et intensité du son, les capacités d'enregistrement permettent de faire surgir des matériaux de toute nature sonore, l'absence potentielle de contact avec l'instrument et sa reconfiguration dynamique permet d'envisager une métamorphose continue des relations de jeu.

\noindent Dans une analyse des stratégies de design d'interface en fonction de leur effet sur le spectateur \cite{reeves_designing_2005}, Reeves et al. distinguent quatre orientations possibles en fonction du degré de visibilité des gestes effecteurs de l'instrumentiste par rapport au degré de visibilité de l'effet produit par l'instrument (cf. figure \label{fig:gesture:Benford}):
\vspace{-1em}
\begin{itemize}[noitemsep]
 	\item \textbf{expressive} une interface expressive propose une relation cohérente et lisible entre le jeu de l'instrumentiste et l'effet; 
 	\item \textbf{occulte}: à l'opposé, une interface sur laquelle les gestes effecteurs sont invisbles ainsi que le résultat est jugée ``occulte'' (``secretive'') — ce pourrait être le cas par exemple lors de gestes discrets visant à rappeler des réglages sur l'instrument et dont l'effet n'est pas directement perceptible; 
 	\item \textbf{à suspens}: si des gestes effecteurs sont visibles manifestement destinés à produire un effet mais qu'aucun effet n'est perçu, cela pourrait être perçu comme un dysfonctionnement, mais également (et c'est l'orientation proposée ici) comme un moment de \textit{suspens}, si le public a conscience que l'action de ces gestes lui seront perceptibles plus tard;
 	\item \textbf{magique}: enfin, et c'est la direction qui m'intéresse plus particulièrement ici, les gestes effecteurs peuvent être cachés tandis qu'un effet est produit par l'instrument, configuration que les auteurs appellent ``magique''.
\end{itemize}
%-----------------
\noindent Analysons ici les différents cas de figures amenant à la perception d'une relation ``magique'' pour le spectateur. 

\subsubsection{Gestes feints}

\noindent Tout d'abord, se présente le cas où les gestes ne sont pas cachés, mais où subsiste un décalage entre les gestes perçus par le public et les gestes effectivement captés par l'instrument. Ce cas de figure peut se présenter sur des instruments acoustiques (et \textit{a fortiori} sur des \glspl{DMI} expressifs). Par exemple, dans certaines pièces pour piano de György Kurtág, le pianiste doit effectuer un geste d'approche énergique vers le piano, laissant supposer qu'il va jouer \textit{forte}, mais le retenir au dernier moment pour jouer \textit{pianissimo}\footnote{Ce mode de jeu a été observé en concert dans une interprétation des \textit{Játékok} de Kurtág. Après vérification, l'indication n'est pas explicite dans la partition, mais fait partie des instructions données par György Kurtág en vue de l'interprétation de certaines parties, d'après György Kurtág Jr. On y trouve par contre l'indication inverse, consistant à jouer \textit{fortissimo} \iquote{en touchant les touches sans faire de son}}. L'oreille se prépare à entendre un son intense et s'en protège par un réflexe stapédien\footnote{Le réflexe stapédien est la contraction involontaire des deux muscles de l'oreille moyenne, le muscle stapédien et le muscle du marteau. En rendant plus rigide la chaîne des osselets, il atténue le niveau des sons transmis à l'oreille interne.}, renforçant ainsi l'intensité du \textit{pianissimo}.\\

\subsubsection{Gestes du playback}

\noindent Nous avons également utilisé des gestes feints dans la performance \textit{FIB\_R} réalisée avec Gladys Brégeon, pour jouer une partie dont la ryhmique très précise et machinique. La rigueur métronomique requise n'étant atteignable qu'en la confiant à l'ordinateur, nous effectuons des gestes en rythme avec cette séquence autonome, ce qui laisse au spectateur l'impression que nous la jouons nous-même et contribue au l'esthétique générale de cette performance dans laquelle le rôle occupé par la machine et par les performeurs demeure ambigüe.

\subsubsection{Capteurs dissimulés}

\noindent Les auteurs mentionnent deux exemples utilisant des contrôleurs invisibles pour le public (des pédales de type \textit{foot-switch} supposées peu visibles d'une part, et l'utilisation de capteurs dissimulés dans le corps d'un instrument traditionnel, dont l'action est masquée par les gestes habituels du musicien.)

\subsubsection{Mapping non-conventionnel}

\noindent Dans ce cas de figure, les gestes ne sont pas cachés à proprement parler, mais le résultat de leur action défie certaines règles faisant partie de l'expérience du spectateur, par exemple si l'on fait correspondre un son très doux à un geste violent et inversement.

\iquote{Dans le domaine du geste, les outils technologiques peuvent bien sûr jouer un rôle complice, démultipliant les perspectives, inversant les conséquences attendues, décelant l'infime ou captant par méthode statistique tel ou tel paramètre du jeu musical.} 
\iquote{(...) s’approprier à la manière d’un mime les gestualités sonores qui, malgré les indications de la partition, ne peuvent être réellement considérées et donc interprétées que via le prisme de l’écoute.}
P. Jodlowsky \cite{jodlowski_geste_2006}

\subsubsection{Métamorphisme du mapping}

\noindent Il n'est pas nécessaire de se baser sur des lois ancrées dans l'expérience collective, mais également possible de déjouer des règles qui ont été établies en cours de jeu, en modifiant dynamiquement le fonctionnement de l'instrument pendant le jeu. Il devient nécessaire dans ce cas de faire en sorte que la ``couture'' entre les différents réglages qui se succèdent soit invisible, soit en procédant par un changement continu (e.g. interpolation vers de nouveaux réglages), soit en transposant dans le nouvel espace de jeu un certain nombre de paramètre de l'ancien espace de jeu afin qu'ils correspondent\footnote{Un exemple concret est donné dans la vidéo présentant les Modèles Intermédiaires Dynamiques (cf. chapitre \ref{ch:algorithms}) : la transition entre les différents modèles présentés se fait en conservant une cohérence visuelle de la représentation (e.g. la position, l'orientation des formes) --~ alors que tout le comportement à changé \url{https://vimeo.com/25740547} .)}.

\indent Cependant, le jeu de l'instrumentiste est polarisé par le fait qu'il joue (généralement) pour un public. La réception de la performance par le spectacteur/auditeur engage une projection active de sa part pour reconstruire du sens : déterminer la cause des sons, anticiper les trajectoires, ...  Cette projection se construit sur la base de l'expérience du spectateur et constitue l'objet central de la recherche en psychologie de la perception.


Celle-ci est prise dans une perception globale impliquant le geste du musicien, la réaction de l'instrument ainsi que les projection active de causalité de la part du spectateur sur la scène à laquelle il assiste.

L'étude de la perception des \glspl{DMI} est arrivée plus tardivement que l'étude de leur contrôle mais a pris une importance croissante ces vingts dernières années, notamment à cause de la considération dans le domaine des \textit{IHM} de scénarios narratifs dans ce que l'on nomme ``l'expérience utilisateur''. 

(Auslander, Croft, Bin, Fyans, Benford)


Au delà du timbre et des modes de jeu que le luthier ``programme'' dans l'élaboration d'un instrument acoustique, les \glspl{DMI} permettent d'élaborer une programmation qui à la particularité de pouvoir se déployer dans le temps avec une certaine autonomie, d'y dessiner un trajet\footnote{On peut mettre en regard la notion de trajet, qui suit un chemin, avec celle de plan   \iquote{la notion de référence s'est développée avec la masse grandissante des faits enregistrés, mais les écrits sont chacun une suite compacte, rythmée par des sigles et des notes marginales, dans laquelle le lecteur s'oriente à la manière du chasseur primitif, le long d'un trajet plutôt que sur un plan.} dans \cite{leroi-gourhan_geste_1964}, p 69}, voire d'y construire une narration. 

\iquote{Les propos des instruments qui nous entourent ne sont pas obligatoirement les nôtres. Ils appartiennent à ceux qui ont fait produire les instruments. Les détourner, c’est se libérer. Les instruments récents sont fascinants parce que, plus que tout autre, ils abritent des virtualités ignorées et parce qu’ils permettent des actions libératrices.} Villem Flusser, in ``les gestes''





Exemples de gestes subversifs :
\vspace{-1em}
\begin{itemize}[noitemsep]
	\item Geste forte, joué piano  : l'oreille se prépare à entendre un son fort et s'en protège par un réflexe stapédien\footnote{Le réflexe stapédien est la contraction involontaire des deux muscles de l'oreille moyenne, le muscle stapédien et le muscle du marteau. En rendant plus rigide la chaîne des osselets, il atténue le niveau des sons transmis à l'oreille interne.}, le son joué piano est du coup perçu encore plus piano.
	\item playback : le son est joué (semi-) automatiquement et le geste l'accompagne, donnant l'impression de le jouer vraiment
\end{itemize}


Les métaphores définissant les relations gestuelles-sonores peuvent être guidées par la recherche d'une certaine lisibilité, comme il a été le cas dans la plupart des travaux pré-cités, qu'elle soit de nature énergétique, spectromorphologique, ou imitative d'instruments existants. Ces relations peuvent toutefois être envisagées dans leur aspect subversif, en profitant de la perception différenciée de l'instrumentiste et de l'auditeur/spectateur.

Un exemple significatif est l'interprétation de ``Hangsimogato N°2'' de György Kurtág Jr\footnote{Vidéo disponible sur \url{https://www.youtube.com/watch?v=MJ8Z5skovLw}}. Dans cette pièce, le développement musical se fait par avancement sur une partition pré-programmé par l'intermédiaire d'un capteur Infra-Rouge (D-Beam). Le capteur lui-même n'est pas sensible à l'orientation de la main ou à quelle main (gauche ou droite) vient couper le rayon, mais György Kurtág Jr développe un vocabulaire gestuel qui établit des relations de correspondance avec les différents son de la pièce. Ces relations de correspondance peuvent s'appuyer sur une similarité de morphologie énergétique, mais dans certains cas (e.g. geste de présentation des mains ouvertes vers le ciel, replis des bras en croix) elle sont purement métaphoriques et poétiques.
\todo{rajouter un screenshot de la vidéo de Gyorgy}



Toutes les composantes du son, de la musique, et de la scénographie sont sujettes à l'invention de gestes, selon ce que le musicien décidera comme d'importance pour sa musique.



Perception d'erreur du point de vue du spectateur \cite{fyans_ecological_2012}

\cite{emerson_gesture-sound_2018}

Reeves et Benford propose un modèle de stratégies pour le design ``d'interfaces spectateurs'' montrant les effets obtenus lorsque la manipulation (les gestes de l'instrumentiste dans le cas qui nous intéresse ici) et l'effet de cette manipulation (le son dans notre cas) présentent chacun des degrés divers de visibilité.

Dans la figure de Benford manque la possibilité que la visibilité ne soit pas causale de l'effet.
Ce qui nous manque ici est la possibilité que la partie visible de la manipulation ne sont pas la cause de l'effet.

La cohérence\footnote{J'appelle ici ``cohérence'' la causalité \textbf{apparente} entre le geste et le son, pour la distinguer de la causalité \textbf{effective}, avec laquelle elle est mise en regard. La littérature en psychologie cognitive utilise cependant davantage le terme de perception de causalité.} entre le geste et le son se base sur un ensemble de valeurs qui ont été analysée dans de nombreux articles de sciences cognitive\footnote{Voir par exemple \cite{michotte_perception_2017} qui compile un grand nombre d'effets amenant à une perception congruente}, en particulier celles de la théorie de la perception, telles que :
\vspace{-1em}
\begin{itemize}[noitemsep]
	\item la congruence temporelle : syncrhonisme, destin commun, si j'entend le son en même temps que le geste, alors c'est ce geste qui a produit ce son.
	\item la congruence spatiale : si j'entend le son venir d'un endroit précis et que l'instrument se trouve à ce même endroit, alors le son vient de l'instrument.
- médiation par un objet intermédiaire dont on pense connaître le fonctionnement
\end{itemize}


\vspace{-1em}
\begin{itemize}[noitemsep]
	\item \textbf{lisible} : la perception du geste et du son apparait cohérente par rapport au système. Par exemple : on voit et on entend une personne parler.
	\item \textbf{illisible} : on voit une personne parler et on entend une autre voix
	\item \textbf{troublante} : c'est le cas du ventriloque : il est bien responsable du son que l'on entend (causalité), mais l'absence de mouvements de lèvre rend la perception visuelle incohérente avec la parception sonore (dissonance cognitive entre vue et ouïe).
	\item \textbf{subversive} : c'est le cas du playback
\end{itemize}

\subsection{Définition} 

Le terme ``subversion'' (du latin \textit{subvertere} : renverser, bouleverser) désigne ``l'action visant à saper les valeurs et les institutions établies'' (dictionnaire Larousse). Les moyens employés par la subversion consiste à diffuser un message contraire à un l'ordre établi, dans le but d'affaiblir celui-ci.

Dans le cas de la musique, si la notion de subversion peut prendre une dimension culturelle ou politique dans certains courants musicaux, c'est ici dans le cadre de la perception que j'emploie ce terme.

La subversion peut intervenir à différents niveaux. Au niveau de la composition, l'écriture musicale permet des modulations qui déjouent les attentes de l'auditeur. (e.g. Pink Floyd, breathe transition). Elle peut également se situer au niveau du jeu, en usant de procédés comme des gestes qui contredisent ce qu'on entend et vont l'amplifier. Gyorgy Kurtag Jr. geste violent pour jouer une nuance pianissimo.

Exemples comparés de Applebaum Aphasia et Vincent Carinola/Jean Geoffroy "Virtual Rhizome".
BBC Classic Album: "Pink Floyd - The Dark Side of the Moon"

Dissonance cognitive.

Synchrèse de Michel Chion.

Parler du playback, du air-guitar, de la synchrèse.

Nattiez, Music and discourse p44 : certain pianistes ont l'impression de donner de la ``profondeur'' à un accord en permettant aux doigts de glisser vers l'intérieur du piano après avoir enfoncé les touches. (...) inversement, Braendel : le son de notes soutenues sur le piano peut être modifié... à l'aide de certains mouvements qui rendent la \textit{conception du cantabile} du pianiste visible pour le public. (Delalande, ``vers une psycho-musicologie'' in L'enfant du sonore au musical):166, (Brendel, A. 1976. Musical Thoughts and Afterthoughts. Princeton: Princeton University Press. p.31)

Bien que ces catégorisations du gestes décrivent adéquatement différents aspects du geste instrumental sur des instruments acoustiques, il semble que le geste musical intègre une aspect subversif souvent négligé.

En particulier, dans le cas des \glspl{DMI}, la relation entre le geste et le son est totalement sujette au design de ce que l'on nomme communément le \gls{mapping} et la part de subversion devient partie intégrante du design de ce mapping. 


Il semble dès lors que l'on peut envisager d'autres types de relation entre le geste et le son, afin de tenter de décrire les différents rapports qu'ils entretiennent selon les situations.

Risset fait remarquer l'importance de l'histoire du son d'origine mécanique dans la perception des sons \cite{risset_son_1992}: 

\begin{quotation}
II semble à première vue que l'acoustique numérique puisse s'affranchir de la mécanique. Cependant notre ouïe a évolué dans un environnements d'objets vibrants: aussi la prise en considération des contraintes et des particularités des vibrations mécaniques est-elle importante pour comprendre les idiosyncrasies de la perception auditive et pour en tirer parti.

Les limites de l'acoustique numérique dépendent des capacités différentielles de perception davantage que des contraintes mécaniques. Pourtant, notre perception auditive est orientée par un monde de sons produits mécaniquement, et la mécanique ne doit pas être écartée de façon cavalière, comme l'ont suggéré les travaux de Gibson et Cadoz : les spécificités des vibrations mécaniques mettent en lumière l'organisation perceptuelle dans le processus auditif


\footnote{The limitations of digital acoustics depend upon the differential capacities of perception rather than upon the constraints of mechanics. Yet our auditory perception is geared to a world of mechanically-produced sounds, and mechanics should not be given a cavalier dismissal, as the work of Gibson and Cadoz has suggested : the specifics of mechanical vibrations shed light on the the perceptual organization in the hearing process.} TODO : traduire l'anglais, plus riche.
\end{quotation}


\textbf{Proposition}
\vspace{-1em}
\begin{itemize}[noitemsep]
\item readable gesture to sound relations
\item confusing gesture to sound relations
\end{itemize}

\vspace{-1em}
\begin{itemize}[noitemsep]
\item Gestes emphatique = en phase avec le mouvement interne du son
\item Geste apophatiques = en opposition de phase avec le son
\item Geste unrelated
\end{itemize}

%---- Figure : Einarsson sculpture ---------
\begin{figure}[!htbp]
	\captionsetup{format=plain}%
	\includegraphics[width=\textwidth]{gfx/03_gesture/gesteReelGesteSuppose.jpg}
	\caption{Geste produit, geste perçu, fonctionnement réel et supposé de l'instrument}
	\label{fig:gesture:RealVsSupposed}
\end{figure}


%-------------------------------------------
\subsection{Sons paradoxaux, gestes paradoxaux}
Risset, Kurtag Jr., Kurtag Père

De la même manière que l'écriture musicale sur papier a permis de développer des processus de compositions difficilement pensables sans ce support visuel \footnote{tels que la rétrogradation ou la fugue}, les ordinateurs ont permi de créer des formes sonores qu'il aurait été impossible de concevoir sans cet outil computationnel, tels que les sons paradoxaux de Risset et Shepard \footnote{Ce qui ne signifie pas que les sons paradoxaux soient impossible à produire sans recourir à l'ordinateur; cf. leur interprétation vocale par Victoria Hart \url{https://vimeo.com/147403169}}.

aspect scénographique de la performance musicale.
\vspace{-1em}
\begin{itemize}[noitemsep]
\item \textbf{gestes de feintes} : (rupture) déception de l'attente, mais visible après coup. E.g. dans le football, faire semblant d'aller à droite et envoyer le ballon à gauche, en musique: ommettre le temps fort d'un rythme bien établi, etc. La mécanique du geste est entièrement visible mais a été confuse par un changement innatendu.
\item \textbf{geste magique} : La mécanique du geste reste invisible et la logique causale entre le geste et son résultat reste inexplicable, et sujette à spéculation imaginaires.
\end{itemize}


%---- Figure : Triggering modes ---------
\begin{figure}[!htbp]
	\includegraphics[width=\textwidth]{gfx/03_gesture/key_modes.pdf}
	\caption{Différents modes de déclenchement: (a) enfoncement de la touche ``n''; (b) comportement habituel d'un clavier (e.g. dans un traitement de texte) (c) suppression de la répétition (d) utilisation d'un réservoir (e) phrases séquencées rythmiquement}
	\label{fig:gesture:triggering_modes}
\end{figure}


%-------------------------------------------
\subsection{Continuités artificielles}

\vspace{-1em}
\begin{itemize}[noitemsep]
\item Contrepoint : relier la mélodie à l'harmonie 
\item Bach et le tempérament = relier les différents modes, via la modulation.
\item les doigtés alternatifs, sur le plan gestuel, permettent de sacrifier la justesse de la note, pour établir une continuité gestuelle fluide
\item La musique sérielle : relier la gamme tempérée au spectre en ordonnant 
\item Stravinsky, Russolo : relier l’harmonie et le bruit 
\item jouer un pattern connu ("qu'on a dans les doigts") tout en substituant les notes permet de jouer de manière fluide une mélodie inhabituelle. => numérique
\item Cage, Murray Schaeffer : relier le déterminisme et le hasard, la musique et l’environnement 
\end{itemize}

(morpho-dynamisme)
Mettre capture d'image du MID qui passe d'une structure rotative à un Verlet.

Comment la continuité s'établit ?

=> voir Théories de la composition musicale au \siecle{20}~siècle
Conjointement à ces explorations compositionnelles se sont développées des techniques et des technologies permettant d’appréhender ces nouveaux espaces. Que cela soit des procédés d’écriture ou des instruments reflétant ces méthodes et modèles.


%%%%%%%%%%%%%%%%%%%%%%%%%%%%%%%%%%%%%%%%%

 
\section{Conclusion}
\label{sec:gesture:conclusion}


\noindent On voit donc que la relation entre le geste et l'instrument a été considérablement affectée par l'introduction de l'électricité, mais plus encore du numérique, dans le corps de l'instrument. Ces relations se construisent sur un medium présentant des caractéristiques radicalement opposées à celles de l'instrument acoustique : l'absence de causalité et de continuum énergétique entre le geste et le son, le métamorphisme du comportement de l'instrument, et l'absence possible de contact physique entre le geste et l'instrument.

Dans ce contexte, plusieurs voies sont possibles. 
La recherche d'une lisibilité de la relation geste/son peut s'appuyer sur des modèles physiques recréant artificiellement les conditions perdues de l'instrument acoustique, ou encore sur des relations basées sur la relation spectro-morphologiques entre mouvements du geste et mouvements du son.

Ces gestes peuvent également s'appuyer sur une connaissance du déroulement interne d'un matériaux enregistré, auquel cas une relation de re-sonnance / résonance vient définir une modalité de relation instrumentale ou les mouvements gestuels ne sont pas nécessairement dans une relation mimétique, mais viennent s'articuler sur différents plans de jeu mêlant écoute, extrapolation imaginaire, [ré]agencements à la volée et anticipatifs, gestes muets dont l'action n'est perçue que de manière différée, etc.



De même que l'on peut travailler une partition et devenir expert dans son interprétation, on peut travailler un instrument pour en devenir expert et travailler les différentes compositions pour cet instrument. On peut aujourd'hui travailler le geste (et l'écoute!) et en devenir expert pour jouer les différents instruments qui s'offrent à ces gestes.


Tout relation semble possible, si tant est qu'il


La relation se définit de manière contextuelle. 

Il est important de comprendre qu'il n'y a plus de relation fixe entre le geste



=> Comment ces aspects influencent le design de l’instrument ?
De cette étude du geste instrumental, on peut retenir plusieurs éléments qui viennent orienter (todo, better word) le développement des briques de bases qui constituent les DMIs.

\vspace{-1em}
\begin{itemize}[noitemsep]
\item transgression des catégories (entre continu et discret, entre audio et non-audio, création de relation arbitraires entre paramètres orthogonaux)
\item absence de limites arbitraires dans les représentations numériques (e.g. ambitus de pitch, polyphonie maximale)
\end{itemize}

\vspace{-1em}
\begin{itemize}[noitemsep]
\item \textbf{le format de données} : doit permettre le polymorphisme (cf. Zicarelli ``numbers without meaning'') entre les diverses formes de captation du geste (signal, événement, présence stable ou éphémère, etc.)
\item \textbf{le mapping} entre variables est lui-même sujet à une reprogrammation dynamique durant le jeu
\item \textbf{les différents gestes} de composition, de performance, d'écoute font partie intégrante des gestes de lutherie
\end{itemize}







\section*{extra material}
Notion de vivadi

Part of the excitment in the domain of new digital musical instruments in the 21st century can be attributed to the fact this fact as the musical creativity goes beyond the sound itself and includes the system through which it is performed. A downside of this situation, however, is that the novelty and digital features if the instruments create a sense of discontinuity with tradition , alienation, and lack of understanding by the audience as to what the instrument or the performer is actually doing.
\cite{magnusson_sonic_2019}



\iquote{Si ça se trouve, cette notion que dans quelques années, ``tout sera possible avec la technologie'' fera que cela sera compliqué de créer un mystère entier et profond, parce que les gens du coup diront ``oui, j'en ai entendu parler, maintenant on peut faire ça''.
 J'ai un ami (...) qui a fait voler un espèce de morceau de tulle au dessus des gens avec des principes mécaniques, et beaucoup de gens disaient ``ah oui, c'était incroyable mais je pense que c'était un drône'', alors que pas du tout. Mais je me suis dit, c'est vrai que d'ici quelques années, un objet qui vole tout seul en silence dans l'espace, n'aura plus le même pouvoir de mystère qu'il y a quelques années.} Yann Frish dans \url{https://www.youtube.com/watch?v=5BqHXbQC36M}





Subversion du geste : Kagel et le théâtre musical.
\url{https://geste.hypotheses.org/gemme}

\iquote{Dans le domaine du geste, les outils technologiques peuvent bien sûr jouer un rôle complice, démultipliant les perspectives, inversant les conséquences attendues, décelant l'infime ou captant par méthode statistique tel ou tel paramètre du jeu musical.} 
\iquote{(...) s’approprier à la manière d’un mime les gestualités sonores qui, malgré les indications de la partition, ne peuvent être réellement considérées et donc interprétées que via le prisme de l’écoute.}
P. Jodlowsky \cite{jodlowski_geste_2006}



\noindent Jakobson (1960) :
\vspace{-1em}
\begin{itemize}[noitemsep]
\item \textbf{expressive function}
\item \textbf{representational function}
\item \textbf{conative function}
\item \textbf{phatic function}
\item \textbf{metalingual function}
\item \textbf{poetic function}
\end{itemize}

\noindent David McNeil : 
\vspace{-1em}
\begin{itemize}[noitemsep]
\item \textbf{Iconics} where the gesture resembles the referent (e.g. describing an action or shape of an object with the hands).
\item \textbf{Metaphorics} where the vehicle (the gesture) relates in one of a number of metaphorical ways to the tenor (non-literal meaning) of the gesture, e.g. indicating a container or conduit for ideas, or a gift of an idea or suggestion (cf. Lakoff et Johnson 1980).
\item \textbf{Beats} where the hand, head, eybrows move roughly in synchrony with the rhythm of often emphatic speech, mark a sequence, or a hiatus such as a change of theme or focus.
\item \textbf{Cohesives} which create a gestalt in gesture space which is coextensive with a spoken utterance or – hierarchically – with its parts.
\item \textbf{Deictics} which may indicate an actual physical position, size, distance or direction, but may also place concepts metaphorically in physical gesture space
\end{itemize}


La mémoire et les gestes:
Leroi Gourhan

\iquote{Quant à l’action relayée (force motrice et transmission), elle domestique pour les utiliser des éléments qui étendent et complètent les effets techniques. Dans ce stade évolué, on n’est plus dans le faire mais dans le faire faire, engagé dans la voie techno-scientifique qui ne garde du geste humain initial que ses épures et en analyse indéfiniment les schèmes.} Michel Guérin, \cite{guerin_philosophie_2018}




Partant de l'idée que le geste et la musique sont deux phénomène impliquant le mouvement, je chercherai donc à définir l'intention gestuelle en fonction du rapport qu'il entretient avec le mouvement musical.
On peut dès lors envisager trois attitudes principales :
\vspace{-1em}
\begin{itemize}[noitemsep]
\item \textbf{jouer avec} : en phase avec le mouvement de la musique (geste emphatique), en soutenant par un rythme ou une harmonie complémentaire à ce qui est joué
\item \textbf{jouer contre} : jouer contre le mouvement pour chercher à l'annuler ou le détruire (geste apophatique)
\item \textbf{jouer indifféremment} : sans chercher à être ni contre, ni avec
\end{itemize}

On pourra nuancer cette catégorisation brutale en ajoutant une quatrième catégorie, qui se situerait entre 
le jeu ``avec'' et le jeu ``contre'' qui consiste à jouer en ``interférence'', c'est-à-dire qui vient infléchir une direction


\iquote{J'appelle technique un acte traditionnel efficace (et vous voyez qu'en ceci il n'est pas différent de l'acte magique, religieux, symbolique). Il faut qu'il soit traditionnel et efficace. Il n'y a pas de technique et pas de transmission, s'il n'y a pas de tradition.} Marcel Mausse, les techniques du corps

Il manque un élément important dans cette considération de la tradition. Pour qu'une tradition soit transmise, il faut que des individus la transmette. Cela peut se faire par la contrainte ou un système doctrinal (e.g. un système religieux), mais en cette absence de coercition physique ou mentale, la tradition sera transmise à la condition que les individus croient en la valeur de cette tradition et qu'ils lui accordent suffisament d'importance pour en mémoriser les principes.


%-------------------------- Figure : Shannon ----------------------------------
\begin{figure}[!htbp]
	\includegraphics[width=\textwidth]{gfx/03_gesture/ShannonCommunicationSystem.png}
	\caption{Diagramme schématique d'un système général de communication, tel que proposé par Shannon.}
	\label{fig:gesture:shannon}
\end{figure}



Ces deux catégories de geste d'action et de gestes perçu sont emprunt de la théorie de l'information proposée par Shannon \cite{shannon_mathematical_1948} qui envisage la communication comme un système émetteur-message-récepteur unidirectionnel. 
L'inconvenient d'envisager le geste comme simple émetteur d'un signal (qu'il soit travail ou signe) est qu'il empêche de considérer le geste dans la rétroaction dans laquelle il s'inscrit avec l'instrument. En particulier dans la performance musicale, la rétroaction multimodale (par l'ouïe, la vue, le toucher) entre les geste d'un instrumentiste et son instrument, le son et le public est essentielle.

Il était tentant de l'appliquer à la situation musicale et de voir dans le phénomène sonore un message circulant du musicien vers l'auditeur, et les analyses du geste instrumental s'appuyant sur cette solide base théorique ont permis de développer un certain nombre de concept encore utile pour l'analyse du geste musical.

Cependant, la situation de performance musicale est loin d'être aussi fonctionnelle que celle qui consiste à vouloir transmettre un flux de données. Notamment, la théorie de l'information s'applique à des machines qui ignore totalement le contenu sémantique de ces données et les aspects cognitifs ou les références culturelles des émetteurs et récepteurs.
%-------------------------- Figure : transparence Fels -----------------------
\begin{wrapfigure}[14]{R}{0.5\textwidth}
	\begin{center}
 		\includegraphics[width=0.48\textwidth]{gfx/03_gesture/Fels-transparency.pdf}
	\end{center}
	\caption{Transparence pour le musicien et l'auditoire, d'après \cite{fels_mapping_2002}}
	\label{fig:gesture:fels_transparency}
\end{wrapfigure}
%-------------------------- Figure : transparence Fels -----------------------

Certains auteurs ont critiqué cette approche \cite{fyans_where_2009} en remarquant que l'idée selon laquelle une connaissance et une compréhension préalable de l'instrument et de l'idiome était nécessaire pour évaluer une performance musicale, ne pouvait être généralisée aux \glspl{DMI}, à cause de l'émergence rapide de technologies, d'instruments et de pratiques de performance dans ce domaine. 

Cependant, cette critique reste ancrée sur une approche qui considère que le spectateur évalue le \textit{succès} d'une performance selon sa compréhension des intentions de l'instrumentiste.


Le jeu musical joue en partie sur l’attente de l’audience (récompensée ou non) sur la base de règles d'harmonies, d’idiomes (e.g. cadences, résolutions, cycles rythmiques), de citations (e.g. via le sampling), etc..
Affordance des instruments ne peut être réduite aux objectifs d’affordance des IHM.


Kurtag Jr. Hangsimotato (video)
Jean Haury Meta-Piano
Applebaum Aphasia

gestes incongruent (Musical gestures, Godoy, p.48)


Charlotte Moorman and Name June Paik performing John Cage’s 26’1.1499” for a String Player (Human Cello section

%%%%%%%%%%%%%%%%%%%%%%%%%%%%


Si le geste est un mouvement accompagné d'intention, il faut prendre en compte cette partie intentionelle et essayer de la qualifier, dans la perspective des conséquences qu'elle porte au design des \glspl{DMI}.

La notion ``d'image de son (i-son)'' de François Bayle exprime la mécanique psycho-poétique de construction de l'œuvre musicale acousmatique. Sa nomenclature ne se prête pas facilement à une application directe dans la lutherie (numérique ou non).
En partant de la classification des fonctions de l'écoute proposée par Pierre Schaeffer et Michel Chion (\cite{chion_guide_1994}, p.26) (Insérer ici le tableau comprendre-écouter-entendre-ouïr), François Bayle retient notamment trois niveaux d'écoute attentive (en regroupant entendre et comprendre dans un seul niveau) qu'il fait correspondre à trois niveau d'intentionalité dans la mise en jeu des \textit{images-de-sons}:
\vspace{-1em}
\begin{itemize}[noitemsep]
\item \textbf{\textit{im-son}}: l'image isomorphe, iconique, référentielle;
\item \textbf{\textit{di-son}}: le diagramme, sélection de contours simplifiés, indiciels ;
\item \textbf{\textit{mé-son}}: la métaphore ou métaforme, reliée à une généralité
\end{itemize}

Ces trois catégories font également écho au catégories de Delalande (gestes effecteurs, gestes accompagnateurs, gestes figurés) — sans qu'il y ait toutefois de relations causales triviales entre ces catégories du gestes et de l'écoute.

%------------------ Bricout: g-son -------------------------
Concept de \textit{g-son} proposé par Bricout \cite{bricout_les_2011}, à partir du concept d'i-son (\textit{image-son}) proposé par Bayle, comme ``dépassement de la suggestion de l'image par le son lui ajoutant de manière beaucoup plus évidente la suggestion du geste, de l'élan physique.''

Romain Bricout : couple ``déclenchement/modulation'' (analogue à l'archétype ``percussion/voix'' Martin Laliberté) comme atomes gestuels constitutif de tout mouvement. => NON tout l'espace gestuel avec toutes les connotations possibles (sémiotiques, mimétiques)

Bricout :
\iquote{Déclenchement et modulation représentent donc ces deux gestes primordiaux, à la base de de n'importe quel autre geste plus complexe. Par voie de conséquence, n'importe quel son renvoie lui-même à un geste producteur qui se rapprochera tantôt du déclenchement, tantôt de la modulation ou, par combinaison, des deux à la fois}
=> qu'en est il d'un field recording ?
%------------------ END Bricout: g-son -------------------------

J'aurais plutôt tendance à employer le terme ``d'image de geste'' (i-geste) pour décrire cette analogie entre les plans d'interprétation du geste et du son.

Le travail de création des correspondances entre geste et son passe ainsi par trois étapes faisant écho à ces différents niveaux de perception/compréhension musicale :
\vspace{-1em}
\begin{itemize}[noitemsep]
\item \textbf{coder} la relation algorithmique (causale ou non), c'est-à-dire concrètement la relation algorthmique qui s'opère entre les signaux captés par l'interface et le contrôle de la synthèse sonore, 
\item\textbf{jouer} la relation sensible, c'est-à-dire pratiquer (chorégraphier) l'ensemble du mouvement gestuel dont une partie seulement sera captée par l'interface de jeu;
\item\textbf{imaginer} la relation poétique, cette relation s'établit sur un ensemble plus complexe de valeurs esthétiques, de références culturelles impliquant de manière plus globale les questions de composition, de scénographie, de métaphores portée par les sons, etc.
\end{itemize}




\subsubsection{Les unités sémiotiques temporelles}

La définition des UST est donnée dans \cite{timsit-berthier_les_2004}:
\iquote{Les UST sont des segments musicaux, qui possèdent une signification temporelle en raison de leur organisation morphologique et cinétique. Elles peuvent êtres considérés comme des représentations iconiques qui entretiennent des rapports de ressemblance avec des modèles temporels naturels. L’UST ne traduit pas le phénomène musical à son niveau acoustique, mais cherche à y trouver en quelque sorte une intentionnalité.}



%%%%%%%%%%%%%%%%%%%%%%%%%%%%%%%%%%%%%%%%%%%%%%%%%%%%%%%%%%%%
%%%%%%%%%%%%%%%%%%%%%%%%%%%%%%%%%%%%%%%%%%%%%%%%%%%%%%%%%%%%
%%%%%%%%%%%%%%%%%%%%%%%%%%%%%%%%%%%%%%%%%%%%%%%%%%%%%%%%%%%%
%%%%%%%%%%%%%%%%%%%%%%%%%%%%%%%%%%%%%%%%%%%%%%%%%%%%%%%%%%%%
%%%%%%%%%%%%%%%%%%%%%%%%%%%%%%%%%%%%%%%%%%%%%%%%%%%%%%%%%%%%
%%%%%%%%%%%%%%%%%%%%%%%%%%%%%%%%%%%%%%%%%%%%%%%%%%%%%%%%%%%%
%%%%%%%%%%%%%%%%%%%%%%%%%%%%%%%%%%
\section*{Espace du geste musical}
La musique a longtemps été considéré comme étant faite d'un sous-ensemble de sons, les sons harmonieux, voire harmonique, avant qu'au \siecle{20}~siècle, les bruits n'y fassent leur place avec les avant-gardes, futuristes. 

John Cage in \cite{cage_silence:_1961}
\begin{quotation}
\noindent If this word, music, is sacred and reserved for eighteenth- and nineteenth-century instruments, we can substitute a more meaningful term: organization of sound.\\
\end{quotation}


Anecdote De Laubier ` le haut parleur ne fonctionne pas'

La musique n'est donc pas faite que de sons, au sens acoustique du terme, mais également (avant tout?) de la perception des sons, qui implique des processus de cognition, des références socio-culturelles, et une sensibilité, une mémoire propre à chacun. 
Ainsi, l'espace de la musique ne se présente non pas comme un sous-ensemble de l'espace des sons, mais probablement comme un sur-espace comprenant à la fois les sons acoustiques mais également tous les liens qu'ils tissent avec notre mémoire.

\Pierre{ voir les 2 définitions de la musique}

L'art musical consiste ainsi à faire entendre des aspects de la musique qui ne sont pas nécessairement présents dans le son, à faire surgir des espaces qui ne peuvent se déployer que dans notre imaginaire, en faisant écho à la trace latente que les sons et la musique ont déjà imprimée en nous.


\begin{quotation}
\noindent Le musical dépasse le sonore en cela qu’il est connecté à une expérience cognitive qui implique la perception et la mémoire.\\
Le sonore dépasse le musical en cela que tout ce qui est sonore ne fait pas nécessairement musique (sauf chez Cage).
\end{quotation}

\Pierre{ ce n'est pas seulement le cas de la musique mais aussi de la parole. Voir définition de Castellengo.}
\Pierre{ Cage -> référence ?}

Là où la présence du musicien sur scène remplissait une nécessité acoustique pour l’écoute, la musique sur support, ou produite par des machines, déplace ce besoin au profit d’une autre fonction, à la fois de compréhension des gestes du musicien (mais est-ce là un jeu de dupes?) et d’un spectacle de l’ordre du funambulisme; le musicien prend des risques [celui de se tromper dans le cas de l’interprétation d’une partition] et la mise en question du corps, réagir au contexte (lieu et au public, ainsi qu’aux éventuels autre musiciens) d’une manière vivante.

L’écoute nous plonge dans des flux sonores, et notre tendance à projeter des causes à ces sons (cf. gestalt) nous emmène sur les lieux — toujours en partie étrangers — de la production de ces flux. sitar indien, crissement de pneu, explosion, acoustique sous-marine ou ambiance de salle de café.
Le musicien crée des passerelles et des agencements entre ces zones liminales.

\Pierre{ si tu parles de gestalt, il faut développer mais c'est aussi une théorie très controversée, donc attention !}

Si les gestes \textit{subversifs} peuvent être assimilés à des gestes accompagnateurs, la plupart des articles de la littérature semble ignorer cette part de subversion au profit de la lisibilité du geste et sa corrélation avec le son \cite{godoy_exploring_2006}.
Cependant, la corrélation n'est pas nécessairement recherchée en tant que telle et si, comme le rappelle Risset, \iquote{la musique est aussi un art du mirage, de l’illusion} \cite{risset_propos_2010}, les œuvres sont nombreuses qui cherchent à dépasser le lien d'apparente causalité entre le geste et le son. 
Il serait alors plus juste d'utiliser le terme de \textit{geste accompagnant la musique}, si toutefois on 

%---- Figure : Einarsson sculpture ---------
\begin{figure}[!htbp]
	\includegraphics[width=\textwidth]{gfx/Einarson-SchumannSculpture}
	\caption{Einar Torfi Einarsson - Schumann-Sculpture (remnants + deracination)}
	\label{fig:gesture:einarsson}
\end{figure}


\iquote{Every music performance is a dramatic presentation for listeners and improvisers alike. In a sense, both groups play interactive roles as actors from their respective platforms. Just as the design of the hall, the stage and the lighting frames the band's activity for the audience's observation, it also frames the audience's activity for the band to observe. Performers and listeners form a communication loop in which the ction of each continuously affect the other.} Paul F. Berliner in \cite{berliner_thinking_2009}


%%%%%%%%%%%%%%%%%%%%
%-------------------------------------------
\subsection*{Dans les instruments numériques}


Les \glspl{DMI} ont souvent été analysés en tant qu'\gls{IHM}, et les conférences académiques qui leur sont consacré reflètent une culture dans laquelle l'interaction s'exprime via un cahier des charges préalablement identifié: une \gls{IHM} est utilisée dans le cas d'une tâche précise et sa qualité (ergonomie, précision, etc.) peut être mesurée de manière quantifiée.
Dans le cas des instruments de musique cependant, cette tâche est plus complexe, car les enjeux de la création musicale dépassent par essence tout objectif identifié et mesurable au préalable. Par ailleurs, les \glspl{DMI} sont destinés à plusieurs types "d'utilisateurs" ayant un rôle différent : le musicien qui joue de l'instrument, mais également le public, qui bien qu'il ne joue pas de l'instrument est amené à en observer la performance.

Low entry fee, high ceiling.

La performance musicale est un "jeu" qui comporte une part de duplicité. Le public d'un concert est toujours le sujet d'une illusion. 

Un des biais de la littérature sur l’affordance des instruments de musique numérique est qu’elle s’inspire souvent des objectifs de l’affordance des IHM en général, avec l’idée que l’instrument doit être compréhensible pour les “autres” utilisateurs potentiels que l’auteur de l’instrument. Pourtant, nombre d’instruments présentés dans la communauté NIME ne sont joués que par leurs auteurs (cf. [1], [2], [3]) et si le fait de vouloir transmettre son instrument aux autres est louable, il n’est pas gage de qualité en ce qui concerne la création qui sera faite avec cet instrument. 


L'art musical procède en partie de la magie et de l'illusion perceptive. Le musicien nous fait entendre des continuités (e.g. une mélodie) là où l'acoustique fait apparaitre une série discrète (e.g. des notes de piano), ou inversement des fissions (e.g. deux voix indépendantes) là où est jouée une série temporelle de notes sur un même instrument. (=> plutôt que des exemple entre parenthèse, mettre une figure illustrant fission e.g. Bach's Violin Partita No. 3, BWV 1006.)

Cette question du jeu entre le continu et le discret dépasse le seul cadre de la musique mais semble trouver dans cet art de nombreuses espaces d'expression.

Les théories de la perception, en particulier du Gestalt, viennent en partie expliquer les mécanismes qui pousse notre perception à créer des continuités où il n'en existe pas physiquement et inversement à catégoriser des événements selon certaines distances perceptives qui ne sont pas nécessairement en lien avec l'unité de source de production du son.

Si donc on analyse le geste musical, il faut nécessairement prendre en compte sa dimension subversive en ce qu'elle se traduit, particulièrement dans le cas des \glspl{DMI} et des productions musicales impliquant l'électronique en général, dans le design des instruments et outils qui servent à la créer.

\cite{bin_show_2018}

Une étude de Tsay \cite{tsay_sight_2013}, dans laquelle des amateurs et experts sont amenés à évaluer une performance musicale sur la seule base d'un enregistrement silencieux, met en évidence le rôle considérable du la part visuelle dans l'appréciation et l'évaluation de la performance.

Carte et guide , frettage adaptatif (cite \cite{goudard_playing_2014})


\iquote{De même, pour un violoniste, la manière dont il lève le bras et dont il va attaquer le son, la rapidité avec laquelle il prépare son coup d’archet nous renseignent un petit peu, mais pas complètement – parce que l’on ne sait pas quelle hauteur il va jouer – sur certaines catégories du son, comme le fait que le son sera agressif, fort, ou délicat et très doux. (...) Dans la musique instrumentale, cette causalité est très importante car cela participe de la façon dont nous la percevons et l’intérêt, avec la musique électronique, c’est que l’on peut remettre en question cette causalité-là : un tout petit geste peut provoquer une tempête. Entre le geste du pianiste qui va appuyer sur une touche du piano et le son qui va sortir, il y a une machine que j’appelle une boîte noire, qui peut inverser les polarités, c’est-à-dire que je peux très bien programmer la machine de manière à ce que plus le son qui va être joué va être minime, pianissimo, plus le son électronique qui va sortir va être au contraire démesuré : dans ce cas-là, le geste ne correspondra pas du tout au son.} Philippe Manoury interviewé par Anne-Sylvie Barthel-Calvet. (\url{https://geste.hypotheses.org/364})

Dans en Echo de Manoury, ce sont les formants de la voir qui contrôlent la partie électronique, c'est-à-dire un geste invisible, sans contact. (extensible au suivi de partition)


Pouvoir transformer tout type de donnée en geste programmé.


L'interface sensible\footnote{Sur la notion d'interface sensible, cf. \ref{ch:interfaces}} doit pouvoir se prêter à des gestes sans intention, c'est-à-dire qu'elle doit permettre des gestes non-réfléchi, mal-contrôlé ou plutôt in-controlés, qui peuvent tomber en dehors de la zone prévue pour capter de le geste, où d'une manière inadéquate. Cela ne signifie pas nécessairement que l'instrument doit ``faire quelque chose'' de ces gestes: il peut les ignorer.  Mais il est utile que l'intrument permette aux gestes de ``déborder'' du cadre prévu pour leur interaction (si toutefois ce cadre existe).


%-------------------------------------------
\subsection*{Tout ce qui bouge n'est pas geste - partie à revoir ou distribuer}

Dans le domaine de la recherche musicale, les mouvements du corps sont associés à la notion de \textit{geste musical}, c'est-à-dire à un concept associant à la fois le \textit{mouvement} du corps et \textit{l'intention} et/ou \textit{la signification} de ce mouvement. 

Cela n'est pas nécessairement et systématiquement le cas et les mouvements de l'instrumentiste peuvent être envisagés et décrits avec d'autres perspectives que celle de leur potentielle intention. \todo{ref ou footnote ici vers des études en ce sens} La notion de \textit{geste musical} semble en effet implicitement suggérer un rapport hiérarchique entre le musicien et son instrument, dans lequel les gestes ne serait produits qu'intentionnellement, à l'initiative du musicien. Les instruments de musique, et en particulier les DMIs, sont envisagés plus récemment comme \textit{agents} qui opèrent dans un système de relations multi-directionnelles\todo{ref}, que Berliner décrit métaphoriquement par une \textit{conversation} dans \cite{berliner_thinking_2009}. \todo{attention,il parle de la relation musicien/public} 

\Pierre{ je doute qu'il y ait beaucoup de gestes non-intentionnels chez le musicien !}


Les gestes du musicien ne sont pas nécessairement remplis d'une intention ou d'une signification \textit{a priori}, ils peuvent s'apparenter aux gestes de la danse.




L'instrument vibre et produit parfois du son sans qu'il soit explicitement déclenché ou controlé. Les mouvements du corps du musicien en témoignent et au dela des effets spectaculaires des DJs qui touchent aux potentiomètres de leurs interfaces comme s'ils étaient brûlants \footnote{Mark J. Butler apelle \iquote{passion of the knob} (\textit{la fièvre du potentiomètre}) ces moments qui surviennent \iquote{lorsqu'un musicien dirige une expressivité exceptionnellement intense vers un petit composante technique associée à l'ingénierie du son} \cite{butler_playing_2014} \url{https://www.youtube.com/watch?v=Nh9C7nQHmII}}, le corps est parcouru de mouvements qui ne sont pas uniquement des \textit{actions} mais des \textit{réactions} à ce qui est produit par l'instrument. 

\Pierre{ l'exemple choisi devrait être un peu plus analyser car les mouvements du DJs sont probablement totalement intentionnels - c-a-d ils font partie du spectacle car le DJ se sait regardé.}
\Pierre{ je crois plus en la séparation geste-signe et geste-action}

Si l'on considère la relation geste/instrument/musique comme un réseau multi-directionnel, le geste peut-être provoqué par la musique, via l'instrument lui même. Deux exemples caricaturaux viennent illuster cette possibilité : la performance \iquote{eletric stimulus to face — test} de l'artiste Daito Manabe\footnote{\url{http://www.daito.ws/work/electricstimulustoface_test.html}} ou dans le système de motorisation des doigts pour apprendre un instrument proposé récemment à la conférence NIME par \cite{zhang_adaptive_2019}.


Notons enfin que les mouvements peuvent survenir également en interaction avec le public\footnote{This reveals that passion-of-the-knob moments and other actions are not interior to the musician’s world, but rather are intensely meaningful communications: they reverberate outward to the audience and then are reflected back to the stage as formative elements of a milieu whose participants seek to actively cultivate and sustain liveness. in \cite{butler_playing_2014}} 



\subsection*{geste d'impression, geste d'expression}
Si le geste peut ex-primer, c'est-à-dire ``faire sortir en pressant'', un mouvement intérieur et le faire exister dans la temporalité de la performance, il peut aussi im-primer (faire rentrer, en pressant) ce geste sur un support à même d'en accueillir la trace.

geste du latin gero qui signifie ``porter''

S'il y a différance (ajournement et différence) dans la grammatisation musicale (la composition, la lutherie, la programmation), il y a enfin le moment de sa performance, de sa répétition.

Classification des controleurs gestuels dans \cite{wanderley_controgestuel_1999}:
\vspace{-1em}
\begin{itemize}[noitemsep]
\item \textbf{Instrument-like controllers},where the input device design tends to reproduce each feature of an existing (acoustic) instrument in detail. Many examples can be cited, such as electronic keyboards, guitars, saxophones, marimbas, and so on.
\item \textbf{Instrument-inspired controllers} that although largely inspired by the existing instrument’s design, are conceived for another use [62]. Fig. 3 presents one example of such controller, the SuperPolm violin developed by S. Goto, A. Terrier, and P. Pierrot [63], [64], where the input device is loosely based on a violin shape, but is used as a general device to control granular synthesis. => emprunts variés de formes et de fonctions.
\item \textbf{Extended instruments} are instruments augmented by the addition of extra sensors [58], [65]. Commercial augmented instruments included the Yamaha Disklavier, used, for instance, in pieces by J.-C. Risset[66], [67]. Other examples include the flute [68]–[70] and the trumpet [71]–[73], but any existing acoustic instrument may be extended to different degrees by the addition of sensors.
\item \textbf{Alternate controllers} (see, e.g., Fig. 4), whose design does not follow that of an established instrument. Some examples include the Hands [52], graphic drawing tablets [74] (cf. Fig. 5), etc. For instance, an unorthodox gestural controller using the shape of the oral cavity has been proposed in [75].
\end{itemize}

These controllers can furthermore be classified into different categories.
\begin{itemize}[noitemsep]
\item \textbf{Touch, expanded range, or immersive} controllers [76], depending on the amount of physical contact required from the performer. Mulder also [76] separates immersive controllers into internal, external, and symbolic controllers according to the possibilities of visualization of the control surface. In a different approach, Piringer [77] classifies immmersive controllers into partial or completely immersive controllers.
\item \textbf{Individual or collaborativecontrollers}[78],depending on whether the instrument is performed by one or multiple performers at one time.
\item \textbf{Metaphorical} or \item{ad hoc} controllers, and so on.
\end{itemize}


Guerino Mazzola frozen gestures



\iquote{Complétons tout d’abord la phrase : l’ordinateur n’est pas un instrument mais une représentation d’instrument. Cette subtile nuance contient l’essentiel. Envisagé ainsi, l’ordinateur donne une nouvelle dimension au processus de création en y intégrant explicitement, en amont de l’acte instrumental, la construction d’une représentation du dispositif instrumental. Cette construction\/représentation offre une latitude nouvelle : la possibilité pour l’homme de se placer dans une relation ``de type instrumental'', une représentation de relation instrumentale où la liberté d’échapper aux contingences du réel lui permet de créer de nouveaux mondes imaginaires. 
Toutefois, le processus de création est également considérablement transformé par le fait que l’aller\/retour indispensable entre le réel et l’imaginaire, lui aussi, se déplace. Dans le cas de l’instrument réel, l’aller\/retour se fait in situ, dans la relation même avec l’instrument. Dans le cas de la représentation d’instrument, il se fait dans une boucle plus vaste : l’activité de représentation instrumentale, de jeu virtuel, de composition façonnent nos sens et notre intelligence d’une nouvelle manière qui sont alors en jeu dans une perception nouvelle du monde réel... à condition que nous y retournions, c’est à dire que nous ne finissions pas par substituer définitivement nos représentations à la réalité.} \cite{cadoz_musique_1999}, p99.

Claude Cadoz défend l'hypothèse que les appareils électroniques ne sont pas des instruments mais des ``représentations d'instrument''.


 % INCLUDE: interface
% !TEX root = ../thesis-example.tex
%
\chapter{Interface sensible / hardware}
\label{ch:interfaces}

\cleanchapterquote{Komponieren heißt: über die Mittel nachdenken.\\
Komponieren heißt: ein Instrument bauen.\\
Komponieren heißt: nicht sich gehen, sondern sich kommen lassen.\\
.}{Helmut Lachenmann}{1986}

\cleanchapterquote{Fingers are not to be despised: they are great inspirers, and, in contact with a musical instrument, often give birth to subconscious ideas which might otherwise never come to life.}{Igor Stravinsky}{1936}

%--------------------------------------------------------------
\section{Interface gestuelle ou interface sensible?}

\noindent Il est souvent question ``d'interface gestuelle'' lorsqu'on pense aux \glspl{IHM} utilisées pour l'interaction musicale. Comme nous l'avons vu au chapitre précédant, le geste occupe une part importante de l'interaction musicale mais les interfaces de jeu ne se restreignent pas nécessairement à la captation du geste : elle peuvent être sensible à la température, la lumière\footnote{Voir par exemple les œuvres \textit{Light Thing} de Leaf Cutter John (\url{https://youtu.be/2jIlLHfSEfs}), ou encore \textit{Light Music} de Thierry de Mey}, à la couleur, et réagir de manière générale à différentes conditions environnementales.\\
\indent Par ailleurs, le geste possède un certain nombre de qualités qui ne sont pas forcément captées par l'interface, alors qu'elle sont effectivement vues et ressenties par le musicien et par le public, et contribuent ainsi à la performance, comme nous l'avons présenté au chapitre précédent.\\
\indent Enfin, le ``geste'' qui vient contrôler les processus sonores dans les \glspl{DMI} peut être de nature virtuelle, prendre la forme de motifs pré-enregistrés qui peuvent être issus de toute sorte de source de données interprétées en tant que flux temporels, tels que peut-être le cas dans la sonification de données ou l'utilisation de modèles intermédiaire. Cet aspect là sera décrit plus en détail dans le chapitre \ref{ch:algorithms}.\\
\indent Il semble ainsi préférable de parler d'\textit{interface sensible}, plutôt que d'\textit{interface gestuelle} pour décrire les dispositifs d'interaction numériques pour la musique, leur caractéristique commune étant l'usage de capteurs (\textit{sensors}).


%%%%%%%%%%%%%%%%%%%%%%%%%%%%%%%%%%%%%%%%%
\section{Composer, interpréter, improviser en live}

\noindent Les \glspl{DMI} intègrent généralement des matériaux pré-composés, que le musicien peut lancer et qui pourront tourner de manière autonome (paysages sonores, processus génératifs, drônes, etc.) et une part créée plus directement, qu'elle soit synthèse ou transformation de matériaux.\\
\indent Les différents accès de l'interface de jeu doivent ainsi permettrent d'articuler au mieux ces deux pôles et permettre la gestion du temps à différentes échelles. En particulier, les interfaces multitouch, si elles permettent de gérer aisément de multiples processus lents (qu'on pourra visualiser et ajuster à l'écran), ne permettent guère une réactivité ``percussive'', telle que le permet un pad \gls{MIDI} ou mieux encore, le microphone.\\
\indent La granularité du contrôle joue également un rôle important dans cette perspective. Entre une note \gls{MIDI} qui ne déclenche qu'un événement ponctuel, un contrôleur continu qui envoie des données chaque fois qu'il est modifié et un capteur échantilloné qui envoie des informations en continu, les possibilités de contrôle seront différentes en terme de réactivité et de finesse de modulation. Il faut donc prévoir les capteurs adéquats pour les processus que l'on souhaite contrôler en aval. Comme le faisait remarquer Max Mathews, cité par Joel Chadabe dans \cite{chadabe_electric_1996}:
\iquote{Il faut penser aux systèmes dans leur globalité pour obtenir une quelque chose d'utilisable musicalement - on ne peut pas vraiment développer un capteur sans le mettre en perspective des programmes avec lesquels on va l'utiliser. \footnote{``One has to think of overall systems to get a musically useful thing — you can't really develop a sensor without relating it to the programs that you're going to use it with.'', p. 230}}\\
\indent Ceci n'est pas sans poser problème dans la perspective de modularité d'un \gls{DMI} que l'on aimerait pouvoir faire évoluer et adapter à différents contextes. Un compromis consiste à disposer d'une palette de capteurs différents et de les agencer selon une ergonomie adaptée aux gestes qu'ils invitent, et d'adapter le logiciel, plus souple, à cette configuration.

\section{La part acoustique de l'interface des DMIs}
\label{sec:interfaces:part_acoustique}

\noindent Il faut ici rappeler que les \glspl{DMI}, s'ils se caractérisent par l'usage de la computation numérique, sont aussi nécessairement des instruments électroniques, électriques et acoustiques. Il portent l'héritage et les contraintes propres à ces différents médias. Mais le phénomène acoustique, omnidirectionnel et tri-dimensionnel dans le monde physique, est réduit dans l'électronique numérique à un signal mono-dimensionnel\footnote{Nicolas Collins, dressant une liste de traits distinctifs entre hardware et software dans \cite{collins_semiconducting_2013} notait: ``Traditional acoustic instruments are three-dimensional objects, radiating sound in every direction, filling the volume of architectural space like syrup spreading over a waffle. Electronic circuits are much flatter, essentially two-dimensional. Software is inherently linear, every program a one-dimensional string of code.''} et mono-directionnel, introduisant une distinction entre les ``entrées'' d'une part et les ``sorties'' d'autre part. La dimension acoustique des \glspl{DMI} s'intègre donc dans un circuit ouvert ou fermé avec le système de computation via des transducteurs acoustiques, qu'ils soient microphones, en entrée, ou haut-parleurs, en sortie.\\
\indent Paradoxalement, si le résultat acoustique d'un instrument est \textit{in fine} ce qui nous intéresse le plus, la part acoustique de l'instrument numérique est aussi sa part maudite. Les phénomènes de résonnance et de rétro-action qui se produisent dans l'instrument acoustique et qui sont également exploités dans les instruments électriques analogiques (e.g. le feedback entre une guitare électrique et un amplificateur), posent un certain nombre de problèmes lorsqu'on introduit un élément de computation numérique dans un tel système\footnote{Une des raisons à ce problème est l'instabilité consubstantielle aux systèmes bouclés discrets et non-linéaires, mais également, de manière plus générale, la transition entre le domaine continu de l'acoustique et le domaine symbolique de l'informatique implique une analyse du son qui nécessite une modélisation des phénomènes perceptifs, tels que la hauteur, le rythme, etc. utilisés pour le contrôle}.\\
\extra{nombreux synthétiseurs analogiques ont été mis sur le marché, qui articulent cette part d'électronique analogique (pour les filtres et la synthèse) et numérique (pour le  contrôle), afin de pouvoir bénéficier des avantages des deux domaines.}

\subsection{la captation : microphones et transducteurs piezo}

\noindent La possibilité de pouvoir introduire du son acoustique dans un système électroacoustique est une pratique très courante chez les musiciens électroacoustiques, et permet des séquences de jeu en prise directe avec des objets concrets dont on pourra transformer le son. En particulier, l'usage de microphones, notamment de transducteur piezo dites ``microphone de contact'', vient redonner une composante acoustique au corps même de l'agencement instrumental. L'intérêt principal réside dans la richesse du signal capté, comme le note Miller Puckette dans \cite{puckette_infuriating_2011}: \iquote{(...) il y a peu chance que le frottement de balais sur un pad de batterie produise quoi que ce soit d'intéressant, alors que faire la même chose sur un instrument qui fonctionne directement avec le signal audio du microphone de contact (comme nous le faisons ici) offre la possibilité de créer une large gamme de sons musicaux utiles\footnote{``(...) sliding a brush over a drum trigger isn’t likely to produce  anything  useful,  whereas  doing  the  same  thingon an instrument that operates directly on the audio signal from the contact microphone (as we do here) has the possibility to create a wide range of useful musical sounds.''}.} L'usage d'algorithmes d'analyse en temps-réel permet, au delà d'effets déjà existants dans le domaine électroacoustique, d'utiliser les caractéristiques du son comme paramètre de contrôle. Une telle approche a été utilisé notamment dans \cite{schwarz_rich_2014} et l'interface Mogees\footnote{\url{https://www.mogees.co.uk}, voir également l'Annexe \ref{appendix:zamborlin}} est principalement basée sur ce principe. 



\subsubsection{Utilisation des piezo dans les différentes DMIs réalisés}
\todo{déplacer cette section dans 4.5?}
\indent \textbf{Sur le Filigramophone}, des cellules piezo sont placées entre une vitre en plexiglass et le chassis contenant la tablette graphique (cf. Figure todo). Cela permet des gestes percussifs, de frottements ou l'utilisation d'objet mis en mouvement (toupies, dés, diques) directement sur la surface, tout en conservant —à travers le plexiglass— l'usage de la tablette qui renvoit les coordonnées horizontale et verticale ainsi que la pression du stylet. Ces transducteurs piezo permettent de capter les différents timbres de la surface plexi, qui présente une certaine élasticité (par rapport au verre) en étant sertie uniquement sur ses bords, et dont la hauteur spectrale est plus grave au centre et plus aigüe sur les bords. Il est également possible de frapper sur le chassis, ce qui permet d'obtenir une autre nuance de timbre (cf. Figure \ref{fig:interface:filigramophone-piezo}).\\
%------------ Figure : filigramophone et xypre piezo -----------
\begin{figure}[!htbp]
	\captionsetup{format=plain}%
	\centering
	\begin{minipage}[t]{0.48\textwidth}
		\includegraphics[width=\linewidth]{gfx/05_interfaces/filigramophone-piezo_72dpi.jpg}
		\caption{Transducteur piezo entre la vitre et le chassis sur le Filigramophone}
		\label{fig:interface:filigramophone-piezo}
	\end{minipage}
	\hspace{.02\linewidth}
	\begin{minipage}[t]{0.48\textwidth}
	    \includegraphics[width=\linewidth]{gfx/05_interfaces/xypre-piezo_72dpi.jpg}
		\caption{Transducteur piezo pseudo-symétrique dans le côté du chassis sur le Xypre v2 (plaque extérieur démontée)}
		\label{fig:interface:xypre_v2-piezo}
	\end{minipage}
\end{figure}
%------------ Figure : filigramophone et xypre piezo -----------
\indent \textbf{Sur le Xypre v1}, la technologie infra-rouge du cadre de détection du multitouch permet d'insérer une vitre plexiglass pour un fonctionnement similaire à celui développé sur le Filigramophone. L'inconvienient qui en résulte est la transmission des vibrations au cadre de détection multitouch et les bruits de plastique qui en résultent.\\
\indent \textbf{Sur le Xypre v2}, l'écran multitouch ne permet pas l'ajout d'un plexiglass les cellules piezo ont été placées sur le côté du chassis, pré-contraints entre le chassis et une plaque de bois plus fine servant de surface de percussion, permettant là-encore de récupérer différentes nuances de hauteur spectrale, utilisée pour la synthèse en aval (cf. figure \ref{fig:interface:xypre_v2-piezo}). La pré-contrainte du piezo est en partie dûe à l'orientation verticale du piezo (qui chuterait, autrement) mais permet également d'utiliser une technique de pseudo-symétrisation du signal du piezo (cf. schéma fonctionnel figure \ref{fig:interface:balancedPiezo} et aperçu figure \ref{fig:interface:xypre_v2-piezo}), qui s'avère très utile quand les transducteurs piezo sont utilisés à proximité d'un écran, source de perturbations électromagnétiques. Romain Michon a étudié différentes possibilités d'effectuer des gestes percussifs et de pression sur une tablette multitouch dans \cite{michon_nuance_2016}, ainsi que proposé une solution hybride par l'ajout de capteurs \gls{FSR} placés sous un iPad, modulés en amplitude et récupéré via l'entrée audio d'un iPad. L'intérêt de cette solution est de permettre le calcul de la vélocité des gestes percussifs, en plus de l'aftertouch, mais la combinaison des informations de pression et de coordonnées X/Y, bien que judicieusement contournée par une triangulation des différents \gls{FSR}, reste problématique. Par ailleurs, cette solution utilisant des capteurs de pression, sous la surface rigide de l'iPad ne permet pas d'exploiter la variation de hauteur spectrale en fonction du lieu de frappe, la surface de l'iPad étant trop rigide et les \gls{FSR} inadaptés à cette gamme de fréquences. C'est cette dernière raison qui a motivé la conception du Xypre v2 avec une séparation entre la surface percussive des piezo et la surface de contrôle sur l'écran multitouch.

%-------------------------- Figure : balanced piezo ----------------------------------
\begin{figure}[!htbp]
	\includegraphics[width=\textwidth]{gfx/05_interfaces/balancedPiezo.pdf}
	\caption{Montage pseudo-symétrique de transducteurs piezo-électriques}
	\label{fig:interface:balancedPiezo}
\end{figure}
%-------------------------- Figure : balanced piezo ----------------------------------

\subsection{la diffusion : haut-parleur et transducteur tactile}

\noindent Dans le cas le plus trivial, l'acoustique des \glspl{DMI} se limite à la membrane du haut-parleur qui transforme \textit{in fine} le signal audio-numérique en son acoustique\footnote{Le choix des haut-parleurs peut jouer un rôle primordial, comme c'est le cas dans la musique acousmatique diffusée sur orchestre de haut-parleur ou ``acousmonium''. Voir en particulier \cite{mooney_sound_2006}}. A la différence des instruments acoustiques, la diffusion et la projection du son est souvent séparée de l'interface gestuelle et, bien souvent, distante du musicien quand elle est ``spatialisée''. La spatialisation du son a en effet joué un rôle essentiel dans la motivation de faire des concert en direct, à une époque où l'arrivée du \gls{CD} et de l'écoute de salon ``haute-définition'', comme le raconte Serge de Laubier en parlant des origines du PSO\footnote{``Processeur Spatial Octophonique'', système de spatialisation inventé par De Laubier en 1986.} et du Méta-Instrument (cf. annexe \ref{appendix:delaubier}) dans les années 1980 : \iquote{Si on entend mieux chez soi, c'est pas la peine de faire des concerts, donc il faut qu'au concert, il y ait une expérience unique qui vaille le coup de se déplacer, d'où réfléchir à un système de spatialisation.}\\
\indent L'essor progressif des transducteurs tactiles à large bande audio depuis la dernière décennie a cependant entrainé l'intégration des haut-parleurs dans le corps de l'instrument, en particulier dans les instruments augmentés tels que la Smart Guitar de HyVibe\footnote{\url{https://www.hyvibe.audio}}. Ce retour acoustique présente plusieurs intérêts :

\vspace{-1em}
\begin{itemize}[noitemsep]
	\item \textbf{offrir au musicien un retour vibratoire }(et/ou auditif), qui lui permet de mieux sentir le résultat sonore et pouvoir plus facilement identifier sa propre production sonore dans le cas de musique d'ensemble;
	\item \textbf{faciliter l'identification et la localisation} auditive de la source sonore et accroître la cohésion entre geste et son et pour le public;
	\item \textbf{bénéficier des propriétés acoustiques des matériaux} structurel de l'instrument, notamment leur rayonnement, beaucoup plus singulier que celui des haut-parleurs (dont la conception est généralement orientée vers un rayonnement et une bande-passante homogène, ``neutre'');
	\item \textbf{introduire du feedback} dans le corps de l'instrument en recaptant cette vibration. Traité avec une latence suffisamment faible, cette possibilité laisse la possibilité de transformer dynamiquement les propriétés acoustiques des matériaux, comme le fait le système HyVibe;
	\item \textbf{communiquer des informations} à l'instrumentiste via un retour vibratoire (cf. TODO mettre un lien vers le développement du frettage virtuel)
\end{itemize}

\noindent Sur le Filigramophone, un haut-parleur tactile a été fixé sur la vitre de plexiglass (cf. figure \ref{fig:interface:filigramophone-hp}) et utilisé à la fois pour le retour vibratoire et la communication d'information, en particulier par frettage virtuel de la surface (cf. section TODO:ref). Le retour vibratoire de l'instrument se distingue toutefois de l'envoi pur et simple du signal audio, la perception tactile ne lui étant pas directement liée. Au lieu de cela, un signal sinusoidal à 70Hz, correspondant à une fréquence de résonnance de la vitre de plexiglass, modulé par l'enveloppe du son final, ainsi que par sa dérivée spectrale, afin de mieux sentir dans les doigts les transitions et les ruptures du son (cf. figure TODO : schéma de principe du patch / développement dans une autre partie ?).\\
\indent Sur le Xypre, la séparation entre la surface percussive et l'interface multitouch de l'écran a conduit à placer le haut-parleur tactile sur une plaque de bois (contreplaqué) à l'avant du chassis ((cf. figure \ref{fig:interface:xypre-hp})). Il se trouve ainsi relié acoustiquement à un transducteur piezo (cf. figure todo), tandis que l'autre transducteur est (relativement) isolé acoustiquement en étant positionné orthogonalement.

%------------ Figure : filigramophone et xypre piezo -----------
\begin{figure}[!htbp]
	\captionsetup{format=plain}%
	\centering
	\begin{minipage}[t]{0.48\textwidth}
		\includegraphics[width=\linewidth]{gfx/05_interfaces/filigramophone_hp_72dpi.jpg}
		\caption{Haut-parleur tactile sur le Filigramophone}
		\label{fig:interface:filigramophone-hp}
	\end{minipage}
	\hspace{.02\linewidth}
	\begin{minipage}[t]{0.48\textwidth}
	    \includegraphics[width=\linewidth]{gfx/dummy.pdf}
		\caption{Haut-parleur tactile sur le Xypre}
		\label{fig:interface:xypre_v2-hp}
	\end{minipage}
\end{figure}
%------------ Figure : filigramophone et xypre piezo -----------

%%%%%%%%%%%%%%%%%%%%%%%%%%%%%%%%%%%%%%%%%
\section{Qualité ergodynamiques}


\subsection{Proprioception}
\noindent La proprioception désigne la perception de la position des différentes parties du corps dans l'espace, recouvrant le sens du mouvement (\textit{kinesthésie}) et le sens de la posture (\textit{statesthésie}). Dans le cas des \gls{DMI}, cette proprioception edt double : il s'agit à la fois d'intégrer la topologie réelle de l'instrument (disposition des capteurs, espace du mouvement autour de ceux-ci, course sensible et courbes de réponse) mais également la topologie virtuelle des algorithmes manipulés. 
En ce qui concerne la topologie réelle de l'instrument, un aller-retour s'opère entre des gestes impliqués par le positionnement des capteurs et des gestes trouvés durant la confrontation avec l'instrument.
Par exemple, le positionnement des capteurs dans Xypre v2 a été revus en fonction de gestes qui venaient naturellement lors du jeu avec le filigramophone. Par exemple, le geste de percussion sur le côté du chassis venait naturellement dans la course du bras pivotant autour de l'articulation de l'épaule, alors qu'aucun capteur n'avait été positionné là. C'est d'ailleurs un geste de percussion qu'on retrouve dans des instruments de percussion comme le Pakhavaj indien ou dans la pratique de la percussion en fanfare (la grosse caisse, frappée de côté).

%------------------------------------------------------------
\subsection{Ergodynamisme}
cf. définition de Thor Magnusson
agencement de l’interface, représentation visuelle, repères tactile, frettage

\subsubsection{Le poids du hardware}

\noindent Le hardware, comme son nom l'indique, est solide, matériel, ``dur''. Sa matérialité, son poids affecte sa transportabilité et constitue un facteur contraignant pour le musicien. Sa dématérialisation dans des alternative logicielles —\textit{soft}, plus douces et légères— permet ou non d'intégrer dans les limites acceptables pour leur transport des fonctionnalités offertes par les équipements \textit{hard}.
En particulier, la virtualisation des outils traditionnellement utilisés par les ingénieurs du son (table de mixage, équaliseurs, compresseurs et autres effets) permet leur intégration dans l'instrument lui-même.

(Cf. interview Mamou-Mani.)
Ainsi, Serge de Laubier qui utilisait une table de mixage Yamaha O2R (31kg) motorisée et contrôlée directement depuis son Méta-Instrument, ainsi qu'un échantilloneur EMU (4,5kg) a progressivement conçu des émulation logicielles de ces équipements pour alléger le transport. Par ailleurs, les émulations logicielles permettent de réaliser d'autres fonctions qui n'étaient pas présentes dans les modèles originaux.


\subsubsection{Choix des capteurs et latence}




\subsubsection{Topologie spatiale}

La topologie des capteurs dans l'espace:
\vspace{-1em}
\begin{itemize}[noitemsep]
	\item topologie corpo-centrée (e.g. exosquelette du MI3, Myo Atau Tanaka)
	\item topologie objet, qu'on peut prendre, secouer, etc. (e.g. ``The Sponge'' de Martin Marier)
	\item topologie ``sur table'' objet qu'on ne peut pas déplacer mais autour duquel on peut tourner (e.g. claviers, pad, machine intona rumori de Bernier/Messier ...)
	\item topologie immersive : installation, vidéo, danse (e.g.) (e.g. ``Machine variation'' de Bernier/Messier)
\end{itemize}



\subsubsection{Intégration de l'instrument}

Est-ce que l'instrument est plug'n'play ou va-t-il va falloir connecter ses différents éléments ? Connections : 
\vspace{-1em}
\begin{itemize}[noitemsep]
	\item entre les capteurs et l'interface de digitalisation (arduino, carte son, etc.)
	\item entre l'interface ADC et l'ordinateur
	\item entre l'ordinateur et la carte son
	\item entre la carte son et les haut-parleurs
\end{itemize}

Les deux premières étapes sont absentes dans le cas du live-coding.


\subsubsection{Temps de montage}

La topologie et l'agencement des capteurs sur l'instrument entraine une facilité variable de transportabilité et un temps de démontage/remontage de l'instrument avant qu'il soit possible de le ``démarrer''.
Par ailleurs, les instruments électrifiés nécessitent souvent d'être branchés (sauf s'ils fonctionnent entièrement sur batterie). Ces branchement prennent du temps, nécessitent d'avoir des alimentations en courant à proximité et d'une puissance suffisante. Pour les outils numériques se rajoute à cela le ``temps de lancement'', c'est à dire généralement le temps de démarrer l'ordinateur, de lancer le(s) logiciel(s) nécessaire(s), d'ouvrir le patch ou le script adéquat, et éventuellement de l'initialiser avec la bonne configuration.

L'idéal d'un instrument numérique est souvent qualifié de ``plug'n play'', mais rare sont les cas d'instrument qui s'affranchissent du ``plug''. Il est important de prendre en compte cette durée dans le design d'un \gls{DMI}, car tout le temps passé sur la partie de montage technique est souvent pris au détriment du temps de répétition (cf. François Dumeaux, Nicolas Bernier, Bruno Zambolin interviews). 

Notons toutefois que les \glspl{DMI} ne nécessitent généralement pas de temps d'accordage, et ne sont pas généralement pas sujets aux conditions de température et d'hygrométrie qui nécessite le ré-accordage des instruments acoustique et le temps de chauffe, particulièrement important pour les cuivres.


\subsection{se repérer au toucher}
Un grand défaut des interfaces graphiques ``tactile'' est que leur surface est par défaut dépourvue de repère tactile: aucune aspérité ne vient guider la main pour qu'elle trouve ses repères et son chemin sans l'aide de la vue. Différentes stratégies peuvent venir partiellement compenser cette carence. Leur application dépend à la fois de la technologie de captation du multitouch, ainsi que du type de repère, statique ou dynamique, que l'on souhaite :

\subsubsection{rajout de repères statiques}
L'ajout de repères statiques ad-hoc et interchageables aide le toucher à sentir une position de repère sur l'écran tacile, ou encore les contours d'une zone d'interaction. Les interfaces ``Joué''\footnote{\url{https://www.play-joue.com}} ou ``Sensel Morph''\footnote{\url{https://sensel.com}} commercialisent différents revêtements (\textit{overlays}) pour leur surface multitouch. Ces surfaces ne sont toutefois pas pourvue d'un écran graphique, ce qui évacue le problème de la transparence de ces repères.\\
Une solution bon marché consiste à utiliser de la bande adhésive (cf. figure todo). Un intermédiaire plus fin entre l'éphémère bande adhésive et une solution fixe consiste à utiliser une plaque de plexiglass intermédiaire entre le doigt et la surface tactile, qui permet de graver des motifs potentiellement plus complexes. Cette dernière option n'est toutefois pas toujours réalisable sur les écrans à technologie capacitive, qui nécessite parfois que le doigt soit effectivement en contact.\\


TODO : mettre une photo de la wacom dans sa version avec les scotch et une photo de la plaque de plexiglass


\subsubsection{repère dynamique : fretting audiotactile}
Le déclenchement d'impulsion dans un haut-parleur tactile peut aider à sentir les paliers dans la progression d'un geste continu (e.g. les différentes ``notes'' d'une échelle de hauteur). Ce type de retour tactile fonctionne uniquement en réponse à un mouvement, le repère n'est pas sensible de manière statique. Son avantage par rapport à un repère physique (e.g. adhésif) .\\
TODO : mettre une photo de la wacom dans sa version avec les scotch.

\subsubsection{vérouillage des composants GUI}
Une solution logicielle partielle à ce problème et largement utilisée dans les systèmes de \gls{GUI} consiste à vérouiller l'interaction avec le composant GUI tant que le doigt est en contact avec la surface tactile, ce qui permet d'utiliser l'intégralité de l'écran tactile pour l'interaction et d'avoir des gestes plus amples. Ce solution nécessite toutefois que le composant ait bien été ciblé au moment du contact initial.


%-------------------------------------------------------------

\subsection{facteurs de formes: héritages et transpositions}

\subsubsection{héritage instrumental}

\noindent Les \glspl{DMI}, en partie libérés\footnote{cf. § \ref{sec:interfaces:part_acoustique}} des contraintes de facteur de forme liées à l'acoustique, présentent un agencement et une topologie liés à des questions d'ergonomie d'une part mais aussi d'héritages, instrumentaux ou non. L'héritage le plus évident est celui des techniques de jeu et du répertoire, qui prend une importance considérable dans le design des \glspl{DMI} inspirés d'instruments pré-existants, en particulier les instruments dits augmentés (cf. Annexes \ref{appendix:turchet} et \ref{appendix:mamou-mani}). Cet héritage est le plus manifeste dans les instruments destinés au commerce, en permettant un effet dilligence\footnote{notion définie par le médiologue Jacques Perriault, décrivant les protocoles mis en place pour l'adaptation d'une innovation en vue de son acceptation sociale (``Les premiers wagons avaient la forme des diligences.'')} entre instruments classiques et instruments nouveaux.\\

TODO : inclure une image de la HyVibe et/ou de l'hyper mandoline de Turchet.\\
Héritage des techniques de jeu et du répertoire : cf. interview Adrien MM et Lucas Turchet) 


\subsubsection{héritage du corps}
\noindent L'ergonomie, pensée en dehors de l'organologie classique, nous amène à considérer directement le corps, et plus particulièrement les mains. Comme l'explique Serge de Laubier (cf. Annexe \ref{appendix:delaubier}) :\\
\iquote{(...) quand on fabrique un instrument il y a une contrainte, c'est le corps et donc forcément, il faut que les instruments, en tout cas ceux qui fonctionnent bien, soient quand même relativement bien adaptés au corps... relativement parce que, même les instruments acoustiques, les instrumentistes se détruisent pas mal mais quand même, malgré tout, il arrivent à les pratiquer pendant des années, plusieurs heures par jour, et en général ils tiennent... donc la contrainte du corps est importante et la première contrainte c'est celle des mains, avant la contrainte du corps... parce que c'est le plus agile je pense, le plus agile, rapide, réactif}\\

Un certain nombre d'instruments comme ``The Hands'' de Michel Waisvisz ou le Méta-Instrument de Serge De Laubier sont ainsi directement conçus à partir de l'ergonomie de la main, de la mécanique du corps.

\subsubsection{héritage de l'objet}

\noindent L'instrument est également un objet de scénographie et c'est parfois sa fonction scénographique qui initie et oriente le développement de l'instrument. C'est par exemple le cas dans l'usage d'objets détournés, tels que ``The Sponge'' de Martin Marier \cite{marier_sponge_2010}, ou dans les installations de Patrick Saint-Denis qui \iquote{part de l'objet} et \iquote{poursuit l'idée des objets animés par le son} (cf. Annexe \ref{appendix:saint-denis}).\\

TODO : mettre une image de The Sponge et des accordéons de Patrick Saint Denis

%%%%%%%%%%%%%%%%%%%%%%%%%%%%%%%%%%%%%%%%%
\subsubsection{héritage poétique}

\noindent Une autre forme d'héritage est celui de l'imaginaire, de la force d'évocation poétique des objets, qui dépasse le strict cadre fonctionnel de la relation instrumentale. Cet imaginaire polarise le jeu du musicien et l'écoute du public, \iquote{comme s'il y avait un arrière plan imaginaire de tout ce que ça draine d'histoire, de projections} (cf. Annexe \ref{appendix:dumeaux}). On retrouve la même fonction de détournement d'objets et d'association imaginaire dans ``l'Olitherpe'', instrument joué par Patricia Dallio, qui se présente comme un agencement de capteurs intégrés dans des objets de récupération. 

\begin{quotation}
	J'ai l'habitude de récupérer des objets qui me parlent. Je ne sais pas pourquoi il me parlent, mais ils me parlent. J'ai pu associer des mouvements aux formes des objets. (...) Le mouvement m'inspirait l'objet, ou l'objet était en corrélation avec le mouvement. C'est aussi une de mes habitudes de récupérer et de savoir ce que j'ai dans mes affaires et de pouvoir l'intégrer à quelque chose d'utile et de pratique qui n'est pas forcément son but d'origine.
\end{quotation}

\noindent Dans un documentaire\footnote{Documentaire ``L'Olitherpe et la teneur de l'air'' : \url{https://vimeo.com/224494409}}, Olivier Charlet qui s'occupe de la construction du dispositif (intégration des capteurs dans un habillage et une structure) évoque plusieurs pistes qui orientent l'intégration des capteurs : 
\vspace{-1em}
\begin{itemize}[noitemsep]
\item \textbf{l'usage qu'en fait la musicienne} dans son jeu, dont on peut imaginer qu'il conditionne l'endroit du dispositif où le nouveau capteur sera intégré; 
\item le \textbf{fonctionnement propre des capteurs} : par exemple un capteur de distance qui envoie un signal infra-rouge et reçoit sa réflection nécessitera un espace libre dans son champ de captation;
\item une \textbf{association morphologique} entre le mouvement de l'instrumentiste et la forme de l'objet;
\item des \textbf{associations imaginaires et poétiques} : le capteur infra-rouge évoque des yeux et c'est dans un vieux phare de vélo récupéré qu'il sera intégré, choisi pour ses qualités évocatoires (``J'ai l'habitude de récupérer des objets qui me parlent.'').
\end{itemize}

\noindent On peut voir dans la démarche artisanale et \gls{DIY} des lutheries numériques l`importance accordée à la charge affective des objets; les contrôleurs numériques (tels que les contrôleurs \gls{MIDI}) vendus dans le commerce sont en effet souvent issus d'une production industrielle et faits de matériaux plastiques qui à l'inverse du bois d'un violon, ne porte pas la trace organique des fibres du bois ou celle du geste artisanal imprimé par le luthier et se retrouve, d'une certaine manière, dépourvu d'histoire.

%-------------------------------------------------------------
\subsection{retour audio-tactile}
Intégration de HP tactile
Spatialisation ou diffusion ad-hoc
Travail avec Pascale Criton à la maison des aveugles


%%%%%%%%%%%%%%%%%%%%%%%%%%%%%%%%%%%%%%%%%
\section{Généalogie d’une interface de DMI}
\label{sec:interfaces:sec1}

Filigramophone : évolution depuis la tablette graphique simple, tablette augmentée, écran multitouch augmenté, intégration de Bela…

De la simple tablette graphique à l’écran multitouch augmenté de capteurs, histoire de l’évolution d’une interface pour la performance électroacoustique.
La conception d’une nouvelle interface pour la performance musicale est une tâche complexe, nécessitant de nombreux aller-retours entre conception, fabrication et pratique musicale. Le filigramophone est une interface qui a connu plusieurs versions, suffisamment différentes pour avoir envie de leur donner un nouveau nom à chaque fois et suffisamment similaire pour y voir la continuité d’un seul et même instrument.

%----------------------------------------------------------------------------------------------------------
\subsection{origine : la tablette graphique}
La tablette graphique (nommément un modèle Sapphire de Wacom) a été l’interface originelle qui a servi de base au filigramophone. J’ai commencé à l’utiliser suite à son utilisation dans le cadre de la Méta-Mallette\footnote{Logiciel pour la pratique collective de musique par ordinateur développé par l’association Puce Muse.}. La raison de ce choix est que la tablette graphique offre une interface relativement bon marché (donc déployable en nombre) qui permet un contrôle assez fin de la synthèse sonore.
Un certain nombre de musiciens, compositeurs et concepteurs de NIME l’ont adopté pour leurs projets \cite{zbyszynski_ten_2007}, et Nicolas d’Alessandro a consacré une partie de son travail de thèse \cite{dalessandro_realtime_2009} à ce sujet, en proposant une étude détaillée des différentes échelles de mouvements dans le geste du dessin et de l'écriture (articulation doigt-poignet-épaule).

La tablette graphique permet de bénéficier de l’expertise du geste d’écriture et de dessin.

Ali Momeni, Jesper Nordin, gestrument (\url{https://gestrument.com/}), Pierre Jodlowski


%----------------------------------------------------------------------------------------------------------
\subsection{augmentation de la tablette graphique}
ajout de piezo et HP tactile pour une réponse immédiate
ajout de MPD pour le changement de comportement
ajout de carte/frettage de la tablette (aux crayon et feutres)

Problème de devoir tout démonter l'instrument pour changer la feuille de frettage. => solution graphique ?

Utilisation du haut-parleur tactile pour un frêttage virtuel de la table (envoi d'une impulsion au passage d'un repère)


freetingSignal = sig(P)*cycle(130)*

%-------------------------- Figure : filigramophone ----------------------------------
\begin{figure}[!htbp]
	\includegraphics[width=\textwidth]{gfx/filigramophone/filigramophone_overview.jpg}
	\caption{filigramophone - vue d'ensemble, débranchée}
	\label{fig:interface:filigramophone}
\end{figure}

%----------------------------------------------------------------------------------------------------------
\subsection{de la tablette à l'écran graphique multitouch}

Format valise-cabine

\subsubsection{première version}
Multitouch overlay de PQlabs, technologie IR, renvoie directement les coordonnées en TUIO. Sensible à la poussière,
%-------------------------- Figure : xypre ----------------------------------
\begin{figure}[!htbp]
	\includegraphics[width=\textwidth]{gfx/05_interfaces/xypre-v1_72dpi.jpg}
	\caption{xypre v1 inauguré durant une performance avec ONE}
	\label{fig:interface:xyprev1_jeu}
\end{figure}

Detecte tout ce qui est en contact (pas seulement les doigts mais aussi les objets) => permet de positionner des éléments de manière stable.

TOTO: insérer photo de la 1ère version (version concert)

\subsubsection{deuxième version}

%-------------------------- Figure : xypre ----------------------------------
\begin{figure}[!htbp]
	\includegraphics[width=\textwidth]{gfx/05_interfaces/Xypre_plan01_72dpi.jpg}
	\caption{Xypre v2 - plans de conception}
	\label{fig:interface:xypre_plans}
\end{figure}

%-------------------------- Figure : xypre ----------------------------------
\begin{figure}[!htbp]
	\includegraphics[width=\textwidth]{gfx/05_interfaces/xypre_overview_unplugged.jpg}
	\caption{xypre - vue d'ensemble, débranchée}
	\label{fig:interface:xypre}
\end{figure}

Ecran multitouch capacitif : détecte les doigts uniquement, pas les objets => motivation pour développer des objets positionnables de manière statique dans mp.TUI

Besoin d'un driver payant pour récupérer le TUIO.

Perte de la pression comme paramètre expressif :\\
intégration de capteur de pression (FSR) et de distance (IR)\\
Utilisation d'un Bela pour une latence très faible du son et un instrument autonome (needs raspberry for the display)\\
Possible extension par l'ordinateur.


%%%%%%%%%%%%%%%%%%%%%%%%%%%%%%%%%%%%%%%%%
\section{Conclusion}
\label{sec:interfaces:conclusion}


\section*{extra material}



%%%%%%%%%%%%%%

\iquote{Though there is a huge range of performer decision, history, and knowledge that will determine their exact method of playing (as established by Jorda [12]), the physical design of the DMI impacts this gesture repertoire by presenting certain affordances.} \cite{bin_hands_2017}

\vspace{-1em}
\begin{itemize}[noitemsep]
\item Faire évoluer une interface en la raffinant (De Laubier, VG).
\item Faire évoluer une interface en rajoutant des choses (Patricia Dallio)
\item Faire évoluer en supprimant des choses (Dumaux)
\item Partir de l'objet (Patrick Saint Denis)
\end{itemize}

 % INCLUDE: interface
%\part{Partie II} 
% !TEX root = ../thesis-example.tex
%
\chapter{Algorithmes interactifs / software}
\label{ch:mapping}

\cleanchapterquote{A picture is worth a thousand words. An interface is worth a thousand pictures.}{Ben Shneiderman}{(Professor for Computer Science)}

\cleanchapterquote{One general effect of the digital revolution is that avant-garde aesthetic strategies became embedded in the commands and interface metaphors of computer software. In short, the avant-garde became materialized in a computer.}{Lev Manovitch}{The language of the new media}

\section{interaction instrumentale}

\section{modèles intermédiaires}
\label{sec:mapping:sec2}
Articles du la notion de DIM (@SMC), article sur le jeu de pitch (@ICMC)


\section{MP : un protocole de connexion modulaire, polyphonique, expressif}
\label{sec:mapping:sec1}

\subsection{motivations et revue des protocoles existants}

\subsection{Description du protocole MP}


\section{Sagrada : extension de MP au DSP}
\label{sec:mapping:sec1}

L’atomisation jusqu’au sample, plutôt que séparation entre DSP et messages de contrôle
=> faust, gen~ et cie : l’export vers des plateformes multiples.


\subsection{motivations et contexte}

\subsection{Le package sagrada pour Max}

\subsection{Performances comparée}
bufgranul
gen~ OLA
granularized	% INCLUDE: mapping
% !TEX root = ../thesis-example.tex
%
\chapter{Représentations visuelles}
\label{ch:visual_representation}

\cleanchapterquote{Ces petits morceaux d’espace visuels,\\
dont la connexion n’est pas donnée d’avance, \\
par quoi voulez vous qu’ils soient connectés, \\
sinon par la main?}{Gilles Deleuze}{\textit{Qu'est ce que l'acte de création?}\\ conférence donnée à la FEMIS \cite{deleuze_deux_2003}}

voir mp.TUI.key pour les images à inclure

%%%%%%%%%%%%%%%%%%%%%%%%%%%%%%%%%%
\section{Aspects visuels du design d'instrument}

\noindent La conception d'un instrument englobe plusieurs aspects qui affectent son allure visuelle. Je passerai ici en revue quelques-uns de ces aspects, en m'appuyant notamment sur des exemples d'instruments acoustiques, pour souligner certaines continuités avec les développements entrepris ici dans le domaine numérique.

\subsection{Adaptation à la production du son}

\todo{éliminer les redites du chapitre hardware}

\noindent L'apparence des instruments est en partie définie par la recherche d'une certaine qualité de son. Bien que cela soit particulièrement évident pour les instruments acoustiques dont les formes ont des conséquences directes sur le rendement sonore (cf. exemple figure \ref{fig:visual_representation:fhole}), les profils particuliers issues de la lutherie traditionnelle ont également donné naissance à un certain nombre d'éléments iconiques (par exemple, les ouïes du violons ou les touches noires et blanches du piano) et de facteurs de forme (par exemple, une taille plus grande produit un son plus grave) associés à l'idée d'un instrument, comme le rappelle Trevor Pinch dans son interview de Robert Moog \cite{pinch_why_2001}: \iquote{Les claviers étaient toujours là, et chaque fois que quelqu'un voulait prendre une photo, pour une raison ou pour une autre, c'était mieux si tu jouais du clavier. Les gens comprennent alors tu fais de la musique. (...) Cette pose [prenant la pose, bras gauche tendu tandis que la main droite joue du clavier] associe graphiquement la musique et la technologie.}\footnote{``The keyboards were always there, and whenever someone wanted to take a picture, for some reason or other it looks good if you’re playing a keyboard. People understand that then you’re making music. (...) This pose here [acts out the pose of the left arm extended while the right hand plays a keyboard] graphically ties in the music and the technology.''}.\\
\indent De plus, les \glspl{DMI} peuvent intégrer des transducteurs acoustiques, tels que des microphones piézoélectriques ou des haut-parleurs tactiles (comme nous l'avons vu à la section \ref{sec:interfaces:part_acoustique}), qui influencent la conception acoustique de leurs composants matériels et donc les facteurs de forme évoqués précédemment.

%------------ Figure : f-hole et boehm -----------
\begin{figure}[!htbp]
	\captionsetup{format=plain}%
	\centering
	\begin{minipage}[t]{0.48\textwidth}
		\includegraphics[width=\linewidth]{gfx/06_visual_representation/f-hole.png}
		\caption{Évolution de la forme des ouïes du violon et influence sur la projection acoustique, d'après \cite{nia_evolution_2015}}
		\label{fig:visual_representation:fhole}
	\end{minipage}
	\hspace{.02\linewidth}
	\begin{minipage}[t]{0.48\textwidth}
	    \includegraphics[width=\linewidth]{gfx/06_visual_representation/Julliot_patent.png}
		\caption{Extrait du brevet de J. Djalma sur \iquote{l'amélioration du clétage des flûtes de Boehm}, 1908.}
		\label{fig:visual_representation:boehm}
	\end{minipage}
\end{figure}


\subsection{Adaptations ergonomiques}

\todo{éliminer les redites du chapitre algorithms}

\noindent L'instrument s'adapte également au corps. Un exemple intéressant est l'évolution du traverso vers la flûte de concert occidentale, à l'aide du système Boehm dans les années 1840 (cf. figure \ref{fig:visual_representation:boehm}). Ce système de clétage découple la topologie gestuelle de la topologie du flux d'air et de la topologie de résonance. En utilisant des manches et des platines, il permet d'acroître la puissance sonore par l'élargissement des trous et leur déplacement à des endroits adéquats pour la résonance, tandis que les touches peuvent être placées à des endroits adaptés à la position des doigts du flûtiste.\\
\indent Le système Boehm peut être qualifié de ``modèle intermédiaire'' entre le geste et la production sonore, fait d'un système mécanique dans ce cas. La plupart des instruments combinent divers ``modèles intermédiaires'' pour amplifier, enrichir, déplacer, focaliser, multiplier les gestes des interprètes et générer des mouvements hors du champ des possibilités du corps humain : pédales de grosse caisse, marteaux et amortisseurs pour piano, archets et plectres, etc.

\subsection{Représentations liées à la théorie musicale}

%------------ Figure : keyboard et TUI - scale -----------
\begin{figure}[!htbp]
	\captionsetup{format=plain}%
	\centering
	\begin{minipage}[t]{0.48\textwidth}
		\includegraphics[width=\linewidth]{gfx/06_visual_representation/Mersenne_clavier.png}
		\caption[Clavier à 27 touches de Mersenne]{Le ``clavier parfait de vingt-sept marches sur l'Octave' de Mersenne (1636)}
		\label{fig:visual_representation:MersenneKeyboard}
	\end{minipage}
	\hspace{.02\linewidth}
	\begin{minipage}[t]{0.48\textwidth}
	    \includegraphics[width=\linewidth]{gfx/06_visual_representation/mpTUI_pitchgrid_72dpi.png}
		\caption[Grille de hauteur micro-tonale dans mp.TUI]{Une grille de hauteur avec une représentation micro-tonale réalisée avec mp.TUI. La luminosité des lignes verticales varie en fonction de la quantité de quantification.}
		\label{fig:visual_representation:pitch_grid}
	\end{minipage}
\end{figure}

\noindent Les instruments de musique intègrent également des éléments de théorie musicale. Par exemple, la partie supérieure d'un clavier (touches noires et blanches) représente la gamme chromatique, tandis que la partie inférieure (touches blanches seulement) représente la gamme diatonique de Do majeur. Le dimensionnement et le positionnement de ces touches est un compromis intéressant entre les contraintes mécaniques du système de marteaux et une représentation uniforme des échelles diatonique et chromatique. De plus, la largeur de l'octave est telle qu'elle tient sous une main tendue et vient réifier dans l'instrument la notion d'équivalence des octaves, en permettant de jouer n'importe quel intervalle à l'intérieur d'une octave avec une seule main. Les claviers ont par ailleurs fait l'objet de nombreux développements expérimentaux\footnote{Voir notamment \cite{haury_petite_1999} pour un historique du clavier.} avec des dispositions de notes utilisant des grilles hexagonales ou plusieurs couches de touches (figure \ref{fig:visual_representation:MersenneKeyboard}) pour permettre le jeu dans des systèmes d'intervalles micro-tonaux.\\
\indent En tant que système symbolique, la théorie musicale peut être facilement encodée dans les ordinateurs. Les logiciels de production musicale contiennent tellement de fonctions et de règles basées sur la théorie musicale qu'il serait difficile de toutes les représenter sur l'interface. Thor Magnusson parle ainsi ``d'outils épistémiques'' pour décrire les \glspl{DMI}, affirmant qu'il sont conçus avec ``un tel degré de pertinence symbolique qu'ils deviennent un système de connaissance et de pensée dans leurs propres termes'' \cite{magnusson_epistemic_2009}. Pour le \textit{musicien numérique}, ce ``système de connaissances'' est un paysage imaginaire à explorer, un territoire sonore pour lequel l'interface de l'instrument peut métaphoriquement prendre le rôle d'une carte géographique, ou d'un cockpit de pilote \cite{vertegaal_towards_1996}.

\subsection{Représentations liées au contexte de performance}

\noindent Si l'on considère les instrument de musique comme des ``instruments pour musiquer'' en reprenant la définition de Christopher Small (cf. section \ref{sec:introduction:preamble}), alors les partitions, les salles de concert, le public et plus généralement, le contexte de la performance influencent aussi la conception et la représentation des instruments. Les partitions orientées (figure \ref{fig:visual_representation:table_music}) sont un exemple d'adaptation de la partition au contexte de la ``musique de table'', permettant dans ce cas aux musiciens de lire une même partition lorsqu'ils sont assis autour d'une table. De même, les \glspl{DMI} collectifs (cf. \ref{sec:ephemeral:origins:collectiveDMIs}) peuvent adapter leur représentation au nombre d'interprètes en présentant à chacun d'eux un groupe d'éléments d'interface utilisateur orientés vers eux (figure \ref{fig:visual_representation:multi_orientation}).\\
\indent Comme exemples de l'influence du lieu de concert sur le design visuel de l'instrument, on peut notamment évoquer les modélisations de l'espace acoustique du lieu pour venir contrôler la spatialisation du son (cf. figure \ref{fig:visual_representation:spat}) ou --~à l'inverse~-- la projection sur le lieu d'un \textit{mapping vidéo} en correspondance avec la musique\footnote{ou une ``musique visuelle'' comme l'appelle Serge de Laubier} (cf. \ref{fig:visual_representation:pucemuse-monument}).

%------------ Figure : keyboard et TUI - scale -----------
\begin{figure}[!htbp]
	\captionsetup{format=plain}%
	\centering
	\begin{minipage}[t]{0.48\textwidth}
		\includegraphics[width=\linewidth]{gfx/06_visual_representation/Dowland-firstBookOfSonges.png}
		\caption[Partition ``de table'' à plusieurs voix]{Partition ``de table'' à plusieurs voix (John Dowland - First Booke of Songes or Ayres. Édition Peter Short, London, 1597)}
		\label{fig:visual_representation:table_music}
	\end{minipage}
	\hspace{.02\linewidth}
	\begin{minipage}[t]{0.48\textwidth}
	    \includegraphics[width=\linewidth]{gfx/06_visual_representation/mpTUI_multi-orientation.png}
		\caption[Instrument simple adapté pour 6 joueurs]{Instrument simple adaptée pour 6 joueurs située autour d'une interface de jeu commune.}
		\label{fig:visual_representation:multi_orientation}
	\end{minipage}
\end{figure}
%------------ Figure : keyboard et TUI - scale -----------

%------------ Figure : spat et PM -----------
\begin{figure}[!htbp]
	\captionsetup{format=plain}%
	\centering
	\begin{minipage}[t]{0.48\textwidth}
		\includegraphics[width=\linewidth]{gfx/06_visual_representation/IRCAM-spat.jpg}
		\caption[IRCAM Spat Revolution]{L'interface du logiciel ``Spat Revolution'' de l'IRCAM modélise l'espace de projection acoustique du lieu de concert.}
		\label{fig:visual_representation:spat}
	\end{minipage}
	\hspace{.02\linewidth}
	\begin{minipage}[t]{0.48\textwidth}
	    \includegraphics[width=\linewidth]{gfx/06_visual_representation/PuceMuse-Facade.jpg}
		\caption[Projection monumentale, Puce Muse]{Projections monumentales de ``musique visuelle'', controlée de manière synchrone à la ``musique sonore''. Photographie © Puce Muse.}
		\label{fig:visual_representation:pucemuse-monument}
	\end{minipage}
\end{figure}
%------------ Figure : spat et PM -----------

\subsection{Représentations liées à l'expérimentation}

\noindent Le processus de conception des instruments de musique contient une grande part de travail empirique. L'ajustement des paramètres d'un instrument nécessitera souvent une rétroaction sonore et visuelle directe pour affiner les réglages à la main, jusqu'à ce que l'instrument sonne et se prête au jeu. Ainsi, l'interface des \gls{DMI} est souvent fournie en valeurs numériques qui permettent d'essayer de comprendre ce qui se passe dans l'instrument et d'écouter la corrélation entre valeurs de paramètres et résultat sonore. C'est notamment pour cette raison que la ``programmation visuelle'' est un paradigme courant dans les environnements de lutherie numérique, tels que Max, PureData, Reaktor, 

\subsection{Aspects esthétiques : l'instrument œuvre d'art}

\noindent Le design de l'interface visuelle de l'instrument se résume rarement à ses aspects fonctionnels. Les instruments de musique sont souvent l'œuvre d'un artisanat très pointu, voire des objets d'art. La finesse des détails et le souci de leur aspect esthétique jouent ainsi une part éminente dans le design final. Dans les interfaces visuelles des logiciels audio, on retrouve souvent ce soin esthétique, que cela soit sous forme d'un skeuomorphisme\footnote{le terme de skeuomorphisme est utilisé pour définir un élément de design dont la forme n'est pas directement liée à la fonction, mais qui reproduit de manière ornementale un élément qui était nécessaire dans l'objet d'origine.} (figure \ref{fig:visual_representation:skeuomorphisme}) cherchant à imiter le bois des instruments acoustiques, le métal et l'aspect physique des boutons sur les équipements hardware, ou d'un design plus épuré\footnote{sur les connexions entre le design d'interfaces et les arts graphiques, voir également les étonnants résultats obtenus en appliquant des modèles AI pour créer une interface à partir d'images photographiques dans \cite{troyer_mondrian_2019}.}  (figure \ref{fig:visual_representation:apparatum}).

%------------ Figure : skeuomorphisme et soin du design -----------
\begin{figure}[!htbp]
	\captionsetup{format=plain}%
	\centering
	\begin{minipage}[t]{0.48\textwidth}
		\includegraphics[width=\linewidth]{gfx/06_visual_representation/Redstair_GEARcompressor.png}
		\caption[Skeuomorphisme dans les logiciels audio]{Le skeuomorphisme dans les logiciels audio témoigne de l'importance accordée à l'esthétique, au-delà des fonctionnalités de l'interface. Photographie Klaus Göttling.}
		\label{fig:visual_representation:skeuomorphisme}
	\end{minipage}
	\hspace{.02\linewidth}
	\begin{minipage}[t]{0.48\textwidth}
	    \includegraphics[width=\linewidth]{gfx/06_visual_representation/2018_06_26_PAN_GENERATOR_APPARATUM0396.jpg}
		\caption[Apparatum, par panGenerator]{Le design minimal et soigné de l'instrument Apparatum. © panGenerator}
		\label{fig:visual_representation:apparatum}
	\end{minipage}
\end{figure}
%------------ Figure : skeuomorphisme et soin du design -----------

\noindent En tant qu'élément scénographique, la représentation visuelle joue un rôle essentiel dans l'esthétique et la poésie de l'instrument.
Exemple dans FIB\_R : toute la performance visuelle re-projetée sur l'écran correspond à ce que les instrumentistes ont face à eux sur leur tablette. A la fois espace d'interaction et création visuelle esthétique/poétique.



%%%%%%%%%%%%%%%%%%%%%%%%%%%%%%%%%%%%%%%%%
\section{L'écran comme interface de jeu}

\subsection{L'écran, cockpit du musicien?}

\noindent En tant qu'\textit{outils épistémiques}, les \glspl{DMI} mettent en jeu un nombre extrêmement élevés de flux de données et de processus qui interagissent entre eux et avec l'interface. La représentation visuelle peut faciliter le \textit{monitoring} du bon fonctionnement de ces processus\footnote{que cela soit lors de l'élaboration d'un instrument, voire durant sa performance, quand la mémoire fait défaut, ou que l'instrument instable est sujet à bug}, de leurs dynamiques, et la sélection de réglages dans des banques de données (presets, samples, etc.).

\subsubsection{Espace infini et savoir cartographique} 

\noindent L'écran donne en effet accès à un espace virtuellement infini, grâce à des systèmes d'onglets, de fenêtres multiples, de menus condensant différentes options, de zoom et déplacement sur des zones d'intérêt. L'exploration y est autant possible ``horizontalement'', c'est-à-dire parmis les différents éléments d'un même niveau de complexité, que ``verticalement'', c'est-à-dire dans la profondeur des différentes couches d'un même élément.\\
\indent Ce vaste territoire possède ainsi sa géographie et ses ``moyens de transport''. Si l'exploration peut se faire horizontalement (de manière ``touristique'', en parcourant les divers processus à disposition) ou en profondeur (de manière ``spéléologique'', en plongeant dans les entrailles d'un processus), elle peut aussi se faire par ``télé-transportation'': un preset peut rappeler une configuration complète correspondant à un terrain de jeu totalement différent.\\
\indent A mesure que les données emmagasinables sur les ordinateurs augmente --~et plus encore, sur le net en tant qu'il constitue une excroissance de l'instrument numérique~-- la fonction de cartographie et de recherche de l'information utile prend de plus en plus d'importance. On peut en observer les signes dans la généralisation de l'intégration de moteurs de recherche dans les logiciels, notamment audio. Le numérique fait évoluer notre rapport aux outils d'une posture consistant à ``savoir le contenu'' à une posture consistant à ``savoir que ce contenu existe et comment le trouver''\footnote{\iquote{(...) my students tell me, the biggest problem they have in music production is finding their samples on their hard drive. They say ``I have so many kick drums on my Drive, that I've downloaded from so many places, that I can never find the one I want''. (...) now maybe it means that you don't need Stradivarius, you need a database programmer, that the most important thing in your life would be like, a really brilliant database that would allow you to intuitively retrieve whatever sample you wanted... In other words, that's in such a different domain but it may be, as I say, the single most important thing for a composer working today.} Nicolas Collins, cf. annexe \ref{appendix:collins}}.\\
\indent Cependant, l'action de rechercher un contenu n'est pas \textit{a priori} un geste expressif. La conception d'un \gls{DMI} pour la performance passe ainsi par la circonscription d'un territoire dans lequel l'interaction soit suffisamment directe et intuitive pour que le musicien n'ait pas à réfléchir à la manière de se rendre à l'endroit désiré\footnote{``(...) it's a difficult thing to articulate but I think that if you give somebody too many choices of things to do, it will not be an expressive musical instrument because there's gonna be too much thinking(...) you have to sort of find a spot in between'' Nicolas Collins, cf. annexe \ref{appendix:collins}}.

\subsubsection{Temporalité de l'interface} 

\noindent À la différence d'une interface physique, l'information sur écran n'est pas seulement représentée de manière \textit{spatiale} mais également \textit{temporelle}, ce qui offre des possibilités tout à faire adaptées pour la représentation des phénomènes sonores et des processus musicaux. Ainsi, l'écran peut assurer une représentation du tempo, des patterns rythmiques ou encore de la progression d'un processus (e.g. l'état d'avancement d'un séquenceur) dans leur évolution dynamique, ce qui facilite leur anticipation (cf. exemple de type ``time timer'' sur la figure \ref{fig:visual_representation:dialButton}).\\
%------------ Figure : skeuomorphisme et soin du design -----------
\begin{figure}[!htbp]
	\captionsetup{format=plain}%
	\centering
	\begin{minipage}[t]{0.48\textwidth}
		\includegraphics[width=\linewidth]{gfx/06_visual_representation/mpTUI-DialButton.png}
		\caption[Visualisation temporelle à l'aide de chronomètres visuels]{Visualisation temporelle à l'aide de chronomètres visuels de la librairie mp.TUI}
		\label{fig:visual_representation:dialButton}
	\end{minipage}
	\hspace{.02\linewidth}
	\begin{minipage}[t]{0.48\textwidth}
	    \includegraphics[width=\linewidth]{gfx/06_visual_representation/LAM-DSW.png}
		\caption[Deux échelles temporelles différentes d'un flux audio]{Visualisation du son ``en cours'' et de sa trace accumulée à plus long terme, en utilisant deux échelles temporelles.}
		\label{fig:visual_representation:DSW}
	\end{minipage}
\end{figure}
%------------ Figure : skeuomorphisme et soin du design -----------
\indent Également, l'image peut faciliter la compréhension de processus musicaux sur le long terme, en particulier le déploiement de grandes formes temporelle, qui nécessite une mémoire auditive faisant parfois défaut. Si le son est évanescent, l'image peut en effet laisser une trace sensible, ``s'écrire'' au fur et à mesure de sa performance\footnote{Un tel procédé à été utilisé lors d'un concert de ONE, durant lequel les formes d'ondes des sons joués par chaque musicien étaient vidéo-projetées et s'aggloméraient telle des stalagmites, rendant visible les intensités de jeu de chacun au cours du concert(L'objet LAM.jit.gl.DSW\textasciitilde{} développé à cette fin est disponible dans la LAM-lib).} (cf. figure \ref{fig:visual_representation:DSW}). De manière générale, la ``partition'', qu'elle soit trace du passé ou pré-figuration de ``l'à-venir'', peut se retrouvée intégrée dans l'interface instrumentale (cf. chapitre \ref{ch:notation}). Cette ``partition'' n'est pas nécessairement une représentation du temps musical \textit{in extenso}, mais peut être présente sous des formes fragmentées de temps enregistré à la volée ou de séquences pré-programmées.\\
\indent Une autre conséquence du dynamisme de l'interface graphique est la possibilité d'afficher des informations de manière contextuelle, sur le plan spatial (e.g. en affichant des informations uniquement quand on interagit avec une certaine zone de l'interface) comme sur le plan temporel (e.g. en affichant des informations en fonction d'événements en cours).\\
\indent Cette idée de l'interface comme ``cockpit du musicien'' \cite{vertegaal_towards_1996} hérite du contexte technico-scientifique dans lequel les \glspl{DMI} ont vu le jour: les interfaces des premiers ordinateurs ainsi que le matériel de l'ingénierie du son ressemblant bien souvent à celles des cockpits d'avion. Pourtant, si l'image d'un cockpit évoque des gestes de réglage et d'ajustement lents, sur un grand nombre de paramètres mesurés sur des échelles absolues (que cela soit sur un avion ou une table de mixage), la notion d'instrument et de jeu musical appelle de son côté des gestes expressifs, vifs, et souvent sur des échelles relatives\footnote{...comme en témoigne l'importance de valeurs relatives telles que les intervalles ou les nuances (crescendo/decrescendo, ralentendo/accelerando, etc.) dans l'écriture musicale. Cf. également les propos de Serge de Laubier à ce sujet dans l'annexe \ref{appendix:delaubier}}.

%%%%%%%%%%%%%%%%%%%%%%%%%%%%%%%%%%%%%%%%%
\subsection{L'écran comme interface tangible}

\noindent Dans la chronologie rassemblée par Bill Buxton\footnote{Bien qu'il ne se soit pas limité à des applications musicales, les développements de Bill Buxton a été également largement orienté par le contrôle de la synthèse et l'idée de lutherie numérique.} sur l'histoire des interfaces multitouch \cite{buxton_multi-touch_2007}, il est intéressant de noter que les premières interfaces intégrant des capteurs capacitifs ont été des instruments de musique électroniques\footnote{L'electronic sackbut de Hugh Le Caine, et les claviers de Don Buchla.} (figure \ref{fig:visual_representation:buchla}), utilisant des capteurs capacitifs en parallèle pour le contrôle polyphonique de la synthèse dans les années 1950. De même, si le premier \textit{écran multitouch} semble avoir été inventé au Bell Labs en 1984, le premier à avoir été commercialisé pour le grand public a été le Lemur (figure \ref{fig:visual_representation:lemur}) développé par Jazz Mutant en 2003, encore pour le contrôle musical.

%------------ Figure : multitouch instruments -----------
\begin{figure}[!htbp]
	\captionsetup{format=plain}%
	\centering
	\begin{minipage}[t]{0.48\textwidth}
		\includegraphics[width=\linewidth]{gfx/06_visual_representation/Buchla_music_easel.jpg}
		\caption[Le clavier multitouch du synthétiseur Buchla music easel]{Le clavier multitouch du synthétiseur Buchla music easel (1972).}
		\label{fig:visual_representation:buchla}
	\end{minipage}
	\hspace{.02\linewidth}
	\begin{minipage}[t]{0.48\textwidth}
	    \includegraphics[width=\linewidth]{gfx/06_visual_representation/Lemur.jpg}
		\caption[Le Lemur de Jazz Mutant et son écran multitouch]{Le Lemur de Jazz Mutant et son écran multitouch (2003).}
		\label{fig:visual_representation:lemur}
	\end{minipage}
\end{figure}
%------------ Figure : multitouch instruments -----------

\noindent Les interfaces tangibles se sont multipliées dans les appareils grands public au tournant du siècle, notamment sous l'impulsion de projets de recherche tels que ceux du \textit{Tangible Media Group} dirigé par Hiroshi Ishii au Media Lab du \gls{MIT}. Si un certain nombre de gestes (la notion de geste étant ici souvent réduite au gestes des doigts, voire du pouce, de l'index et du majeur seulement) sont devenus des standards pour l'interaction sur ce type d'interfaces (tel le \textit{pinch-zoom}, le \textit{swipe}, etc.), la recherche sur les interactions possibles reste un domaine récent et encore largement inexploré.\\
\indent Les écrans multitouch offrent de nombreux avantages qui ont déjà été soulignés par Bill Buxton \cite{buxton_multi-touch_2007} et Sergi Jorda \cite{jorda_digital_2005}, notamment :

\vspace{-1em}
\begin{itemize}[noitemsep]
	\item l'utilisation possible de plusieurs doigts, plusieurs mains, plusieurs personnes en même temps, en comparaison de la souris, qui permet une interaction polyphonique, parallèle plutôt que séquentielle;
	\item la cohésion entre le contrôle d'un objet et sa représentation et la relation sensuelle d'intimité qui se créé en l'absence de distance entre l'objet virtuel et celui/celle qui le manipule;
\end{itemize}

\noindent La cohésion entre la représentation et le contrôle d'un objet virtuel\footnote{... qui, rappelons-le, n'est qu'une ``synchrèse'', c'est à dire un artéfact liant les deux phénomènes par leur concommittence spatio-temporelle} facilite l'intuitivité de l'interface en abolissant la distance entre ces deux aspects de l'interaction. Au delà de cet intérêt ergonomique, cela permet d'imaginer de nouveaux scénarios d'interactions dynamiques. Même de simples interactions usuelles, comme celle de la bureautiques (sliders, boutons, etc), peuvent être ré-inventées avec des comportements alternatifs, dans une perspective d'utilisation musicale, comme nous le verrons plus loin.\\
\indent Cette idée est notamment poursuivie dans le domaine des \gls{IHM}, avec le dessein d'intégration du numérique dans les objets du monde physique, telle que présenté notamment dans l'article ``Tangible Bits'' d'Ishii et Ullmer \cite{ishii_tangible_1997} et reprise par Sergi Jordà dans les 25 principes qu'il adopte pour le développement de la ReacTable\footnote{``For including realtime interactive visualizations and, at the same time, overcoming
mouse limitations without adding indirections, interfaces should be able to reflect their own states and behaviors. They should integrate, like the abacus, both representation and control.'' \cite{jorda_digital_2005}}.\\
\indent Ce faisant, la cohésion entre le geste et la représentation ``acquise'' dans les écrans tactile laisse la place à des scénarios d'interaction subersifs (sur le geste subversif, cf. section \ref{sec:gesture:subversion}) qui se jouent de cette synchrèse artificielle\footnote{voir par exemple la série de performances ``Screen to screen'' de l'artiste Vincent Brocquaire, basées sur cette duplicité de l'écran/interface. \url{http://www.vincentbroquaire.com/screen-to-screen.html}}.\\


%%%%%%%%%%%%%%%%%%%%%%%%%%%%%%%%%%%%%%%%%
\subsection{Topologie dynamique}

\subsubsection{Repères statiques, amovibles, dynamiques}

%------------------- Visual markers -----------------------
\begin{figure}[!htbp]
	\captionsetup{format=plain}%
	\makebox[\linewidth][c]{%
		\begin{subfigure}[b]{.34\textwidth}
			\centering
			\includegraphics[width=.95\textwidth]{gfx/06_visual_representation/guitar-frette.jpg}
			\caption{\textbf{statiques}, sur un manche \\de guitare}
		\end{subfigure}%
		\begin{subfigure}[b]{.34\textwidth}
			\centering
			\includegraphics[width=.95\textwidth]{gfx/06_visual_representation/filigramophone-visualMarkers.jpg}
			\caption{\textbf{amovibles}, sur une tablette \\graphique}
		\end{subfigure}%
		\begin{subfigure}[b]{.34\textwidth}
			\centering
			\includegraphics[width=.95\textwidth]{gfx/06_visual_representation/mpTUI-visualMarkers.png}
			\caption{\textbf{dynamiques}, sur un écran \\multitouch}
		\end{subfigure}%
	}
	\caption{Repères visuels sur différentes interfaces instrumentales.}
	\label{fig:visual_representation:visual-markers}
\end{figure}
%------------------- Visual markers -----------------------

\noindent Le positionnement des différents capteurs de l'interface d'un \gls{DMI} définit une topologie de l'interaction qui peut être soulignée par des repères visuels (cf. figure \ref{fig:visual_representation:visual-markers}). On peut ainsi distinguer trois cas de figure, concernant le positionnement de ces repères visuels :
\vspace{-1em}
\begin{itemize}[noitemsep]
	\item \textbf{statique} : les repères sont fixes et l'algorithme sous-jacent pourra être ``ré-accordé'' dans les limites imposées par cette configuration (e.g. il est possible de changer la tension des cordes d'une guitare, sans que la position des frettes soient invalidée);
	\item \textbf{amovible} : les repères peuvent être changés ``physiquement'', soit que leurs positions soient réglables (e.g. comme c'est le cas pour les frettes du sitar), soit qu'ils soient interchangeables (e.g. les différentes revêtements de l'interface \textit{Joué} ou du \textit{Sensel Morph}, ou le calque glissable dans une tablette graphique);
	\item \textbf{dynamique} : cette solution est rendue possible sur les écrans tactiles, mais n'offre pour l'instant pas les qualités haptiques de repères en relief permises par les repères statiques ou amovibles\footnote{Un certain nombre de prototypes d'écrans à formes dynamiques ont été développés cette dernière décennie (voir \cite{follmer_inform_2013, siu_shapeshift_2018}), mais restent encore très rustiques et ne semblent pas encore prêts pour une diffusion commerciale};
\end{itemize}


\subsubsection{Nouvelles formes et nouvelles fonctions}

\noindent De nombreuses règles ont été formulées dans le domaine du design, de l'architecture, des \gls{IHM}, et de la visualisation de données, telles que le célèbre adagage ``la forme suit la fonction'', pris comme paradigme dans le design et l'architecture moderne, autant que dans le design des interface graphiques. Ces règles de design prônent généralement le minimalisme, la lisibilité et l'efficacité du design. L'ouvrage d'Edward W. Tufte ``The visual display of quantitative information'' \cite{tufte_visual_2001}, une des principales références dans le domaine de la visualisation de données, rappelle ainsi ces ``Principes de l'Excellence Graphique'' :
\vspace{-1em}
\begin{itemize}[noitemsep]
	\item l'excellence graphique est une présentation bien conçue de données intéressantes — une affaire de \textit{substance}, de \textit{statistiques} et de \textit{design};
	\item l'excellence graphique consiste en la communication claire, précise et efficace d'idées complexes;
	\item l'excellence graphique est ce qui donne le plus grand nombre d'idées dans le temps le plus court, avec le moins d'encre possible et dans le plus petit espace;
	\item l'excellence graphique est presque toujours multivariée;
	\item elle requiert de dire la vérité sur les données.
\end{itemize}

\noindent On retrouve des règles similaires dans les (nombreuses) préconisations concernant le design d'interfaces graphiques, guidées par une recherche de fonctionnalité, lisibilité et transparence maximale. Bien que toutes ces règles puissent être utiles pour le design de la partie graphique d'un \gls{DMI}, il faut cependant garder en tête que les instruments de musique ne sont pas des interfaces dont la raison d'être est définie par leur fonctionnalité (on n'achète pas un instrument pour se rendre la vie plus facile), et que les notions d'\textit{efficacité}, de \textit{lisibilité} et de \textit{vérité} y sont particulièrement sujettes à caution.


\subsubsection{Mettre les doigts dans la prise}
% \begin{wrapfigure}{R}{0.5\textwidth}
% 	\captionsetup{format=plain}%
% 	\centering 
% 	\includegraphics[width=0.48\textwidth]{gfx/06_visual_representation/Xypre-Live.jpg}
% 	\caption[Modèle intermédiaire stochastique manipulé sur écran]{Modèle intermédiaire stochastique manipulé directement sur écran, durant la performance FIB\_R.}
% 	\label{fig:visual_representation:xypre-live}
% \end{wrapfigure}
% \par
%-------------------------- Figure : Fingers on MID ------------------------
\begin{figure}[!htbp]
	\captionsetup{format=plain}%
	\includegraphics[width=\textwidth]{gfx/06_visual_representation/Xypre-Live.jpg}
	\caption[Modèle intermédiaire stochastique manipulé sur écran]{Modèle intermédiaire stochastique manipulé directement sur écran (interface Xypre présentée au chapitre \ref{ch:interfaces}), durant la performance FIB\_R.}
	\label{fig:visual_representation:xypre-live}
\end{figure}
%-------------------------- Figure : Fingers on MID ------------------------

\noindent Si dans un instrument acoustique, l'énergie se transmet des gestes du musicien à l'instrument qui produit le son, cette chaîne d'interaction peut se retrouver perceptivement renversée dans les \glspl{DMI}. Le processus, ou ``modèle intermédiaire'' (cf. \ref{sec:algorithms:MID}) qui \textit{tourne} sur la machine possède son propre mouvement autonome, que les doigts viennent perturber, infléchir, canaliser, étouffer, aiguiser, filtrer... La représentation visuelle de modèles intermédiaires dynamiques vient rendre manifeste ces \textit{forces invisibles}\footnote{\iquote{En art, et en peinture comme en musique, il ne s’agit pas de reproduire ou d’inventer des formes, mais de capter des forces. (...) La tâche de la peinture est définie comme la tentative de rendre visibles des forces qui ne le sont pas. De même la musique s’efforce de rendre sonores des forces qui ne le sont pas.} Gilles Deleuze \cite{deleuze_francis_1981}} et appelle ainsi des gestes \textit{en réponse} aux mouvements internes du modèle, qui redéfinit en permanence l'espace de son interaction, dont la topologie n'est pas donnée d'avance et qu'il s'agit apprivoiser (cf. figure \ref{fig:visual_representation:xypre-live}).


\subsection*{Extra material : todo remove}

Apprendre indépendamment de la transposition
\iquote{I love the piano sound but not the difficulty of learning the variations from one key to another.  The LinnStrument with its 4th tuning avoids all of those issues.} Jeff Moen about the linnstrument (\url{http://jeffmoen.com/how_i_got_here.html})

%%%%%%%%%%%%%%%%%%%%%%%%%%%%%%%%%%%%%%%%%
\section{La librairie mp.TUI pour Max}

\todo{Rajouter des figures pour édition directe (shipped with handles) et modèle voronoi}

\noindent La bibliothèque mp.TUI\footnote{Sources disponible sur \url{https://github.com/LAM-IJLRA/ModularPolyphony-TUI/}.} pour Max propose des composants graphiques permettant le contrôle et la représentation d'une interaction polyphonique et modulaire. Elle offre également (et surtout) un environnement ouvert pour la programmation de nouveaux objets graphiques et de nouvelles interactions en mettant à disposition des briques de base prenant en charge les fonctions élémentaires usuelles. Cette section présente une brève revue des outils existants et des motivations avant de détailler le fonctionnement de la librairie.

\subsection{Motivations}

\subsubsection{Antécédants} 

\noindent Plusieurs développements ont été réalisés depuis 2003 permettant de contrôler, de différentes manières, des patchs Max via une interface \textit{multitouch}. Parmi les réalisations, on peut noter notamment :
\vspace{-1em}
\begin{itemize}[noitemsep]
	\item \textbf{le Lemur} : une interface \textit{multitouch} personnalisable développé par la société JazzMutant entre 2003 et 2011, d'abord sur une interface hardware dédiée avant d'être portée sous la forme d'une application pour pour tablettes et smartphones\footnote{distribuée par la société Liine \url{https://liine.net/en/products/lemur/}}. Cette interface était pionnière à la fois à une époque om le \textit{multitouch} grand public était naissant, et par la possibilité de personnaliser l'interface à l'aide d'un éditeur tournant sur un ordinateur standard. La personnalisation était cependant limitée au fait de choisir les composants graphiques parmi un ensemble de composants alignables sur une grille;

	\item \textbf{la librairie MMF} : ``Max Multitouch Framework''\footnote{Vidéo présentant MMF: \url{https://www.youtube.com/watch?v=EEkj85GU_is}} développé par Mathieu Chamagne dans le cadre du projet \gls{ANR} Virage (2008-2010), qui permettait de contrôler certains éléments de \gls{GUI} de Max. La contrainte principale réside dans le fait que le contrôle des éléments de \gls{GUI} de Max est assez gourmand en \gls{CPU} et rend difficile l'usage de plusieurs doigts.

	\item \textbf{TouchOSC} : développé en 2008, touchOSC\footnote{\url{https://hexler.net/products/touchosc}} est basé sur les principes du Lemur mais développé comme une application pour tablette et smartPhones, permettant de renvoyer des données \gls{MIDI} ou \gls{OSC}.

	\item \textbf{Mira} : En 2013, Sam Tarakajian présente Mira\footnote{dont la vidéo de présentaion est intéressante à plus d'un titre \url{https://vimeo.com/63846055}}, qui sera distribué ensuite par Cycling'74\footnote{\url{https://cycling74.com/products/mira}}, la société développant Max. Mira simplifie grandement la connection entre un iPad et un patch Max dans la mesure où les éléments de \gls{GUI} d'un patch Max sont directement transposés sur l'interface \textit{multitouch}, sans qu'il y ait besoin de recréer des connections manuellement entre deux interfaces différentes, comme cela pouvait être le cas avec le Lemur ou TouchOSC.

	\item \textbf{Max multitouch} Enfin, Cycling'74 a considérablement améliorer la mécanique de patching depuis la version 8, et a commencé fin 2018 l'implémentation native du \textit{multitouch} dans l'éditeur de patch, pour l'instant limitée à un certain nombre d'objet et à la version Microsoft Windows.
\end{itemize}

\subsubsection{De la nécessité de réinventer la roue} 

\noindent À la vue de ces développements, on peut légitimement se demander quel est l'intérêt de développer une librairie graphique pour le \textit{multitouch} dans Max. La réponse est essentiellement liée à la conviction que les interfaces graphiques, tout comme le domaine du mapping, ne consiste pas en un simple agencement de composants standards mais un champ ouvert à la créativité pour développer de nouvelles relations. \\
\indent Si l'on prend l'exemple d'un objet aussi basique et courant qu'un \textit{slider}; l'objet paraît \textit{a priori} simple et univoque. Pourtant, si l'on détaille le mécanisme de corrélation entre le geste et le comportement de l'objet, on s'aperçoit rapidement que de nombreux scénarios sont non-seulement possibles, mais pertinents selon le contexte \footnote{Un exemple notoire sur l'usage du \textit{slider} est l'introduction, en 2011, de ce qu'Apple nomma ``scrolling naturel'', en inversant le sens de défilement des documents pour répondre aux nouvelles interfaces tactiles et en balayant plus de 25 ans de convention d'usage. Les réactions fûrent largement hostiles au début, mais le nouvel usage finit par être non-seulement accepté mais également considéré comme plus naturel. Dans ce cas précis, la question se pose sur la partie du slider sur laquelle on agit : le contenu ou bien le cadre. Si l'on prend un équivalent dans le domaine de l'audio, le déplacement d'une tête de lecture dans un échantillon, tel que couramment représenté dans les interfaces de synthèse granulaire, est aussi approprié que l'idée de déplacer l'échantillon ``sous'' la tête de lecture, et qui constitue la manière dont fonctionnait les lecteurs à bande.}, de sorte que les implémentations d'interactions basiques --~telle que celle d'un slider~-- dans la plupart des logiciels de bureautique ne sont pas nécessairement les plus adaptées au contrôle musical. Ainsi, un musicien utilisera une telle interface linéaire avec \underline{tous ses doigts}, par exemple pour jouer des intervalles mélodiques, usage pour lequel la mémoire des intervalles s'avère très ergonomique. Il suffit de tester le fonctionnement des sliders sur les tablettes dans la plupart des applications pour constater que ce genre d'interaction n'y est, sans surprise, pas prévu. De même, la plupart des logiciels contraignent l'orientation des composants graphiques selon les axes verticaux et orthogonaux (et c'est pour l'instant de toutes les librairies sus-mentionnées), en se basant sur l'usage habituel de l'écran comme une feuille de papier vertical qui se lit de haut en bas et de gauche à droite\footnote{On peut noter le choix original d'un design basé sur le cercle dans la ReacTable, inspiré de celui de l'AudioPad développé en 2002 au \gls{MIT}, pour faciliter un usage collectif}.

Ainsi, la prise en charge native du \textit{multitouch} dans l'interface de Max ne saurait être satisfaisante en terme créatif si elle ne laisse pas les relations entre les objets graphiques et le geste ouvertes à la reprogrammation. 

\subsection{Utilisation de MP pour le contrôle \textit{multitouch} de GUI}

\noindent Comme son nom l'indique, la librairie mp.TUI est construite sur le protocole MP (cf. section \ref{sec:algorithms:MP}). Elle fournit un cadre, basé sur les logiques de patching de Max, pour créer de nouveaux composants \gls{GUI} \textit{multitouch} dans un contexte graphique OpenGL et surmonter certaines limitations de l'interface graphique native de l'environnement de patching de Max. Par exemple, les interfaces graphiques sont généralement orientées sur une disposition horizontale/verticale avec une orientation de lecture du haut vers le bas alors qu'on peut souhaiter avoir plusieurs orientations, comme dans la situation présentée sur la figure \ref{fig:visual_representation:multi_orientation}. La superposition de divers composants peut nécessiter des couleurs et des transparents personnalisés, et l'on peut souhaiter inclure des interfaces visuelles plus complexes que les curseurs et les boutons, par exemple des particules, des vidéos, des modèles 3d, des shaders (cf. figure \ref{fig:visual_representation:phonetogramme}), etc.\\
\noindent Les composants de la bibliothèque sont d'un ensemble d'abstractions de trois types :
\vspace{-1em}
\begin{itemize}[noitemsep]
	\item \textbf{des composants système}, qui implémentent les fonctions de base permettant la communication entre les objets de la \gls{GUI}. En particulier, l'objet \verb|mp.TUI.hub| récupère les données de la souris ainsi que les messages \gls{TUIO} reçus par \gls{UDP} et les envoie aux composants graphiques sélectionnés;
	\item \textbf{des éléments de GUI}, qui sont des instances prêtes à l'emploi de composants courants ou moins courants tels que curseurs, claviers, graphes, etc.;
	\item \textbf{des outils}, un ensemble d'abstractions qui permettent de créer facilement de nouveaux composants en proposant des fonctions utiles pour la conception d'interaction (transformation de la visualisation, gestion de la polyphonie sur un élément, interaction tels que pinch-zoom, calcul de dérivées, etc.).
\end{itemize}

\subsection{Les composants de la librairie mp.TUI}

\noindent Les composants utilisent des transformations géométriques hiérarchiques\footnote{à l'aide de l'objet jit.anim.node de Max}, qui permet d'obtenir des coordonnées relatives au monde ou à l'objet indépendamment de la position, de l'échelle et de l'orientation du composant de l'interface utilisateur. Cela permet également de créer des groupes de composants, comme on le ferait dans n'importe quel logiciel de CAO. Suivant la nature empirique de la lutherie numérique revendiquée ci-dessus, un "mode édition" est également disponible pour manipuler rapidement à la main la position, l'échelle et l'orientation des composants de l'interface utilisateur (figure \ref{fig:visual_representation:groups_patch} et \ref{fig:visual_representation:groups}).

%-------------------------- Figure : phonétogramme ----------------------------------
\begin{figure}[!htbp]
	\captionsetup{format=plain}%
	\includegraphics[width=\textwidth]{gfx/06_visual_representation/Phonetogramme.png}
	\caption[``Le phonétogramme'', une application réalisée à l'aide de la librairie mp.TUI]{``Le phonétogramme'', une application muséographique conçue pour la Cité des Sciences, dont la GUI est réalisée avec la librairie mp.TUI.}
	\label{fig:visual_representation:phonetogramme}
\end{figure}

\noindent La possibilité de concevoir des objets audiovisuels en Max en étroite relation avec la programmation de l'interaction entre le geste, l'audio et le visuel permet de les intégrer dans des scénarios dynamiques personnalisés : histoires narratives pour des ateliers éducatifs avec des enfants, scénarios réactifs, visualisations personnalisées pour les malvoyants, expositions muséographiques avec chartes graphiques spécifiques, adaptation réactive aux formats d'écran, graphismes expérimentaux pour l'esthétique des performances artistiques live, etc.
%------------------ Figure : mp.TUI : simple slider ---------------------
\begin{figure}[!htbp]
	\makebox[\linewidth][c]{%
		\begin{subfigure}[b]{.5\textwidth}
			\centering
			\includegraphics[width=.95\textwidth]{gfx/06_visual_representation/mpTUI_slider-patcher.png}
			\caption{L'objet Max créant un slider}
		\end{subfigure}%
		\begin{subfigure}[b]{.5\textwidth}
			\centering
			\includegraphics[width=.95\textwidth]{gfx/06_visual_representation/mpTUI_slider-onscreen.png}
			\caption{Rendu du slider dans une fenêtre OpenGL}
		\end{subfigure}%
	}
	\caption{Un simple slider dans la librairie mp.TUI}
\end{figure}

\subsection{Outils pour l'interaction multitouch}

%-------------------------- Figure : mp.TUI overview ------------------------
\begin{figure}[!htbp]
	\includegraphics[width=\textwidth]{gfx/mpTUI/mp-TUI-preview.png}
	\caption{Aperçu de quelques composants graphiques de la librairie mp.TUI}
	\label{fig:visual_representation:mp.TUI}
\end{figure}

% \begin{figure}
% 	\captionsetup{format=plain}%
% 	\centering
% 	\begin{minipage}[t]{0.48\textwidth}
% 		\includegraphics[width=\linewidth]{gfx/mpTUI/mp-TUI-preview.png}
% 		\captionof{figure}{Exemples de composants}	
% 		\label{fig:visual_representation:overview}
% 	\end{minipage}%
% 	\hspace{.02\linewidth}	
% 	\begin{minipage}[t]{0.48\textwidth}
% 	    \includegraphics[width=\linewidth]{gfx/mpTUI/mp-TUI-voronoi.png}
% 		\caption{Modèle de Voronoi}
% 		\label{fig:visual_representation:voronoi}
% 	\end{minipage}
% \end{figure}


\subsection{Groupement d'objets graphique}

\noindent L'objet mp.TUI.groups permet de rattacher différents éléments de GUI à un groupe, comme on le fait dans la plupart des logiciels d'édition vectorielle, afin de gérer leur position, échelle et orientation de manière globale. Cela permet de définir un ensemble de composant dans des coordonnées relative, puis de venir ajuster l'emplacement d'un seul et unique bloc. Il est possible de procéder à l'ajustement de ces coordonnées par de valeurs envoyées explicitement, ou bien par une manipulation directe du groupe, selon le même mode opératoire que pour les objets usuels.\\
\indent Un exemple est présenté sur la figure \ref{fig:visual_representation:groups_patch} et son rendu graphique figure \ref{fig:visual_representation:groups}, associant des curseurs et un texte affichant leurs coordonnées dans une zone précise définie par le canvas.

%-------------------------- Figure : groups ----------------------
\begin{figure}[!htbp]
	\captionsetup{format=plain}%
	\centering
	\begin{minipage}[t]{0.48\textwidth}
		\includegraphics[width=\linewidth]{gfx/06_visual_representation/mpTUI_groups_patcher.png}
		\caption{Patch Max présentant des objets groupés}
		\label{fig:visual_representation:groups_patch}
	\end{minipage}
	\hspace{.02\linewidth}
	\begin{minipage}[t]{0.48\textwidth}
	    \includegraphics[width=\linewidth]{gfx/06_visual_representation/mpTUI_groups.png}
		\caption{Groupement d'objets et édition ``à la main''}
		\label{fig:visual_representation:groups}
	\end{minipage}
\end{figure}
%-------------------------- Figure : groups ----------------------

\subsection{GUI composites}

\noindent Il peut s'avérer nécessaire de coordonner plusieurs éléments d'interaction graphique dans un seul et même ensemble, afin qu'un élément puisse réagir à une interaction sur un autre élément. L'objet mp.TUI.canvas est destiné à cette fin. C'est un objet graphique vide, qui définit simplement une zone ajustable en position, échelle et orientation. Comme tous les autres objets de la librairie mp.TUI, il emet en sortie les \textit{MP-events} qui lui arrivent, ce qui permet de les renvoyer sur les entrées MP d'autres composants, même si ceux-ci ne sont pas directement touchés par les curseurs de position.\\
\indent Un exemple simple de \gls{GUI} composite est présenté sur la figure \ref{fig:visual_representation:canvas}, associant des curseurs et un texte affichant leurs coordonnées dans une zone précise définie par le canvas.


%------------------ Figure : canvas ---------------------
\begin{figure}[!htbp]
	\captionsetup{format=plain}%
	\includegraphics[width=\textwidth]{gfx/06_visual_representation/mpTUI_canvas.pdf}
	\caption[Exemple de GUI composite avec mp.TUI.canvas]{Exemple de GUI composite avec mp.TUI.canvas: les curseurs ne sont pris en compte que dans la zone définie par le canvas. Patch Max en haut, rendu en bas}
	\label{fig:visual_representation:canvas}
\end{figure}
%------------------ Figure : canvas ---------------------

%------------------ Figure : canvas ---------------------
% \begin{figure}[!htbp]
% 	\captionsetup{format=plain}%
% 	\centering
% 	\begin{minipage}[t]{0.38\textwidth}
% 		\includegraphics[width=\linewidth]{gfx/06_visual_representation/mpTUI_composition-canvas.png}
% 		\caption[Exemple de GUI composite avec mp.TUI.canvas]{Exemple de GUI composite avec mp.TUI.canvas}
% 		\label{fig:visual_representation:canvas-patch}
% 	\end{minipage}
% 	\hspace{.01\linewidth}
% 	\begin{minipage}[t]{0.58\textwidth}
% 	  	\includegraphics[width=\linewidth]{gfx/06_visual_representation/mpTUI_canvas_window.png}
% 		\caption[Exemple de GUI composite : rendu graphique]{Exemple de GUI composite : rendu graphique}
% 		\label{fig:visual_representation:canvas-window}
% 	\end{minipage}
% \end{figure}
%------------------ Figure : canvas ---------------------


\subsection{Instanciations dynamiques}

\noindent La gestion de processus musicaux en parallèle créés à la volée nécessite l'instanciation dynamique d'objets, matérialisant de manière tangible ces processus afin de pouvoir les contrôler durant leur existance et les supprimer lorsqu'on le souhaite.
\indent Il est parfois possible de poser des objets physiques sur la surface multitouch pour créer et maintenir en vie de tels processus, sans avoir à maintenir le contact des doigts\footnote{c'est le cas par exemple sur la ReacTable qui utilise des objets cylindriques pour instancier des modules} (cf. fig \ref{fig:visual_representation:objectOnTable}), mais cette solution n'est pas toujours réalisable\footnote{Les surfaces multitouch basés sur une techologie capacitive ne sont pas sensible à tout type d'objet physique, contrairement par exemple, au écran basés sur une technologie infra-rouge.} ni souhaitable (pour des raisons d'encombrement de l'espace de la tablette par exemple).\\
\indent Les objets graphiques de la librairie mp.TUI peuvent facilement être créés à la volée pour répondre à ce besoin, en les insérant dans des objets MP, tel que présenté dans le chapitre précédant. Ceci permet de concevoir relativement simplement un scénario d'interaction graphique tel que la possibilité de définir des zones ``réservoir d'objets'' où il est possible de venir piocher des éléments créés dynamiquement, et représentant des processus complexes incluant leur propre mode de destruction (exemple figure \ref{fig:visual_representation:dynamicInstanciation}).

%-------------------------- Figure : groups ----------------------
\begin{figure}[!htbp]
	\captionsetup{format=plain}%
	\centering
	\begin{minipage}[t]{0.48\textwidth}
		\includegraphics[width=\linewidth]{gfx/dummy.pdf}
		\caption{TODO : Maintien d'un contrôle par un objet physique}
		\label{fig:visual_representation:objectOnTable}
	\end{minipage}
	\hspace{.02\linewidth}
	\begin{minipage}[t]{0.48\textwidth}
	    \includegraphics[width=\linewidth]{gfx/dummy.pdf}
		\caption{TODO : Maintien d'un contrôle par un objet virtuel, supprimable par un double clic}
		\label{fig:visual_representation:dynamicInstanciation}
	\end{minipage}
\end{figure}
%-------------------------- Figure : groups ----------------------

\subsection{Performances}

\noindent La bibliothèque mp.TUI est entièrement développée avec des objets natifs de la distribution Max. Cette approche, bien que plus coûteuse en charge \gls{CPU} que des objets compilés, a l'avantage de permettre à tout utilisateur de Max de modifier facilement les composants et de les adapter à ses besoins. De plus, les composants mp.TUI s'appuient essentiellement sur OpenGL, de sorte que la majeure partie de la charge de calcul est laissée au \gls{GPU}, ce qui en fait une solution plus réactive que la solution envisagée dans la libririe MMF. L'interaction tangible avec les objets de la \gls{GUI} se fait à l'aide du moteur physique Bullet-Physics\footnote{\url{http://bulletphysics.org/}} intégré dans Max. Bien que cela puisse être plus coûteux pour certaines formes simples, cela nous permet de concevoir des composants \gls{GUI} de n'importe quelle forme et orientation, comme des \textit{sliders} courbes ou des formes creuses, et de les animer potentiellement, comme dans l'exemple des "balles rebondissantes" où plusieurs curseurs 2D peuvent être déplacés et lancés dans une zone délimitées.

\subsection{Travaux futurs}

\noindent Des optimisations sont très probablement possibles pour améliorer les performances de mp.TUI, notamment au niveau du calcul de l'intersection entre position des pointeurs et objets graphiques, en traitant les cas simples séparément plutôt que dans le modèle générique du moteur \textit{Bullet-Physics}.
\noindent La librairie mp.TUI permet d'introduire la notion de programmation gestuelle de type multitouch dans l'environnement Max, en bénéficiant des passerelles dont celle-ci peut bénéficier avec les possibilités audio, graphiques et de mapping disponibles dans Max. De nombreuses stratégies de contrôle basées sur ce paradigme restent à implémenter, explorer et inventer, notamment sur la gestion d'événements conjoints définissant des primitives gestuelles, tel que présenté dans \cite{oney_implementing_2019}, ou la détection d'objets tagués tels que présenté dans \cite{yu_tuic_2011}.


\section{Conclusion}


%%%%%%%%%%%%%%%%%%%%%%%%%%%%%%%%%%%%%%%%%
\section*{miscellanées (temporaire à supprimer)}
citations :

The  keyboards  were  always  there...  for  some  reason  or  other  it  looks  good  if  you’re playing a keyboard. People understand then you’re making music.” Robert Moog in Trevor Pinch, “Why You Go to a Piano Store to Buy a Synthesizer: Path Dependence and the Social Construction of Technology,” in Path Dependence and Creation


NIME 2019 : ``From Mondrian to Modular Synth: Rendering NIME using Generative Adversarial Networks''

Chamagne

\begin{quote}
Visual language is one of the oldest forms of knowledge representation and predates conventional written language by almost 25,000 years
\end{quote}
\cite{tufte_visual_2001}

\cite{moody_physics_2009}


\begin{quotation}
If you ask the man on the stree "What's a synthesizer?" He will reply "A synthesizer is a keyboard instrument"... If you go in a retail store ad say I want to see some electronic instruments, they'll send you to the keyboard department, because a synthesizer is a keyboard instrument by default. — Don Buchla, synthesizer pioneer (interview)
\end{quotation}
\cite{pinch_why_2001}

Roel Vertegaal, Tamas Ungvary et Michael Kieslinger utilisait ainsi le terme de ``\textit{musician's cockpit}'' dans un article de 1996, une métaphore qui laisse imaginer l'instrument comme un véhicule que l'on pilote. % INCLUDE: visual_representatio
%\part{Partie III} 
% !TEX root = ../thesis-example.tex
%
\chapter{Notation}
\label{ch:notation}

\cleanchapterquote{(...)building a musical instrument becomes indistinguishable from designing a music-theoretical framework; the musical instrument is a theory of music (...)}{Thor Magnusson}{Sonic Writing (2019)}

\cleanchapterquote{(...) comment je me suis adapté à ce problème de la radio dans l’environnement : comme les peuplades primitives se sont adaptées aux animaux qui les effraient et qui constituaient probablement, comme tu dis, des intrusions. Ils ont dessiné des images d’eux sur les murs de leurs cavernes ; et donc moi, j’ai simplement composé une pièce avec des radios. À présent, à chaque fois que j’entends des radios – même une seule, pas simplement douze à la fois, comme tu as dû en entendre à la plage, au moins – je pense : “Tiens, ils jouent ma pièce”}{John Cage}{John Cage / Morton Feldman. \\ Radio Happenings 1966 \cite{cage_radio_2015}}


% This article presents “John”, an open-source software designed to help collective free improvisation. It provides generated screen-scores running on distributed, reactive web-browsers. The musicians can then concurrently edit the scores in their own browser. John is used by ONE, a septet playing improvised electro-acoustic music with digital musical instruments (DMI). One of the original features of John is that its design takes care of leaving the
% musician's attention as free as possible.
% Firstly, a quick review of the context of screen-based
% scores will help situate this research in the history of contemporary music notation. Then I will trace back how improvisation sessions led to John's particular “notational perspective”. A brief description of the software will precede a discussion about the various aspects guiding its design.


%%%%%%%%%%%%%%%%%%%%%%%%%%%%%%%%%%%%%%%%%
\section{Notes sur la partition}

\subsection{Du geste vers le son et vice versa}

Une partition est généralement pensée comme un document servant aux musiciens à interpréter une œuvre musicale
Description du geste\\
description du son\\
prescription du geste\\
prescription du son

Dans les partitions classiques de la notation musicale occidentale, la partition peut être \textit{descriptive}, par exemple en donnant la hauteur et la durée de notes, ou \textit{prescriptive} en indiquant par exemple quel geste 

Dans le cas de l'utilisation de \gls{DMI} ayant la capacité à produire des sons sans qu'il y ait nécessairement ``geste d'excitation'' (selon la terminologie de Cadoz) préalable, la partition peut servir à noter des \textit{gestes accompagnateurs} 

Dans la performance audio visuelle FIB\_R, le cheminement vers la partition a été d'abord de construite un instrument, puis de jouer de cet instrument jusqu'à arriver à des formes intéressantes. Mais à un certain moment, ce que l'on imagina ne correspondait plus à quelque chose de jouable directement, car la précision rythmique désirée dépassait ce qu'il était possible d'obtenir avec les interfaces et l'algorithme de jeu actuel.
Par ailleurs, penser le geste indépendemment de sa capacité à être capté par la machine laissait la possibilité d'utiliser des gestes plus libres et plus en phase, musicalement, avec ce que nous souhaitions.\\
Il y a donc eu un renversement entre des gestes qui servaient au début à produire des sons, pour arriver à une composition écrite et séquencée, dont nous jouons les gestes en playback sur les sons.
La machine laisse la possibilité de créer des mouvements enregistrés dont la précision et la vitesse dépasse celle possible par un être humain. Musicalement, il est intéressant de se tenir à la limite de ces possibilités du corps, car cela créé une tension par (empathie?) ou (paréidolie?) due au fait que chacun puisse se projeter dans ces mouvements.


%-------------------------- Figure : FIB_R Lui+Elle ----------------------------------
\begin{figure}[!htbp]
	\includegraphics[width=\textwidth]{gfx/notation/FIBR-Chat2-Elle.pdf}
	\caption{Partition gestuelle pour la partie II de FIB\_R, Elle}
	\label{fig:notation:FIBR-chat2-Elle}
\end{figure}

\begin{figure}[!htbp]
	\includegraphics[width=\textwidth]{gfx/notation/FIBR-Chat2-Lui.pdf}
	\caption{Partition gestuelle pour la partie II de FIB\_R, Lui}
	\label{fig:notation:FIBR-chat2-Lui}
\end{figure}


%--------------------------------------------------------------------
\subsection{Des partitions traditionnelles aux partitions graphiques}

\noindent La partition est généralement considérée comme un outil permettant au compositeur de créer une œuvre musicale pour un interprète. Elle décrit le résultat sonore attendu et prescrit les gestes à effectuer\footnote{Eric Maestri propose les termes “phonographique” et “ergographique” pour décrire ceux deux aspects.}. Elle sert ainsi de moyen mnémonique pour garder une trace de ce qui est indépendant du contexte de la performance\footnote{En considérant ici que l'interprétation fait partie du contextuel.} et qui est souvent assimilé à l'œuvre elle-même dans la tradition musicale occidentale.\\
\indent La partition possèdent cependant beaucoup d'autres fonctions. Elle permet notamment de transposer le temps musical dans un espace visuel, permettant au compositeur d'agencer des éléments musicaux "hors du temps" afin de produire des pièces qui ne pourraient être conçues sans ce support visuel\footnote{Un exemple notoire est le rondeau "Ma fin est mon commencement" (14e siècle) de Machaut, dans lequel les deux voix sont rétrogrades l'une à l'autre.}.\\
\indent Si le système de notation occidental inventé par Guido d'Arezzo au XIe siècle n'a cessé d'évoluer, s'enrichissant de nouveaux symboles et de nouvelles techniques jusqu'au début du XXe siècle, les révolutions technologiques et culturelles qui ont suivi ont bouleversé à la fois les moyens de production et le champ de l'expression musicale, désormais étendu au bruit et au spectre sonore entier.\\
\indent On peut noter le développement de ce qu'on appelle les "partitions graphiques" \footnote{... c'est-à-dire l'utilisation de signes graphiques autres que les symboles habituels de la notation conventionnelle des notes sur une portée.} au milieu du 20ème siècle, qui reflète cette évolution musicale pour laquelle la notation traditionnelle est insuffisante.  Pour des raisons qui peuvent sembler opposées, la partition graphique a contribué à repousser à la fois les limites de ce qu'il était possible de "fixer" dans une composition - en la spécifiant entièrement sur un système de synthèse, et les limites de ce qu'il était concevable de varier - la partie confiée à l'interprétation par l'interprète.  Les partitions de Mycene Alpha de Iannis Xenakis et de Décembre 1952 de Earle Brown soulignent ces deux directions (cf. Figure \ref{fig:notation:brown-xenakis}).

%-------------------------- Figure : Brown-Xenakis ----------------------------------
\begin{figure}[!htbp]
	\includegraphics[width=\textwidth]{gfx/notation/Brown-Xenakis-Paysage.png}
	\caption{Extraits des partitions de \textit{December 1952} de Earle Brown (à gauche) et \textit{Mycène Alpha} de Iannis Xénakis (à droite)}
	\label{fig:notation:brown-xenakis}
\end{figure}

Cette opposition apparente entre une œuvre totalement figée et une œuvre totalement soumise à la créativité des interprètes semble plutôt le résultat d'approches complémentaires visant à explorer les nouveaux domaines sonores et musicaux, tant dans leurs manifestations que dans leurs potentialités, réifiées ou fantasmées.\\
\indent Dans ce continuum de possibilités entre œuvre fixe et improvisation libre, que Richard Dudas appelle \iquote{comprovisation} dans \cite{dudas_comprovisation:_2010}, différentes ``perspectives notationnelles''\footnote{j'emprunte ici cette expression à Baghwati \cite{bhagwati_notational_2013}} peuvent être envisagées. Les différentes finalités de la représentation musicale jusqu'alors intégrées dans la partition traditionnelle gagnent en indépendance et prennent une importance variable, s'adaptant aux contextes de l'œuvre musicale et de l'interprétation. La partition définit le terrain de jeu, qui n'est pas nécessairement linéaire et qui, grâce à la possibilité de produire des images animées en temps réel, n'est plus nécessairement fixe.

%----------------------------------------------------------------------------------------------------------
\subsection{Œuvres ouvertes}

\noindent Une partition peut contenir à la fois une description/prescription linéaire qui décrit le déroulement de la partie de manière exhaustive, mais les œuvres ``ouvertes''\footnote{Pour une présentation plus générale du concept d'œuvre ouverte, voir \cite{eco_oeuvre_2015}}, qui fournissent aux musiciens des ``règles du jeu'' plutôt que le résultat du jeu de ces règles, présentent ainsi des processus génératifs dont les résultats varient à chaque performance. Jean-Louis Giavitto, dans \cite{giavitto_du_2014}, nomme ces deux types de partitions ``intensionelles'' et ``extensionnelles'' en référence à la formulation d'ensembles en mathématique, soit extensivement i.e. en en donnant la liste, soit intensivement i.e. en les caractérisant par une propriété. 

%----------------------------------------------------------------------------------------------------------
\subsection{Screen scores}

\noindent La disponibilité croissante des appareils numériques a conduit au développement d'un certain nombre d'applications destinées à la création de partitions à l'écran. Comme le note Lindsay Vickery dans \cite{vickery_limitations_2014} :

\begin{quotation}
Ces développements suggèrent une tendance, en particulier chez les jeunes compositeurs dont la pratique s'est développée exclusivement sur ordinateur, de passer logiquement à l'étape de présenter des matériaux notationnels à l'écran.
\end{quotation}

Cat Hope résume les principales caractéristiques offertes par ce nouveau média dans les termes suivants \cite{hope_screen_2011}: \textit{les capacités de défilement, de permutation, de transformation, de génération et de mise en réseau du support numérique}.\\
\indent L'utilisation de l'infographie pour la représentation musicale semble être un médium de choix pour enrichir les possibilités d'écriture de partitions graphiques. En particulier, la fluidité d'adaptation du support virtuel permet d'envisager de multiples "vues" d'une même partition selon les contextes auxquels elle est destinée. Ainsi, la composition, la performance ou l'analyse d'une œuvre musicale ne nécessitent pas nécessairement les mêmes représentations. En termes d'interprétation musicale, on peut ajouter une distinction entre l'interprétation d'une partition par un humain et une machine, ces deux types d'"interprètes" ayant des capacités relativement différentes.\\
\indent De la même manière que les technologies numériques ont atomisé l'instrument de musique en découplant ses différentes composantes (contrôleur gestuel, cartographie, synthèse, etc. devenant modulaire), elles ont également atomisé la partition en ses différentes fonctions, de support pour la composition, la performance ou l'analyse. Il est alors nécessaire de préciser quel cas d'utilisation est en jeu et Cat Hope définit à cet effet le terme ``partition sur écran'' (\textit{screen-score}) \cite{hope_screen_2011} comme le medium présenté aux musiciens pour une performance:

\iquote{Les screen-scores sont des compositions musicales écrites, conçues pour être interprétées ; elles ne doivent pas être confondues avec des représentations visuelles de la musique ou l'interprétation musicale des arts visuels.}

\indent Le concept de ``partition sur écran'' a été étudié en profondeur par plusieurs auteurs, compositeurs et musicologues (voir Winkler \cite{winkler_real-time_2004}, Clay \cite{adams_inventing_2008} ou Lee \cite{lee_real-time_2012}), qui ont discuté des avantages et des inconvénients de l'utilisation des technologies numériques pour la représentation musicale, tant dans ses aspects techniques que dans ses conséquences musicologiques. Lindsay Vickery propose une revue très détaillée dans \cite{vickery_limitations_2014}, des latences critiques permettant à un instrumentiste de lire en temps réel le matériel musical affiché et donne des conseils sur ce à quoi le compositeur doit faire attention lorsqu'il compose avec ce support.\\
\indent Ces études offrent des descriptions pertinentes et précieuses pour le compositeur qui souhaite utiliser des partition sur écran. Cependant, il semble qu'elles puissent être complétées par une approche de la partition différente de celles envisagées dans la plupart de la littérature sur le sujet, où le point de vue est souvent celui du compositeur. La conception d'un système de partition à écran est donc polarisée par l'importance centrale de la partition, elle-même considérée comme une condition préalable à l'exécution musicale, situation qui reflète également une forte tradition de la musique classique occidentale \footnote{Une exception notable est la contribution de Georg Hajdu \cite{hajdu_disposable_2016} qui propose le concept de "musique jetable" pour qualifier les formes musicales \iquote{qui reposent dans une moindre mesure sur des partitions entièrement notées, telles que la "comprovisation" ou la performance sur laptop}\footnote{\iquote{that rely on a lesser degree on fully notated scores, such as "comprovisation” or laptop performance}}. Cependant, même lorsqu'elle est "\textit{jetable}", la partition occupe ici encore une position préalable à la performance et sur laquelle l'attention reste focalisée, à la différance de l'approche proposée avec ``John''.}.\\
\indent Dans le cas des performances de ONE (Figure \ref{fig:notation:one-fullband}), qui sont basées sur une pratique d'improvisation libre sans composition préalable, la focale est déplacée du côté de l'instrumentiste. L'élément central n'est pas la partition mais l'écoute et la compréhension du son et des autres musiciens. La partition (s'il est encore possible de l'appeler ainsi) émerge souvent après les séances d'improvisation et sa présence ne doit pas se faire au détriment de l'attention mutuelle. Dans cette perspective, il est possible d'envisager que le musicien adapte lui-même la représentation musicale à ses propres besoins, en fonction des parties qu'il doit jouer, de ses préférences personnelles, des différents mouvements de la partition, etc.\\
\indent Dans le cas particulier où les instruments sont numériques et programmables, l'utilisation d'un système de partition en réseau offre également la possibilité de déléguer certains paramètres de l'instrument à un contrôle externe pris en charge par la partition. Dans une situation d'improvisation, la négociation entre ce contrôle automatisé et le choix du musicien implique une médiation dont je parlerai dans la section \ref{sec:notation:score_for_humans_and_machines}.

%----------------------------------------------------------------------------------------------------------
\subsection{Improvisation dans l'ensemble ONE – naissance d'une notation}

\noindent Les sept musiciens de l'ensemble ONE (cf. figure \ref{fig:notation:one-fullband}) sont tous profondément impliqués dans le domaine de l'informatique musicale avec des spécialités diverses dans les domaines de la pratique instrumentale, de la composition, de la facture instrumentale, de la recherche en sciences musicales et de l'éducation. Nous pratiquons tous des instruments de musique numériques dont nous avons conçu le logiciel et parfois aussi le hardware, dans une certaine mesure. A l'origine de notre collaboration, il n'y avait pas d'autre projet que celui de tenter l'expérience de jouer une ``musique de sons'' (todo : ref vers cette expression) avec cet instrumentarium numérique hétéroclite, sans grille, sans théorie musicale, sans accord préalable sur la forme et le contenu.\\

%-------------------------- Figure : ONE ----------------------------------
\begin{figure}[!htbp]
	\includegraphics[width=\textwidth]{gfx/notation/ONE-fullBand.png}
	\caption{The members of ONE with their digital musical instruments.}
	\label{fig:notation:one-fullband}
\end{figure}

\indent Plusieurs séances d'improvisation ont été l'opportunité de découvrir nos sons, nos styles de jeu, notre vocabulaire musical. Ces moments de répétition ont été avant tout l'occasion de performances anarchiques, guidées uniquement par le fil de notre écoute, de confrontation, de mélange, de collision, de superposition d'objets et d'espaces sonores, ainsi que de moments de discussion et de réglages de nos dispositifs de jeu.\\
\indent Ces sessions ont également fait l'objet d'exercices d'improvisation classiques : recherche de fusion timbrale et de contrepoints, passages fugués entre musiciens, accompagnement d'un soliste, travail sur les nuances pianissimo, ou improvisation ``dans le style'' d'une pièce connue. Finalement, les enregistrements audio nous ont permis de ré-écouter les improvisations parfois longues et ininterrompues pour en extraire des idées intéressantes.\\
\indent La question de la structure globale d'un concert en mouvements musicaux est apparue durant la préparation de la première performance en public de ONE. L'absence d'une partition structurant la durée du concert nous a conduit à suivre un scénario narratif inspiré d'un roman de Jules Verne. Ainsi, le concert consistait en une série de chapitres, simplement identifiés par des intertitres tenant lieu de paysages sonores exotiques et imaginaires à explorer.\\
\indent Peu à peu, ces expériences ont donné lieu à l'émergence d'un vocabulaire musical plus atomique, représentant des atmosphères et des mouvements définis collectivement, que nous avons appelés ``karmas''\footnote{La relation avec ce concept indien est lointaine, mais elle comporte un sens séduisant qui fait écho à la façon dont nous les voyons dans la performance : l'ensemble des actions représentées par le karma influence l'avenir de l'individu. De même que l'interprétation musicale d'un \textit{karma} (tel que nous le définissons) est soumise aux actions des musiciens et tout accident, la bifurcation par rapport à la partition prévaudra sur l'évolution musicale plus que la partition elle-même.}. Les différents moments de jeu et de discussion nous ont amenés au développement d'autres objets conceptuels qui ont été en partie réalisés sous la forme d'un logiciel surnommé ``John, le semi-conducteur''\footnote{en anglais ``John, the semi-conductor'', ``conductor'' signifiant chef d'orchestre, ce subtil jeu de mot se perdant malheureusement dans sa traduction française...}.\\
\indent L'origine du développement de John est ainsi liée au désir de trouver un moyen de structurer le temps musical en différents mouvements dans la perspective de concerts librement improvisés d'une durée assez longue. Une autre motivation réside dans la capacité à varier les improvisations afin de ne pas toujours répéter les mêmes textures et structures formelles telles que des séquences de cycles ascendants-descendants.\\
\indent De plus, nous cherchions des moyens de stimuler l'exploration de combinaisons et d'idées musicales inhabituelles qui nous poussent hors de notre zone de confort. La proposition de diviser mathématiquement le temps en séquences pour permettre à tous les ensembles possibles (solo, duo, trio, ... jusqu'au tutti), a été la première impulsion pour le développement d'un générateur de partition capable de produire automatiquement de telles distributions.\\
\indent Comme les opinions divergeaient au sein du groupe sur l'équilibre entre règles et absence de règles, un principe clé a permis de trouver un terrain d'entente : John est un ``semi-conducteur'', un ``sous-chef d'orchestre'. Cela signifie que les partitions créées avec John ne sont qu'une proposition que chaque membre du groupe est libre de suivre ou non, selon le contexte musical qui ne prend véritablement forme qu'au moment même de la performance. L'écoute reste donc la règle essentielle du jeu, l'emportant sur un suivi aveugle de la partition. En particulier, sont laissés à l'appréciation de chaque musicien l'articulation entre les différentes parties de la partition, qu'elles soient tuilées ou disjointes, ou encore la décision de jouer quand il n'est pas censé le faire (ou inversement), etc.\\
\indent Ce principe a pour conséquence directe une épuration dans le design visuel de la partition, dont le but est de permettre à chaque musicien de se situer en un coup d'œil dans la partition, sans monopoliser son attention au détriment des autres musiciens et du son. Le but est donc très différent de celui poursuivi dans d'autres systèmes de notation musicale sur écran, comme ceux explorés dans des travaux impliquant du déchiffrage en direct de partition dynamique \cite{freeman_extreme_2008}.\\
\indent Essentiellement, John permet la gestion collective du temps, que ce soit lors des répétitions, de la composition ou des performances, en fournissant un support de représentation partagé. Une brève description du logiciel pour en saisir les grandes lignes précédera une discussion sur les différents aspects liés à cette gestion de groupe.

%%%%%%%%%%%%%%%%%%%%%%%%%%%%%%%%%%%%%%%%%
\section{A propos de John}

\noindent John s'appuie sur une architecture client / serveur, dans laquelle chaque musicien visualise une interface client dans un navigateur web, sur laquelle il peut agir. Cette interface se compose de deux parties, un générateur de partition d'une part et une visualisation interactive de la partition d'autre part.
%-------------------------- Figure : John client interface ----------------------------------
\begin{figure}[!htbp]
	\includegraphics[width=\textwidth]{gfx/notation/John-snapshot.png}
	\caption{Snapshot of John's client interface running in a web-browser.}
	\label{fig:notation:john-snapshot}
\end{figure}

%----------------------------------------------------------------------------------------------------------
\subsection{Générateur de partitions}

\noindent Le générateur de partitions permet de créer très rapidement des propositions musicales en ne spécifiant que des contraintes globales :
\begin{itemize}[noitemsep]
\item la durée globale de la partition;
\item le nombre minimal et maximal de joueurs;
\item la durée minimale et maximale des blocs;
\item une liste de \textit{karmas} identifiant une ambiance musicale particulière, selon un vocabulaire défini en commun durant les séances d'improvisation;
\item une liste de nuances de \textit{pianississimo} à \textit{fortississimo}.
\end{itemize}

\noindent Une fois ces contraintes spécifiées, le générateur de partition produit une proposition aléatoire respectant ces conditions, composée d'une séquence de blocs temporels associant un karma et une nuance. Cette proposition peut ensuite être ajustée dans l'interface d'édition / visualisation.

%----------------------------------------------------------------------------------------------------------
\subsection{Visualisation interactive}

\noindent Cette interface représente des blocs sur disposés sur une abscisse chronologique. Elle se compose d'une \textit{vue globale} réduite d'une part, offrant une vue d'ensemble partagée de la partition dans son intégralité, et d'une \textit{vue locale} zoomable, située au-dessus de la \textit{vue globale}. Sur la \textit{vue globale} se trouve une tête de lecture commune et synchrone à tous les clients (en rouge sur la Figure \ref{fig:notation:john-snapshot}), ainsi qu'un empant temporel (en bleu sur la Figure \ref{fig:notation:john-snapshot}) définissant la durée affichée sur la \textit{vue locale}. Cet empant est défini individuellement par chaque musicien sur son client et varie généralement d'une dizaine de secondes à quelques minutes selon la granularité temporelle de la partition et la préférence de chacun.\\
\indent Tous les paramètres de contrôles sont accessibles dans l'ensemble des clients, permettant à chacun d'éditer la partition : générer une nouvelle instance, déplacer et modifier la durée des blocs et leur contenu (\textit{karma} et \textit{nuance}), démarrer la lecture, modifier la vitesse de lecture, déplacer la tête de lecture pour démarrer à un moment donné de la partition. Ces changements sont immédiatement appliqués à l'ensemble des autres clients de John.\\
\indent L'utilisateur peut également définir des paramètres locaux qui n'affecteront que son interface client : la visibilité des différentes pistes, la durée de sa \textit{vue locale} et la synchronisation (ou non) de sa vue locale au curseur de lecture, à l'aide du bouton ``link''.

%----------------------------------------------------------------------------------------------------------
\subsection{Implementation}

\noindent Après une première version développée en Max\footnote{\url{https://cycling74.com}}, l'application a été portée en HTML5 réactif à l'aide de l'environnement Meteor\footnote{\url{http://meteor.com}}. Ceci permet l'édition collective sur toutes les plates-formes (y compris les plates-formes mobiles) connectées à un réseau local, via un simple navigateur web. La visualisation a été réalisée à l'aide de la bibliothèque D3.js\footnote{\url{https://d3js.org}}.\\
\indent Les partitions sont sauvegardées au format \gls{JSON} sous la forme d'une liste d'événements avec un identifiant unique, un index de piste, une temps de début, une durée et un certain nombre de propriétés telles que le karma et les nuances. Pendant la lecture, le temps et les événements de la paritition sont envoyés en tant que messages MP (cf. section \ref{sec:algorithms:MP}) sur le réseau.

%%%%%%%%%%%%%%%%%%%%%%%%%%%%%%%%%%%%%%%%%
\section{John en pratique}
%----------------------------------------------------------------------------------------------------------
\subsection{Composition générative}

\noindent Le générateur de partitions nous a fait gagner beaucoup de temps pendant les répétitions, en nous offrant immédiatement une structure musicale possible. Aussi arbitraire que soit cette structure, sa fonction principale est de stimuler la performance musicale via sa prescription la plus minimale : quand jouer (ou ne pas jouer). Ainsi, les propositions sont souvent testées telles quelles avant d'être ajustées collectivement en fonction de ce que les membres du groupe trouvent intéressant ou non. Il est alors possible de faire évoluer cette structure musicale, avec apparemment plus d'efficacité que si l'on partait de rien.

%----------------------------------------------------------------------------------------------------------
\subsection{Distribution de la participation}

\noindent Le fait que John propose explicitement une distribution du temps de jeu entre chaque musicien a conduit à des configuration d'ensemble que nous n'aurions pas nécessairement essayées, en particulier les ensembles réduits (solo et duo), chacun d'entre nous ayant tendance à jouer trop souvent pour laisser s'installer ces configurations minimales.\\
\indent De plus, avoir des moments de pause explicites permet de mieux anticiper ses entrées. En effet, les \glspl{DMI} ont souvent une dimension ``méta-instrumentale''\footnote{C'est-à-dire qu'il peut être totalement reconfiguré pendant la représentation pour offrir un tout autre ensemble de sons, de processus et de modes de jeu.}, et plus généralement ils exposent un grand nombre de paramètres. Ces moments de pause planifiés permettent de mieux gérer le temps dont dispose chaque instrumentiste pour configurer de telles reconfigurations de paramètres. (todo: expliquer un peu mieux la distinction entre paramètres accessibles immédiatement et paramètres moins accessibles)

%----------------------------------------------------------------------------------------------------------
\subsection{Synchronisation}

\noindent Dans une situation d'improvisation libre, la synchronisation entre les musiciens est entravée par l'absence de règles idiomatiques. En particulier, l'absence de pulsation ou de mesure rend cette synchronisation plus difficile encore quand le nombre de musicien augmente et prive souvent l'improvisation libre de transitions franches dans la pratique d'ensemble.

\indent Le chef d'orchestre, lorsqu'il y en a un, fournit des repères temporels précis, par la battue et d'éventuelles indications pour le jeu. Outre les questions éthiques posées par le rôle d'un leader d'un groupe d'improvisation, et analysées par Clément Canonne dans \cite{canonne_improvisation_2012}, confier la direction d'une improvisation à une personne\footnote{comme c'est le cas dans le \textit{Soundpainting} de Walter Thomson ou dans une composition comme ``Cobra'' de John Zorn} reste limité par le fait qu'elle ne peut agir que dans le présent, et que cela exige une attention quasi-permanente des musiciens envers le chef, au détriment de celle qu'ils peuvent porter à leurs pairs. A cet égard, la représentation offerte par John condense d'une certaine manière la partition et le chef d'orchestre dans un seul et même support visuel. Cette partition animée offre en effet des repères visuels qui indiquent la simultanéité de plusieurs événements musicaux, et son défilement sous la tête de lecture permet une synchronisation précise entre les musiciens lors des transitions.\\

%----------------------------------------------------------------------------------------------------------
\subsection{Support visuel pour des repères musicaux}

\noindent Malgré la disponibilité d'outils d'analyse\footnote{Tels que E-Analysis \cite{couprie_eanalysis:_2016} ou l'Acousmographe du \gls{GRM} \cite{favreau_lacousmographe_2010}.} et l'existence d'un certain vocabulaire pour décrire les objets sonores et musicaux dans la musique électroacoustique\footnote{en particulier les ``objets sonores'' de Pierre Schaeffer \cite{schaeffer_traite_1966}, les ``images-son'' de François Bayle \cite{bayle_musique_1993} ou encore les \gls{UST} du \gls{MIM} \cite{delalande_les_1996}.}, il n'existe aucune norme de notation prescriptive pour les \glspl{DMI}. L'absence d'un vocabulaire unanime, la singularité des instruments et la formidable palette sonore qu'ils offrent, ne facilitent pas l'exercice consistant à identifier et discuter ce qui vient d'être joué lors d'une longue séance d'improvisation (manquant de pouvoir parler ici de ``répétition''). Une partition minimale telle que celle proposée par John facilite cette identification et permet de retravailler des moments précis après une longue performance. La réduction que la notation symbolique opère sur le résultat sonore complexe d'une performance permet à chacun de se retrouver rapidement dans l'espace temporel d'une improvisation, plus rapidement du moins que si l'on devait se référer à l'enregistrement sonore.

%----------------------------------------------------------------------------------------------------------
\subsection{Une écologie de l'attention}

\noindent L'improvisation libre électroacoustique requiert une attention considérable des musiciens envers les autres musiciens, leur instrument et, de toute évidence, au son. À cet égard, les \gls{DMI} présentent souvent l'inconvénient supplémentaire, par rapport aux instruments acoustiques, de capter une partie de l'attention visuelle en raison de la présence fréquente d'un écran, de nombreux paramètres d'interaction et d'une interface parfois dépourvue de retours ou de repères tactiles qui permettraient d'y accéder sans avoir besoin de les regarder. De plus, les musiciens numériques préparent souvent leur instrumentarium juste avant la représentation\footnote{ce que Thor Magnusson et Kris Kiefer nomme ``pre-grammation'' dans \cite{kiefer_live_2019}} avec un ensemble choisi d'éléments musicaux \textit{ad hoc} (lorsqu'ils ne le codent pas en direct, comme c'est le cas dans le live-coding), ce qui complique encore la connaissance ``kinesthésique'' de l'ergonomie de l'instrument, sans aucune aide visuelle.\\
\indent La conception de ``John'' a été ainsi motivée par une économie de la charge cognitive des musiciens. Pouvoir en partie personnaliser son interface de visualisation ne signifie donc pas y ajouter davantage de données visuelles, mais plutôt n'afficher que ce qui est nécessaire, au profit de l'attention mutuelle.

%----------------------------------------------------------------------------------------------------------
\subsection{Partitions pour humains \emph{et} machines}
\label{sec:notation:score_for_humans_and_machines}

\noindent Pendant la lecture de la partition, le serveur envoie des données aux clients lorsque des événements commencent ou se terminent (cf. figure TODO). Ces informations peuvent être utilisées par l'instrument du musicien (si toutefois son \gls{DMI} est connecté au réseau). Mais, comme John n'est qu'un ``semi-conducteur'', ses messages peuvent tout aussi bien être soumis à l'approbation du musicien/client pour permettre une certaine flexibilité dans la façon dont le musicien adhère à la partition.\\
\indent Ainsi, il est possible d'imaginer qu'un \textit{karma} spécifique rappelle un pré-réglage correspondant dans l'instrument du musicien, correspondant à l'esprit de ce \textit{karma}. Mais si le musicien est encore en train de jouer le \textit{karma} précédent, il/elle ne voudra probablement pas que cette notification change automatiquement sa configuration avant d'avoir terminé la phrase musicale en cours. Cette ``évaluation paresseuse''\footnote{J'emprunte ici ce terme utilisée dans le domaine de la programmation récursive, qui consiste à évaluer une expression uniquement quand le résultat de cette expression devient nécessaire. Dans notre cas toutefois, c'est possiblement le musicien qui décide de cette évaluation.} rend l'utilisation de John un peu différente de celle des séquenceurs traditionnels.

%----------------------------------------------------------------------------------------------------------
\subsection{Montrer la partition ?}

\noindent Rendre lisible les interactions entre les musiciens dans les performances d'improvisation peut contribuer à l'appréciation globale de la performance par le public. Pourtant, avec les \glspl{DMI}, le découplage spatial et énergétique entre les gestes de l'instrumentiste et la localisation de l'énergie sonore (sur un haut-parleur possiblement distant) brouille cette lecture. Les systèmes de partition sur écran offre la possibilité de partager l'affichage de la partition avec le public plus facilement que ne le permettent les partitions imprimées et peuvent ainsi aider à cette lisibilité avec le risque, cependant, qu'elle ``entrave les aspects performatifs dramatiques de l'œuvre'' parmi d'autres raisons suggérées par Cat Hope dans \cite{hope_screen_2011}.\\
\indent Bien que la partition de John n'ait jamais été montrée directement au public pour cette raison particulière, elle a été utilisée pour contrôler des effets vidéo et de lumières\footnote{Il s'agissait par exemple d'éclairer les musiciens censés jouer, de modifier la teinte de la lumière en fonction des karmas, de projeter des ondes sonores agrégées comme traces de la partition, de synchroniser des vidéos, etc.}, à la fois des raisons scénographiques et pour aider l'écoute à la compréhension de la musique.

%%%%%%%%%%%%%%%%%%%%%%%%%%%%%%%%%%%%%%%%%
\section{Perspectives}

\noindent Les membres de ONE ont reconnu que John aidait le processus créatif. Cependant, il reste des questions ouvertes comme la synchronisation collective sur les passages rythmiques. En particulier, anticiper un processus dynamique n'est pas une tâche triviale et nécessiterait probablement des outils spécifiques à cette fin, telles que les animations proposées par Ryan Ross Smith dans \cite{smith_atomic_2015}.\\
\indent Le concept de vues \textit{locale} et \textit{globale} pourrait probablement être généralisé à d'autres paramètres partageables. Par exemple, pouvoir démarrer une lecture locale pour s'entraîner ou préparer son instrument par soi-même. De même, il serait utile de travailler sur une autre partition que celle chargée sur les autres clients. Cette désynchronisation soulève cependant des problèmes de conflits de versionnage, dont la prise en compte dépasse pour l'instant les possibilités offertes par John.\\
\indent Le portage de John sur une technologie web a été en partie motivé par la possibilité de futurs concerts impliquant un grand nombre de musiciens et dans lesquel chaque musicien pourrait voir sa partie avec un simple navigateur web. D'autres développements seront nécessaires pour pouvoir réaliser de telles performances, qui posent là-encore des questions d'ergonomie visuelle.
\indent Dans l'ensemble, les partitions informatisés laissent place à de nombreuses interactions possibles pendant la durée de la performance. Leur design pourrait probablement tirer partie du fait de les considérer comme un instrument collectif, dont chaque musicien, ceci incluant public et son écoute active, pourrait jouer.
 % INCLUDE: notation
% !TEX root = ../thesis-example.tex
%
\chapter{Conclusion}
\label{ch:conclusion}

\cleanchapterquote{
\textnormal{MUSIQUE.} Coup de baguette et récapitulation des musiques précédentes ou musique source seule.\\ 
Un temps.\\ 
\textnormal{PAROLES. — Encore. (}Un temps. Implorant.\textnormal{) Encore ! \\
MUSIQUE.} Répète dernière musique telle quelle ou à peine variée.\\ 
Un temps.\\ 
\textnormal{PAROLES.} \textit{Profond soupir.}
}
{Samuel Beckett}{(dans: paroles et musique. 1962.)}

Maurice Conti, TED talk the incredible inventions of intuitive ai:
\vspace{-1em}
\begin{itemize}[noitemsep]
	\item Things fabricated => things farmed
	\item constructed => grown
	\item isolated => connected
	\item extraction => aggregation
	\item obedience => autonomy
\end{itemize}
 % INCLUDE: conclusion

%% !TEX root = ../thesis-example.tex
%
\chapter{Latex Sandbox}
\label{ch:latex_sandbox}

Exemple de \hl{texte surligné}. Et voilà


\section*{titleboxes}

\begin{titlebox}{A physical explanation of the \emph{dynamic matrix}}
lots of text\\
a new line\\
equation
where  is a unitary matrix (each column is one of the eigenvectors of the dynamic matrix is the product of the number of particlces, and the number of dimensions,
\end{titlebox}

\begin{notebox}
A physical explanation the \emph{dynamic matrix}\\
lots of text\\
a new line\\
equation
where  is a unitary matrix (each column is one of the eigenvectors of the dynamic matrix is the product of the number of particlces, and the number of dimensions,
\end{notebox}

\begin{notebox}
A physical explanation the \emph{dynamic matrix}\\
lots of text\\
a new line\\
equation
where  is a unitary matrix (each column is one of the eigenvectors of the dynamic matrix is the product of the number of particlces, and the number of dimensions,
\end{notebox}

\Pierre{\blindtext}

\section*{figures}

\subsection*{figure simple dépassant sur les marges}

\begin{figure}[!htbp]
	\centerline{
		\includegraphics[width=1.2\textwidth]{gfx/06_visual_representation/mpTUI_pitchgrid_72dpi.png}
	}
	\caption{à gauche : évolution de la forme des ouïes du violon d'après \cite{nia_evolution_2015}. à droite : Extrait du brevet de J. Djalma sur \iquote{l'amélioration du clétage des flûtes de Boehm}, 1908.}
	\label{fig:sandbox:single}
\end{figure}

ou bien

\begin{center}
  \makebox[\textwidth]{
  	\includegraphics[width=\paperwidth]{gfx/06_visual_representation/mpTUI_pitchgrid_72dpi.png}
  	}
\end{center}

\subsection*{figures multiples, dépassant sur les marges}

\begin{figure}[!htbp]
	\makebox[\linewidth][c]{%
		\begin{subfigure}[b]{.6\textwidth}
			\centering
			\includegraphics[width=.95\textwidth]{gfx/06_visual_representation/f-hole.png}
			\caption{a test subfigure}
		\end{subfigure}%
		\begin{subfigure}[b]{.6\textwidth}
			\centering
			\includegraphics[width=.95\textwidth]{gfx/06_visual_representation/Julliot_patent.png}
			\caption{a test subfigure}
		\end{subfigure}%
	}\\
	\makebox[\linewidth][c]{%
		\begin{subfigure}[b]{.6\textwidth}
			\centering
			\includegraphics[width=.95\textwidth]{gfx/06_visual_representation/Julliot_patent.png}
			\caption{a test subfigure}
		\end{subfigure}%
		\begin{subfigure}[b]{.6\textwidth}
			\centering
			\includegraphics[width=.95\textwidth]{gfx/06_visual_representation/Julliot_patent.png}
			\caption{a test subfigure}
		\end{subfigure}%
	}
	\caption{A figure with four subfigures}
\end{figure}

/blindtext

\begin{figure}[!htbp]
	\centering
	\begin{minipage}{.45\linewidth}
	    \includegraphics[width=\linewidth]{gfx/06_visual_representation/f-hole.png}
	    \caption{First caption}
	    \label{img1}
	\end{minipage}
	\hspace{.05\linewidth}
	\begin{minipage}{.45\linewidth}
	    \includegraphics[width=\linewidth]{gfx/06_visual_representation/Julliot_patent.png}
	    \caption{Second caption}
	    \label{img2}
	\end{minipage}
\end{figure}


\begin{figure}[!htbp]
	\captionsetup{format =plain}%
	\centering
	\begin{minipage}[t]{0.48\textwidth}
		\includegraphics[width=\linewidth]{gfx/06_visual_representation/f-hole.png}
		\caption{first figure but with more comments than the second picture to see what the different is.}
	\end{minipage}
	\hspace{.02\linewidth}
	\begin{minipage}[t]{0.48\textwidth}
	    \includegraphics[width=\linewidth]{gfx/06_visual_representation/Julliot_patent.png}
		\caption{second figure}
	\end{minipage}
\end{figure}
\bookmarksetupnext{level=part}
\begin{appendices}
	%\chapter{Interviews}
\label{appendix:interviews}

\section*{qu'est ce qui t'a amené au numérique ?}

\vspace{-1em}
\begin{itemize}[noitemsep]
\item Bernier : l'infini des possibles du son
\item Saint Denis : le bateau qui va vite (la modernité, la puissance)
\item Dumeaux : envie de faire de la musique "avec du plaisir" après avoir entendu un concert, entre 7 et 12 ans
\item Zamborlin : DJ-ing
\item Turchet : étendre les possibilités d'un instrument qu'il connaissait déjà
\item Collins : écouter comment l'électronique sonnait (composing inside electronics Tudor)
\item Mamou-Mani : Instruments classiques, musique concrète, recherche
\item Fernandez : trouver des nouvelles sonorités
\end{itemize}


	\chapter{Interview : Serge De Laubier}

\section*{Biographie}


\section*{Transcript}
 %ok 17/07/2017
	\chapter{Interview : François Dumeaux}
\label{appendix:dumeaux}

\section*{Biographie}


\section*{Transcript}

François Dumeaux, interview du 30/07/2017, dans son studio, Cuzorn, France.



 séquence musicale 

VG —  C'était très bien de commencer par une séquence musicale car cela me permet de te poser tout de suite la question de décrire un petit peu cet instrument, ces instruments, je ne sais pas comment tu les appeles, s'ils ont un nom 

FD —  alors celui là, je ne fais pas preuve d'originalité, je l'appelle le synthétiseur modulaire … mais j'y adjoins cet instrument qui est un instrument de musique traditionnel qui est un tambourin à cordes que j'ai détourné complètement de sa fonction de départ. Normalement, ça se joue avec une flute harmonique et ça sert à faire un bourdon rythmique, et moi je m'en sers pour faire de la matière… Mais pour moi, c'est comme s'il y avait un arrière plan imaginaire de tout ce que ça draine d'histoire, de projections… même si au final le son que j'en sors est assez éloigné de ce pourquoi c'est fabriqué au départ 

VG —  dans le son et dans l'objet, tu veux dire ? 

FD —  Oui, voilà tout ce que ça peut représenter, pour moi hein, ça nourrit mon imaginaire. Je ne sais pas, c'est comme s'il y avait un lien avec qqch qui fait vraiment partie de ma pratique musicale au même titre que la musique expérimentale, il  y a la musique trad et j'ai toujours fait les deux, mélanger un peu, et plus ça va plus je mélange les deux en fait...Je fais aussi des trucs purement l'un ou purement l'autre mais les deux m'intéresse, en fait. Plus ça va et plus j'ai tendance à effacer les frontières entre mes pratiques. Donc là, ce que j'ai utilisé là c'est ce que j'utilise le plus souvent. Tu vois, tous les modules ne sont pas patchés, il y en a dont je ne me suis pas servi. Et ça, c'est ce dont j'aime bien me servir quand je fais de l'improvisation… donc c'est des oscillateurs, il y en a des analogiques, il y en a des numériques… et c'est un peu les deux opposés, c'est à dire, il y en a des très simples, des sinus, et il y a des oscillateurs à table d'onde… 

VG —  dans lesquels tu charges des samples, tu veux dire ? 

FD —  non, il a toute une banque de … je sais plus combien il en a, 64 je crois…. (les faisant écouter une à une) on balaie …  

VG —  ça interpole entre les tables d'onde ? 

FD —  voilà c'est ça… il fait du morphing entre les tables d'onde… voilà et puis après il y a des trucs de modulation, de l'aléatoire, des cycles très longs, et puis… euh… et puis j'utilise celui-ci (joignant le geste à la parole) qui … euh… un truc qui marche avec l'électricité statique (petite séquence de jeu, sons très bruitistes, corrosifs, à changements rapide)… voilà… ça je l'utilise beaucoup en improvisation… et je sais pas si tu as fait gaffe, il y avait un paysage sonore…de vent…  

VG —  oui j'ai entendu 

FD —  et alors c'est ce module là, qui est un truc qui était un kit… que j'ai soudé moi, ça s'appelle le « radio-musique » et c'est basé sur la pièce de John Cage… c'est l'idée, tu sais, de composer une pièce où tu … où dans la partition, tu avais « ouvrir la radio à telle fréquence, à tel moment» … et alors comme ça a beaucoup changé la radio et même que bientôt ça risque de plus exister, la radio… analogique, ils se sont dit, on va faire le même  truc mais avec nos propres banques de son donc tu coup, tu as… tu peux avoir plein de canaux différents (séquence de jeu) là par exemple, c'est une banque, c'est que des paysages sonores et je joue le tuner, je vais passer sur un autre paysage sonore… et pendant ce temps, l'autre il continue comme si c'était une vraie radio, donc si je reviens sur  l'autre, ça reprend pas la lecture où c'était, ça a continué entre-temps de compter…des cycles… (faisant la démo, sons d'oiseaux) …  

VG —  c'est radio-John… 

FD —  ouais… (rires) … alors évidemment, c'est de trucs que tu peux moduler donc tu peux … déclencher… un changement de fréquence .. et moi je m'en sers surtout comme d'une banque de sons pour… euh… de la même façon que quand j'utilise un ordinateur pour faire des citations parce que j'ai envie sur le moment… je l'ai sans avoir besoin de trainer un ordinateur … et je disais il y a plusieurs banques.. là c'est la banque paysages...(extraits)… bon là j'ai ça.. (bleeps)… là c'est des instru acoustiques avec pas mal d'instru traditionnels… ça c'est de la flûte… des séquences de jeu que j'ai enregistrées… du stick, une cornemuse landaise… pas tout à fait jouée comme prévu mais… voilà… tu vois le truc quoi … euh… et, oui, je fais passer le son du tambourin à corde dans un filtre résonant...euh… ici… (démo) et j'ai \hl{de l'aléatoire} (l'aléatoire est une matière, NDT) qui change la coupure du filtre … (extraits sonores)…. Bon là j'ai des réglages un peu exagéré mais… voilà… cet espèce de chassé-croisé comme ça… et... en fait, c'est euh… je me suis rendu compte en réfléchissant à cette interview qu'on devait faire que j'utilisais en fait exactement les mêmes choses, euh… j'étais allé vers les mêmes matériaux que du temps où je travaillais avec des patchs Max, çàd majoritairement des sinus, de la synthèse type forme d'onde, donc FM, modulation d'amplitude… et des samples... et du geste !… 

VG —  paysages sonores et des sons produits en direct… 

FD —  ouais… 

VG —  ensemble.. 

FD —  ouais… et après, ce que ça a fait … moi presque pendant 20 ans j'ai utilisé quasiment que l'ordinateur… c'était aussi pour raisons financières, hein… j'avais ça et ça me permettait d'avoir un maximum de synthètéiseurs sans avoir à les acheter en fait… et ce qui m'a fait basculer,en fait,  c'est l'arrivé du MiniBrute (de Arturia, NdE) qui est un petit synthétiseur analogique… au début je l'ai pris plus pour m'amuser… et en fait ça m'a vraiment accroché...pour retrouver le geste...et puis quelques mois après j'ai commencé le synthé modulaire… 

VG —  De retrouver le geste, parce sur l'ordinateur tu étais plutôt avec la souris ? 

FD —  eh ben, non j'ai beaucoup utilisé les interfaces … mais… il me semble qu'il y a toujours euh… un moment où t'as un peu la flemme… où tu vas pas aller au bout, où tu dis 'ah ce serait pas mal, là de faire une automation de volume mais bon, je vais plutôt le faire à la souris... » … quand tu composes par exemple une pièce, combien de fois je me suis retrouvé à aligner des belles courbes… c'était pas avec les oreilles que je faisais ma courbe de volume, c'était avec mes yeux. Voilà… donc c'est un peu… tu te retrouves à faire des trucs absurdes...d'être tatillon… (avec le ton tatillon, NdE) « ah non c'était à 127 » alors qu'en fait on entend pas la différence… par exemple… mais du coup ouais, quand j'ai commencé le synthé modulaire, ça m'a….fait prendre un chemin que j'avais pas prévu au départ …. pas à ce point, disons. Moi, comme je te disais, j'ai toujours mélangé musique expérimentale et musique trad mais auparavant j'amenais plutôt mon instrumentarium électronique dans des formations traditionnelles… et où je venais un peu jouer le trouble-fête...mais je chantais pas, je faisais pas d'instrument acoustique ni rien...et en fait… donc au début quand j'ai commencé le synthé modulaire c'était plus dans une idée d'avoir des beaux matériaux, que j'allais assembler, continuer mes compositions pareil dans l'ordinateur, sur plusieurs pistes et tout ça… et en fait je suis retrouvé très vite à mordre un autre hameçon qui est le jeu en direct … et pendant 2 ou 3 ans, ma pratique musicale c'était que ça, j'étais devant mon modulaire, je faisais des patchs, je les enregistrais quand ça me plaisait, c'était très éphémère, et…. 

VG —  tu enregistrais … le son ? 

FD —  j'enregistrais ce qui sortait oui… et parfois je faisais des patchs génératifs que je laissais tourner pendant des jours ...et que je changeais un petit micro-poil.. alors là on était plus proche de l'installation par exemple… mais il y a eu beaucoup un truc de « jouage »… de vraiment jouer en direct des trucs… de plus en plus, de plus en plus, jusqu'à arriver à un point où le …. le logiciel que j'utilisais d'habitude pour du montage se retrouvait juste un magnétophone multi-pistes… et je jouais mes pièces, de A à Z, quoi. Bon maintenant je fais… 

VG —  plus trop un outil pour composer, mais plutôt pour le cas où tu veuilles enregistrer ou… 

FD —  en tout cas la composition ne se faisait plus, euh… de manière formelle, en pensant à « tiens je vais rajouter ça... », c'était plus préparer un … euh… un réservoir de jeu, et en jouer. Donc un truc entre composition et improvisation un peu, on pourrait dire…bon maintenant, je fais encore des compositions plus… classiques… voire aussi des compositions sans utiliser du tout le synthé modulaire, mais j'ai eu une espèce de période comme ça un peu exclusive où je ne faisais plus que ça, pendant 2 ou 3 ans. Mais donc ça m'a amené vers d'autres choses, parce que je me suis dit,, tiens, ce serait chouette d'y adjoindre un tambourin à cordes… donc il y a un copain qui me l'a fabriqué, et je l'ai aidé … c''est Romain Colautti… et à partir de là, ça a été encore plus la fuite en avant, je me suis mis à chanté et j'ai chanté de plus en plus et j'ai même été après, du coup, jusqu'à avoir envie de m'inscrire dans un DEM de musique traditionnelle, et voilà j'ai fait ça… et alors c'est marrant parce que du coup, ça a encore relativité l'outil… Donc comme je te disais j'ai eu une période exclusive où il n'y avait plus que ça, et là il est revenu à sa place d'outil… et maintenant par dessus je fais du tambourin à corde, je chante, je commence même à apprendre le violon, etc. et lui je m'en sert pour, euh, ça fait partie de mon ... instrumentarium, on va dire, maintenant… et par exemple avant tout ça j'ai fait beaucoup de prises de son, de paysages sonores, de trucs comme ça... et pareil quand j'ai eu ma période très obscessionnelle du modulaire, ja faisais plus du tout de prise de son… et là maintenant j'ai repris, tu vois, donc y'a un espèce de truc qui s'est ré-équilibré, ça a généré tout un tas de trucs, ça m'a fait bifurqué dans plein de direction que j'avais pas prévues, donc ça c'est vraiment pas mal, et maintenant ça a repris sa place… parmi d'autres… 

VG —  ok tu as fait un peu l'interview dans le sens inverse de ce que j'avais prévu (rires), parce que je voulais te demander comment ça avait commencé et on est parti, pas des derniers, mais des presque derniers outils que tu utilises pour créer tes enregistrements… tu parlais d'enregistrement je ne sais pas s'il faut commencer par ça mais la question que je voulais te poser, alors tu y as déjà répondu en partie mais si je reprends le fil dans l'autre sens, une question par laquelle je voulais commencer, c'est qu'est ce qui t'a poussé à la base à faire de la musique avec des outils comme ça plutôt qu'avec des instruments pré-..., avec une histoire, des cours, une pédagogie, pourquoi prendre un instrument bizarroïde et mal fini… 

FD —  ouais, ouais, ouais… euuuuh…. En fait au début, je pense qu'il y a plein de raisons, et une des raisons c'est que j'avais envie de faire de la musique de manière… euh… avec du plaisir en fait… moi depuis gamin je voulais faire de la musique, et … bon, d'ailleurs j'avais envie de construire des synthés modulaires quand j'étais petit, parce que j'avais vu Jean-Michel Jarre à la télé, et j'avais dit « ah ouais, ça c'est super » … mais bon sans savoir du tout à quoi ça correspondait mais… et donc j'étais allé à la bibliothèque de mon village et j'avais demandé à la bibliothécaire « est-ce que vous avez des livres pour construire ses propres synthétiseurs ? » et elle devait se dire « qu'est ce qu'il a lui ? » (rires) et du coup elle disait « ah non, on n'a pas ça, désolé » et du coup j'attendais un mois et je revenais « et maintenant vous en avez ?» … donc au bout d'un moment elle a dit « non mais ça n'existe pas en fait » (rires)… voilà 

VG —  tu avais quel âge ? 

FD —  là, j'avais, euh, je sais pas, je devais avoir entre 7 et 12 ans, un truc comme ça… 

VG —  et tu jouais d'un instrument de musique ? 

FD —  non, alors au départ non, et alors du coup j'enquiquinais mes parents pour apprendre à jouer du synthétiseur, tout ça… et ils me disaient que c'était pas possible et gnagnagna … et puis en fait, euh… du coup j'ai essayé de me dire mais comment je peux faire pour faire quelque chose qui ressemble à ça donc j'ai cherché, cherché, et dans mon cerveau d'enfant, de ce que je voyais à la télé ou quoi, ce qui se rapprochait de ce que j'imaginais du synthétiseur, c'était la guitare électrique. Donc j'ai dit à mes parents que je voulais faire de la guiatre électrique. Alors ils m'ont dit «  mais il n'y a pas de cours de guitare électrique, faut que tu fasses des cours de guitare classique » … et donc me voilà à prendre des cours de guitare classique, c'était assez éloigné de ce que je voulais faire au départ… 

VG —  … pour faire du synthétiseur… 

FD —  et évidemment la déception a été grande, d'autant que le mec dont je ne citerai pas le nom était un piètre pédagogue, donc il m'a fait jouer les trois mêmes morceaux pendant trois ans, non, pendant deux ans… Pendant deux ans les mêmes morceaux en boucle, ça le dérangeait pas… il faisait ses heures et… et voilà au bout d'un moment j'ai dit que je voulais arrêter parce que c'était décevant justement … et donc là mes parents ont décidé que je n'avais pas vraiment envie de faire de la musique … donc ça aurait pu s'arrêter là… et ça s'est pas arrêté là… et donc après, à l'adolescence, découverte du rap, de la techno, des musiques électroniques en général, et j'ai commencé à bidouiller des machins… je les ai tannés assez longtemps pour qu'ils me filent un peu d'argent pour acheter mes premiers trucs, et j'ai commencé avec un 4 pistes à K7, un synthé … parce qu'entre temps ils avaient quand même acheté un synthé à ma sœur — qui n'avait jamais demandé ça… mais qui jouait du piano, donc elle avait le droit de faire du synthé analogique parental… et elle ne s'en servait pas du tout, et d'ailleurs je l'ai là … il est là, derrière… (rires)… et c'était un synthé FM, figures toi… un truc pour enfants, mais qui avait un moteur de synthèse FM rudimentaire...donc là j'ai vraiment beaucoup passé de temps à jouer sur le spectre et tout ça… donc il y a un espèce de début comme ça… j'achetais Keyboard magazine et je bavais sur de trucs qui étaient moins intéressants en fait, mais qui avaient l'air mieux parce que c'était « professionnel »…  mais je faisais mes armes là-dessus… après j'ai acheté des boites à rythmes et tout ce genre de trucs… et ensuite … 

VG —  c'était un synthé avec des boutons du coup ? 

FD —  Euh, il y a des tirettes… 

VG —  c'est celui à gauche là ? 

FD —  c'est pas celui là, il est derrière, là, attends je te le sors… c'est vraiment le synthé-jouet, de la gamme PSS de Yamaha, c'est des trucs, c'était pour enfant quoi, c'est des petites touches… (me le montrant) Donc là, il y avait le spectre, l'intensité de modulation, et puis l'enveloppe, attack, decay, release, le vibrato et un volume… c'est rudimentaire mais ça commence à … 

VG —  tu peux sculpter tes sons… faire un peu de cuisine… 

FD —  ouais… et tu pouvais transformer la banque de sons que tu avais à l'intérieur… tu partais de sons qui étaient déjà plus ou moins complexes et tu pouvais… ouais.. je l'ai ré-utilisé, j'ai fait des pièces électroacoustiques avec ce synthé, il n'y a pas si longtemps… en fait, il est très très bien … (rires)… et voilà et… bon après voilà j'ai fait de la musique électronique plus ou moins techno et tout ça… et par la rencontre d'un copain, je suis rentré dans la classe d'électroacoustique de Bordeaux, et alors qu'au début je pensais juste venir chercher des techniques, faire une espère d'espionnage industriel, encore une fois, tout ce qu'on fait transforme nos pratiques, et là beaucoup plus que ce que j'aurais imaginé parce que ça m'a vraiment beaucoup plus plût que ce que je croyais au départ… j'arrivais avec des a priori, de trucs élitistes, chiants… un peu les clichés, quoi… et en fait, notamment la musique de Bernard Parmeggiani, où j'ai complètement plongé dedans, et dans les cours, dans le cursus, il y avait une initiation à Max... et alors là ça a été super parce que tout à coup je pouvais vraiment aller dans la matière, beaucoup plus qu'avec des synthés fabriqués par d'autres… donc là il y a eu une période, là aussi encore une fois bien intensive, et  bien obsessionnelle… 

VG —  que tu utilisais pour te créer tes propres sons du coup ? 

FD —  ouais, je faisais mes patchs… 

VG —  … ou tu t'en servais pour du live ?  

FD —  alors il y avait les deux… il y avait les trucs de matières que j'utilisais comme réservoir pour mes pièces composées et il y a eu aussi l'improvisation très très vite… parce que l'improvisation ça a fait partie assez vite de ma pratique musicale… ouais, dès que j'ai commencé à faire de la musique vraiment sérieusement, j'ai composé des trucs « et » fait de l'improvisation … les deux m'intéressent… pour des raisons différentes… et donc ouais, il y avait des patchs de matière, il y avait un patch que j'avais commencé où il y avait des couches et des couches … je sais pas, j'ai du l'améliorer sur 7 ans un truc comme ça… je pourrais en rejouer… 

VG —  le même patch, le même noyau qui s'est …. 

FD —  ouais la même base qui s'est enrichie, complexifié… qui au début était un truc très simple, juste un jeu de lire un sample à des vitesses différentes de manière aléatoire… et puis petit à petit… et là je l'utilisais avec une interface de contrôle UC33 (de la marque XXX TODO, NdT)  

VG —  c'est des faders, non ? 

FD —  oui, il y avait 8 faders et dessus t'avais 8 fois 3 potentiomètres rotatifs…que tu pouvais assigner à ce que tu voulais… et c'est quand même une interface vachement bien… 

VG —  ouais… qui a eu ses grandes heures de gloire… 

FD —  ouais… et que j'ai trouvé en fait souvent plus pertinente que des choses qui voulaient aller plus loin, du type Méta-Instrument (inventé par Serge de Laubier, NdE) et tout ça... qui, pour moi, il y a un côté ou presque on se perd dans le contrôle et on oublie presque le sonore… alors il y a des gens qui en font des choses super mais… quelque part le côté hyper-simple de « un fader », moi ça me plait plus… d'ailleurs je me retrouve encore avec un fader ici (en montrant les faders de la petite table de mixage à côté de son modulaire interface modulaire) voilà… une espèce de robinet à son…  

VG —  ok… parce que sur un fader d'UC33 tu peux y mettre plein d'autres choses qu'un volume sonore… 

FD —  oui… c'est vrai… c'est vrai… mais je m'en servais, dans mon patch principal d'improvisation, c'était exactement ça… c'était des canaux, que je dosais … donc voilà, mais donc oui après quelques années, je sais pas, j'ai du faire presque sept ans de Max, et quand j'ai basculé là dessu (le modulaire NdE) voilà… ça m'a rendu un peu triste, Max… je me retrouvais devant un écran (en prenant un dos vouté, NdE) à faire des trucs… là j'avais… j'ai presque retrouvé une sensation que j'avais au début en découvrant Max, de (mimant des gestes de connections) « ah et si je fais ça qu'est ce que ça fait ? »  

VG —  de patcher ? 

FD —  de patcher, et en même temps comme je te disais, avec un truc de corps, un truc direct, et un truc éphémère qui a quand même sa valeur… tu peux pas rappeler un patch en un clic, alors ça peut être un inconvénient mais ça peut aussi être un avantage parce que … euh… je sais très bien comment refaire le même type de truc que j'ai fait une autre fois mais je ne vais jamais le faire exactement pareil, donc ce sera « le truc de ce jour là », quoi… un côté comme ça qui me plait pas mal… 

VG —  et tu retrouves des chemins là-dedans quand même ? 

FD —  ouais, j'ai mes espèces de routines, de logiques, de programmation… je sais très bien que si je veux faire par exemple un truc très … des nappes dans la durée, je vais partir là-dessus, si je veux faire quelque chose de très dynamique, je vais patcher autrement… 

VG —  et le patch, tu le fais en amont de ton jeu, ou bien tu patches en cours de jeu des fois ? 

FD —  ça dépend… en concert, j'aime bien ne rien avoir prévu et juste je démarre avec du sinus et on va voir ce qui se passe… 

VG —  et rajouter … 

FD —  … je rajoute, je patche en jouant,  

VG —  .. partir quasiment sans cap tu veux dire, ou bien avec un truc très simple … 

FD —  ouais, juste, ouais, très simple, et je rajoute, là  j'avais prévu quelques trucs mais c'était aussi dû à la contrainte de faire une improvisation courte, donc, si je dois commencer à patcher, ça prend du temps quoi… ouais de longues improvisation de 1 heure, 2 heures, 3 heures... 

VG —  il y a un truc dont tu parlais de rappeler un preset en un clic, qui n'est pas forcément qu'un avantage, et donc je parlais de chemin tout à l'heure, parce que si tu veux passer d'une config à une autre, tu peux pas physiquement dé-pluguer tout et re-pluguer en un clic justement, et donc, enfin moi je ne pratique pas le synthé modulaire, mais ça me fait penser à cette caractéristique à laquelle je n'avais pas pensé avant, qui est que tu prends des chemins que tu tricotes et dé-tricotes, et d'une certaine manière, quand tu veux aller quelque part, tu … (34:03) 

FD —  ouais… ben par exemple, mettons, j'ai ça.. (début de séquence musicale) si lui par exemple je veux le transformer, je sais pas, euh, j'aurais envie de le faire interagir avec lui, ce que je vais faire, je vais faire une espèce de tour de passe-passe, je vais venir faire un autre événement donc pendant ce temps je fais ça, donc il y a un truc musical qui est en train de se passer… je débranche mon truc… je prends un câble… et… je branche… je vais pouvoir ré-introduire mon son … (fin de la démo) et par exemple des fois je peux…. Ça peut m'arriver d'être moins dans le jonglage que ça et patcher alors qu'on entend le son… mais… c'est plus périlleux… faut être un peu joueur quoi, des fois ça ne fait pas du tout ce que tu avais prévu… 

VG —  ouais.. (rires) 

FD —  hehe… donc là faut jouer avec, quoi … mais c'est des trucs que j'aime bien en improvisation justement, les accidents, les surprises… des fois ça m'est arrivé en improvisant de… d'être surpris par le résultat et de trouver que c'était un peu moche, que c'était presque de mauvais goût, quoi, par rapport à mon goût personnel… donc là du coup, démerdes toi avec ça !… (rires) 

VG —  et… t'as une sorte de pavage… c'est une autre chose qui m'intéressait qui est aussi caractéristiques des DMI de pouvoir changer la config en un clic et de ne pas se retrouver avec les mêmes choses sous les doigts, c'est à dire qu'un même bouton peut changer une fonction…  

FD —  Ouais … et bien là ça peut être aussi le cas… enfin pardon, je te coupe… 

VG —  il y a à la fois cette question là, et il y a aussi la question de la construction musicale là dessus où … en fait dans ces instruments qui sont un peu des méta-instruments, que tu ré-assembles, que ça soit en un clic ou via un patch, tu as une sorte de programmation plus ou moins en live, avec ton instruments…  

FD —  ouais 

VG —  et sur des longs set, sur la durée d'un concert, du coup tes stratégies pour passer d'un… enfin, est ce que tu fonctionnes plutôt avec un set continu ou est ce que tu as des… 

FD —  ça dépend… 

VG —  parce que quand tu as des pistes, des morceaux séparés, comment tu passes d'un morceau à l'autre, alors que dans les logiciels de type (Ableton) Live, les gens rappelle le presets du deuxième morceau… 

FD —  non, alors je ne l'utilise pas du tout pour faire des morceaux précis que je rejoue… ça je ne fais pas du tout ça...en fait ce serait hyper compliqué de faire ça, d'ailleurs… c'est pas du tout prévu pour… je dis pas que c'est impossible mais à mon avis faut beaucoup s'entrainer pour un résultat pas forcément génial… par contre ça m'est déjà arrivé de faire un set en plusieurs fois, et au milieu de couper le son, débrancher, et repartir de zéro… comme si c'était un autre morceau, ou une autre pièce… enfin, on met les mots qu'on veut mais… et par contre ça m'arrive de l'utiliser dans des groupes, par exemple, où là il y a des choses qui sont prévues mais c'est plus de l'ordre de … à tel moment, à tel tableaux, je sais que je fais des vagues de sons modulés en FM… et c'est pas plus prévu que ça… 

VG —  c'est pas forcément une config qui est prévue mais plutôt un résultat sonore que tu atteins d'une manière ou d'une autre ? 

FD —  alors, si… en général quand je fais des trucs avec des groupes, s'il y a des tableaux vraiment prévus, là je note des patchs de manière très succinte, c'est à dire, telle modulation sort de tel connecteur et va à tel connecteur, et en gros les positions des potars… donc on n'a pas exactement le même résultat mais on a quelque chose approchant, quoi… et de toute façon ça a vocation à jouer, j'en joue quoi… les positions de potars sont des positions moyennes quoi, après j'en change.. peut-être pas tous, mais… 

VG —  et ça me fait penser à une autre question qui est la question de l'ergonomie de l'instrument, parce que dans ces DMI, il y a une ergonomie de l'instrument qui n'est, a priori, pas dictée par des critères de facture acoustique, à l'inverse des instruments acoustiques… et du coup c'est souvent plus défini, à la fois par des contraintes techniques, du fait que tel module a besoin de tel bouton là, tout ça, et aussi par la contrainte personnelle, enfin les désirs ou les contraintes liée à ta propre pratique, à ton propre corps, à comment tu organises les choses que tu as sous les doigts… 

FD —  ouais, aussi ta façon de le voir, de l'imaginer en amont... 

VG —  oui… et par exemple pour ce dispositif là (le modulaire dont on parle, NdE) qu'est ce qui fait que tu as organisé les modules de cette manière là plutôt que d'une autre ? 

FD —  j'ai essayé d'être au maximum logique… j'ai mes sources en haut, tout ce qui est modulation… donc les oscillateurs, et puis le sampleur dont je te parlais qui fait la radio, là…  là, j'ai toute les sources qui vont venir transformer, envoyer du signal ou de l'aléatoire, de la commande un peu quoi… j'ai par exemple ici un mini séquenceur euclidien… 

VG —  pour piloter les sources ? 

FD —  euh… qui envoie des impulsions, quoi 

VG —  c'est à dire, ceux du haut c'est plutôt des générations horizontales de … 

FD —  c'est plutôt eux qui vont me générer du son, et eux vont venir faire des variations de voltage, donc selon où je le branche ça fait quelque chose ou quelque chose… 

VG —  des enveloppes ? 

FD —  ça peut faire des enveloppes, ça peut faire des LFO, ça peut faire des… ouais. 

VG —  qui vont venir sculpter tes sources… 

FD —  ça peut venir changer la fréquence de ma radio… 

VG —  d'accord 

FD —  où ouvrir le filtre qui vient filtrer mon tambourin à corde 

VG —    

FD —  ouais… je sais pas...euh… 

VG —  c'est à dire quand tu as un LFO, je dis générateur de geste parce que on pourrait le moduler … 

FD —  oui, quelque part c'est comme si on déléguait à quelqu'un de faire … (geste d'oscillation de la main) 

VG —  … sur le potar… 

FD —  oui voilà c'est ça… et après en bas, c'est plutôt les trucs de geste, en fait là c'est bêtement parce que \hl{c'est là où je met les mains…} là sur le patch que j'ai montré tout à l'heure, je ne me suis pas servi de ça mais celui là je m'en sers pas mal en impro… 

VG —  Sur es capteurs électrostatiques aussi non ? 

FD —  Voilà … qui lui, sert par exemple à … c'est comme des banques de mémoire de voltage… donc là j'ai branché, y'a trois lignes, j'ai branché la sortie de la première ligne sur le pitch de ce premier oscillateur… (démo de l'effet) mais si je le branche sur la fréquence de coupure de mon filtre, je vais faire résonner le filtre (du tambour, NdE) 

VG —  c'est une sorte de clavier que tu peux assigner à n'importe quel paramètre 

FD —  voila, tout le truc qui est super avec ce système, c'est que tu as des voltages et selon où tu le branches, c'est des voltages qui vont contrôler une hauteur, un volume, une intensité, une vitesse, enfin bon comme dans Max en fait, quand tu as des trucs qui vont de 0 à 127, c'est pas 127dB, c'est juste à 127  

VG —  c'est une course en 0-5V 

FD —  oui, je sais plus, c'est plus ou moins 12V je crois 

VG —  ton patch orange c'est ton patch de modulation ? 

FD —  non j'ai pas de codage de couleur, mes câbles oranges c'est juste des jack-jack simples 

VG —  mais tu as une différenciation entre ce qui est signal audio et modulation ? 

FD —  non pas sur ceux là, sur certain oui, sur les Buchla il y a carrément un format différent de câble pour ce qui est signal et ce qui est euh… 

VG —  là c'est tout au même niveau 

FD —  oui 

VG —  c'est à dire tu peux prendre un signal audio et le mettre comme une modulation d'autres trucs… 

FD —  oui, et d'ailleurs je trouve que c'est un avantage parce que du coup tu peux faire des trucs qui n'étaient pas prévus et moduler un filtre en audio, par exemple, c'est un truc que j'adore faire… par exemple (faisant la démo) là j'ai une sortie audio de ce générateur d'aléatoire et là comme tu entends ça … 

VG —  c'est la fréquence de coupure que tu modules avec le signal audio ? 

FD —  oui… ou ça pourrait être aller récupérer un sinus… donc tu déconstruis un peu des trucs de…  « ah les filtres, c'est ... »  

VG —  ça existe aussi dans Max entre ce qui est signal et ce qui est message de contrôle 

FD —  oui, tu peux faire la traduction et du coup c'est hyper bien … 

VG —  tu peux la faire, mais quand même elle est présente comme qui n'est pas … 

FD —  elle est marquée oui c'est vrai… 

VG —  certains paramètres que tu voudrais pouvoir contrôler directement en audio et tu te retrouves à devoir les convertir en message avec ce que tu perds… 

FD —  oui c'est ça, du coup tu as une fréquence d'échantillonnage un peu 

VG —  oui.. 

FD —  après qui peut générer une autre esthétique mais… 

VG —  et… j'ai perdu le fil de ce que je disais avant, je parlais d'ergonomie et… 

FD —  oui, mais ça c'est marrant, je vois aussi des points commun entre quand je faisais du Max et ça c'est, euh, déjà c'est évolutif… c'est pas figé… et… euh, il y a quelque chose que j'ai pu trouver hyper pratique à un moment me semble absurde, peut-être un an après, et je vais changer des trucs… j'ai pas mal discuté avec des gens qui pratiquaient ça, le synthé modulaire, et il y en a certains qui ré-organisent de manière totalement aléatoire leur modules et ils disent que ça les fait  faire des choses différentes qu'ils ne faisaient pas du tout avant… du coup ça leur change leur logique, ça leur fait faire… ré-essayer des trucs qu'ils ne faisaient plus… ou qu'ils n'avaient jamais essayé… du coup c'est rigolo… 

VG —  ah, tu as un truc qui n'est pas propre au son électronique, mais qui est fortement encouragé par ce genre de dispositif où tu es dans l'exploration sonore beaucoup, et c'est parfois la critique inverse qui est entendue à propos du fait que les instruments électroniques sont des instruments sur lesquels il est difficile d'avoir une pratique comme celle d'un violoncelliste, qui joue de son violoncelle qui ne bouge pas … 

FD —  oui, qui va rejouer exactement la même chose oui… 

VG —  oui, ou alors explorer son instrument, mais elle est dans rechercher des choses fines et inaccessibles avec son instrument … 

FD —  ben pour moi en fait c'est différent…   % ok 30/07/2017
	\chapter{Interview : Lucas Turchet}
\label{appendix:turchet}

\section*{Biographie}
\noindent Né à Vérone, Italie en 1982, Lucas Turchet obtenu une maîtrise en informatique de l'Université de Vérone en 2006, une maîtrise en guitare classique et composition du Conservatoire de musique de Vérone, en 2007 et 2009, un doctorat en technologie des médias de l'Université Aalborg, Copenhague, Danemark, en 2013, et une maîtrise en musique électronique du Royal College of Music de Stockholm, Suède, en 2015. Il est actuellement boursier Marie-Curie au Centre of Digital Music, Queen Mary University of London, Londres, Royaume-Uni. Il est également professeur adjoint au Département de génie de l'information et d'informatique de l'Université de Trente, Italie. Ses recherches scientifiques, artistiques et entrepreneuriales ont été soutenues par de nombreuses subventions de différents organismes de financement, dont la Commission européenne, le ministre italien des Affaires étrangères et le Conseil danois de la recherche. Ses principaux centres d'intérêt de recherche sont la technologie musicale, l'interaction homme-machine et la perception multimodale. Il fait partie des fondateurs de Mind Music Labs, une entreprise basée à Stockholm fabriquant des instruments augmentés.

\noindent Site web : \url{http://www.lucaturchet.it}

\section*{Transcript}

\noindent Lucas Turchet, interview du 23/08/2017, à Queen Mary University, en marge de la conférence AudioMostly'17, Londres.

LT — To answer to your question basically I... what is the motivation for a person to to build these instruments that are so unique and peculiar and why play them, right? and simply because I find I had this motivation because I wanted to explore something new it was a need for me to express myself in other ways by using an interface that I knew already a lot but I was not satisfied only with the possibilities of the acoustic sound itself and I was not satisfied with the possibilities of using external equipment such as a conventional foot pedal, stompboxes that are used to to affect the sound process the sound and I also wanted to add further sounds not contemplated in the typical set of commercially available products and I think you can't play something typically a trigger drummachine or a sequencer or a synthesizer and above all control it in real time fromthe instrument in your hand and not with your foot or pressing a button on the computer so these things contributed to me, to my vision and my needs were such that such that... so great that I really wanted tobuild something that was unique customized for me, for my hands, for my playing style so in short this was just a research to primarily respond to a need that I had as a composer and as a musician and also not only this was the primary need motivation but the second motivation also was that I wanted to give this instrument one day in the hands of someone elseso composers could avail themselves of this instruments, this new interfaces and use I mean to composeI mean I didn't want to be the only one to use this instrument and I wanted ...in my ideals this instrument should be used by so this would be a more complete goal I would say

VG — so that it extends the availibility of works for this instrument ?

LT — Exactly a repertoire including more performances so when we build something new you would also to give a little bit of dignity and I mean I'm happy that I play it but it would be nice not to be the only one it would be nice for instance to have an orchestra of mandolin, or mandolin an cellos, mind sciences and technology and do something together... why not?

VG — and see what other people do with the same instrument

LT — yeah exactly, see also how people are reacting for instance with the same interface same technology, with the different ideas and sure that the ideas that they have explored arenot the only ones that are available that are possible certainly when you give an instrument to someone else he will have his own mental models, his own needs,expression needs and so on that will bring him or her to discover another set of gestures program certain types of algorithms and so on that will lead to different results than those than I have achieved

VG — When you say that you wanted to explore something new, new ways to expresswere you thinking more of sound, or gesture, or a little bit of both ?

LT — the two things go hand in hand, but I think that the very beginning was mostly a matter of control. So okay I might be happy with the delay, but I would like tohave the control at the note level, meaning that I play, I want a musical sentence and in the, I don't know, twenty notes there are in musical sentences only on three of them I want to apply a the delay effect. With the common foot pedals you can't achieve this level of detail, of note level control. So the thing was that I could play, I press a button, a button or a sensor,in the exact moment which this noteis going to happen and I release it when this note is elapsed but the effect is applied, the next note to comehas no effects and this will avoid me to do strange tip-tap movements that can't be done of course in front of an audience.

VG — So the design that you made was... did you make a study of ... as you play the mandolin I guess you knowwhat kind of movements do exist 

LT — exactly

VG — but the location where you put the buttons and sensors you chose them according to existing mandolin gestures or on the opposite, not to interfere too much ? because it can be a drawback that if you do a gesture witha conventional instrument you would trigger things without willing to do so

LT — Absolutely. The second finger can be also exploited as a compositional parameter by the way, but of course in general no you want to add the controller and applying effect when you want... okay so first of all, my investigation has started by a very simple consideration that is some mandolin players in particular myself when we play, we pose the little finger on the part on the bottom of the strings to make stability on the wrist in some occasions okay ... so it was already easy to press something to activate something and it was with no effort. So from that is easy to extend this concept too the pressure sensors that are nearby and of course with a set of ergonomical studies to understand where the sensors were better placed and a lot of researching was done to discover this, precisely this and they found this configuration then in the very end is also applied in the startup company, in the smart guitar that is produced by my startup company

VG — Another thing that I am interested in is the fact that in digital musical instruments under any sensor, you can map any sensor to any control. I don't know if you have various configurations, various presets... can you change or is it fixed ?

LT — no it's not fixed at all, and moreover there are also even different layouts, physical layouts of the sensors. I have developped different layouts but in the very end, I typically used mostly one I feel more comfortable or that the response is most much better to some needs for a set of pieces that are composed so far. It is not fixed though and the problem is that when you change the assignment of a parameter, or an effector a drum-machine, sampler or whatever synthesizer to a particular sensor and in the next piece you change or in the next part even of the same piece you change this mapping placing, I don't know, a phaser rate, I don't know, that parameter instead of the pitch of a whammy pitch-shifter of course you have to re-learn the instrument because the instrument changes. It's beautiful because you have a power in that moment, at the cost of one click, that your instrument will change totally timbre, and we switch between it very very quickly, but in the other hand you have to re-learn the instrumentand sometimes it takes a lot of practice also, especially if you do a concert with different pieces that are rather different between each other, because typically a musician want to vary. So you have to remember, you have to study it is not an invented instrument, but it is an instrument that has ``software'' (?) something is not the (...) the same problems of any instrument,so you have to study, you change piece,you have to practice, study, rehearse, learn and be perfect... (laughs)

VG — Well, digital musical instruments are more quick to change that's maybe... once you have this kind of stable behavior of an acoustic instrument and the constant evoluting behaviour on the digital musical instrument. But then what you say is that you would choose to rely only on memory for the kind of behavior modification. I mean you could have for example screens telling you ``this is preset 2'' or whatever, or do not have presets...

LT — no, no, I do. I have a setting interface, I have two interfaces, one is for the expressive control, the other is for the settings and they have were six buttons, they are capacitive sensors with LEDs which allows me to understand in which bank I am and I can navigate between the banks with up and down and there are four presets within each bank

VG — do you have like a visual feedback?

LT — no because at the moment I have only LEDs that give you this, but I don't have a small display, this is something if they want to have.

VG — but those LEDs are a visual feedback giving you information about which bank you're using, you said?

LT — no I don't have it on the mandolin I have and I don't have because I have a two lights two buttons with its own LEDs I know that if I press one time I will have a LED that will blink if I press two times the LED will not blink and I would be in the second and I typically use two banks, four banks in general : two navigating upwards and two navigating backwards, in total four but the fact is also that I am having a computer near meso the visual feedback is there I mean I can always check it in which piece I amwhere I am because I have a second screen over that on the table near me

VG — but you'd preferably skip this screenand have an embedded solution somehow ?

LT — this is precisely the reason because I am building these ...(interruption) this is precisely the reason because I am... I have developed this novel family of instruments that are called ``smart instruments''. I don't know if you are familiar with them?

VG — a little bit

LT — ok the first examplar is the ``smart guitar'' by mind music labs this company that I mentioned before and that I co-founded so just in a nutshell what are smart instruments, they are instruments that have an embedded computational unit so they have intelligence and this intelligence is responsible primarily for the sensors processing so it is an augmented instrument with sensors and actuators or loudspeakers, these in many cases placed where the sound source is directly in the instrument itself and they have the feature of having a multi-directional wireless connectivity in this context if you have an embedded computational unit that powerfulit is easy to connect a touch display with visual information and more importantly you do not need an external computational unit such as a laptop traditionally, typically the most traditional setup for augmented instrument is where there is the sensors augmentation place on the instrument itself but then the computational unit is placed outside

VG — so you mean this connectivity is also a way I guess to update the instrument bank... 

LT — for instance this is a precisely the new line of research and I'm going to present a poster now( at audio mostly conference 2017, NdE) out the manifold interaction possibilities that are enabled by an instrument that has a sound engine inside, a system to deliver electronically generated sounds placed on the instrument itself this connectivity feature that allows people to join together with this instrument for instance and people for instance with a smartphone or a tablet can or even on a wireless keyboard can connect to the instrument play together with the instrument players that is playing the guitaror whatever other smart instrument and all the sounds are mixed and generated by the guitar itself so, this is a simple application

VG — maybe we're running out of time but maybe one last question I usually ask people ... a pronostic on what's important to them, what they feel is important to them in the field of digital musical instrument for the next ten years and how do they think this could change the way we make music in general

LT — oh this is a one million dollar question in the sense that ... mmm ... musicians in general are the most conservative people of all categories of human beings okay

LT — you think so ?

LT — for instance guitarists want the the sound of the Stratocaster of the fiftees and... and that's it. They are not open to anything else for instance most a musician nowadays are like that then there are the experimental musicians within those we should do analysis but I do believe however that many possibilities of producing sounds and spatialization and augmented instruments are somehow already being discovered there is a lot of research to do I agree but the core concept is not groundbreaking any more on that side in my humble opinion. What is the future in my opinion is the possibility of connecting things together okay so it's not by chance that we are living in this Internet of Things world now. This domain is more and more important in any level and also music is definitely affecting the by these tools, technologies but also behaviors people want to be connected anytime everywhere and with anyone and musicians are people and they, in my opinion they will feel more and more this need. okay? yeah, that sounds good, hope that I've answered to your questions

VG — yes, sure, thanks you. % ok 23/08/2017
	\chapter{Interview : Bruno Zamborlin}
\label{appendix:zamborlin}

\section*{Biographie}

\noindent Bruno Zamborlin (né en 1984) a effectué une thèse entre Goldsmith Univerity et l'IRCAM sur le sujet de l'appropriation des interfaces numériques pour la musique, avant de créer en 2013 une startup pour commercialiser les Mogees, un dispositif captant les vibrations pour ``transformer tout objet physique en instruments de musique''. En 2018, la startup Mogees est devenue ``HyperSurfaces'' et étend la technologie des Mogees à d'autres domaines que la musique, pour convertir des objets usuels en interfaces Homme-Machine via le traitement de leur vibration.

\noindent Site web HyperSurfaces : \url{https://www.hypersurfaces.com}

\section*{Transcript}

\noindent Bruno Zamborlin, interview du 23/08/2017, dans les bureaux de Mogees, Londres.

VG — first I wanted you to tell me about what, in the first plac, gave you the will to design your own instrument rather than choosing the piano or other ready-made instruments 

BZ — So I think the first time I thought of Mogees was... the idea that I had was really trying to overcome some of the limitations of music creation and specifically electronic music creations, something I've been always very jealous I was see my friends playing acoustic or electric instruments and being able to you know just improvise and just go to the beach or to the middle of the street and just improvise and just jam together, you know, very easy, very spontaneous, very visual, very powerful... and then I was the guy you know behind the laptop [laughing] so was the kind that always had to prepare for ages actually to to you know to load my samples and load my setup and actually always had to think about what to play beforehand, I never had that spontaneity that they had. So I wanted to come up with some tool that actually at the very beginning was for myself. Some tool that could actually enabled me to be so spontaneous with electronic music as they were with a guitar or a drum kit. Is that one minute yes? 

VG — So basically you wanted mobility? Is that what you mean ?

BZ — I wanted spontaneous... yes I wanted something that was simple for me to carry on, that I could fit in my pocket and at the same time I wanted something that was flexible that allowed me that a variety of different sounds so MIDI was very good because I you know this idea of actually controlling multiple sound sources was was very interesting for me and that, you know, I experimented with cameras and with other types of sensors but I never managed actually to find something it was actually easy to set up and plug and play... then I had this idea of actually using the sound of physical objects so actually using existing physical objects as a ... as an interface and physical objects are great because they are surprising because you don't carry them with you just find things around you and after you try actually playing Mogees with different objects more and more you you find that there are actually commonalities, common patterns... so every time you play on a new table or a new tree or a new bike or a thing, you're always like, you know, looking at easier and easier actually, and I really tried to design Mogees so it could be as customizable as possible. So you know there is more than 15,000 pieces, like there are 15,000 people that use it in a completely different way one from the other and I love this concept of reusing the skills that someone has so... if, you know, dancers for example that use Mogees, they... they use like the skills of dancing, you know, for example they stick it on the table they jump on a table and they starts actually you know stepping on the table with their feet and there and they are triggering different sounds... I really like this idea of like okay I am a dancer and I'm free using the skills that I already acquired what something completely different to make music. The same thing happened with all the guitar players that actually stick a Mogees on the body of their guitar and then you know they used it they kind of extend the capabilities of the guitars, they're still playing the guitar but they add some new gestures... for example to trigger a pattern or whatever and they are reusing that skills to play Mogees as well ... I love that variety, you know, that flexibility. And you can see that with artists but you can see that even with with kids so every time I see kids like you know playing it I am amazed by the curiosity and the ideas that they have, like, one time, like, they embedded like a Mogees inside a basketball like a spongy ball, with the phone inside and then it started actually play the ball or they would come up with any crazy idea, using toys or LEGO construction to create their own live-set, they do amazing things and like not be constrained by what you learned...

VG — You mentionned that you were ``the guy behind a laptop'', so I guess you were essentially using keyboard and mouse and maybe some MIDI interfaces to play music ?

BZ — Yes exactly, and I started more more to move to a mobile setup where I had like a couple of Mogees, a couple of phones, connected to a speaker and ... yes, that's it... that's so much more portable ... 

VG — but still that's a pretty radical shift ... you just abandonned all the buttons and knobs-style interfaces which are still the most common interfaces for playing electronic music ...

BZ — yes you're right, it was a shift yes... so at the beginning, I did it for me, you're right... and it was really like a big jump to letting go the more classic MIDI interfaces... I mean I still use them, of course, you know, they give extra much control, they're super useful in many situation but on the other side, for more... uh... in other musical situations, I just like to have nothing and just improvise completely and not having anything prepared beforehand ... so yes I didi it for me, and I did this jump and I started playing with a few Mogees... and then I think that was the first video that I made that got popular and many people asked me if they could actually get a Mogees and I said well, no I don't have any... it was just, you know, just for me ... and then I did the very first kickstarter project and that's when basically I raised the funds to make the first 2000 maybe or something... that was back in ... like few years ago... and that's when actually people started using it and that's when, of course like it always happens, I realized that that the instrument was simple for me but was not simple at all for other people... so I had to improve a lot the usability, do a lot of tests... and theses guys, the very first backers, they did help so much in ensuring that actually the instrument was usable by anyone ... and then, after that, we improved so much the usability 

VG — you mean the software interface ? 

BZ — both ... yes, well mostly software, mostly software ... but also ...

VG — because it looks to me like the object still looks like it was in the first place 

BZ —  yes...  the hardware yes ... we did like very minor tweaks ... but yes, hardware-wise yes it's pretty much exactly the same, yes, and all the work has been about the apps ... 

VG — can you describe a little bit how Mogees works ?

BZ — so the way it works is it's a piezo transducer that you stick on to the object you want to play and like every piezo transducer basically it transforms all the vibrations that you make when you touch this physical object into an electric signal ... like a microphone... and then this signal goes to your phone ... and in this phone, our app is running and our app has these algorithms that basically analyzes the signal and transform ... and try to understand in real-time how we are interacting with the physical object ... and it lets you program this physical objects so as to react to the sound you want, with the gestures that you want ... So you can train, say, this table and you can train this table so that it learns when you tap with your thumb or with your other thumb, or when you scratch... any gesture you want and then that's the machine learning, that actually learns all these gestures and then allows you basically to trigger different sounds every time you do those gestures ... It's like, if you want it's like if you were programming these physical objects so as to react exactly the way you want ... So it's one of the first instruments really that allows you to, you know, that kind of learns from you, so that doesn't impose you any particular behavior, you know, if you have a MIDI controller, you'll have to move that fader or that knob or push that button, you know... So the gesture you do are always the same...  so you can, I mean, if you want you can push the button with your nose, if you want to, but of course it's much easier to push it with your finger, and MIDI controller has been designed for your fingers ... you know ... While this table hasn't been designed for me to make music out of it, you know, so it's really like a hacking, in a way, it's like I am actually seeing this table and I think wow! ... what kind of music, what kind of interaction can I make with this table ? And it's up to me to make it happen, the app is just a tool to fulfill what I have in mind. So that's why, for me, what we made is not a musical instrument at all... It's a tool to allow you to make your own instruments 

VG — mm... that's a good pitch... One of the specifics of DMIs is the fact that they are programmable, and that's a very interesting point that you raised, that you somehow program the objects ... which is, in a funny way, a kind of twist on the Internet of things somehow... because we usually see digital objects replacing the usual everyday ones and this turns the usual object into a digital interface... 

BZ — yes!

VG — with this instrument, you offer people the possibility to plug anything on anything somehow ... which sounds like softwares that we use, like Max or whatever, that allow you to plug anything on anything... and there are a lot of different strategies for the mapping between interfaces and synthesis ... and so, if I understood well, the app is containing instruments that you can load and the choose your synthesis ?

BZ — in terms of sound, you mean ? So .. the way you can see it... in technical terms, you know in Mogees, you have the analysis and the synthesis, right ? So the analysis is on the part that actually learns your gestures and it triggers a... a MIDI event, if you want... then we have our own internal synthesizers, that we made, so there are couples that can triggers old-school drum machine sounds, then we have a lot of physical model syntheses, that are our famous ones that you may have seen on our videos, thoses are really the more ... uh... peculiar ones, because you can actually use them only with Mogees, because the idea of these physical models is a physical model that is actually excited by the real sound that you make when you interact with the physical objects ... You have really continuous interaction, like a scratch, like a physical scratch will actually sound like a scratch ... like you're scratching a string of a piano, or the string of a violin... but yes, you can program it so you can associate for example the model of a, I mean, you know, it's not a real physical model, but you can imagine, like, if you were to associate, like, the string of violin to this table, so this table would sound like a violin... 

VG — this acts like a filter, in the general sense of it...

BZ — yes... it's a synthesizer... 
 
VG — since you can plug anything on anything, and you have this particular thing with digital musical instrument that the same gesture, depending on what virtual stuff you loaded on the computer, your phone, or whatever, the same gesture will have totally different behavior on sound. I mean, not only sound but but also the behavior .... maybe that's a little less true for Mogees, since you are very close to the material, but I mean if you take a MIDI interface, turning a button could as much change the frequency of cutoff filter, as scrolling in a sound Bank for example, which are just mentally totally unrelated actions...

BZ — yes, there is no connection between the controller and the effect that the controller is having on the system, yes. In Mogees, so... it's up to you as well, so you define the mapping, so it's similar in that way but the analysis and the synthesis, if you want, like the gesture, and the effect that it has on the system are more tied together than in a normal MIDI controller, because... all these synthesizers that we made, that are excited by the real sound of the object, are definitely connected with the gesture that you do ... So, you know, we allow this freedom to users, the user can decide to just trigger a sample that has nothing to do with this table, so if you want to disconnect completely the gesture and the sound, he can do that. It's very useful sometimes, you know. But ... you're right when saying that actually you also have this possibility to trigger sounds that are strictly dependent on the gesture they make. 

VG — on the gesture energy somehow ...

BZ — Yes, the sound really, yes ... it's like, you know, we are exciting the physical model with real sound that is picked up by the microphone. 

VG — another thing that is specific to digital musical instruments is that you can embed memory, knowledge, into the body of the instrument itself. This could be sound samples but this could be also rules of musical composition like scales or whatever ... and how do you see, as far as Mogees is concerned, this part of musical science that you can embed into the instrument ? 

BZ — well, a little bit less than in other instruments because I expose the mapping to the user, so a little bit less than, you know, a ReacTable, or other instruments that actually have a behavior that is very musical, in a way. In Mogees, I like the user to define this behavior. But... there are course constraints like for example you have to tap on object, right? So that's already like a very specific behavior that I'm imposing ... so it is a percussion, a percussive instrument in a way, I'm actually constraining the platter of actions that you do, because you have to tap on objects. So the kind of way I am implicitly pushing someone to use it is percussive... Anyway... we can define that as a behavior or a constraint, if you want... In terms of musical composition, no, because you trigger the sounds you want... So there's no.... it's quite free, I think

VG — yes... I guess that is more about sound design, somehow, that you embed in Mogees example and the composition is...

BZ — ... is up to you... I'm not defining any scale, for example, or any ...

VG — I had in mind this video where you tap the table and the note would change automatically ...

BZ — yes, we also made an app for kids where you preload a MIDI file, and then every time you tap, you just step through a MIDI sequence... right? I would... struggle... to call that a musical instrument... because for me it's more like a musical tool to explore the sensation of playing ... a piano piece ... you know? ... I find it fascinating from a pedagogical point of view, in the sense that, you know, teachers are really excited about that app, because it allows basically a kid  is creating music for the first time or is playing music first time, actually, to focus on the aspects, like the tempo and the accents, that ... usually you focus on a very ... like, after a few years you're actually playing piano... and at the same time, they know this is always right, because it's actually pre-recorded in the app... so you don't have to learn other more difficult skills, like, you know, how to sit, and the coordination between the two hands and where all the notes are... So it's a way actually to stimulating at you ... stimulating the creativity of the kids, in terms of tempo and accent and personalization of the playing music. For me is really like a an exercise of active listening, between listening to music and playing a piece. It is a little bit in between of these two worlds. I don't know if people call it a musical instrument or not, I'm usually not interested in definitions, anyone has his own definitions but it's ... yes... I'm not into that...

VG — among the ... 5000 users, you said ?

BZ — 16000 users

VG — 16000 users, oh my god

BZ — between ``pro'' and ``play'', we have two products, like, ``pro'' is more targeting musicians and artists and ``play'' is more for kids and gamers... 

VG — and how much do you estimate the part of ``pro'' versus ``play'' users? 

BZ — uh... half and half roughly... very roughly...

VG — and amateur playing, is it somehow more related to music education ?

BZ —  yes, well we have four apps. If you buy Mogees Pro, you have all the apps, if you buy Mogees Play, you have the three games, but you don't have the music creation app which is the more complex one

VG — which lets you define the mappings...

BZ — yes... exactly...

VG — and regarding these two poles of amateur versus pro practices, one of the girl from the Bela-platform team, was mentionning that they try to bridge the gap between amateurs and experts, well, in the quite expert field of programming your own embedded low-latency device...

BZ — yes, this one is quite different that mine, leaning toward the experts, yes (rires) and I couldn't be onto [Bela] (rires)

VG —  still, you find people that would like to make their instruments and are interested in the musical side of it and would not want to learn assembly code... 

BZ — of course, it's very important...

VG — one need that we encouter with digital tools nowadays, is to be able to start using a tool as an amateur and explore further on later... keep the same technological tool and reach expert practice after practicing the amateur version... starting from learning the basics, but still being able to discover more advanced functions 

BZ — hacking the system you mean ?... with something simple that has constraints but that can be appropriated by the user... right ?

VG — that an amateur instrument can become an expert instrument, in short.

BZ — yes, oh absolutely...  yes so that's what I think... yes there's like a study, you know like one of the guys behind Bela was also my PhD examiner, Andrew McPherson, and Michael Gurevitch from the NIME community, like they started this very interesting concept that is the concept of constraints in musical instruments.... it's like, you know, a project you may have heard of is this one-button instrument ... so they said, okay let's try to design the instrument that has the highest level of constraint ever : button... just a button that has no velocity, makes just one simple sound, one beep, that's it. Let's give it to musicians and see what they come up with. And they notice that of course, because the instrument is so constrained, everyone was actually inventing their own techniques of playingn, with very very different ... to hands, to feet, to nose... and they tried to find ways to make something special, you know...  and I think it's a little bit what you say, it's... it's a very interesting approach, this classic, you know, ``low entry fee with no ceiling to virtuosity'', a motto where you try to make something simple that everybody can use but that also allows for some customization and someone to appropriate a little bit ... a little bit of rules and try to make it his own instrument. That's a fascinating concept, yes. With Mogees, it is almost like forced ... because, like, the instrument has no shape, has no rules, is just a sensor, yes? And you ... like the first action I ask everybody to do is : find an object, stick it (the Mogees, NdE) to that object and then, think about how you can play that object, you know, so in a way I kind of push in this behavior. Because I'm not telling anyone how to how to play it, you know, while if you have a piano, you know that you have to sit down and you got to play it in a certain way... so I kind of opened this out completely, you know...  the first time this can be quite terrifying, you know, like ... as a commercial product it can give a lot of promise because users are so used to be guided... used to being told what to do and I'm giving you like a blank sheet... I say okay you can do whatever you want and they're like ``for example what ?'' ... you know... like, they wanna know ... So we had to make a lot of videos and a lot of, like, ideas just to stimulate them a little bit because, you know, of course the experts ones have no problem, they were just uh yes walk in the streets finding crazy objects and start playing them, but others are more, like ... the need for some stimulus, some guidelines first... 

VG — yes... I was saying that another specific thing of this digital instrument making activity of the 21st century, is the existence of communities which are helping both instrument makers and instrument players.... I wanted to know, in your case, how do you feel that it helped you, in the first place, to design your instrument 

BZ — sure ... I think... music instrument designers always need to talk to the musicians and to actually have this constant feedback from the users, otherwise their job will be impossible. So it has always been the case, 100 years ago... now it just became easier, if you want, because we can read all the comments on the socials and we actually know what they think, in a more unbiased way, because we can actually watch that talking about your instrument between them... So that it's a definitely easier now ... In my case, Kickstarter helped so much, because you know, you have like a very direct way of actually engage with your community... But this is a thing that has been the case for ages, like, think about the first people who started actually appropriating the turntable, to use it as a musical instrument ... I thinkg that all the pioneer, all the manufactors of turntables were really interested, and saying ``oh my god, I'm seeing a new market there... let's try to make a turntable that was actually designed to make scratches, so we can sell more''. So like, all this feedback is crucial, we are always looking for endorsers that can actually become virtuoso of our instruments and show how amazing our instruments are, to the rest of the world.... We're also looking for very honest feedback from the musicians that actually try to do something with our instrument, they have suggestions and criticisms...

VG — you mean it's just the scale that is changing ?

BZ — the scale is changing, yes... well, you know, it depends... it depends... I mean, you ... in the past, you know, there were fewer manufacturers and these manufacturers were selling very huge numbers ... now there are many small, very small manufacturers, that have like more specific communities, if you want... so I don't know if the scale changes actually, I don't know ...   there's always been a lot of musicians, it's more like the now is easier to gather information, that's what I think... it's more like, musicians have always be keen to buy new things and talk about it, but we couldn't hear them... and now we can...  

VG — I was thinking of the community of musician but also of the community of makers. I mean, this technological objects need a lot of competence, knowledges from from various fields, like electronics, computer science, design ... So, how did you find your way to gather all of this knowledge and to come up with an instrument that is at the same time brand-new and ready-made ? Like the time-span between the idea that you have and the commercial product is pretty short, if you compare it to, say, the time needed for a luthier to produce one violin, a hundred years ago ...

BZ — sure, yes .... now is much quicker, yes...

VG — how did you connect with all these people who have this knowledge of design, of electronics ... Or did you do it on your own ? 

BZ — No, definitely not... In my case, yes, I started a startup, with a Kickstarter to raise funds ... Kickstarter is an amazing way of doing this, because you basically sell it first, you're raising the funds and then you can actually pay experts that can help you in all the areas that you have no idea... Like, I'm not a designer, I am not a graphist, I am not a manufacturer... so you need all these people to work with you... and you need funds for that and that's where the crowd-funding really helps... 

VG — but did you have enough knowledge to make the protoype by yourself ?

BZ — no I had enough knowledge to know ... that I couldn't (rires) ... 

VG — even during your PhD at IRCAM ?

BZ — yes, first prototype, yes... of course, yes... I made it myself, yes... that was basically, with Max/MSP and a few contact mics, you know, custom made contact mics... but then you know, from a proof of concept to a product that you can give to people, there are years of work... so much ... it's huge

VG — maybe one last question to conclude this interview, I'd like to ask you a pronostic about something that you feel will be important in the next 10 years, not necessarily ``The'' future, but something that you feel is developping and interesting at some point, in terms of instrument making, or that you feel related to instrument making...

BZ — musical instruments are more and more defining music genders... I think... sorry... let me correct it, what I just said... I think new interfaces for musical expression are more and more versatile and can enable someone to make very diverse music genders... the opposite of, you know, 100 years ago, the piano plays piano-music, the flute plays flute-music and now this is so different... I think there's gonna be more and more people that will want to act of music creation by music creation ... I guess these boundaries will blur with other fields like gaming, education... I can see a huge disruption coming in the world of education towards a much more personnalized, ad-hoc, free, gamified way of teaching to kids ... much less standardized like it is now where we are just, you know, the huge classrooms and we all have to do the same thing, at the same pace, with the same program... That is going to disappear... And because of that, because of that freedom, maybe we will enable the next generation to enhance their creativity much more and I hope music will be a very good test-bed to stimulate our creativity, that can then be applied to many many other fields in our lives ... % en cours 23/08/2017
	\chapter{Interview : Nicolas Collins}
\label{appendix:collins}

\section*{Biographie}
New York born and raised, Nicolas Collins spent most of the 1990s in Europe, where he was Visiting Artistic Director of Stichting STEIM (Amsterdam), and a DAAD composer-in-residence in Berlin. He has beena Professor in the Department of Sound at the School of the Art Institute of Chicagosince 1999, and a Research Fellow at the Orpheus Institute (Ghent) since 2016.From 1997 -2017 he was Editor-in-Chief of the Leonardo Music Journal. An early adopter of microcomputers for live performance, Collins also makes use of homemade electronic circuitry and conventional acoustic instruments.  His book, Handmade Electronic Music –The Art of Hardware Hacking(Routledge), has influenced emerging electronic music worldwide. (source \url{www.nicolascollins.com})

\section*{Transcript}

Nicolas Collins, interview du 29/11/2018 à l'IRCAM.

VG — okay so I was saying that I make these interviews out of interest and curiosity for the fact that people making digital musical instruments or  playing them often have a very personal way of doing them, there's no not much tradition as compared to cello or whatever instrument... so the first question I ask usually is what in the first place led you to build your own instrument rather than using existing ones? what drove you to this weird activity?

NC — well that's a good question... This goes back before digital I have to say because I actually started out in music doing electronic music from very early on when I was still in high school, and by the time I got to university which was in 1972, there was a kind of a movement in America of homemade and handmade circuitry for music and the reason was that it was primarily economic which is that the electronic music equipment of the time, synthesizers, were too expensive for a person to buy. In other words studios bought them and pop stars bought them but a high school student couldn't buy a synthesizer. It wasn't like now where you can get a Casio, Yamaha synthesizer for, you know, less than you would pay for a trumpet, right? That was not the case in the 70s, so a lot of us started learning how to make circuits just for the reason of economy. But then a kind of a movement started about a kind of an alternative electronic music that was based not so much on using electronic sound to realize an existing vision but as David Tudor called it : "composing inside electronics" which is that we would make a circuit and then sort of figure out what the circuit did well and have the circuit as it were embed a certain amount of the score and structure of a piece as much as it would be the sound. So from a relatively young age for me building electronic devices was not a question of making an instrument like a violin but it was also about making a composition that the composition or the or the rules for the improvisation were built into that. So it was a little bit different from instrument building because the things you built were dedicated to a particular piece. As I say they essentially contain the essence of a work... so that is a little different from most of the history of musical instrument design because most musical instruments have been designed to be broadly useful, otherwise if you are in the business of making musical instruments, you will not succeed if you make an instrument that can only play one composition... yeah? Because only one person may buy it, right?... but if you're building your own instruments there was that character as I say that that was a... that was a sound device but it was also... let's call it a composition, for want of better term so that tradition was in my blood from as I say relatively young age like probably by the time I was 20 I was... I was thinking that way. So sometimes it's very impractical because it meant that, you know, you had to design one set of circuits to do one piece and something else to do another and if you're a traveling musician it means you have to travel with all this equipment instead of just having a flute and playing repertoire, right? So what happened is as you build up this sort of collection you begin to discover that... that, you know, this object and this object interact in a way that makes a third thing... You know, this was designed for one piece of music, this was designed for another and when I use the two together I get a third piece, you know? So sometimes we got a little bit... more than one application out of a given circuit or a given system and gradually I began to design things that were somewhat more like instruments in that I could use them in different pieces but it was kind of a backwards thing for me it started out with with circuits that were devoted to one piece and then they kind of broadened out a bit. So I, in the 80s, I basically designed two or three sort of things that you could think of as instruments it could be adapted to perform different pieces and the first were these instruments I called "backwards electric guitars" where they are guitars that have electro- magnetic resonators under each string so that you can play sound into the strings to vibrate the strings so that instead of strumming the guitar you used it as a signal processor so you could talk into it or sing into it you could play sounds into it and then by using your left hand you could change the filtering and the resonating of it... okay? Kind of thing you can do with digital filters these days but in those days it was a very unusual sound a little like shouting into a piano with the sustain pedal man but with able to change it... And I used those in several pieces both solos and small ensembles of these instruments... and then in the early 80s I got very interested in live sampling and signal processing doing live transformation of found sound material either coming from other musicians or I used radio very often, live radio, and I built a few systems and then I ended up building something that really was like a musical instrument and it was probably for your purposes the most relevant because it was a digital musical instrument and it was around 87 that I made this...1987... and I took a digital reverb an early digital reverb and I hacked into the operating system of the reverb so that I could drive various algorithms and processes that it did from a micro computer that I put inside the box actually a Commodore 64 of all things, and to control this I wanted something that was bigger than little knobs and sliders because it was at a time in the music scene in New York where kind of post punk music was a big thing and there was a lot of very visual action on stage, a lot of violent action, so \hl{I wanted something that was bigger, something that would make me visible and I was going around my loft saying: "I need a really big slide pot, I need to slide pot that's like this big" and I thought "oh a trombone trombone is a big slide pot"} stupid idea but um so what I did was I took what's called a optical shaft encoder it's like the data wheel on a rack mount device — it's half a mouse basically half a mouse — and I coupled it to the movement of the slide with a retractable dog leash. This was a very mechanical system. So that as I move to slide the knob turned and this was read by the computer so you always knew where you were. And then I put a small keypad on the slide with like 20 key-buttons that I could press. And I wrote a program. It was basically like a graphic interface without graphics where I would press this switch and I could click and drag the pitch, or the length of the sample, or the frequency of the cutoff filter and it was just like clicking and dragging on a computer screen only I didn't have to look at anything because I knew that this switch was always the button for pitch and this was always for something else yeah? 

VG — Very direct mapping ... 

NC — Very direct mapping of keystroke and mouse. That was it. It was essentially, you know this was early Macintosh days I was basically taking another road, another branch on the Macintosh interface. And very nice because of course you didn't have to look at a computer screen to perform so you could be free. And then what I did was I put a loudspeaker on the mouthpiece of the trombone, so that the sound could play back through the instrument. And what this meant was that I could aim the sound anywhere I wanted. Right? In other words it wasn't like fixed coming out of a speaker I could walk around anything. I had an acoustic presence on the stage, so if I was playing with another musician we had this acoustic identity together, even though it was an electronic instrument. When I moved the slide of course it filtered the sound, because it's like a resonant tube that we're changing the center frequency and I could use a mute to do kind of like a wah-wah filtering as well. So it had a very vivid acoustic presence. And what I did with the instrument was live sampling of other musicians and it was very fast I mean was instantaneous to go "pshht" and make a loop and do stuff, okay? So whereas I developed this like all of my hardware and software, I developed it for one specific composition, okay? I can send you the URL for it it's called "Tobabo Fonio" and it's processing of recordings from the peruvian altiplano of brass bands players. But I found that it was a very flexible instrument for working with improvising musicians. The thing is that this was a time when there was very little being done actually with live electronics and improvised music. I mean it was, yes amplifiers and maybe a few effect pedals, but very little and I would go out on stage with another musician and I would grab the first few sounds they made, maybe they were just tuning or, you know, checking the instrument. I would grab it and then I would do variations on it for three minutes. Speeding it up, slowing it down, make it go backwards, all the standard vocabulary of sampling — but live. And instantaneous. And it was very flexible, very playable and it had this quality of fitting in, in the context of acoustic instruments, where a lot of computer music and electronic music, where you try to combine acoustic instruments with electronics, there is always this very clear distinction between, you know, this electronic sound coming out of the speaker and there's this beautiful beautiful cello on the stage and they never quite get together. As it turned out, I also had a line output from this instrument, you know, that when I wanted to go loud, when I wanted something to be very Hi-Fi, I could, you know, move a button and send it to the speaker system. But the charm of it was, you know, doing essentially acoustic duets with people doing computer transform, which is a very odd idea and still is; almost nobody doing it. There was a guy talking at the conference yesterday about a piece using the same idea, but it's... you know, he had no idea I did it 30 years ago [laughter]. You know, he's just stumbled in right now. But it's rare to have that acoustic presence of computer sounds onstage. So that was... I think for your purposes, that was, you know, kind of a landmark computer music instrument — digital musical instrument, because it was, there was nothing like it industrially available. In other words if I had gone shopping and said I want an instrument for live sampling and signal processing I wouldn't have been able to buy one and much less would I have been able to buy one that had, you know, this sort of combination of being almost a musical instrument... So... I built it. 

VG — And do you think you would  have been able to sell it ? 

NC — One time, I worked very closely with the engineers of the company that designed the digital reverb — "Ursa Major" and they were a small company, I represented them through a job I had in New York I was friendly with the engineers when I ran into a problem I'd call them and they'd make a suggestion, they gave me documentation... One day they, a letter arrived in the mail from them, and they had passed on a letter that had arrived at their factory by some... it was sent to them by some kid in middle-school saying: "I've seen pictures of this digital trombone that was built using your circuitry, I'm a trombone player in a middle school band and I'd really like to get one of these instruments. Is it for sale?" So I have had *one* inquiry for a sale. 

VG — It's really spontaneous. 

NC — Yeah. 

VG — Without advertising for it?  

NC — Yeah, and it was written in hand on lined paper in a pencil it was like, litterally, the kid  was probably ten years old, right? So now, it ... er... again I mean maybe that's the flip side of my saying that you know these early instruments in my community were not just instruments they were compositions. And what that meant was that they were kind of limited, that, you know, \hl{I would build a circuit no one else would want this circuit the way somebody wanted a Moog, you know or wanted another a Theremin because the circuit was really not usable for lots of things. It was usable for what I wanted to do.} This instrument, for me, it was important because I could do lots of different things with it and at one point I considered commissioning other composers to write pieces for it but it was really... still, it was kind of too personal... it would never have been hugely popular. The electronic end of it was, in the sense that this was like a precursor of all of the looping pedals that were developed, say, I don't know, during the late 90s and noughts. Looping pedals are everywhere now. Well this was sort of like the first live looping-pedal, yeah?  and it was way before any of those things. There was only one thing that was at all like it, which I had worked with, which was this famous pedal of Electro-Harmonix called the "16 second delay" which was never meant as a looping pedal but people figured out how to use it as one, and that was... I bought one of the very first ones that was ever made and it was after working with that, that I went on to design this system, so... But yeah, I could have taken the practical part of this project and developed a commercial instrument, but I think I would have been way too early in the development of the aesthetic of electronics in popular music to have had many sales. And as a friend of mine once said "it never pays to be too early" and I've always been too early, you know, I've always been too early and you never get rich being too early, unless you file fundamental patents and, you know, what artist has ever had the time to do that? So yeah... so that was, as I say the closest thing to a ... to an instrument. I can give you the URL, I wrote a paper on this, that you can look at. And then, um ... around the same time I started hacking CD-players to turn CD-players into sort of sample manipulation devices. And I used those in several compositions... manipulating CD recordings of music as a way to stretch it out. So you know, that was instrumental in the sense that it, you know, it did this particular thing but I could use it in a number of pieces rather than the one piece. But it never had the same flexibility of application as the trombone instrument did. 

VG — What do you mean by "CD... "  er... how did you use it ? 

NC — How did I do it? Right... What I did was... I had this idea... Yasunao Tone, very important Japanese composer working in New York at the time had done these things where he damaged the CDs with scotch tape and he made this very beautiful glitch music. His work was based on manipulating the recording, the CD itself. And using a stock player. I was interested in modifying the player to do ... what I wanted to do was I wanted to do DJing with CDs before there were DJ players... all right... this was 87-88 and I wanted to be able to scratch CDs and it took me years to figure out how to do that but I did figure out how to do this thing where you could put it in pause and continue to hear the sound and it would be sort of a suspension of the sound and a loop... you hear it when you're when your disc gets dirty and sometimes it'll get caught in a loop and I got so I could control it so that I could sort of suspend a little loop and then move forward a little bit, and move forward a little bit, forward a little bit and one of the classic clichés of the avant-garde is "slow it down" you know... That's one of the things we do all the time, we slow things down. And so I got very interested in taking recordings primarily of early music like late Renaissance and early Baroque music by period ensembles and stretching out these performances and using that as a backdrop for live performance by instruments. And you have to understand that within five years, this kind of stuff you could do with... well, not within five years, maybe within eight years... you could do it on a computer but at that time you needed a dedicated DSP to do this kind of stuff or a sampler you couldn't do it with the CPU on your computer. So the CD manipulation was sort of halfway between making a piece for instrument and fixed media, like what we used to call "instrument and tape", you know, where you have a tape playing and the instrument play, and true interactive music like we would have by the end of the 90s where the computer would listen and respond and do stuff... before the computers became powerful enough to do that, this was a way of sort of having a manipulatable backing tape where you could kind of change how often it would change and everything like that... 

VG — There is something with what you describe for the trombone that strikes me is that the mapping that you made is really direct and simple... Well, simple — I don't know but very direct. And it is something that is very often said in the litterature about digital musical instruments that er... direct mappings don't work so well... 

NC — Ahaa... 

VG — That you can't map one fader to one sine wave or something. And... this is said more often than the opposite, at least. 

NC — Right, yeah. 

VG — And I was always a bit suspicious about that claim... 

NC — Well, it has to do with the nature of the interface. In other words the problem is that a mouse or a trackpad is really only good for you know one parameter at a time. Maybe two. It's very difficult to do multi-access control with a standard computer interface. So it means first of all you have to use an external interface. And thinking commercially, if you're making software for a computer you want it to run on any computer you don't want it to have to have a custom interface to work with because that will limit the number of sales you have because, you know... If they have a choice between two recording and editing softwares for a computer, and one will work immediately when you put it in and the other one says oh you have to spend 300 euro on this external controller or it won't work at all *but!* once you have it, then you have direct control. You would think that the direct control would be a better deal but people will always buy the cheap one, yeah? And I think that the development of computer tools has generally been for non-real-time  production... yeah? In other words, hard-disk based editing and workstations is what's called "nonlinear editing". In other words you don't have a strip of tape that's in sequence, you're jumping around you're doing everything. And if you don't care, if it's not really that important how fast you work, you don't need to have 28 faders for the 28 tracks because 90 percent of the time, you're only moving one fader at a time, so why pay for all those faders? So you know you run a mix and you say "oh you know I think that second track should have been a little quieter from 30 seconds to 33 seconds" ... So you go back and you fix it. In other words they have developed tools to get around the problem of direct control and save people a lot of money and investment in alternative technology to interface to the system. But performing music is different. Performing music is a real-time event. Yeah. You can't ask the audience to sit there while you go back and edit something. Do you know what I mean? And performance traditionally... most musical instruments are about direct control... right? And I think that .. you know, I was watching these videos yesterday of two projects here at IRCAM where, you know, they're using a graphics tablet whose advantage is that you can have multiple controllers on it and the other thing was with something working with a Kinect for visual tracking, which meant that you could you know track two points multiple axes, you know, these things are things that musicians are thinking about... and it's just that ... em... yeah again maybe I was just very early with the idea that what I wanted was to have instant access to all the parameters of the DSP without having to say navigate through menus or anything else like that. And, you know, for example your typical rackmount effect processor, like an Eventide harmonizer, you have one wheel and you have a few buttons and you have a hierarchy of menus, and it's designed so that the things you need to do most are button click and the wheel. And then the things that you don't have to do quite so often, you have to do two button clicks go to another menu and then use the wheel and they design it so that the deeper you go it's the stuff that's less often accessed. If they had 28 buttons on the front of the thing, a) it would cost more, from a hardware standpoint and b) it would be more confusing for most people because they don't want to control all those things at once. But I don't think you want to tell a pianist who's coming out on stage: "All right, look, we're gonna cover up a bunch of the keys because in this piece you're not gonna play them and then we'll take the covers off for the next piece and put them on something else. You won't need those keys, will you?" That isn't the way instruments work! 

VG — That's a good comparison. [laughter] But... er... I'm not sure it was meant this way in the litterature I was talking about... I was thinking of articles that were published in the NIME papers where it's dedicated  to mostly live instruments and for example, I don't know if it's a paper by Marcello Wanderley, or... there are a few of them actually  that say that if the mapping is too simple the result is not rich enough to be interesting... 

NC — Right, are these players ... writing these ? I mean you know it's... I deal with this all the time there's what we call a "sweet spot" in terms of the nature of mapping and curving and everything else like that and... in most conventional instruments it's been worked out over centuries, you know, of like, exactly what is the tuning like, what is the response, what is the touch everything else... We're in the, you know, when you're designing a new instrument, you know, you don't have that history, you don't have that big database of users who have said, you know, this works better than this, you know, I mean look at the evolution of say like, I don't know, the flute or the clarinet, you know, and you see all these like little alleys people went down... the "klappen-trumpet", you know, evolving into the trumpet, I mean it's like, all this weird stuff, but um... were you at the talk that I did yesterday morning? 

VG — Yeah 

NC — OK, so I showed a picture of the trombone and I also showed a picture of a new instrument that I've made based on a trumpet and I have these... unlike the trombone which had something like 28 switches on it, there's just, I think, there's just six buttons on this. And I went through a few different ideas about how to make it work as an instrument. And the first ideas were always too complicated because you know one button would change what the other buttons did, like a function button and this and that not... and it was... you have to kind of like think to make your way through it in a non-intuitive way and I ended up bringing it down to a much much simpler, a much much simpler set of mappings but I could only do that because I changed very much my idea about what the sort of sound property and the kind of nature of the interaction was. I decided to pull back and have less direct control, have more nuance of its own... yeah? and the other thing is that I was working with these sensors on the valves so that I could have basically three axes of continuous control at the same time. So with a little thinking I was able to have a much more direct mapping where there was less steering by buttons. And it just became a somewhat more fluid instrument. I don't ... it's a difficult thing to articulate but I think that if you get... if you give somebody too many choices of things to do, it will not be an expressive musical instrument because there's gonna be too much thinking. If there's too little, it'll be like a snare drum — or worse, a drum machine snare drum you know, which is all you can do is press the button and it always comes out exactly the same. You have to find... you know, you have to sort of find a spot in between... yeah? I mean think about,  for example, um...  think about if for an electric guitar instead of just plucking and fingering you had to move the fret to wherever you wanted it to be for that note, you know, like on a sitar, or you know, you had to put the chord into three different jacks depending on whether you wanted to send it to this amplifier, this amplifier, or the PA... You know, these things would  slow you down ... you know ... 

VG — Yeah, that's one of the drawback of digital stuffs...  one of the main drawback is this non-directness of tools ...

NC — Right 

VG — ... that you have to start the computer, launch the program, open you patch, recall the settings ...

NC — I wrote a paper ... 

VG — That reminds me of a friend's quote about how to make a good digital instrument design is when you are able to play it drunk... 

NC — Yeah! ... No, exactly... exactly... um I wrote a paper in the early 90s called "Exploded view" that \hl{was about how MIDI had done, was it had exploded the musical instrument that it used to be, that everything was integrated,} in other words that there was a string that was making sound and there was a finger board that was determining pitch and it was all combined into this one ... you know, like, like self-contained system and now with MIDI you know we have one instrument that's just a controller and the media is coming out and it's going to a thing that's generating sound and it gives you tremendous power on the one hand, you know,  because it means that if you have technique on one instrument you can play all these other sounds but at the same time there was none of the same feedback that you would have from a conventional instrument and I mean, again, you know, with the electric guitar ... the electric guitar lives in close proximity to the amplifier and every good electric guitarist knows how the guitar and the amplifier in the body interact to form a complete system... well... if you break it up into more separate pieces you know, a fingerboard, string-sensors, a string-synthesizer, a DSP, an amplifier... you tend to lose that clear-cut feedback network and you have to start designing haptic feedback into the system ...

VG — that makes me think of another question that I usually ask to people about ... I don't know how much it applies in your case, in your music but... many people using digital musical instruments, when they make a concert, they have several tracks or songs or whatever pieces, and ... um ... with those digital instruments, you have the ability with one click, to completely change sounds and mapping between the sensors, so you have to figure out your own way with this issue... and there is not only one way, I think, to go from one end to end of a concert, so... is it relevant in your case ? From what you told me, I tend to realize that you mostly use one self-contained object with... 

NC — yeah and very often my instruments have... have a very... em... have a very limited sound palette, yeah? In other words or they have what we might call a limited instrumental palette which is to say they produce a particular type of sound or a particular type of process and they don't do anything else so, for example, this...  this hacked CD-player only does one thing which is it does this sort of looping and drawing out of sound... now, if I put a baroque music CD in, obviously it's gonna sound different than if I put a heavy-metal CD in but the process will be the same, you know, the nature of the transformation... with the trombone it had a finite vocabulary of what it could do you know, it could could make a loop, slow it down, it could speed it up do various sort of multi-tap processes which were very beautiful in terms of changing the sound but it couldn't, for example... I don't know, break the signal down into constituent sine waves or, you know, do a Fourier transformation or vocoding or you know, track the pitches of the sample that I made and map them to an accordion sample, you know... no... there were lots of things it couldn't do but as a result as a performance instrument, it was very reliable, it was the kind of thing that you knew what you were gonna get, at a given moment. 

VG — That's somewhat of a design decision 

NC — That *is* a design decision... so for example, if you... 

VG — you could have make this instrument evolve, could you? 

NC — um, well... at that point uh... I was up against the very limit of what you could do with DSP. I don't think I could have gotten many more types of processing because nothing was available... so it was self limited but on the other hand it made for a very performable instrument you know, it's a little bit like if you're a guitar player and you're playing a six string guitar and you press a button and suddenly three more strings are added to the side of the neck, you have to think before you go, right? now if you're playing a keyboard and one time you have a sample of the piano, yeah?, and the next time it's a sample of a hammond organ... that's less... of a break but at the same time pianists have a very different touch on a keyboard then an organ player has... and to play a Hammond organ and to play a piano require thinking about your physical interaction with the keyboard and what kind of gesture will become what ...  and switching between the two ... musicians would talk about this in the old days when you'd have you know like a piano in an organ, or a piano in a synth or something, you know... they talk about, if you've got them talking about it, how they had to kind of like just think a little bit about the performance styles differences, even though it's still a keyboard, right?, so you have a MIDI keyboard connected to a number of different things and you know if you're playing, say, you know, a piano and the next you're trying to do a bass line and it's a different kind of articulation because you're trying to get pop bass or you're doing percussion samples for a piece, you're gonna have to bring a different technique to bear... So I think that there is the problem when an instrument gets too flexible, that you lose the ability to be virtuosic you know that it can stand in the way of attainment... Now you know, one of the people that I'm sure you've looked at in terms of digital musical instruments is \hl{Michael Waisvisz}... and Michael developed the first version of his "Hands" I think in the mid-1980s maybe 84-85, something like that, and continued to work with ,you know, subtle variations on it until his death... and after a few years, people said : "So, when are you going to design another instrument ?" and he said \hl{"I don't want to design another instrument,  I want to get better at playing this one."} yeah? ... um ... that was  an instrument that had a lot of sophisticated mapping and variable mapping because, you know, he didn't have a keyboard he could go all over, he had to limit his buttons to what he could fit within two hands on  ping-pong paddle type devices... he had these other factors of the tipping of the thing and how far apart they were spread but you know, he spent a long time thinking about what will I map to what, how much control do I want, how sensitive do I want it to be, you know... how far do I have to move it to effect a change... and then he spent years practicing and making pieces for it that, you know, each piece probably involved changing the instrument slightly, but always keeping a core structure that he was familiar with. 

VG — One of the feeling that I have regarding  this question of versatility of... and ephemerality of the thing that you have below your fingers is that it somehow... em... displace the things that you have  to learn and remember. I remember discussing this issue with people about why there is no or very few instruments in this digital realm that last in time more than 10 years apart from the keyboard... which has this piano-cousin... there are many, many (instruments) coming out on the market which fail to succeed ... and... so this question of what you learn when you learn  an instrument which is a set of many things, you learn gestures, you learn the timbre of your instrument the sensibility of your sensors and I was discussing this with another friend who is musician and he uses many different types of interfaces, he was telling me that, somehow, processes that he used under the various interfaces, he got to know them, that ... like if you play with FM-synthesis you can play with a joystick, or with a FSR sensor or whatever, that would make lot of changes,  and some things you will have to re-learn, but there is a common thing to them... that you know that the FM-synthesis will be that shaky at some settings and that stable at some other settings, so you kind of know the sound  generator as an object even if its a purely virtual algorithm... this kind of mental switch  between virtual instruments that you may use in the time of the performance and ... I don't know at all how Michael Waisvisz used... because he had several instruments, like, he switches between instruments... 

NC — towards the end of his life he developed... well, he developed this... a variation on the Hands that was a conducting instrument that was sort of a stripped-down version of the Hands;  fewer direct controls and ... then he did something that was based on the idea of a spiderweb with four sensors on the threads, it looked a little bit like a funny harp and... I don't think he ever made terribly effective use of that as an instrument... um... there's an expression in American-English maybe even English-English, from the music industry called "the second album syndrome" that for most pop bands the second album is always a disastrously bad record and the first one can be a magnificent success and  the second one is inevitably terrible... and it's called the curse of the second album  and it's like everybody has to get through it and if the band is lucky they go on to make a third record which is a little bit better but it takes them years to be as good as their first and I think when you design instruments, I think most inventors will probably tell you the same thing, you know, whether they're designing toasters or you know something else, that their second invention is probably a dog, you know, and I think that Michael stuck to his idea about becoming a virtuoso on this instrument but then at a certain point said "okay, you know, I now know how to play this instrument, I'm really good, I'd like to do some more experiments you know he was at STEIM when all these crazy controllers were being made so he sort of would look at what's going on and said "you know that's kind of interesting, I don't have that in the Hands, that would be something interesting"... so you know that's the motivation... why the instruments don't survive is a very interesting question, okay, and it will be interesting to look in a hundred years to see, for example, are people still playing the Theremin? which is probably the oldest electronic instrument that's still in production and has a base of users. It's not as popular as the saxophone but the average you know, educated music person when you say Theremin they'll say "oh yeah that thing you wave your hands around" whereas if you say "Nick's, you know, trombone propelled electronics" they'll say "what?" you know, it's like... no!... it's like a completely different world... you know, I think there have been these drum pad controllers that I think are kind of a sort of like, they are to drums what a MIDI keyboard is to the piano, you know, in other words they're so closely related it's you can barely call them separate instruments and a lot of the drum controllers now basically use drums and just put sensors on the heads you know, because people don't want to play some weird rubbery thing... The guitar controllers have all been miserable failures, you know, none of them seem to work, even though potentially it's a huge market... huge market, you know, if you could get one that worked well, you get a lot of people buying it because electric guitars are still you know the hugely popular instrument wind controllers, you know, brass and wind controllers very small market for them and again they are  to the brass instruments and the wind instruments what the MIDI keyboard is to the piano, that is to say, they only succeed because they're very close to the instrument they model, but you know, I mean you've probably looked at the instruments that Don Buchla invented, the Lightning which is his conducting instrument, the Thunder which was a sort of a very very touch sensitive control pad system... I have no idea how many of those he sold... not a lot... they were you know, the conducting instrument bore a resemblance to conducting but was really you know, quite different; the control surface was very different from any existing keyboard or drum pad... not a huge market for it I think that the ... kind of proliferation of affordable electronic instruments, synthesizers, in the 80s was followed so quickly by sequencing and ... digital audio workstation softwares that I think that we now have a split in a sort of ... economy of music where, in certain worlds most notably classical music there is still a lot of performance taking place, you know, symphony orchestra still exists, opera companies exist, string quartets exist and tour, pianists tour... more and more pop music is a studio based practice and when you're producing music in a studio you don't worry so much about the instruments, you know, in other words there's a lot of non-linear music being produced now where you know, the drummer is not drumming, you know... they're not putting down their drum tracks by hitting buttons in real time, you know, everything is being done in an editing, oriented fashion... um... an awful lot of the work is removed from the traditional idea of performance and I think what it meant was that just at the time that you might have seen a proliferation of new electronic instruments, the need for instruments sort of went down, you know, because ... from a practical standpoint it's ... it's easier to perfect that music in a non real-time way, you know,  it's easier to do those drum tracks with editing than it is by hitting drums unless you're a great drummer... yeah? ... even vocal performances now as you know are so highly edited and processed that you know we're really only this far off from having it be essentially a sample based technology and techno it's been that way for years...  it's just little shouts and hits you know... 

VG — aligned text-to-speech.... 

NC — Yeah, or it's somebody singing once. 

VG — There's a whole scene in Japan ...

NC — Oh yeah I know ...

VG — ... with stars and fans, so yeah, it exists but you're right that there is a real cut between  electronic, or not even electronic, music production and ... "instruments" or so called instruments 

NC — oh yeah and so for example I think IRCAM is an interesting place in which to be asking these questions because historically it's had a very high level of institutional support for extremely serious live performance of music incorporating electronics and looking to get beyond the kind of excellent European model of live instrumental performance against a backing recording and you know, starting with the 4X, clearly a huge amount of effort has taken place in IRCAM to get this stuff working. Now as it turns out, and I'm sorry to say this from within the hallowed halls the advances that have been made in the commercial music industry and other research facilities like STEIM for example have taken place much much faster than the work here at IRCAM. In other words, outside IRCAM we're miles ahead of what's being done here but it's being done for rather different communities and I think that the sort of the community of kind a more academic classical -contemporary classical- composition is going a little more slowly and perhaps more methodically  at looking at this 

VG — more conservative ...

NC — yeah ... I think it's a question of ... people want a kind of a certain reliability and proof to stuff I think there's less interested in experimenting with the technology per se they want the end product whereas outside you have two things : number one in commercial pop music the economic rewards are so high because the sheer number of buyers out there that you know stuff is being pushed out much faster... the development of these things just takes place much faster ... um ... and then you have a bigger user base that shows you all the variations and all the bugs and everything else like that so stuff gets very very ... 

VG — there's a real community 

NC — yeah... it is a lot of community support, I mean, you read you know reviews of looping pedals and within four months another company has come out with a pedal that responds to the critique of the previous one, you know, I mean, this is very very quick moving but .... you know it's... I think one of the questions that you have to face doing your work is what the role of the instrument is in different musical communities because it's very different here inside IRCAM for very similar technology as it would be in a techno studio in Cologne, on the stage of CBGB in New York if it were still alive, you know, at the Kitchen in New York in other words, find like six different venues around the world that are presenting music of some sort using electronics and think about what is the consistency of how we view a musical instrument in those different in those different places and as I say I think non-linear music production has just changed the equation hugely... 

VG — yeah, that's for sure... to me this is obviously a big topic in my research, that the frontiers, the borders, of what we call, what we may call, a digital musical instrument are very blurred,  and should we call an iPod... to paraphrase John Cage asking if a truck passing in the street is more of a musical event if it is listened in a conservatory of music... is downloading MP3 a musical activity ?... so the number of situations we are facing, not necessarily listening or participating, but dealing with music has increased so much, the number of devices which allow us to interact with music has also increased so much that there is a diversity of situations which is just non-quantifiable ... 

NC — and also there are very different set of problems emerging, so for example, I'll give you a few examples, I started teaching for the first time in my life in 99, I'd been out of university for 20 years and I took this job at this art school, School of the Art Institute, in Chicago.  These were artists, art students, not music students but they were working with sound, a lot of them chose to come to Chicago to go to school because Chicago is a very important city for music, for a large number of communities. It's the birthplace of a lot of african-american music forms in America every post rural blues music form in the african-american music culture is essentially originated in Chicago, or has a strong connection to Chicago from electric blues through various stages of so called jazz music... 

VG — to techno... 

NC — ... to techno and that's the other thing very big, especially at the end of the 90s for techno house garage and this funny sort of indie pop music the band "Tortoise" was a sort of classic example of record label called Thrill Jockey... "Clicks and cuts" was a Chicago invention I mean all of these these things... so... the kids come there because they like music even if they're only amateur musicians I have this seminar, undergraduate composition seminar, and there are people working on soundtracks for films, there are people doing sound sculpture installation and soundtracks for performance, some are doing essentially contemporary music composition though they don't call it that and there's a bunch of house producers ... and what's interesting is that most of them can produce a reasonably competent backing track. That is their rhythm construction and their basic bass lines and jabs are solid. They have no problem. Why? Because the software for doing that is by that time very very good, you know, your looping and editing software for doing rhythm based metronomic dance music has reached the point that you don't really have to have a lot of skill to do it you have to use the same editing skills you would use for editing a video to do this you know we know that has nothing to do with music per se or needless to say it has nothing to do with soul ... right? On the other hand,  not one of them could write a hook for the singer. Not one of them could write a three note melody that you would want to listen to more than once. And I went nuts because after like listening to 20 of these things I finally just got... I yelled at them I said "if you cannot write an interesting melody, don't put one into the song" and it wasn't like I was expecting them to be Schubert I would play them techno that had a three note hook that worked and I would play the mere track that had a three note hook that didn't work and I'd say "why can't you do that?" you know what they said ? "we need a class on how to write hooks" so I had to hire someone to write a class on how to write hooks and the reason was that that was like an old-fashioned musical idea that was not part of their background, you know, if I was doing the same class at a conservatory, maybe they would be rhythmically incompetent but would be able to write the hook, I don't know! All I know is that there was a displacement between the way the technology could speed them up in certain aspects of what they were doing and not help in other ones. You know what I mean ? 

VG — too specialized... 

NC — yeah... and then it was interesting because of course then we would look deeper into what they would do and you know they would play their track and I'd say "bring in your favorite track at the moment" and then we'd sit down and we'd analyze the two and I'd say "I want you, in your track, to tell me where the cowbell falls" and they'd say "it's always on this beat and  this beat" and I'd say "all right now listen to the track you love and tell me where the cowbell falls" and they say "oh! it moves..." aaah...it moves...  In other words you start to rely upon the mechanism of these tools and your music goes in a particular direction because it answers questions for you, it makes decisions for you. So,you know, each one of these advances brings one in the same time kind of liberation and power and the other hand it kind of suppresses certain decision-making processes. 

VG — yeah, thats one global issue about those digital tools, about technology... Stiegler, the philosopher, deals with that issue that we are ... being handicapped... we are becoming handicapped and the fun thing with disability we are all facing is ... I've been working on a number of projects which were dedicated to the accessibility of blind or deaf people  for music and the solutions that we find to improve the access they have to music also work for non-disabled people... You can always ease the work that you have to do to to reach a goal, like for example, I remember this MIT team which has developed glasses which would transform color into sound for color-blind people but actually the sensors are able to detect ultraviolet and infrared so that this person now sees more than non-disabled people can see... so the limit... the difference between enhancing the body and the... where is the limit to a normal body ? oh I wanted to say something about that but ... what I wanted to say had to do with with... yeah ... I wanted say that you develop things to reach a particular goal — yeah — these tools but the thing with music is that the goal is always somewhere else somehow... you said it somehow relate to sex in this irrational thing that there is no such goal for perfect music or perfect sex,  and it always move to somewhere else to a new place, to unexplored places... So all the tools that we might develop to ease the task of making music is always failing at producing innovative... 

NC — well, yes and no, I mean I think that kind of post post electronic music has made very interesting misuse of Technology you know, that that is sort of, I mean you know, everything from the 808 in techno to a certain type of samplers and synthesizers by being pushed to the extremes of their behavior have become signature sounds in particular styles of music ...um... the guitar amp in an over-driven state is not something that the engineers who designed the early guitar amps wanted to have happen and yet it became the signature sound of electric guitar. Look at the prepared piano in classical music you know the early developers of the of the pianoforte would never have anticipated anybody doing anything like that to the instrument, it's you know, I think a lot of people still think it's horrible misuse of the instrument and yet you know, those sort of adaptations have been powerful liberating forces and in some cases have led to an entirely, you know, it's an entire new genre of music yeah right, at the same time you know there there are drawbacks a lot of musicians will talk about auto-tune and you know what it's done to pop singing you know, as it generates an artifact you know, it generates an artifact there's no question, and yet people will accept that with the interest of sort of having better pitch there's beautiful research has been done on the impact of recording on performance style and how you know, you have pianists who had a very long career and their career spanned from before recording to two or three stages of recording and you have them heard on recordings that were made, say, 20 years apart from each other over a 60-year period and there's been this comment made that they became more conservative in their performance style after the advent of recording because of course before recording nobody remembered your mistakes if you made a mistake in a concert maybe three people in the audience would know and the critic might comment on it and there are lots of pianists for example it were famous for hitting wrong notes all the time in performance you can't get away with that in a recording studio because it's gonna keep coming back people will hear that note again and again and again ... so what happened is that you had less rubato, you had passages being played slower than they would be on stage and this and that... adapting to the technology ... so yeah you know, at the same time this is the technology that brought music to the masses and brought profit to the musician you know, so it's not a... it's it's not a good thing or a bad thing it's a mixed thing and I mean this is a little bit off the topic of what we were talking about, I mean initially we were you know, trying to adjust sort of the more overt technical issues of what makes a digital instrument, what it is, you know, how do you limit it, how do you facilitate performance and how to be flexible but you know in the end, I suppose you have to see all of these parts in terms of the bigger picture and the bigger picture is you know, sadly it's the economy of music, and that can happen in a number of ways, number of levels it can simply be you know literally how many records will you sell if you use this technology versus that technology, yeah, or by using this technology will my music change in a way that defines a big thing that people love or something that people hate, yeah, in other words every bifurcation in the evolution of dance music in the last 20 or 30 years you know, you see one spin-off that dies and another one that flourishes okay, and then in non-commercial music it has to do with you know, the way you value your own work in other words I, as a composer, I work at a piece, I perform a piece some pieces bring me greater pleasure the act of presenting them than others and I look at it and I say well why is that ? you know, is it the sound world ? is it the nature of the performance experience ? is it the nature of the audience response ? sometimes my public and me are in agreement about what is good yeah, and sometimes we're not sometimes I'm convinced that this is really a good piece and the audience just doesn't get it okay, and other times it's the opposite I'm thinking "that's the one they liked??" you know, that's kind of the stupidest thing I did all night and that's the one they remember... everything seems to have that sort of like economic trade-off in the in the larger sense of the term 

VG — Yeah... maybe I should ask one last question, I think we are late already, yeah 30min late, so, I usually ask people a tricky question about making a pronostic or a vision of... 

NC — ... the future ? [sigh] 

VG — ... but you made it the whole time  during the discussion so... 

NC — I think that... yeah..  I really do think... that... we're in a very odd position at the moment in terms of this sort of the evolution of music technology, which is on the one hand I'm quite serious when I say that you know more and more music's going to become non performance-based and therefore all of this obsession about instruments is sort of gonna disappear in a large amount of the community, people aren't gonna worry about it, stuff's gonna become very generic, yeah, in other words it's like, pop bands in the 1960s didn't worry about the instruments; yeah, guitar, bass, drums, keyboard, yeah, boom! that's fine, we can do it! synthesizers in the 80s everybody was "oh are you using this using this, using this..." I think that's gonna kind of disappear, except for you know, a very small specialized sector of the population that's interested more in live performance but live performance is going to be less and less and less of our music world... I think that the advances in technology are not going to be in instrumental performance but in non-linear work those tools that improve and speed up the production of music on a computer, for distribution as a sound file and whatever it takes to make that work better just the way people are working on making better search engines, or you know, better word processor, better spreadsheet programs, it's gonna be in that market... it's not going to be in the market of Stradivarius and and Leo Fender... yeah? at the same time when I was at STEIM we had the sensor lab and this was essentially the first Arduino, yeah? and it cost three thousand guilders, that's sixteen hundred euro and you had to kind of work with the STEIM engineer for a few weeks to figure out how to make an interface with it... yeah? very robust, very reliable, but ridiculously expensive. Arduino comes out, not only is it cheap, you don't even have to know how to program to use it. Why? you know I have an art student,  never programmed before gets an Arduino says "I want to control the speed of a motor" they type "motor speed control Arduino" into the search engine, they copy, they paste, they download, it works. And maybe you know, they have to change the speed and they post a question "how do I change speed on Arduino?" they get an answer, they do it. In other words you have this very very easy entry point. \hl{This place [IRCAM] is a little odd because it still follows this old model of the engineer does that not the composer you know, but... [whispering: In the rest of the world the composer does it!]} ok? and what it means is that at the same time \hl{if you are doing live performance if you do need a specialized instrument it's ... it's almost more like cooking than it is like building a musical instrument you know, everybody cooks ! you don't think "oh did you go to chef school ? you made me a spaghetti bolognese... did you go to the cordon bleu school? - no man I just cook you know, I got to eat, I got cook."}  Then I think that you're gonna have much more sort of low level,  low pain construction of alternate interfaces and some of them may be  very simple it may just be a button, a pressure pad... some people  may build up something you know, like that looks like a damn sitar you know, but it's a very fluid thing and it doesn't necessitate an institution. It doesn't need IRCAM and it doesn't even need STEIM anymore, you know, STEIM was always like the budget-IRCAM; we did 60 projects a year when I was artistic director... you know, you don't do that at IRCAM. We did that at STEIM. No, it's not really necessary you don't need to come and work with engineers to make this stuff happen, okay? but here's the other thing I was gonna mention when I talked about the problem of my techno producers not being able to write a hook; you know, if like where did ... where are these weird problems coming from ? You know my students tell me \hl{the biggest problem they have in music production is finding their samples on their hard drive.} They say "I have so many kick drums on my Drive, that I've downloaded from so many places, that I can never find the one I want. Now that's really weird. That's like the first time you ever look at a harp and you say "how do you know which string is which ?" you know, you got so many strings, right?... \hl{Guitar! That's why people like guitar and bass... six strings, four strings, I can deal with that.} Harp?... Phew! You know... who would have thought that that would become the biggest musical problem ? ... not staying in tune, yeah, so now... 

VG — it's the library of Jorge-Luis Borges,  where people are in this infinite library containing all the possible  books in the world... 

NC — Exactly!... Exactly, it's the digital version, it's Borges' light, you know which is that... who'd have thought that, you know,  now maybe it means that you don't need Stradivarius you need a database programmer, that's the most important thing in your life would be like, a really brilliant database that would allow you to intuitively retrieve whatever sample you wanted... yeah?... but that's not an instrumental  idea right? In other words that's in such a different domain but it may be, as I say, the single most important thing for a composer working today ... Funny idea, huh? [tacet] 

VG — Okay. Should we go to the opening ... 

NC — Go hack ? Let's go see what's there! 
 % ok 10/11/2017
	\chapter{Interview : Adrien Mamou-Mani}

\section*{Biographie}

\section*{Transcript}

VG — Qu'est ce qui à l'origine a motivé cette idée qu'on pourrait dire farfelue de vouloir faire de la musique avec des instruments avec des outils numériques et des ordinateurs plutôt que de prendre un instrument existants... qu'est ce qui a motivé sa en premier lieu ?


AM : il y a 2 aspects complètement différents qui m'ont motivés. Le 1er aspect est qqch qui vient de la recherche, c'est peut etre un peu original, cela n'a pas été au départ une problématique musicale, mais née de mon histoire de chercheur sur la physqiue des instruments de musique. Ma thèse était sur la vibration des tables d'harmonie et j'arrivai bien à analayser comment ça vibre et comment le geste de lutherie (des luthiers) vient influencer certaines particularités vibratoires. J'ai pas mal travaillé avec des luthiers sur la relation entre leurs techniques de fabrication et le résultat vibratoire des instruments.
Et donc le 1er aspect qui m'a motivé ensuite à aller commencer à convevoir des choses comme ça c'était un peu dire "est ce que ces choses là qu'eux font par leur savoir faire, est que par de l'électronique on est capable de la même chose. C'est à dire que moi, sans être luthier, est ce que je suis capable de faire comme eux. Ca ça a été une motivation importante dans la conception. Et en fait il y avait tout un champ, il y avait Charles Besnainou qui avait déjà commencer à travailler la dessus, Steven Grifin (?) à CalTech. 
Je me suis rendu compte qu'il y avait des gens qui avaient déjà essayé de faire ça avec, d'avoir une approche dison d'accousticien qui veut essayer de comprendre comment la chose fonctionne mais qui va aborder son sujet d'étude par un autre biais qui était de l'électronique au lieu d'être de la mécanique si tu veux. 
Donc ça, ça a été un truc qui m'a amené à faire ça. Et deuxième aspect ...

VG : Juste pour le 1er aspect, pour poursuivre ce que tu disais, quand tu dis "aborder ce point de vue de mode résonance de table d'harmonie avec de l'électronique, c'est (ou ce n'est pas?) une envie de modéliser la table d'harmonie via des méthodes numériques

AM : Non non non, moi ce qu'il m'intéressait dans les savoir-faire des fabricants, c'est comment  est-ce que eux, quand ils déformaient les choses ça changeait la qualité de l'instrument. Parce que moi je regardais les rasonnances et comment ça les modifiait mais ... et donc je me suis dit est-ce que c'est possible de changer les qualités, mais non pas mécaniquement comme eux ils font, mais électroniquement.

VG : en les pilotant avec des vibreurs ...

AM : en les pilotant, voilà, c'est ça ... c'est euh... un peu arriver au ... au départ c'était ça du moins, arriver au mêle résultat qu'eux, et d'ailleurs moi, pdt mes recherches c'était intéressant de comprendre aussi ce qui fait la qualité, mais par qqch de complètement empirique, direct, sur l'instrument, moi je règle comme eux ils font mais par de l' électronique, donc j'ai une maitrise, je sais ce que je fais par l'électronique.
Ça, ça a été qqch d'important, euh... ouais, de qqch qui vient disons de pronlématiques de recherche, disons ... on pourrait presque dire c'est presque opportuniste, c'est à dire comment le ... une thématique de recherche de l'époque m'a fait sentir qu'il y avait des choses qui par mes connaissances à moi, par mes compétences dans un domaine spécifique pouvait m'amener à faire comme les luthiers, enfin je sais pas comment te dire tu vois...

VG  : oui, oui ...

AM : c'était vraiment un peu en temps que chercheur qui d'un coup bascule en recherche appliqué"e ou disons développement par rappprot à ma curiosité à la base de chercheur. Donc ça ça a été un des truc mais ça n'a pas été ça qui a été le déclencheur vraiment.
Ce qui a été le déclencheur, après je suis pas mon propre psy mais... (rire) ya.. euh... enfin c'est très précisément en 2010, euh, ma femme était enceinte de ma fille, ça a été  je pense  qqch de très important pour moi, càd elle elle était, enfin on était en train de construire qqchose quoi, son ventre qui grossissait et je sentais que je servais à rien, donc j'ai eu envie de faire qqchose avec mes doigts, bon...
et ça a été au moment où, donc là je travaillais à Londres à ce moment là, je travaillais sur les hautbois, donc des choses qui n'avaient rien à voir et à ce moment là il y avait de plus en plus d'appli qui sortaient sur iphone, il y avait, tu sais l'ocarina là des gens de Stanford, comment ça s'appelle ? Smule... il y avait, il comment à y avoir plein d'appli qui sortaient et je me souviens très bien car il y a mon frère qui m'a appelé et qui m'a dit "bon Adrien, il y a plein de trucs qui sortent sur les iphones, toi, avec tout ce que tu connais des instruments de musique t'es pas capable de nous faire une petite appli là, un truc sympa ?" ... et ça ça a été un déclencheur en fait.
Donc il y avait ce truc là, j'avais mon projet qui m'intéressait à Londres mais ça arraivait à, enfin j'avais envie d'avancer... ma femme (fait un geste montrant son ventre enceinte)... mon frère qui m'dit ça .... et ... et dans le labo il y avait qqn qui utilisait... tiens c'est la 1ère fois que je le dis ça...

VG : faut faire un bébé en fait ...

AM : ouais c'est ça... d'ailleurs la 1ère guitare je l'ai appelé Annabelle comme ma fille ... enfin bon tout ça était un peu mélangé...
et dans mon labo il y avait ... et en fait moi je me suis dit bon mon frère c'est marrant ce qu'il dit mais moi je veux pas faire un truc que je considère comme un gadget, parce que justement pour moi tout ce qui était numérique, toi c'est ça qui t'intéresse, mais moi à cette époque là j'ai complètement ça, la musique nuémrique c'était pas du  tout mon truc à cette époque là, j'ai eu des phases tu vois.
et quand il m'a dit ça je me suis sdit non je vais pas encore fair eun gadget,  mais je me suis dit par conrtre est ce qu'il n'y a pas moyen de connecter les 2 mondes, moi ce que je fais sur l'acoustique et ce qu'on est capable de faire avec du numérique et... 
et donc c'est là que mes recherches  sont intervenues et c'

 ------- unparsed

revenu j'ai pensé à ce que l'intéressé est dans mon labo il y avait quelqu'un d'autre ça parfois que le jaudy médial des rôles ans qui était un post doc d'angleterre qui avait bossé avec des boîtes en aéronautique je sais que l'équipementier de d'airbus ou de boeing qui utilisé des petits actionnaires électrodynamique nxt les choses qui maintenant sont vachement répondu même en 2010 tout ne s'est pas trop débourser 2000 à peu près des premiers noms de là dessus et donc moi je connaissais pas donc je moi je lui disais j'aimerais bien trouver un truc qui vibre j'allais trouver des choses dans les boutiques à londres cité des gros truc qui n'allait pas et il m'a dit avec moi j'avais utilisé dans pour les avions d'avoir utilisé des petits trucs regarde la marque est mixte et qu'il marque anglaise lui ce moment et donc j'ai pris un iphone j'ai pris un actionneur et 10 des jeux les culés sur une guitare à deux balles de match et j'ai commencé à balancer des trucs dans la guitare ça c'était mi 2010 ou 3ème trimestre 2010 france j'ai fait ce truc là dans la guitare la maison et il a ça dans ma tête ça explose et ça a tout fait exploser parce qu elle quand j'ai fait cette expérience là je me suis rendu compte qu'il y avait quelque chose d'assez peu invasif où je pouvais continuer à jouer de la guitare acoustique j'avais en plus du numérique allait dont je me suis rendu compte que tout ce que j'avais fait avant sur comment ça résonne au couplage avec les cordes les propriétés vibratoire de la caisse tout ça avait un impact sur enfin tout ça me servait à comprendre ce qui se passait là donc voilà j'ai repris j'ai rappelé charles et ses mots jeudi total et fait des beaucoup de feedback et dans l'est dans les guitares pour moi ce que tu avais fait et apparaît là je monte un projet de recherche avec la nr aussi grâce à une super rencontre avec baptiste se met à nu st pierre et marie curie était lui spécialiste du contrôle vibratoire pour les remercient pour autre chose et pour vous cette série de choses entre 2010 et début 2011 j'ai monté le projet nerfs ça a marché et après donc j'ai créé ce truc smart instruments se produisant un siècle donc voilà ce qui m'a motivé allait commencer à jouer avec les technologies là c'est voilà c'était tout bête c'était juste je pense quand j'ai commencé à jouer avec cette technologie là c'est tout l'adieu au fil de nos discussions que j'ai joué de la toujours de la guitare mois joudelat j'ai joué de la guitare mais vraiment amateur dans un groupe de rock je lui toute la basse et chanteuse de rock les choses un petit tapis stick j'ai commencé à rappeler enfin des chansons c'est toujours bien mais si moi je veux plutôt du violoncelle ils ont jeté dans cette classique petit outil et ensuite j'ai fait une année n'est qu'un instrument avec serge de l'aubier les tentes étaient pas encore là mais je crois qu'en cd croisée après soit fin à la faim et à fait plaisante pas avec les voilà il ya eu ça avec pierre le goux on allait commencer à faire aussi un peu de musique électronique à la maison voilà google il ya plusieurs choses comme ça entre d'instruments classiques électronique et moi sinon sur tout ce qu'il ya eu c'est que moi tout ce qui m'a motivé depuis 
j'ai 19 20 ans c'était la musique concrète pierre schaeffer j'étais un grand fan de de pierre schaeffer donc en fait c'est ça surtout qui au bout du compte ce que je voulais c'était réussir à trouver les clients un équivalent de la musique concrète aujourd'hui c'est un peu naïf et entrer c'était ça qui m'a motivé à la base dans mon histoire musicale disons c'est c'est ça qui a joué souvent plastiques musique concrète lecture de pierre schaeffer et désireuse sommer quelque chose de l'heure des airs sonore qui peut être cernée par la musique concrète qui n'a pas vraiment ses instruments comme on parlait des cimetières par des versements dans les désirs dans les motivations qui ont poussé à hama à créer des objets comme ça enfin je sais pas non c'était pas désir c'était plutôt essayer trouvé que ce qu'est ce que c'est qu'est ce que c'est 
la musique aujourd'hui c'est tout ça c'est pas parce que j'ai des banalités vous reste valable voilà c'est fait les grecs de bon et donc voilà je me suis juste dit si pierre schaeffer il était là aujourd'hui avec ce que lui évoque cette rencontre entre technologie perception et et sciences ils ont qu'est ce qu'il aurait fait aujourd'hui qu'est ce qui serait un peu le la petite boule guînoise de notre de notre temps si je devais faire quelque chose qui représenterait mon époque musicalement ça va faire un objet musical disons comment dire au début je pensais plutôt d'un point de vue composition quand j'avais 18 ans après j'ai pensé à musique électronique après le poussif un monde mais ça pouvait être composition ça pouvait être un instrument mais je voulais essayer de trouver quelque chose qui pour moi représente et qui représentait pour moi notre époque et avec sa fille jade voici embarqués sont très cools avait l'impression que j'avais débloquer quelque chose que vous 
alliez quoi ça a donné un sens à ces trucs là tout ce que je cogite et sur pierre schaeffer je me suis bon finalement peut-être que ce truc là qui est la rencontre entre le monde physique et non numérique c'était ça notre pour tout résumer et voilà alors par rapport à l'habitude de changer j'ai interviewé récemment nous emmener des gens qui pratiquaient donc mes questions étaient toutes orientées sur comment le musicien qui est une ce genre d'instrument oui gère un certain nombre de propriétés parfois contraignante du numérique le fait que ça marche ou ça marche pas c'est oui ou off le fait que qu'on peut passer de manière brutale d'un contexte à un autre qu'il ya une continuité par rapport à peut-être que tu favori que toi sauf à cette même question ressurgit chance et tu parlais de connecter les deux mondes entre l'acoustique lumière et demi c'était une manière de promo de formuler de manière chronique ans ça serait de te demander si c'est pas introduire un peu la bergerie km du numérique dans l'acoustique dans la torre renaissance numérique dans l'acoustique du numérique 
dans le coup si la bergerie c'est la critique je ne sais pas je pense notamment le nick a peut-être al'avantage d'être répétable donc on a quelque chose si ça marche une fois ça marche de la même manière à chaque fois c'est parfois de aussi partie de ces lignes est l inverse une table d'harmonie c'est les moyens de par elle il faudrait il a je repensais notamment à une conférence de données collins qui a cherché les musiciens qui espéraient tout moment de beaucoup de manières très empirique avec l'électronique et qui a fait une présentation où il est allé un peu les différences entre hardware et software oui notamment le fait que software s'est constamment dans le présent c'est constamment mis à jour que ça marche ou ça ne marche pas alors qu'elle va vers ça que le marché même en étant un peu cassé pour wii donc comment comment tu vois un petit peu cette fonction terme dans le dos de l indice que huawei hockey comment est ce que je vois c'est demandent ensemble alors une question comme ça le premier truc qui me viendrait ça serait te dire que tu perdes approche je veux dire en une phrase en approche c'est avoir une acoustique programme c'est tout simple c'est à dire le numérique est au service de l'acoustique ce qui m'épate le plus c'est qu'en fait tu te sens même pas qu'il ya du numérique c'est ça qui m'intéresse le plus c'est bien parfois enflés on s'amuse des cliniciens on leur fait des traitements et lui ils ont juste 
l'impression que c'est une autre qualité acoustique maison pendant l'impression que du numérique ça c'est ça qui m'intéresse le plus c'est numérique le service il pourra voir pour donner au monde physique la cap des capacités de programmation du numérique c'est à dire dans les diverses attitudes de ne pas trouver en france exactement comment traduire ça à côté voilà polyvalent au modifiable qui veulent avoir une vie voilà te dire simplement il aurait sûrement plein de trucs mais en tout cas ce qui est vrai pour moi la vérité c'est que je le vois les choses comme ça utiliser la technologie au service de la russie pour pouvoir le programme pour cette métaphore peut-être mal droite de loudes radin je vais plutôt d'un côté j'ai eu la bonne surprise de voir que pour les luthiers quand donc les productifs et dur de l'association lui qui est violement quand je leur faisais des démonstrations il ya eu il visitait comme nous tu es juste un luthier où on fait de la lutherie pour du mozart et toi pour que tu as dû trier sont intéressantes il faut qu'elles servent la musique de maintenance de demain mais de me voir faire des traitements sur des violons échanger leurs talents pour en temps réel pendant curieuse louis dessus ça les a pas choqué donc ya pas vraiment lu ce truc de hacker ce qui nous amène pour me nourrir l'ordinateur dans l'audio ya pas eu ça j'ai pas rencontré de salé seule chose alors peut-être que c'est plus que tu es musicien alain rolland ce que tu dis toi parce que des gens qui pratiquent les seules choses qu on a eu pas sur les guitares pop et maintenant les sur les prototypes d'avant en particulier sur les clarinettes par exemple et 
sur les violons aussi c'est qu'en fait il ya eu certains musiciens qui ont senti que on a mené une énergie concurrence à ce que le en fait elle et ça ça les à dire à certains qui ont été gênés dire que normalement un instrument et les filles et d'instruments acoustiques et la cdefi ge cee ensuite qui vont de 300 des faire émerger des choses maîtrisé en connaissant bien à propos en reprenant bien tous sont un peu différentes notes à l'exploiter au maximum et un nom quand on arrivait avec le traitement d'un coup l'instrument ce n'est plus le même dix mois je peux pas faire ça j'ai travaillé le truc est maintenant en fait ça sonne plutôt pas et bon pour ça devient gênant d'avoir lancement refuge puisque d'habitude c'est eux qui donne ce côté souvent dynamique dira seulement la façon de jouer ça c'est un problème et un deuxième aspect ça a été des gens qui ouais il met de l'énergie dans l'instrument est d'avoir une énergie concurrence qui vient corriger ils ont perdu 5,6 avec quelqu'un d'autre qui joue avec legault et donc ça ça peut donner aussi donc pour lui dans la bergerie qui pensait plutôt côté interprète côté musiciens que ça ça a pu dans certains prototypes pour certains certaines pièces qui avaient été écrits de s'appeler génie aujourd'hui on n'a plus tout ce problème là avec nos produits a vraiment dit tard parce que en termes fonctionnels c'est des choses qui se faisait déjà ce qu'on propose là c'est des choses qui se faisait déjà avec du traitement électronique mais que non vu son enterrement l'hectare donc pour rien côté a encore été transparent de saoû déjà connue entre ici c'est pas un nouvel instrument pour eux la rendait dans un cadre où un peu comme ce que je visais annoncé le numérique au service de la guitare de ce que peut imaginer déjà que ça soit des pédales analogique que ça soit des tomes acoustique qu'on change nous on leur fait revivre ça mais avec une expérience plus simple et plus direct mois donc là on n'est pas ce côté ou dans la bergerie parce que parce qu'en tant fonctionnels à mon avis juste c'est quelque chose qui se représentent déjà mais il ya un côté high-tech qui est un peu problématique la plus cotée lutherie quand on parle avec des grandes marques et de leur dire qu'on va intégrer ça dans la revanche seulement voilà ils n'ont jamais encore plus à tapas
 dans les instruments acoustiques ce genre de l'ordinateur avec une technologie très particulière que nous on est en place justement parce qu'on fait du numérique mais avec du hardware pour du numérique très spécifique par des latences donc il ya quand tu parlais de cette distinction annuelle sa victoire pour moi dans le numérique jeu mais aussi la partie hardware disons qu'on a besoin de pouvoir faire dans le traitement numérique et pour par couple et ça avec lui est donc là par rapport à eux les luthiers aujourd'hui installé ces trucs là dans delta c'est quand même c'est pas loup dans la bergerie c'est plutôt au bois ou comment on va faire certains musiciens j'ai l'impression de l'habitarelle elle rassure en grande partie à cause du fait qu'elle est liée à un objet qui est connue qui est identifié et qui a fait l'histoire et qu'il y à un minimalisme de l'interface numérique qui vient de sujets comment tout est pensé qui donne une place assez discrète à sa demande et vidéos de démonstration que j'ai vu où il y avait les vibratos chorus reverb on reste sur des effets qui reviennent un peu incorporer ce qu'on pourra voir avec une pédale d'effet dans ce sens que je suis j'imagine qu'il est possible de faire des choses que pour présenter au début et pour rassurer tout le monde c'est probablement vient de commencer car son heure de feu d'attaqués nas trapu imagine pas autre chose possible oui et par rapport à ça et là aussi une question qui se pose par rapport aux équipes je veux tu mets ton site est en situation qui sont 
 toujours en mouvement de la question de l'apprentissage de ce que tu disais avec les violonistes oui mais moi je le fais et certains auraient tendance à prétendre que on ne peut pas les apprendre parce que l'objet ne change sans arrêt oui ok comment tu fais alors par rapport à ce que tu disais au départ donc en effet à ça c'était une magique dans les tests qu'on a fait ça me rappelait les trucs là où justement la façon de faire des tests et tous les protocoles mais il faut toujours rester très ouvrir parce que tu as des surprises et nous des premières surprises qu'on a vu çà a été donc au début on s'était très fixées sur des effets connus et on s'est rendu compte que dès que tu arrives à un certain niveau de musiciens des effets connus et vidéos qu emi group je les ai déjà et moi j'ai mon matos qu'il faut jamais bonne pédale est donc ce qui est ressorti c'est que des trucs que nous on pourrait considérer comme des défauts c'est ça que les intéressés ils ont essayé de tirer au maximum sur tous les trucs qu'ils allaient j'avais entendu en particulier le truc qui marche le mieux c'est tout ce qui est su stein tu sais un peu les niveaux là tu vois ce genre de système donc nous on a des sorts de ibo dynamique équipe est bien loin sur toutes les cordes et que mettra différemment sur chaque note et tout ça et au savon bachand exploiter ça et c'était le truc qui justement est à la marge et qui ne ressemblait à rien et finalement dès qu' on est arrivé à un certain niveau de musiciens c'est ça qui est intéressant c'est que l'on utilisé pour rassurer les gens oui mais c'est pas ça qui va faire tes pro vie vont l'acheter parce que eux ils 
 veulent justement le truc unique qui d'habitude est impossible à faire on ça il ya eu ce côté sue steyn et aussi des côtés très originaux c'est la transformation de l'acoustique de l'instrument donc ça c'est particulier aussi donc on a en charge le temps j'ai pas voulu mettre l'accent là dessus au départ parce que c'est très subtil de booster un peu ton show acoustique transformée et donc voilà j'ai pas mis en avant met donc en effet il ya ces aspects là qui sont d'ailleurs beaucoup plus révélateur de nos technologies que les choses que l'on a mis en avant aujourd'hui qui la technologie importante de plus implicite là dedans dans le contrôle du feedback met donc en effet il ya tous ces aspects là et on s'est dit que pour c'est exactement ce que tu as dit juste pour vous rassurer pour commencer à tirer les gens il fallait déjà il fallait pas les envoyer sur des choses qui étaient complètement étrangère pour qu'ils puissent commencent à s'approprier leur interface pour qu'ils puissent garder leurs références avait dit à tout le temps il faut sincèrement de la guitare augmentés quoi tout en ayant des capacités en plus donc ça c'est la première partie de ce que tu disais donc en effet il ya ça ensuite sur le côté si j'ai bien compris comment plus parler d'eux comment essayer d'avoir quelque chose qui est stable et afar et qui va pouvoir s'en crée au fur et à mesure parce qu'avec ces technologies à ça bouge en permanence et donc il ya toujours le risque enfin en tout cas est ce que dans sa nature même est ce 
 que ces trucs-là reste sur des choses éphémères et qu'il faut toujours renouvelée renouvelé c'est bien ça tu voulais dire 30 attrition un peu est ce que je voulais dire c'était ici j'apprends à ramener sa d'une chose concrète c'est je parlais de la pédagogie dans le sens oui si tu vas un conservatoire tue pas oui tu peux apprendre la guitare on t'explique un certain nombre de techniques de jeu toussaint avec ces nouveaux instruments enfin les instruments électroniques de manière générale la question se pose vraiment franchement c'est comment comment tu enseignes et les trois coups stick dans le cas d'un instrument ou purement numérique vraiment vraiment une poêle dans mon cas il ya un enseignement électro acoustique qui existent mais qui est une école assez particulière dans la semaine il va musique électronique avec ce genre d'instrument que j'ai compris est ce qu'il ya une pédagogie de décevoir parce que votre contrat aujourd'hui avec ce premier produit là tout ce qu'on fait c'est basique c'est juste que des choses qu'ils ont dans leur paie dame qu'ils ont avec leurs leur la rance en bluetooth ils ont dans leurs guitares donc il n'y a pas je cherche on cherche pas dans ce dans cette solitaire on cherche pas à les exploiter les spécificités du monde pour essayer de ensuite les apprendre les transmettent et ou non nous on veut pour eux au lieu d'avoir leur pédalier ils ont l'air plus d'indiquer dans leurs écrans c'est tout ces salles de vie sain ça quoi voilà c'est un travail du son qui font déjà en plus c'est que les guitaristes ont déjà l'habitude de faire par deux dispositifs que nous on intègre dont des guitares acoustiques pour profiter de des questions de profiter des propriétés des caisses des questions de qualité sonore des questions pratiques sortir de la guitare est une question de connectivité pour qu' il puisse échanger entre eux et que quand ils écoutent un cours en ligne au lieu d'avoir sa leçon qui sombra dans l'intérim en soi une guitare avec le maximum de qualité possible question qualité sonore de connectivité et de simplicité 
 du dispositif donc pour ce produit là on n'est pas dans disons un un procès à un produit pour aider à processus créatif que nous on viendrait proposées pour faire des choses vraiment nouvelle et qu on va essayer ensuite d'ap de faire apprendre et voir comment partager ça on n'est pas là dedans on est plutôt sur quelque chose pratique part à tapis sur un bouton et et voilà alors après on peut discuter sur ce que je pense que j'ai fait avant bien sûr des aspects les plus créatifs ou là il ya un an il ya d'autres sujets et d'autres sujets qui arrive je sais pas si tu veux qu'on parle de ça qu'on se concentre sur ce brûlant par le forcément mais peut-être la question que l'algérie a trait à ça c'est dans le design de la smart guitare ces nouveaux apprentissages sont aussi des fois liée à des nouveaux gestes oui et hills à la smala on a reçu à l'elysée revenus oui l'interface est problématique pour ça j'imagine que c'est un choix qui a des raisons bonnes ou mauvaises sans parler qu'elles sont bonnes ou mauvaises mais vous auriez pu j'imagine facilement et acéré des capteurs gyroscopiques il ya des capteurs oui pression des glissières toutes sortes de choses gillett la table complète pour être farci de boutons et ce n'est pas le cas je me l'examen fois c'est vers une forme alter de ce minimalisme exact et qu'est ce qui est important je suis pas m'avoir que c'était ça c'est le premier qui me posent des artistes sont là parce que dans le monde et guitariste eux ils ont une vision très normal de ça mais tu as raison c'est bien normal quand vient plutôt niveau de technologie il y avait au départ on avait plein plein plein de possibilités justement c'est super intéressant et d'ailleurs il y en a certains qui nous pousse à dire j'aurais bien plus si ça et ça et là moi je suis mais pour l'instant tout cas pas dans ce flot de lave alors bye sa revanche ce que je disais avant en fait ces dons une philosophie de réalité augmentée que son ebit a augmenté moi mon idéal c'est le numérique au service de la gousse ticket personne se rend compte de rien en fait c'est ça ça l'idéal que j'ai 
 en tête c'est un numérique intégré en fait dans notre monde physique parce que moi j'ai toujours été très frustré ou critique sur face à toute la complexité du monde physique on a et qu'on a encore du mal à analyser que quand on fait des instruments quand après finalement un appui sur un clavier sur une touche avec ce qu'on a j'ai jamais cru à la sur du long terme comptes andré vers quelque chose avec la qu'on arrivera à la complexité d'un physique moi j'ai toujours vu les choses en laissant partir de déjà de notre monde à nous et je trouve que ça rentre dans une philosophie plus actuelles de développement durable et de foot voilà les gens il se met à apprendre la guitare pendant 10 ans on a on a tout un répertoire on a tourné on a toute une richesse je veux quand ils prennent ça ils se disent qu'ils ont le potentiel d'utiliser toute cette richesse qu'ils ont déjà donc c'est dans cette logique là de s'appuyer vraiment sur l'existant dans une logique de réalité augmentée ce sera autre chose derrière qui va venir compléter cette chose là et qui doit être le moins invasive possible le moins intrusif possible et qui rejoint ce qu'on appelle aujourd'hui les objets connectés l'internet des objets par rapport à ceux visés précédemment sur qu'est-ce que c'est notre époque moi j'essaie de le voir comme ça quelque chose comme ça où finalement les ordinateurs il va en avoir toute façon partout ça va être intégré par tous et qu'il n'ya plus de distinction de ces demandes-là du monde physique et du monde de l'ordinateur et elle ça c'est un démonstrateur de sages le vit comme ça viendrait j'aimerais que ça soit un démonstrateur de ça ou après avoir fait des objets dédiés dont les ordinateurs après avoir commencé à porter ça dans des smartphones et quelques objets ben maintenant on est rentré dans le truc où ça commence à être de plus en plus partout ça commençait de moins en moins intrusif donc on peut avoir un idéal comme ça où on peut s'appuyer sur n'ont pas le savoir faire qui dans le monde numérique mais savoir faire qu'ils ont du monde physique en l'occurrence des instruments acoustiques et ou industrielles sinon des trucs l'industrie troisième dépenses et où j'ai encore un peu le même hymne est lui même idéal un peu comme ça ou ou l'industrie existantes a juste intégré ça et s'arrête complètement intégré tu vois c'est renier un idéal complètement de fusion des choses et tout et partez le symbiose mondiale semblable symbiotique au prince deux titres de très bons les deux jolies de renationaliser sage possibles tels que j'ai pas lu mais voilà c'est ça d'accord je vois qu'ils étaient déjà presque une heure au pied ouais j'aurais deux cas soumis du groupe avait ses salles pas d'acheter malin bon voisinage parmi beaucoup cette année environ seulement des barres noires un petit aussi le gênent pas pour se moquer la première question s'est elle est liée au contexte actuel de production musicale oui dans lequel une très grande un très grand 
 pourcentage de la musique on ne serait pas là précisément mais qui a tendance à grossir encore de la musique qui est produite enregistrés sur cd est enregistrée en studio à l'aide de logiciels de montage qui permettent de tout recomposée et bien souvent d'ailleurs aller programmé à la main c'est à dire que plutôt de faire jouer un batteur on va enregistrer avec la difficulté qu'on va voir après remettre son jeu en place avait ajusté pour les besoins de production les programmer directement avec salut mauvais coups qui permettent d'ajouter du rubato ou que sais-je de tout à tout un pan de la musique produite sur les enregistrements d'accord qui est fait par des méthodes offline démission en fait will play du jeu wii et par rapport à ça c'est dans la fabrication d'un instrument a un côté caen paris quelque part en fait de miser son asthme en dix ans la somme on n'est pas morts on engagement elle en parle il ya un an les gens comment tu vois le la smam guitare ou la position de manière plus générale des instruments dans cette production musicale ou où le numérique a tendance à assister ouais de plus en plus la production musicale que ce soit par des méthodes a rolling ou pas des méthodes voilà c'est un nouveau palais mais je ne connais pas grand chose moi je veux dire juste très simplement comment ça se passe un peu aujourd'hui que l'on parle doucement avec des gens qui sont justement côté studio et quand ils voient seront chez eux ce qu'ils nous disent globalement c'est moi le maximum que je peut choper à la source je choque sur l'instrument parce que tout ce qui se fait en flammes derrière sur les modifications du timbre et tout pour eux et nous disons c'est si on peut éviter on préfère les cons donc ils sont contents quand on leur montre ça pour l'instant on n'a jamais travaillé on n'a jamais été au bout d'un projet comme ça mais en tout cas premier abord pour par d'autres gens plutôt côté studio de ce qu'il nous dit c'est un oui ça m'intéresse parce que moi je son pied d'eau c'est toujours un problème donc si vous pouvez me faire quelque chose programme qui fait que morales ouvrage est pas ce côté pied de ça m'arrange moi j'aime pas trop rajouté les traitements sur une guitare acoustique a posteriori si vous les aviez en amoureux moi j'ai juste à mettre un micro devant avec déjà le truc et dernis et comme il faut très ça m'arrange aussi donc là pour l'instant moi je suis dit tu pas du tout spécialiste ont créé tout ce que j'entends c'est plutôt que par rapport à un instrument de la prise prises de sons disons d'un an suivant donc ça ne couvre pas tout ce que tu as dit avec la pré poulain tous les aspects aussi de jeu instrumental mais disons prises de sons sur par deux timbres de son instrument voilà plus on parle d'une bonne source musée donc si long silence en plus ils sont plus de votre voix la paix on va mener en amont quand même en ce sens avec notamment l'avocat du diable là dessus et je conçois très bien que si on avait rajouté l'effet waouh sur les guitares de jimi hendrix studio n'a pas du tout marché vp ventes à 
 réméré way oui oui mais comme tu dis il ya tout un pan de la musique qui se fait comme ça maintenant et après je peux dire que aussi à tout un retour aussi à la musique live et aux concerts et plus maintenant que la monétisation par le studio est de plus en plus complexes et sinon je sais pas tout cas côté guitare acoustique service à ce que je peux te dire quand même c'est que j'ai découvert un monde de gens qui se retrouvent dans dans des salons affaires guitare acoustique et les associations de gens qui se retrouvent comme s'avance même en soirée pour aller présenter leurs nouvelles compos avec or petit en petit comité comme ça à d'anglais dans les appartements il n'ya plein plein plein plein de choses qui se montent dans le monde comme ça donc par rapport à la guitare je pense que pour ce produit leur particulier quoi ça colle quand même avec c'était pas la grosse industrie je ne sais pas aujourd'hui mais en tout cas ça colle avec une pratique musicale qui est très très très très très courante encore voilà juste qu'on me dit ça reste un truc est vrai que la production automatisée de musique se boit surtout enfin là où elle avait plus flagrante c'est dans la production typiquement de musiques de films ou économiquement de faire venir un orchestre part avoir acheté la wii de son usine à la guitare et de ce point de vue là est un instrument assez populaire que tu peux prendre les cinq îles manière récemment qu'elle est moins sujette à être peut-être un après puisqu'ils sont deux guitares touche france admire pas mal de choses peut-être on est dans un contexte qui leur matérialisation être question à dix mille dollars vas-y on a commencé à perdre ton passé ah oui oui au parcours on va terminer par le l'avenir oui la question elle est très ouverte c'est cunac lankaise qui selon toi faliez as quand même assez sujet de l'origine musique numérique ouais qu'est-ce qui est selon toi et hélas la voie vers laquelle on va où tu voudrais aller ou ok alors la manière dont ça va transformer notre monde et noël aux réalités qui est perceptible alors on déjà si on arrive à faire un truc où on est capable d'avoir une acoustique programmables quelque chose pas simplement sur les instruments ont discuté avec d'autres secteurs à voir son côté design sonore par programmation numérique pourrait le design sonore des objets relation numérique en discute avec l'automobile avec d'autres secteurs qu'ils veulent tu vas tu rentres ans transformer des temples des objets n'est non pas mécaniquement comme ils le font d'habitude en changeant des choses mais par des traits pong bec donc ça bon voila si on est capable 
 d'avoir des choses en particulier ici on arrive à avoir des choses comme je disais que tu n'as pas qu'ils sont moches mais comme on dit une identité sonore qu'on associe aux nouvelles technologies si on arrive à faire savoir plutôt vraiment quelque chose avec les objets qui résonne plus kazan - a plus aiguë plus grave est que tout ça soit programmables ça déjà c'est pas mal donc appliquer design sonore ce qu'on appelle aujourd'hui le design sonore vraiment à la matérialité les autres qui faisaient ça exactement donc ça ça ça m'excite bien déjà et et je pense qu'il ya un enjeu dans nos sociétés où les objets qui sont ni peut avoir des fonctions différentes il va avoir dix diables rouges est donc tout ça c'est bien d'avoir quelque chose qui peut s'adapter pour qu'ils s'adaptent dont la sonorité tennis ça peut servir tout ça c'est un désastre c'est pas exactement ta question sur les instruments numériques alors moi sur sur les vrais instruments au sens large ouais ok on va aussi voiture ça peut être un instrument pour toi ok voilà donc tu as là dessus donner cette flexibilité ses diverses activités averse est liquide aux objets sas à les rendre des instruments état d'une certaine manière parce qu'il est pour ça ça m'intéresse bien sinon après pour la la suite de ça migros ça serait de ne pas se limiter aux sons mais à intégrer les choses plus aller plus vers de l'interprétation ou aller vers par exemple à vacant par les philadephia quand j'étais là la semaine dernière la cité parce que j'aimerais bien dans la boucle ajouté peut-être un accompagnement qui va s'adapter automatiquement à ce que tu fais donc il va aller le chercher dans des bases de données telle musique que tu as envie de jouer donc on va aller te mettre directement des accompagnements couple et ça avec des technologies qui sont pas sonore mais plus de la musique au sens large du symbolique musical si tu peux avoir en même temps tu as partie sur pendant que toi tu es en train de jouer et qu'il reconnaisse que tu fais qui va te faire les effets correspondant on voit la banlieue et tout ça ensemble collectant qu'est ce qu'il faut à côté voilà donc ces genres de choses qui font à voile à ses employés imaginer son hôte chinois qu'est ce que vous aimez collaboration mais à pas comptés ce coq à la bière il ya des gens de ma musique je vois qu'ils étaient là qui était un cube est ici a lancé louise itunes qui étaient là peut-être pourquoi pas discuter avec les gens de guitare pro l'algérie c'était en france étaient des français en fait mais qu'ils font un peu la tow 2006 parfait pour guitaristes où tu as ta partition tablatures les pédales qui correspond avec les tout ça et nous nous intégrer là dedans en disant que ben en fait tout ça ça peut être tout directement ta guitare sur le manche est sûre et voilà donc du point de vue musical intégrer ce qu'on fait nous dans le sont intégrés à des problématiques plus général qui est qui pour l'apprentissage mobiliser et à la pratique 
 la composition et voilà il proposait un outil quasiment alternatif on pourrait dire à des outils peu écran clavier mais qui sont l'instrument lui mais dans tout ça ça passe à doha ce moment c'est pas ça qui fera que tu verras ta partition n'a pas encore prévu que sa démarche est affichée partition mais en tout cas voilà que à voir l'instrument comme interface aussi bien pour l'entrée coup la première sortie de tout au moins toutes altitudes en tant que musicien ça m'intéresserait bien et est sa manière connecter sur tout c'est à dire que l'on est un des enjeux qu'on voit derrière ça ces cubes par exemple l'un des copains qui a great art avec ses propriétés avec tous sont ces données propres que tu peut charger dans une autre guitare donc à changer de timbres avec le même temps que loeb que ta musique à toi tu vas pouvoir la partager plus facilement tu as une sorte de twitter de la musique ou petit à petit pas besoin de rien et est tout de suite la communication a fait très vite entre les gens et de partage se fait facilement ça c'est sur le long terme parce qu'il ya plein plein plein de problèmes techniques ça encore aujourd'hui mais je pense que ses habitudes là elles vont pas partir en quoi le fait que les gens commencent à voir le plus en plus l'habitude d'être en contact en permanence avec du savoir avec deux avec les autres et de grâce au numérique je pense que et ça ça entraîne de plus en plus sûre plus dans la pratique musicale qu'on va vouloir avoir pas instantanément je fais ça je vais pas me prendre la tête et comme avec mon smartphone appuie sur un bouton et j'ai tout de suite le résultat que je dans nos pratiques musicales tout ce qu'on imagine on va vouloir la voir matériellement donc c'est une vision un peu cognitive delà de la pratique sur et de et du numérique dans la pratique un instrument ou le rêve pour moi le truc ultime de sa guitare ou d'autres instruments c'est du sculpteur intrigante est juste un peu simple que ça et quand tu apprends à différents morceaux souvent la musique pop jazz je vais apprendre toi même des choses t'imagines des choses est d'ailleurs la construction mentale elle intéressante en soi mais quand même temps tu puisses avoir matériellement directement j'ai travaillé le solo de jimmy page de services aux jeunes comme tout le monde l'a tous les guitaristes l'on fait un moment tout le monde fait ça avec sa guitare acoustique parce que bon bah tu as a sous la main étaient en train de travailler que tu as fait le début avant qui lui est complètement avec un son clair la chance n'a pas envie de manger voilà que tu puisses avoir un oui tiens là il est passé en dix tours en tête juste à 10 puis voilà quoi c'est ça que j'imagine sur le long terme que tout ce que tu imagines bas juste ça soit détenu puisse la voir matériellement la grande tu vois je vois beaucoup plus inspiré ouais je les joue très très 
 bien vu c'est d'autant mieux que tu as répondu au passage à une question dont je me rends compte que j'ai oublié de te poser que je m'en faisais concernant mais j'ai répondu à temps mais c'est vrai que la surface donnant sur la mésentente mal sur l'importance de la communauté danser dans le monde numérique un oui de manière plus générale dans les instruments comme ça je crois pas moi 7 le groupe aux enjeux comme dans le reste quoi selon lui ce que les gens s'ils sont pas différents quand il quand il ment sur facebook avec leurs copains quand ils font la guitare ça c'est même genre d'intervention de génériques seynod sur notre communauté de plateformes wii voilà peut-être ça à partir mais moi je pense fait je pense que c'est quelque chose de très humain ça s'accélère avec technologie numérique mais toulouse on a 2 bal est de plus en plus on est là dedans et même pour une efficacité dont voit l'ennemi je veux dire mais moi en terme d'efficacité technique vous pouvez presque dire parce que parce que dans ta pratique musicale tu as besoin tu as besoin d'un train dans ton apprentissage dont tu as besoin d'échanger un maximum ici quand tu as pris ton nouveau morceau si ce n'est pas tout de suite le jouer à tes potes et ben tu sais pas si tu peux enfin je sais pas c'est ma vision d'être à moi des choses mais en général je pense que quel que soit le truc quel que soit nouvelles connaissances tu as besoin ensuite de la mettre en oeuvre juste pour avancer quoi même pour apprendre le morceau d'après martin a besoin d'avoir présenté on audition ou avec des pâtes à la maison la trouve beau morceau voir comment c'est reçu va échanger avec quelqu'un qui va jouer une partie du truc le fait de le faire en live ça tu vas le récent tir différemment ça donne une nouvelle idée c'est par cette interaction en permanence même en termes d'efficacité mais courtois pour pour avancer pas besoin d'être connecté en permanence termine en a besoin on travaille bien dit de moi aussi mais très vite il faut que ces choses là se mettent en place et les technologies m'a permis de faire ça bonfol tous les youtube heures et tout ça c'est incroyable comme donc et je pense que nous notre réponse à nous par 
 rapport à ça c'est que la qualité sonore vallée rarement rendez-vous dans tout ça en fait elle n'a pas été elle n'a pas pris de l' la vague du numérique en fait la qualité sonore elle l île élèves voilées youtuber on a des super truc tu as des supers mecs qui font décompose super intéressante sur youtube mais il souvent le son est pourri c'est con mais c'est complètement dépendant de ces technologies de la cua donc ou bien pour avoir un bon sont souvent un matériel très sophistiqué mais c'est pas intégré dans le dans ces nouveaux objets là et nous on aimerait que ce qu'on propose ça soit le truc qui permet de sable parce que si tu es que tu as composé ton morceau avec notre guitare et que tu joues avec le youtuber il joue j'ai bien kiffé sa vidéo après si tu charges sa musique que lui il a écrit tous ont un cover à deux je ne sais pas quel morceau que l'on ait à van houtte charge dans la guitare et après tu l'écoutés matata guitare qui jusque le mec a joué donc en termes de qualité sonore c'est incomparable avec ce que tu as et lui la ligue ou en riant s'il a la même guitare chez lui toi parce que le dire celui là la guitare il s'est enregistré avec ça quand tu charges ce qu'il a fait tu les écoutes dans ta guitare on a une super qualité sonore alors peut-être qu'aujourd'hui c'est un peu con record de mettre en oeuvre donc il va falloir qu'on arrive à faire ça vite et des points techniques vous êtes sur le long terme mais je crois beaucoup plus à ça que de rester complètement avec des matériels qui sont indépendants chez dans la communauté je vois ce que je veux dire si tu as une unité dans l'instrument et déjà la base dans les gens qui partagent les choses ça va simplifier énormément simple d'un côté normatif cieux de type de captation sip de réécoute qui types d'enregistrements plus fait et qui après qu'ils partagent et les autres sont dans les mêmes conditions que toi ça je pense que ça peut accélérer ça comme des outils numériques voilà des plateformes sauf que là la plateforme elle passe par un revers spécifique et une sorte de contre arguments à la modularité qui est caractéristique de tous les voir avec mon nico voilà l'état compliquer la vie en ce moment que ça crève kloten que l'on utilise jeudi à 5 13 5 reste c'est un terme de chien qui parle du fait que l'image et et son fusionnent mais pour les raisons du sinaï qui sont a priori de choses indépendante a les seins espère jouer deux ans à fusionner ok vous nous faire une démo bah oui quand même attente est venu enfin des mots saint chrême causer des crises un crime s y sommes synthèse les ch à la place de thc when it in america hop alors je vais te l'a présenté comme on la présente au blizzard tu valoir alors justement le type de façon de de partager sur la 
 technologie donc voilà huit a aussi dit à fait correct du coup ils s'arrogent ans nous avons tu vois je commence toujours par dire ses guitares normal jeu là c'est vraiment à des tabous même en bill doigt sur ces trucs là ensuite je l'allumé et jeudi mais regardez vous pouvez aller d'un côté sportif dont nous les blue tooth on se connecte à yagg je suis connecté ensuite je l'emmenais souvent toujours le même morceau obtenu l'entendre on est à lyon à chaque fois qu'ils aient que je vais chercher donc je vais chercher ensuite morceaux qui ont enregistré là je leur dis à chaque fois les truites dont +10 comme dans son l'ue continue doubler parce que on n'a pas eu facile seule la guitare ça marche très bien en vie pour les pâquis pour l'instant on oriente très quand même guitariste du tarissement on dit ça on dit les backing tracks donc donc ça on leur présente en disant pour te réécouter par exemple ou bien tu as ton prof qui joue justement est appelé à jouer dans ta guitare mais sinon il ya aussi leur back in tracks donc les accompagnements des guitaristes par exemple ça ça fait beaucoup coûté bizarrement je pensais pas c'est ça qui plaisait plus mais oui c'est ça donc donc ça c'est le premier aspect jeu diffuser du son je te cache pas que pour l'instant si on l'a pas orienté complètement enceintes bluetooth c'est parce que ça pas une super bonne qualité ainsi parce que on profite de la réponse en fréquence d'une guitare donc c'est très bien pour que ta guitare mais sinon 20 ans tant les résonances d'égalité quand tu mets un peu n'importe où et n'importe quoi ici donc là on a travaillé district a croisé qui travaillent sur des halles ou de correction voilà donc la maintenant dans un modèle ans on a ça ça sonne plus comme une enceinte moi j'aime ça de jeu mais d'un autre côté sur l'utilisation ça peut être intéressant aussi donc ça c'est premier aspect deuxième aspect looper quand on entend développer nous pour l'instant c'est juste un en enregistrant rybak et ensuite les effets donc là la petite rivière là où tu peux changer donc en terme d'interface donc la vie si on fait juste plus est moindre dans nos banques qui sont prédéfinis quand on est finie en bluetooth en amont en fait on envoie à l'époque où on décide aujourd'hui depuis un ordi là on développe là mais ça c'est un volume et ça c'est un paramètre de l'effet donc dans la réserve j'aime les tickets je connaîtrai ça c'est un paysage change hélas la vitesse suffisante il traverse un peu de son sang c'est intéressant le faiseur sympathique avec pékin et marche à tous les coups avec les effets c'est quand les enlève je pars à l'acoustique on souhaite augmenter là tu vois quand après qui dit merde ça c'est ce qu'elle touche avec ça je fais ce que tu sais utiliser l'e10 aujourd'hui ça va toujours très vite si en juste l'envié il est remarqué oui l'afrique il a pensé comme 
 un truc et qui fait là ajouté d'arrêt le corps si on peut dire 1 c'est décidé encore une contrôle ibérique de doré normal et j'ai aussi ce que je disais ce qui nous a valu mettre une disto et donc la lys ouvrons ce avait plein de feedback tu vois c'est donc c'est là qu'on s'est rendu compte remettre les cheveux examen rue le système concert l'histoire s'impose finalement on va en ville à l'histoire et et donc voilà que ça chiffre de demain après l'app temps ça tirait un peu toujours les guidances une m mais entre choisir quelqu'un on va zy total de choristes il aimait bien d'être là donc voila tu oses tu fais plus et moins ici et à volume tu peux laisser toute façon je pense ça tu vois chorus et reverb yama ils ont fait france abou seada connaît yama trans acoustique ils ont fait une guitare et m avec un actionneur qui signeront l'habitat ça n'est pas tous nos trucs mais tu peux commencer à faire des petits effets donc t'as pas une fusion comme ça parce que s'arrogent du trou mais tu es tu fonda de la guitare mais tu pourrais comparer regard des cieux j'ai même passé le truc parce que ça c'est aussi un gros sujet en tant qu'entreprise la coordination avec les gongs les compétiteurs les concurrents de fait que j'ai créé la boîte c'est pas cela va commencer à se mettre sur ce marché là ça c'est un moulin d'ascq dont on appelle tous parlé que du côté des moteurs quand même pour aller au bout le type de produit quitte c'est vraiment par rapport à ça aussi l'environnement les concurrents le technologies qui sont utilisés aujourd'hui nous ce qu'on vient apporter il ya un côté très très peu nantis rubin ouais ça c'est très important dans les jungles invite à lever je suis tombé en fait j'ai croisé plus le connais bien je connais sont dans la boîte mais mieux etc lab qui fait là c'est moche oui un super ouais j'ai compris mais je les ai jamais eus mais ouais bah ouais on est sur un truc de quatre chefs waits attaque d'accord sur la longue pour une conférence qu'il était là bas pour présenter une mandoline macally parce qu'il a un projet de recherche ont du talent oui différents instruments traditionnels italiens qui est le nom de l'hymne sorte de cornemuse dont j'ai oublié le nom et je sais pas encore ok a battu voient eux on exactement je pense dans la même vision de cet avenir j'ai connecté guitare en même temps après il a vu la scène suite avec les hommes et essai avec les vidéos et je vais lui la vague l'été pour présenter sa mandoline ok donc j'ai vu qui reprend un peu d'hésiter ils sont un peu plus sur le geste moi j'avais lu ils ont la journée de jérusalem est où les yeux c'est un instrument acoustique un talus en fait commencé c'est ça n'a pas de sens du 6ème juste un manche et à la caisse actions mais passera la casse est découplé quoi le mans millions mais c'est la même après après donc est donc bon c'est vrai qu'à maquiller comme un maquereau ouais gros bonnets ouais je les cherche un peu et auditionné vachement là dessus en piano ils ont fait un truc vachement bien keanu 37b est parce que la transat et klein est aussi par vous même prix que nous assimile 2009 et puis tu as juste envie une petite phrase de la volaille urgence les combats ont commencé 500USD est là mais ça va être un peu plus cher pour la commande c'est sûr énigme pour les premiers mon rêve injuste dans les 750 c'est l'un des pendus des volumes qu'on arrive à faire pour nous c'est très con traire de yamaha qui font et où les traditionnelles quoi moi même qui ne s'est pas grandie terry sera jugé sur mes pattes qui sont venus tarifs pour les heures qui nous achètent une autre scène professionnelle oh en fait ce qu'elle particulier celle de latence la latence au mur elle est hyper fait comparer tout ce qui existe en musique parce que comme on en voit nominé traiter rapidement c'est c'est plus des microsecondes plutôt que des millisecondes a donc ça ça fait une carte électronique qui pas habituelles en bleu pas juste faire du dsp pour nous prouver quoi et cd tout le projet de recherche a dit agréable gallo sa propre carte ouais mélange des choses en mélange des composants traditionnels des choses plus industrielles et des choses plus au nord on est ensemble et c'est ça qui est un peu fait que c'est cher on a vraiment concevoir un truc spécial donc on espère que à terme ça a bien des choses qu standard tranquillement sur leur zone rouge sa jauge énorme bévue ouest qu un peu sur ce cas là vous apporter c'est vraiment ça ce côté non là encore 50 fois plus qu'il ya dix ans ce que nous nous on a vraiment un round 3 les matic jouer vraiment super vite lui ont dit web interactive et oui ils sont multi lots multi connu un bon match car cet acte est là bas c'est intéressant ce qui fait ne pas croiser en drôme et rencontre joliot et

 % ok 20/12/2017
	\chapter{Interview : Patrick Saint-Denis}

\section*{Biographie}


\section*{Transcript}


PSD — Je me suis rendu compte que je faisais des IM slash scénographie, et puis là je suis un peu plus conscient que c'est ça que je fais, mais au début c'était pas dans le but de faire... c'était vraiment de la scénographiqe pour de la musique instrumentale … donc j'ai … 

VG —  instruments acoustiques tu veux dire ? 

PSD — comment ? 

VG —  instruments acoustiques ? 

PSD — ouais, instruments acoustiques, donc j'avais fait des trucs c'était 2008, 2009, c'était Processing, OpenFrameworks... de l'image audio-réactive...puis j'ai commencé à jouer avec des plumes d'oiseaux, qui étaient montées sur des moteurs et qui tournaient en fonction des amplitudes qui faisaient tourner les moteurs... et puis... que j'amplifiais visuellement avec une caméra, je traitais … donc une espèce de workflow qui était celui de la musique mixte, si tu veux, prendre le son d'un instrument puis le modifier puis l'envoyer dans les HP, mais je faisais la même chose visuellement, avec des éléments qui bougeaient en même temps que le son ; je commence à apprendre l'arduino... donc je commence à développer des scénos avec des objets et de la vidéo qui réagissait au son... et puis j'ai tout de suite vu que … qu'il y avait quelque chose de plus dans les objets que dans l'image vidéo... quelque chose qui attire plus, qui... euh... plus surprenant... avec un objet, à l'époque c'était une plume d'oiseau, c'était tout petit comme ça, et puis ça out-stageai (sic) à peu près les projections architecturales, tout ça... donc j'ai décidé d'aller là dedans.. poursuivre ça, poursuivre l'idée des objets animés par le son. En fait là j'ai fait un gros écran physique, qui est derrière, just de l'autre côté [du mur de l'atelier dans lequel nous sommes NDR]  

VG —  avec les feuilles de papier là ? 

PSD — ça prend de la place.. et là c'est juste des petits ventilateurs... [me montrant le dispositif] (incompréhensible) donc c'est un écran physique... 

VG —  et qui fait du son ? 

PSD — ben oui c'est ça, qui fait du son, et que tu peux toucher aussi... pour le spectateur, tu sens la draught d'air là..   

VG —  oui 

PSD — il y a 192 petits ventilateurs qui font ffffffffffoooooo... y'a quand même quelque chose de fun là.... puis ce fait qu'là j'ai transposé dans le monde physique l'interaction de type audiovisuel... c'est comme ça que je l'imaginais, et puis là j'ai fait d'autres dispositifs avec des HP, des robots, et puis des … encore des genres de transpositions là... des trucs vidéos dans de la... dans le monde physique et puis, et là j'ai commencé à travailler aussi avec la danse, et à utiliser ces dispositifs là vraiment comme de la scénographiqe et puis de fil en aiguille ces machines là, qui se voulaient de la scéno, sont aussi des instruments de musique, par exemple celui-là [montrant], donc cette série là d'accordéons-robots qui est vraiment... ça a été pensé comme étant de la scéno, mais tout le son vient de la scénographie et puis donc la scéno, la scène est un instrument. Et puis c'est un peu... c'est ça que je désire poursuivre maintenant, en tout cas c'est ça que je … conçois mon travail mais tu sais j'aime beaucoup garder une bonne part de mystère dans ce que je fais, donc souvent je fais des choses et puis je découvre plus tard... souvent mon travail devance un petit peu ce que je pense de mon travail... ce qui fait que les projets qui s'en viennent sont des projets qui sont plus pensés au début comme étant des instruments scénographiques. 

VG —  et... qu'est ce qui … une question que je pose souvent quand je commence une interview... tu réponds en partie à ça mais qu'est ce qui... t'a amené à utiliser des instruments ou des outils numériques pour ton travail plutôt que d'utiliser des objets acoustiques ou classiques... 

PSD — ouais... je sais pas t'as quel âge toi, moi j'ai 42 … 

VG —  Moi 37.. 

PSD — … quand je suis arrivé dans la composition... instrumentale... je suis rentré au conservatoire, j'avais 19ans, à Québec... clairement j'avais l'impression de rentrer dans un bateau qui coule... celui de la musique contemporaine... le discours... encore avec le poids des vieilles avant-gardes... et puis un discours assez... défaitistes là... “ fallait être là dans l'temps... les grandes œuvres sont faites... “ … puis vraiment c'était ça le discours ambiant et très défaitistes... et puis évidemment t'as pas envie, toi, d'arriver dans un bateau qui coule... Et puis d'autant plus que culturellement cette musique là n'a jamais pris non plus dans la culture là... au Québec c'est encore assez colonial pour la musique... et puis là est arrivé... tu sais y'a un bateau qui coule et puis un espèce de bateau avec des moteurs, qui coule pas du tout, qui va vite, y'a des gens dessus qui font l'party [la fête, en Québécois, NDT] c'est celui des arts numériques, et puis ça m'intéressait beaucoup le fait que … le fait que je me sois intéressé aux instruments c'est un peu... par cette mouvant là.. 

VG —  l'effevescence qu'il y avait autour... 

PSD — ouais, les arts numériques et … voilà... Aujourd'hui, c'est différent, on a tout l'intérêt avec les instruments tout le temps un peu  … sachant se renouveller... par exemple je travaille beaucoup avec des danseurs pour jouer avec des instruments que je fais... assez peu avec des musiciens... et puis il y a quelque chose avec la technologie, que le corps disparaît... notre corps disparaît... devient invisible, à travers les écrans, ou même la musique éléctronique, le corps est quand même complètement... sorti de la boucle. C'est à dire que l'énergie, en musique électronique, l'énergie ne vient pas du corps.. le corps contrôle peut-être le son ? avec l'énergie pour faire le son qui vient de l'électricité... c'est pas une mauvais chose en soi... et puis des fois le corps est tellement gardé en dehors de ça que on se met à danser et puis on essaie de communiquer la musique avec le corps... toute la culture DJ par exemple, où on essaie de se garder occupé là, à tourner des pots (“ potentiomètres rotatifs des interfaces ”, NDT) … y'a une autre façon d'engager le corps que ça.. et puis probablement la pire façon d'engager le corps c'est l'interface... [sortant son laptop et montrant le clavier] cette interface là qui est faite pour faire de la bureautique... écrire des courriels, faire des  fichiers excels... et puis c'est ce fait qu'en travaillant avec des danseurs, le fait c'est que tu as un mouvement, le mouvement du corps mais un mouvement qui n'est pas nécessairement instrumental, qui est comme une autre façon d'arranger … de faire participer le corps avec le son... 

VG —  il n'y a peut être pas tant de volonté de contrôle, tu veux dire ? … par rapport à un instrumentiste qui contrôle son instrument ? 

PSD — ouais, c'est ça, ben c'est un autre genre de … c'est un autre type de contrôle effectivement qui est moins fin [faisant un geste avec le bout de ses doigts] … effectivement qui peut-être plus conceptuel... par exemple quand je prends des... par exemple avec les accordéons, avec le chandail qui permet d'aller la respiration de l'interprète et puis de le transférer aux accordéons... donc là y'a un transfert “ Homme-Machine ” [ajoutant des guillemets à l'expression avec ses mains, NDT] … un anthropomorphise de... projections du corps sur le robots. Donc on n'est oas dans un contrôle d'interprétation de type instrumental. 

VG —  Je pensais au fait que le danseur, dans sa pratique, est plutôt dans une pratique où le corps s'exprime en tant que tel, pour lui même, sans avoir besoin d'un instrument, alors que l'instrumentiste a besoin d'un instrument … 

PSD — oui...effectivement, y'a … et puis il y a plein d'autres façon par exemple avec la vision par ordinateur d'aller chercher le corps en mouvement, avec OpenCV, la Kinect... et puis de … de transposer ces gestes là au son.. ç m'intéressait beaucoup.. peut-être un peu moins maintenant.. je sais pas, peut-être qu'on le voit trop...euh... mais oui, donc cette gestuelle là, sur scène et puis aussi toute le workflow, toutes les méthodes de travail qu'il y a en danse ça m'intéresse beaucoup... le temps qu'ils vont passer ensemble...parce qu'il n'y a pas vraiment de notation pour la danse... y'en a mais mais elles sont pas pratiquées hein, y'a personne qui prend la notation Laban et puis qui fait ah.... [mimant le fait de lire et comprendre une partition Laban] … donc ils sont obligés de se parler dans le processus de création, et puis d'arriver comme ça et puis de.... de.. et puis là émergent des discussions de … en laboratoire... ce qu'on fait très très très peu en musique … en musique faut que ça soit efficace [tapant des main pour indique le rythme].. un show faut qu'ça se monte en trois répèts...etc... Passer seize semaines à monter un show c'est … [mimant le fait que c'est impensable avec ses mains]... très rare... les chanteurs le font un petit peu plus... se pointer à une répétition sans savoir ce qu'on va faire...  

VG —  ce serait plutôt l'équivalent de faire un album, ou un truc comme ça, presque ? 

PSD — oui, tout à fait...avec un groupe... oui, oui... dans les musiques populaires... tout à fait c'est plus près de ça... ce monde là que j'ai trouvé dans la danse, qui était réceptif à ce genre de travail que je faisais, qui était prêt à passer du temps, à développer une gestuelle qui... parce que c'est sûr qu'eux vont développer... si j'arrive avec une “ scéno slash instrument “ , eux vont aussi développer une gestuelle qui fait sens en terme d'interaction, par exemple, donc les deux se nourrissent... et voilà... c'est un peu l'espère de filon dans lequel je travaille en ce moment... 

VG —  et par rapport à cette idée d'instrument, une autre différence que j'imagine entre ce qui est scénographique et l'instrument, c'est que l'instrument... acoustique en tout cas, l'instrument traditionnel il est réutilisé, pratiqué, pendant toute une vie, pour plein d'instrumentistes... alors que là j'imagine que ce sont des dispositifs qui sont plus éphémères.... ou.. comment tu travailles par rapport à ça ? 

PSD — absolument... ben la question de... premièrement la séparation entre l'instrument et l'œuvre, donc l'instrumentiste peut jouer, va jouer dans sa vie plein d'œuvres... etc...  moi je conçois plutôt que l'instrument c'est l'œuvre... et puis, bien sûr que je peux jouer plein de musiques différentes avec.. bon,.les accordéons, surtout... mais ça ferait pas sens. Y'a … ouais, c'est pas quelque chose qui m'intéresse... moi ce qui m'intéresse c'est de développer... de faire un instrument avec lequel je vais peut-être aller chercher, allez, deux, trois performances maximum, la plupart du temps, une, si je fais un instrument qui est très limité, donc c'est un “ œuvre -instrument ”, que je vais performer. Donc ça c'est un volet. Après ça, la question de la pérennité des œuvres... euh... c'est sûr que la musique instrumentale est habituée à... [début de raillerie] ben on écrit une partition et là elle peut partir pour les siècles des siècles (rires)  quelqu'un dans 600 ans dans une autre culture complètement différente va pouvoir reprendre ton œuvre et la rejouer... [fin de raillerie] … ça ça ne m'intéresse pas du tout, du tout, c'est pas comme ça que je veux contribuer à la culture, pas du tout. Durer... ben sur Youtube... je vais durer par l'archive visuelle de mes œuvres, qui est donc très importante.. documenter son travail c'est … donc une façon de contribuer à la culture, comme ça... mais pas dans une optique de re-performance. Donc moi, je veux faire mes trucs, je veux m'amuser avec mes amis, on va faire des spectacles, les spectacles vont avoir une durée de vie, qui va être la leur, qui va trouver sa résonance dans le milieu ici... Et puis après ça, ça se retrouve sur Youtube, sur Facebook, et puis ça influencera peut-être quelqu'un d'autre, et puis ça aura la résonance que ça va avoir... donc … c'est sûr c'est un autre euh... le théâtre c'est comme ça, le théâtre fallait... fallait être là ! Quand c'est arrivé. Le spectacle de danse c'est comme ça. Ou il y a des danses qui sont reprises, etc, mais … 

VG —  des pièces aussi de théâtre … (non?) 

PSD — oui, mais le metteur... la mise en scène est scène est tellement importante...  

VG —  oui 

PSD — par exemple, je sais pas... le cirque.... fallait aller au spectacle de cirque, pour voir le truc...si on n'était pas là, on n'était pas là, voilà, c'est pas grave, y'en a un autre (qui vient après) donc c'est un peu comme ça que je perçois mon travail...  

VG —  donc, pour cet instrument là (les écrans de feuilles), tu vas te cantonner à une esthétique, comme tu disais tu pourrais faire plein de choses avec, mais tu veux faire une œuvre musicale d'un choix de jeux, de performances qui vont être spécifiques … 

PSD — ouais, si tu veux ça c'est mon genre de devoir, que j'essaie de faire, c'est à dire de … une fois que je me rends compte de ce que j'ai fait, comme instrument, tu sais quand c'est un projet, tu avances dans ton projet, et puis là tu as des itérations, et puis à un moment donné oups, le projet devient un instrument, et puis c'est là que tu prends un peu conscience de “ oh ! J'ai fait ça?! ” et puis là j'essaie déjà d'écouter, ce que j'ai fait, d'aller trouver c'est quoi les affordances spécifiques à la patente ["truc", "bidule", "machin" au Québec] que j'ai fait. Et puis par exemple dans le cas des accordéons, le fait que j'en ai 5, le fait qu'ils sont sur roues, avec des batteries de Skidoo... je sais pas si tu sais c'est quoi un Skidoo ? C'est un moto-neige... donc tu peux avoir les accordéons en mouvement dans l'espace sur un plateau scénique, donc ça vient influencer le jeu aussi, le fait que ça peut jouer vraiment plus vite qu'un être humain... donc il y a là une zone... c'est la zone que je recherche, la zone qui est unique à ça... et puis le fait qu'il y en ait 5... par exemple dans le spectacle de flamenco si tu veux, le spectacle commence ...(attrapant des accessoires) ... puis tu vois j'étais une peu habillé en marin, et puis ici j'ai une lumière de marin …. avec une petite centrale inertielle là dessus... et puis je la fait tourner comme ça (mimant un geste de tournoiement au dessus de sa tête)... pendant longtemps... et puis à un moment donné, quand le truc passe devant un accordéon, ça fait “ prrrrr ” et puis l'accordéon se met à faire des notes... et puis si y'a, dépendemment de la position de l'accordéon, [mime les bruits de déclenchement de chaque accodéon à différentes positions de l'espace] … ça c'est une chose qui m'intéresse, à développer pour l'instrument... mais juste envoyer des notes pour envoyer des notes [non, de la tête].. ça c'est pas l'instrument... par exemple, comme le truc des instruments-robots qu'y en Europe...c'est en Allemagne qu'il ya  … [cherchant] … un truc de fou..  

VG —  le projet de Squarepusher ? 

PSD — Pat Metheny.... [son projet Orchestion, NDT] 

VG —  Pat Metheny a fait ça aussi ? … Y'a pas mal de musiciens qui ont fait des trucs comme ça... Stephan Eicher a fait des trucs du genre.. 

PSD — Oh mon dieu ! … Stephan Eicher... je suis allé cherché dans les années 90s 

VG —  Oui, il fait un truc avec Max/MSP... 

PSD — ah oui ? 

VG —  oui, je l'ai entendu interviewé à la radio, j'étais impressionné de ses connaissances là dessus... 

PSD — oh ! Par exemple (sortant son laptop), je vais te montrer, j'avais fait une patente à Mutek, où j'ai fait des “ compilations ” pour parler de ce que je viens de te parler... 

VG ...
(more transcript to come) % ok 27/05/2018
	\chapter{Interview : Nicolas Bernier}
\label{appendix:bernier}

\section*{Biographie}

\noindent Nicolas Bernier crée des performances et des installations audiovisuelles visant à sculpter un dialogue entre le son et la matière tangible. Formé par son travail dans les domaines du cinéma, de la littérature, de la danse et du théâtre, son propre langage mêle des éléments de musique, de photographie, de design, de science, d'art vidéo, d'architecture, de lumière et de scénographie. Au milieu de cet éclectisme, ses préoccupations artistiques restent constantes : l'équilibre entre le cérébral et le sensuel, entre les sources organiques et le traitement numérique.\\
\indent Lauréat du prestigieux Golden Nica au Prix Ars Electronica 2013 (Autriche), son œuvre est largement reconnue et présentée dans le monde entier : SONAR (Espagne), Mutek (Canada), Elektra (Canada), ZKM (Allemagne), Transmediale (Allemagne) et LABoral (Espagne) pour n'en citer que quelques-uns. Ses compositions sonores sont largement publiées sur les labels de musique électronique : 901 Editions (Italie), LINE (États-Unis), leerraum (Suisse), Entr'acte (Royaume-Uni) et empreintes DIGITALes (Québec).\\
\indent Il est titulaire d'un doctorat en arts sonores de l'Université de Huddersfield (Royaume-Uni). Il est membre des centres de recherche et développement en arts médiatiques Perte de signal, CIRMMMT et Hexagram basés à Montréal. Il enseigne dans le cadre du programme de musique numérique de l'Université de Montréal. 

\section*{Transcript}

\noindent Nicolas Bernier, interview du 28/05/2018, dans un café à coté l'Université de Montréal, Canada. Les termes quebécois sont traduit en français au fil du texte, à leur première occurence.
 
VG — maintenant tout ce qu'on dit est enregistrée 

NB — ``tout ce que vous dites peut être retenu contre vous'' 

VG — ... avec votre accord 

NB — que puis-je ?

VG — alors... je vais te poser des questions assez générales, mais elles n'appellent pas du tout à une réflexion générale, c'est vraiment ton approche qui m'intéresse ... 

NB — voyons voir si j'ai quelque chose à dire 

VG — en premier lieu, qu'est ce qui t'a amené aux instrument numériques? est-ce que tu avais une pratique musicale acoustique avant de t'intéresser au son digital ?

NB — oui, ça on peut dire mais pas... je viens du rock \textit{grosso-modo}... 

VG — et qu'est ce qui t'a amené à utiliser les technologies numériques plutôt que de faire de la guitare électrique ou du piano ? c'était quoi la motivation ?

NB — ouais, c'est quand même une bonne question ... ça remonte à quelques années quand même... tu sais faut que tu te demandes à cette époque là quand j'ai commencé, qu'est ce qui m'a amené ... je peux pas donner une réponse comme récente là... mais je pense que c'est juste \textit{l'infini des possibles} qui ... avec l'instrumental puis avec ma ... c'est une question de capacité moi je venais de la musique pop hein... donc trois accords et quelques rythmes différents mais tandis qu'avec les sons... dans le fond c'est pas tant le numérique... c'est ça, ça c'est une bonne réponse quand même, le numérique je m'en fous un peu mais les sons, les autres sont intéressants, puis quand on se met à pouvoir faire de la musique avec tous les sons... voici... ben c'est sûr que ça ouvre... tout à coup il ya plus de limites donc ça je pense c'est une des grandes motivations... 

VG — pour autant, enfin du peu que je connais ton travail, c'est très électronique ce que tu fais... ou bien tu fais du field recording, des sons concrets je veux dire... quand tu parles de "tous les sons"

NB — quand je parles de tous les sons, je parle de tout ce qui est pas nécessairement instrumental, puis après est-ce qu'il a des ... quel type de son j'utilise, ben justement j'utilise tous les sons, je travaille avec des sons d'instruments, j'ai travaillé du field recording, j'ai travaillé avec l'enregistrement de non-instrumentale en studio, j'ai travaille avec des sons électroniques, mais ça c'est plus récent en fait... mettons, la synthèse c'est vraiment récent, j'ai eu vraiment ... en bon enfant de l'école Schaefferienne, pour moi c'était... ce qui est tout à fait irrationnel de toute façon c'est une façon romantique de vendre la musique concrète qui serait basée sur l'enregistrement acoustique... mais dans le fond c'est juste notre façon de traduire la musique concrète parce que la musique concrète ça voulait pas dire... ça n'était pas anti-synthèse mais moi j'étais anti-synthèse... c'était un \textit{statement} ... tu sais j'étais pas de l'école allemande... Stockhausen ... j'étais de l'école ...française ...donc c'est ça, mais récemment j'ai pu me débarrasser de mes démons et puis...

VG —  qu'est-ce qui faisait que tu étais anti-synthèse ?

NB — ben c'est ça, je pense qu'il y a quand même ... la synthèse j'associe ça à ... et puis encore une fois c'est toujours ça, on a des biais qui sont plus ou moins justes, mais j'associais ça peut-être au contrôle absolu sur tous les paramètres. Tandis qu'avec l'enregistrement acoustique, j'ai l'impression que ça dévoilait, par transformation même simple, j'ai l'impression que ça dévoilait tout le temps dès aspects inouïs qu'on n'aurait jamais pu imaginer dans son et sur lesquels j'aurai pas nécessairement de contrôle, en tout cas...

VG — une part d'imprévisible ?

NB — ouais... donc voilà ... mais ... puis peut-être, je reviens à ta question, ta première question, c'est quoi ta première question ? ah oui, vers le numériques c'est ça...  donc c'est ça c'est pas tant le numérique, et moi à la rigueur faire de la musique de bande, j'aurais vraiment aimé ça, j'en ai fait une... un désastre total... (rires) ... mais j'aurais aimé ça travailler à la bande 

VG — à la bande ... magnétique ?

NB — avec un couteau, oui, un couteau et du papier collant... mais tu vois la raison, une autre des raisons en fait c'est ça c'est drôle parce que, moi initialement je m'intéressais à la musique contemporaine en général, sans éducation musicale, tu sais dans le fond moi c'est ça, je viens de... je vais essayer de faire l'histoire courte là... mais disons t'es adolescent, tu fais du rock, après t'arrives à Montréal, tu viens d'une région ... de banlieue, il n'y a pas grand chose qui existe, t'arrives à Montréal, tu découvres un peu l'impro, tu découvre qu'il ya d'autres sortes de rock ou ça chante pas, donc là tu penses au post-rock et puis là finalement t'incorpore un peu de jazz, t'incorpores l'impro, là tu te rends compte qu'il y a la musique répétitive qui existe, puis là tu te rends compte que la musique contemporaine existe et puis moi dans le fond, je m'intéressait *aux* musiques, au pluriel, contemporaines, et puis je voulais plus me diriger vers la musicologie, sauf qu'à un moment donné je me suis rendu compte que 1) que pour rentrer à la faculté ici, donc moi j'avais pas d'étude en musique, donc pour rentrer à l'université en électroacoustique, on n'avait pas nécessairement besoin d'un fort bagage en théorie musicale, ça prend un minimum mais c'est tout... puis ensuite je me suis dit plutôt que d'étudier la musique, cette musique là m'intriguait beaucoup, l'électroacoustique, l'acousmatique, je comprenais pas ... je comprenais pas vraiment ... tu sais, c'est un des moments marquants, quand même, de ma vie le concert acousmatique où on s'assoit, il y a personne sur la scène, t'es encore là tu viens du rock, de ta région, et puis t'arrives à Montréal et tu t'assois dans l'concert, personne sur la scène, du son partout partout, et puis le concert finit, les gens applaudissent,  tu te demandes vraiment... \textit{what the fuck} ?  Donc je me suis dit à la place de l'étudier, je vais la jouer, je vais l'apprendre et c'est comme ça que je suis rentré un peu ... donc c'est une combinaison de... ça me semblait être plus ouvert parce qu'après ça, tu sais avec tous les groupes de musique pop ben c'est sûr que si tu fais du reggae, tu fais du reggae, et puis tu fais du reggae longtemps, et puis si tu fais du ... métal, tu fais du métal longtemps, et puis si tu...  c'était difficile de sortir des carcans ... de la musique électro-quelque-chose m'a semblé plus... plus encline à faire ce qu'on veut... 

VG — des mélanges... des collaborations?

NB — ouais, et puis qui interdit pas la récupération d'idiomes pop ou rock non plus ... voilà, ce qui fait que ça, plus combiné au fait que c'était relativement facile d'intégrer le milieu ... c'est un peu ça qui a fait que je me suis ramassé là dedans ("se ramasser" : au Québec, "se retrouver dans un endroit sans l'avoir prévu ni voulu", NDT)...

VG — et quand tu parlais de bandes, et de montages que tu as travaillé au ciseau tout ça, il y a un côté cinématographique là-dedans ?

NB — Non... pas... en tout cas pas avec ... non pas tu tout en fait...  je veux dire dans mon travail il y a un côté cinématographique, j'ai commencé avec la vidéo en fait c'est plus... c'est des choses qui se sont oubliées un peu mais dans le fond toutes mes premières œuvres c'était vidéo et puis j'ai un background en design graphique en fait donc j'ai toujours été très visuel, ça a toujours été  assez important ... mais quand j'ai travaillé avec la bande, et puis même si je travaillais encore aujourd'hui avec la bande, je pense que il n'y a pas de relation avec le film en tant que tel... je penserai "sonore"...

VG — pas de "cinéma pour l'oreille"... 

NB — ouais, non c'est ça... non... le cinéma pour l'oreille... 

VG — ça ne te parles pas plus que ça 

NB — non ... mais tu sais, je trouve ça très correct, les parallèles sont super intéressants... mais après, moi je vais pas m'asseoir je sais pas trop où... quand je me mets à travailler, je me dis pas que je vais faire du cinéma pour l'oreille, je me dis pas que je vais faire rien en fait... S'inscrire dans un courant, là, je sais pas trop de quoi tu vas me parler...  J'en parlais tantôt avec avec quelqu'un, je lui disais c'est drôle cette conférence là, TENOR (conférence sur les technologie de la notation et de la représentation, NDR), où t'as des gens qui vont comme, revendiquer leur "appartenance" au monde de la partition graphique et puis ... là je me rends compte que moi je me suis ramassé, j'ai un ensemble, je sais pas si t'es au courant mais j'ai un ensemble ici sur des vieux oscillateurs des années 50... par défaut on s'est ramassés dans la partition graphique parce qu'il faut bien qu'on trouve des façons de jouer ensemble, puis de lire, puis de transmettre, puis... j'ai jamais été là-dedans mais là tout à coup avec un groupe, faut qu'on se structure un peu... donc là, tout à coup je me ramasse un peu à mon insu dans ce milieu là, il ya plein de gens qui me contactent et puis qui me disent (ton emprunté) "ah oui, toi aussi tu travailles sur la partition graphique, tu peux tu me donner... c'est quoi qui t'intéresse, et puis tu travailles sur quoi... " ... mais, moi c'est juste un moyen parce que bon faut que l'on fasse des musiques, mais ce n'est pas une fin en soi, c'est pas un intérêt plus qu'il faut... c'est juste que c'est un peu... une obligation (rire) en quelque sorte...

VG — c'est des outils dont tu as besoin pour arriver à une finalité..

NB — ben pour, ouais, pour faire de l'art... ce qui m'intéresse c'est l'art, c'est la seule chose qui m'intéresse ... \textit{quote} : "la seule chose qui m'intéresse c'est l'art. Nicolas Bernier, en face de l'église, 2018" (rires) 

VG —  dans les choses qui m'intéressent dans le numérique, ce qu'il y a de particulier notamment, c'est le fait que par rapport à des instruments classiques tu as une possibilité de disruption très forte, tu peux faire des ruptures toutes les nanosecondes si tu veux, toutes les millisecondes, on va dire... et ça change un peu le rapport... 

NB — Quoique la disruption... Là je pense en temps réel mais ... je ne sais pas si c'est un...  parce que tu sais au début, excuse moi je te laisse même pas finir ta phrase, je renchéris déjà, mais allons-y ... disruption...  parce que tu dis, bon, disruption, ça "permet" la disruption ... le numérique... bon, on dit numérique mais l'analogique le permettait déjà, de 1, donc déjà quand on utilise le numérique faut faire attention avec l'utilisation du terme, et puis de deux, je me dis ouais mais les instruments acoustiques aussi permettaient la disruption, et puis là tout à coup, sauf que tu dis ouais mais l'on peut à la milli-seconde ou à la nanoseconde, et là je me dis ah ouais ok c'est vrai, on peut peut-être pas faire ça avec des instruments acoustiques...  sauf que là si on fait des disruptions à la nanoseconde... tout à coup c'est peut-être plus de la disruption, en fait parce que pour qu'il y ait une disruption faut qu'il y ait un certain temps, ça devient, tu sais pas comme la granulation, on pourrait dire que c'est de la disruption à la nano-seconde mais dans le fond c'est plus de la disruption, c'est de la création de masses, qui elles, pour être rompues, devrait avoir un... donc en tout cas "\textit{food for thoughts}" ... "nourriture à réfléchir" peut être... je te laisse continuer... 

VG —  peut-être que ce n'était pas un exemple très bien choisi, même si effectivement ça permet de faire ça à des fréquences qu'on ne pouvait pas faire avant, mais ce n'est pas uniquement au niveau sonore que je pense à ça mais au niveau de la relation entre le geste éventuel qui va générer un son, ou d'autres choses d'ailleurs, les outils qui permettent de contrôler le son, la lumière, la vidéo, ont tendance à fusionner un peu, dans des logiciels comme Max ou on manipule des données... les relations que tu établies du coup entre la personne ou la machine qui contrôle la musique, qui produit la musique entre le geste et le résultat, tu as une diruption possible, tu peux changer tout le mapping n'importe quand ...

NB — ça oui, on est d'accord ... (mimant sur la table un geste très doux, et faisant subitement un bruit très saturé avec la bouche) petit côté théâtral... 

VG — du coup ça a des conséquences...

NB — ... sur la transmission, oui

VG — il n'y a pas une tradition... si tu donnes des cours à la fac de musique sur les technologies numériques, il y a un contexte assez différent entre le fait d'enseigner le piano ou le violon où tu as des traditions et des techniques qui sont pérennes, en tout cas plus ou moins établies depuis des dizaines ou des centaines d'années, et des outils où il faut ré-inventer les choses à chaque fois, à la fois parce qu'elles "permettent" des relations qu'il faut re-définir à chaque fois et puis les technologies eux-même sont moins stables que le bois et le cuivre, et sujets à des mises à jour de système et des choses comme ça. C'est à la fois des contraintes et des possibilités de création, mais qu'est ce qui t'amène à utiliser des outils qui ne sont pas forcément plus stables et facile à utiliser ...

NB — ouais, sauf tu sais, pour quelqu'un comme moi... enfin ça dépend de ta culture aussi ... mais pour moi c'est beaucoup plus facile de manipuler un outil disons informatique que ... d'apprendre le violon avant vingts ans avant de peut-être pouvoir jouer un peu correctement ... il y a une certaine facilité quand même qui vient avec...  on peut s'adapter, on peut changer, il y a pas le fardeau de centaines d'années de musique tonale... encore là... toutes ces questions là... je suis comme ça moi... je me dis qu'il n'y a pas d'absolu

VG — et tu parlais tout à l'heure du fait de toi tu te fichais, on a évoqué ça tout à l'heure (avant l'interview NDR) de comment transmettre les oeuvres, comment faire qu'elles durent, qu'elles puissent être rejoué dans dix ans, tu disais que tu t'en fichais complètement 

NB - ouais... voilà, c'est dit ... officiellement... (rires) non mais c'est parce que moi j'ai... c'est ça j'ai peut-être aussi parce que justement ça c'est, justement, un réflexe qui vient de la Musique, de la Musique avec un grand M, d'instruments de bois et de métal, la pérennité donc moi je suis tellement pas là-dedans, surtout que quand ... ça dépend... moi je fais surtout bon, de la performance, des installations... la performance, je l'écris pour moi, pour jouer, pour ... c'est moi qui joue... si c'est pas moi, ça se peut là qu'un jour j'écrive des performances pour d'autres personnes mais... ce n'est pas le premier réflexe en tout cas...  j'ai envie d'être sur scène..  ça fait partie de ... je suis compositeur, mais  en même temps je suis musicien pop aussi donc ... donc pour moi la performance vient avec, quand je meurs, la performance meurt avec moi... je vois pas vraiment l'intérêt que mon oeuvre soit jouée encore, interprétée par quelqu'un d'autre ... et puis ensuite les installations, ça c'est peut-être plus intéressant, mais là c'est comme un autre domaine c'est plus proche des arts visuels, tout ça... et puis j'ai une installation qui a été achetée, qui est dans une collection permanente, en France justement, et puis je me dis que quand je vais mourir ben y'a quelqu'un que lui c'est son travail dans la vie c'est de faire en sorte que cette installation là existe indépendamment de mon être et c'est \textit{correct} (en Québecquois: ``vraiment bien'', NDT).

VG — c'est quelle installation ?

NB — c'est tout petit, ça s'appelle "Frequencies (a / friction)"  c'est un oscillateur sur une table lumineuse, un oscillateur, un diapason... l'oscillateur est à 438Hz et le diapason est à 440 ... l'oscillateur est constant, et le diapason qui tape dessus à interval aléatoire et donc quand le diapason est activé on entend le batement... c'est tout simple ... c'est tout simple mais je l'aime beaucoup...puis voilà c'est quand même chouette ... mais non, moi j'ai pas d'ego d'artistes, il y a des artistes qui se disent je veux laisser ma marque sur la terre, et on va se rappeler, on va se souvenir de mon nom ... j'ai pas cette prétention là, je fais mon petit truc et voilà ...

VG — mais tu donnes des cours...

NB — ouais

VG — ... donc d'une certaine manière tu es dans la transmission à ce niveau là... qu'est ce que tu estimes utiles ou intéressant de transmettre dans une pratique comme ça ?

VG — oui, dans un cours où tu as beaucoup à inventer, 

NB —  ouais - les cours que je donne c'est grosso modo des cours de composition, mais au sens élargi ... je fais des cours au bac on appelle ça des "cours-projet", c'est à dire que ton projet ça peut être la composition stricto-sensu, de la composition musicale, mais ça peut être aussi construire un instruments pour faire une performance, ça peut être de l'installation. Voilà, une des choses — c'est un peu abstrait, l'une des choses qui m'intéressent, c'est la cohérence du propos artistique, qui fait que peu importe que les outils, les méthodes, les traitements... que tout ce que tu va utiliser pour faire de l'art soit en phase avec ... un processus... des intention qui auront peut-être changé en cours de route, c'est essayer de garder un peu la cohérence dans tout ... ça c'est la première chose qui m'intéresse de transmettre, c'est un peu abstrait peut-être ?...  pas tant que ça ...

VG — du coup, c'est pas forcément... c'est plutôt un travail de guide, d'encadrement? 

NB — oui oui c'est ça... c'est à peu près tout ce que je fais, l'encadrement. Il y a le cours d'ensemble, ça c'est un peu différent, je me ramasse un peu avec le chapeau, un peu chef d'orchestre, en même temps pas trop chef non plus parce que toutes les partitions sont écrites, je suis plus comme un ... si je veux le dire de façon pas glamour, je suis plus comme un "coordinateur", ces trucs là, "directeur artistique"... mais sinon même mes cours, mon grand groupe, "grand" est toujours relatif mais en musique j'ai un groupe de 20 personnes à peu près, ça c'est un "grand groupe",  mais c'est quand même un cours d'initiation à la composition que j'essaie quand même d'inculquer ce dont je viens de parler, sauf que là c'est des étudiants qui entrent en  première année, qui ont vraiment peu d'expérience ou pas du tout, donc l'autre chose que j'essaie de transmettre c'est juste d'avoir... une conscience du développement du temps... c'est un peu... un classique du "compositeur" ... je pense que c'est quelque chose que je maîtrise relativement ... pas pire...(rires) la conscience du développement du temps ou de l'évolution d'énergie dans le temps... et puis je le dis comme ça parce qu'encore une fois l'évolution de l'énergie dans le temps, le développement du temps, ça s'applique pas juste à la musique, ça s'applique... à tout... ça s'applique à la façon dont je suis en train de te parler et puis je vais mettre l'emphase à un moment donné, je vais prendre le temps quelque part d'autre, ça c'est quelque chose qui m'intéresse, que je transmets... qui m'intéresse mais cela dit, sur laquelle j'ai jamais réfléchi, ou j'ai jamais été... si tu me demandes "oui  mais c'est quoi ta conception de..." j'en ai aucune idée, c'est quelque chose que ... que je sens... grosso modo... puis après ça bon c'est sûr qu'à un niveau, euh... ce cours là disons de groupe de composition son, c'est sûr qu'il y a quand même des choses techniques c'est la partie qui m'intéresse pas en fait... mais on va parler d'espace, on va parler de montage, on va parler de filtrage, on va parler de toutes ces choses qui vont aider éventuellement à développer ton temps comme tu veux... mais c'est pas ...

VG — tu n'attaches pas plus d'importance à ça...

NB — non

VG — une question par rapport aux objets techniques, est ce qu'il y a des outils que tu utilises de manière récurrente, et s'il y en a pourquoi, qu'est ce qui t'intéresse dans ces outils et qui fait qu'ils reviennent dans ton travail ?

NB — mouais, mon petit côté "baveux" (arrogant, méprisant en Québecois, NdT) aimerait répondre "non" à cette question ... mais bon forcément on utilise... mais pour vrai, moi j'utilise le moins d'outils possible ... je te disais tantôt que j'ai un passé en design graphique un peu, mais j'ai un passé aussi en programmation pour le web, c'est fin des années 90... à l'époque c'était ASP (cf. \gls{ASP}) qu'on faisait... ASP et SQL (cf. \gls{SQL}) ... et puis donc quand j'ai commencé à travailler sur le marché professionnel à 17 ou 18 ans dans un bureau avec un salaire... et puis là, ) un moment donné tu te dis bon, t'as 22 ans ça fait déjà cinq ans que tu travailles, tu te dis je vais pas faire ce jusqu'à ma mort... ça n'a pas de bon sens ... donc làj'ai décidé de retourner en musique, faire de l'art... pourquoi je dis ça?... ah oui! par rapport aux outils... qui qu'on programme beaucoup en musique maintenant ou dans les arts numériques, et puis moi c'est une chose que j'ai pas particulièrement le goût de faire, j'ai quitté un milieu pas pour... bon ça m'a aidé dans mes études j'avais de l'aisance là dedans plus que d'autres gens, sauf que je suis pas là pour ça tu sais... comme je disais tantôt ce qui m'intéresse, c'est la finalité qui m'intéresse, c'est pour ça que je vais engager des gens pour faire les choses, c'est une culture qu'on n'a pas beaucoup en tout cas à Montréal mais en France plus, tout l'assistanat musical là, nous autres ici c'est quelque chose qui n'existe pas vraiment, on est vraiment une culture \gls{DIY}, on fait tout ... il y a quelque chose d'un peu macho là dedans, tu sais, si t'as pas fait tout, si t'as pas ...  c'est pas "authentique", tu sais... mais moi je suis pas vraiment là dedans parce que au final moi, j'aimes ça jouer... j'aime ça entendre les choses mais je ne tiens pas à avoir fait la mécanique en arrière, bon en même temps j'en fais une bonne partie quand même parce que mes moyens sont limités ... tout ça pour dire que... pouvez-vous répéter la question monsieur? (rires)

VG — je te demandais s'il y avait des outils qui revenait de manière récurrent...

NB — ah ouais, c'est ça...  donc longue histoire pour dire que j'utilise le moins d'outils possible ... puis moi depuis plusieurs années ma plateforme c'est Live (Ableton Live, NDR), pourquoi Live? entre autres parce qu'il y a Max for Live dedans puis moi avec ces deux choses là, écoute, j'ai pas besoin de beaucoup plus... après... 

VG — c'est la rapidité du fait d'arriver à tes fins qui est la motivation principale ?

NB — ouais, puis de pouvoir faire un maximum de choses en changeant le moins, en ne changeant pas d'environnement constamment... ça évite des bugs je pense, tu sais, le fait de pas avoir untel qui communique avec untel, qui communique avec untel, avec untel qui revient à untel et puis... tout est dans la même fenêtre, tout est... ce qui fait que moi, ça me convient vraiment ... puis après, bon ça c'est un peu  les outils de base disons, mais après c'est sûr que chaque projet est quand même différent donc il va toujours y avoir... mais tu sais, je vais te donner un exemple où je n'ai pas fait... où je me suis pas écouté dans un projet récent, un gros projet, bon j'ai travaillé beaucoup son et lumière ces dernières années, et puis je travaille avec un microcontrôleur, que tu peux même pas dire le nom, je sais même pas si ça a un nom,  je suis tombé là dessus par hasard un moment donné dans mes recherches il y a dix ans, un gars dans un sous-sol aux États-Unis qui fait des petites cartes... à l'époque j'étais en train de travailler avec des dimmer-packs... je dis dimmer-pack, tu vois c'est quoi ? je ne sais pas le terme en français ...  les trucs pour les gradateurs de lumière, normalement t'as des \textit{shovel} où tu peux mettre quatre ampoules puis tu peux envoyer du DMX et contrôler ton éclairage... tu vois c'est quoi cette boite là, ce qu'il y a dans les théâtre pour contrôler l'éclairage ... tout ça pour dire que là j'étais avec ça et puis je me suis rendu compte, ça faisait pas très longtemps qu'on travaille avec les LEDS et puis à un moment donné, je tombe sur ce micro-contrôleur là, qui à la place de peser 10 livres, faire 4 canaux c'est juste une carte ça fait trente deux canaux, puis ça pèse rien, puis donc j'ai commencé à travailler avec ça, puis là tous mes projets sont construits sur cette carte là que je connais, avec un objet ... puis là c'est l'autre affaire, tu sais ya un objet dans Max qui communique avec machin et puis là est-ce que l'interface DMX, USB-DMX qui parle à l' objet qui ... (soupirs) ... là j'ai une formule puis bon il ya d'autres choses qui rentrent en ligne de compte moi j'ai besoin d'une rapidité à toute épreuve et puis les interfaces commerciales, style ENTTEC, qui sont assez connues dans le monde de l'éclairage semi-professionnel, ben ça va pas assez vite il y a des dropped-frames tu sais ça perd des images, et ça me convient pas, je suis tombé un peu par chance sur cette formule là, qui fonctionne super bien pour mon usage, donc la plupart de mes projets sont construits là dessus. Quand je pars sur un gros projet j'ai pas besoin de me dire/il y a beaucoup de choses qui/chaque projet est différent, moi la partie compliquée c'est la partie mécanique, tu sais la partie, euh...  j'ai besoin d'un parasol, telle couleur, tel matériau ... ça je trouve ça compliqué et puis la partie design industriel ... j'apprends, je connais rien là dedans et puis j'ai quand même pas le choix de ... la partie technique, technologique un peu, qui se renouvelle tout le temps c'est plus ça tu vois... la partie "outil" parce que je construis mes outils... c'est ça ... qu'est ce qu'on entend par "outil"... mes outils c'est toujours grosso-modo les mêmes, mais mon "dispositif" il est jamais pareil ...  ce qui fait que là je réutilise, je sais pas si tu me suis dans mon histoire, je réutilise tout le temps le même truc sauf que le dernier projet, gros projet, je me fais convaincre par des jeunes... trop ambitieux... que je dois changer tous mes outils... parce que ... X Y raisons ... ``ok ... ok ... on va tout changer'' ... mais là au final il ya des bugs, ya des machins qui communiquent pas avec d'autres machins, puis d'autres machins qui communiquent pas, puis là y'a trop de data, et puis on perd du data, puis là gna gna gna... Au final on finit par réussir à faire un prototype de peine et de misère qui annoncerait un projet somme toute quand même assez intéressant, sauf qu'on en parlait tantôt, faut que j'aille chercher les sous pour aller... puisqu'on a réinventé les outils, l'argent est parti dans les outils et pendant l'art ... parce que je te dis, moi ce qui m'intéresse c'est l'art...  puis là je me ramasse avec un prototype... pas d'art ...puis faudrait que j'aille chercher des sous encore pour faire l'art, puis je réussi pas, je sais pas pourquoi, j'ai pas le don, les gens n'aiment pas le projet, je sais pas trop...


VG — peut-être parce que tu ne demandes pas des sous pour du matériel, peut-être

NB — ouais c'est ça... ben oui, voilà, c'est ça... exactement... tu as mis le doigt sur quelque chose...

VG — un problème récurrent oui...

NB — donc ce projet là, ben ... poubelle ... et puis j'ai travaillé fort longtemps et comme je te dis, c'est pour avoir un prototype quand même assez chouette et puis là finalement, j'ai juste fait comme... arf ... poubelle... et puis tous ces processus sont assez longs... tu fais une demande de subventions, tu reçois la réponse un an après ... le projet démarre six mois après... tu travail pendant un an ... ce qui fait que ça veut dire qu'une oeuvre, s'il faut que tu fasses un protoype, ça prend deux ans et demi, ça veut dire que t'es cinq ans sur le même truc... moi je peux pas travailler comme cela... parce que je suis rendu après cinq ans ça ne m'intéresse plus en fait, mon projet m'intéresse plus me ... donc ça répond un peu la question, peut-être ? 

VG — oui, d'une manière tout à fait intéressante...

NB — ok tant mieux ....

VG — c'est moi qui ait perdu le fil de mes questions du coup ... 

NB — mais je peux peut-être te relancer en fait, je serai peut-être curieux, parce que par rapport à ce que je te dit, tu sais moi dans le fond les outils c'est pas tant ça la question, qui est plus ... dans le fond moi, c'est comme la scénographie, quelque part... moi c'est ça, je dis ``scénographie'' mais le dispositif, l'objet ...  parce que je crée des objets, je ne créé pas des instruments ... je crée des objets, c'est plus comme de... je pense plus proche de la sculpture que de la musique... 

VG — c'est des objets dont tu joues quand même un peu en live...tu fais des performances avec...

NB — ben c'est des objets, ouais c'est ça, ça va être ....  je sais pas c'est quelque chose comme .... un dispositif, mettons pour simplifier, de visualisation ou ... c'est ça, ça peut se rapprocher un peu d'un décor de théâtre, des fois ça peut être un instrument de musique, des fois ça peut être un peu tout ça...  mais pour moi c'est ça qui est compliqué... c'est à dire il faut que j'habite une scène, que j'habite un espace... moi j'ai une pratique grosso modo solo, j'ai pas une équipe de concepteurs qui vont me proposer ``ah! voici on a pensé travailler l'aluminium avec euh... ''. Il faut que j'essaye d'imaginer tout ça moi-même et c'est vraiment pas évident. C'est faire, tu sais aller voir les designers industriels, faire des plans, dire ``ok j'ai besoin d'eux un tube d'aluminium de cette grandeur, avec tel genre d'ancrage, qui va prendre tel genre de boulons, qui va se ranger dans telle genre de caisse, qui va pas peser plus que tant de livres, ou de  machin, ou de...'' ce qui fait que tout ça... et puis en plus, faut que je fasse de l'art avec ça... ça devient compliqué... j'ai un bel objet mais là qu'est-ce-que je fais avec ?... Ça m'a coûté cher à faire, il faut bien que je trouve quelque chose à faire avec ça. Je me suis mis une hypothèse qui me disait que je ne trouverai quelque chose à faire... Est ce que ça va fonctionner?... Je ne sais pas tant que l'objet existe pas. Et puis comme j'ai pas des budgets de recherche et développement qui font que je peux flamber de l'argent et puis dire ``ah oui j'essaye des choses''  mais c'est... ça passe ou ça casse tout le temps... 

VG — tu fais beaucoup les choses toi même ?

NB —  ben comment dire... je les imagine moi-même... après je coupe pas l'aluminium moi-même et je fais pas de travail manuel moi-même. La programmation, je vais en faire partie quand c'est trop ... moi d'habitude la programmation c'est vraiment plus, euh .... pour moi c'est très utilitaire... je fais pas des ``créations programmatiques'' tu sais, j'ai besoin que ça fasse ``ça'', et puis ça, ça va être une étape quand même assez simple, ok je vais le faire. Quand ça deviennt plus complexe, je m'embarque plus là dedans... ça m'intéresse pas, et puis j'ai pu le temps de toute façon. Ce qui fait que ça je vais engager quelqu'un pour le faire... J'ai pas engagé tant de monde que ça pour faire la programmation de ce projet là que j'ai mis à la poubelle... Il y en a un autre que j'ai mis à là poubelle. Et puis il y en a un que j'utilise ... cette application pour, justement, de suivi partitions graphiques. Ça je l'ai faite faire. C'est à peu près tout. Le reste, tu sais j'ai besoin de déclencher des choses, de lire telle donnée...ça je m'arrange, grosso-modo...

VG — par rapport à ce que tu disais tout à l'heure, du fait qu'avec la programmation, le numérique, tous ces outils hardware ou software où tu vas, enfin ou tu peux potentiellement interconnecter plein de choses pour faire...  et qui donnent des projets qui sont lourds à monter... parce que du coup, en voyant ça d'une manière optimiste et naïve on peut se dire que justement ça permet de faire une écriture de la métamorphose, parce que tu peux avoir un processus qui se transforme complètement durant le cours de l'installation, mais quelque part le fait que ce soit ... plus lourd ... en fait cette complexité et cette lourdeur de la programmation du coup, est-ce que tu as l'impression que ça influe la manière dont tu vas concevoir tes œuvres? 
Je veux dire, est ce que tu fais une œuvre très directe par rapport au fait que c'est justement, lourd techniquement, au fait que ça prend du temps

NB — ouais je pense que ... je pense qu'il n'y a pas nécessairement de corrélat à faire, enfin ça va dépendre de chaque projet... des fois oui, des fois tu dis bon ben là... mais des fois ... comme là, tu viens de parler, je sais pas comment t'as dit, t'as parlé d'écriture de la métamorphose? comme un peu une écriture algorithmiques? c'est ça que tu entends un peu par là ?  une forme qui va se développer, euh...? 

VG — on peut faire des choses interactive qui aient un scénario qui se développe dans le temps 

NB — ouais, imprévu tu veux dire ?

VG — qu'elles soient prévues ou scénarisées 

NB — ok 

VG — mais avec ces outils qui permettent, dans un ordinateur au sens large du terme, tu peux mettre beaucoup de mémoire et du coup ça permet potentiellement de faire des choses qui vont avoir une durée de performance où les choses se renouvellent changent et... ça permet potentiellement de le faire, mais peut-être que de le faire, ça génère des projets qui peuvent être plus lourds techniquement et longs à monter... et pour garder une certaine fraîcheur et que tu puisse arriver à ton idée avant que cinq ans soient passés, est-ce que ça a influencé ta manière de travailler ?

NB —  je sais pas, je n'ai... ça ne m'a jamais vraiment posé de problème mais peut-être que je ...  mais effectivement je vais essayer de forcer toujours, de toute façon, dans tout les projets, une certaine simplicité. Tu sais, j'essaie de ... Je veux dire, de toute façon un projet ça fonctionne tout le temps pareil, il y a tout le temps, bon t'as plein de belles idées, et puis il y a tout le temps un moment donné où tu te rends compte que tout ça n'est pas réalisable, et puis là faut tout refaire... et cette prise de conscience de ... ou cette façon de travailler, cette zone là, où il faut réorganiser les idées de départ, il y a des gens qui vont pousser la frontière le plus loin possible, tu sais ils vont essayer de faire le truc impossible jusqu'à la dernière seconde, et puis peut-être qu'à la toute fin ils vont réussir à faire ce qui était impossible ou peut-être qu'ils vont juste revenir à la case départ ... moi j'ai tendance à rapidement au plus simple. Tu vois j'ai plein d'idées et puis à un moment, qui fait que peut-être j'ai jamais eu, euh... Je me suis jamais rendu à cette complexité là, où je me suis senti un peu menotté par la complexité du projet, mais par exemple un projet avec Martin Messier, qui est un genre de... on peut appeler ça une machine ... une structure en bois, c'est assez haut, je sais pas si tu vois, c'est seize pieds de haut, seize pieds de large... je peux pas te dire en mettre mais... (me montrant dans l'espace ce que cela représente) c'est un instrument assez imposant, qu'on ne peut pas sortir dans notre chambre à coucher et puis se mettre à jouer ... donc un projet comme ça, c'est compliqué mais c'est pas tant au niveau des outils qu'au niveau de... si on veut répéter ce spectacle là, il faut louer une salle, idéalement faut qu'il y ait de l'éclairage donc idéalement il faut qu'il y ait un technicien... donc là si tu veux, et puis nous on n'a pas les moyens d'une compagnie de théâtre, on a les moyens/on partage... on est dans la ``catégorie musiciens'' donc un musicien, ou un artiste disons numérique, ça s'organise tout seul, tu sais...  nos budgets sont pas cinq fois plus gros parce qu'on a un instrument qui demande ça... donc ça c'est compliqué, et puis ça je pense que c'est un frein... c'est un projet duquel je suis plus ou moins content artistiquement d'ailleurs ... et puis d'après moi c'est, entre autres, parce qu'il était tellement complexe à gérer et puis là tout seul, faut que ça ce range. Tu sais ça va dans des caisses, des grosses caisses qu'il faut que t'entrepose, et puis quand tu l'entrepose ça coûte de l'argent ... Ça c'est un frein à l'art. Et puis tu vois, Martin, lui ç'en est un, c'est un peu à lui que je pensais, lui va pousser la limite le plus loin possible. Dans ce projet là que j'ai fait avec lui, ça fait longtemps que j'aurais fait comme ``ok, regarde, notre machine, elle va être deux fois plus petite et puis on va pouvoir la faire dans mon petit local, puis au moins, ça va être moins imposant visuellement, mais au moins on va pouvoir plus travailler avec, et puis apprendre à jouer notre instrument, et puis travailler le mapping, et puis travailler la composition, et puis travailler la... '' Mais lui... mais c'est un projet collaboratif ...  par définition on fait des personnes pas heureuses... (rires) Ce qui fait que lui a du mettre de l'eau dans son vin, et puis moi aussi... pour découvrir quelque chose qui est ni lui ni moi... Ce qui fait que c'est peut-être plus à ce niveau là, je dirais que je sens où la complexité des projets, à un moment donné, va avoir un frein, mais... en fait je dis ça, j'ai donné cet exemple là parce que c'est un peu le plus... pour moi, je te montrerai une photo tantôt, c'est le plus flagrant... mais tous mes projets c'est comme ça en fait... Chaque projet, comme là, tous mes projets, c'est tout des dispositifs sur lesquels je peux jamais répéter... qui sont tous un peu gros, ou tous entreposés euh... parce que j'ai pas les moyens, ça coûte cher à envoyer outre-mer...  donc la plupart de mes projets, je les fais une fois à Montréal, après je l'envoie en Europe, puis ça reste en Europe... j'essaie de... je le fais entreposer à gauche à droite pour pas avoir à payer le voyage du retour. Donc ça qu'est ce que ça veut dire, ça veut dire que je peux pas travailler, puis là évidemment, quand je le fais, disons la première fois à Montréal mais c'est tout le temps... c'est une première, puis une première, ben c'est jamais une version finale, ce qui fait que là ça veut dire que je peux pas répéter avec mes dispositifs... ça pour moi c'est un gros gros gros problème en fait... 

VG — qui conditionne la manière dont tu travailles plus que ce dont je parlais avant ?

NB — ouais ...  je sais pas si ça conditionne la manière dont je travaille mais en tout cas ça... ça met des freins dans le processus. Et puis à l'achèvement de mon travail. Contrairement à une compagnie de danse ou de théâtre, qui ont des budgets pour avoir un lieu, des techniciens... Martin Messier, lui a sa compagnie maintenant et puis il fait ça, quand il travaille sur un projet, il peut se permettre de louer une salle et puis de vraiment ... d'expérimenter son projet en salle  et puis de faire le plan d'éclairage, et puis de savoir comment ça va se passer pour de vrai. Mon dernier projet, c'est drôle j'en parlais avec un diffuseur, en espagne récemment, et puis qui me disait ``mais comment...'' —parce tu sais moi j'ai de l'éclairage un peu— et il me disait ``mais comme tu fais pour faire ça ??'' parce que lui fait de la musique électronique, il a son studio mais y'a pas nécessairement de positif physique qui va avec son travail... et je lui disais, écoute —comme le projet que j'ai fait dans son festival— je lui disais je l'ai monté, c'est comme trois murs à peu près de cette largeur là (me montrant avec ses mains), qui sont l'un en arrière de l'autre au début, et puis dans la performance, je vais ouvrir les murs comme ça, ce qui fait qu'à à la fin ils vont être tous placés comme sur une ligne, ce qui fait qu'au début, c'est juste large comme ça, mais à la fin c'est quand même 20 pieds de large, donc ça prend une certaine largeur, que j'ai pas dans mon appartement... donc moi je croise les doigts en me disant que ça va fonctionner, j'espère que ça fonctionne une fois sur place... Ça c'est une chose, et puis après, une fois sur place je sais que je suis dans des théâtres, je me dis ah, bon, donc je vais essayer de mettre une lumière, tel type, là, je parle avec les techniciens, et tel type de couleurs, peux tu m'essayer un spot ici et un spot là, et puis ... ce qui fait qu'à chaque représentation j'améliore des choses, j'ajoute une lumière, j'en ferme une là, je change un peu la disposition ... et puis à un moment donné je finis par avoir quelque chose de satisfaisant mais je le travaille en live dans mes répétitions, juste avant que le public entre en salle et puis c'est comme ça que je finis par concevoir mes chose parce que j'ai pas le luxe d'avoir un lieu... 

VG — je te pose une dernière question, en totale disruption, qui est peut-être une boîte de pandore aussi qui est la question d'Internet, du fait que ... j'ai interviewé Nick Collins qui me disait qu'avec l'accessibilité des outils numériques sur Internet, faire de la lutherie numérique ressemblait davantage à faire de la cuisine. Il prenait l'exemple d'un de ses étudiants qui tape sur google ``asservir un moteur'', qui télécharge le fichier arduino, et ça marche... Comment considères tu ça, et notamment le fait qu'on peut travailler très facilement en faisant du copié/collé, où tu récupères le travail de quelqu'un d'autre qui n'est pas forcément ton idée originale, donc on travailles avec quelque chose qui a été pensé, orienté d'une certaine manière par quelqu'un d'autre... est ce que ça ré-affecte le résultat de ce que tu fais ?

NB — c'est sûr que ça affecte le résultat mais en même temps, c'est l'histoire de l'humanité ... On ne fait que ça, du copier-coller

VG — et de s'influencer les uns les autres...

NB — ben oui ... tant mieux si c'est plus facile ... après est ce que c'est garant d'un intérêt de ton œuvre, évidemment non. Si tu prends une pièce pour guitare de tel compositeur célèbre et puis que tu la recopies... tant mieux, on peut progresser là-dedans...  est ce que c'est que garant de l'intérêt de l'œuvre pour guitare, ben non, ben c'est la même chose avec l'électronique.  Après c'est au public, ou aux pairs, ou un mélange de ces deux choses là de dire, de voir s'il y a un intérêt là-dedans, d'une part... Ou sinon, si la personne qui le fait grandi là-dedans, tant mieux ... mais je sais pas... Nick Collins voyait ça d'un mauvais oeil ? 

VG — c'était pas forcément un jugement positif ou négatif

NB — plutôt un constat ?

VG — c'était plutôt positif, en tout cas lui vient d'une période où il n'avait pas toute la documentation qu'il y a et il a du créer des choses en faisant du reverse engineering...

NB — ouais ça m'aurait étonné, parce que c'est sûr il ya des gens qui pourraient avoir un discours un peu péjoratif par rapport à ça... mais Nick Collins, il me semble, c'est son but justement que ça devienne accessible, l'électronique, et je me dis il devrait être content, il devrait se réjouir...

VG — oui, c'était plutôt positif

NB — l'autre truc positif, c'est que si on n'a pas à apprendre la mécanique, si la mécanique se fait toute seul en tapant dans google, mais ça veut dire qu'on peut faire de l'art peut-être plus rapidement, et ça veut dire qu'on peut peut-être passer moins de temps à parler des outils, et puis qu'on peut passer peut-être plus de temps à parler de qu'est ce que tu aurais fait avec ces outils là, de forme, de temps, de matières, des choix, des intentions, qui pour moi — c'est personnel—  je dis ça parce que je me dis peut-être que lui il se dit ben là ça m'intéresse pas tes affaires, de forme et puis de matières... mais je pense que c'est une bonne nouvelle si on peut parler plus de l'art plutôt que de demander ``mais comment c'est fait?...'' Je suis pas obligé de te demander comment c'est fait, je comprends tout maintenant, je sais, c'est pas compliqué...  t'as branché un moteur dans un arduino avec une ligne de code qui fait \textit{random} sur le temps, machin... Mais qu'est ce que tu voulais dire avec ça? Ç'est la discussion qui me semble plus enrichissante. 

VG — Pour faire l'avocat du diable, j'aurais tendance à dire qu'Internet a des bons côtés et des mauvais côté, mais je ne sais pas si elle favorise un esprit geek, mais elle ne l'empêche pas en tout cas... c'était aussi par rapport à ce biais là...

NB — ouais... favorise un truc geek ..?...

VG — il y a plus de discussions peut-être autour des objets techniques et des plateformes...

NB — ouais, je sais pas, j'ai 40 ans, j'ai vécu 40 ans sur cette planète, est ce que ça n'a pas toujours été là les discussions techniques? Je veux dire, on se retrouve autour d'une auto puis on parle de la mécanique, et puis on part du moteur, et puis on parle du...  on est un peu fait comme ça je pense... A défaut de parler de l'esthétique... ``Pourquoi as tu mis cet aileron? Pourquoi ce type d'aileron?'' (ton emphatique) ``— c'est pour la métaphore de... (rires)''

il y a une facilité des outils qui est là avec Internet... Ça rend la recette facile.. C'est un peu la même chose que dire tout le monde peut faire de la musique, tout le monde peut faire de la vidéo, tout le monde peut faire de la photo...  Est ce que ça veut dire que parce que tout le monde peut faire de la photo avec son téléphone, que toutes les photos qui se font ont la même valeur? ... Pas vraiment...

VG — oui... il y en a plus en quantité...

NB — L'histoire va connaître celles qui ont vraiment de l'intérêt...

VG — Et si je peux te demander un pronostic, sur ces instruments numériques qui n'ont pas l'équivalent dans leur durée historique des instruments acoustiques, est ce que tu penses, souhaites, ou ne souhaites pas que les instruments numériques vont se cristalliser dans des formes particulières, comme pour les instruments acoustiques qui se sont établis de manière assez stable ...

NB — ouais... est-ce ce que je souhaite... je pense pas que je le souhaite... mais tu sais... j'en ai rien à cirer....  il va sûrement avoir des instrument, c'est sûr qu'il va y avoir des instruments numériques qui vont se standardiser... C'est aussi bête que les controleurs à boutons... qu'est ce qu'un contrôleur, qu'est ce qu'un instrument .... un contrôleur avec un ordinateur, ça va se standardiser, si on prend ça pour acquis, on le considère même pas comme un instrument...  mais quelque part ... j'ai des boutons rotatifs, des push buttons puis c'est relié à des paramètres, si ça c'est standardisé, tant mieux ... c'est sûr qu'il va en avoir, et puis il y aura sûrement des trucs un peu fou quand même un jour qui vont être inventés et puis qui vont êtres normaux et puis qu'on va jouer ... ouais, c'est cool... tant mieux....  mais j'espère en fait qu'il va continuer à y avoir tout ce pan là d'inventions qui vont être éphémères peut-être... Encore une fois, je ne vais pas dans le sens du développement d'un instrument qui va être utlisé, ma pensée va vers créer un objet d'art, et pour créer cet objet d'art là, j'ai peut-être besoin d'instruments qui existe mais j'ai peut-être besoin d'instruments qui n'existe pas puis le projet d'après ben j'aurai sûrement pas le goût de réutiliser le même instrument parce que ça va être une autre œuvre d'art, donc ça veut dire que ça va être des idées différentes, avec un objectif différent, et puis ça j'espère que ça va continuer à exister... Quelque part c'est un peu... euh... un peu moche la période du synthétiseur où tout le monde, si tu faisais de la musique électronique, tu faisais du synthétiseur. C'était la seule option, oui y'avait plein de sortes de synthétiseurs, mais ... c'est cool en ce moment qu tu te dis je peux faire ce que je veux... et puis en même temps cette époque là on l'a déjà vécu, la révolution industrielle, ils inventaient des instruments... sans arrêt... du côté de la musique on en inventait... tu sais je suis un peu là dedans...les vieux livres, le matériel scientifique, ils inventaient plein de... c'est fou, ça ressemblait quand même drôlement notre époque d'aujourd'hui où on avait l'impression qu'on pouvait tout inventer...  ça je trouve ça cool... on voit que la boucle, ça revient au début de la conversation quand tu me demandais pourquoi je faisais ça... parce que tout est possible... Il me semble que j'aimerais garder ça... garder cet aspect là... de sentir que tu tout est possible parce que si ça se perd et puis on commercialise, et que tous les instruments sont pareils ... ``not my cup of tea''... 

VG — La biodiversité de la technologie ...

NB — ouais...tu vois quand tant de forme —tantôt je te parlais de forme— l'infini des possibles... là c'est \textit{basic}, on finit par le plus ennuyant de la forme, mais dans une conversation c'est peut-être un peu moins ennuyant que musicalement... Je suis bon là dedans à créer de la forme...(rires) quand on retourne au point de départ... voilà... y a quelque chose à faire avec ça ?



 % ok 28/05/2018
	\chapter{Interview : Jose-Miguel Fernandez}
\label{appendix:fernandez}

\section*{Biographie}

\noindent José Miguel Fernandez a étudié la musique et la composition à l’université du Chili et au Laboratoire de recherche et de production musicale (LIPM) de Buenos Aires, Argentine. Il suit ensuite les cours de composition au Conservatoire National Supérieur de Musique et de Danse de Lyon et participe au Cursus de composition de l’IRCAM. Il compose des œuvres de musique instrumentale, électroacoustiques et mixtes. Ses œuvres sont créées en Amériques, Europe, Asie et Océanie, et il réalise des concerts de musique mixte et électroacoustique dans plusieurs festivals. Il a été sélectionné au concours international de musiques électroacoustiques de Bourges (2000) et il est lauréat des concours internationaux de composition Grame-EOC de Lyon (2008) et Giga Hertz Award du ZKM en Allemagne (2010). En 2014 il a été sélectionnée par l'IRCAM pour suivre le programme de résidence en recherche artistique sur l'interaction en musiques mixtes. Il est actuellement doctorant du Doctorat de musique : recherche en composition, organisé en collaboration par Sorbonne Université, l’UPMC et l'IRCAM. Son projet de recherche se concentre principalement sur l’écriture de l’électronique et la recherche de nouveaux outils pour la création en musiques mixtes et électroacoustiques. Parallèlement à son activité de compositeur, il travaille sur divers projets pédagogiques et de création en lien avec l’informatique musicale.

\section*{Transcript}

\noindent Jose-Miguel Fernandez, interview du 13/06/2018, à l'IRCAM, Paris.

VG — Comme je te disais j'ai une liste de questions assez ouvertes, sur les raisons qui t'ont amené à faire ce que tu fais et là ou tu vas avec les instruments numériques, et les choses qui m'intéressent spécifiquement ce sont les caractéristiques du numérique dans ces instruments qui dès qu'on utilise les instruments sont présentes de toute façon... 

JMF — oui, parce que moi je n'ai pas d'interface physique sauf mes capteurs, mais sinon je fais tout dans l'ordinateur... après bon je fais de la musique mixte donc il y a les instruments vrais qui vont être traités ou de la musique acousmatique pure où il n'y a pas de ...mais je n'ai pas d'interfaces tactiles ou des choses comme ça même si j'en ai fait, j'ai bidouillé pas mal avec arduino, tout ça, mais maintenant je suis plus orienté vers le développement, tous ces trucs là, plus informatique, Antescofo, comme je t'ai montré... 

VG — oui, après tout cela s'inscrit dans un écosystème d'outils que tu utilises... l'idée à la base était motivée par le fait que chacun a sa manière très spécifique de faire de la musique avec des instruments numériques, tout le monde peut programmer à sa sauce donc il y a toujours une part de personnalisation, et pas un truc que tu achètes tout fait... même quand tu achètes Ableton Live, tu vas quand même prendre tes propres chemins... ce qui m'intéressait là-dedans, c'est qu'à chaque fois il y a un côté très singulier dans la manière dont chacun s'approprie ces outils 
 
JMF — ok 

VG — ma première question c'est qu'est ce qui t'a amené à utiliser des instruments de musique numériques plutôt qu'une guitare, un piano, un instrument qui a déjà une tradition, prêt à l'emploi d'une certaine manière... qu'est ce qui t'a motivé à utiliser de tels instruments ? 

JMF — je viens du Chili et je pense que tout vient de la composition, d'essayer de trouver de nouvelles sonorités, toujours, et surtout des interactions avec des instruments et d'autres types de choses, les gestes, l'utilisation de Kinect (l'interface de Microsoft, NdE) mais au tout début, j'étais au Chili et je savais qu'ici en Europe et aux Etats Unis, on utilisait un logiciel qui s'appelle Max, car j'avais écouté quelques compositions faites ici à l'IRCAM et ailleurs, qui utilisaient ça, donc moi j'étais en quatrième année à l'université, on avait un cours d'électroacoustique et j'avais demandé à mon prof est-ce que tu connais ce logiciel qui s'appelle Max ? Et il me dit oui oui, regarde et il ouvre un tiroir et il y avait Max 2.5, donc avec le gros manuel qu'il y avait à l'époque...  

VG — en papier 

JMF — oui, en papier c'était relié avec une spirale... et donc je l'ai pris, et il m'a dit oui il est ici parce qu'on a gagné un projet donc on a acheté un classic 2, un Mac Classic II et Max parce que voilà il l'avait étudié en Allemagne, il venait d'arriver, donc il disait oui je sais que ce logiciel, tout le monde dit que c'est un peu l'avenir... déjà à l'époque, c'était en 1994 ... et donc il m'a dit mais moi j'ai aucune idée de comment ça marche donc, tiens, débrouilles toi... donc il m'a donné le truc et moi j'ai commencé à regarder, donc j'ai vu le premier tutorial où il y avait « plus » et après « multiplier », deux boites, donc je me dis bon mais ça... ça ne sert pas à faire de la musique ce truc... et bon après je me suis rendu compte qu'il y avait un métro (objet Max métro, servant de métronome, NdE) donc ah ok, on peut faire les rythmes, et après il y a un random (objet Max, NdE) les trucs de bases quoi... donc je peux faire random, et et après je peux contrôler des trucs externes, des synthétiseurs, des samplers... parce qu'à l'époque c'était que MIDI, c'était la version Opcode MIDI... et donc il y avait des samplers et donc je commençais à enregistrer plein de trucs et après les piloter avec Max et après les synthés, tout ça ... et petit à petit je commençais à rentrer dans le monde de la fabrication, et en temps réel, parce qu'on peut dire que j'ai vraiment commencé avec Max, j'avais fait un peu de Csound aussi... et en même temps... et après je suis allé en Argentine où il y avait le LIPM (Laboratorio de Investigación y Producción Musical, à Buenos Aires, NdE), c'était un laboratoire d'électroacoustique, le meilleur à l'époque c'était en 1996 ... et c'était le meilleur d'Amérique du Sud, il y avait toutes les semaines des gens, John Chowning, Chadabe, l'IRCAM, Boulez venaient tout le temps... j'étais pendant six mois et j'ai vu passer toute l'informatique musicale, il y avait Max Mathews, tous les gens très très importants, donc ça bougeait beaucoup, ils avaient beaucoup d'argent à l'époque et il faisaient des concerts toutes les semaines, tous les mercredis, même deux parfois... et donc là j'ai continué à faire de la programmation dans Max ... et voilà je pense qu'à partir de là, je me suis mis complètement dans l'informatique et quand je retournais au Chili, je travaillais avec le gars que je te disais, il avait plein d'ordinateurs, j'ai travaillé avec Nato qui était la première librairie pour faire de la vidéo dans Max et le premier système ambisonique aussi, on avait mis une sphère de haut-parleurs, et ... bon après un certain moment, je suis dit bon là il y a une personne qui peut m'apprendre plus de choses et ici voilà... parce que l'autre personne était plutôt dans la musique électronique pour danser plutôt...et moi je voulais continuer la musique plutôt de recherche et d'exploration ... expérimentale d'une certaine façon... et bon je me suis dit ici il n'y a plus rien à faire, il faut que je parte un peu plus loin... donc je suis allé à Lyon, au CNSM, j'ai fait la postulation et j'ai été accepté, et là voilà j'ai continué encore à développer Max et tout ça .... et bon l'idée c'était toujours de faire des choses les plus interactives possibles entre les musiciens donc j'ai beaucoup travaillé avec des percussionnistes, même aujourd'hui je continue à le faire, et donc mettre toujours des capteurs, des piezos, des choses pour avoir le plus d'information possible dans l'ordinateur pour la synchronisation, pour les traitements et tout ça et ça m'a amené à commencer à développer de plus en plus de différents types de patchs, soit pour la synthèse, soit pour les traitements... et là c'est arrivé que je suis venu à Paris, faire le cursus de composition de l'IRCAM et là, après le cursus, comme ça arrive à beaucoup de compositeurs, bon je ne suis pas si jeune que ça, mais qu'on a rien... d'un jour à l'autre, j'avais une super bourse pour toute cette année, et d'un jour au lendemain je n'avais rien donc j'ai demandé ici, bon est ce qu'il y a un peu de boulot, et il y a Eric Daubresse qui m'a dit oui, il y a un peu de travail, donc j'ai commencé à travailler avec Emmanuel Nunes, qui avait une notion de l'électronique très fine, et tout de suite j'ai été amené à faire des patchs assez complexes, pour pouvoir ... donc lui son paradigme, c'était que chaque note allait avoir un ou plusieurs traitements et chaque note allait se spatialiser dans un ensemble de haut-parleurs, un peu une sphère aussi de haut-parleurs, un enveloppement de haut-parleurs, et donc la première pièce que j'ai faite avec lui, c'était pour ensemble et électronique donc il y avait des couches de superpositions d'instruments qui partaient dans tous les haut-parleurs, donc c'était à l'époque j'utilisais encore des boites de messages, mais il y en avait partout et donc c'était assez, disons grand comme travail, et donc à partir de ça après j'ai continué à travailler à l'IRCAM pendant une dizaine d'années en tant que RIM (réalisateur en informatique musicale, NdE), intermittent du spectacle bien sûr... mais mon idée c'était toujours de voir quelle électronique avoir dans le monde de la musique contemporaine, plus ou moins l'équivalent que la musique instrumentale, c'est-à-dire que la musique instrumentale a un long parcours historique et aussi des techniques, et des virtuosités... on voit même des concerts pour piano de la musique romantique, il y a un rythmique, harmonique et oui, au niveau de la virtuosité, donc normalement c'est ce qu'on a l'habitude, parce que bon c'est quand même assez jeune je pense l'informatique musicale temps réel, et souvent c'est comme si l'électronique reste toujours un accompagnement ou qui est toujours plus simple, on fait des nappes, ou.... Et donc mon idée c'était pourquoi ne pas arriver au même, même si bon, c'est une idée, c'est peut-être encore une utopie, mais pourquoi ne pas arriver au même niveau de sophistication du monde instrumental dans le monde électronique ... et donc à partir de là, j'ai commencé à faire justement des patchs de plus en plus complexes, mais à un certain moment, plutôt récemment, je me suis rendu compte que j'avais besoin quand même d'un autre paradigme, de trouver une autre façon de faire, qui est d'utiliser par exemple Max dans ce cas, parce que je voulais des choses dynamiquement, très rapidement et avoir une grande superposition, de faire une espèce d'orchestration de l'électronique, au même titre que l'instrumental, et ... aussi avec les travaux que j'avais fait avec Emmanuel Nunes, donc lui m'a d'une certaine façon marqué par cette idée de virtuosité électronique, et d'arriver à un système le plus dynamique possible... et donc j'ai commencé à travailler là avec Antescofo, qui est le suivi de partition, donc là on pouvait avoir vraiment un suivi de partition très très précis et même à un niveau rythmique, même si c'est très rapide l'ordinateur va arriver à suivre, bien sûr si c'est des notes, si c'est des modes de jeu un peu particulier, il faut passer à d'autres systèmes mais au moins le suivi de notes marche très bien et aussi on avait le suivi de tempo donc avec toutes ces données on pouvait déjà faire des choses très très sophistiquées mais il restait quand même la contrainte qu'on a, voilà, sur l'ordinateur, souvent, ou sinon sur des ordinateurs en réseau, mais que la CPU monte assez rapidement, dès qu'on commence à utiliser par exemple des phase-vocoder, SuperVP on appelle ça (un objet Max implémentant un algorithme de vocodeur de phase en temps réel, développé à l'IRCAM, NdE) pour faire du time-stretch en temps-réel par exemple ou un traitement FFT, etc. donc on s'aperçoit que dès que tu commence à faire des choses un peu plus dynamiques et rapidement la CPU part et même Max commence à avoir des problèmes au niveau du timing, de la précision du temps... et donc petit à petit j'ai commencé à migrer vers SuperCollider que j'ai au début utilisé comme un synthétiseur, parce que j'étais assez choqué la première fois que j'ai chargé SuperCollider, bon j'ai commencé avec la version 2, mais qu'il y a une seule ligne de code et que ça envoyait tout de suite un son hyper-complexe et donc je me suis dit bon, là il y a quelque chose à explorer, parce qu'avec une seule ligne de code très réduite, on arrive à faire des sons assez complexes avec des rythmes et surtout la qualité du son, ça m'a beaucoup choqué parce qu'on avait plutôt l'habitude de Max/MSP, à l'époque ça venait de sortir aussi, c'était en 1998, quelque chose comme ça, 1999, je ne me souviens plus... et donc le son de SuperCollider quand même il y avait une richesse qui était assez particulière, qu'on n'avait pas dans Max/MSP à l'époque, peut-être que maintenant c'est différent, peut-être que maintenant, c'est aussi un paramètre un peu abstrait, mais c'est un paramètre qui est quand même perceptible et que pas mal de gens ont quand même... c'était même une discussion dans les forums de logiciels spécialisé, quel est le logiciel qui sonne le mieux et donc toujours il y avait une tendance vers SuperCollider quand on faisait la relation entre des environnements temps-réel ... et donc comme je t'ai dit, j'ai commencé à utiliser SuperCollider que pour faire des sons, parce que ça m'intéressait la partie synthétique et tous les UGens, qui sont comme les objets dans Max, les objets qui génèrent du traitement du signal, du son, des enveloppes et tout ça, et donc il y avait et il y a toujours une grande richesse de différents types de modules pour faire différents types de choses, stochastiques, randomiques et déterministes, etc. et à un certain moment, j'ai commencé à me dire ah mais pourquoi je fais pas les sons, au lieu de les faire en... parce que je les faisais dans SC et je les enregistrais pour les utiliser comme des bandes pour déclencher après, je me suis dit pourquoi ne pas les faire en temps réel et c'est là que je me suis aperçu qu'il n'y avait pas que la synthèse mais qu'il y avait tout un mécanisme de gestion du temps réel dans SC qui était complètement dynamique et vraiment, James McCartney, la personne qui l'avait créé avait pensé du début à toute cette organisation... donc le logiciel est pensé du début, bon comme Max aussi bien sûr pour faire du temps réel, mais ici pour faire du temps réel dynamiquement... c'est-à-dire qu'on va pouvoir créer des instances de synthèse, les détruire, les faire évoluer dans le temps, créer plusieurs superpositions et avec une consommation de CPU assez réduite... et donc ça, ça m'attirait beaucoup l'attention ... et bon après je fais le mix, j'ai commencé à faire ça il y a deux ou trois ans... entre les logiciels, parce qu'Antescofo bien sûr a évolué, la première pièce où je l'ai utilisé ça devait être en 2010 ou 2011 et aujourd'hui ce n'est plus qu'un suivi de partition mais aussi un langage de programmation en entier, donc un langage où on va pouvoir gérer sur tout le temps, il y a différentes façon de gérer le temps... comme je t'ai montré il y a le temps absolu, le temps relatif, il y a différents types de courbes, des multi-courbes, multi-dimensionnelles,etc. Donc le fait que c'est un langage de programmation va me permettre de créer des processus temporels ou rythmiques ou musicaux, pour créer des accords ou n'importe quoi mais aussi pour créer de la synthèse et c'est là où je me suis dit bon, ce que je vais faire c'est marier le monde du suivi de partition avec ce langage de programmation c'est un espèce de méta-séquenceur, parce qu'on peut l'utiliser comme séquenceur aussi, on n'est pas obligé de l'utiliser qu'avec le suivi de partition mais tu peux créer tes séquences, des séquences où tu vas générer aussi en même temps des processus, donc qu'est ce que c'est un processus, ça va être par exemple déclencher une séquence qui va être crée avec un algorithme déterminé ou séquence avec des notes que tu vas mettre dans un réservoir... et bon ça c'est vraiment la base, mais ça va être aussi changer la structure d'une séquence où d'une synthèse avec des inputs, donc c'est là où j'ai commencé à utiliser aussi des capteurs, donc je rentre par OSC (Open Sound Control, NdE) direct les capteurs dans Antescofo et lui il va créer les synthèses et tout ça... donc l'idée d'unir les deux mondes, c'est que les deux sont des systèmes dynamiques dans le sens où dans Antescofo je peux créer un petit processus, et ce processus je vais pouvoir l'appeler, mon processus bien sûr va s'appeler « toto » (rires) et je peux l'instancier toutes les fois que je veux... par exemple toto va faire Do, Mi, Sol ... mais je peux le lancer cinquante fois, il va faire cinquante fois Do Mi Sol et même il va y avoir du chevauchement, ce qui veut dire que la polyphonie est déjà intégrée d'office parce que je peux l'instancier autant de fois que je veux... je peux changer ses paramètres, que ça ne soit pas Do Mi Sol, mais Do Ré Mi Fa Sol La Si Do, et donc je peux aussi en temps réel changer la morphologie d'une certaine façon ... la ligne mélodique, si on fait une mélodie, je peux la moduler aussi en temps réel et je peux connecter cette modulation avec le monde extérieur, apr exemple les capteurs ou le clavier ou n'importe quoi, par exemple l'analyse du son, j'ai beaucoup travailler avec l'analyse en temps réel des flux audios avec des descripteurs ou des trucs beaucoup plus simple, comme enveloppe follower, etc. donc tu vas pouvoir moduler toutes ces structures qui sont polyphoniques et ce qui m'intéresse c'est que dans SC c'est plus ou moins l'équivalent parce que je vais pouvoir aussi créer des synthèses, tout ce que je veux, dynamiquement, sans avoir besoin de faire un patch par exemple... Bien sûr avant j'ai programmé tout pour que je puisse faire les interconnexions tout ça, il y a beaucoup de programmation avant, mais au moment de la performance on va pouvoir déclencher autant de choses qu'on veut, à la vitesse qu'on veut, bien sûr il y a des limites, mais ça donne un environnement très flexible et je m'approche de mon idée que je te disais au début que je peux pouvoir manipuler l'électronique de manière aussi souple qu'avec les instruments... bien sûr ça ne va pas être quelqu'un qui va le jouer, sauf si on utilise le suivi de partition, à ce moment là je peux utiliser les caractéristiques de l'instrumentiste comme les gestes, le son etc. pour piloter et créer dynamiquement des synthèses, des spatialisations et tout ça mais je veux aussi pouvoir créer des processus qui vont pouvoir créer plusieurs couches de synthèse, donc c'est ce que je suis en train de faire dans lequel il y a le système ambisonique et je veux, voilà, créer des masses sonores qui vont d'un endroit à l'autre, après peut-être aussi avoir une écoute, que je puisse me balader dans la pièce elle-même, la pièce électronique je veux dire, mais qui est un espace virtuel et créer des masses qui vont évoluer, un peu à la Xénakis d'une certaine façon, utiliser des masses sonores et créer par des synthèses granulaires ou différents types de synthèses que je peux envelopper, envoyer dans différentes parties de l'espace en 3D et voilà... mon idée c'est justement de pouvoir donner à l'électronique une vie que normalement on n'a pas l'habitude de le faire, on ne peut pas le faire par exemple avec ProTools, si je me mets à couper des petits échantillons et à faire des processus, peut-être que je vais passer un an à faire une pièce de cinq minutes... donc c'est vraiment pas pratique de faire comme ça et c'est pas adapté parce que je vais avoir des superpositions de, je sais pas, 200 trucs en même temps... et par exemple gérer dans proTools 200 tracks c'est un peu compliqué et après si tu veux faire du grouping, bien sûr on peut mais c'est pas le but et ce n'est pas des instruments qui sont adaptés pour faire ce genre de choses et c'est pour ça que j'ai eu cette tendance d'aller vers cette richesse de l'électronique qu'on puisse voir comme un espèce d'orchestre électronique mais aussi comme une possibilité d'intégrer des choses de modèles extérieurs comme des modèles mathématiques ou stochastiques ou des... je sais pas , des orbites, pour l'instant je suis en train de faire un catalogue des librairies dans lesquelles je suis en train d'injecter ou programmer différents modules pour différents types de mouvements dans l'espace ... des mouvements rythmiques etc. et après biensûr des enchainements de synthèse et voilà... c'est ça l'idée, c'est de faire une électronique qui soit très très souple et virtuose... ou pas... parce qu'on peut faire des nappes très lentes mais lesquelles il y a beaucoup de superpositions, c'est pas que des choses très articulées rythmiques mais les deux parce que bon la musique que je fais, il y a souvent des parties qui sont un peu plus statiques, mais après ça rentre dans un chaos où ça aprt dans tous les sens, où peut-être ça peut revenir, donc ça passe d'un monde à l'autre... donc l'idée c'est pour aller dans cette densité, de pouvoir utiliser qui soient un peu plus performants que ceux qu'on a l'habitude d'utiliser... et voilà c'est plutôt où je vais... 

VG — oui... en faisant un grand saut jusqu'au début de ce que tu disais, tu as commencé par la composition instrumentale classique, hors électronique ? 

JMF — oui, disons ma première pièce c'était classique, un sextuor à cordes, mais je pense que ma deuxième composition déjà c'était une pièce mixte... et après j'ai fait une pièce électronique pure, acousmatique et après une pièce instrumentale et je pense que depuis que je suis rentré au CNSM de Lyon, j'ai fait peut-être une seule pièce acoustique seule, un quatuor de saxophone, le reste c'est que des pièces mixtes ou électroniques, donc oui, on peut dire que j'ai tout de suite commencé à faire de la musique mixte avec des ordinateurs, c'était presque en même temps... 

VG — dans ce que j'entends de ce que tu me racontes, tu étais dans un cursus de composition où du coup tu as appris à composer avec les instruments acoustiques et avec la notation classique, j'imagine, déjà étendue par les symboles de la musique contemporaine, et du coup l'électronique, c'était du coup un moyen d'avoir des outils plus souples, plus dynamiques, pour l'écriture ? Il y avait vraiment cette idée de l'écriture musicale ? Je te pose cette question car parmi les gens que j'ai interviewés, certains sont arrivés dans la musique électronique uniquement pour des raisons de son, par exemple... c'était la possibilité de faire des sons qu'on ne pouvait pas faire sans l'électronique, il y avait une souplesse au niveau du son ... quelque part les deux se rejoignent.... La souplesse d'écriture des processus participe de la richesse des sons qu'on peut produire, sûrement, mais je pensais à cela par rapport à cet exemple que tu donnais de ProTools où c'est très difficile de faire 200 pistes parce que tu veux déclencher 200 notes en même temps, ces systèmes électroniques, tu dis que tu souhaites un système plus ouvert, plus dynamique, plus vivant... et du coup c'est des outils pour lesquels il devient très difficile de noter cela avec une notation classique, voire impossible ... on serait autant embêté avec une partition papier pour dessiner 200 notes qu'on le serait avec ProTools...  

JMF — oui bien sûr 

VG — donc ce n'est pas tant le fait que ProTools ne soit pas pratique, c'est que ... 

JMF — oui, c'est pas adapté à la notation bien sûr... après pour revenir aussi à ce que tu dis, pour moi, le plus important c'est le son, aussi ... le résultat sonore c'est ce qui est le plus important, après il y a des moyens pour y arriver... donc l'écriture instrumentale tu peux faire quelque chose plus ou moins complexe, mais moi ce qui m'intéresse c'est le son, ce n'est pas l'écriture pour l'écriture elle-même... par exemple je peux créer des systèmes qui vont me donner des résultats d'écriture, bien sûr je peux utiliser de la CAO, de la composition assistée par ordinateur, qui va me générer automatiquement des choses mais c'est pas ça vraiment qui m'intéresse... je l'ai utilisé, et je vais continuer à l'utiliser, c'est comme avoir une espèce de réservoir de sons... toute l'écriture pour moi, c'est quelque chose qui va vers le résultat final qui est le son... par exemple, moi, ça ne m'intéresse pas de créer des relations hyper complexes au niveau rythmique, mélodique, etc. sans avoir un retour sonore... je pense que la musique algorithmique pure, pour moi, n'a pas trop d'intérêt parce que on se concentre beaucoup trop sur quelque chose de théorique et on laisse de côté justement le son... et le son c'est ça qu'on va finalement entendre... c'est ça qui est la musique, ce n'est pas la théorie qui est derrière... bien sûr après on peut créer des super théories qui vont créer aussi des sons très intéressants et qui peuvent donner aussi des compositions magnifiques, mais après voilà ça dépende du compositeur de comment il arrive à avancer... surtout aujourd'hui où on a tellement de possibilités... mais voilà c'est comme à partir, je sais pas, des années 60, il y a dans la musique contemporaine, pas dans les autres musiques où il n'y a pas eu cette division entre la partie théorique et le son, mais dans la musique contemporaine au moins européenne et occidentale, on peut dire qu'à partir des années 1950, il y a eu un retour vers le son et je suppose qu'on est encore là-dedans parce que finalement on s'est rendu compte que si on veut pas ennuyer un public au bout de cinq minutes d'entendre quelque chose qui finalement change, par exemple c'est ce qu'on appelle la musique structuraliste de années 1950, où tous les paramètres étaient calculées par des combinaisons calculatoires, et au bout de cinq minutes on perd l'attention (la tension?) parce que ça donne toujours la même chose... bon après il y a des gens qui apprécient beaucoup ça, mais en général on a plus je pense la sensation, ou je ne sais pas comment dire, l'intuition de rentrer dans une musique qui peut te prendre et te ramener et te faire voyager, ballader par différents endroits, sonores bien sûr, dans un espace sonore et... voilà c'est ça qui m'intéresse... donc pour moi tous ces outils, qu'ils soient plus ou moins sophistiqués, c'est pour arriver au résultat final qui va être le son et cet agencement de sons qui est finalement la composition, c'est-à-dire comment je vais pouvoir prendre quelqu'un, ou même moi-même, parce que peut-être que je vais montrer la musique que j'ai écrit à quelqu'un qui va dire c'est n'importe quoi, c'est pas de la musique ... mais au moins pour moi, et j'espère pour quelqu'un d'autre, va pouvoir rentrer dans cet état quand la musique te prend et va te faire comme une montagne russe, et qui soit quelque chose qui ait une émotion .... le plus important c'est la musicalité et c'est le truc un peu magique, car elle a la capacité de te prendre et te faire rentrer dans des espaces mentaux, psychologiques, ou je ne sais pas quoi qu'on ne vit pas dans la réalité normale... quand tu es en train d'écouter attentivement la musique, comme dans une salle de concert, ou quand tu mets un casque, avoir une concentration et te laisser porter par la musique ... pas la musique d'ascenseur ... mais voilà c'est le paradigme du concert qui, bon est aujourd'hui un peu fragilisé car que les gens peuvent sortir de tout ça... mais la musique acousmatique, comme on dit le cinéma pour l'écoute, je suis assez d'accord avec ça, voilà c'est comme quand on va au cinéma, un film dans lequel tu rentres... et le son va réagir dans ta psyché, tes émotions, et toute ta perception...  

VG — c'est la part de sensualité, de souvenirs, d'évocations...  

JMF — oui, d'une certaine façon très hédoniste comme pensée... 

VG — ... qui ne sont pas dans les mathématiques 

JMF — oui, pour moi, tout ça c'est des outils... donc les mathématiques, je peux utiliser différents types d'algorithmes chaotiques, stochastiques, etc. mais ça va être seulement des éléments, comme le bruit ou les sinusoïdes, c'est seulement des outils qu'on a la chance aujourd'hui d'avoir toute cette palette qui est des fois mêmetrop énorme, tu peux te perdre parce que tu ne sais pas par où commencer, tellement il y a de possibilités, parce qu'il y a les instruments, il y a les traitements des instruments, tu as le son de synthèse et maintenant en plus tu peux avoir des choses qui peuvent se produire automatiquement de façon très rapide en temps réel, mais... voilà, ces outils si on arrive à se les approprier, et leur donner une directionnalité, une forme musicale ça peut donner je pense des choses très très riches... voilà c'est ça mon idéal ... après si j'arrive à le faire ou pas, bon c'est une autre histoire... 

VG — et tu parlais du fait que tu avais commencé avec Max, puis après SC, as tu l'impression qu'il y a différents instruments numériques — je ne sais pas si tu les appelles instruments d'ailleurs, que tu as développés et utilisés et qui sont identifiés comme entité... est-ce que tu pourrais les compter par exemple ? Est-ce que tu pourrais dire que jusqu'à aujourd'hui tu as fait, une dizaine, une cinquantaine, 200 instruments électroniques ... ou bien est-ce que c'est quelque chose qui était toujours en évolution ? Est-ce qu'il y a des étapes où tu fais un instrument et il a une fin ? 

JMF — non je pense que je n'ai jamais fini un instrument, c'était toujours un \textit{work-in-progress} et je me souviens que... bon je pense que s'il y a quelque chose avec les gens qui font du Max et du SC, la première chose qu'ils essaient de faire — bon, pas tous—, c'est de se créer un environnement ... par exemple, je me souviens dans Max, mon premier environnement, dès qu'est sorti Javascript, ça a été de créer un système de scripting, parce que j'avais l'idée de créer automatiquement les modules et donc j'avais ce que j'appelais un méta-patch, et je l'ai —peut-être pas ici, mais quelque part dans un disque dur, c'est un patch qui disait combien d'entrées tu veux et combien de sorties, est ce que tu veux un spatialisateur, est ce que tu veux un \textit{frequency shifter}, donc j'ai commencé à donner des listes de traitement et après je cliquais un bouton qui s'appelait « build », et il créait tout le patch automatiquement avec les vumètres, les sliders, la matrice audio, la matrice de contrôle avec tous les trucs dessinés, les inputs, les outputs, et donc d'une certaine façon ça c'était un premier instrument, parce que pour moi l'environnement c'est l'instrument et après tu vas seulement lui créer des plugins de cet environnement... et donc c'est un paradigme assez connu, tous les logiciels marchent comme ça, comme Live ou même ProTools, ou Studio Vision pro, ou Cubase déjà avait la notion de plugins et je pense même les mixeurs, on peut dire d'une certaine façon que c'est des plugins qui sont fixes parce que tu as l'équalisateur, le compresseur, etc. mais voilà on peut dire que cet instrument a déjà les plugins incorporés, et mon idée c'était d'avoir un instrument modulaire mais comme je t'ai dit, il n'y a pas que moi, je pense qu'il y a plusieurs gens qui font de l'informatique musicale dans Max, qui se sont créé, j'en ai vu passer surement toi aussi, plusieurs environnement de ce genre, dans lequel tu peux soit créer comme je t'ai dit, une représentation et après le truc va se créer automatiquement, soit que tu vas les faire plus ou moins dynamiquement... donc ça c'était mon premier instrument... et après quand je suis rentré ici à l'IRCAM, ce que je faisais c'est que j'avais un patch qui avait plein de traitements, donc après je le montrais au compositeur et après je faisais par exemple une réduction ... ou bien je créais en même temps des plugins parce qu'il voulait faire un truc déterminé, une abstraction Max ou un patch Max, que je vais rajouter au patch principal pour générer des processus.... Si je veux faire un truc qui fait des rythmes automatiquement, je vais intégrer cette machine dans l'environnement général de Max... et après j'ai beaucoup travaillé avec les capteurs, comment rentrer des données du monde extérieur pour aussi moduler toutes ces machines en tep réel etc. et mon deuxième instrument c'est cet environnement dans SC qui est plus ou moins la même chose, parce que d'un côté j'ai tous mes modules, c'est des plugins, et je pense qu'il n'y a rien de nouveau par rapport à ça... c'est le même concept qu'on a eu de toujours, mais c'est la possibilité que je vais pouvoir construire des environnements tout de suite et dynamiquement et même tout ce que je faisais avant dans le scripting Max, ça prenait des fois quelques minutes, où tu avais le truc qui tournait, la pizza (le sablier sur OSX, NdE), parce qu'il était en train de créer tous les patchs et bpatcher pour la visualisation etc. tandis que là, ça se fait instantanément.... Je n'ai pas calculé combien de temps ça prend... Peut-être que si je fais des graphes audio très complexes ça prend quelques milli-secondes mais c'est instantané, on ne le voit pas... et donc c'est la même chose mais augmenté parce que je le fais en temps réel, automatiquement, je fais une description des processus ou des chaînes de traitement, dans une ligne de code et donc c'est ça pour moi maintenant l'avantage de ce deuxième instrument, je peux dire, que j'ai fabriqué c'est le dynamisme...  

VG — d'avoir l'instrument le plus réactif possible... 

JMF — oui... et que je peux changer comme je veux, le plus souple possible ... j'y ai passé un peu de temps mais je suis assez content du résultat au aussi du fait qu'il y a aussi le langage Antescofo qui va me permettre de piloter cet environnement... donc dans le futur on pourrait imaginer qu'on pourrait faire tout avec Antescofo, donc ils ont déjà fait le test d'intégrer FAUST (Functional Audio Stream, développé par le Grame, NdE), il y a Pierre qui n'est pas là aujourd'hui mais ils ont créé des SynthDef —des définitions de synthèse— dans Antescofo mais qui appelle le compilateur FAUST, donc tu peux créer \textit{just in time} (cf. JIT compiler, NdE) comme ils appellent ça, des modules FAUST... c'est encore trop expérimental et primitif mais peut-être dans le futur on peut avoir un système qui soit plus unifié, c'est le sujet de ma thèse, de faire des partitions centralisées, ça s'appelle mon sujet de thèse...  

VG — Partitions centralisées ? 

JMF — oui, c'est-à-dire d'avoir toute la description de ce qui va se passer dans un seul environnement, ici en l'occurence c'est une partition électronique, c'est-à-dire où je vais avoir tous les processus temporels et les processus de synthèse audio, je vais tout écrire dans un seul environnement... et éventuellement on pourrait aussi aller plus profondément et faire du DSP, du traitement du signal directement aussi ... mais bon, ça dépend de niveaux de complexité... 

VG — du coup Antescofo devient un peu le container de tes partitions... 

JMF — voilà... Antesofo devient un peu le séquenceur, parce que bon la musique se déroule dans le temps, donc je vais lui dire que maintenant je vais commencer tel truc, après je vais avoir un deuxième événement, un troisième etc. mais pour chaque événement, je veux pouvoir faire quelque chose de déterministe, dire déclenche moi une séquence qui va faire 1,2,3,4 mais le 2ème événement je peux lui dire, fais moi un événement qui va dépendre de la vitesse avec laquelle tu as bougé la main, et le 3ème, je vais utiliser des algortihmes mathématiques, et un quatrième où je vais faire interagir différents processus, peut-être déterministes qui vont déclencher des trucs aléatoires ou à l'inverse, des trucs aléatoires qui vont déclencher des trucs déterministes, des séquences ou peu importe, et donc ça devient un espèce de méta-séquenceur, parce que je epxu en même temps créer des processus qui vont s'autogénérer et qui vont aussi être influencés par le monde externe... donc comment agencer tout ça, ça va me permettre d'expérimenter, voir, aller vers la musique... peut-être que je vais avoir une idée et après la faire jouer en temps réel, la faire jouer, je peux la faire jouer en temps réel, et si ça ne me plait pas, je dis non ça, ça ne marche pas, je la jette et .... mais bon ça c'est aussi l'expérience, qu'après tu sais plus ou moins où tu vas, quand tu as une idée, c'est tout l'apprentissage qui est un peu interne et subjectif aussi... 

VG — tu parlais dans Antescofo, du fait qu'on reste sur une notion du temps linéaire, par rapport à cette souplesse d'écriture, est-ce que ce n'est pas quelque chose qui te contraint d'avoir un développement linéaire, plutôt que d'avoir différents fragments qui pourraient être reconnus... 

JMF — disons que tu peux faire les deux... ce n'est pas complètement linéaire et disons tu peux faire des sauts, tu peux appeler différents trucs... et aussi, un autre avantage, c'est que tu peux avoir plusieurs temps en parallèle, donc tu peux faire de la polyrythmie sans aucun problème, disons quelque chose qui va aller dans un rythme, un autre qui va être très rapide, donc tu peux avoir quelque chose qui va se manger, je sais pas, qui va polluer l'autre, le transformer... tu vois tu peux faire ce genre de choses... mais bon moi, ce qui m'intéresse c'est quand même la linéarité, donc je suis peut-être dans ce cas assez classique, on peut le dire, parce que je veux rester dans le mode de la composition où je commence là et je finis là, et milieu j'ai tout un parcours... la montagne russe dont je te parlais, le voyage... j'aime construire ce voyage, même si dedans il y a plusieurs dont je n'ai pas le contrôle, volontairement je dis que je ne veux pas avoir absolu que tous les grains que je veux déployer dans l'espace, parce que c'est l'effet global qui m'intéresse... mais il y a quand même la notion de linarité dans la composition elle-même, après bien sûr on peut faire des trucs plus ou moins complexe... je peux dire que ce qui est arrivé là, après il va arriver là d'une façon transformée ou renversée... mais ça c'est des choses qu'on fait dans la composition depuis la nuit des temps... donc ça reste quand même dans cette idée de composition... parce que ce qui m'intéresse ce n'est pas qu'on ne sache pas, ou que je ne sache pas où je vais aller... bien sûr, après le champ est libre pour ceux qui veulent, Antescofo va te permettre même de faire des états d'improvisation et tu peux faire plein plein de choses... tu vas par exemple pouvoir reconnaître des patterns, par exemple si tu fais Do, Sol, Mi, à chaque fois que tu vas jouer dans ton instrument Do, Sol, Mi, l'ordinateur peut être réactif et jouer quelque chose par exemple... ou à chaque fois que, bon, ça on ne l'a pas encore expérimenté mais, par exemple, s'il reconnaît un geste que je fais avec les capteurs, à chaque fois que je fais un rond avec un geste, l'ordinateur va dire ah maintenant je vais faire telle chose... ou tu peux lui donner un réservoir de trucs à faire, à chaque fois que je fais un kick de main droite vers la droite par exemple... donc tu peux créer des états d'improvisation assez poussé aussi, parce que voilà tu peux avoir plein de réservoirs de trucs qui vont se passer en fonction de comment tu vas jouer ou simplement laisser libre et faire qu'à chaque fois que je regarde la caméra, il va faire un truc tu vois... mais musicalement, ça ne m'intéresse pas beaucoup, mais j'ai quand même expérimenté des choses comme ça, mais je pense que le problème c'est qu'on perd un peu le contrôle... j'ai beaucoup fait d'improvisation et c'est super, mais surtout je trouve pour les gens qui le font... mais des fois pas trop pour les gens qui l'écoute... donc ce qui m'intéresse aussi c'est qu'il y a le public qui écoute, et j'aimerais créer quand même quelque chose qui peut ramener à rentrer dans ce truc qui te prend... c'est ça qui m'intéresse... que la musique te prenne et te... 

VG — qu'est ce qui te fait dire que l'improvisation n'est pas forcément intéressante pour le public ? 

JMF — parce que j'ai joué longtemps dans un groupe d'improvisation et c'était super on était hyper contents mais on le montrait au gens et ils disaient ah oui mais votre truc c'est nul ... ou des fois oui c'est super... mais bon oui, c'est très ... mais bon tu vas me dire toute la musique est comme ça... mais j'ai assisté à beaucoup de concerts de musique improvisée et ça m'est arrivé que, bon pareil je peux avoir écouté un concert et peut-être que je n'aime pas du tout... mais des fois je pense qu'on perd un peu cette espèce de continuité ou je ne sais pas comment dire... mais bon c'est une appréciation personnelle... mais je pense que c'est aussi mon parcours qui m'a amené à faire ça, parce que j'ai aussi expérimenté, mais bon je jouais la clarinette surtout, je ne faisais pas de l'électronique... par exemple... je n'ai jamais eu l'occasion, peut-être que ce serait intéressant... tu dois connaître ça mieux que moi, de jouer, faire une impro électroacoustique ... 

VG — en même temps ta remarque je la comprends... bon c'est difficile de parler de manière générale alors qu'il y a plein de musiques improvisées, mais il y a en tout cas des écueils récurrents dans ces musiques là, surtout en ensemble... par exemple, la raison pour laquelle on a inventé John (John the Semi-Conductor, cf. chapitre Notations, NdE), c'est que parmi les écueils de la musique improvisée, il y a la question des grandes formes qui souvent prennent la forme d'un grand mouvement en cloche, parce que c'est très difficile d'avoir des moments de rupture synchrone avec tout le monde...  

JMF — Oui, par exemple souvent il y a des problèmes de grande forme parce qu'il y a une tendance toujours et donc l'avantage pour moi de la musique écrite entre guillemets, parce que bon on ne peut peut-être plus parler de... je ne sais pas bien ce qu'est l'écriture ou pas maintenant, j'ai une confusion en ce moment... peut-être que je saurai plus dans quelque années... ou pas... peu importe... mais en tout cas l'avantage, c'est que tu sais que là, paf, je veux une coupure nette et que tout le bordel qu'on a fait va se synchroniser et se couper là, à ce moment déterminé et que je vais recommencer un truc, qu'il soit hyper complexe ou pas, et j'ai un contrôle sur le temps que j'aime beaucoup... et c'est ça le truc du compositeur ... j'ai fait aussi des compositions à plusieurs... donc j'ai fait l'année dernière avec un autre compositeur, mais bon c'était séquentiel, disons que chacun a fait une séquence... donc ce n'est pas comme si on avait composé à deux en même temps, mais quand même il reste qu'on a dit tous les deux « ah ici il va y avoir une partie qui va être plus bruiteuse et après une partie plus statique... et après une partie très articulée et après une partie très forte avec un beau crescendo » ... donc tu vois on a créé d'une certaine façon une espèce de forme et après chacun a rempli... après bien sûr au milieu tu peux changer, tu peux dire ah oui, mais ça non, ça peut-être on va les inverser parce que ça va mieux... 

VG — ... un squelette pour la grande forme... 

JMF — oui, un squelette... peut-être que c'est ça ce que tu es en train de faire avec John, un squelette ... mais aussi l'avantage de l'écriture c'est que tu peux faire le squelette mais aussi aller vers la micro-structure, tu peux même composer la micro-structure et je trouve que c'est quand même pas mal, parce que tu créé une sorte d'architecture... là je rejoins encore Xénakis que j'aime beaucoup ... lui étant architecte en même temps, il parlait aussi de cette création de formes, d'architecture de la micro-forme à la structure générale, au bâtiment ou à la pièce ... ça je trouve que c'est l'avantage de l'écriture, et c'est la chose qui m'intéresse en ce moment... c'est pour ça qu'avec Antescofo, tu peux écrire très précisément tous les événements électroniques, avec SC les jouer, et créer les rendus audio, etc.  

VG — tu utilisais l'expression « cinéma pour l'écoute », tu parlais de montagnes russes, tu as utilisé différentes métaphores et tu parlais aussi du fait que tu utilisais des capteurs... une des choses singulières dans les instruments électroniques et numériques en particuliers c'est que lorsque tu branches des choses ensemble, que cela soit des processus, des capteurs, etc. tu peux les brancher de différentes manières, et donc à la différence des instruments acoustiques où tu retrouves le même comportement quand tu jours plusieurs fois les mêmes touches, là chaque touche peut déclencher quelque chose de différent à chaque fois et donc toute l'interaction est scénarisée... Quand tu travailles avec des interfaces, quelle genre de métaphore tu vas utiliser ? Est-ce que par exemple tu as donné des noms à des mappings qui revenaient de manière récurrente ? Comment fais-tu ton chemin là-dedans ? 

JMF — Je pense, encore une fois, que c'est de la structuration, de l'écriture... les capteurs que j'utilise sont déterministes, ce sont des MO (Modular Musical Objects, développés par l'équipe ISMM à l'IRCAM) 9 axes, donc je vais avoir toujours la même donnée, bon, plus ou moins la même, comme je les mets dans un gants, l'idée c'est que ce soit fixe et le plus déterministe possible ... si je fais ce mouvement de la main, je veux savoir que la position de la main est comme ça, et donc du moment où j'ai les capteurs déterministes, ce serait différents avec des capteurs un peu plus ... par exemple des capteurs physiologiques... 

VG — mais même avec des capteurs déterministes, tu peux mapper cette position x à plein de choses différentes qui vont changer à chaque pièces, ou même durant la pièce... 

JMF — oui, c'est là où il y a une espèce d'écriture aussi... parce que je fais d'abord les tests, l'expérimentation et tout ça et travailler avec des musiciens ... et je me dis ah, à ce moment là, je vais utiliser tel type de mapping ... 

VG — c'est-à-dire qu'un geste est associé à un moment dans l'écriture... 

JMF — ... voilà, j'ai même des partitions où j'écris le mouvement des gestes, bouger la main comme ça, etc. Il y a même dans l'écriture des gestes où des fois je mets des traits pour qu'ils jouent des gestes très brusques de percussions ou de karaté, ou j'ai un symbole de repos... 

VG — c'est-à-dire qu'un geste va jouer tel son à un moment de la pièce, mais à un autre moment de la pièce, le même geste pourra déclencher autre chose ? 

JMF — oui, c'est ça... il y a une écriture du mapping, aussi... 

VG — d'accord... donc les métaphores que tu utilises ont une inscription temporelle locale à chaque fois, dans le déroulement de la pièce ? 

JMF — c'est ça oui... mon truc en ce moment, c'est ça, et donc même les mappings, et même les transformations de mapping, parce que je peux dire qu'un mapping va se transformer dans le temps en un autre type de mapping, et tout ça, même des choses très complexes qui peuvent arriver, comme l'utilisation de machine learning, tout va être quand même déterminé à un certain moment de la partition... même si je peux dire que si je fais ça (faisant un geste de la main, NdE) à un certain moment, je peux avoir par exemple un réservoir de sons ou de synthèses, mais à chaque fois que je fais ça, ça va être différent mais dans une certaine ambiance on va dire... donc par exemple je vais utiliser des petits sons aïgus, et passer à des sons graves très aggressifs... 

VG — et généralement, c'est toi qui joue les capteurs ou bien c'est quelqu'un d'autre ? 

JMF — c'est quelqu'un d'autre... je suis encore dans ce paradigme instrumentiste/compositeur, que je donne ma partition et quelqu'un la joue... parce que des fois je me dis je vais faire une pièce où je bouge seulement les mains, mais après il faut le faire... et on sait que les instrumentistes ont un \textit{background} des gestes, des mouvements, beaucoup plus précis que moi... bon j'ai joué du piano et de la clarinette pendant des années, mais je ne joue plus et je n'ai pas la même réactivité et des fois peut-être la même musicalité... même si des fois, il y a des instrumentistes qui sont un peu coincés, parce que quand tu leur mets des capteurs, il faut aussi un entrainement pour eux, pour maîtriser et savoir jouer avec un instrument électronique... et donc ça, c'est un désavantage on peut dire, par rapport à quelqu'un qui les a fabriqués, qui sait jouer son propre .... mais peut-être que cette personne là va jouer toujours de la même façon, parce qu'il a ses conditionnement, etc. et c'est intéressant quand tu construis un instrument de le faire jouer à quelqu'un d'autre, à différentes personnes, et même pour toi si tu es instrumentiste, ça va te donner aussi d'autres idées, sortir un peu... c'est un truc naturel de l'être humain qu'on reste toujours dans un petit espace parce qu'on se sent protégé, on sait qu'on maîtrise ça, mais il y a peut-être d'autres mondes à explorer qui pourraient être très intéressants et c'est pas aussi compliqué non plus de rentrer dedans, c'est juste que voilà, il fallait donner le truc à quelqu'un d'autre... peut-être un musicien africain ou indien, ils vont changer complètement ta perception du truc et ça va te donner un autre retour qui va, voilà peut-être te faire partir dans de nouvelles voix... et ça c'est très important je pense, on peut faire un parallèle entre la musique et la vie, on est toujours comme ça (faisant des signes de changements de directions, NdE), après on a des bas et des hauts, et des fois on se sent super bien, des fois on se sent très mal... donc la vie c'est tout lié pour moi de toute façon... donc le fait de regarder d'autres cultures ou d'autres gens dans ce cas là qui jouent, peut t'apporter beaucoup personnellement et musicalement...  

VG — je parlais des métaphores que tu peux utiliser pour la composition, mais quand tu confies ton instrument à quelqu'un d'autre, même si tu peux lui laisser découvrir l'instrument par lui-même, tu es amené à lui transmettre des instructions, et ce ne sont pas des instruments qui ont une histoire, donc c'est à toi de dire ça fonctionne comme ça et peut-être utiliser des métaphores ? 

JMF — oui... il y a justement la tradition orale, qui est quelque chose de très ancien, mais qui est toujours très très moderne aussi... parce que voilà on construit un instrument ou un logiciel, mais il faut quand même les apprendre, faire des tutoriels dans YouTube pour que les gens apprennent à l'utiliser... donc même si on utilise des technologies très sophistiquées, mais il y a toujours ce rapport qui était là avant l'écriture ... ou dans d'autres traditions... parce qu'il y a des traditions où il n'y a pas d'écriture du tout, d'autres où il y a plus ou moins, et l'occidentale où tout est hyper précis... mais même dans l'hyper-précision il y a toujours le prof qui va enseigner la technique, jouer le piano avec un certain toucher, qui a ce rapport de tradition orale, et qui est fondamental... 

VG — je pensais notamment au fait que tu as un certain vocabulaire qui existe pour la musique comme \textit{staccato, pizzicato, sul ponticello}, etc... qui sont utilisés pour les instruments qu'il est possible de réutiliser dans d'autres contexte... on discutait une fois de la manière dont on peut enseigner les instruments électroniques sachant qu'il n'y a pas vraiment de standard, de forme réifiée et pérenne d'instrument électronique, comme il existe pour les instruments acoustiques... les interfaces sont à chaque fois variées, certains jouent avec des claviers, d'autres avec des pads, d'autres avec des capteurs et arduino... et comment faire pour ne pas repartir de zéro à chaque fois qu'on en discute ensemble... et il y a peut-être un vocabulaire qui se développe pour la musique électronique, par exemple si tu joues de la synthèse FM, si tu la joue avec des capteurs arduino ou avec un clavier, cela ne va pas être la même ergonomie du tout, mais quelque part il y a un objet similaire que tu retrouves, avec certains timbres comme des cloches, des cuivres, etc. le « son » de la FM...  

JMF — oui c'est par rapport au résultat sonore dans ce cas... 

VG — il y a un objet abstrait qui existe qu'on peut transposer sur différentes interfaces, et dans lequel on peut retrouver des chemins... et c'était aussi par rapport à ça que je te posais cette question des métaphores, de la manière dont tu développes un langage pour parler avec les instrumentistes avec qui tu travailles pour pouvoir échanger, dans la mesure où, cet espace de timbres de la FM par exemple, n'est pas enseigné en formation musicale... 

JMF — oui c'est sûr qu'il n'y a pas vraiment de relation entre les instruments traditionnels et les instruments numériques ... ce qui rend aussi peut-être des fois une vie assez courte parce qu'il n'y a pas beaucoup d'instruments qui restent dans le temps, comme ça évolue toujours... le violon est arrivé à son maximum de perfection avec je sais pas, Stadivarius, et le piano avec, je sais pas, Pleyel ... et après il est resté dans cet état là... et personne s'est demandé, oui, maintenant on peut faire les pianos électroniques et tout ça mais ça reste quand même le violon... dans les musiques avec instruments numériques, il y a souvent une vie courte, parce que la personne qui les fait, c'est souvent une personne et une fois qu'il l'a fait, il dit ok je vais faire un autre, je passe à autre chose et celui là il reste là, comme une pièce de musée, presque, même si ça a été fait l'année dernière, et après on passe à un autre, et on passe à un autre, et on passe à un autre... c'est comme les logiciels... 

VG — Oui, il y a beaucoup d'instruments qui sont fait par pièce, en fait, qui ne sont peut-être pas tant des instruments que des agencements d'instruments, parce que comme tu le disais tout à l'heure quand tu parlais de ton « instrument », tu parlais de ton environnement, quoi... qui j'imagine ne change pas à chaque pièce...  

JMF — non ce qui change c'est le contenu disons de comment je vais le gérer... 

VG — tu changes les plugins, tu changes les capteurs... 

JMF — ... mais la structure va être la même oui... c'est une base assez souple mais j'utilise constamment deux logiciels, Antescofo et SC dans un cadre quand même déterminé... même si on peut faire tout et n'importe quoi avec, mais je suis dedans... et par exemple dans les instruments que toi, tu connais bien, il y a vraiment très peu d'instruments qui perdurent dans le temps, et paradoxalement c'est peut-être les premiers, le Théremin, bon les Ondes Martenot peut-être un peu moins, je ne sais pas s'il y a encore le cours au CNSM de Paris, mais c'est plutôt ces instruments plus anciens dont je vois qu'ils perdurent dans le temps... bon et après bien sûr il y a les synthés classiques comme le Moog et les autres qu'utilisent les rockers, mais... 

VG — j'ai l'impression qu'il y a des « organes » d'instruments qui restent, par exemple le clavier ... 

JMF — Voilà ...  

VG — et dans les choses qui survivent, ce n'est peut-être pas tant les organismes complets que les organes eux-mêmes, c'est-à-dire que le clavier c'est un des organes du piano, du clavecin, de l'orgue... et il reste le clavier, on le retrouve sur le Seaboard (de Roli, NdE), et d'autres instruments nouveaux... 

JMF — oui, on a continué à ... oui... 

VG — et il y a à la fois des objets physiques comme le clavier, mais il y a aussi des objets abstraits comme les notions de théorie musicale, comme les gammes, qui sont abstraites mais qui s'inscrivent dans le corps des instruments, et qu'on va retrouver dans les algorithmes de traitements tels que les arpéggiateurs, les grilles de quantisation... c'est un élément abstrait de l'écriture mais qui se traduit physiquement dans des outils de lutherie numérique, qui prennent corps dans les instruments numériques... et du coup ma question, comme tous ces instruments sont très disparates, qu'il y a plein d'éléments de partout et qu'on va utiliser Antescofo, Max, SC, sur un environnement et qu'il est très difficile de tout connaître, il y a ces communautés qui s'organisent sur internet en forums de discussions et si on ne sait pas se servir de quelque chose, on peut poser une question et avoir la réponse sur un forum... Nick Collins disait quand je l'ai interviewé que faire des instruments s'apparentait aujourd'hui à faire de la cuisine, qu'il n'y avait pas besoin d'être un grand chef, car tout le monde sait cuisiner, en récupérant les ingrédients ici et là... 

JMF — de la bidouille... 

VG — oui, il y a un côté bidouille... et il y a beaucoup d'échanges, si on regarde la page Facebook de Max, il y a plusieurs milliers d'utilisateurs dessus, la page d'Ableton Live, n'en parlons pas... donc il y a vraiment un fonctionnement collaboratif, avec des gens qui font des tutoriels et tout ça... ce qui contraste un peu par rapport à une vision un peu plus solitaire du luthier qui fabrique son instrument seul dans son atelier, enfin c'est l'époque qui est comme ça, plus connectée, et la question que je voulais te poser c'est comment toi tu as vécu ça, car quand tu as commencé dans les années 1990, internet n'était pas encore là et tu parlais du manuel papier de Max 2.5, c'était peut-être un travail un peu plus solitaire de lire ça ... 

JMF — oui... mais je pense que j'ai toujours été assez solitaire, bon sauf quand j'étais un Maxeur à fond, parce que là je suivais la mailing-list Max, j'étais beta-testeur, mais ça a duré une période et ça fait des années que je ... bon de temps en temps je vais aller regarder quand il y a un truc qui m'échappe dans Max mais c'est très rare d'aller dans les forums... maintenant je suis le forum de SC, et je ne le vois peut-être pas tout le temps, parce que bon Max, je suivais vraiment tous les différents \textit{thread} etc. bon il n'y avait peut-être pas autant d'utilisateurs, et peut-être que maintenant c'est impossible de suivre tout sinon tu y passes toute ta journée comme sur FaceBook, mais à l'époque voilà, il n'y en avait pas autant... mais maintenant je suis un peu SC, mais par exemple Antescofo, c'est quelque chose qui n'a pas encore une grande communauté, du moins sur le langage de programmation, donc déjà Antescofo, il y a des gens qui le connaissent mais pour qui c'est un suiveur de partition, tout le monde ne sait pas qu'Antescofo, sauf peut-être les gens du forums IRCAM, possède un langage de programmation qui te permet de faire plein de trucs et donc en ce moment là, on est vraiment une communauté très très réduite... SC, c'est différent, il y a une communauté active avec plusieurs emails par jour, ce n'est pas le débit de Max, et ne parlons pas de Live, mais dans ce chemin d'Antescofo je pense qu'on reste d'une certaine façon ésotérique, une secte d'une certaine façon, pas dans le sens péjoratif du terme, mais dans le sens qu'on est un peu fermé, parce qu'on n'a pas encore eu l'occasion de montrer tout ce qu'on fait parce que c'est relativement nouveau... et aussi il y a le code, qui est une syntaxe que tout le monde n'aime pas, parce qu'il y a des lignes où tu te dis ouh la la, ça peut me rappeler mes cours de mathématique ou de physique quand j'étais à l'école... et si je n'aimais pas ça, peut-être... ou je ne sais pas, il peut y avoir plein de raisons, ça c'est un truc que je viens d'inventer bien sûr... on peut avoir des fois peur du code, plus que d'un langage comme max où c'est visuel et peut-être tu dis dit ah oui, je sais ce que je suis en train de connecter, je sais que c'est la sortie là qui va rentrer là, et qui va sortir par là... donc c'est beaucoup plus direct comme approche 

VG — Antescofo, il y a peut-être aussi un côté confidentiel dans le fait que... le code est public? 

JMF — oui, c'est public, tu peux le télécharger sur l'IRCAM  

VG — ah je ne savais pas ça 

JMF — mais tu n'es pas le premier à me le dire  

VG — parce qu'il y a une startup qui a été créé et je suis sur la mailing list donc je pensais que quand il sortirai le logiciel je serais au courant, mais je n'ai pas vu passer le code... 

JMF — oui, ils ont sorti un logiciel qui s'appelle Metronome... non, Metronaut... et maintenant la distribution IRCAM est gratuite, je ne suis jamais allé la télécharger sur le forum IRCAM, mais théoriquement tu peux t'inscrire sur le forum et la télécharger... et il y a des tutoriels ... mais voilà ça reste encore un peu une niche parce que d'un côté c'est très nouveau, et d'un autre il y a l'aspect code qui peut être un peu aussi, euh... c'est pour ça que SC il y a beaucoup moins d'utilisateurs, parce qu'apprendre c'est un peu plus compliqué... Max a ce côté un peu comme un jouet où tu t'amuses comme un Mécano, c'est beaucoup plus amical et c'est même ludique, je pourrais dire ... tandis que quand tu dois écrire des lignes de code et penser « ah, ça ça doit rentrer là » c'est un peu plus un truc d'information, de geek... 

VG — oui, tu oublies le point virgule... 

JMF — oui, voilà, tu oublies le point virgule ou quelque chose comme ça et ça ne marche plus, donc oui tu perds des heures sur ça... bon après sur Max tu peux perdre des heures sur autre chose, mais je pense qu'il y a cette contrainte là, qui est un peu illusoire à mon avis, parce que bon moi, comme je t'ai dit j'étais complètement addict à Max, et finalement il n'y a pas trop de différence dans la programmation graphique et la programmation textuelle, c'est finalement la même chose mais voilà l'un à l'air plus pour jouer et l'autre plus un truc de geek, et tu peux créer des préjugés très facilement, parce que peut-être on a un peu de paresse aussi, parce que peut-être il faut réfléchir un peu plus, bon avec Max aussi il faut réfléchir parce que c'est quand même de la programmation... bon mais c'est ça que je veux dire, qu'à la fin les deux sont des langages de programmation, l'un est visuel, graphique ou block-diagram, et l'autre c'est textuel et en série, mais c'est quand même un raisonnement logique, si tu veux que ça, ça rentrer là et que ça fasse un multiplier et diviser par machin-truc, c'est la même chose, sauf que l'un tu écris une ligne et l'autre tu connectes des boites...  

VG — sur Max, c'est comme on dit le « low entry fee »  

JMF — comment tu dis ? 

VG — c'est le concept du « low entry fee » et « high ceiling » qui avait été formulé dans un article par David Wessel (cf. \cite{wessel_problems_2001}, NdE), de pouvoir rentrer facilement dans un logiciel, ce que ça va te coûter pour installer le logiciel et faire tes premiers sons d'une certaine manière... 

JMF — oui, voilà ... on a l'habitude, surtout dans notre société, qui nous ramène de plus en plus aux trucs déjà prêts, on a ça déjà presque dans les gènes, et donc quand tu vois un truc qui ne marche pas du premier coup, ou qui est trop compliqué, tu dis non c'est pas la peine si avec Live je peux faire la même chose ou beaucoup plus et je ne vais pas avoir toute cette courbe d'apprentissage pour faire un son qui fait pouet pouet, tandis que dans Max for Live j'ai tout un environnement déjà prêt où je prends un module et je fais un gros son tout de suite, tu vois... et je peux faire des \textit{break-point} et moduler et donc dans deux secondes je te fais une compo géniale et dans l'autre truc, ça fait un an que je tape mon clavier et voilà, j'arrive à faire quelques sons mais je n'ai pas la flexibilité et ce que j'ai dans Live... sauf que ... bien sûr tu peux faire tout, c'est mon avis encore une fois, les gens peuvent le partager ou non, et surtout pas ceux qui font du Live, mais dans Live le problème c'est quand même que tu as une boite à outil qui est très fermée, même si tu as l'impression que c'est très ouvert, parce que tu peux prendre, bon, comme dans mon système, des plugins et tu les mets, mais... mais c'est tout, tu ne peux pas faire plus que ça... après si tu veux faire des trucs un peu plus complexes avec le rythme de là qui va aller contrôler ça, ou même récursif... là tu ne peux pas parce qu'il est fait seulement pour avoir une timeline... c'est fait pour faire un type de musique assez spécifique, pour faire des loops, bien sûr maintenant ça c'est beaucoup étendu donc tu peux faire des trucs plus complexes que faire des loops ou tu peux faire des loops très complexes aussi... mais ça reste quand même orienté dans un truc déterminé... et moi je n'aime pas ça, justement, parce que je veux peut-être expérimenter, avoir des curseurs allant à différentes vitesse, et qui vont faire peut-être des rubatos par exemple qui vont peut-être changer dans le temps, qui vont, voilà, s'entremêler et moduler d'autres curseurs qui vont être en bas par exemple, c'est l'idée de... de tempos qui vont être moduler entre eux, des choses comme ça, et chaque groupe va être différent, de séquences, de tracks, de break-points... mais voilà si je veux faire ça, je ne peux pas si je veux suivre un instrumentiste avec les inputs, il n'y a pas que je sache, ou peut-être on peut le faire, un plugins Max for Live avec Antescofo, je ne sais pas mais pourquoi pas... mais bon c'est peut-être plus difficile après pour gérer... donc tu vois c'est vraiment très... 

VG — c'est comme l'autoroute... tu vas plus vite mais tu n'as pas le temps d'aller explorer les paysages que tu traverses...  

JMF — voilà... c'est parfait comme métaphore, je vais la noter celle là... elle est parfait celle-là, parce que justement c'est une autoroute qui peut te permettre de faire passer des gros camions, mais ... 

VG — ... tu rateras les chevreuils, les sangliers, et les autres espèces inconnues sur ta route... 

JMF — Oui, et c'est finalement ce qui arrive dans notre société où on nous façonne notre façon de faire... dès petit on nous dit il faut faire ça, ça, ça, et puis quand tu es plus grand, il faut faire ça, ça, ça... et après il y a peut-être tout un autre monde beaucoup plus riche à découvrir qui est à côté... et voilà c'est exactement, c'est parfait ta métaphore... 

VG — La création c'est peut-être aller voir sur les chemins de côté... 

JMF — voilà... après n'empêche que peut-être l'outil va évoluer et que peut-être tu peux faire des petits sauts sur les côtés, juste pour voir un peu et après revenir dans ton autoroute, mais disons c'est ça l'idée que ça continue à évoluer pour peut-être se ramifier un petit peu, mais tu ne vas pas pouvoir non plus aller voir peut-être le truc qui est vraiment là... 

VG — ... caché au fin fond de la forêt... 

JMF — ... au fond de la forêt, derrière la dune où tu peux te baigner et il y a le soleil magnifique, une cascade... (rires) 

VG — c'est pour ça qu'on fait du SC, c'est pour trouver la cascade au fond de la forêt... (rires) 

JMF — voilà... bon... je ne sais pas si c'est vraiment ça mais on peut faire de la métaphore et de la comparaison et c'est surtout que voilà, moi je ne me suis pas du tout intéressé à Live, même ici, ils l'utilisent en prod à fond, tu vois les RIMs utilisent à fond Live, Max for Live, parce que tu mettre... il y a quelques RIMs qui l'utilisent qui ont tout un environnement dans Max mais pour faire du prototypage... donc tu es avec le compositeur qui dit je veux faire ça et ça, et donc au lieu de faire le patch Max, tu prends tes plugins, tu les mets dans ta chaine et après tu fais quelques courbes, et donc là le compositeur dit « ah, super, c'est génial c'est ça » et après il repassent dans Max par exemple ... donc ça c'est une pratique, je ne sais pas s'ils le font encore, mais c'était une des pratiques qu'il y avait, avec en parallèle les mêmes traitements dans l'un que dans l'autre, sauf que dans Max c'est plus difficile parce qu'il faut mettre les boites, bon sauf si tu as un système de scripting comme je te décrivais avant, où tu peux faire des patchs dynamiquement... et surtout oui, l'idée c'était que je puisse enlever, remettre, tu vois, des modules, automatiquement dans le patch...  

VG — oui, Serge Lemouton en parlait hier de ça, du fait que non seulement les gens utilisaient Live mais que certaines pièces qui étaient finalisées avec Live et que ça devenait aussi un problème pour la préservation, la conservation et la documentation des œuvres... il parlait des pièces qui ont vingt ans et déjà avec Max, qui est un langage d'assez bas niveau, donc tu peux arriver à reconstituer une pièce même si les versions de Max ont changé, mais Live c'est beaucoup plus complexe et touffu comme environnement, donc ça pose plus de problèmes... 

JMF — oui, oui, c'est un problème, s'il y a un nouveau, c'est très risqué, bon moi je m'en fous un peu mais si jamais on peut penser à un logiciel qui tout d'un coup est un grand concurrent de Live et que tout le monde commence à passer sur cet autre logiciel, Live d'ici dix ans n'existe plus... je ne pense pas qu'ils vont mourir tout de suite, mais ça peut arriver... et comme souvent dans les marchés, c'est les marchés qui bouffent tout, et s'il n'est plus compétitif, on ferme la boite... et tu es obligé de garder les ordinateurs, parce que ça ne va pas marcher sur l'OS suivant... donc voilà, il faut garder tout... ce qu'il va falloir faire c'est garder les ordinateurs avec les configs, tout, dans des espèces de stockage... bon après il va y avoir des problèmes de stockage... et aussi les maintenir dans le temps, parce que si tu laisses des machines qui ne marchent plus pendant dix ans ou vingt ans, quand tu va l'ouvrir tu vas appoyer sur power et il va y avoir un silence... plutôt que l'accord de démarrage de mac... donc c'est sûr que c'est un gros problème... l'avantage des langages textuels c'est justement ça, que les définitions sont faites dans un fichier texte et donc c'est rien du tout, mais toutes les abstractions et toutes les définitions sont un langage plus ou moins informatique donc c'est facile de le ré-interpréter... même si tu veux faire un portage, tu peux facilement prendre le code, le transformer, ou même créer un script qui va transformer un langage comme par exemple Antescofo à cet époque, et peut-être dans cinquante ans quelqu'un voudra jouer cette pièce et il prendra le langage et il va le traduire dans peut-être un autre qu'il y aura à l'époque... mais comme c'est textuel, il faudra juste un petit script qui va scanner, analyser le truc et comme c'est que des fonctions logiques, ça va le ... 

VG — ...le porter dans « hyper collider » ... 

JMF — voilà... je ne sais pas ce qu'il va y avoir dans cinquante ans ... mais bon pour moi, ça n'est pas trop un problème, pour Serge bien sûr parce qu'il est en plein dedans, pour moi ce n'est pas ça plus que le problème plus que c'est une boite fermée... il y a des environnements qui sont des boites fermées et moi je ne veux pas faire ça... 

VG — oui, il y a aussi un problème personnel de se trouver déposséder de ton travail parce que le logiciel meure... 

JMF — oui, bon là tu migres après dans un autre truc, je l'ai fait plusieurs fois... j'utilisais ProTools à la base et un jour je n'en pouvais plus avec DigiDesign, c'était à l'époque, parce que je voulais avoir un truc plus puissant, mais il fallait acheter une carte qui valait, je ne sais pas, 5000€ ou 10000€ et j'ai dit non, laisse tomber je n'utilises plus ce truc... et j'ai arrêté du jour au lendemain, et j'ai fait du Logic (Apple Logic Pro, NdE) que je pouvais craquer plus facilement, je n'avais pas besoin d'avoir une carte... bon, après je l'ai acheté, bien sûr... mais ça marchait, c'était presque pareil... bon maintenant il y a Ardour qui est pas mal aussi... bon je ne l'utilises pas mais... un des avantages des logiciels Open-Source ou gratuit, c'est qu'il y a toujours, c'est le cas pour SC, il est gratuit depuis 15 ans à peu près, quand James McCartney a décidé de le donner en Open-Source et il y a toujours eu cette communauté, mais si ça a changé, mais le logiciel reste toujours Open-Source, gratuit et il continue à avancer ... peut-être que ça n'avance pas à des pas gigantesque, mais là, voilà si on voit Max8, on voit aussi que même si c'est un truc payant, ça n'avance pas non plus énormément, ça reste plus ou moins la même chose... le moteur de base c'est toujours le même et le code pour faire du DSP, ils ne l'ont pas optimisé ... et ce truc là justement ne permet pas trop de faire de trucs dynamiques, parce que ça peut commencer à cliquer, il n'est pas fait pour ça, tu vois... donc tu ne peux pas commencer à créer et détruire des patchs à la volée parce qu'il n'est pas optimisé pour faire ce genre de trucs... 

VG — ça a commencé peut-être justement comme un logiciel qui permettait d'aller dans des petits chemins à explorer et ils essaient un peu de faire des bretelles vers l'autoroute... ils ont optimisé le workflow pour que tu puisses travailler plus rapidement ... 

JMF — oui, là en plus c'est acheté par Ableton... donc ils sont passé du côté ... je ne sais pas si c'est obscur ou clair de la force, mais ça va vers l'autoroute, ça c'est mieux comme métaphore ... où le graphisme est très important, tellement important que c'est même dérangeant des fois, des trucs automatiques parce que tu ne voulais pas connecter là et il te le fait... j'ai eu un peu de mal à passer à Max7 justement, parce que je faisais les câbles et je voulais connecter là et ça partait vers le haut, parce que maintenant tu peux connecter le haut et le bas, donc tu as plus de possibilités mais des fois tu n'as pas l'habitude donc c'est plus des gadgets ... je ne sais pas si dans Max8 par exemple ils ont fait le truc pour connecter automatiquement, donc tu sélectionnes comme avec le plugins Toolbox (package Max Toolbox de Nathanaël Lécaudé, NdE)... 

VG — en partie oui... 

JMF — oui, parce que j'ai vu que tu pouvais insérer des objets... ça c'est super, ça existe dans Pd-extended depuis vingt ans je pense... c'était quelqu'un qui avait fait des extensions pour Pd où tu pouvais faire ça, il y a vingt ans je te dis...  

VG — oui, je ne sais pas pourquoi ils ne l'ont pas intégrer plutôt, la Max Toolbox, même maintenant ce n'est pas encore intégré... les connections un vers multiple ce n'est pas encore ça... 

JMF — oui, et ça c'est des trucs de base, parce que la tendinite que j'ai eu à force de câbler, je l'ai encore...  

VG — je vois très bien ce dont tu parles... ça s'améliore, mais ce n'est pas encore ça... il y a un truc qu'ils ont mis en place, encore à titre expérimental, c'est d'essayer de moins faire de distinction entre le mode d'édition et le mode d'action...  

JMF — entre lock et unlock 

VG — oui... que tu puisses faire un bang ou une number box alors que tu es en mode d'édition... 

JMF — mais ça on peut le faire avec la touche Pomme... 

VG — oui, c'est un peu ça mais dans l'autre sens en fait ... 

JMF — oui, mais tu vois ça reste superficiel... oui ça peut être sympa mais bon on a déjà des habitudes donc ... 

VG — c'est des choses pour faciliter le \textit{workflow} et le \textit{low entry fee}... parce que toutes les choses de scripting dynamique et tout ça, ce sont des choses très avancées, il n'y a pas tant de gens qui font des choses comme ça en fait... je serai curieux de savoir quel pourcentage d'utilisateurs ça représente... je ne pense pas que cela ne soit lié qu'à ça mais il y a une tendance générale dans les logiciels avec l'arrivée de tablettes et d'iOS etc. d'application monotâche... qui font un seul truc... avant tu avais des logiciels qui permettaient de faire plein de choses, des logiciels comme Word ou des suites Office ou photoshop, qui existent toujours pour la production, mais il y a aussi maintenant plein de logiciels qui permettent d'appliquer un certain nombre de filtres sur une image ou de faire ... 

JMF — ... oui pour faire plus vite et qu'il n'y ait pas besoin de connaître... 

VG — des logiciels très limités avec peu de fonctions mais qui vont être connectés par exemple ... qui vont faire un seul truc très spécifique... 

JMF — oui 

VG — et pour l'audio, il y a aussi parmi les apps sur iOS, tu vas trouver un granulateur par exemple, un programme qui fait uniquement de la synthèse granulaire... c'est un peu un plugin autonome...  

JMF — Peut-être qu'on va vers ça... on est en plein dedans je pense même... parce qu'ici l'IRCAM c'est une espèce de niche avec peu de monde, donc on est peut-être un peu des marginaux, d'une certaine façon... et en relation à la tendance générale, comme Live ou même Max for Live, c'est pas pour tout le monde non plus, on en parle comme ça, mais... mais peut-être pour un utilisateur qui fait que du Live, un logiciel comme Max4Live il va trouver que c'est pour les geeks... tu vois... que pour nous, c'est un truc qu'on ne veut même pas utiliser... mais tu vois, il y a peut-être une tendance à ce genre d'outils qui sont prêt à utiliser... plug'n play ... où tu as deux boutons et voilà... et peut-être que nous on est des extra-terrestres résistants qui ne veulent pas s'aligner dans le courant, dans l'autoroute justement ... je ne sais pas pourquoi mais on ne veut pas prendre l'autoroute et on continue à... bon, après bien sûr c'est... 

VG — en tout cas il y a une tendance à y avoir plein de logiciels très simples qui ne vont réaliser qu'une seule tâche... mais ce n'est pas forcément l'autoroute dans le sens où c'est aussi un peu, même si le contexte économico-social est très différent, le fonctionnement de Linux ... dans Linux il y a aussi cette tendance à faire des petits packages unitaires qui vont faire juste une fonction particulière, mais avec l'idée que tu peux utiliser des \textit{pipes} pour envoyer le résultat de la sortie d'une application dans l'autre, ou comme JACK (JACK Audio Connection Kit, développé par Paul Davis, NdE) qui s'est développé aussi comme ça ... 

JMF — oui donc tu peux créer un réseau fait de petits bouts, quoi... 

VG — voilà... tu as un logiciel qui ne fait que éditeur de notes, qui ne fait aucune synthèse, et tu vas envoyer ça avec JACK-MIDI dans un logiciel qui ne fait que la synthèse ... et ce fonctionnement très modulaire correspond aussi à un développement par des gens qui font ça de manière bénévole ... Développer un gros logiciel, c'est quelque chose qui est difficile pour une communauté anarchique et informelle ... Si tu veux développer un truc comme Live, c'est un peu les limites, par exemple Ardour existe mais ça a été très compliqué je crois pour Paul Davis de porter un truc comme ça tout seul parce que c'est une grosse usine... et c'est difficile de maintenir ça tout seul ou même à plusieurs sans qu'il y ait au moins une organisation, pas forcément lucrative, pas forcément une entreprise mais une fondation comme pour Wikipedia où se décident les politiques de développement... alors que les petites applications, c'est possible de les faire en tant qu'indépendant ...  

JMF — oui... mais bon j'ai vu quelques séquenceur qui ont l'air pas mal dans iOS et aussi un espèce de JACK où tu peux connecter différentes trucs par MIDI ou audio...  

VG — oui, c'est des choses qui se développent... 

JMF — oui... c'est peut-être une voie mais je pense qu'il faut aussi des choses centralisées parce que tu peux avoir différents satellites, on peut dire, mais il faut quand même un truc central qui va donner les informations et qui va recevoir les informations de chaque satellites, qu'est ce qu'il est en train de calculer, qu'est ce qu'il est en train de faire, mais moi je vais lui donner l'ordre maintenant de faire telle tâche... donc c'est peut-être une voie mais il faut quand même un qui gère d'une certaine façon ... après il peut y avoir un fonctionnement un peu comme internet où il n'y a pas vraiment un serveur central mais que tout le monde peut parler avec tout le monde sans hiérarchie... mais bon il faut quand même il faut l'exprimer quelque part et pour moi pour l'instant c'est Antescofo... après oui, les satellites peuvent par exemple les synthétiseurs dans SC c'est un satellite, mais lui quand même il créé des sons mais aussi il me renvoie des informations parce qu'il y beaucoup d'analyseurs, par exemple d'\textit{onset detection} ou je sais pas quoi ... et il va aussi me renvoyer ... et aussi tous les niveaux des vumètres donc je peux avoir par exemple un iPad et je vais créer une interface où je vais tout voir les groupes de groupes ... où chaque slider c'est un groupe de groupes et donc je n'ai pas besoin d'avoir l'interface sur l'ordinateur parce que je l'ai sur iPad mais je reçois les VUmètre, la somme de trucs qui sont en train de marcher sur ce \textit{track}... dans les concerts je fais comme ça, j'ai mon iPad avec les différentes \textit{tracks} dont je peux jouer avec les niveaux de sons si je veux un peu plus de niveau... 

VG — ...comme je commence par poser la question de ce qui t'a amené à faire ça, la question finale si on en reste là pour aujourd'hui, c'est si tu fais un pronostic pour dans 10 ans, 20 ans ou 50 ans, comme tu disais tout à l'heure... qu'est ce que tu vois, sachant qu'on n'est pas encore arrivé aujourd'hui encore, y arrivera t on jamais, à un langage ou un environnement idéal... qu'est ce qui te manque, toi ? 

JMF — non je pense que moi ce qui me manque dans l'immédiat c'est peut-être d'avoir plus de facilité d'intégrer tout ce \textit{workflow} dans quelque chose d'un peu plus souple ... parce que bon là, c'est toujours le code ou l'interface graphique d'un côté... donc pour moi, ce qui serait intéressant, ce serait d'avoir un logiciel qui soit mixte, comme un miroir... avec d'un côté une représentation graphique, mais d'un autre le code... donc tu peux faire des allers-retours... et donc si je veux faire un slider je le fais tout de suite avec le code ou je le fais comme dans Max, je prends le slider, je le créé et puis voilà automatiquement il va me faire la sélection au niveau graphique ... mais ça peut être aussi au niveau d'une chaîne de traitement, où je peux avoir une chaîne de traitement et de l'autre côté je vais avoir une représentation en graphe et les connections et que je peux même changer et ça va automatiquement changer le code... un truc comme ça où le \textit{workflow} c'est beaucoup plus dynamique, la façon d'interagir... parce que voilà mon truc ça a l'air très dynamique et je peux faire beaucoup de trucs mais je suis toujours dans le code et des fois voilà le code, si tu oublies une virgule ou un crochet, voilà il faut ... bon au moins le programme est intelligent et il te dit où tu as fait une erreur et tout ça ... mais peut-être que j'aimerais des fois éviter de faire du code et donc ... parce que bon, moi aussi, je suis toujours un Maxeur mais je pense que la partie graphique c'est quand même important comme un retour aussi d'expérience, des connections, voilà toutes les machines audio qu'on a, même les analogiques, on connecte des câbles... c'est peut-être pour ça aussi que ça marche très bien , parce que c'est très analogique en fonction de comment marche les machines en général, au moins pour l'instant... parce que dans le futur, justement, il n'y aura peut-être plus de câbles mais pour l'instant il faut câbler, et tu sais c'est très intuitif comme raisonnement ... le code c'est beaucoup plus abstrait comme mode de raisonnement, c'est beaucoup plus bas niveau ... et donc peut-être qu'il manque dans mon truc de pouvoir passer de l'un à l'autre, je ne sais pas si je vais le faire un jour... bon là je fais des interfaces qui se créent automatiquement pour avoir un peu la main sur différent trucs, mais aller encore plus profond même dans le code, dans les structure et trouver des façon de créer des algorithmes d'une façon graphique qui vont après être interprétés en code ... après bien sûr il y a des trucs que tu ne vois pas tout simplement, du moins au jour d'aujourd'hui, je ne vois pas comment l'imaginer de faire un graphisme des types de processus par exemple... que dans le code, ce n'est qu'une ligne de code, mais comment je vais représenter ce que fait ce processus d'une façon graphique, je ne sais pas... peut-être qu'il y a des logiciels comme Mathematica ou je ne sais pas, des choses qui font ça, ou d'autres trucs des mathématiciens... mais au moins pour tout ce qui est du contrôle, ce serait bien d'avoir ce logiciel idéal, en ce moment, qui serait voilà, de passer de l'un à l'autre avec les deux représentations... comme Gen (l'extension de Max permettant de créer de manière visuel des graphes convertis en code compilé, NdE) aussi, tu vois... voilà Gen, il y a un peu aussi cette idée... 

VG — FAUST aussi ... 

JMF — FAUST oui, mais dans FAUST tu ne peux pas aller interagir sur la représentation graphique... mais Gen tu peux agir et changer les câbles et ça va te changer le code automatiquement, donc ça par exemple c'est un truc qui m'intéresse dans Max... et puis bon après, je n'ai pas la boule de cristal avec moi pour te dire dans 50 ans mais je pense que ça dépend beaucoup, pas que de la musique, mais au moins il y a une partie de la société et de la politique, et de comment va le monde ... et par exemple on ne sait pas si des structures comme l'IRCAM vont exister dans 50 ans... c'est une structure qui soutient une sorte de recherche fondamentale, d'une certaine façon, c'est vraiment des réflexions... peut-être que ça va être chacun chez lui... parce que je ne crois pas que ça va exister encore... mais bon, c'est mon idée, mais j'espère que non, que ça va durer pendant des siècles... mais toujours les choses ont une durée de vie, tu nais, tu vis, et puis tu meurs et ça peut arriver pour tout, même pour la Terre, on sait bien que ça va finir un jour et même le système solaire... donc ce genre d'expérimentation, c'est je pense chacun qui va la faire chez lui, parce que bon on n'a plus finalement besoin de grandes structures comme l'IRCAM parce que tu peux bidouiller chez toi, et après s'il y a la communauté, comme tu dis, ça c'est très important de garder les communautés, de les agrandir, des communautés de gens qui veulent expérimenter ou bidouiller, parce que c'est ça qui est intéressant je pense... Si tu restes dans ton autoroute, bon c'est pas très ... tu peux peut-être gagner de l'argent, faire plein de trucs, mais c'est pas très intéressant pour certains genres de personnes, donc voilà... pour d'autres peut-être c'est super, c'est leur vie, ils s'éclatent à fond mais il y en a d'autres que ça n'intéressent pas... et voilà je pense que comme va le truc, il ne faut pas être je pense magicien ou avoir la boule, pour voir qu'il y a une tendance vers que les trucs culturels disparaissent de plus en plus, parce que tout simplement ça n'intéresse plus les politiques et comme c'est les politiques qui décident ou les gens qui sont classés à un haut niveau, il y a de moins en moins de... il y a toujours ce truc d'aller vers le plus simple, donc c'est assez naturel finalement, le logiciel avec un seul bouton mais c'est pareil partout, donc on ne va pas aller se compliquer l'existence à exister qui demande des efforts et donc je pense qu'on perd un peu la partie plus culturelle d'une certaine façon et ... peut-être que ça a été toujours comme ça, mais peut-être avant les politiciens, et surtout ici en France, ils avaient quand même une culture de ce qui se faisaient dans les arts, la littérature, la musique... il y avait quand même, bon, le truc qui a fait exister l'IRCAM c'est parce qu'il y a eu une rencontre politique et qui a fait que ce truc existe, mais ce genre de personnes ça a l'air qu'elles n'existent plus ... donc si tu vas demander à un politicien qui est Maessian par exemple, il va dire qu'il ne sait pas... c'est peut-être un compositeur très ancien et hyper connu qui a formulé toutes les notes et peut-être même Debussy, ils ne savent pas qui c'est où dans la musique classique, on peut parler d'autres genre de musiques, qui existaient dans le passé ... et donc il y a une espèce d'aculturisation... je vois ça, même si je ne suis pas un grand lecteur de philosophie parce que je n'y comprends rien, mais on voit ça très net, je pense, et c'est ce qui va faire que, et c'est le cas, qu'on a de moins en moins d'argent par rapport aux années 1980 et 1970... il y a eu une grande chute et ce qui est arrivé en Italie, c'est que la musique contemporaine, ou les centres d'expérimentation de ce genre là n'ont plus de subventions et si tu n'as pas de subvention, tu ne peux pas monter le moindre truc quoi... donc là je pense que, quand même, la curiosité ça fait partie de l'être humain, ça ne va jamais disparaître même s'il y a une tendance à faire toujours le truc avec le bouton, il y a je pense toujours quelques uns qui sont en train de bidouiller derrière et voilà je pense que c'est ça qui va continuer ... et bon il y a aussi la partie universitaire qui peut aussi des conférences internationales où on peut se recontrer... parce que des fois il y a des choses super intéressantes, il y a des gens qui vont là et qui montrent des trucs super intéressants... bon si tout continue à peu près comme ça, on ne sait jamais s'il y a une bombe nucléaire qui tombe ou une météorite... ça ferait une autre situation... 

VG — oui, probablement que ça changerait la situation... 

JMF — on pense toujours qu'on est immortels, mais on ne sait jamais ce qu'il va se passer demain donc on ne peut pas peut-être faire un pronostic de trop longe durée... mais si on pensait que tout continue plus ou moins comme maintenant, dans cet espèce de stabilité, même s'il y a plein de trucs qui se passent, mais bon l'histoire de l'humanité ça a toujours bougé pas mal... et donc l'IRCAM on voit qu'il a besoin de se renouveller, parce qu'il ne peut pas continuer avec les mêmes paradigmes des années 1970 et 1980 parce que c'était très focalisé sur le compositeur... maintenant le compositeur, bon on ne sait même pas ce que c'est, qu'est ce qu'il faut faire, quel type de musique... donc il y a aussi une crise d'une certaine façon, dans la musique elle-même, parce qu'il y a toujours ce truc de vouloir faire un truc nouveau, nouveau, nouveau... mais bon après il y a aussi des limites ... bon, moi je suis aussi positiviste de ce côté là parce que je crois encore que mon système et peut-être ce truc que je peux générer plein de choses va pouvoir créer des musiques, peut-être pas nouvelles, mais qui vont être très souples et qui vont faire ça, c'est mon idéal, te prendre et te faire voyager... mais voilà... je ne sais pas... mais je pense aussi que, c'est une évidence aussi, qu'il y a pas mal de gens qui bossent chez eux finalement ... donc je pense que ça, ça va continuer... et il y a toujours des gens qui vendent des Bela ou des Arduinos donc on peut continuer à bidouiller sans problème... après sûrement l'informatique, si tout va bien, va évoluer aussi... donc il y a des ordinateurs quantiques qui se profilent et donc peut-être dans 50 ans on aura les premiers prototypes d'ordinateurs quantiques... 

VG — ... on les aura probablement avant, quand même... 

JMF — oui, peut-être avant... ils disent qu'il y en a déjà quelques uns mais que c'est pas vraiment des quantiques ... mais oui espérons avant... et donc sûrement quand ça arrivera il va y avoir une réduction dans l'informatique, dans la façon du rapport aussi avec les machines, et donc tout ça fait que les gens qui font de la musique et qui bidouillent ça va continuer, et du moment qu'il y aura l'ordinateur quantique, il y aura des gens qui vont aussi bidouiller avec ça pour faire de la musique mais je ne sais pas quel genre de son ou systèmes de spatialisation ultra-sophistiqués ... ou le rêve de Stockhausen, qu'il disait dans un entretien, qu'il va penser la musique et que la musique va se créer tout seul, et on va peut-être pouvoir créer directement sans aucune interface tactile ni rien, tu vas faire de la musique ou de l'improvisation... on va se regarder et on va faire des sons qu'avec le regard... En tout cas, moi je pense que tant que l'humanité va exister, il va y avoir des nouveaux trucs, bon même si on revient aussi en arrière, c'est aussi pour aller vers d'autres chemins... parce qu'on a toujours dit que, ah oui l'homme va pouvoir aller sur la lune, pensant que l'homme ne va jamais aller sur Mars, donc on pense comme ça maintenant mais on ne pas ce qui va arriver dans 100 ans, tu vois, peut-être qu'un homme va pouvoir aller à l'autre bout de l'univers parce qu'il va passer par un trou noir ou je ne sais pas quoi... disons que chaque chose que l'Homme a pensé qu'il ne pouvait pas faire, ou qu'il pouvait faire, finalement il l'a fait ... 

VG — c'est souvent au moment où on pense que ce n'est pas possible de le faire que ça devient possible de le faire... parce que quelqu'un dit que ce n'est pas possible de faire quelque chose, il y a toujours quelqu'un d'autre qui a envie de lui donner tort... et à partir du moment où les choses sont pensées, elles existent déjà ... 

JMF — oui, voilà, à partir du moment où tu as l'imagination, tu vas tout faire pour aller jusque là... et nous on est confronté à ça parce que voilà, tu as ton idée de ton interface et tu commences à la faire et voilà tu l'as construite même si tu y passes beaucoup de temps... il faut taper, y passer des heures, mais bon on a une espèce d'obsession que ce truc c'est possible de le faire et puis tu sais intuitivement que c'est possible ... parce que bon je sais qu'il y a que je ne peux pas faire... par exemple ce logiciel qui a la partie graphique et le code, je sais que au vu de l'état actuel de l'informatique que c'est absolument possible de le faire, mais moi je n'ai pas les compétences de programmation pour le faire pour aller bidouiller du C++ ou je ne sais pas quoi, donc je peux avoir l'idée, mais j'aurais besoin de collaborateurs ou de dire à quelqu'un de le faire et puis faire un groupe de travail sur ça, mais donc voilà... mais bon en tout cas je suis très positiviste, donc je ne crois pas du tout quand quelqu'un me dit « ah c'est tout déjà inventé » je lui dit c'est n'importe quoi parce que c'est la même chose qu'on a dit toujours et puis le jour d'après, il y a un nouveau truc qui bouleverse tout... donc dans ce cas là, ça sera peut-être les ordinateurs quantiques ou une autre technologie, je ne sais pas quoi, ou quelque chose qu'on n'a même pas pensé... et dans la musique c'est pareil, la musique c'est très plastique et une théorie que j'ai, c'est que la musique tu peux faire tout ce que tu veux... la musique c'est comme une pâte à modeler, et plus tu as d'outils, plus tu peux la modeler comme tu veux, faire des métamorphoses et elle va continuer à se modeler au fur et à mesure que l'être humain va continuer à vivre je pense... c'est inifini... tout est possible...  

VG — c'est le mot de la fin ? Tout est possible ? 

JMF — voilà c'est bien... tout est possible 
 % ok 13/06/2018
	
%	\chapter{Algorithms}
\label{appendix:algorithms}

\section*{}
\subsection*{}
\subsubsection*{frettage audio-tactile}

Algorithme pour déclencher des impulsions audio pour donner l'impression de frettes virtuelles.
Valeurs normalisé entre 0 et 1.

\vspace{-1em}
\begin{itemize}[noitemsep]
\item  S = série ordonnée des valeur de position de frettes
\item  x[n] = position x du stylet, sous forme de signal synchrone, au temps n;
\item  p[n] = pression du stylet, sous forme de signal synchrone, au temps n.
\end{itemize}

Détecter la frette courante :
 $$zone[n] = \sum_{i=1}^{length(S)} x[n]>S(i) $$ 

Détecter un changement de frette :
 $$absdelta[n] = abs(zone[n] - zone[n-1]) $$ 

Créer une enveloppe à partir de cette impulsion :
 $$pulse[n] = pulse[n-1] + \frac{(absdelta[n] - pulse[n-1])}{slide} $$ 

Modulante d'impulsion de fréquence $$f_mod$$ :
$$cycle130[n] = sin(2\pi\frac{f_m}{SR})$$

Signal audio de frettage, avec P[n] la pression :
$$fretsignal[n] = P[n] * cycle130[n] * pulse[n]$$

\noindent
\begin{minipage}{.5\linewidth}
	\begin{equation}
   		Z[n] = \sum_{i=1}^{length(S)} x[n]>S(i)
	\end{equation}
\end{minipage}%
\begin{minipage}{.5\linewidth}
	\begin{equation}
  		P[n] = \lVert zone[n] - zone[n-1] \rVert > 0.5
	\end{equation}
\end{minipage}
 % 
\end{appendices}
\bookmarksetup{startatroot}

%\setlength{\parskip}{0mm} % moins d'espace entre les items (mais affect la suite ?)
%\singlespacing
\printnoidxglossaries
%\printglossaries
% \cleardoublepage
%\printglossary[type=domain]

\printindex[people]
%\printindex[works]

% --------------------------
% Back matter
% --------------------------
{%
\setstretch{1.1}
\renewcommand{\bibfont}{\normalfont\small}
\setlength{\biblabelsep}{0pt}
\setlength{\bibitemsep}{0.5\baselineskip plus 0.5\baselineskip}
\printbibliography[title={Bibliographie}, nottype=online]
%\printbibliography[heading=subbibliography,title={Sites web},type=online,prefixnumbers={@}]
}

\clearpage

\phantomsection
\addcontentsline{toc}{chapter}{\listfigurename}
\listoffigures

\noindent Les illustrations dont l'auteur.e n'est pas précisé ont été réalisées par l'auteur de la thèse et sont sous licence Creative Commons CC BY-NC-SA.

%\clearpage

\phantomsection
\addcontentsline{toc}{chapter}{\listtablename}
\listoftables

\clearpage

% !TEX root = ../thesis-example.tex
%
\pagestyle{empty}
\hfill
\vfill
\pdfbookmark[0]{Colophon}{Colophon}
\section*{Colophon}

This thesis was typeset with \LaTeXe.
Its stylesheet was inspired byt the \textit{Clean Thesis} style developed by Ricardo Langner.
The design of the \textit{Clean Thesis} style is inspired by user guide documents from Apple Inc.

%\cleardoublepage

%\input{content/declaration}
%\clearpage
\newpage
\mbox{}

% **************************************************
% End of Document CONTENT
% **************************************************


\end{document}
