% !TEX root = ../thesis-example.tex
%
\chapter{Algorithmes interactifs / software}
\label{ch:algorithms}

\cleanchapterquote{One general effect of the digital revolution is that avant-garde aesthetic strategies became embedded in the commands and interface metaphors of computer software. In short, the avant-garde became materialized in a computer.}{Lev Manovitch}{The language of the new media}


\noindent La citation de Lev Manovitch en exergue de ce chapitre en reflète la question centrale: le code est l'élément définissant le fonctionnement d'un \gls{DMI} et si sa nature virtuelle rend son usage très malléable, la nature des objets virtuels que l'on manipule est en grande partie conditionnée par les représentations que l'on se fait de l'interaction musicale. Cependant, les machines ont aussi leur contraintes et limitations, qui influencent la conception des protocoles et les choix d'implémentation. Après avoir présenté le contexte informatique dans lequel s'inscrivent les développements logiciels d'informatique musicale temps-réel, je détaillerai trois développements en lien avec ces questions : 
\vspace{-1em}
\begin{itemize}[noitemsep]
	\item \textbf{le concept de ``modèle intermédiaire dynamique''}, qui au dela des questions de mise en relation de variables, propose un modèle abstrait pour l'interaction geste/son;
	\item \textbf{le protocole MP}, qui propose une solution alternative au \gls{MIDI} pour répondre aux questions d'agencement et de connexions de modules de traitement;
	\item \textbf{la librairie \textit{sagrada}}, qui propose un système original de synthèse granulaire modulaire dans Max contrôlée de manière synchrone ou asynchrone;
\end{itemize}

%%%%%%%%%%%%%%%%%%%%%%%%%%%%%%%%%%%%%%%%%
\section{Matériaux algorithmiques}

\noindent Dans le cas des \glspl{DMI}, les données sont essentiellement :
\vspace{-1em}
\begin{itemize}[noitemsep]
	\item \textbf{des signaux audio}, enregistrés dans des tables d'ondes ou sous forme de flux;
	\item \textbf{des algorithmes}, définissant les opérations mathématisques à effectuer pour traiter les données audio
	\item \textbf{des objets}, définissant les opérateurs, les réglages et xxx todo;
	\item \textbf{un graphe}, définissant l'agencement des opérateurs.
\end{itemize}

Dans les environnements de programmation audio, les flux de données sont généralement séparées en deux types majeurs\footnote{C'est notamment le cas dans Max, SuperCollider, Ableton Live, Csound ... qui traitent le signal par blocs plutôt qu'échantillon par échantillon, pour des raisons d'optimisation.} : 
\vspace{-1em}
\begin{itemize}[noitemsep]
	\item \textbf{des signaux synchrones}, représentant par exemple des signaux audio, échantillonés à une fréquence précise, et traités généralement par bloc dans un \gls{DSP};
	\item \textbf{des événements asynchrones}, arrivant de manière sporadique, tels que les messages \gls{MIDI}, \gls{OSC} ou d'autres types de messages propres au programme;
\end{itemize}

\noindent Ces deux types de flux sont généralement associé aux notions d'\textit{audio-rate} et de \textit{control-rate}, définissant deux granularité indépendantes dans le traitement du signal.


\textbf{keywords} ergodynamisme, résonance, mapping.\\

En général, découplage conceptuel (et logiciel) entre la partie DSP et la partie "contrôle". Dans la réalité, ce n'est pas si simple et c'est la raison pour laquelle ce qui est habituellement rangé du côté "mapping" et du côté "synthèse" est rassemblé dans un seul et même chapitre.


\todo{Présentation de MP, de sagrada et de leur articulation (et par là même de l'articulation entre contrôle et synthèse)}

L’atomisation jusqu’au sample, plutôt que séparation entre DSP et messages de contrôle
=> faust, gen\textasciitilde{ } et cie : l’export vers des plateformes multiples.

\subsection{La notion de mapping}

\noindent Dans le domaine des lutheries numériques, le terme de ``mapping'' est très couramment utilisé pour décrire la relation entre les données issues des capteurs (en particulier de l'interface gestuelle) et les paramètres de la synthèse sonore. 
Le terme de mapping est issu des mathématiques dans lequel il se traduit en français par le terme ``d'application'', c'est à dire de relation entre deux ensembles pour lequel chaque élément du premier (appelé ensemble de départ) est relié à un unique élément du second. Andy Hunt le définit donc dans \cite{hunt_towards_2000} comme \iquote{la liaison ou la correspondance entre les paramètres de contrôle (dérivés des actions de l'interprète) et les paramètres de synthèse sonore}\footnote{``the liaison or correspondence between control parameters (derived from performer actions) and sound synthesis parameters''}.\\
\indent Ce terme a glissé dans l'usage du design d'instrument numérique en raison de la relative simplicité des relations dans les premiers développements et est resté en usage alors que les relations de correspondance sont aujourd'hui bien plus complexes, comme le notait déjà Joel Chadabe en 2002, dans \cite{chadabe_limitations_2002}: \iquote{Le mapping décrit la façon dont un contrôle est connecté à une variable. Mais à mesure que les instruments deviennent de plus en plus complexes pour inclure de grandes quantités de données, une sensibilité au contexte, ainsi que des capacités de génération sonore et musicale, le concept de mapping devient plus abstrait et ne décrit pas les réalités plus complexes des instruments électroniques\footnote{Mapping describes the way a control is connected to a variable. But as instruments become more complex to include large amounts of data, context sensitivity, and music as well as sound-generating capabilities, the concept of mapping becomes more abstract and does not describe the more complex realities of electronic instruments.}.}\\
\indent D'après Hunt et Wanderley \cite{hunt_mapping_2002}, deux directions dans le design de mapping :
\vspace{-1em}
\begin{itemize}[noitemsep]
	\item \textbf{generative mechanisms}, such as neural networks to perform mapping;
	\item \textbf{explicitly defined mapping strategies}.
\end{itemize}

Exemples de schéma de mapping de Wanderley dans \cite{wanderley_escher-modeling_1998}.

MnM : \cite{bevilacqua_mnm_2005}

``complex mappings are more satisfactory, after a learning phase, than one-to-one mapping'' in \cite{wanderley_mapping_2002}

\extra{\cite{hunt_mapping_2002}: \iquote{(...) we define mapping as the act of taking real-time performance data from an input device and using it to control the parameters of a synthesis engine.}}

%-----------------------------------
\subsection{Sons de synthèses ou échantillons ?}

\noindent La question est souvent posée, concernant la synthèse sonore de \glspl{DMI}, s'ils utilisent des sons de synthèse ou des sons enregistrés, oubliant dans le même temps que les sons enregistrés sont, lorsqu'ils sont rejoués, des sons de synthèse.\\
\indent Il serait alors tentant de faire une distinction entre des sons synthétisées à partir de purs modèles mathématiques et des sons synthétisés à partir d'échantillons enregistrés dans des tables d'onde. Pourtant, là encore, cette distinction serait vaine
\indent Cette distinction est plus adéquate lorsqu'on parle de synthèse  d'une analogie avec la différence qu'il existe en image vectorielle, composée de primitives géométrique qui la définisse et une image matricielle, définie directement par la matrice de ses pixels.\\
synthèse extensive ou intesive.

\todo{reformuler ça}

%---------------------------------------
\subsection{Synthèse sonore ou contrôle}

Sound  \cite{di_scipio_sound_2003}

\noindent On a tendance à séparer \todo{réf?}, dans la description et le design des \glspl{DMI}, une partie \gls{DSP}, traitant un \textit{signal synchrone}, souvent associée voire confondue à la synthèse audio, et une partie \textit{contrôle}, souvent associé à des données de type \textit{messages asyncrones} (e.g. typiquement, des messages \gls{MIDI}) venant contrôler le \gls{DSP}.\\
\indent Il existe une différence de nature assez profonde entre ces deux types de données, à la fois du côté de leur implémentation technique\footnote{programmation synchrone vs. asynchrone, priorité d'horloge, etc.} mais également dans leur rôle et leur fonction dans l'interaction musicale. 
Ainsi, le signal se présente souvent comme l'\textit{image d'un signal analogique échantillonné} avec l'idée de continuité qui lui est associée. L'autre un graphe asynchrone, se présentant souvent comme l'image d'un ordre d'éxecution (e.g. note-on). \\
\indent D'un côté on a un signal continu et monophonique (un seul échantillon est traité à la fois), de l'autre un signal sporadique et potentiellement polyphonique (plusieurs messages peuvent être envoyé au même instant logique).\\
\indent Cependant, du point de vue de leur utilisation, les frontières entre ces deux types de données numériques d'une part, et ces deux natures d'objets conceptuel (le signal continu et l'ordre événementiel) ne se recouvrent pas exactement. \\
\indent D'une part, il est courant d'utiliser les messages asynchrones de Max pour convoyer des variables continues (telles que les données d'un contrôle MIDI-CC ou d'un capteur envoyé par \gls{OSC}).\\
\indent D'autre part, il est également possible d'utiliser un signal synchrone pour déclencher des événements (musicaux) discrets. C'est par exemple le cas lors qu'on envoie une impulsion dans une synthèse de type \gls{KarplusStrong}, ou quand on utilise directement une impulsion audio pour déclencher une enveloppe ou la lecture d'un échantillon.
Cette technique notamment utilisée dans \cite{bascou_gmu_2005} permet de déclencher des flux de grains à haute fréquence (plus rapidement que le système de message de Max ne le permet), ce qui permet notamment de le passage progressif entre flux rythmique et son continu (ref Stockhausen Kontakte)\\
\indent L'arrivée du multi-canal dans le logiciel Max a également conforté cette notion d'utilisation de signaux audio comme messages de contrôle.

\todo{Mettre une explication et une ref vers FTM.}

%%%%%%%%%%%%%%%%%%%%%%%%%%%%%%%%%%%%%%%%%
\section{Modèles intermédiaires}
\label{sec:algorithms:MID}
Articles du la notion de DIM (@SMC), article sur le jeu de pitch (@ICMC)


augmenter la quantité nombre de dimensions (de peu de paramètres à beaucoup de paramètres)
augmenter la qualité

\cite{momeni_dynamic_2006}, \cite{winkler_making_1995}, 

Toolboxes : ESCHER (Wanderley et al)\cite{wanderley_escher-modeling_1998}, Mapping library for Pd (Steiner) \cite{steiner_towards_2006}, MnM (Bevilacqua et al.) \cite{bevilacqua_mnm_2005}, Libmapper

%------------------ Figure : représentation du mapping #1 #2 ---------------------
\begin{figure}[!htbp]
	\captionsetup{format=plain}%
	\centering
	\begin{minipage}[t]{0.48\textwidth}
		\includegraphics[width=\linewidth]{gfx/04_algorithms/Wanderley_Schema1.png}
		\caption[Représentation du mapping \#1]{Mapping d'après Wanderley xx todo date}
		\label{fig:algorithms:DynamicMappingLayer1}
	\end{minipage}
	\hspace{.02\linewidth}
	\begin{minipage}[t]{0.48\textwidth}
	  \includegraphics[width=\linewidth]{gfx/04_algorithms/Wanderley_Schema2.png}
		\caption[Représentation du mapping \#2]{Mapping d'après Wanderley xx todo date}
		\label{fig:algorithms:DynamicMappingLayer2}
	\end{minipage}
\end{figure}
%------------------ Figure : représentation du mapping #1 #2 ---------------------

%------------------ Figure : représentation du mapping #3 #4 ---------------------
\begin{figure}[!htbp]
	\captionsetup{format=plain}%
	\centering
	\begin{minipage}[t]{0.48\textwidth}
	 	\includegraphics[width=\linewidth]{gfx/04_algorithms/Momeni-DynamicMappingLayers.png}
		\caption[Représentation du mapping \#3]{Mapping d'après Momeni et Henry}
		\label{fig:algorithms:DynamicMappingLayer3}
	\end{minipage}
	\hspace{.02\linewidth}
	\begin{minipage}[t]{0.48\textwidth}
	  \includegraphics[width=\linewidth]{gfx/04_algorithms/complexPatch.png}
		\caption[Représentation du mapping \#4]{Mapping, dans la vraie vie\footnote{Image postée sur le forum Max, en tant qu'offre d'emploi visant à remettre le patch en ordre \url{https://cycling74.com/forums/job-offer-1}}.}
		\label{fig:algorithms:DynamicMappingLayer4}
	\end{minipage}
\end{figure}
%------------------ Figure : MP-event et MP-block ---------------------

\subsection{Motivations}

\noindent Les \glspl{DMI} nous mettent face à une équation bancale. D'un côté quelques flux de données issus de capteurs rendent péniblement compte de la finesse du geste physique, de l'autre la possibilité (et le désir) de produire une musique plus complexe qu'aucun instrument acoustique ne peut le faire. La réponse à cette équation réside dans la question centrale de ce que l'on nomme habituellement le \gls{mapping}, c'est à dire la relation de correspondance entre valeurs issues de capteurs et paramètres de synthèse. Cependant, et malgré sa popularité\footnote{Entre 2001 et 2015, le terme "mapping" a été employé dans plus de 750 articles de la conférence \gls{NIME} alors que "design d'interaction" (interaction design) n'était employé que dans 160 articles.}, le terme de mapping semble assez peu représenter la complexité de ce qui n'est pas une simple mise en relation et nous préférerions parler d'un design d'interaction, c'est à dire d'une programmation des relations entre gestes et sons faisant intervenir modèles dynamiques, scénarios évolutifs, fragments de partitions et réglages d'un nombre extrêmement élevé de variables.\\
\indent Les algorithmes de synthèse et de transformation peuvent présenter un nombre de paramètres trop élevés pour qu'il soit possible de les contrôler individuellement. Diverses méthodes de réduction du nombre de paramètre à gérer par le musicien sur son interface de jeu ont été proposées, telles que l'utilisation de paramètres intermédiaires abstraits \cite{wanderley_escher-modeling_1998}, la projection de l'espace des paramètres sur un espace perceptif \cite{wessel_timbre_1979}, ou d'exploration par voisinage \cite{tubb_divergent_2014}, de modèles physiques \cite{todo}, de modèles d'apprentissage \cite{lee_real-time_1991, fiebrink_real-time_2011}.

\indent Par ailleurs, la qualité du geste capté ne rend pas forcément compte de la granularité gestuelle qu'on peut observer, par exemple, dans l'interaction fine qui se produit au contact de la surface des matériaux physiques : la souplesse d'un plectre, la rugosité du filetage d'une corde, ou la souplesse du feutre d'un marteau de piano viennent apporter une qualité particulière au son qui ne se résume pas à la vibration d'une corde théorique idéale. Il existe ainsi, dans les instruments acoustiques, divers ``intermédiaires'' entre le geste et le résonateur qui viennent influencer la qualité de leur interaction.\\
\indent Ces systèmes intermédiaires peuvent parfois être interchangeables (on peut jouer sur une guitare avec les doigts, un plectre, un \textit{bottleneck}, etc.) et contribue à élargir la palette de timbres et de styles de jeu d'un même instrument. En transposant cette idée dans le domaine numérique, on peut envisager les modèles intermédiaires dynamiques comme de tels systèmes interchangeables et visant à modifier la qualité des mouvements captés par une interface sensible.

Les modèles génératifs peuvent amener à une liaison trop incertaine entre le geste et le résultat. Les modèles physiques\footnote{Voir par exemple, dans le cas de la synthèse temps-réel les librairies \gls{STK} développée par le \gls{CCRMA} et PMPD initialement développée par Cyrille Henry.} sont une alternative qui pallie ces deux problèmes en offrant à la fois un modèle intermédiaire entre le geste et le résultat qui réduit le potentiellement grand nombre de paramètres de synthèse à un plus petit nombre de paramètres gestuels.\\
\indent Un problème des modèles physiques est d'une part leur instabilité dans les régimes non-linéaires et d'autre part, que la cohérence de la relation entre le geste et le son souhaitée par le musicien ne s'appuie pas nécessairement sur les règles de la mécanique newtonienne \footnote{On pourrait dire à certain égards que les instruments de musique cherchent à subvertir les lois de la physique, cf. section \ref{sec:gesture:subversion}}. Par exemple, on peut assigner à un modèle géométrique des mouvements, un tempo, une cinétique qui épousent la géométrie de l'objet sans toutefois qu'il y soit question de masse et que l'objet ait une ``cohérence physique''.

\subsection{Caractéristiques souhaitées}

\noindent Les MIDs sont ainsi définis par les caractéristiques suivantes:
\vspace{-1em}
\begin{itemize}[noitemsep]
	\item \textbf{enrichir le geste} capté avec des modulations opérant à des fréquences au-delà des fréquences gestuelles humaines (par ex. rebonds rapides);
	\item \textbf{démultiplier le geste} et pouvoir, à partir d'un seul geste d'entrée, contrôler une multitude d'agents;
	\item une intégration facile dans une architecture logicielle modulaire, qui permette de tester et ajuster ses modèles en direct;
	\item la possibilité de piloter simultanément différents moteurs de synthèse et de rendu avec les mêmes algorithmes, afin d'améliorer la cohérence de l'ensemble;
	\item être bidirectionnel : le modèle intermédiaire doit pouvoir communiquer dans les deux sens avec l'interface, les autres DIM ou le moteur de synthèse (cf. Fig. 1) afin de réguler les interactions (non linéaires) entre les différents étages.
\end{itemize}

Les processus à l'œuvre dans un modèle intermédiaire agissent sur plusieurs aspects de la transduction de mouvement attendue dans un instrument de musique.
\vspace{-1em}
\begin{itemize}[noitemsep]
	\item qualité du mouvement: Pour donner un exemple acoustique, un plectrum pour une corde
offre une qualité de pincement particulière et son attaque est différente de celle des doigts. De plus, la position des cordes sur le corps de l'instrument permet de transformer le mouvement apparemment linéaire de la main en une variation de ce mouvement particulier, à savoir une succession de pincées dont le rythme est lié à l'espacement des cordes. Les dispositifs d'interaction sont souvent dépourvus de rugosité de surface (par exemple, une tablette à stylo n'a pas la rugosité du crin d'un archet). Un DIM devra intégrer cet "aspect de surface" de l'objet virtuel.
	\item mouvements non linéaires : La richesse du son est en partie liée à l'ensemble des phénomènes non linéaires en action dans un instrument de musique et contribue à rendre le son riche et subtil. Ces non-linéarités dues en partie aux matériaux des instruments acoustiques (et électroniques) manquent souvent des valeurs numériques standardisées des paramètres des instruments logiciels. Dans le domaine numérique, les non-linéarités se retrouvent en effet souvent dans les bugs et les abus de codecs, toutes sortes de "pépins" qui sont précisément recherchés par une veine de musique électronique portant désormais ce nom, pour produire des sonorités riches et des paysages sonores imprévus. Le DIM devrait donc réintroduire la saturation, les courbes exponentielles, la distorsion et autres gigue dans les fonctions de transfert des instruments numériques. 
	\item Mouvement augmenté : En agissant sur des modèles complexes, un modèle unidimensionnel
peut être convertie en un mouvement polyphonique multidimensionnel. Par exemple, sur un instrument à cordes, les nombreuses notes d'un accord peuvent être jouées d'une seule touche. Ce renforcement du mouvement peut agir sur les dimensions verticales (poly-phoniques), horizontales (polyrythmiques) ou sur les nombreuses dimensions du timbre. On peut par exemple contrôler un "paramètre cible" qu'un ensemble d'éléments atteindrait par une logique de déplacement qui leur est propre.
\end{itemize}

\subsection{Implémentation}
Première implémentation en 2009.
Re-implémentation avec MP.

\subsection{Evaluation}
Utilisation dans la MM. Utilisation dans FIB\_R pour contrôler image et son.


%%%%%%%%%%%%%%%%%%%%%%%%%%%%%%%%%%%%%%%%%
\section{MP : un protocole de connexion modulaire, polyphonique, expressif}
\label{sec:algorithms:MP}

%---------------------------------------------------------------------------
\subsection{Motivations et revue des protocoles existants}

\noindent Nous avons vu dans la section précédente comment le concept de modèle intermédiaire dynamiques pouvait enrichir le geste capté et améliorer l’ergonomie des instruments de musique numériques. Cependant, un des facteurs critiques rencontré lors de ces développements se situait dans la manière de faire communiquer différents modules polyphoniques\footnote{Par « module polyphonique », on entend ici des processeurs traitant simultanément plusieurs flux de données de contrôle de même nature en parallèle, e.g. le filtrage des points de contact sur une interface multi-touch ou encore la modulation des différentes notes d'un accord.} entre eux.\\
\indent Un certain nombre de protocoles asynchrones\footnote{Les protocoles de contrôles que nous envisageons ici sont asynchrones, fondés sur l’idée que les systèmes musicaux envisagés dans les lutheries actuelles sont des systèmes complexes composés d’éléments hétérogènes et intégrant notamment des interfaces \textit{hardware} elle-mêmes asynchrones. Bien que la synthèse audio soit un processus synchrone, le design global d'un \gls{DMI} est le plus souvent un système \iquote{globalement asynchrone, localement synchrone}, tel que défini par Daniel M. Chapiro \cite{chapiro_globally-asynchronous_1984}.} dédiés au contrôle temps-réel de la synthèse numérique ont vu le jour depuis les années 1980. Au delà de proposer des solutions techniques concrètes, ces protocoles sont porteurs d’un modèle implicite représentant les objets en présence dans un contexte d'interaction musicale. Une brève revue montrera comment ceux-ci se sont progressivement ouverts, à mesure que les capacités de calcul se sont accrues et que la notion même d’instrument s’élargissait à de nouveaux champs tels que les installations sonores ou les applications musicales interactives.

\subsubsection{MIDI}

\noindent La norme \gls{MIDI}, proposée en 1983, reste encore aujourd'hui le protocole le plus répandu pour le contrôle de la synthèse audio. La profusion de nouvelles interfaces et applications l'auront tout juste fait évoluer pour permettre la prise en charge de nouvelles technologies de réseau (rtpMIDI\footnote{ Encapsulation du \gls{MIDI} dans des messages \gls{RTP} permettant une communication sur des réseaux ethernet et WiFi.}) ou de nouvelles interfaces (\gls{MPE}, voir plus bas)).
Les limitations du \gls{MIDI} ont pourtant été identifiées peu de temps après son apparition\footnote{Voir par exemple \cite{moore_dysfunctions_1988}, \cite{mcmillen_zipi_1994} ou \cite{selfridge-field_beyond_1997}}, notamment :
\vspace{-1em}
\begin{itemize}[noitemsep]
	\item sa précision et son espace de nommage sont limités;
	\item l'identifiant d'une note est assimilé à son (éventuelle) hauteur;
	\item l'état actif d'une note est assimilé à sa vélocité;
	\item la modulation individuelle des notes est fastidieuse;
	\item sa nomenclature fait référence aux instruments acoustiques.
\end{itemize}

\noindent Le \gls{MIDI} élude une partie de la question du mapping en reliant intrinsèquement le geste à la production sonore à travers le concept de note \gls{MIDI}\footnote{ Une note \gls{MIDI} est composée d'une valeur de hauteur et d'une valeur de vélocité associées à canal \gls{MIDI}.} qui assimile les deux côtés de l'interaction : la notion de vélocité se rapportant au geste et celle de pitch au son.\\
\indent La dernière évolution du \gls{MIDI} a été motivée par la commercialisation récente d'interfaces dites expressives\footnote{ Citons le LinnStrument, Roli Seabord, Haken Audio’s Continuum, Eigenharp Alpha, Madrona Labs’ Soundplane, le KMI K-Board Pro4 ou prochainement Joué.}, c'est à dire permettant la modulation indépendante de chaque note jouée. Bien qu'elles ne soient pas les premières interfaces permettant un tel contrôle, un effort conjoint a été entrepris par plusieurs fabricants pour définir standard nommé \gls{MPE}. Cette évolution n'est cependant qu'une normalisation de l'usage des canaux \gls{MIDI} actuels permettant un tel contrôle dans le cadre existant, et non un nouveau protocole qui dépasserait les limitations intrinsèques au \gls{MIDI}.

\subsubsection{ZIPI}

\noindent En 1994, Zeta Instruments et le \gls{CNMAT} proposèrent \gls{ZIPI}\cite{mcmillen_zipi_1994} pour dépasser les limitations du \gls{MIDI}. \gls{ZIPI} fait ainsi la distinction entre note, hauteur, canal et vélocité, augmente la précision des données, introduit des messages de modulation par note, la possibilité d'un réseau en étoile (plutôt que le chaînage linéaire du \gls{MIDI}) ainsi qu'une méthode d'interrogation des instruments connectés.\\
\indent \gls{ZIPI} propose également une organisation hiérarchique à trois niveaux, héritée d'une classification traditionnelle où les \iquote{orchestres} sont des ensembles de \iquote{familles d'instruments}, composées \iquote{d'instruments}, définis par un ensemble de \iquote{notes}. \gls{ZIPI} introduit enfin deux espaces de nommage distincts pour la description du geste d'une part et de la synthèse audio d'autre part.\\
\indent Malheureusement, le public ciblé --~les fabricants et utilisateurs de synthétiseurs hardware~-- n'était pas prêt pour un tel changement alors que l'avènement du protocole \gls{firewire} cette même année palliait le faible débit de données du \gls{MIDI}\footnote{ Le débit d'un bus \gls{MIDI} était jusqu'alors de 31,25 kbit/s en connection DIN uni-directionnelle; le \gls{firewire} proposait jusqu'à 400Mbit/s tout en étant bi-directionnel.}.

\subsubsection{Max et les ``nombres asignifiants''}

\noindent Dès 1985, un logiciel introduisait une nouvelle manière de connecter des dispositifs entre eux. Héritant à la fois de la logique des \textit{opcodes} de la série de logiciels \gls{MUSIC-N} et d'une ergonomie calquée sur le câblage hardware, Max\footnote{initialement développé par Miller Puckette sous le nom ``The Patcher''.} offre la possibilité de connecter des opérateurs de bas niveau sur des flux de données libérés de toute référence à ce qu'il pouvait représenter (geste, son ou autre), d'où l'expression ``\textit{meaningless numbers}''\footnote{Cette notion de nombres asignifiants fait éminemment écho aux propos de Gilles Deleuze, exprimées au début des années 1980 dans Mille Plateaux \cite{deleuze_mille_1980} : ``Un agencement machinique est tourné vers les strates qui en font sans doute une sorte d'organisme, ou bien une totalité signifiante, ou bien une détermination attribuable à un sujet, mais non moins vers un \textit{corps sans organes} qui ne cesse de défaire l'organisme, de faire passer et circuler des particules asignifiantes, intensités pures, et de s'attribuer les sujets auxquels il ne laisse plus qu'un nom comme trace d'une intensité.'' p. 10} formulée à leur propos par Zicarelli dans \cite{zicarelli_communicating_1991}.\\
\indent Un facteur ayant contribué au succès d'un logiciel comme Max est la possibilité qu'il laisse de pouvoir brancher plus ou moins n'importe quelle variable sur une autre, permettant ainsi une grande souplesse dans l'élaboration de scénarios d'interaction ainsi qu'une approche expérimentale dans l'élaboration des \glspl{mapping}.\\
\indent Cependant, la gestion de processus polyphoniques y reste délicate. Si un objet comme \verb|poly~| permet effectivement de créer des instances multiples d'un même processus, son adressage reste fastidieux\footnote{La gestion de la polyphonie a été améliorée par la prise en charge de messages au format \gls{MPE}, ainsi que par l'ajout de signaux audio polyphoniques dans la dernière version. Cependant, le format \gls{MPE} reste contraint par les limites pré-citées}. En particulier, la distribution d'un processus polyphonique dans plusieurs modules élémentaires indépendants, permettant leur ré-agencement, n'est pas intégrée de manière aussi souple que la connection de modules non-polyphoniques. L'utilisation de matrices pour le contrôle de variables en nombre est une option possible, mais elle n'est adaptée, en terme de performances, que lorsque l'on a affaire à un ensemble homogène de variables de taille pré-determinée\footnote{Le changement dynamique de taille de matrice étant une opération couteuse, mal-adaptée aux exigences du temps-réel de la performance.} plutôt qu'à des événements sporadiques arrivant à la volée et ne se prête pas à la diversité des situations rencontrées dans le design interactif de \glspl{DMI}.

\subsubsection{OSC : Open Sound Control}

\noindent En 1997 au \gls{CNMAT}, un groupe incluant d'anciens concepteurs de \gls{ZIPI} ré-utilisa cette recherche pour développer le protocole \gls{OSC} \cite{wright_open_1997}, motivé par les possibilités offertes par la communication en réseau et le désir de prendre en compte l'extension des types de données alimenté par l'utilisation grandissante de logiciels comme Max. En proposant une syntaxe intelligible et facile à utiliser, \gls{OSC} a connu un certain succès: il a été adopté par un certain nombre de logiciels et interfaces hardware\footnote{ Comme le Lemur, le Monome ou l'Ethersense.} et utilisé pour définir d'autres protocoles tels que \gls{GDIF}, \gls{TUIO}, ou encore la librairie ``o.''\footnote{ « Oh dot » : package pour Max, développée au \gls{CNMAT}.}.\\
\indent Cependant, sa relative lourdeur en terme de débit comparé au \gls{MIDI} \cite{fraietta_open_2008} et une absence de nomenclature rendant fastidieuse les branchements plug'n play l'ont pour l'instant privé d'une adoption par l'industrie et le grand public.

\subsubsection{TUIO: Tangible User Interface I/O}

\noindent Dans cette brève revue, il faut mentionner \gls{TUIO} \cite{kaltenbrunner_tuio:_2005} (fondé sur \gls{OSC}) comme le premier protocole\todo{vérifier à quel point c'est le premier} à introduire un indice incrémentiel pour identifier de manière unique des événements dynamiques et éphémères tels que les touchés de doigt sur une interface utilisateur tangible (\gls{TUI}).\\
\indent Il se distingue également de la logique des événements \gls{MIDI}, dont l'utilisation de messages distincts pour les note-on et -off peut produire des notes qui restent ``bloquées'' si un message note-off est perdu. Il propose une solution simple et pratique pour résoudre l'incertitude d'arrivée des messages envoyés sur \gls{UDP}, consistant à envoyer systématiquement la liste des événements actifs.

\subsubsection{Le signal}

\noindent S'il n'est généralement pas envisagé tant comme un signal de contrôle que comme un support pour l'audio numérique\footnote{David Zicarelli parle encore d'\textit{audio channels} dans une conférence présentant la nouvelle version de Max sensée changer la manière de les considérer : \url{https://www.youtube.com/watch?v=Y4YLy7kqcr8}}, le signal peut être utilisé à cette fin et même permettre la transmission d'événements asynchrones. Le développement de la librairie ``Sagrada'', présentée dans la section \ref{sec:algorithms:sagrada}, en donne un example.\\

%-----------------------------------------------------------------------------------
\subsection{Description générale}

\noindent Le système MP (\textit{Modular Polyphony})\footnote{Disponible sur: \url{https://github.com/LAM-IJLRA/ModularPolyphony}} est un protocole et un ensemble d'outils facilitant la connexion modulaire de processus polyphoniques. Il se compose de blocs de traitements polyphoniques nommés \textit{MP-blocks}, communiquant par des messages asynchrones nommés \textit{MP-messages} et représentant des objets temporels abstraits nommés \textit{MP-events}. Le but de cette librairie est d'améliorer la modularité en conservant l'indépendance des blocs individuels de traitement par rapport au design général de l'interaction.\\
\indent MP s'inspire de certaines des idées présentes dans les protocoles pré-cités. En particulier, il reprend un concept général du \gls{MIDI} qui conçoit le contrôle musical temps-réel comme une séquence d'événements temporels ayant un début et une fin. Il emprunte aussi à \gls{ZIPI} l'idée d'un découplage entre identifiant, hauteur, vélocité, canal ainsi que la nuance entre l'activation d'une note et sa modulation. Le protocole MP est ainsi fondé sur un paradigme à trois états permettant la modulation expressive de tout paramètre.\\
\indent Par ailleurs, ce système étant destiné à une lutherie expérimentale et exploratoire, MP laisse le typage et l'espace de nommage des paramètres ouverts, sans le restreindre à une nomenclature arbitraire. Néanmoins, il propose une syntaxe plus orientée qu'\gls{OSC} et que les ``nombres asignifiants'' de Max, pour faciliter une gestion cohérente de l'interconnection de modules.\\
\indent Enfin, MP propose une stratégie originale permettant l'association dynamique d'événements entre eux, de telle sorte qu'il soit possible de contrôler les paramètres par groupes (et sous-groupes).\\

%-----------------------------------------------------------------------------------
\subsection{MP-events}

\noindent Un \textit{MP-event} est un objet temporel abstrait qui peut être traité par des \textit{MP-blocks}. Il est défini par un ensemble de \textit{MP-messages} (figure \ref{fig:algorithms:MP-event-model}). Ces messages sont composés de paramètres de contrôle précédés par un identifiant unique propre au \textit{MP-event}. Le format de message est minimaliste et tous les messages utilisent la même syntaxe : un identifiant unique, un nom de paramètre suivi d'une liste de valeurs. Par exemple :
\vspace{-1em}
\begin{itemize}[noitemsep]
	\item{\verb|[42 pitch 112| : le \textit{MP-event} \#42 règle le paramètre de pitch à la valeur 112;}
	\item{\verb|[123 scale 0 2 4 5 7 9 11]| : le \textit{MP-event} \#123 définit une gamme diatonique.}
\end{itemize}
\todo{formater les messages code en verbatim et ou avec une box}

\noindent Deux noms de paramètre sont réservés pour un usage particulier: \texttt{state} et \texttt{guests}. Ils seront détaillés dans les prochaines sections. Nous verrons également qu'un \textit{MP-event} peut suivre plusieurs chemins de traitement en parallèle et être fusionné avec d'autres \textit{MP-event}. 

%------------------ Figure : MP-event et MP-block ---------------------
%\begin{figure}[!htbp]
%	\captionsetup{format=plain}%
%	\centering
%	\begin{minipage}[t]{0.48\textwidth}
%		\includegraphics[width=\linewidth]{gfx/04_algorithms/MP-event-model.pdf}
%		\caption[Représentation schématique d'un \textit{MP-event}]{Représentation schématique d'un \%textit{MP-event}. Les cercles pleins représentent les états \textit{on}, les cercles vides les %états \textit{off} et les lignes continues les états \textit{update}.}
%		\label{fig:algorithms:MP-event-model}
%	\end{minipage}
%	\hspace{.02\linewidth}
%	\begin{minipage}[t]{0.48\textwidth}
%	  \includegraphics[width=\linewidth]{gfx/dummy.pdf}
%		\caption[dummy]{dummy}
%		\label{fig:algorithms:dummy-1}
%	\end{minipage}
%\end{figure}
%------------------ Figure : MP-event et MP-block ---------------------

%------------------ Figure : MP-event ---------------------
\begin{figure}[!htbp]
	\captionsetup{format=plain}
	\includegraphics[width=\textwidth]{gfx/04_algorithms/MP-event-model.pdf}
	\caption[Représentation schématique d'un \textit{MP-event}]{Représentation schématique d'un \textit{MP-event}. Les cercles représentent des \textit{MP-messages}, les courbes représentent l'évolution du paramètre dans le processus. Noter que les paramètres ne sont pris en compte que lorsqu'un message \textit{state} est reçu.}
	\label{fig:algorithms:MP-event-model}
\end{figure}
%------------------ Figure : MP-event ---------------------


\subsubsection{Identifiant}

\noindent L'identifiant unique (ID) sert à identifier un \textit{MP-event} tout au long de ses traitements. Le \textit{MP-event} peut provenir d'un capteur physique (e.g. une touche de clavier) ou d'une source virtuelle (e.g. représentant le contact d'un doigt sur une \gls{TUI} ou le produit d'un algorithme génératif). L'ID est présent dans chaque \textit{MP-message} pour éviter toute erreur de routage dans le cas où un \textit{MP-block} reçoit parallèlement des \textit{MP-events} de plusieurs sources indépendantes. Les ID peuvent être définis explicitement --~auquel cas la gestion de conflit d'adressage est laissé à la charge du développeur~-- ou générés de manière unique par un objet dédié : \verb|mp.uID.maker|.

\subsubsection{Etat}

\noindent Le message ``\verb|state|'' est réservé et deux missions lui sont attribuées :
\vspace{-1em}
\begin{itemize}[noitemsep]
	\item il sert d'horloge asynchrone en déclenchant l'envoi des paramètres au processus;
	\item il spécifie la manière dont le processus doit interpréter ces paramètres.
\end{itemize}

\noindent Les paramètres peuvent ainsi être interprétés de trois façons :

\vspace{-1em}
\begin{itemize}[noitemsep]
	\item \textbf{state 1} : début de modulation de paramètre;
	\item \textbf{state 2} : mise à jour de paramètre;
	\item \textbf{state 0} : fin de modulation de paramètre.	
\end{itemize}

\noindent Le modèle musical sous-jacent envisage ainsi un \textit{MP-event} comme la modulation d'un ensemble de paramètres et prend en compte les phénomènes transitoires qui peuvent apparaître au début et/ou à la fin d'une modulation\footnote{ Un exemple évident est l'attaque d'un son, mais en ce qui concerne un processus non-sonore comme le filtrage de données, cela peut concerner l'initialisation du filtre.}. Ces discontinuités peuvent causer des réponses non-linéaires, trop rapides pour être contrôlées manuellement et parfois mieux traitées séparément.\\
\indent Dans l'algorithme de traitement, ces états peuvent correspondre à l'initialisation de variables internes, l'activation d'un lissage (\textit{portamento}) entre deux valeurs consécutives, le déclenchement d'un processus transitoire spécifique (e.g. l'attaque d'un son), etc. Tout paramètre peut donc être envoyé à un \textit{MP-block} en spécifiant s'il doit être considéré comme le début, la continuation ou la fin d'une phrase de modulation\footnote{ La confusion entre le \textit{pitch} et l'identifiant de note dans le protocole \gls{MIDI} rend le résultat de la même opération incertain : alors que certains synthétiseurs re-déclencheront la même voix, d'autres alloueront une nouvelle voix et attendront le même nombre de \textit{note-off} qu'il y a eu de \textit{note-on}.} Ce modèle à trois états semble correspondre par ailleurs aux intentions de la \gls{MMA} qui a annoncé un message de \textit{note-update} dans le futur protocole MIDI-HD\footnote{Rapporté par le site web Synthopia: \url{http://www.synthtopia.com/content/2013/01/20/midi-manufacturers-testing-new-high-definition-midi-protocol/}}.\\
\indent Notons toutefois que dans notre cas, le message d'état peut être rattaché à n'importe quel paramètre, ce qui diffère encore de l'implémentation \gls{MIDI}, notamment en ce qui concerne l'allocation de voix. Alors que le premier message \verb|state 1| reçu pour un ID causera l'allocation d'une voix dans le \textit{MP-block}, le message \verb|state 0| ne libérera pas nécessairement cette voix, cette décision revenant au processus en cours, comme nous le verrons plus loin.

\subsubsection{Guest-list}

\noindent La spécification MP ne suit pas d'organisation hiérarchique telle que les canaux \gls{MIDI} ou les familles d'instruments de \gls{ZIPI}. A la place, elle laisse la possibilité à tout \textit{MP-event} de déclarer une liste de \textit{MP-events} ``invités'' (\textit{guestlist}) à la volée. Ces invités pourront avoir accès à la voix allouée au \textit{MP-event} hôte et contrôler ses paramètres. Cette fonctionnalité nous offre une solution flexible pour le groupement d'événements, permettant un nombre arbitraire de niveaux hiérarchiques, sans pour autant être limité par une relation de subsumption.\\
\indent La \textit{guestlist} peut être utilisée dans le cas de \textit{MP-blocks} génératifs, où l'ID du \textit{MP-event} ``parent'' peut être ajouté à la guestlist des \textit{MP-events} ``enfants''. Ceci permet une modulation cohérente de plusieurs voix associées à des \textit{MP-events} générés par une même source. Des exemples concrets de cette situation sont les pistes \gls{MIDI} ou la hiérarchie orchestre/famille/instrument/note proposée par \gls{ZIPI}. Ils correspondent à un ``\textit{\gls{mapping}} divergent'' dans le schéma proposé par Rovan, Wanderley, Dubnov et al. \cite{rovan_instrumental_1997}.\\
\indent Un ``\textit{\gls{mapping}} convergent'' est également possible : plusieurs \textit{MP-events} peuvent être déclarés comme invités d'un \textit{MP-event} tiers. Par exemple, un \textit{MP-event} ``enfant'' peut être créé à partir du résultat de l'interaction de plusieurs \textit{MP-events} ``parents'' (cf. exemple section \ref{sec:algorithms:many-guests-to-one}).
\indent Notons qu'une solution alternative aurait pu consister à transmettre systématiquement la liste de tous les paramètres d'un \textit{MP-event} parent à ses enfants. Dans le cas de chaînes de mapping assez longue, ou de polyphonie élevées, cet héritage systématique s'avérait cependant trop lourd pour être une solution efficace.


\subsubsection{Master-ID}

\noindent Le paramètre guests ne nous laisse toutefois pas un accès aisé à l'ensemble des voix de polyphonies d'un \textit{MP-block}. A cette fin, un identifiant spécifique indexé 0, permet ce contrôle global. Il revient implicitement à considérer que le \textit{MP-event} \#0 (nommé master-ID) fait systématiquement partie de la guestlist de tout \textit{MP-event}. Dans le cas où les \textit{MP-events}, ses \textit{guests} et le master-ID tentent de modifier les mêmes paramètres, l'ordre de priorité est donné du plus spécifique au plus général, c'est à dire, au \textit{MP-event}, puis aux \textit{guests}, puis au \textit{Master-ID}\footnote{On retrouve également cette idée, quoique limitée par les contraintes du \gls{MIDI}, dans la notion de \textit{master channel} du protocole \gls{MPE}.}.

\subsubsection{Ordonnancement des MP-messages}

\noindent Le cycle de vie de la voix d'un \textit{MP-event} suit la séquence suivante de \textit{MP-messages} :
\vspace{-1em}
\begin{enumerate}[noitemsep]
	\item envoi des paramètres de début de modulation;
	\item envoi du message state 1;
	\item envoi des paramètres de modulation;
	\item envoi du message state 2;
	\item envoi des paramètres de fin de modulation;
	\item envoi du message state 0.
\end{enumerate}
\noindent Cependant, comme la libération d'une voix ne suit pas nécessairement un message \verb|state 0| (dans le cas où le processus a sa propre stratégie d'extinction), il est possible d'envoyer différents messages d'état plusieurs fois durant la durée de vie d'une voix, jusqu'à ce que la voix soit effectivement libérée.

%-----------------------------------------------------------------------------------
\subsection{MP-blocks}

\noindent Un \textit{MP-block} se compose de deux parties (cf. schéma \ref{fig:algorithms:MP-block-model} et implémentation dans Max, figure \ref{fig:algorithms:MP-simpleSynth}) : le routeur et le traitement polyphonique que nous décrivons ici.

%------------------ Figure : MP-block ---------------------
\begin{figure}[!htbp]
	\captionsetup{format=plain}
	\includegraphics[width=\textwidth]{gfx/04_algorithms/MP-block-model.pdf}
	\caption[Représentation schématique d'un \textit{MP-block}]{Représentation schématique d'un \textit{MP-block} avec le fonctionnement du routeur.}
	\label{fig:algorithms:MP-block-model}
\end{figure}
%------------------ Figure : MP-block ---------------------


\subsubsection{Le routeur}

\noindent Le routeur accomplit les fonctions suivantes :
\vspace{-1em}
\begin{itemize}[noitemsep]
	\item le stockage des paramètres reçus pour un \textit{MP-event};
	\item l'allocation de voix pour un nouvel \textit{MP-event};
	\item l'envoi des paramètres à cette voix;
	\item l'envoi éventuel des paramètres des \textit{guests};
	\item d'enregistrer la libération de la voix;
\end{itemize}

\noindent L'ordonnancement des messages envoyés à la voix de traitement assure une synchronisation du calcul afin qu'il ne soit fait qu'une seule fois. Ainsi, la séquence suivante est envoyée par le routeur à la voix cible :
\vspace{-1em}
\begin{enumerate}[noitemsep]
	\item start X state Y : ce message permet à la voix de se préparer à traiter les paramètres à venir selon l'état Y;
	\item tous les paramètres du master-ID;
	\item tous les paramètres des guests;
	\item tous les paramètres du \textit{MP-event} déclencheur;
	\item end X state Y : ce dernier message clôt la séquence et sert de signal d'horloge déclenchant le calcul.
\end{enumerate}

\noindent L'allocation de voix est réalisée à l'arrivée du premier message \verb|state 1| pour un \textit{MP-event} donné. La libération de la voix intervient quand le processus de traitement indique au routeur qu'il a terminé sa tâche. Trois scénarios sont alors possibles :
\vspace{-1em}
\begin{itemize}[noitemsep]
	\item Le processus se termine dès qu'un message \verb|state 0| est reçu. Ce sera le cas, par exemple, pour un processus tel qu'une addition ou un filtre médian. Ce scénario est pris en compte de manière automatique en spécifiant un attribut \verb|@automute 1| au routeur;
	\item Le processus déclenche son extinction à la réception d'un message \verb|state 0|, typiquement le release d'une enveloppe \gls{ADSR};
	\item Le processus a sa propre durée et peut avoir terminé sa tâche avant ou après avoir reçu un message state 0. C'est le cas, par exemple, lors de la génération d'un signal audio de durée fixe comme un échantillon de percussion.
\end{itemize}

\subsubsection{Traitement par voix}

\noindent Le traitement par voix est effectué par un patch chargé dans les multiples voix d'un objet Max \verb|poly~|. En dehors des fonctions permettant le traitement à proprement parler, un objet nommé \verb|mp-muter| permet de ventiler les messages envoyés par le routeur en fonction du message d'état et de renvoyer au routeur l'information de fin de tâche que le processus doit fournir dans le cas où il possède une extinction propre (mode \verb|@automute 0|). Ce principe est exposé sur la figure \ref{fig:algorithms:MP-simpleSynth-inside}, dans laquelle l'objet Max \verb|adsr~| vient notifier la fin de l'enveloppe à l'objet \textit{mp.muter}.

%------------------ Figure : simple synth ---------------------
\begin{figure}[!htbp]
	\captionsetup{format=plain}%
	\centering
	\begin{minipage}[t]{0.510\textwidth}
		\includegraphics[width=\linewidth]{gfx/04_algorithms/MP-reallySimpleSynth.png}
		\caption[mp.simpleSynth : encapsulation de la synthèse]{mp.simpleSynth : routeur et encapsulation de la synthèse.}
		\label{fig:algorithms:MP-simpleSynth}
	\end{minipage}
	\hspace{.02\linewidth}
	\begin{minipage}[t]{0.450\textwidth}
	  	\includegraphics[width=\linewidth]{gfx/04_algorithms/MP-reallySimpleSynth-inside.png}
		\caption[mp.simpleSynth : patcher de synthèse]{mp.simpleSynth : patcher de synthèse. Le route est notifié de la fin d'activité de l'envoi d'un message ``mute'' par l'objet ADSR.}
		\label{fig:algorithms:MP-simpleSynth-inside}
	\end{minipage}
\end{figure}
%------------------ Figure : simple synth ---------------------


\subsubsection{Paramètres d'un MP-block}

\noindent Les \textit{MP-blocks} répondent à une liste de paramètres contrôlant le processus de traitement par voix. Ces paramètres sont stockés par identifiant dans le routeur jusqu'à ce qu'un message d'état soit reçu. Ce message entrainera l'allocation d'une voix (si disponible et si elle n'est pas déjà active), puis l'envoi de tous les paramètres partageant cet identifiant à cette voix.\\
\indent Il est possible d'envoyer d'autres paramètres que ceux contrôlant le processus. Dans ce cas il traverseront le \textit{MP-block} inchangés, tout en restant synchrones avec d'éventuels autres paramètres générés par le processus qu'ils traversent. Dans le cas où ce processus génère de nouveaux \textit{MP-events}, ces paramètres peuvent automatiquement être ajoutés à chaque \textit{MP-event} généré, en une sorte d'héritage similaire à celui opéré par la librairie ``o.'' \cite{freed_composability_2011}, mais non-automatique et laissé à l'appréciation du développeur du \textit{MP-block}.

%-----------------------------------------------------------------------------------
\subsection{Exemples}

\noindent Les exemples proposés dans cette section présentent des cas concrets d'utilisation du système MP. Ils sont implémentés dans le logiciel Max et inclus dans les examples du package \textit{ModularPolyphony}. Ces exemples s'appuient sur l'utilisation d'une tablette \textit{multitouch}, dont les données reçues dans l'objet \verb|mp.TUIO.input| sous la forme de \textit{MP-events} correspondent au contact de chaque doigt, avec les coordonnées en X et Y de leur position sur la tablette. Le synthétiseur \verb|mp.simpleSynth| est celui présenté sur les figures \ref{fig:algorithms:MP-simpleSynth} et \ref{fig:algorithms:MP-simpleSynth-inside}.

\subsubsection{Mapping simple avec une fonction pure}

\noindent Cet exemple (figure \ref{fig:algorithms:MP-pure}) montre l'utilisation la plus simple du système MP. On contrôle ici la valeur de vélocité sur l'axe vertical tandis que l'axe horizontal contrôle la hauteur.\\
\indent Comme l'objet \verb|scale|, qui opère une mise à l'échelle entre les données de position (entre 0 et 1) et les données de hauteur et de vélocité (en valeur \gls{MIDI} 0-127), est une fonction pure\footnote{En informatique, une fonction pure, appelée également algortihme déterministe, est une fonction dont la sortie ne dépend que de la valeur d'entrée.}, il peut être utilisé pour traiter directement l'ensemble des \textit{MP-messages} lui parvenant. Ici, l'avantage d'utiliser MP est de bénéficier d'un système d'adressage similaire au \gls{MPE} sans être limité par le typage, la précision et l'espace de nommage des données.

%------------------ Figure : simple synth ---------------------
\begin{figure}[!htbp]
	\captionsetup{format=plain}%
	\centering
	\begin{minipage}[t]{0.485\textwidth}
		\includegraphics[width=\linewidth]{gfx/04_algorithms/MP-mappingPure.png}
		\caption[Exemple de patch MP : fonction pure]{Exemple de patch MP : mapping avec une fonction pure}
		\label{fig:algorithms:MP-pure}
	\end{minipage}
	\hspace{.01\linewidth}
	\begin{minipage}[t]{0.485\textwidth}
	  	\includegraphics[width=\linewidth]{gfx/04_algorithms/MP-mappingRecursive.png}
		\caption[Exemple de patch MP : fonction récursive]{Exemple de patch MP : mapping avec une fonction récursive}
		\label{fig:algorithms:MP-recursive}
	\end{minipage}
\end{figure}
%------------------ Figure : simple synth ---------------------

\subsubsection{Mapping avec une fonction récursive}

\noindent Cet exemple (figure \ref{fig:algorithms:MP-recursive}) montre le cas d'un mapping avec une fonction récursive. L'objet \verb|slide| de Max opère un filtrage logarithmique définit par l'équation :
 $$y[n] = y[n-1] + \frac{(x[n]-y[n-1])}{slide}$$ 
\noindent Dans ce cas là, il n'est plus possible d'utiliser une seule instance de l'objet \verb|slide| pour traiter l'ensemble des \textit{MP-messages}, en raison de la mémoire interne d'un état précédant propre à un \textit{MP-event} particulier. L'objet \verb|mp.slide| permet alors d'instancier plusieurs voix traitant les valeurs individuelles de \textit{pitch} en parallèle.


\subsubsection{Création dynamique de MP-events et guests}

\noindent Dans cet exemple (figure \ref{fig:algorithms:MP-mappingDivergent}), nous créons à la volée des \textit{MP-events} ``enfants'' que nous contrôlons en parallèle avec le \textit{MP-event} ``parent''.
Dans un premier temps, nous associons les coordonnées x et y d'un \textit{MP-event} aux paramètres de hauteur (pitch) et de vélocité, respectivement. Le module \verb|mp.note2chord| génère un accord à partir de la hauteur (ici en ajoutant une quinte à la hauteur d'entrée). Cet accord prend la forme de deux nouveaux \textit{MP-events} enfants qui vont venir activer chacun une voix de polyphonie de notre synthétiseur. Le module \verb|mp.note2chord| ajoute également aux paramètres des \textit{MP-events} enfants, l'ID du \textit{MP-event} parent, en tant qu'invité. Ceci va nous permettre de contrôler le paramètre de vélocité des deux notes simultanément à partir du \textit{MP-event} parent. 

\vspace{-1em}
\begin{itemize}[noitemsep]
	\item à gauche, le module  génère un ensemble de \textit{MP-events} "enfants". Le premier niveau en génère deux et le deuxième niveau en génère 3. Le résultat final est la génération d'un accord de six notes à partir d'un seul \textit{MP-event};
	\item à droite, le \textit{MP-event} venant de \textit{mp.TUIO.input} est envoyé sur deux chemins où ses valeurs sont mises à l'échelle par le module mp-scale pour définir la fréquence et l'offset d'un LFO respectivement. Le \gls{LFO} contrôlera ensuite la fréquence de coupure d'un filtre passe-bas du module de synthèse;
	\item nous avons deux occurrences d'un contrôle global avec le Master-ID : une pour le paramètre \textit{depth} du LFO et une pour la transposition du deuxième étage \textit{note2chord};
	\item en dernier lieu, nous dirigeons les messages résultants de ces deux chemins vers un système de représentation graphique. La position d'objets graphiques est assignée aux valeurs de pitch des \textit{MP-events} générés, tandis que leur couleur est associée au \textit{MP-event} parent. Ceci résulte en une représentation graphique de toutes les hauteurs en tant qu'objets dont la couleur nous dit à quelle source commune (ici le contact d'un doigt sur un écran) ils sont rattachés
\end{itemize}

%------------------ Figure : simple synth ---------------------
\begin{figure}[!htbp]
	\captionsetup{format=plain}%
	\centering
	\begin{minipage}[t]{0.38\textwidth}
		\includegraphics[width=\linewidth]{gfx/04_algorithms/MP-mappingGuest.png}
		\caption[Exemple de patch MP : guestlist]{Utilisation de la guestlist pour un mapping divergent}
		\label{fig:algorithms:MP-mappingDivergent}
	\end{minipage}
	\hspace{.01\linewidth}
	\begin{minipage}[t]{0.58\textwidth}
	  	\includegraphics[width=\linewidth]{gfx/04_algorithms/MP-mappingConvergent.png}
		\caption[Exemple de patch MP : guestlist]{Utilisation de la guestlist pour un mapping convergent}
		\label{fig:algorithms:MP-convergent}
	\end{minipage}
\end{figure}
%------------------ Figure : simple synth ---------------------


\subsubsection*{Mapping polyphonique convergent: many[guests]-to-one}
\label{sec:algorithms:many-guests-to-one}

\noindent Alors que l'exemple précédent nous montrent une utilisation de la \textit{guestlist} comme hiérarchie de subsumption, ce dernier exemple (cf. figure \ref{fig:algorithms:MP-convergent}) fournit un exemple concret de relation \textit{many-to-one}.\\
\indent Imaginons que l'on souhaite contrôler avec deux doigts un son dont l'intensité est proportionnelle à l'espacement de nos doigts. Il faut alors que deux \textit{MP-events} indépendants (correspondants à la captation de mes deux doigts) soient combinés pour que leur présence conjointe donne naissance à un nouveau \textit{MP-event} ``enfant''. C'est ce qui est réalisé dans les deux chaînes de traitement centrales sur la figure \ref{fig:algorithms:MP-convergent}, la chaîne la plus à gauche déclarant les ID des \textit{MP-events} parents dans la \textit{guestlist} du MP-event ainsi créé.\\
\indent Imaginons maintenant que l'on souhaite que la hauteur du son produit change aléatoirement à chaque fois que l'un deux nos doigts franchit une frontière virtuelle sur l'interface, en l'occurence une ligne horizontale de coordonnée $y=0,5$. Cela nécessite que les \textit{MP-events} correspondant à chacun de mes doigts puissent agir sur le son créé par le \textit{MP-event} ``enfant''. C'est l'opération réalisée par la châine de traitement la plus à droite sur la figure \ref{fig:algorithms:MP-convergent}, avec l'objet \verb|mp.past| qui détecte le franchissement de seuil pour chaque événement en parallèle (cette détection n'étant pas une fonction pure).


\subsubsection*{Contrôle d'interfaces graphiques}
\label{sec:algorithms:example-mpTUI}
\noindent Le système MP est également amplement utilisé dans la librairie MP.TUI, détaillée au chapitre suivant.


\subsection{Limitations et optimisations}

\noindent Séparer les différents processus de traitement d'un design d'interaction polyphonique a l'avantage de permettre une meilleure modularité et, par suite, une meilleure stabilité des processus mis en œuvre, qui n'auront pas à être modifiés en interne. Le choix a également été fait de ne pas s'appuyer sur une mémoire globale ou des pointeurs externes aux modules (e.g. pour sauver la guestlist), de sorte que les \textit{MP-blocks} sont réellement indépendants et autonomes et que le design d'interaction général puisse être réparti sur plusieurs applications et/ou machines en réseau.\\
\indent Cependant, cette modularité a un coût et est moins optimisée, en terme de computer et d'utilisation de la mémoire, qu'un design qui consisterait à d'inclure toutes les fonctions nécessaires dans les traitements \textit{ad-hoc}. Certaines optimisations peuvent être réalisées pour des processus ne nécessitant pas de mémoire interne (e.g. une mise à l'échelle statique), pour lesquels il n'est pas nécessaire d'allouer des voix. Cela se fait cependant au prix de certaines fonctionnalités (pas d'utilisation du \textit{master-ID} possible dans ce cas). Par ailleurs, le framework MP a entièrement été réalisé à l'aide d'objets natifs Max et pourrait sûrement être optimisé en le portant sous la forme d'objets compilés.\\
\indent Le modèle de messages asynchrones distinguant début et fin d'événement ne résoud pas directement le problème de note ``bloquées''\footnote{Ce problème surgit dans les systèmes MIDI, quand une note-off n'arrive pas.} tel qu'on le connait avec l'utilisation du \gls{MIDI}. Cependant, l'ouverture de l'espace de nommage permet l'implémentation \textit{ad-hoc} d'un système d'acquittement\footnote{tel qu'il existe dans le protocole \gls{TCP}}, ou de \textit{heartbeat}\footnote{Nom donné au signal périodique généré par un programme pour indiquer son fonctionnement normal ou pour sa synchronisation avec d'autres parties d'un système informatique.}. Également un message \textit{flush}\footnote{Equivalent de la fonction ``MIDI-panic'' généralement présente sur les système MIDI et permettant d'éteindre toutes les notes actives (quand elles sont bloquées).} est présent sur chaque module MP afin de gérer au besoin l'urgence lors d'une performance.


%%%%%%%%%%%%%%%%%%%%%%%%%%%%%%%%%%%%%%%%%
\section{Sagrada : extension de MP au DSP}
\label{sec:algorithms:sagrada}

%--------------------------------------------------------------------------------
\subsection{Motivations et contexte}

\noindent Si le développement de la librairie MP a permi d'apporter du contrôle continu à l'intérieur d'une logique basée sur des événements discrets, la démarche inverse permet d'explorer, dans l'autre sens, les passerelles expressives possibles entre le synchrone et l'asynchrone, entre le continu et le discret, le lisse et le strié\footnote{Pierre Boulez utilise ces termes pour évoquer la dialectique entre le continu et le discontinu et définir la nature des ``espaces'' (dont le temps, en particulier) en musique dans \cite{boulez_penser_1987}.}. En particulier, la synthèse granulaire est basée sur le découpage du continuum sonore en grains de son et les premières œuvres utilisant ce type de synthèse\footnote{Par exemple, ConcretPH de Iannis Xénakis en 1958, ou Kontakte de Karlheinz Stockhausen en 1960, dont le passage où la fréquence des grains, d'abord perçue comme une hauteur tonale, décroit jusqu'à être perçue comme un rythme est un exemple concret notre propos.} sont très nettement marquée par cette recherche de passage entre le continu et le discret.\\
\indent Ma pratique personnelle m'avait amené à utiliser l'environnement GMU\footnote{Environnement pour la synthèse granulaire dans Max/MSP, développé par Charles Bascou et Laurent Pottier au \gls{GMEM}, cf. \cite{bascou_gmu_2005}} en raison des possibilités qu'il offre pour le contrôle du déclenchement des grains, réalisé par un signal \gls{DSP}, ce qui permet des fréquences de grains très élevées (potentiellement, jusqu'à la moitié de la fréquence d'échantillonage) tout en conservant une précision temporelle à l'échantillon près.\\
\indent La question était donc : comment relier ces deux domaines polyphoniques s'appuyant sur des supports hétérogènes ? La possibilité de pouvoir contrôler les grains individuellement dans GMU, bien que très fine, ne permettait pas d'accéder à leur modulation en dehors de l'interface fournie par l'objet Max. Les grains y sortent mixés sur deux canaux\footnote{Une version 8-canaux existe aussi, mais destinée plutôt pour l'octophonie de haut-parleurs, plutôt que pour le traitement individuel des grains.} et l'ajout d'un filtre en aval est nécessairement appliqué à l'ensemble des grains.\\
\indent Suivant la même logique de modularisation suivie dans le développement de MP, Sagrada propose un ensemble de modules indépendants, traitant les grains de manière indépendante. Le passage de paramètre est réalisé à partir d'une horloge qui envoit des impulsions sur un signal synchrone de Max.

%--------------------------------------------------------------------------------
\subsection{Implémentation}

\subsubsection{Types de modules}
\noindent Sagrada est basé sur trois types de modules, dont le fonctionnement est détaillé par la suite :
\vspace{-1em}
\begin{itemize}[noitemsep]
	\item \textbf{une horloge}, gérant le déclenchement des grains;
	\item \textbf{des modules de traitement}, contrôlés par l'horloge;
	\item \textbf{des modules de gestion de flux}, permettant de grouper les ticks d'horloges en flux indépendants;
\end{itemize}


\subsubsection{Interconnexion des modules}

\noindent Un patch Sagrada (figure \ref{fig:algorithms:MP-ExamplePatch}) fait intervernir une série de modules reliés par une variable définissant un \textit{contexte}. Ce contexte permet de :
\vspace{-1em}
\begin{itemize}[noitemsep]
	\item \textbf{redéfinir la polyphonie} de l'ensemble des modules, en la déclarant seulement auprès de l'horloge ;
	\item \textbf{synchroniser} l'ensemble des modules à une horloge commune;
	\item \textbf{éviter les conflit d'adressage} entre plusieurs contextes Sagrada;
\end{itemize}

\noindent Les modules de Sagrada sont interconnectés par la spécification du nom de leurs entrées et sorties sous forme d'arguments du module (à l'instanciation), ou de messages Max pour changer cette interconnexion à la volée.

%------------------ Figure : Sagrada-trigger ---------------------
\begin{figure}[!htbp]
	\captionsetup{format=plain}
	\includegraphics[width=\textwidth]{gfx/04_algorithms/Sagrada-examplePatch.png}
	\caption[Sagrada : exemple de patch]{Example de patch Sagrada montrant les différents modules, interconnectés par la variables de contexte et d'entrées/sorties}
	\label{fig:algorithms:MP-ExamplePatch}
\end{figure}
%------------------ Figure : Sagrada-trigger ---------------------

\subsubsection{Horloge et assignation de voix}

\noindent L'horloge de Sagrada est implémentée dans l'abstraction Max \verb|sagrada.trigger~|. Son principe de fonctionnement consiste à envoyer un \textit{tick} de déclenchement de grain à une nouvelle voix de polyphonie à chaque fois qu'elle reçoit une impulsion en entrée. Ces impulsions peuvent donc être déclenchées soit de manière asynchrone (e.g. en transformant un message \gls{MIDI} en une une impulsion grâce à l'objet Max \verb|click~|), soit de manière synchrone (e.g. en utilisant un train d'impulsion controlé en fréquence). L'incrément fonctionne selon l'équation suivante, dans laquelle $x$ représente le signal de délenchement, $y$ la voix de polyphonie cible et $N$ la polyphonie maximale autorisée :
$$ y[n] = \big(x[n] + y[n - 1]\big) \mod N $$
\noindent Dans cette version simple, l'assignation des voix se fait donc de manière cyclique, en ``usurpant'' au besoin la voix à un grain encore actif. Un mécanisme de non-usurpation est également proposé sur le mode du \textit{round-robin}\footnote{Le \textit{round-robin} est un algorithme d'ordonnancement consistant à attribuer une opération à un processus faisant partie d'une file d'attente, en choisissant le premier disponible dans l'ordre de la file d'attente.}. Il s'appuie sur l'utilisation d'un tableau, contenant l'incrément nécessaire pour passer de la voix prévue par l'algorithme simple à la prochaine voix disponible. Si la voix prévue est libre, cet incrément est nul; si l'incrément est égal à la valeur de polyphonie, on n'envoit pas le grain.
$$ z[n] = \Big(y[n] + jump\big[y[n]\big]\Big)\mod N $$
\noindent L'implémentation de ce mécanisme d'horlge synchrone a été réalisé dans l'environnement \verb|gen~| de Max, afin de pouvoir travailler à l'échantillon (cf. figure \ref{fig:algorithms:MP-TriggerClock}). Le tableau contenant l'information de disponibilité des voix de polyphonie est un module indépendant, pouvant être placé à n'importe quel endroit dans la chaîne de traitement des grains. (cf. objet \verb|sagrada.busy~| sur la figure \ref{fig:algorithms:MP-ExamplePatch}) Par exemple, si un grain subit une opération de filtrage résonnant, sa durée finale sera plus longue que le grain original. On pourra donc, au besoin, contrôler l'état occupé d'une voix par les grains issus de ce filtre résonnant.

%------------------ Figure : Sagrada-trigger ---------------------
\begin{figure}[!htbp]
	\captionsetup{format=plain}
	\includegraphics[width=\textwidth]{gfx/04_algorithms/Sagrada-TriggerClock.png}
	\caption[Sagrada : horloge synchrone et assignation]{L'objet sagrada.trigger (à gauche) et l'implémentation des horloges dans \textit{gen}, avec usurpation de voix (en haut à droite), et sans usurpation de voix (en bas à droite).}
	\label{fig:algorithms:MP-TriggerClock}
\end{figure}
%------------------ Figure : Sagrada-trigger ---------------------

\subsubsection{Gestion des flux de grains}

\noindent Si l'on considère l'ensemble des grains générés comme un flux, se posent alors deux questions concernant leur gestion :
\vspace{-1em}
\begin{itemize}[noitemsep]
	\item \textbf{la gestion d'un flux dans sa durée} : concètement, si je considère un flux de grain comme un macro-objet temporel ayant un début et une fin, une gestion particulière des coupures de début et de fin est souhaitable pour adapter la nature du son à un niveau micro-temporel à la forme macro-temporelle (par exemple en choisissant un grain particulier pour l'attaque);
	\item \textbf{la gestion de la multiplicité des flux} : concrètement, et pour faire le lien avec les développements présentés dans MP, si un doigt peut contrôler un flux de grains, un autre doigt doit pouvoir contrôler un autre flux de grains (et ainsi de suite) sans qu'il y ait de confusion dans le contrôle de chacun de ces flux.
\end{itemize}

\noindent La gestion de flux de grains est prise en charge par l'objet \verb|sagrada.multilayer~| (figure \ref{fig:algorithms:MP-multilayers}), qui permet de multiplexer plusieurs horloges en leur attribuant un index. Cet index de flux permet le décompte des grains par flux et en particulier, d'utiliser cet index pour gérer des grains individuels tels que le premier et le dernier grain d'un flux. 
\indent Un mécanisme de feedback interne permet d'éviter le conflit potentiel entre plusieurs \textit{ticks} d'horloge qui seraient déclenchés exactement au même instant (i.e. sur le même échantillon d'un signal). Dans ce cas précis, la priorité est donnée au flux de plus haut niveau\footnote{c'est à dire concrètement à celle gérée dans la voix de polyphonie la plus élevée de l'objet poly\textasciitilde{} gérant les différent flux.} Toutfois, ce cas de figure étant peu probable statistiquement, le multiplexage permet de n'utiliser qu'un seul et unique signal \gls{MSP} pour la gestion de l'ensemble des flux et d'économiser ainsi en ressources \gls{CPU}.

%------------------ Figure : Sagrada-multilayers ---------------------
\begin{figure}[!htbp]
	\captionsetup{format=plain}
	\includegraphics[width=\textwidth]{gfx/04_algorithms/Sagrada-multilayers.png}
	\caption[Sagrada : gestion de flux de grains]{Gestion de flux de grains par multiplexage d'horloges.}
	\label{fig:algorithms:MP-multilayers}
\end{figure}
%------------------ Figure : Sagrada-multilayers ---------------------

%--------------------------------------------------------------
\subsection{Performances comparée}

\noindent L'avantage principal de Sagrada est de pouvoir composer une synthèse granulaire \textit{ad hoc} en traitant chaque grain de la manière que l'on souhaite. En adoptant un système de déclenchement des grains via un signal \gls{MSP}, Sagrada bénéficie des même avantages que le système \gls{GMU} (présentés dans \cite{bascou_gmu_2005}), en particulier la possibilité de ré-utiliser les systèmes de déclenchement stochastiques de grains fournis dans \gls{GMU}\footnote{Une réimplémentation partielle en a été faite dans la LAM-lib, notamment sa fonction de probabilité à densité [rand\_dist\_list\textasciitilde{}] dans l'objet [LAM.pdf\textasciitilde{}]}. Un avantage d'utiliser des impulsions par rapport au \textit{zéro-crossing} utilisé dans GMU est de diminuer de moitié (un échantillon au lieu de deux) l'intervalle entre deux grains successifs.\\
\indent Cependant, cette modularité implémentée dans le langage Max a un coût. A nombre de voix actives égales, le système \gls{GMU} consomme trois fois moins de ressources \gls{CPU} que Sagrada.


gen\textasciitilde{ } OLA, granularized, ftm, mc.*


\section*{extra material}

la production de hauteur dans les instruments acoustique est systématiquement liée à un phénomène de résonance, c'est à dire un phénomène de bouclage, de feedback. 

La sélection de différentes hauteurs peut ensuite être obtenue 
- en jouant sur une multitude d'éléments accordés différemment (les cordes d'une harpe, les lames
d'un marimba, etc.) ;
- en modifiant les caractéristiques d'un élément résonnant, le plus souvent sa longueur (tube des
instruments à vent, corde d'un violoncelle) ;
- en sélectionnant des harmoniques précis d'un son riche (didgeridoo, chant diphonique).


Dans les DMIs la production de hauteur est essentiellement possible de deux manières: 
- par la lecture, éventuellement en boucle, d'une table d'onde
- par le délai de réinjection d'un filtre résonnant (synthèse soustractive, Karplus, etc.)
- par un calcul mathématique impliquant des fonctions périodiques (telles que les sinus dans la FFT)

L'acoustique physique présente naturellement des non-linéarité qui contribuent à la richesse du son et l'identité de son timbre. L'électronique analogique permet également d'obtenir —sur l'espace restreint d'un signal mono-dimensionnel par rapport à l'acoustique physique généralement bi- ou tri-dimensionnelle— des systèmes résonants non-linéaire et stables, via l'utilisation de composants électronique passifs.

Les choses sont plus compliquées pour le son numérique, s'il est possible d'obtenir des systèmes résonant stables (en maintenant les pôles dans le cercle unitaire du le plan complexe), la garantie de stabilité devient très complexe à évaluer dès que l'on introduit de la non-linéarité.
Malgré de récentes avancées dans ce domaine (cf. travaux de Thomas Hélie sur les systèmes Hamiltoniens à ports)

