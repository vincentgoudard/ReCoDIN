% !TEX root = ../thesis-example.tex
%
\chapter{Geste instumental}
\label{ch:transparency}

\cleanchapterquote{Any sufficiently advanced technology is indistinguishable from magic.}{Arthur C. Clarke}{Profiles of the Future (revised edition, 1973)}



\section{Revue des théories sur le geste instrumental}

Auteurs : Cadoz, Godoy, Leman, Wanderley

Les instruments de musique numériques (DIM) présentent non seulement un découplage énergétique entre les gestes du musicien et le résultat sonore, mais également une partie computationnelle couplée à une mémoire qui permet un re-programmation complète de l'interaction, rendant leur fonctionnement cryptique. Cet aspect peut se révéler être aussi bien un inconvénient, dans la mesure où il prive le public d’une lecture possible d’une performance musicale. Cela peut cependant aussi s’avérer un avantage, dans la mesure où la performance musicale comporte une part scénographique dans laquelle l’illusion a toute sa place.
Je présente ici quelques aspects liés à l’interaction avec de tels instruments dans le contexte d’une performance audio-visuelle réalisée avec la plasticienne Gladys Brégeon. Cette performance implique différents types d’interfaces, et pour chacune de ces interfaces, des interactions diverses et hybrides qui joue en partie sur la lecture souhaitée ou fantasmée de ce qui s’opère sur scène.

Pendant longtemps (TODO : combien?), les instrumentistes utilisant des DIMs ont été caché derrière des machines, à la position souvent occupé par l'ingénieur du son, ne laissant rien voir ou si peu de ce qu'il faisaient vraiment. 
Expliquer en quoi le découplage énergétique, qui a amené à "un sens de la discontinuité avec la tradition, aliénation et manque de compréhension par le public en ce qui concerne ce que l'instrumentiste ou l'instrument fait en réalité". (T Magnusson, sonic wrting pp. 61) a amené à une contre-réaction faisant passer la lisibilité de la relation geste/son comme un critère pertinent de design instrumental.

La subversion peut intervenir à différents niveaux. Au niveau de la composition, l'écriture musicale permet des modulations qui déjouent les attentes de l'auditeur. (e.g. Pink Floyd, breathe transition). Elle peut également se situer au niveau du jeu, en usant de procédés comme des gestes qui contredisent ce qu'on entend et vont l'amplifier. Gyorgy Kurtag Jr. geste violent pour jouer une nuance pianissimo.

Exemples comparés de Applebaum Aphasia et Vincent Carinola/Jean Geoffroy "Rhizome".
BBC Classic Album: "Pink Floyd - The Dark Side of the Moon"

Dissonance cognitive.

Synchrèse de Michel Chion.

\section{Les limites d'une analyse des instruments en terme de HCI}
\label{sec:transparency:limitesHCI}

Plusieurs articles des NIME analyse les DMI comme HCI : 
Louange de la transparence : \cite{fels_mapping_2002}

Les instruments de musique numériques (DIM) ont un fonctionnement non-causal qui rend leur fonctionnement cryptique. Cet aspect peut se révéler être aussi bien un inconvénient, dans la mesure où il prive le public d’une lecture possible d’une performance musicale. Cela peut cependant aussi s’avérer un avantage, dans la mesure où la performance musicale comporte une part de scénographie dans laquelle l’illusion a toute sa place.
Le jeu musical joue en partie sur l’attente de l’audience (récompensée ou non) sur la base de règles d'harmonies, d’idiomes (e.g. cadences, résolutions, cycles rythmiques), de citations (e.g. via le sampling), etc..
Affordance des instruments ne peut être réduite aux objectifs d’affordance des HCI.


Kurtag Jr. Hangsimotato (video)
Jean Haury Meta-Piano
Applebaum Aphasia

gestes incongruent (Musical gestures, Godoy, p.48)


\subsection{Les musiciens ne sont \emph{pas} des utilisateurs d'instruments}
\subsection{La scène et le laboratoire}


Passé d'un rapport à la musique enregistrée du disque qui cherche une fidélité par rapport à l'enregistrement à la performance qui cherche à être fidèle au disque. \cite{??} 
Large développement d'outils pour la gestion "offline" de la musique (déplacement, copié/collé,etc) et de l'ergonomie de ces outils.
Hybridation des instruments entre du controle instrumental direct ("traditionnel") et des techniques issues de la production musicale offline.


%%%%%%%%%%%%%%%%%%%%%%%%%%%%%%%%%%

\section{Espace du geste musical}
La musique a longtemps été considéré comme étant faite d'un sous-ensemble de sons, les sons harmonieux, voire harmonique, avant qu'au XXème siècle, les bruits n'y fassent leur place avec les avant-gardes, futuristes. 

La musique n'est donc pas faite que de sons, au sens acoustique du terme, mais également (avant tout?) de la perception des sons, qui implique des processus de cognition, des références socio-culturellles, et une sensibilité, une mémoire propre à chacun. 
Ainsi, l'espace de la musique ne se présente non pas comme un sous-ensemble de l'espace des sons, mais probablement comme un sur-espace comprenant à la fois les sons acoustiques mais également tous les liens qu'ils tissent avec notre mémoire.

L'art musical consiste ainsi à faire entendre des aspects de la musique qui ne sont pas nécessairement présents dans le son, à faire surgir des espaces qui ne peuvent se déployer que dans notre imaginaire, en faisant écho à la trace latente que les sons et la musique ont déjà imprimée en nous.

Là où la présence du musicien sur scène remplissait une nécessité acoustique pour l’écoute, la musique sur support, ou produite par des machines, déplace ce besoin au profit d’une autre fonction, à la fois de compréhension des gestes du musicien (mais est-ce là un jeu de dupes?) et d’un spectacle de l’ordre du funambulisme; le musicien prend des risques [celui de se tromper dans le cas de l’interprétation d’une partition] et la mise en question du corps, réagir au contexte (lieu et au public, ainsi qu’aux éventuels autre musiciens) d’une manière vivante.

L’écoute nous plonge dans des flux sonores, et notre tendance à projetter des causes à ces sons (cf. gestalt) nous emmène sur les lieux — toujours en partie étrangers — de la production de ces flux. sitar indien, crissement de pneu, explosion, acoustique sous-marine ou ambiance de salle de café.
Le musicien crée des passerelles et des agencements entre ces zones liminales.


Illustration : 

%%%%%%%%%%%%%%%%%%%%%%%%%%%%%%%%%%


\section{Subversion sonore, subversion gestuelle}

Bien que ces catégorisations du gestes décrivent adéquatement différents aspects du geste instrumental sur des instruments acoustiques, il semble que le geste musical intègre une aspect subversif souvent négligé.

En particulier, dans le cas des \gls{DMI}, la relation entre le geste et le son est totalement sujette au design de ce que l'on nomme communément le \gls{mapping} et la part de subversion devient partie intégrante du design de ce mapping. 

Il semble dès lors que l'on peut envisager d'autres types de relation entre le geste et le son, afin de tenter de décrire les différents rapports qu'ils entretiennent selon les situations.


\quote{The limitations of digital acoustics depend upon the differential capacities of perception rather than upon the constraints of mechanics. Yet our auditory perception is geared to a world of mechanically-produced sounds, and mechanics should not be given a cavalier dismissal, as the work of Gibson and Cadoz has suggested : the specifics of mechanical vibrations shed light on the the perceptual organization in the hearing process.}

\textbf{Proposition :
- readable gesture to sound relations
- confusing gesture to sound relations}

Gestes emphatique = en phase avec le mouvement interne du son
Geste apophatiques = en opposition de phase avec le son
Geste unrelated

\subsection{Continuités artificielles}

- Contrepoint : relier la mélodie à l'harmonie 
- Bach et le tempérament = relier les différents modes, via la modulation. 
- les doigtés alternatifs, sur le plan gestuel, permettent de sacrifier la justesse de la note, pour établir une continuité gestuelle fluide
- La musique sérielle : relier la gamme tempérée au spectre en ordonnant  
- Stravinsky, Russolo : relier l’harmonie et le bruit 
- jouer un pattern connu ("qu'on a dans les doigts") tout en substituant les notes permet de jouer de manière fluide une mélodie inhabituelle. => numérique
- Cage, Murray Schaeffer : relier le déterminisme et le hasard, la musique et l’environnement 

=> voir Théories de la composition musicale au XXe siècle 
Conjointement à ces explorations compositionnelles se sont développées des techniques et des technologies permettant d’appréhender ces nouveaux espaces. Que cela soit des procédés d’écriture ou des instruments reflétant ces méthodes et modèles.



\subsection{Dans les instruments traditionnels}

\subsection{Dans les instruments numériques}


Les DMI ont souvent été analysés en tant qu'interface homme-machine, et les conférence académique qui leur sont consacrés reflètent une culture dans lequel l'interaction s'exprime via un cahier des charges préalablement identifié: une interface homme-machine est utilisée dans le cas d'une tâche précise et sa qualité (ergonomie, précision, etc.) peut être mesurée de manière quantifiée.
Dans le cas des instruments de musique cependant, cette tâche est plus complexe, car les enjeux de la création musicale dépassent par essence tout objectif identifié et mesurable au préalable. Par ailleurs, les DMI sont destinés plusieurs types "d'utilisateurs" ayant un rôle différent : le musicien qui joue de l'instrument, mais également le public, qui bien qu'il ne joue pas de l'instrument est amené à en observer la performance.

Low entry fee, high ceiling.

La performance musicale est un "jeu" qui comporte une part de duplicité. Le public d'un concert est toujours le sujet d'une illusion. 

Un des biais de la littérature sur l’affordance des instruments de musique numérique est qu’elle s’inspire souvent des objectifs de l’affordance des HCI en général, avec l’idée que l’instrument doit être compréhensible pour les “autres” utilisateurs potentiels que l’auteur de l’instrument. Pourtant, nombre d’instruments présentés dans la communauté NIME ne sont joués que par leurs auteurs (cf. [1], [2], [3]) et si le fait de vouloir transmettre son instrument aux autres est louable, il n’est pas gage de qualité en ce qui concerne la création qui sera faite avec cet instrument.     


L'art musical procède en partie de la magie et de l'illusion perceptive. Le musicien nous fait entendre des continuités (e.g. une mélodie) là où l'acoustique fait apparaitre une série discrète (e.g. des notes de piano), ou inversement des fissions (e.g. deux voix indépendantes) là où est jouée une série temporelle de notes sur un même instrument. (=> plutôt que des exemple entre parenthèse, mettre une figure illustrant fission e.g. Bach's Violin Partita No. 3, BWV 1006.)

Cette question du jeu entre le continu et le discret dépasse le seul cadre de la musique mais semble trouver dans cet art de nombreuses espaces d'expression.

Les théories de la perception, en particulier du Gestalt, viennent en partie expliquer les mécanismes qui pousse notre perception à créer des continuités où il n'en existe pas physiquement et inversement à catégoriser des événements selon certaines distances perceptives qui ne sont pas nécessairement en lien avec l'unité de source de production du son.

Si donc on analyse le geste musical, il faut nécessairement prendre en compte sa dimension subversive en ce qu'elle se traduit, particulièrement dans le cas des DMIs et des productions musicales impliquant l'électronique en général, dans le design des instruments et outils qui servent à la créer.

\cite{bin_show_2018}

Carte et guide , frettage adaptatif (cite pitch)

\section{Morpho-dynamisme dans les DMI}


\clearpage

\section{Co-performance}
Ainsi, la nature dynamique et générative des DMIs déplace l'agentivité de l'interaction instrumentale, qui ne saurait être réduite à une relation unidirectionnelle, où le retour haptique et/ou vibratoire n'aurait qu'une dimension épistémique et informative pour le musicien.
L'instrument peut se retrouver en position de mener le jeu et imposer sa cadence à l'instrumentiste. La performance musicale avec un DMI est donc une co-performance où la distribution du contrôle de la synthèse et de la gestuelle qui la provoque, ou en découle, peut se distribuer de manier polymorphe. Une partie de la dynamique de jeu peut être prise en charge par la machine et une autre partie par l'instrumentiste dans une relation qui peut parfois s'approcher d'un duo.

 La performance musicale avec un DMI est donc une co-performance où le contrôle de la synthèse est distribué entre la machine et le musicien. 
 Les gestes du musiciens ne sont ainsi par nécessairement des gestes de contrôle (puisqu'il peuvent être "libérés" de cette fonction là) mais peuvent \textit{découler} de la synthèse opérée par la machine. Les gestes du musicien peuvent alors entretenir des relation de nature différentes :
\begin{itemize}
\setlength\itemsep{-0.5em}
\item une relation d'accompagnement, cohérente ou dissonante;
\item une relation de réaction épidermique;
\item une relation neutre.
\end{itemize}
 
 
\section{Conclusion}
\label{sec:transparency:conclusion}
=> Comment ces aspects influencent le design de l’instrument ?


