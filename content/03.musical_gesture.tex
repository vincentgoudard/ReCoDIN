% !TEX root = ../thesis-example.tex
%
\chapter{L'instrument et le geste}
\label{ch:gesture}

\cleanchapterquote{La percussion de ce pseudo-gong est illusoire :\\
rien ne tape sur rien dans l’ordinateur.\\
Schumann qualifiait le legato au piano\\
de ``trompe-l’oreille'' : la musique est aussi un art\\
du mirage, de l’illusion.}{Jean-Claude Risset}{Discours invité aux JIM, 2010.\\\cite{risset_propos_2010}}
\index[people]{risset@Risset, Jean-Claude}
%L'expression musicale a été contrainte (et non ``prisonnière'', car la contrainte peut être fertile) par la relation causale entre le geste d'excitation et le son produit par l'instrument dans les lutheries acoustiques.

\vspace*{\fill}

\noindent Poursuivant des évolutions organologiques latentes, telles que présentées dans la section \ref{sec:ephemerality:landscape}, les \glspl{DMI} finissent d'opérer un découplage énergétique, une ``dislocation du contrôle\footnote{Eduardo Miranda utilise ce terme de \textit{control dislocation} dans \cite{miranda_new_2006}.}'', rompant avec plus de 40~000 ans de tradition musicale \cite{conard_new_2009} et avec l'unité de temps, de lieu et d'action, érigée en règle par le théâtre classique, mais qui reflète également la relation qui existait entre l'auditeur et le phénomène sonore, comme le rappelle Hugues Genevois dans \cite{cance_what_2012}.\\
\indent L'introduction de la mémoire et de la computation numérique permet une re-programmation complète de l'interaction, rendant leur fonctionnement à la fois complexe et cryptique. Cet aspect peut être vu comme un inconvénient, dans la mesure où il prive le public d’une lecture possible de la performance musicale. Il peut cependant s’avérer être un avantage, si l'on considère que la performance musicale comporte une part scénographique dans laquelle l’illusion a toute sa place. \\
\indent Nous allons voir dans ce chapitre comment le geste instrumental s'en trouve affecté, par un examen critique des catégories gestuelles déjà proposées, la proposition de nouvelles catégories prenant en compte l'aspect subversif de l'art musical et les conséquences que nous pouvons en tirer en termes de conceptions des \glspl{DMI}.

\clearpage

\section{Introduction : L'étude du geste en musique}

\noindent Si l'étude du rythme et plus encore, des hauteurs, a été particulièrement importante dans la culture classique occidentale, l'étude du geste a longtemps été négligée voire méprisée, comme le rapporte Jean-Marc Warszawski\footnote{Texte publié d'après la conférence ``La musique et le geste'' en octobre 2012, disponible en ligne : \url{http://www.musicologie.org/18/la_musique_et_le_geste.html}.}: \iquote{La tradition savante occidentale, grâce à l'écriture, dématérialise le geste créatif musicien, et impose une ligne de démarcation entre musique écrite et non écrite, en quelque sorte une frontière entre le primitif et le civilisé.}\\
\indent Peut-être faut-il également y voir une autre raison : le geste ne se laisse pas aussi facilement mesurer, encore moins définir, que la hauteur. Si cette dernière peut se  réduire en première approximation à une grandeur physique mesurable --~sa fréquence fondamentale\footnote{La perception de la hauteur est évidemment plus complexe que l'évaluation de la fondamentale et un des sujets d'étude de la psycho-acoustique, voir notamment les travaux de Michèle Castellengo \cite{castellengo_ecoute_2015}. L'écriture musicale classique s'est toutefois largement construite sur cette réduction, que Robert Francès nomme ``abstraction notale'' \cite{frances_perception_1984}.}~-- le geste ne se laisse pas aussi facilement simplifier. Il entraîne depuis sa production jusqu'à sa réception toute la complexité du vivant : sa multidimensionalité, son instabilité, sa relativité, sa combinatoire et l'ensemble des aspects culturels qui lui confèrent ou non une signification dans un contexte donné.\\
\indent L'étude du geste se développe dans le courant du \siecle{19}~siècle, sous l'effet de la révolution industrielle, de l'analyse mécanique du mouvement et la création de conservatoires qui développent des techniques d'apprentissage où le geste est pris en compte\footnote{Voir en particulier la collection rassemblée sur le site de la Bibliothèque Nationale de France: \url{https://gallica.bnf.fr/html/und/partitions/oeuvres-theoriques-et-pedagogiques}. L'intérêt pour le geste à cette période de développement industriel donne lieu par ailleurs à l'invention d'étonnants systèmes mécaniques servant à guider, contraindre et fortifier les gestes, en particulier pour le piano, tels que le ``chiroplaste'' ou le ``dactylion'', mais qui s'avèrent être davantage des instruments de torture, endommageant parfois les mains de manière irréversible.}. L'étude du geste a progressivement gagné en importance, d'une part avec l'émergence de l'anthropologie au début du \siecle{20}~siècle, et d'autre part suite à l'explosion des technologies de télécommunication, lorsque sa compréhension et sa modélisation sont devenues nécessaires pour le développement des \gls{IHM} dans la seconde moitié du siècle, jusqu'à devenir un domaine d'étude à part entière, les \textit{gesture studies}, soutenu notamment par l'\gls{ISGS} créée en 2002.\\
\indent Dans le domaine de la musique, c'est l'arrivée du ``temps-réel'' et l'émergence des \glspl{DMI} dans les années 1980, qui entraînent la parution croissante d'articles traitant de la question du geste instrumental. Au tournant du siècle (du millénaire), le geste devient un objet d'étude majeur dans le domaine de l'informatique musicale, ce qui se traduit notamment par 
des projets de recherche interdisciplinaires\footnote{Comme le projet européen ConGAS en 2003 (``\textit{Gesture Controlled Audio Systems}'', COST Action 287), le projet ANR Gemme en 2012 (``Geste musical : modèles et expériences''), dont un carnet de recherche en ligne est disponible : \url{https://geste.hypotheses.org/gemme}, ou la chaire thématique pluridisciplinaire GeAcMus (``Geste - Acoustique - Musique''), créée en 2015 à Sorbonne Université}, la parution d'ouvrages collectifs dédiés\footnote{Voir en particulier : \cite{genevois_les_1999}, \cite{wanderley_trends_2000} et \cite{godoy_musical_2010}}, ainsi par que l'apparition de la conférence \gls{NIME} en 2001, qui lui accorde une place importante.\\


%\indent Plusieurs projets de recherche interdisciplinaires ont également été menés, comme le projet ConGAS\footnote{``Gesture Controlled Audio Systems'', projet financé entre 2003 et 2007 dans le cadre de la Coopération Européenne en Sciences et Technologies (COST Action 287)}, ou plus récemment le projet Gemme\footnote{``Geste musical : modèles et expériences'', projet de recherche financé par l'ANR de 2012 à 2016, dont un carnet de recherche en ligne est disponible : \url{https://geste.hypotheses.org/gemme}} et la chaire thématique pluridisciplinaire GeAcMus\footnote{La chaire ``Geste - Acoustique - Musique'' a été créée en 2015 à Sorbonne Université \url{http://www.sorbonne-universites.fr/actions/recherche/chaires-thematiques/geacmus.html}} créée à Sorbonne-Université.

%\extra{Le terme Gesture est utilisé dans 62\% des articles publiés à NIME (cf.Jensenius paper : To Gesture or Not? An Analysis of Terminology in NIME Proceedings 2001–2013) <= update this}

%%%%%%%%%%%%%%%%%%%%%%%%%%%%%%%%%%%%%%%%%%%%%%%%%%%%%%%%
\section{Geste instrumental et geste musical}

\subsection{La musique et ses instruments}

\noindent La définition du geste musical pose le double problème de définir ce qu'on entend par ``geste'' et par ``musique''. Pour ce qui est de la musique, j'adopterai ici la définition proposée par Christopher Small, non pas du \textit{nom} ``musique'', mais du \textit{verbe} qu'il invente : ``musiquer'' \cite{small_musicking:_1998}:
\iquote{Musiquer, c'est participer, à quelque titre que ce soit, à une performance musicale, que ce soit en jouant, en écoutant, en répétant ou en pratiquant, en fournissant du matériel pour la performance (ce qu'on appelle ``composer''), ou en dansant. Nous pourrions même parfois en étendre le sens à ce que font la personne qui prend les billets à l'entrée, les gros bras qui déplacent le piano et les tambours, les roadies qui installent les instruments et font les balances ou les personnes qui nettoient après que tout le monde soit parti. Ils contribuent tous, eux aussi, à la nature de l'événement qu'est une performance musicale.}\\
\indent Si j'adopte ce néologisme, ce n'est pas pour l'exhaustivité que semble conférer une définition aussi large, mais essentiellement pour deux raisons: la première est le choix d'utiliser un verbe plutôt qu'un nom, c'est-à-dire d'identifier une pratique plutôt qu'un objet, ce qui dans le cas de la performance musicale semble mieux adapté. La deuxième raison est liée au contexte technico-culturel dans lequel ces pratiques s'inscrivent: la reconfiguration des modes de production et de réception de la musique\footnote{Voir sur ce sujet l'ouvrage de Paul Théberge \cite{theberge_any_1997}} qu'ont entraînées les évolutions technologiques du \siecle{20}~siècle ont rendu poreuses les frontières entre les différentes pratiques liées à la création musicale. Je restreindrai toutefois le champ de cette définition aux pratiques qui gravitent directement autour de l'instrument de musique, tel que présenté dans le chapitre précédent.\\
\indent Partant de cette définition et reprenant une proposition d'Hugues Genevois \cite{genevois_geste_1999}, on peut distinguer plusieurs phases au cours desquelles se manifeste un geste musical :
%---------------------
\vspace{-1em}
\begin{itemize}[noitemsep]
\item \textbf{la composition} : la production de structures musicales hors-temps de leur rendu sonore;
\item \textbf{la lutherie} : la réalisation de l'instrument et sa préparation pour le jeu;
\item \textbf{la performance} : le jeu instrumental qui produit, modifie, mélange, dans le temps même de l'écoute, la matière sonore, les gestes de l'instrumentiste, du chef, de l'ingénieur du son;
\item \textbf{l'écoute} : qui construit l'intelligibilité de notre environnement sonore et se mobilise sans répit pour en garantir la cohérence;
\item \textbf{la pédagogie} : durant laquelle la complexité du geste musical se transmet progressivement, en étant guidée, soutenue, encouragée, critiquée, en se pratiquant parfois sous la forme d'exercices idiomatiques et systématiques (e.g. faire ses gammes), et à travers des allers-retours entre performance et écoute.
\end{itemize}
%---------------------
\noindent Le concept de geste a ainsi été utilisé pour décrire différents aspects de la relation qu'entretiennent le mouvement, l'instrument, l'intention, la morphologie et la signification du son ou encore les formes d'écritures compositionnelles. L'instrument de musique se trouvant à la croisée de ces différents chemins, la notion de ``geste instrumental'' a souvent été liée voire confondue avec celle de ``geste musical''. Dans la pratique, ces différents gestes se superposent souvent, mais pourront se traduire par différentes formes de configurations instrumentales, adaptées aux spécificités de chaque situation.

\subsection{Aspects du geste}

\noindent Dans sa définition générale, le Littré, le Larousse ou le dictionnaire de l'Académie Française s'accordent à le définir comme \iquote{un mouvement du corps, principalement de la main, des bras, de la tête, porteur de sens ou non} (Larousse). Plusieurs aspects s'en dégagent selon l'angle d'étude adopté que nous allons détailler ci-après.

\subsubsection{Aspects phénoménologiques}

\noindent Un premier aspect du geste est sa nature de mouvement, son déploiement temporel, dont les qualités spatiales, cinématiques, cinétiques, synchroniques, fréquentielles, balistiques font écho à la mémoire de tous les mouvements dont nous faisons l'expérience: battements, ondulations, chutes, éclosions, envols, éruptions, roulements, contractions, extensions, sursauts... Le geste possède une force expressive, en dehors de toute interprétation sémiotique, génératrice de sensations dont la logique ne peut être décrite que par la métaphore\footnote{C'est en particulier l'entreprise de Gaston Bachelard dans l'ensemble de ses œuvres constituant une ``phénoménologie de l'imagination'', dont par exemple ``L'air et les Songes, Essai sur l'imagination du mouvement'' \cite{bachelard_air_1943}. C'est également le projet, mené de manière plus systématique, du \gls{MIM} dans l'élaboration des \gls{UST} \cite{delalande_les_1996}.}.\\
\indent Ce mouvement est l'expression même du vivant et il reflète à la fois la relation du corps à son environnement externe (force de gravité et cinétique, contournement d'obstacles, frictions sur les matériaux, contraintes formelles chorégraphiques...) et l'expression des mouvements internes du corps (vigueur ou mollesse, souple ou crispé, émotion se traduisant par des gestes tels que haussement d'épaule, sursaut de surprise, grimaces diverses, etc.). La maîtrise de ces mouvements est un aspect fondamental de la danse qui, comme la musique le fait avec le son, met en œuvre le corps dans des intensités, des rythmes, des trajectoires. Si les gestes de la danse ne sont pas \textit{a priori} instrumentaux, les \glspl{DMI} rendent plus que jamais possible leur utilisation à des fins d'interaction musicale\footnote{Voir par exemple: \cite{bevilacqua_gesture_2011, alaoui_movement_2012, silang_maranan_designing_2014, hsueh_understanding_2019}.}, comme nous le verrons plus loin.

\subsubsection{Aspects sémiotiques}

\noindent Un deuxième aspect du geste est indiqué par le paradoxal épithète de la définition du Larousse : ``porteur de sens ou non''. S'il a semblé utile d'évoquer ici une qualité potentiellement absente, c'est précisément parce les différentes définitions attribuées au geste dépendent en partie de cet aspect sémiotique, qui reste soumis à une interprétation subjective et contextuelle dans un système de valeurs. Les significations d'un geste peuvent en effet être attribuées par l'auteur·e du geste ou bien par celui ou celle qui observe ce geste, sans qu'il n'y ait nécessairement de correspondance, ni en terme de signification, ni sur la part du mouvement considérée signifiante. La signification d'un geste peut également être attribuée par la machine, en particulier dans les systèmes d'apprentissage, comme nous le verrons plus loin.\\
\indent Cependant, par rapport aux théories de la sémiologie du geste communicationnel développées notamment par David McNeil \cite{mcneill_gesture_2005} ou Adam Kendon \cite{kendon_gesture:_2004}, le geste musical semble davantage se situer à un niveau pré-sémiotique\footnote{Guerino Mazzola propose une analyse conséquente de cet aspect pré-sémiotique du geste dans le troisième volume de \textit{Topos of Music} \cite{mazzola_topos_2018} pp. 859-865.}, en condensant d'innombrables plans de significations possibles dans leur déploiement, et permettant leur interprétation esthésique\footnote{Jean-Jacques Nattiez développe une sémiologie de la musique qui s'appuie notamment sur les concepts d'analyse poïétique, concernant les processus de fabrication, et esthésique, concernant ceux de leur réception. Voir \cite{nattiez_musicologie_1987}.} à de multiples niveaux. Umberto Eco évoque cette idée à propos de l'œuvre d'art dès la préface de ``L'œuvre ouverte'': \iquote{L'œuvre d'art est un message fondamentalement ambigu, une pluralité de signifiés qui coexistent en un seul signifiant.}\footnote{\cite{eco_loeuvre_2015}.}, et Maurice Merleau-Ponty résume élégamment cette force créative par la formule \iquote{La parole est un geste et sa signification un monde}\footnote{\cite{merleau-ponty_phenomenologie_1976}.}.
\todo{sur la notion de geste pré-sémiotique, voir aussi Eco, U. (1999). Kant et l'orithorynque.}

\subsubsection{Aspects ergotiques}

\noindent Une autre définition du geste relève sa dimension opératoire: \iquote{Manière de mouvoir le corps, les membres et, en particulier, manière de mouvoir les mains dans un but de préhension, de manipulation} (Larousse). Cette définition met l'accent sur la relation possible entre le geste et un objet, un outil, un instrument, et s'applique ainsi particulièrement bien à la situation instrumentale. Un aspect qui lui est lié est la fonction épistémique du geste, qui permet de prendre connaissance de son environnement, des matériaux, de se situer et de s'ajuster en permanence. C'est à ces aspects que s'intéresse, en particulier, la typologie gestuelle proposée par Claude Cadoz que je discuterai plus loin. \todo{note sur les récepteurs de Pacini, la motricité fine ?}

\subsubsection{Aspects (dia)grammatiques}

\noindent Enfin, si le geste \textit{ex-prime} (du latin \textit{ex-premo} : ``faire sortir en pressant'') des mouvements internes du corps, il \textit{im-prime} et laisse une trace dans la matière, dans la machine, dans les esprits. C'est parfois cette trace résultante qu'on nomme geste, par métonymie et analogie avec les qualités de mouvement qui sont à son origine, lorsqu'on parle par exemple du \textit{geste de composition}. Les capacités des machines à enregistrer le geste dans sa dynamique --~auparavant éphémère et seulement visible sous la forme de traces statiques: littérature, peinture, partition~-- confèrent à cet aspect \textit{diagrammatique} du geste une importance particulière à l'époque de sa reproductibilité technique\footnote{L'expression utilisée ici renvoie bien évidemment à l'œuvre de Walter Benjamin \cite{benjamin_loeuvre_2013} qui analyse, pour sa part, les conséquences de cette révolution technique sur la réception des œuvres d'art, selon une perspective socio-politique.}\todo{ref Benjamin bof}, qui sera analysée à la lumière de la notion de (dia)gramme proposée par Gilles Deleuze et Bernard Stiegler.

%--------------------------------------------


% Jensenius in Musical Gesture : "Based on the above viewpoints, it seems straightforward to define musical gesture as an action pattern that produces music, is encoded in music, or is made in response to music.Qualifications can be added to the term musical gesture whenever needed to avoid misunderstandings.For example, one can speak about sound-producing gestures, sound-modifying gestures, sound-accompanying gestures, sonic gestures, playing gestures, and so on."

%-------------------------------------------
\subsection{Spécificités des DMIs}

\subsubsection{Les DMIs ne sont pas (que) des instruments acoustiques}

\noindent Dans le contexte particulier des lutheries numériques, un certain nombre de problèmes se posent par rapport au geste instrumental sur les instruments acoustiques, liés à des spécificités du numérique que nous avons déjà en partie évoquées dans le chapitre précédent\footnote{De manière plus globale et sociologique, les révolutions technologiques électroniques ont également modifié les modes de production et de réception de la musique à l'échelle industrielle, entrainant d'autres conséquences que celles listées ci-après, mais qui dépassent le sujet que nous traitons ici. Je renvoie pour ces questions-là aux analyses de Paul Théberge \cite{theberge_any_1997}, Phillip Auslander \cite{auslander_liveness:_2008}, ou plus récemment de Rebecca Bennett et Angela Cresswell Jones \cite{jones_digital_2015}.}:
%-----------
\vspace{-1em}
\begin{itemize}[noitemsep]
	\item \textbf{le découplage énergétique} : ce découplage est la différence la plus saillante par rapport aux lutheries acoustiques. Certains de ses aspects sont déjà présents dans les instruments électriques et électroniques, mais l'informatique numérique ajoute un découplage de plus: les signaux (gestuels, sonores) n'y existent plus sous forme analogique et se présentent sous la forme de représentations numériques du signal, c'est-à-dire \textit{quantifiées} et encodés dans un alphabet;
	\item \textbf{la mémoire et le temps différé}: si le terme de ``temps réel'' est abondamment utilisé quand on parle des \glspl{DMI}, il ne faut pas oublier que l'informatique introduit avant tout la possiblité du ``temps différé'', de l'enregistrement exploitable dans le futur. Si la possibilité d'enregistrement est également présente sur les instruments acoustiques (les jeux d'orgues sont, d'une certaine manière, des \textit{presets}), c'est la différence d'échelle qui fait ici la particularité du numérique;
	\item \textbf{la computation} : le traitement algorithmique permet --~ou impose~-- une reconfiguration permanente des modèles et des représentations, qui entraîne une instabilité, un métamorphisme des relations entre les données.
\end{itemize}
%-----------
\noindent Par ailleurs, les \glspl{DMI} ne sont encore souvent pas considérés par les institutions musicales au même titre que les instruments acoustiques: leur pratique n'est pas enseignée en tant que telle dans les conservatoires\footnote{La pratique musicale des \glspl{DMI} est généralement reléguée à une pratique optionnelle dans les classes de composition électroacoustique ou \gls{MAO}}, et la partie électronique dans les concerts institutionnels est souvent absente de la scène et confiée à un·e \gls{RIM} (et non à un·e ``musicien·ne'') au statut ambigu\footnote{C'est particulièrement le cas pour les concerts de musique mixte dans la tradition de l'\gls{IRCAM}, beaucoup moins pour les concerts organisés par le \gls{GRM} dont l'histoire n'a pas cet ancrage dans la musique classique acoustique. Voir notamment l'analyse de Laura Zattra sur l'origine du statut de \gls{RIM} à l'\gls{IRCAM}: \cite{zattra_les_2013}}, et qui opère généralement dans l'ombre. Cette situation n'a guère favorisé la considération des interfaces électroniques et des nouveaux gestes qui leur étaient propres en tant qu'instruments de performance musicale à part entière.

%-------------------------------------------
\subsubsection{Les DMIs ne sont pas des ``interfaces utilisateur''}

\noindent Un biais de l'approche fonctionnelle des \gls{IHM} est lié au fait que l'interaction a longtemps été définie (et l'est parfois encore) par la perspective d'une tâche que l'utilisateur doit accomplir. Ce contexte écologique a contribué au développement de tout un champ d'étude dans le domaine des \gls{IHM}: les \textit{user studies}, transposé dans le domaine de l'ingénierie industrielle sous le nom ``d'Expérience Utilisateur''\footnote{En abrégé : UX pour User eXperience; le métier de ``UX designer'' étant désormais très répandu dans le domaine du développement logiciel mais également appliqué à d'autres domaines tels que la grande consommation.}. L'efficacité dans l'accomplissement d'une tâche et l'absence d'effort --~qui constitue \iquote{une vertu cardinale dans la mythologie de l'ordinateur}\footnote{\iquote{Most computer instruments in use are those provided by the commercial music industry. Their inadequacy has been obvious from the start; emphasizing rather than narrowing the separation of the musician from the sound. Too often controllers are selected to minimize the physical, selected because they are effortless. Effortlessness in fact is one of the cardinal virtues in the mythology of the computer.} dans \cite{ryan_remarks_1991}}, comme le notait Joel Ryan~-- ont longtemps été les principales motivations ayant guidé le design d'interaction dans les \gls{IHM}. La notion ``d'expérience utilisateur'' est venue étendre cette analyse éco-systémique en intégrant des aspects esthétiques ou émotionnels dans leur design, mais elle privilégie généralement la recherche d'une utilisation facile, efficace et confortable pour l'utilisateur.\\
%---- Figure : instruments abusers---------
\begin{figure}[!htbp]
	\captionsetup{format=plain}%
	\includegraphics[width=\textwidth]{gfx/03_gesture/instrumentabusers.png}
	\caption[Les instruments ne sont pas des interfaces utilisateur]{Les instruments ne sont pas des \textit{interfaces utilisateur}. De gauche à droite : John Cage, Jimi Hendrix, Paul Simonon, Charlotte Moorman.}
	\label{fig:gesture:abusers}
\end{figure}
%---- Figure : instruments abusers---------
\indent Or dans une performance musicale vivante, cette situation n'existe pas. L'instrumentiste ne saurait être défini comme ``l'utilisateur de son instrument'' (figure \ref{fig:gesture:abusers}), pas plus que son instrument comme une ``interface utilisateur'', et comme le note Ryan, il serait tout aussi intéressant de rendre son contrôle aussi difficile que possible\footnote{\iquote{In designing a new instrument it might be just as interesting to make control as difficult as possible.} \cite{ryan_remarks_1991}}. Les instrumentistes cherchent bien souvent à produire quelque chose à la limite des possibilités de leur instrument\footnote{Dans les entretiens que j'ai pu mener, ce désir d'outrepasser les limites revenait régulièrement, comme ici dans la réponse de Nicolas Bernier à la question de ce qui l'avait amené à utiliser les technologies numériques: ``je peux pas donner une réponse comme récente là... mais je pense que c'est juste \textit{l'infini des possibles} (...) le numérique je m'en fous un peu mais les sons, les autres, sont intéressants, puis quand on se met à pouvoir faire de la musique avec tous les sons... ben c'est sûr que ça ouvre... tout à coup il ya plus de limites donc ça je pense c'est une des grandes motivations...'' [entretien personnel]. Jamais en revanche n'a été mentionné la recherche de facilité ou de confort.}.
%Dans les entretiens que j'ai pu mener, ce désir de toucher voire d'outrepasser les limites revenait régulièrement
%VG — Qu'est ce qui t'a amené à utiliser les technologies numériques plutôt que de faire de la guitare électrique ou du piano ? c'était quoi la motivation ?
% NB — ouais, c'est quand même une bonne question ... ça remonte à quelques années quand même... tu sais faut que tu te demandes à cette époque là quand j'ai commencé, qu'est ce qui m'a amené ... je peux pas donner une réponse comme récente là... mais je pense que c'est juste \textit{l'infini des possibles} qui ... avec l'instrumental puis avec ma ... c'est une question de capacité moi je venais de la musique pop hein... donc trois accords et quelques rythmes différents mais tandis qu'avec les sons... dans le fond c'est pas tant le numérique... c'est ça, ça c'est une bonne réponse quand même, le numérique je m'en fous un peu mais les sons, les autres sont intéressants, puis quand on se met à pouvoir faire de la musique avec tous les sons... voici... ben c'est sûr que ça ouvre... tout à coup il ya plus de limites donc ça je pense c'est une des grandes motivations... 
Que cela soit le contre-fa de la Reine de la Nuit\index[people]{mozart@Mozart, Wolfgang Amadeus!fluteenchantee@\textit{La Flûte enchantée}}, les pianos préparé de John Cage\index[people]{cage@Cage, John}, le scratch sur les platines vinyle d'un Q-bert\index[people]{qbert@QBert (Quitevis, Richard alias~--)} ou les mises en larsen de table de mixage ``\textit{no-input}'' de la scène Onkyokei\index[people]{onkyokei@Onkyokei (scène)}, les exemples abondent dans l'histoire de la musique qui font preuve d'une démarche allant dans l'outrepassement des possibilités instrumentales lors du jeu musical. Dans une conversation durant la dernière conférence \gls{NIME}, Paul Stapleton utilisait ironiquement le terme ``\textit{interface abusers}'' pour souligner l'erreur du terme ``\textit{interface users}''.\\
\indent Une conséquence en terme de design instrumental est la possibilité d'envisager la conception d'interfaces bancales ou inadaptées comme une stratégie volontaire, plutôt qu'elle soit un artéfact résultant d'une connaissance approximative des outils. Anthony Dunne souligne l'intérêt d'une telle démarche dans ce qu'il nomme des ``objets post-optimaux'' présentant une ``inconvivialité'' (\textit{user-unfriendliness}) recherchée : \iquote{Si la convivialité caractérise la relation entre l'utilisateur et l'objet optimal, l'inconvivialité dans ce cas, une forme de douce provocation, pourrait caractériser l'objet post-optimal. L'accent n'est plus mis sur l'optimisation de l'adéquation entre les personnes et les objets électroniques par le biais d'une communication transparente, mais sur la délivrance d'expériences esthétiques par le biais des objets électroniques eux-mêmes.}\footnote{\iquote{If user-friendliness characterizes the relationship between the user and the optimal object, user-unfriendliness then, a form of gentle provocation, could characterize the post-optimal object.The emphasis shifts from optimizing the fit between people and electronic objects through transparent communication, to providing aesthetic experiences through the electronic objects themselves.}, \cite{dunne_hertzian_1999} p.~35.}


\subsubsection{Les limites du contrôle}

\noindent La plupart de la littérature sur les \glspl{DMI} met l'accent sur la notion de contrôle par le geste (et les interfaces y sont souvent nommées ``contrôleurs gestuels''). La performance musicale n'est toutefois pas exclusivement gouvernée par une relation de contrôle absolu du son. De nombreux exemples dans des styles musicaux très divers viennent démontrer l'absence de contrôle --~voire sa recherche~-- sur certains aspects musicaux : les \textit{couacs} dans le jeu de John Coltrane\index[people]{coltrane@Coltrane, John}, le ``laisser-jouer'' de la machine dans la musique noise\footnote{Voir en particulier les analyse de Paul Hegarty et Sarah Benhaim sur le non-contrôle dans la musique noise \cite{hegarty_noise_2007, benhaim_aux_2018}.}, les partitions impossibles de Brian Ferneyhough\index[people]{ferneyhough@Ferneyhough, Brian}\footnote{On notera au passage, en poursuivant la critique de l'analogie entre instrument et \gls{IHM}, que la partition \textbf{n'est pas} un mode d'emploi.}, etc.\\
\indent Si l'on considère la pratique des \glspl{DMI} par rapport à celle des instruments acoustiques, on est amené à faire le constat que là où la présence de l'instrumentiste sur scène remplissait une nécessité acoustique pour l’écoute, la musique sur support, ou produite par des machines, déplace ce besoin au profit d’une autre fonction qu'elle exacerbe. La performance musicale est le théâtre\footnote{Le ``théâtre instrumental'' développé par Mauricio Kagel\index[people]{kagel@Kagel, Mauricio} dans les années 1970 prend à bras le corps cette question de la place des musicien·ne·s la scène théâtrale, actant le fait que leur présence n'y est plus nécessaire pour l'interprétation d'une musique d'accompagnement.} d'un corps plongé dans l'instabilité propre au médium sonore et l'instrumentiste peut s'y trouver dans différentes postures face à ce qui jaillit de son instrument (parfois à son insu), pilote à la maîtrise totale, chimiste opérant des mélanges incertains de fluides sonores, chirurgien·ne disséquant le son, funambule en équilibre sur le fil de l'audible, dompteur de cirque face à un instrument sauvage, cascadeu·r·se se jetant dans le vide, explorat·eur·rice dans une jungle inconnue, chamane faisant naître la transe... autant de postures différentes qui partagent toute une attention extrême à leur environnement.

\subsubsection{[In]fidélité instrumentale}

\noindent La notion de ``fidélité'' a également été vantée dans les dispositifs d'enregistrement et de reproduction audio, depuis leurs origines, comme un gage de qualité. Fernando Iazzetta \cite{iazzetta_meaning_2000} analyse la manière dont la notion de fidélité dans l'industrie musicale, originalement utilisée pour désigner la capacité d'un enregistrement à reproduire les qualités sonores de la performance originale, a progressivement évolué vers une notion ne faisant plus référence au son original, mais établie en fonction de la technologie d'enregistrement disponible. Les conditions contemporaines de production musicale ont à certains égards renversé le sens de la fidélité : un morceau produit intégralement en studio à l'aide d'outils d'édition numériques, précède souvent sa performance (si performance il y a un jour) et amène ainsi les musicien·ne·s à chercher à reproduire dans leurs performances live les qualités sonores de leurs disques et parfois même, ironiquement, à devoir en inventer les gestes.


%%%%%%%%%%%%%%%%%%%%%%%%%%%%%%%%%%%%%%%%%%%%%%%%%%%%%%%%%%%%%%%%%%%%
%%%%%%%%%%%%%%%%%%%%%%%%%%%%%%%%%%%%%%%%%%%%%%%%%%%%%%%%%%%%%%%%%%%%
%%%%%%%%%%%%%%%%%%%%%%%%%%%%%%%%%%%%%%%%%%%%%%%%%%%%%%%%%%%%%%%%%%%%
\section{Le modèle ergotique et l'héritage acoustique}
\label{sec:gesture:ergotic}

\subsection{Continuum énergétique : typologie de Cadoz}

\noindent Claude Cadoz fait partie des pionniers de l'analyse du geste instrumental, en prenant en compte les spécificités propres à cette pratique gestuelle et en les mettant en perspective des technologies informatiques. En particulier, comme son nom l'indique, le geste instrumental n'est pas un ``geste nu'' mais se retrouve couplé à un instrument qui polarise les termes de leur interaction (sans toutefois les définir totalement). Le geste est instrumentalisé, médiatisé, et sa description passe par l'analyse du rapport qu'il entretient avec l'interface instrumentale.\\
\indent Dès 1978, Cadoz décrit avec Jean-Loup Florens dans un article séminal\footnote{``Fondement d’une démarche de recherche informatique / musique'' \cite{cadoz_fondement_1978}.} un grand nombre des  enjeux soulevés par le contrôle gestuel de la synthèse audio-numérique. Ils y explicitent notamment les spécificités induites par l'ordinateur présentées précédemment et proposent, dans la continuité des travaux de Pierre Schaeffer, la notion d'\textit{objet gestuel}\footnote{Le titre de la section : ``L'objet gestuel - champ expérimental'' laisse entendre que tout reste à y faire, et ce concept sera au final assez peu repris par Cadoz.}. Bien que la rupture du numérique et notamment \iquote{l'artifice [du continuum énergétique]} y soit exposée avec lucidité, Cadoz et Florens remarquent aussi que : \iquote{La perception des objets musicaux a ses racines, ses références, ses codages dans la pratique traditionnelle}\footnote{\cite{cadoz_fondement_1978}.}. C'est probablement cette volonté d'ancrage dans la pratique traditionnelle acoustique qui amène Cadoz à définir dès 1981\footnote{\cite{cadoz_synthese_1981}.} des catégories gestuelles basées sur la notion de continuum énergétique, qui polariseront fortement les développements de l'\gls{ACROE}. Claude Cadoz, Annie Luciani, Jean-Loup Florens et Sylvie Gibet définiront progressivement la nomenclature suivante pour décrire les différents types de \iquote{gestes instrumentaux} :
\vspace{-1em}
	\begin{itemize}[noitemsep]
		\item \textbf{gestes d'excitation} qui fournissent l'énergie qui sera présente dans le son \textit{in fine}. Ils peuvent être de nature ``continue'', quand le son et le geste co-existent (e.g. frottement de l'archet, souffle dans un instrument à vent), ou de nature ``instantanée'', si le son commence quand le geste finit (e.g. percussion, pincement de corde) \cite{cadoz_gesture_2000};
		\item \textbf{gestes de modulation} venant modifier les propriétés de l'instrument. Une distinction est apportée par la suite dans \cite{cadoz_synthese_1983} entre modifications ``paramétriques'', telles que le vibrato, et les modifications ``structurelles'' (e.g. ajout d'une sourdine sur une trompette, sélection d'un jeu d'orgues, etc.);
		\item \textbf{gestes de sélection}, ajoutés à cette nomenclature en 1984 dans \cite{luciani_modelisation_1984}, ils consistent à choisir parmi plusieurs éléments similaires d'un instrument (e.g. quelle touche de piano, quel corde de harpe, quel fût de batterie, etc.);
		\item \textbf{gestes de polarisation ou de maintien}, ajoutés en en 1999 dans \cite{cadoz_gesture_2000}, ils consistent à assurer des conditions normales de fonctionnement à l'instrument (e.g. le geste du bras qui assure un niveau de pression suffisant pour le jeu de cornemuse).
\end{itemize}

\subsection{Intérêts et limites du modèle ergotique}

\noindent Cette classification du geste instrumental ``producteur de son'' peut sembler relativement bien adaptée aux instruments acoustiques. Elle est devenue une référence sur le sujet et abondamment citée dans la littérature des \gls{NIME}. Assez paradoxalement, elle a été prise comme modèle pour le design de l'interaction des \glspl{DMI} développés à l'\gls{ACROE} mais également par de nombreux autres luthièrs numériques (e.g. \cite{arfib_strategies_2002}, \cite{schwarz_sound_2012}), alors même que ses auteurs précisent dans \cite{cadoz_geste_1994, cadoz_gesture_2000} que :
\vspace{-1em}
	\begin{itemize}[noitemsep]
		\item il est nécessaire que l'instrument soit stable durant la performance;
		\item un continuum énergétique doit exister entre le geste et le phénomène perçu;
		\item le geste doit être appliqué à un objet matériel et il doit exister une interaction physique avec lui (le cas du Theremin étant considéré comme une exception rare).	
\end{itemize}
\noindent Or, ces trois aspects sont précisément mis en défaut dans le cas des \glspl{DMI}: le continuum énergétique est \textit{a priori} rompu, les instruments sont sujets à des possibles reconfigurations dynamiques\footnote{qu'elles soient souhaitées ou dûes au contexte d'obsolescence (cf. \ref{sec:ephemeral:ephemerality_in_musical_context})}, et le geste est bien souvent capté en dehors de tout contact physique\footnote{Par des capteurs de distance, accéléromètres, caméras, etc. Voir le chapitre \ref{ch:interfaces}.}. Cette relation pluri-millénaire étant cassée, l'ambition de l'\gls{ACROE} a été de tenter de la recréer artificiellement par des systèmes de capteurs, d'actionneurs et des stratégies de mapping servant la définition de cette relation. Ces recherches ont notamment abouti au système \gls{CORDIS-ANIMA}, dispositif pionnier en matière de retour d'effort et de synthèse par modèle physique, qui n'a malheureusement pas connu une grande utilisation hors du laboratoire. On se garderait donc bien de dire que l'inaptitude de cette catégorisation à décrire le geste instrumental dans le cas des \glspl{DMI} ait été un obstacle à l'avancement des travaux de l'\gls{ACROE}. Elle a au contraire été une direction idéologique motrice\footnote{Et d'une certaine manière, on peut aussi y voir un choix esthétique, Cadoz étant aussi compositeur.} pour le développement de modèles physiques et de systèmes à retour d'effort avancés.\\
\indent Pour autant, après bientôt quarante ans de pratiques musicales numériques, nous pouvons observer que cette direction n'était pas la seule possible et de nombreuses stratégies de jeu se sont développées, de manière ``sauvage''\footnote{Le terme de ``lutherie sauvage'' est couramment employé pour désigner des lutheries expérimentales pratiquées hors des méthodes industrielles, académiques ou traditionnelles, notamment avec des objets de récupération et du détournement de dispositifs électroniques.} dans des pratiques \gls{DIY}, sans être freinées par l'absence de continuum énergétique, ni l'instabilité de l'instrument, ni l'absence de contact --~au contraire, les artistes ont embrassé ces artéfacts. L'idée du continuum énergétique est donc insuffisante pour comprendre les termes du geste instrumental numérique et ses traductions en terme de lutherie\footnote{Il ne faudrait cependant pas déduire de cette critique que Cadoz n'est pas conscient des limites de ce modèle; même s'il leur refuse généralement le statut de ``geste instrumental'' au sens qu'il a donné à ce terme, il a contribué dans de nombreux articles à analyser de manière nuancée d'autres aspects du geste présentés ci-après.}.


\section{Expressivité et sémiologie du geste nu}

\subsection{Du fonctionnel au symbolique : typologie de Delalande}

\noindent Dans son analyse des gestes de Glenn Gould au piano, François Delalande a établi une autre typologie de geste abondamment citée\footnote{Notons toutefois que si Delalande utilise ces différents termes, il ne les présentent pas explicitement comme des catégories gestuelles absolues, comme peut le faire Cadoz, pas même comme une liste, comme ils sont souvent présentés --~et ici encore. Il prend soin de préciser que cette division est \iquote{un artifice de description ne correspondant à aucune dissociation effective dans la conduite expressive} \cite{delalande_geste_1988}.}, sur \iquote{au moins trois niveaux, qui vont du purement fonctionnel au purement symbolique}. Il distingue ainsi :
\vspace{-1em}
\begin{itemize}[noitemsep]
	\item \textbf{des gestes effecteurs} responsables de la production du son (le toucher du clavier dans le cas de Gould) et correspondant, d'une certaine manière, à la notion de geste instrumental de Cadoz mentionné plus haut;
	\item \textbf{des gestes accompagnateurs}, qui engagent le corps entier et semblent reliés à une \iquote{attitude affective}. S'ils semblent en apparence moins directement responsables de la production du son, Delalande les considère comme des \iquote{schèmes expressifs} incarnant le lien entre le plan de l'imagination et celui de la production effective du son;
	\item \textbf{des gestes évocateurs} perçus dans la musique par l'auditeur, tels qu'un appui de phrase ou une envolée, et qui ne semblent pas directement liés aux mouvements du corps. Delalande évoque notamment ces gestes comme traduction possible d'un imaginaire associé à la musique, tel qu'une dimension orchestrale dans le jeu pianistique de Gould.
\end{itemize}
\noindent Marcelo Wanderley a mis en évidence que les gestes accompagnateurs, qu'il appelle \iquote{gestes ancillaires} ou \iquote{gestes non-évidents}\footnote{\iquote{Non-obvious gestures}. Voir \cite{wanderley_non-obvious_1999}.} avaient une influence mesurable sur le résultat sonore, par exemple en terme de projection acoustique. Il a par ailleurs été montré que les doigts ne sont pas un simple système mécanique indépendant des mouvements du reste du corps et que la performance d'une phrase musicale requiert des inflexions \textit{facilitées} par ces mouvements du corps\footnote{Voir en particulier \cite{chadefaux_experimental_2012}.}. Par ailleurs, la perception du résultat sonore est influencée par la perception visuelle\footnote{\label{fn:mcgurk} Un exemple connu en est l'effet McGurk, mettant en évidence l'interférence entre l'audition et la vision lors de la perception de la parole. Cf. \cite{macdonald_visual_1978}.Exemple audio: \url{https://www.youtube.com/watch?v=PWGeUztTkRA}}. Les gestes accompagnateurs, outre leur influence sur le jeu et le son, ont également une influence sur la manière dont l'auditeur les perçoit.

\subsection{Imaginaire du geste et du son}
\label{sec:gesture:imaginaire-du-geste-et-du-son}

\noindent Dès 1995, Todd Winkler proposait de repenser le geste instrumental dans les \glspl{DMI} en proposant des contraintes et des idiomes libérés du modèle de l'instrument acoustique. Sa formulation de cette problématique est intéressante en ce qu'elle envisage déjà la sonorité des gestes au-delà du modèle acoustique excitation/résonance. Cependant, ses propositions reflètent son attachement au paradigme physique et à la corrélation entre l'énergie du geste et du son : \iquote{Qu'est ce que la musique des doigts ? Qu'est ce que la musique de course ? Quel \textit{est} le son d'\textit{une seule} main qui claque ? On peut répondre à ces questions en permettant à la physicalité du mouvement d'avoir un impact sur le matériel et les processus musicaux. Ces relations peuvent être établies en considérant le corps et l'espace comme des instruments de musique, libérés des relations dans les instruments acoustiques, mais avec des contraintes similaires qui peuvent donner du caractère au son par des mouvements idiomatiques.}\footnote{\iquote{What is finger music? What is running music? What \textit{is} the sound of \textit{one} hand clapping? These questions may be answered by allowing the physicality of movement to impact on musical material and processes. These relationships may be established by viewing the body and space as musical instruments, free from the associations of acoustic instruments, but with similar limitations that can lend character to sound through idiomatic movements.} dans \cite{winkler_making_1995}.}\\
\indent En analysant les gestes induits par le son, Rolf Inge Godøy a souligné la manifestation de deux autres types de gestes d'accompagnement : les ``gestes de tracé sonore''\footnote{\iquote{sound-tracing gestures}, cf \cite{godoy_exploring_2006}.} qui suivent le contour des morphologies sonores (e.g. le contour mélodique) et les ``gestes d'imitation des gestes de production du son''\footnote{\iquote{mimicry of sound-producing gestures}, cf. \cite{godoy_playing_2005}.} en prenant notamment comme exemple les performances d'\textit{air-guitar}\footnote{\label{fn:airguitar}Le \textit{air guitar} est une activité qui consiste à mimer le geste d’un guitariste, typiquement de guitare électrique dans un groupe de rock ou de métal, sans avoir l’instrument en main, dans une sorte de playback instrumental.}. Il est intéressant de s'arrêter ici sur ces deux types de gestes. En effet, ils ne sont pas \textit{a priori} des gestes instrumentaux dans la mesure où ils ne sont pas effectués par quelqu'un en situation de jeu instrumental (mais sans aucun doute par quelqu'un qui \textit{musique}, au sens de Christopher Small). Il s'agit là de gestes qui relèvent en partie d'une forme de théâtralité mais aussi, comme l'explique Godøy, d'une forme de ``geste d'écoute'' : \iquote{(...) nous pouvons donner un sens à ce que nous entendons parce que nous devinons comment les sons sont produits (...) des études récentes sur la neuro-imagerie semblent appuyer l'idée que la perception est un process actif de la cognition motrice\footnote{\iquote{(...) we can make sense out of what we hear because we guess how the sounds are produced. (...) recent neuro-imaging studies seem to support the idea of perception as an active process involving motor cognition.} \cite{godoy_exploring_2006}.}.} Godøy développe cette idée dans ce qu'il appelle des \iquote{objets gestuels-sonores}\footnote{\iquote{Gestural-sonorous objets} décrits dans \cite{godoy_gestural-sonorous_2006}.}, qui étendent la typologie Schaefferienne des ``objets sonores''\footnote{\cite{schaeffer_traite_1966}.}, à l'étude des gestes associés aux différents objets sonores.\\
\indent Cette fonctionnalité gestuelle, à rapprocher des \textit{gestes évocateurs} évoqués par Delalande, présente ici l'intérêt de décrire des gestes \textit{physiques} exprimant des relations \textit{imaginaires} entre le geste et le son. La morphologie du geste y découle de l'écoute musicale, renversant ainsi la perspective de causalité entre le geste et le son, telle qu'elle existe dans les instruments acoustiques. Dans le cas des \glspl{DMI} où cette relation est \textit{a priori} dépourvue de causalité, ces catégories s'avèrent ainsi très pertinentes, en ce qu'elles nous renseignent sur des axes possibles sur lesquels cette relation peut se construire en l'absence de toute contrainte physico-énergétique. C'est notamment sur le principe de telles correspondances qu'ont été développés des systèmes de contrôle musical par suivi de gestes\footnote{Voir les travaux menés au sein de l'équipe \textit{Interaction Son Musique Mouvement} (ISMM) de l'\gls{IRCAM}, en particulier le projet \textit{Modular Musical Objects} \cite{caramiaux_mapping_2014, francoise_motion-sound_2015}, ou ceux de Rebecca Fiebrinks, en particulier le projet Wekinator, \cite{fiebrink_wekinator:_2010} à l'Université Queen Mary de Londres.}.\\
\indent Le suivi de geste et sa reconnaissance par apprentissage-machine sur la base d'un vocabulaire de formes gestuelles pré-définies, rend possible le design d'une interaction à mi-chemin entre ce que Cadoz appelle \textit{gestes de sélection} et \textit{gestes de modulation}. Les systèmes d'apprentissage permettent en effet de calculer, non pas la catégorie d'un geste, mais la probabilité de présence de différentes catégories dans le geste, et leur écart par rapport aux formes typiques pré-définies. Les capacités d'interpolation de la machine permettent ici de reconstruire un espace continu à partir d'un espace catégoriel, à l'inverse de ce qui est communément réalisé sur les instruments acoustiques, à savoir le striage d'un espace continu en des catégories discrètes (touches de piano, frettes de guitare, etc.)


%%%%%%%%%%%%%%%%%%%%%%%%%%%%%%%%%%%%%%%%%%%%%%%%%%%%%%%%%%%%%%%%%%%
%%%%%%%%%%%%%%%%%%%%%%%%%%%%%%%%%%%%%%%%%%%%%%%%%%%%%%%%%%%%%%%%%%%%

\section{Geste programmé, geste re-sonné}
\label{sec:gesture:instrumental_to_musical}
%-------------------------------------------
\subsection{L'outil comme externalisation de la mémoire}
\label{sec:gesture:instrumental_to_musical:externalisation}

\noindent Dans son essai ``Le geste et la parole'' paru en 1964\footnote{\cite{leroi-gourhan_geste_1964}}, le paléo-anthropologue André Leroi-Gourhan met en lumière la manière dont l'invention et l'utilisation d'outils techniques contribuent à l'évolution de l'humain, par une externalisation progressive des processus opératoires dans les outils :
\vspace{-1em}
\begin{quotation}
	Au cours de l’évolution humaine, la main enrichit ses modes d’action dans le processus opératoire. L’action manipulatrice des Primates, dans laquelle geste et outil se confondent, est suivie avec les premiers Anthropiens par celle de la main en motricité directe où l’outil manuel est devenu séparable du geste moteur. À l’étape suivante, franchie peut-être avant le Néolithique, les machines manuelles annexent le geste et la main en motricité directe n’apporte que son impulsion motrice. Au cours des temps historiques la force motrice elle-même quitte le bras humain, la main déclenche le processus moteur dans les machines animales ou les machines automotrices comme les moulins. Enfin au dernier stade, la main déclenche un processus programmé dans les machines automatiques qui non seulement extériorisent l’outil, le geste et la motricité, mais empiètent sur la mémoire et le comportement machinal.\footnote{\cite{leroi-gourhan_geste_1964} pp. 41-42.}
\end{quotation}
\indent Il note ainsi que l’externalisation des facultés de l'humain s’est étendue à tous ses organes, jusqu'aux fonctions cérébrales de la mémoire, et prédit les opérations computationnelles rendues possibles par le numérique : \iquote{Les fichiers à perforations sont des machines à rassembler des souvenirs, elles agissent comme une mémoire cérébrale de capacité indéfinie, susceptible, au-delà des moyens de la mémoire cérébrale humaine, de mettre chaque souvenir en corrélation avec tous les autres\footnote{\cite{leroi-gourhan_geste_1964} p. 74.}.} Cette idée est développée en ce qui concerne les \glspl{DMI} par Thor Magnusson qui les envisage comme des \iquote{outils épistémiques}\footnote{Cf. supra, section \ref{sec:ephemeral:vessels}} et qui souligne, lors de la création d'un \gls{DMI}, le processus de création d'affordance permettant l'interaction avec le matériau que constitue cette mémoire musicale numérisée\footnote{\iquote{As opposed to the acoustic instrument maker, the designer of the composed digital instrument frames affordances through symbolic design, thereby creating a snapshot of musical theory, freezing musical culture in time.} \cite{magnusson_epistemic_2009}.}.

%-------------------------------------------
\subsection{Le geste programmé}
\label{sec:gesture:instrumental_to_musical:geste_programme}

\noindent Le mouvement qui anime le son peut être réalisé explicitement par un instrumentiste humain ou bien produit de manière automatisée par la machine; on propose alors de parler de gestes ``programmés''. Leur définition peut être ``extensive'', par exemple sous la forme d'enregistrements (samples, courbes d'automation, etc. Figure \ref{fig:gesture:automation}) ou ``intensive'', c'est-à-dire définie par une règle générative\footnote{Sur les notions de notation ``intensive'' et ``extensive'', voir Giavitto \cite{giavitto_du_2014}.}. Si toutefois la définition du geste implique que le mouvement soit associé à une intention, on ne peut prêter une intention à la machine qu'à travers la ``programmation'' de ce mouvement machinique par la personnne qui compose l'instrument.\\
%------------------ Figure : geste programmé ---------------------
\begin{figure}[!htbp]
	\captionsetup{format=plain}%
	\centering
	\begin{minipage}[t]{0.48\textwidth}
		\includegraphics[width=\linewidth]{gfx/03_gesture/AbletonLiveAutomation_72dpi.png}
		\caption[Une courbe d'automation dans le logiciel Ableton Live]{Une courbe d'automation dans le logiciel Ableton Live, synchronisée à un échantillon audio.}
		\label{fig:gesture:automation}
	\end{minipage}
	\hspace{.02\linewidth}
	\begin{minipage}[t]{0.48\textwidth}
	  \includegraphics[width=\linewidth]{gfx/03_gesture/EnriqueThomas-TangibleScore.jpg}
		\caption[Partition tangible d'Enrique Tomás]{Partition tangible d'Enrique Tomás.}
		\label{fig:gesture:tangible_score}
	\end{minipage}
\end{figure}
%------------------ Figure : geste programmé ---------------------
\todo{clarifier la définition de grammatisation et les références afférentes}
\indent Bernard Stiegler développe le concept de ``gramme'' comme \iquote{corps organisé de signes et de symboles} et emprunte à Sylvain Auroux\footnote{Voir \cite{auroux_revolution_1994}.} le concept de ``grammatisation'' comme le processus par lequel le continuum temporel des comportements humains est transformé en un ``discret spatial'', qui permet de les intégrer dans les outils\footnote{Stiegler recourt à ce terme dans plusieurs textes. Il souligne notamment l'intégration de ce réel discrétisé dans les outils dans \cite{stiegler_for_2010}: \iquote{(...) grammatization process which enabled the discretization of corporeal flows, in turn enabling their calculation via machine tools and the apparatus of production, management and conception (...)}. L'idée de ``grammatisation'' fait également écho à la \textit{grammatologie} \cite{derrida_grammatologie_1967} de Jacques Derrida (qui fut aussi le mentor de Stiegler) et au concept de \textit{différance} qu'il y développe, lui-même inspiré par la pensée de Leroi-Gourhan.}. Ainsi en est-il de l'informatique, qui dissocie en symboles et en catégories discrètes ce qui est continu et intégré dans le geste, comme dans le son. Pour Stiegler, les objets sont des enregistrements\footnote{Stiegler utilise les termes de ``rétentions tertiaires'' pour décrire cette inscription de la mémoire dans les objets, afin de la mettre en perspective des ``rétentions primaires'' que sont la conscience du flux temporel et les ``rétentions secondaires'' que sont les souvenirs qui constituent l'expérience d'un individu.}, dans lesquels la mémoire de ce que nous faisons et de ce que nous connaissons est déposée sous la forme d'une mémoire technique.\\
\indent Le concept de Stiegler fait également écho à celui de ``diagramme'', proposé par Gilles Deleuze dans son étude sur la peinture de Francis Bacon:
\begin{quotation}
Le diagramme, c'est donc l'ensemble opératoire des lignes et des zones, des traits et des taches asignifiants et non représentatifs. Et l'opération du diagramme, sa fonction, dit Bacon, c'est de suggérer. Ou, plus rigoureusement, c'est d'introduire des «~possibilités de fait~» : langage proche de celui de Wittgenstein. Les traits et les taches doivent d'autant plus rompre avec la figuration qu'elles sont destinées à nous donner la Figure. C'est pourquoi elles ne suffisent pas elle-mêmes, elles doivent être «~utilisées~» : elles tracent des possibilités de fait, mais ne constituent pas encore un fait (le fait pictural).\footnote{Voir \cite{deleuze_francis_1981}.}
\end{quotation}
\indent On peut voir un parallèle frappant entre cette notion de diagramme et les fonctions assumées à la fois par l'instrument et par la partition musicale.\footnote{Il faudrait encore rapprocher cette notion de diagramme du \iquote{di-son} dans la tripartition des ``modes d'écoute attentive'' (im-son, di-son, mé-son) de François Bayle, qui définit le \iquote{di-son} comme \iquote{le diagramme, sélection de contours simplifiés, indiciels}. Cf. \cite{bayle_musique_1993}} L'instrument de musique contient ainsi l'enregistrement de la théorie musicale qui lui est propre (ses ``traits et ses taches'' que représentent son organisation des hauteurs, sa signature timbrale, son ergonomie, etc.) et que le ou la luthièr·e lui imprime. De même, on peut voir la partition comme un ``enregistrement'' de la pensée et du travail du ou de la composit·eur·rice, une trace de ses ``gestes sédimentés'' comme le disent Jean-Paul Olive et Susanne Kogler\footnote{\cite{olive_expression_2013}}, et qui suggèrent des ``possibilités de fait'', laissés à l'interprète. La composition et la lutherie sont deux formes d’écriture diagrammatiques du geste et du son, qui s’inscrivaient jusqu’alors (avant le numérique) sur des médiums distincts: papier pour la composition, matériaux physiques pour la lutherie. Le numérique offre un médium commun qui permet leur recomposition mutuelle: l'instrument est ``composé''\footnote{Voir \cite{schnell_introducing_2002}.}, la partition est ``instrumentalisée''\footnote{Voir à ce sujet le travail explicite de Enrique Tomás sur les ``partitions tangibles'' \cite{tomas_tangible_2014}.}\index[people]{tomas@Tomas, Enrique!tangiblescores@\textit{Tangible Scores}} et les gestes de l'expression compositionnelle et de l'expression performative s'interpénètrent\footnote{Voir \cite{dobrian_e_2006}.}.\\
\indent Ces gestes programmés ne sont pas de simple enregistrements linéaires à reproduire tels quels, mais des modèles complexes et dynamiques\footnote{Cette idée est développée au chapitre \ref{sec:algorithms:MID}} qui invitent à l'interaction, au jeu\footnote{Cf. les propos de Stiegler dans \cite{stiegler_circuit_2004} \iquote{Quant à la duction de l'instrumentiste, elle vient retemporaliser ce qui ne peut être que spatial : le travail de la composition, ce n'est que spatial, c'est du temps spatialisé, et en cela, essentiellement en défaut d'être. C'est du virtuel pur. C'est du temps discrétisé et détemporalisé dans cette mesure. Discrétisé, il devient manipulable dans sa détemporalisation temporaire telle que la pratique le compositeur, mais il n'est que virtuel. Il ne peut devenir actuel qu'avec l'interprète, qui doit le re-temporaliser.}}, l'instrumentiste qui n'est pas, justement, un \textit{utilisateur} qui démarre, par exemple, la lecture d'un enregistrement audio. La performance musicale consiste justement à faire entendre ce qui n'est pas calculable comme le dit Stiegler: 
\blockquote{``Un musicien, c'est quelqu'un qui d'abord entend, c'est-à-dire qu'il est primordialement affecté par l'oreille, une oreille qui a cependant des yeux et des mains, et un corps qui les relie. Il ne se contente pas de calculer. Il peut calculer, il doit même calculer, mais s'il le fait, c'est pour donner à entendre ce qu'il a lui-même entendu comme l'incalculable même.''\footnote{Voir \cite{stiegler_circuit_2004}.}}
\indent Ce jeu de la main et de l'oreille, qui vient mettre en mouvement ces \textit{gestes programmés} par un geste expressif incalculable, m'amènent à introduire l'idée de geste de ``re-sonnance'', qui se départit du modèle acoustique d'excitation/modulation, en ce qu'il nourrit une relation avec le son qui n'est pas simplement causale mais intègre l'idée d'agencement compositionnel et d'aller-retour entre la temporalité du geste re-sonnant et celle intrinsèque au geste programmé.

%-------------------------------------------
\subsection{Le geste de re-sonnance}

\noindent Si les gestes programmés s'apparentent au résultat d'un processus de ``composition instrumentale''\footnote{Ce processus qui se décline sur différentes échelles temporelles est toutefois réalisables durant le temps même de la performance, en particulier dans le \textit{live-coding}.}, comment donc nommer ces gestes qui permettent de faire sonner un \gls{DMI} ? On ne peut se réduire à les nommer ``gestes d'excitation et de modulation'' (même si la métaphore qui les sous-tend peut s'appuyer sur cette idée), car de nombreux gestes n'évoquent en rien cette dimension physique, comme le déclenchement d'un échantillon ou d'une courbe d'\textit{automation}. On pourrait choisir le terme de ``gestes effecteurs'' de Delalande, mais il faudrait les coupler d'une part à un processus dynamique, et d'autre part à une métaphore définissant leur logique. Or dans le cas de l'analyse du jeu pianistique de Gould, le piano est déjà sa propre métaphore, car l'instrument acoustique \textit{est} sa propre interface et son propre modèle intermédiaire à la fois\footnote{Ce qui n'empêche pas d'autres métaphores extra-pianistiques de se superposer à l'instrument et d'en orienter les gestes effecteurs, comme le note Delalande : \iquote{(...) les différents touchers sont pour lui [Gould, NdE] l'équivalent d'une orchestration pour différencier les parties polyphoniques; ainsi le staccato joue-t-il le rôle des \textit{pizzicati} de violon et le grand \textit{legato} de la basse, celui des violoncelles. Il n'est donc pas exclu que certains gestes puissent être dictés par cette imagination orchestrale (...)}}. La notion ``d'objet sonore-gestuel'' de Rolf Inge Godøy s'approche le mieux de l'idée présentée ici, en tant qu'elle intègre cette notion de métaphore intermédiaire entre le geste et le son, mais d'une part sa nature ``d'objet'' semble plutôt s'appliquer au modèle intermédiaire lui-même qu'au geste, et d'autre part Godøy semble (malheureusement) en limiter le cadre à une relation de congruence spectromorphologique entre le geste et le son, avec le dessein d'offrir, là encore, une lisibilité de la relation énergétique.\\
%---- Figure : Guqin ---------
\begin{figure}[!t]
	\captionsetup{format=plain}%
	\includegraphics[width=\textwidth]{gfx/03_gesture/ShangyangBird.png}
	\caption[Métaphore poétique et animale dans la pédagogie du Guqin]{Un feuillet du \textit{Tayin da quanji} : les gestes instrumentaux du Guqin y sont décrits par un aphorisme poétique et illustrés par le mouvement d'un animal. Source : cf. note \ref{fn:guqin}.}
	\label{fig:gesture:guqin}
\end{figure}
%---- Figure : Guqin ---------
\indent On pourrait élargir ce cadre au-delà de la spectromorphologie en s'inspirant d'un système métaphorique un peu plus ancien et plus ouvert: le ``Tayin da quanji''\footnote{\label{fn:guqin}``L’Encyclopédie des sons suprêmes'', ouvrage ayant probablement vu le jour durant la dynastie Song, mais n'ayant survécu qu'à travers diverses éditions, une datant du \siecle{16}~siècle. Une reproduction est disponible en ligne sur le site de John Thompson, qui en a également traduit le texte en anglais: \url{http://www.silkqin.com/02qnpu/05tydq/ty3.htm}. Voir aussi \cite{picard_chine:_1991}.} décrivant la relation gestuelle-sonore pour l'apprentissage du Guqin dans un ensemble de feuillets comprenant une illustration de la position de main, associée à l'image d'un animal et d'un texte poétique condensant l'esprit du mouvement (cf. figure \ref{fig:gesture:guqin}), par exemple : \iquote{À la manière d'une grue qui danse parce qu'elle est effrayée par une brise.} Considérons maintenant un hypothétique \gls{DMI} dont le modèle intermédiaire représenterait la danse de la grue durant la dynastie Song; le geste de re-sonnance d'un tel instrument pourrait consister à jouer la brise qui effraie la grue. Ce geste pourrait alors se concrétiser de différentes manières, que cela soit en simulant les mouvements du vent par le mouvement d'un stylet sur une tablette graphique, par le mouvement des mains munies d'accéléromètres, en soufflant dans un microphone, ou encore en pinçant des cordes sur lesquelles, s'il est possible d'y évoquer la danse de la grue au Guqin, il serait sûrement possible d'évoquer le mouvement du vent.

\noindent J'utiliserai donc ici le terme de ``gestes de re-sonnance'' pour désigner un geste qui consiste à faire sonner un \gls{DMI} basé sur un ``geste programmé'', c'est-à-dire un modèle dynamique préalablement enregistré, encodé dans le \gls{DMI} et définissant son comportement. Son apparente similitude avec le terme \textit{résonance} n'est pas fortuite: le geste de re-sonnance a affaire à un processus qui possède sa propre dynamique avec lequel il doit s'accorder (ou non) et se mettre ``en résonance''.\\
\indent Le ``geste de re-sonnance'' s'appuie donc sur un \textit{modèle intermédiaire} projeté dans l'imaginaire par \textit{une métaphore} : modèle physique appelant des gestes d'excitation, modèle de lecture de vinyle appelant le \textit{scratching}, modèle de robinets-à-son\footnote{Je reprend ici l'expression utilisée par François Dumeaux\index[people]{dumeaux@Dumeaux, François} pour décrire une partie de son interface de jeu: ``(...) alors il y a des gens qui en font des choses super [en parlant de \glspl{DMI} avec des interfaces et mappings complexes, NdE], mais quelque part le côté hyper-simple de ``un fader'', moi ça me plaît plus... d'ailleurs je me retrouve encore avec un fader ici [en montrant les faders de la petite table de mixage à côté de son modulaire, NdE] voilà... une espèce de robinet à son.'' [Entretien personnel].} controlés par l'ouverture de \textit{faders} sur une table de mixage, modèle cartographique (e.g. interpolation par boules) définissant un terrain à parcourir sur une tablette, modèle de grue effrayée dansant dans la brise, etc. Le geste de re-sonnance se déploie dans un espace polarisé par cette métaphore.\\
\indent Ce modèle est rendu tangible via l'interface, mais l'interface ne définit pas l'ensemble de la métaphore : ainsi un clavier \gls{MIDI} peut servir au déclenchement de notes de piano, mais si le déclenchement des notes est contrôlée par un autre processus (e.g. une pédale d'expression, ou une foule d'agents virtuels autonomes), le clavier ne sera plus envisagé comme une surface ``de percussion'', mais comme un filtre, un crible ne laissant passer que certaines fréquences. La métaphore du piano avec ses marteaux projetés par l'enfoncement des touches disparait pour laisser place à un autre rapport sensible à l'instrument.\\
%\textbf{contextualité}
\indent À la différence de la typologie gestuelle de Claude Cadoz, à prétention universelle, le geste de re-sonnance est un geste \textit{contextuel}, dépendant de l'interface, du modèle intermédiaire, de la musique jouée, du contexte de performance. Cette contextualité n'empêche cependant pas qu'une typologie soit établie pour un contexte donné, en s'appuyant sur l'observation des pratiques qui s'y rattachent. C'est par exemple l'objet de la thèse de Baptiste Bacot\footnote{\cite{bacot_geste_2017}}: son analyse, qu'il définit comme une \iquote{organologie située}, s'appuie sur l'observation de situations concrètes de performance musicale chez différents instrumentistes numériques, et prend en compte les spécificités de leur configurations instrumentales.\\
\indent C'est aussi le cas dans le travail mené par Nathanaëlle Raboisson\index[people]{raboisson@Raboisson, Nathanaëlle} et Pierre Couprie, sur l'étude du geste performatif dans l'interprétation de musiques acousmatiques sur table de mixage\footnote{Voir \cite{raboisson_experience_2017}.}. Une analyse partant de l'observation méthodique des gestes et de leur corrélation (ou non) avec le résultat musical leur permet d'esquisser une typologie propre à cette pratique, incluant des \iquote{gestes de placement} qui participent à la (dé)construction de l'espace sonore, ou encore des \iquote{gestes d'accompagnement}\footnote{Notons que le sens est ici légèrement différent que celui des \textit{gestes accompagnateurs} définis par Delalande, mentionnés précédemment, et pourrait être rapproché des \textit{gestes de tracé sonore} de Godøy (cf. \ref{sec:gesture:imaginaire-du-geste-et-du-son}
).} caractérisés par une synchronie entre le geste et la morphologie sonore.\\
\indent De même qu'il existe un répertoire gestuel associé aux instruments acoustiques, les modèles intermédiaires définissent, de manière modulaire, un vocabulaire d'interaction qui leur est lié. Le geste de ``re-sonnance'' s'appuie donc sur les affordances et la structure musicale associées à certains modèles intermédiaires: un geste venant contrôler une boucle de batterie telle que le ``\textit{Amen Break}''\footnote{L'\textit{Amen break} est un échantillon de batterie issu du morceau ``Amen, Brother'', joué par G.C. Coleman du groupe ``The Winstons''. Ce sample figure parmi les plus utilisées dans le domaine du hip-hop, et des musiques électroniques telles que la \textit{drum and bass}. Pour un aperçu de l'importance de cette boucle, voir notamment l'installation sonore \iquote{Can I Get An Amen?} de Nate Harrison\index[people]{harisson@Harrison, Nate!canigetamen@\textit{Can I Get An Amen?}}, qui retrace son histoire (version audio accessible sur \url{https://youtu.be/5SaFTm2bcac}).} peut s'appuyer sur l'ensemble des techniques de \textit{chopping}\footnote{Dans les musiques basées sur l'utilisation du \textit{break-beat}, le \textit{chopping} consiste à découper la boucle de certaines manières en segments plus petits afin de les reconfigurer dans un ordre différent.}, \textit{cutting}, filtrage, \textit{stuttering}, \textit{varispeed}, etc. qui lui sont historiquement associées.\\
\indent Norbert Schnell développe l'idée de ``ré-animation d'enregistrements audio''\footnote{Le terme anglais \textit{reenactment} utilisé dans son travail reflète mieux que ``ré-animation'' le lien avec la théorie de l'enaction sur laquelle s'appuie notamment son travail. Voir \cite{schnell_playing_2013}.} et propose une étude très riche sur le plan théorique comme sur le plan pratique des modalités d'engagement dans une relation gestuelle/sonore par le biais d'une action métaphorique. En particulier, plusieurs exemples s'appuient sur une pièce musicale complète (la \textit{Variation Goldberg n°18} de Johann Sebastian Bach\index[people]{bach@Bach, Johann Sebastian!variationgoldberg@\textit{Variations Goldberg}}, en l'occurence) et présentent diverses stratégies possibles d'avancement dans la pièce, sur la base d'une même structure commune définie par l'œuvre de Bach.


\subsection{Résonance entre gestes re-sonnants et programmés}

\noindent Les instruments acoustiques présentent des modes de résonance que l'instrumentiste apprend à connaître, à apprivoiser, pour obtenir les qualités sonores qu'il ou elle recherche. Ces modes de résonance influent sur le timbre qui sera, par exemple sur un instrument à corde, rond et ample si l'on joue la corde au milieu de sa longueur, tandis qu'il sera plus grêle et chuintant si l'on joue \textit{sul ponticello}. La résonance de l'instrument acoustique peut également affecter d'autres paramètres, comme le rythme : un percussionniste peut mettre à profit le rebond de ses baguettes sur la peau tendue, lors d'un jeu de roulements, et adaptera le poids et/ou la tension qu'il applique sur ses baguettes en fonction de la zone qu'il frappe et de son élasticité.\\
\indent Cette adaptation dynamique du geste à la résonance de l'instrument prend d'autres dimensions encore dans des \glspl{DMI} pouvant générer de manière autonome toute une phrase musicale, tout un morceau. La nature dynamique et générative des \glspl{DMI} déplace l'agentivité\footnote{La notion d'agentivité dans la performance musicale dépasse sa simple implémentation opérante dans les \gls{IHM}. Par exemple, les figures dialogiques dans la musique classique ont été également analysée à travers le prisme de cette notion, voir notamment \cite{graybill_whose_2016}} de l'interaction instrumentale sur un terrain où elle se distribue entre des processus ``qui tournent'' et qu'il s'agit ``d'attraper''\footnote{Guerino Mazzola rapporte cette phrase du mathématicien Jean Cavaillès dans \cite{mazzola_topos_2018}: \iquote{Comprendre, c'est attraper le geste et pouvoir continuer}.} pour en jouer. Un exemple évident de cette situation est l'alignement rythmique entre deux morceaux mixés par un \gls{DJ}, ou encore l'utilisation d'un \textit{looper} permettant à un·e musicien·ne de superposer plusieurs couches musicales et devant se jouer en rythme avec ce qu'il a enregistré.\\
\indent Le \gls{DMI} peut se retrouver en position de mener le jeu et imposer sa cadence à l'instrumentiste. La performance musicale avec un \gls{DMI} est donc une co-performance où la distribution du contrôle de la synthèse et de la gestuelle qui la provoque (ou en découle), peut se définir de manière polymorphe. Une partie de la dynamique de jeu peut être prise en charge par la machine et une autre partie par l'instrumentiste dans une relation qui peut parfois s'apparenter à un duo\footnote{Cette redistribution dialogique devient explicite dans des dispositifs interactifs d'apprentissage et d'improvisation, tels qu'Omax (\cite{assayag_omax_2006}) ou le Continuator (\cite{pachet_continuator:_2003}). Voir par exemple la session d'improvisation entre György Kurtág père et fils avec le Continuator pour la pièce \textit{Zwiegespräch}: \url{https://youtu.be/pqfKGlRvddg}.\index[people]{kurtagjr@Kurtág, György, Jr.!zwiegesprach@\textit{Zwiegespräch}} \index[people]{kurtag@Kurtág, György!zwiegesprach@\textit{Zwiegespräch}}}.\\
\indent La part d'agentivité respective de la machine et de l'instrumentiste dans la production du son des \glspl{DMI} définit en fait toute une échelle de nuances, allant de la posture de la personne écoutant la musique ``malgré elle'' (dans un supermarché, typiquement) à celle du ou de la musicien·ne engageant tout son corps (physique et mental) dans l'interaction musicale. Cette recherche de la résonance avec l'instrument, qui vise à en comprendre l'organisation des sons et les rythmes propres, à en attraper les gestes programmés, à y trouver les \textit{sweet-spots} est peut-être, davantage que le médium constitutif de l'instrument, ce qui définit vraiment son instrumentalité. La possibilité des \glspl{DMI} de pouvoir générer du son en continu sans qu'aucun effort physique ne soit nécessaire pose en effet la question de l'engagement des musicien·ne·s de manière plus critique encore qu'avec les instruments acoustiques car, à l'inverse, \textit{ne pas jouer} peut requérir un effort physique.\\
\indent Les gestes de l'instrumentiste peuvent alors entretenir diverses relations expressives ``en phase'' avec le mouvement autonome de l'instrument : dialogique, d'accompagnement, d'accentuation, d'infléchissement, d'opposition... mais il peut également s'affranchir d'une relation lisible au profit d'une subversivité expressive.

%%%%%%%%%%%%%%%% FIN de cette section ? %%%%%%%%

%%%%%%%%%%%%%%%%%%%%%%%%%%%%%%%%%%%%%%%%%%%%%%%%%%%%%%%%%%%%%%%%%%%%
%%%%%%%%%%%%%%%%%%%%%%%%%%%%%%%%%%%%%%%%%%%%%%%%%%%%%%%%%%%%%%%%%%%%


\section{Subversion sonore, subversion gestuelle}
\label{sec:gesture:subversion}

\subsection{Geste produit, capté, perçu}

\noindent Dans le domaine des \gls{IHM}, peu d'importance était accordée au geste en dehors de son interaction directe avec la machine jusqu'à la fin des années 1990, comme le dénote l'analyse de Gordon Kurtenbach dans l'ouvrage \textit{The art of human-computer interface design} : \iquote{Un geste est un mouvement du corps qui contient de l'information. Faire un signe d'au-revoir est un geste. Appuyer sur la touche d'un clavier n'est pas un geste car le mouvement d'un doigt dans sa course pour frapper une touche n'est ni observé ni signifiant. Tout ce qui importe est quelle touche est enfoncée.}\footnote{\iquote{A gesture is a motion of the body that contains information. Waving goodbye is a gesture. Pressing a key on a keyboard is not a gesture because the motion of a finger on its way to hitting a key is neither observed nor significant. All that matters is which key was pressed.} \cite{kurtenbach_gestures_1990}}. \\
\indent Pourtant, comme le rappelait à la même époque Richard Leppert, la nature intangible du son et de la musique est polarisée par l'expérience visuelle: \iquote{Précisément parce que le son musical est abstrait, intangible et immatériel (...) que l'expérience visuelle de sa production est cruciale, tant pour les musicien·ne·s que pour le public, pour situer et communiquer la place de la musique et du son musical dans la société et la culture. (...) La musique, malgré son immatérialité sonore phénoménologique, est une pratique incarnée, comme la danse et le théâtre.}\footnote{\iquote{Precisely because musical sound is abstract, intangible, and ethereal [...] the visual experience of its production is crucial to both musicians and audience alike for locating and communicating the place of music and musical sound within society and culture. [...] Music, despite its phenomenological sonoric ethereality, is an embodied practice, like dance and theater.} \cite{leppert_sight_1993}. Une assertion confirmée expérimentalement par de nombreuses études perceptives, par exemple \cite{tsay_sight_2013}.}\\
%TODO - BOF : \textbf{Différence entre geste effectué et geste capté} : un son électroacoustique, par exemple une figure de ``delta'', pourra ainsi, avec un même geste de balaiement de la main, et une même interface de captation (e.g. un simple \textit{slider}), être engendré via un seuillage sur la valeur du \textit{slider} qui déclenche un échantillon, être produit de manière continue, en lisant l'échantillon comme si l'on déplaçait une tête de lecture, pourra avoir son intensité sonore fonction de la vitesse du geste ou pas, etc. On voit que pour un même geste, un même son, et un même capteur, les possibilités de relations entre le geste et le son sont multiples.\\
%------------------ Figure : Geste produit, capté, perçu ---------------------
\begin{figure}[!ht]
	\captionsetup{format=plain}%
	  \includegraphics[width=\linewidth]{gfx/03_gesture/Benford_expected-sensed-desired.pdf}
		\caption[Gestes attendus, désirés, et captés sur une interface]{Recoupement partiel entre les gestes attendus sur une interface, les gestes effectivement captés et les gestes désirés par l'instrumentiste, d'après Steve Benford \cite{benford_performing_2010}.}
		\label{fig:Benford_expected-sensed-desired}
\end{figure}
%------------------ Figure : Geste produit, capté, perçu ---------------------
\indent Nous avons vu dans les sections précédentes différents aspects concernant la fonctionnalité du geste, considérée du point de vue de l'instrumentiste. Ce point de vue privilégie généralement la recherche d'une relation lisible et cohérente entre le geste et le son, se situant à cette intersection où les actions permises par la configuration physique de l'agencement instrumental, celles effectivement captées par l'interface et celles désirées par l'instrumentiste se correspondent, ce que Steve Benford représente par le diagramme de Venn de la figure \ref{fig:Benford_expected-sensed-desired}. Ce diagramme fournit une représentation intéressante des compromis qui se présentent aux luthièr·e·s numériques pour la définition des gestes d'un \gls{DMI}. Benford remarque que si les zones où ces cercles ne se recoupent que partiellement peuvent apparaitre comme problématiques (interface pas assez sensible, gestes captés mais non-désirés), elles sont aussi des espaces permettant d'inventer de nouvelles relations, en particulier quand des gestes désirés qui peuvent être captés ne sont pas des gestes \textit{a priori} attendus.\\
\indent En se plaçant dans la perspective de l'instrumentiste, ce diagramme ne met pas en évidence la relation entre les gestes de l'instrumentiste et la traduction sonore de ces gestes par l'instrument, ni la perception de ce résultat par le spectateur-auditeur. Or, la compréhension du mapping entre gestes captés et production de l'instrument joue un rôle essentiel dans l'attribution de l'agentivité. En particulier, la représentation visuelle du fonctionnement de l'instrument peut aider à cette compréhension comme l'ont montré Florent Berthaut et collaborateurs dans \cite{berthaut_rouages:_2013, berthaut_liveness_2015}. Benford propose un autre diagramme pour prendre en compte ces aspects que nous analyserons plus loin, mais il nous faut d'abord éclaircir un point sujet à différentes orientations et partis pris.

\subsection{Une critique de la transparence}
\label{sec:gesture:critique_transparency}

\indent Dans son essai ``\textit{Theses on Liveness}''\footnote{\cite{croft_theses_2007}.}, John Croft opère une distinction intéressante entre ce qu'il appelle ``expressivité procédurale'' (\textit{procedural liveness}), définie par la manipulation concrète du son en temps-réel, et ``expressivité esthétique'' (\textit{aesthetic liveness})\footnote{La traduction du terme anglais \iquote{liveness} par le terme ``expressivité'' ne rend pas tout à fait compte de la connotation de ``vivacité'' qu'il contient et dénote la confusion et/ou l'intrication de cette notion avec celles d'\textit{agentivité} et de \textit{présence} directe.} définie par la traduction des valeurs d'entrée en des variations esthétiques significatives. Cependant, Croft écarte immédiatement l'idée de considérer cette ``expressivité esthétique'' dans ce qu'elle peut contribuer à façonner l'interaction instrumentale, considérant que cette expressivité nécessaire à la musique peut tout aussi bien s'appuyer sur des sons pré-enregistés. Ce point de vue l'amène à une conclusion similaire à celles de Fels, Dobrian ou Cadoz\footnote{Voir par exemple les remarques de Sydney Fels \cite{fels_mapping_2002} \iquote{We consider transparency as a predictor for expressivity. (...) We identify transparency as a quality of a mapping.}, Christopher Dobrian \cite{dobrian_e_2006} \iquote{Another basic need is that the software provide correspondences between input data and output sound that are sufficiently intuitive for both performer and audience.}}, fondée sur une stabilité de l'instrument et un rapport de causalité, de cohérence et de proportionnalité entre le geste et le son.\\
\indent La poursuite exclusive d'une telle ``transparence instrumentale'', comme condition nécessaire à l'expressivité d'un instrument, pourrait masquer le fait que la relation entre l'instrumentiste et son public n'est pas exclusivement, sinon principalement, motivée par la recherche de cette transparence, comme l'ont notamment montré Cavan Fyans, Michael Gurevitch et Paul Stapleton dans une série d'articles\footnote{Cf. \cite{fyans_where_2009, fyans_examining_2010, gurevich_digital_2011}}. Il mettent notamment en évidence que la confiance apparente de l'instrumentiste joue un rôle prépondérant dans l'évaluation que fait le spectateur de la maîtrise de l'instrumentiste et, de manière plus générale, que les modalités d'engagement des spectateurs dans l'appréciation et la compréhension d'une performance sont multiples et contextuelles.\\
\indent On arguera même ici du contraire: la subversion\footnote{Le terme ``subversion'' (du latin \textit{subvertere} : renverser, bouleverser) désigne \iquote{l'action visant à saper les valeurs et les institutions établies} (dictionnaire Larousse). Les moyens employés par la subversion consiste à diffuser un message contraire à un l'ordre établi, dans le but d'affaiblir celui-ci. Dans le cas de la musique, si la notion de subversion peut prendre une dimension culturelle ou politique dans certains courants musicaux, c'est ici au sens cognitif de la subversion de la perception que j'emploie ce terme. Par exemple, on pourra dire qu'un groupe de musique pop interprétant un tube en playback sur un plateau de télévision use d'un principe subversif, en laissant croire que ce que l'on écoute est le résultat de leur interprétation en direct (valeur établie), alors qu'il s'agit d'un enregistrement. } est une composante essentielle de la création artistique, qui se manifeste déjà dans le cas des instruments acoustiques. Leur design est en effet orienté par l'obtention d'une homogénéité de timbre tout à fait artificielle, autant que la discrétisation du \textit{continuum} des hauteurs en notes. De même, les gestes de l'instrumentiste et de la compositrice créent des continuités là où il y a discontinuité (e.g. le \textit{legato} sur des notes de piano, la modulation tonale pour passer d'un mode à un autre, la fusion des timbres dans le travail d'orchestration, etc.), et inversement introduit des ruptures pour déjouer les attentes du public (e.g. les jeux de contre-temps rythmiques, le \textit{subito-piano}, le montage \textit{cut}, etc.). La musique est un art du mirage, de l'illusion --~comme le rappelle Risset\index[people]{risset@Risset, Jean-Claude} dans la citation en exergue de ce chapitre~--, dans lequel la transparence n'est qu'un alibi de la subversion, pour dessiner de nouvelles relations du sensible.\\
\indent L'attribution de qualités expressives aux \glspl{DMI} sur la base de la perception explicite de causalité entre ce que fait l'instrumentiste et le son produit, traduit ainsi une référence implicite à un modèle instrumental acoustique qui n'est qu'à moitié justifiée. Loin de prétendre ici que le modèle causal de l'instrument acoustique soit obsolète, il semble nécessaire de s'en détacher si l'on veut saisir l'étendue réelle des possibilités créatives offertes par les \glspl{DMI}. L'expressivité ne saurait être réduite à la simple traduction sonore immédiate des gesticulations d'un·e instrumentiste. Les films de la Nouvelle Vague ont suffisamment montré que la collision de plans sonores et visuels n'entretenant \textit{a priori} aucun rapport de causalité ou de synchronicité, sont tout à fait capables de susciter une émotion, de traduire une pensée, ou d'exprimer ce qui, précisément, ne peut être exprimé qu'en sortant d'un rapport de causalité ou de cohérence, comme le décrit Gilles Deleuze dans ``L'image-temps'': 
\begin{quotation}
(...)~le visuel et le sonore ne reconstituent pas un tout, mais entrent dans un rapport ``irrationnel'' suivant deux trajectoires dissymétriques. L'image audio-visuelle n'est pas un tout, c'est une ``fusion de la déchirure''.\footnote{\cite[p.~351]{deleuze_image-temps_1985}.}
\end{quotation}
\noindent C'est à partir de cette déchirure qu'il devient possible de faire jaillir quelque chose d'inouï, qui se cachait dans la fêlure des normes, comme le fait le ``Coup de dés'' de Mallarmé\index[people]{mallarme@Mallarmé, Stéphane!coup@\textit{Un coup de dés n'abolira jamais le hasard}}\footnote{``Un coup de dés n'abolira jamais le hasard'' de Stéphane Mallarmé, \cite{mallarme_coup_1914}.} qui s'affranchit de la syntaxe habituelle pour inventer la sienne propre, ou que rappelle encore ce refrain de Leonard Cohen\index[people]{cohen@Cohen, Leonard!anthem@\textit{Anthem}} dans \textit{Anthem}:\iquote{Ring the bells that still can ring / Forget your perfect offering / There is a crack, a crack in everything / That's how the light gets in}.\\
\indent Les \glspl{DMI} offrent de nouvelles possibilités de subversion de la perception --~comme autant de fêlures du modèle acoustique~-- qui peuvent s'appuyer sur les caractéristiques qui leur sont propres. Ainsi, le découplage énergétique permet des correspondances inattendues entre force du geste et intensité du son, les capacités d'enregistrement permettent de faire surgir des matériaux de toutes natures sonores, l'absence potentielle de contact avec l'instrument permet d'associer plus librement gestes effecteurs et gestes d'accompagnement, et la reconfiguration dynamique des algorithmes permet d'envisager une métamorphose continue des relations de jeu.

\subsection{Stratégies expressives d'interaction gestuelle}

\noindent Dans une analyse des stratégies de design d'interface en fonction de leur perception par le spectateur\footnote{\cite{reeves_designing_2005}.}, Stuart Reeves, Steve Benford et al. distinguent quatre orientations possibles en fonction du degré de visibilité des gestes effecteurs de l'instrumentiste par rapport au degré de perceptibilité (visible, audible) de l'effet produit par l'instrument (cf. figure \ref{fig:gesture:Benford}):
\vspace{-1em}
\begin{itemize}[noitemsep]
 	\item \textbf{expressive}: lorsque les gestes de manipulation et les effets sont perceptibles, l'interface est dite \textit{expressive} (terme que nous trouvons mal choisi, et reformulerons plus loin); 
 	\item \textbf{secrète}: à l'opposé, une interface sur laquelle les gestes effecteurs et leur effet sont invisibles est jugée ``secrète'' --~Benford donne comme exemple le cas d'une performance qui ne serait accessible qu'à une partie du public. Ce pourrait être aussi le cas par exemple lors de gestes discrets visant à rappeler des réglages sur l'instrument et dont l'effet n'est pas directement perceptible; 
 	\item \textbf{à suspens}: si des gestes effecteurs sont visibles et manifestement destinés à produire un effet mais qu'aucun effet n'est perçu, cela pourrait être considéré par le public comme un dysfonctionnement, mais également (et c'est l'orientation proposée ici par Benford) comme un moment de \textit{suspens}, si le public a conscience que l'action de ces gestes lui seront perceptibles plus tard. Benford prend l'exemple d'un public faisant la queue devant une installation dont il ne peuvent faire l'expérience que lorsque leur tour arrive. On pourrait aussi observer une telle situation dans des gestes de live-coding, où le code s'écrit de manière visible pour le public\footnote{Une pratique courante dans le live-coding consiste à montrer son écran au public, généralement sous la forme d'une vidéoprojection du code composé en direct (cf. manifeste TOPLAP : \url{https://toplap.org/wiki/ManifestoDraft})}, dont l'activation est imminente, mais dont on ne perçoit le résultat qu'une fois le code envoyé dans le \gls{DSP};
 	\item \textbf{magique}: enfin, les gestes effecteurs peuvent être cachés tandis qu'un effet est produit par l'instrument, configuration que les auteurs appellent ``magique''. Ce cas de figure est si courant dans la musique électronique qu'on pourrait se demander ce terme reste approprié~--les spectateurs ayant depuis longtemps déjà intégré le fait qu'une machine puisse produire des sons sans qu'un geste en soit la cause directe.
\end{itemize}
%-----------------
% %------------------ Figure : geste lisible ou subversif ---------------------
% \begin{figure}[!htbp]
% 	\captionsetup{format=plain}%
% 	\centering
% 	\begin{minipage}[t]{0.9\textwidth}
% 		\includegraphics[width=\linewidth]{gfx/03_gesture/ManipulationVsEffect2.pdf}
% 		\caption[Stratégies de design d'interfaces pour le spectateur (schéma de Benford)]{Stratégies de design d'interfaces pour le spectateur, d'après \cite{reeves_designing_2005, benford_performing_2010}}
% 		\label{fig:gesture:Benford}
% 	\end{minipage}
% \end{figure}

% %-------------------------- Figure : transparence Fels -----------------------
% \begin{wrapfigure}[14]{R}{0.5\textwidth}
% 	\begin{center}
%  		\includegraphics[width=0.48\textwidth]{gfx/03_gesture/Fels-transparency.pdf}
% 	\end{center}
% 	\caption{Transparence pour le musicien et l'auditoire, d'après \cite{fels_mapping_2002}}
% 	\label{fig:gesture:fels_transparency}
% \end{wrapfigure}
% %-------------------------- Figure : transparence Fels -----------------------
\begin{figure}[!htbp]
	\captionsetup{format=plain}%
	\centering
	\begin{minipage}[t]{0.48\textwidth}
		\includegraphics[width=0.9\linewidth]{gfx/03_gesture/Fels-transparency.pdf}
		\caption[Transparence pour l'instrumentiste et l'auditoire (Fels)]{Transparence pour l'instrumentiste et l'auditoire, d'après \cite{fels_mapping_2002}}
		\label{fig:gesture:fels_transparency}
	\end{minipage}
	\hspace{.02\linewidth}
	\begin{minipage}[t]{0.48\textwidth}
		\includegraphics[width=1.1\linewidth]{gfx/03_gesture/ManipulationVsEffect2.pdf}
		\caption[Stratégies de design d'interfaces pour le spectateur (Benford)]{Stratégies de design d'interfaces pour le spectateur, d'après \cite{reeves_designing_2005, benford_performing_2010}}
		\label{fig:gesture:Benford}
	\end{minipage}
\end{figure}
%------------------ Figure : geste lisible ou subversif ---------------------
\noindent Ce schéma, à rapprocher de celui déjà proposé par Sydney Fels, Ashley Gadd et Axel Mulder trois ans plus tôt (figure \ref{fig:gesture:fels_transparency}), présente l'intérêt de dépasser une vision uni-dimensionnelle de l'expressivité instrumentale, envisagée comme le seul résultat du degré d'agentivité de l'instrumentiste perceptible par l'auditeur/spectateur, en proposant une caractérisation de la relation geste/son prenant en compte l'absence potentielle (et potentiellement souhaitée) de congruence entre les gestes effecteurs de l'instrumentiste et leur traduction sonore perçue par le public.\\
\indent On pourrait cependant ajouter une dimension à ce schéma en y insérant le paramètre de cohérence de la relation entre les gestes effecteurs et l'effet produit\footnote{À cet égard, le schéma de Benford semble se placer dans le cas particulier où cette relation  \underline{\textbf{est}} cohérente.}. Cette cohérence peut être obtenue par la congruence énergético/spatio/temporelle du geste et du son\footnote{Telle que préconisée par Cadoz (Cf. supra p.\pageref{sec:gesture:ergotic}.), ou John Croft dans \cite{croft_theses_2007}}, mais de manière plus générale, elle se caractérise par la capacité des spectateurs-auditeurs à comprendre cette relation, qu'elle soit congruente ou non\footnote{Ce qui semble être démontré par l'étude de Berthaut et al. dans \cite{berthaut_rouages:_2013}}. Pour conserver les mêmes unités d'axe, je propose d'assimiler ce paramètre de \textit{cohérence} au degré de ``visibilité des gestes programmés'', qu'il faut ici entendre comme une compréhension de la mécanique interne de l'instrument, de son mapping. On obtient ainsi un espace tri-dimensionnel dans lequel d'autres axes d'interprétation de la relation expressive se dessinent (cf. figure \ref{fig:gesture:expressive-space}).\\
\begin{figure}[H]
	\captionsetup{format=plain}%
	\centering
	\begin{minipage}[t]{0.8\textwidth}
		\includegraphics[width=\linewidth]{gfx/03_gesture/SubversiveCube.pdf}
		\caption[Stratégies de design d'interaction instrumentale pour le spectateur]{Stratégies de design d'interaction instrumentale pour le spectateur en fonction de la visibilité des \textit{gestes re-sonnants} de l'instrumentiste, des \textit{gestes programmés} dans l'instrument et des effets sur le son.}
		\label{fig:gesture:expressive-space}
	\end{minipage}
\end{figure}
%---------------
\noindent Ainsi, le schéma que je propose ici se fonde sur celui de Benford, en reformulant certains de ses axes et en l'étendant selon l'axe de la visibilité du \textit{geste programmé}:
\vspace{-1em}
\begin{itemize}[noitemsep]
	\item \textbf{transparente} : correspondant à la relation que Benford nomme \textit{expressive}, en indiquant ici que la relation est comprise (du moins en apparence) par le spectateur. Ce renommage nous permet d'éviter un abus de language qui empêcherait de considérer d'autres formes d'expressivité que celle passant par la transparence de la relation instrumentale;
	\item \textbf{autonome} : de même, la zone désignée comme \iquote{magique} dans le schéma de Benford semble davantage correspondre à une situation \iquote{d'autonomie} de l'instrument. En effet, la notion de geste magique\footnote{Les principes du ``geste magique'' ont été analysés par de nombreux anthropologues, notamment par Marcel Mauss dans \cite{mauss_esquisse_1902}, et également dans le cadre des \gls{IHM} par \cite{lokuge_dynamic_1995, marshall_deception_2010}.} implique d'autres variables que celles de l'occultation et de la révélation, en particulier l'inscription dans une temporalité permettant de créer un contexte favorable à la subversion (par la narration, le détournement d'attention, etc.). La notion d'autonomie, elle, se traduit par exemple dans le rôle pris par une platine vinyle jouant un enregistrement en l'absence de geste venant perturber sa lecture;
	\item \textbf{cryptique} : une relation sera vue comme cryptique si les gestes re-sonnants et le résultat sonore ne présentent aucune relation apparente. C'est un état perceptif très instable car nous sommes naturellement happés par une recherche de sens dont il est difficile de s'affranchir. Cette situation \textit{cryptique} dévie donc rapidement vers les autres pôles;
	\item \textbf{externe} : dans le cas où des effets sonores sont perceptibles sans qu'aucune action ne semble les produire, ni celles de l'instrumentiste, ni celles provoquées de manière autonome par l'instrument, le son sera perçu comme externe à l'instrument. Cette situation est typiquement renforcée dans le cas des instruments électroniques par la distance fréquente entre l'instrument et les haut-parleurs qui supprime cette congruence de localisation entre l'instrument et le son, naturelle dans les instruments acoustiques. La présence habituelle d'un espace scénique mis en lumière focalise bien évidemment l'attention sur l'instrument[iste], mais ce point de fuite de l'attention peut précisément être déplacé par ces mêmes procédés (e.g. hors de la scène);
	\item \textbf{déconnectée} : un·e instrumentiste procédant à des mouvements sur un instrument muet et dont le public ne peut pas comprendre le fonctionnement renvoie l'image d'un instrument déconnecté. Cette situation se rencontre par exemple lorsqu'un·e musicien·ne numérique procède à des réglages en amont de sa performance, entre deux morceaux musicaux ou lorsque, effectivement, l'instrument ne répond pas (e.g. bug informatique, câble débranché, ...);
	\item \textbf{potentielle} : enfin la seule présence de l'instrument, sans qu'aucun geste ne lui soit adressé, sans qu'il ne produise aucun son, ni que son fonctionnement soit compréhensible peut s'apparenter à la \textit{rencontre} d'un instrument inconnu, ou d'un objet dont nous envisageons l'instrumentalité. Que nous portions intentionnellement notre attention sur cet objet, où que notre attention soit guidée sur elle (e.g. par sa mise en lumière dans un espace dédié tel qu'une scène ou une galerie d'art), il subsiste le mouvement de notre imagination qui projette les affordances, le jeu, les sonorités potentielles. La pièce \textit{4'33''} de John Cage\index[people]{cage@Cage, John!aaa@\textit{4\textquotesingle33\textquotedbl}} ou encore l'installation \textit{Handphone table} de Laurie Anderson\index[people]{anderson@Anderson, Laurie!handphonetable@\textit{Handphone table}} (cf. figure \ref{fig:gesture:HandphoneTable}) se situent à la limite de ce cas de figure.
\end{itemize}
%------------------ Figure : geste lisible ou subversif ---------------------
\begin{figure}[!htbp]
	\captionsetup{format=plain}%
	%\includegraphics[width=\linewidth]{gfx/03_gesture/LaurieAnderson-HandphoneTable-PhotoLuisaRibas.jpg}
	\includegraphics[width=\linewidth]{gfx/03_gesture/laurie-anderson-the-handphone-table.jpg}
	%\caption[\textit{Handphone table} de Laurie Anderson]{\textit{Handphone table} de Laurie Anderson. En posant ses coudes sur de petites zones vibrotactile et ses mains sur ses oreilles, le public peut écouter, par conduction osseuse, la musique composée par Laurie Anderson. Photographie: Luisa Ribas}
	\caption[\textit{Handphone table} de Laurie Anderson]{\textit{Handphone table} de Laurie Anderson. En posant ses coudes sur de petites zones vibrotactiles et ses mains sur ses oreilles, le public peut écouter la musique composée par Laurie Anderson par conduction osseuse. Coll. MAC de Lyon. Photographie: Blaise Adilon.}
	\label{fig:gesture:HandphoneTable}
\end{figure}
%---------------
\indent Il faudrait rajouter à ce schéma une variable permettant de définir si le comportement de l'instrument est désiré ou subi, en particulier lorsque ses modes de jeu n'ont pas la ``transparence'' attendue sur ledit instrument. En présentant ces différents rôles comme des stratégies plutôt que seulement des problèmes, nous prenons ici le parti de les considérer comme des modes de fonctionnement souhaitables, ne serait-ce que temporairement, par l'instrumentiste.\\
\indent Dans la mesure où ils seraient subis, on pourrait assimiler ces comportements à des ``fausses notes''. Mais il nous faudrait ici préciser que la ``fausse note'', dont la possibilité même est utile aux musiques qui les proscivent --~car elle ajoute une tension évidente, la même que celle qui tient le funambule sur un fil~--, fait également partie des fissures évoquées plus haut, qui laissent entrer la lumière. Le jazz l'a tôt compris et ses improvisateurs déroulent en permanence leur fil de funambule, en les domptant dans la dynamique même du jeu,
%dont le musicien peut tout à fait s'accommoder, en les réutilisant dans la dynamique du jeu, 
selon le désormais célèbre adage répété par nombre de musicien·ne·s de jazz, dont Miles Davis, qu'il n'y a pas de fausses notes en tant que telles; ce sont les notes jouées ensuite qui les rendent justes ou fausses\footnote{Herbie Hancock\index[people]{hancock@Hancock, Herbie} en rend compte dans une interview, en notant le caractère ``magique'' de la résolution trouvée par Miles Davis\index[people]{davis@Davis, Miles} : \iquote{Au milieu d'une des chansons, pendant le solo de Miles, j'ai joué cet accord... qui était tellement faux! Je pensais que j'avais tout détruit et réduit cette formidable soirée en miettes. Miles a repris son souffle et il a joué quelques notes, et il a récupéré mon accord. Je n'ai pas compris comment il a fait ça, mais ça sonnait bien comme par magie.} cf. \url{https://youtu.be/C-GrRIgdmW8}}. Peut-être aurait-on dû d'emblée les nommer ``notes imprévues'', en s'épargnant un qualificatif bourgeois qui ferme les écoutilles à celles-ci.\\
\indent L'assertion de Mile Davis (et d'autres) sur le caractère de ``justesse'' d'une note\footnote{Le terme de justesse est bien évidemment utilisé ici au sens de sa pertinence, son intérêt dans la grille d'accords du morceau, et non pas de la précision de sa hauteur. } dépendant de ce qui est joué à sa suite fait apparaître deux autres lacunes importantes de ce schéma : sa dimension temporelle d'une part et la connaissance préalable des idiomes musicaux et instrumentaux sur lesquels se construit à la fois la performance et la construction de sens de la part du spectateur. Ainsi, on ne peut subvertir la perception de l'auditeur qu'à la condition qu'il y ait un système de valeur à déconstruire. Ce système peut exister préalablement à la performance (e.g. ``un piano produit un son de piano accordé selon une échelle chromatique tempérée''), mais aussi se construire durant le temps même de la performance, en installant une relation qui semble cohérente et stable, au point de sembler établie au moment de la mettre en défaut.\\
%\indent Il me faut préciser ici une dimension esthétique sous-jacente à cette entreprise: l'intérêt de subvertir le système des valeurs sur lesquelles la musique s'appuie ne se borne pas à une duperie du spectateur-auditeur, qui se trouverait victime d'une farce, mais bien de permettre l'extrapolation du réel dans un espace imaginaire, en y ouvrant une brèche qui permet en retour de le rendre tangible.

%-------------------------------------------
\subsection{Inférences perceptives et soudure subversive}
\label{sec:gesture:inferences-soudure}

\subsubsection{Inférences entre perception du geste, de l'instrument, du son}
\label{sec:gesture:inferences-soudure:inferences}

\noindent De la même manière que l'écriture musicale a permis l'émergence de processus compositionnels difficilement concevables sans ce support visuel\footnote{On pense ici à des procédés tels que la fugue ou la rétrogradation, telle que dans le premier ``canon à l'écrevisse'' de \textit{L'Offrande Musicale} de Jean-Sébastien Bach.}, les ordinateurs ont permis de créer des formes sonores qu'il aurait été difficile de concevoir sans cet outil computationnel, tels que les sons paradoxaux de Risset\index[people]{risset@Risset, Jean-Claude} et Shepard\footnote{Ce qui ne signifie pas que les sons paradoxaux soient impossibles à produire sans recourir à l'ordinateur; cf. leur interprétation vocale par Victoria Hart\index[people]{hart@Hart, Victoria!shepardtones@\textit{Shepard Tones}}: \url{https://vimeo.com/147403169}.}. Les nouvelles possibilités d'écriture du geste permettent également d'imaginer des \textit{gestes paradoxaux}, notamment par la redéfinition dynamique de leur traduction sonore\footnote{Voir par exemple l'étude de François-Xavier Féron sur les ``rythmes paradoxaux'' \cite{feron_lart_2010}.}.\\
\indent L'agentivité de l'instrumentiste, de l'instrument et du son permet d'envisager tous les cas de figures d'influence réciproque entre ces agents : le geste de l'instrumentiste peut autant produire des sons directement que par l'intermédiaire de l'instrument, et inversement, le son et l'instrument peuvent influencer son geste. De même, l'instrument peut produire des sons (de manière autonome) autant que le son peut influencer le comportement de l'instrument (s'il est par exemple capté par microphone). L'absence de relation fixe et prévisible dans l'interaction gestuelle avec les \glspl{DMI} entraîne donc l'audit·eur·rice/spectat·eur·rice, dont la cognition est tendue vers la compréhension de ce qu'il observe, dans un réseau d'inférences entre ces différents agents que sont l'instrumentiste (son expression gestuelle), l'instrument, et le son perçu.\\ \todo{mal dit}
\indent Construisons de manière systématique un schéma de ces relations d'inférence (figure \ref{fig:gesture:Inferences}) et examinons l'ensemble des inférences possibles et à quelles situations réelles elles correspondent quand elles sont subverties\footnote{Par souci de concision, j'ai noté ``inférence des gestes à partir du son'' (et ainsi de suite), mais il faut évidemment comprendre ``inférence des gestes \textit{(construite)} à partir \textit{(de notre expérience)} du son''.}.
% --------------- tableau des inférences---------------
\begin{figure}[!htbp]
	\captionsetup{format=plain}%
	\makebox[\linewidth][c]{%
		\begin{subfigure}[b]{.33\textwidth}
			\centering
			\includegraphics[width=0.9\linewidth]{gfx/03_gesture/gesture-inference-legendeL.pdf}
		\end{subfigure}%
		\hspace{.02\linewidth}
		\begin{subfigure}[b]{.33\textwidth}
			\centering
			\includegraphics[width=0.9\linewidth]{gfx/03_gesture/gesture-inference-m.pdf}
			\caption{Gestes/son/instrument vers soi-même}
		\end{subfigure}%
		\hspace{.02\linewidth}
		\begin{subfigure}[b]{.33\textwidth}
			\centering
			\includegraphics[width=0.9\linewidth]{gfx/03_gesture/gesture-inference-legendeR.pdf}
		\end{subfigure}%
	}\\
	\makebox[\linewidth][c]{%
		\begin{subfigure}[b]{.33\textwidth}
			\centering
			\includegraphics[width=0.9\linewidth]{gfx/03_gesture/gesture-inference-a.pdf}
			\caption{Son vers gestes}
		\end{subfigure}%
		\hspace{.02\linewidth}
		\begin{subfigure}[b]{.33\textwidth}
			\centering
			\includegraphics[width=0.9\linewidth]{gfx/03_gesture/gesture-inference-b.pdf}
			\caption{Instrument vers gestes}
		\end{subfigure}%
		\hspace{.02\linewidth}
		\begin{subfigure}[b]{.33\textwidth}
			\centering
			\includegraphics[width=0.9\linewidth]{gfx/03_gesture/gesture-inference-c.pdf}
			\caption{Gestes vers gestes}
		\end{subfigure}%
	}\\
	\makebox[\linewidth][c]{%
		\begin{subfigure}[b]{.33\textwidth}
			\centering
			\includegraphics[width=0.9\linewidth]{gfx/03_gesture/gesture-inference-d.pdf}
			\caption{Gestes vers son}
		\end{subfigure}%
		\hspace{.02\linewidth}
		\begin{subfigure}[b]{.33\textwidth}
			\centering
			\includegraphics[width=0.9\linewidth]{gfx/03_gesture/gesture-inference-e.pdf}
			\caption{Instrument vers son}
		\end{subfigure}%
		\hspace{.02\linewidth}
		\begin{subfigure}[b]{.33\textwidth}
			\centering
			\includegraphics[width=0.9\linewidth]{gfx/03_gesture/gesture-inference-f.pdf}
			\caption{Son vers son}
		\end{subfigure}%
	}\\
	\makebox[\linewidth][c]{%
		\begin{subfigure}[b]{.33\textwidth}
			\centering
			\includegraphics[width=0.9\linewidth]{gfx/03_gesture/gesture-inference-g.pdf}
			\caption{Gestes vers instrument}
		\end{subfigure}%
		\hspace{.02\linewidth}
		\begin{subfigure}[b]{.33\textwidth}
			\centering
			\includegraphics[width=0.9\linewidth]{gfx/03_gesture/gesture-inference-h.pdf}
			\caption{Son vers instrument}
		\end{subfigure}%
		\hspace{.02\linewidth}
		\begin{subfigure}[b]{.33\textwidth}
			\centering
			\includegraphics[width=0.9\linewidth]{gfx/03_gesture/gesture-inference-i.pdf}
			\caption{Instrument vers instrument}
		\end{subfigure}%
	}\\
	\makebox[\linewidth][c]{%
		\begin{subfigure}[b]{.33\textwidth}
			\centering
			\includegraphics[width=0.9\linewidth]{gfx/03_gesture/gesture-inference-j.pdf}
			\caption{Gestes/son vers instrument}
		\end{subfigure}%
		\hspace{.02\linewidth}
		\begin{subfigure}[b]{.33\textwidth}
			\centering
			\includegraphics[width=0.9\linewidth]{gfx/03_gesture/gesture-inference-k.pdf}
			\caption{Gestes/instrument vers son}
		\end{subfigure}%
		\hspace{.02\linewidth}
		\begin{subfigure}[b]{.33\textwidth}
			\centering
			\includegraphics[width=0.9\linewidth]{gfx/03_gesture/gesture-inference-l.pdf}
			\caption{Instrument/son vers gestes}
		\end{subfigure}%
	}
	\caption[Inférences entre gestes, instrument, son et leur perception.]{Inférences croisées de la perception des gestes (G), du fonctionnement de l'instrument (I), du son (S), et de la perception de l'auditeur-spectateur (A).}
	\label{fig:gesture:Inferences}
\end{figure}
% --------------- tableau des inférences---------------

\vspace{-1em}
\begin{itemize}[noitemsep]
		\item \textbf{a) Auto-inférence à partir de la cohérence gestes/son/instrument}\\
	Tout d'abord, la situation dans laquelle les gestes de l'instrumentiste et le son produit sur un instrument donné apparaissent cohérents se pose comme la situation la plus ``habituelle'' lors de l'écoute d'un instrument acoustique traditionnel, ou plutôt, comme une sorte de socle sous-jacent, offrant une certaine stabilité, à partir duquel des variations peuvent être (en)tendues. 
	Cette cohérence nous permet de nous situer dans l'espace et le temps: j'entends différemment selon que je suis proche ou loin de l'instrument, à sa gauche ou à sa droite; je l'entends plus fort si l'instrumentiste déploie plus d'énergie, j'entends une phrase musicale familière, etc. Notre perception travaille en permanence à \textit{reconnaître} la musique, c'est à dire à saisir cette cohérence, qui joue un rôle essentiel dans notre capacité à anticiper ce qui arrive ensuite. L'analyse des processus cognitifs liés à la perception de la musique montre qu'elle est \textit{radicalement} ancrée dans notre système sensori-moteur\footnote{Ce qui explique notamment la propension des humains à hocher la tête ou taper du pied en rythme à l'écoute d'un groove --~phénomène que l'on ne constate que rarement lorsqu'on est exposé à un motif rythmique sous un forme visuelle. Sur ce sujet, voir notamment \cite{zatorre_when_2007, maes_action-based_2014, maes_sensorimotor_2016}.}. Les gestes de l'instrumentiste font écho à notre propre expérience, incarnée, de l'effort, du mouvement et de ses dynamiques. Bien évidemment, l'art musical a cœur à jouer de ces forces, en déjouant celles d'inertie et de gravité qui façonnent notre proprioception: l'impression de lourdeur ou de légèreté d'une mélodie en rend bien compte, en l'absence de quelconque ``poids'' du phénomène sonore.\\
	\indent Cette cohérence peut être subvertie; un exemple commun est l'usage du \textit{playback} par des groupes de musique sur les plateaux TV: la cohérence entre les gestes qu'on peut voir et ce qu'on entend, ainsi que la présence des instruments --~pourtant souvent débranchés~-- poussent le spectateur à croire à une performance live\footnote{Cette subversion peut elle-même être subvertie par des groupes refusant de se prêter à ce jeu.}. Mais une cohérence entre plusieurs éléments qui vont à l'encontre de nos pré-supposés peut aussi nous amener à ré-évaluer notre propre perception et, par suite, nos attentes. C'est ce qui se passe en particulier lorsque qu'un élément musical qui nous paraissait \textit{a priori} bancal, inharmonieux, étrange, \textit{imprévu} parce qu'\textit{anormal} est légitimée par sa répétition: on \textit{apprend à le connaître}. Les accords dissonants de Stravinsky\index[people]{stravinsky@Stravinsky, Igor}, qui étaient inhabituels au début du \siecle{20}~siècle, sont depuis rentrés dans notre quotidien, par la quantité de musiques (en particulier de film) qui ont repris ses motifs. Dans un autre registre, le groove bancal du \textit{Dilla Feel}\footnote{Le \textit{Drunk feel} ou \textit{Dilla Feel}, en référence au style du producteur de hip-hop J Dilla\index[people]{jdilla@J Dilla (James Dewitt Yancey, alias~—)}, consiste en un léger décalage de certains temps et donne au rythme un style nonchalant, comme s'il était joué par une personne ivre. L'album ``Fan-Tas-Tic Vol. 1'' du groupe de rap \textit{Slum Village}\index[people]{slumvillage@Slum Village (groupe)!fantastic-vol1@\textit{Fan-Tas-Tic Vol. 1}} est un des premiers à introduire ce style en 1997.}, original dans le paysage de la fin des années 1990, est aujourd'hui omniprésent dans la musique hip-hop.

	\item \textbf{b) Inférence des gestes à partir du son}\\
	Le lexique servant à décrire les sons recourt souvent à des expressions dénotant l'inférence gestuelle: ``sons percussifs'', ``craquements'', ``souffles''. En-deçà même de leur verbalisation, nous associons naturellement les sons à une ``source'' (réelle ou imaginée), dont nous caractérisons le mode d'excitation\footnote{Voir notamment le travail de thèse d'Anne Faure \cite{faure_sons_2000}, dont les études perceptives montre que les personnes interrogées évoquent fréquemment la source supposée du son (le geste ou l'objet vibrant) pour décrire celui ci, voire, miment le geste supposé à défaut de trouver les mots.}. Non seulement nous inférons les gestes, mais également des schèmes gestuels de plus haut niveau impliquant l'articulation temporelle des gestes\footnote{Par exemple, l'expression ``le son de Gould'' sur la figure \ref{fig:gesture:Gould} se réfère davantage au style particulier du touché pianistique de Glenn Gould (notamment l'agogique) qu'au timbre du piano. Sur un plus haut niveau encore, les études d'Ani Patel et al. (e.g. \cite{patel_comparing_2006}) tendent à montrer que des corrélations rythmiques et mélodiques entre la musique et les accents toniques propres à une langue permettraient d'inférer l'origine culturelle du compositeur.}. Ce type d'inférence donne en particulier lieu aux \textit{gestes d'imitation de production du son} et aux \textit{gestes de suivi de son} décrits par Godøy et présentés précédemment\footnote{Cf. supra \ref{sec:gesture:imaginaire-du-geste-et-du-son}}.

	%------------------ Figure : Gould CD cover ---------------------
	\begin{figure}[!htbp]
	\begin{flushright}
		\begin{minipage}[t]{0.93\linewidth}
			\captionsetup{format=plain}%\centering
			\centering
			%\setlength{\fboxsep}{0pt}%
			%\setlength{\fboxrule}{1pt}%
			\fbox{\includegraphics[width=0.6\linewidth]{gfx/03_gesture/CD-cover-theSoundOfGlennGould2.jpg}}
			\caption[``The Sound of  Glenn Gould'', une inférence gestuelle]{Pochette d'un CD audio de Glenn Gould dont le titre indique l'inférence des gestes à partir du son (© Sony Classical)}
			\label{fig:gesture:Gould}
		\end{minipage}
	\end{flushright}
	\end{figure}
	%------------------ Figure : Gould CD cover ---------------------
\index[people]{gould@Gould, Glenn!variationgoldberg@\textit{Variations Goldberg}}

	\item \textbf{c) Inférence des gestes à partir de l'instrument}\\
	Elle correspond aux ``gestes attendus'' dans le schéma de Benford (figure \ref{fig:Benford_expected-sensed-desired}). Par exemple, un instrument comme le piano entraîne l'attente d'un jeu sur les touches (et non directement sur les cordes), ou une interface multitouch à des gestes en contact avec l'écran (et non à ce qu'on frappe dessus, ou que les mains s'agitent en l'air au-dessus de l'interface). Les multiples manières de capter le geste dans les \glspl{DMI}, à partir de capteurs très discrets, permettent aisément de subvertir cette inférence, mais elle suppose que l'interface de l'instrument soit suffisamment référente pour qu'un certain ensemble de gestes soit effectivement attendus.

	\item \textbf{d) Inférence des gestes à partir des gestes}\\
	Nous avons déjà évoqué cette inférence qui pourrait paraître étrange au premier abord: la perception des gestes effecteurs de l'instrumentiste est influencée par la perception que nous avons de ses postures, ses expressions faciales, les mouvements accompagnateurs, etc. Par exemple, la confiance apparente de l'instrumentiste donne l'impression qu'il maîtrise mieux son instrument\footnote{Cf. l'étude de Fyans et al. mentionnée précédemment supra \ref{sec:gesture:critique_transparency}.}; on peut aussi imaginer le geste de frappe d'un·e percussioniste dont on s'attend à ce qu'il entre effectivement en contact avec la peau d'une caisse claire, quand ce geste peut être arrêté juste avant, sans que cela soit perceptible par le public. Dans le domaine de la magie (en particulier du \textit{close-up}), cette inférence est constamment mise en œuvre par le ou la magicienne qui réalise des gestes contradictoires avec ce que les mains font effectivement. Il existe également dans le jeu musical des gestes plus ou moins cachés, comme le jeu de pédale d'un·e organiste ou encore, dans le cas des instruments à vent, tous les mouvements effectués dans la bouche et dans la gorge, qui ont un effet considérable sur le son malgré leur invisibilité\footnote{Voir à cet égard le morceau ``Spindrift'' du saxophoniste Colin Stetson qui, en plus de jouer des multiphoniques du saxophone, amplifie le son percussif des tampons de clés et chante à travers son instrument, sans que cette performance de chant ne soit explicitement visible. Vidéo en ligne:\url{https://youtu.be/KJHr2DlRog8}\index[people]{stetson@Stetson, Colin!spindrift@\textit{Spindrift}}.}.

	\item \textbf{e) Inférence du son à partir des gestes}\\
	L'inférence des gestes à partir du son fonctionne également par retournement : la vision d'un geste amène assez naturellement à imaginer un son associé à ce geste. Cette inférence est notamment la base du travail du bruiteur au cinéma, qui (re)créé un design sonore en cohérence avec les gestes visibles à l'écran, mais généralement de manière à la fois exagérée (comme le bruit des coup de poing dans les films d'action) et épurée des bruits considérés comme parasites\footnote{Michel Chion propose le terme de \textit{synchrèse} pour désigner ce travail de ``synthèse synchronisée'' \cite{chion_audio-vision:_2013}.}. L'exagération, dans ce travail de sonification du geste, est encore plus manifeste dans les créations musicales qui accompagnent les \textit{looney tunes} --~ceux de Tex Avery\index[people]{avery@Avery, Tex} en particulier~-- au point d'avoir rendu certains sons iconiques pour les gestes auxquels ils sont associés (\textit{glissando} sifflant pour une chute, bruits de ressort pour un arrêt soudain, notes isolées de xylophone pour des pas discrets, coup de triangle pour une idée lumineuse, etc.).

	\item \textbf{f) Inférence du son à partir de l'instrument}\\
	Il est également fréquent d'inférer un résultat sonore en fonction des qualités prêtées à un instrument. Par exemple, on aura tendance à s'imaginer que le son ``cuivré'' caractéristique de la famille d'instruments éponyme (tuba, trompette, trombone, mais aussi didgeridoo) est lié au matériau métallique dont ils sont (parfois) fabriqués, de même pour le son des ``bois'' (flûte, clarinette, mais aussi saxophone). Pourtant, dans un cas comme dans l'autre, le son est essentiellement lié à la géométrie de la perce et à son mode d'excitation. De même, on prête souvent aux instruments anciens, tels que les cordes, une meilleure sonorité, comme si la patine du bois avait amélioré le son, un \textit{a priori} récemment invalidé par une étude perceptive rigoureuse\footnote{Voir l'étude de Claudia Fritz et al. sur ce sujet \cite{fritz_listener_2017}}.

	\item \textbf{g) Inférence du son à partir du son}\\
	Là encore, cette auto-inférence pourrait sembler étrange, mais se manifeste clairement dans tous les effets psychoacoustiques\footnote{Pour plus de détails sur ces effets, se référer notamment aux travaux de Michèle Castellengo \cite{castellengo_ecoute_2015} ou Diana Deutsch \cite{deutsch_psychology_2013}.} utilisés tant dans la composition musicale que dans les effets de traitement audio: phénomène de perception de ``fondamentale manquante''\footnote{On dit d'un son harmonique qu'il a une ``fondamentale manquante'' lorsque ses harmoniques suggèrent une fréquence fondamentale mais que le son n'a pas de composante à la fréquence fondamentale elle-même.} utilisé pour renforcer les basses sur les systèmes audio à bande passante limitée; phénomène d'atténuation du son par le réflexe stapédien, utilisé dans la technique de \textit{sidechaining} de la compression dynamique\footnote{\label{fn:gesture:sidechain}Cet effet utilisé notamment pour renforcer l'impression de puissance des percussions est aujourd'hui omniprésent dans la production des musiques actuelles, en particulier électroniques. Un exemple sonore significatif, dans le morceau \textit{Ascending} du \gls{DJ} anglais Actress\index[people]{actress@Actress (Cunningham, Darren J., alias~—)!ascending@\textit{Ascending}} \url{https://youtu.be/8F-v9RZt1Z8}}; phénomène de fission mélodique utilisé par les compositeurs pour créer une illusion de polyphonie dans un jeu monodique\footnote{Par exemple, son utilisation est manifeste dans la Partita pour violon seul nº 3 de Bach\index[people]{bach@Bach, Johann Sebastian!partita@\textit{Partita pour violon seul nº 3}}: \url{https://youtu.be/5tjl07RmEQg}}, etc.\todo{Simplifier le jargon technique}

	%------------------ Figure : Partita ---------------------
	\begin{figure}[!htbp]
	\begin{flushright}
		\begin{minipage}[t]{0.93\linewidth}
			\captionsetup{format=plain}%\centering
			\centering
			%\setlength{\fboxsep}{0pt}%
			%\setlength{\fboxrule}{1pt}%
			\includegraphics[width=\linewidth]{gfx/03_gesture/BWV1006_preludio_autograph_manuscript_1720-intro.jpeg}
			\caption[Fission mélodique dans une partita de Bach]{La ``Partita pour violon seul nº 3'' de J.S. Bach est un exemple d'utilisation du phénomène psychoacoustique de fission mélodique: on peut entendre une polyphonie à deux voix jouées dans une même monodie.}
			\label{fig:gesture:Gould}
		\end{minipage}
	\end{flushright}
	\end{figure}
	%------------------ Figure : Partita ---------------------

	\item \textbf{h) Inférence de l'instrument à partir des gestes}\\
	Les gestes propres à un instrument traditionnel sont souvent caractéristiques au point où nous n'avons aucun mal à deviner duquel il s'agit si une personne en mime le jeu devant nous, comme dans le cas de l'\textit{air guitar}\footnote{cf. note de bas de page infra \label{fn:airguitar}}. Pour un \gls{DMI} encore inconnu, la tâche s'avère plus compliquée mais peut s'appuyer sur un ensemble de schèmes gestuels transposés, tels que les gestes instrumentaux traditionnels, mais également les gestes techniques de contrôle tels que la rotation d'un potentiomètre. Certains types de gestes étant plus couramment associés à certains types de contrôle, on peut inférer en partie le fonctionnement à partir des gestes qu'on observe. Par exemple, des gestes percussifs laissent présumer une surface qui se prête à la percussion, comme une peau tendue, tandis que des gestes déictiques sembleront davantage associés à des fonctions de sélection. Les technologies de capteurs utilisés dans les \glspl{DMI} permettent de reprogrammer en grande partie ces schèmes gestuels, et de subvertir ainsi le modèle d'instrument sous-jacent.

	\item \textbf{i) Inférence de l'instrument à partir du son}\\
	\label{sec:gesture:inferences-soudure:inferences:instrument-sound}
	La situation d'inférence la plus courante consiste à déterminer la source des sons que l'on perçoit: par exemple, si j'entends un crissement de pneus strident, je perçois \textit{avant tout} qu'une voiture risque de me foncer dessus (et je ne m'intéresserai que plus tard au timbre de ce son). Ce réflexe animal de survie est si profondément ancré que l'\textit{écoute réduite} Schaefferienne s'apparente à un exercice contre-nature\footnote{Les remarques de Pierre Schaeffer sur les efforts nécessaires pour s'affranchir ce qu'il appelle ``l'écoute banale'' sont explicites: \iquote{quelle application il y faut, quels exercices répétés, quelle patience et quelle nouvelle rigueur!} \cite{schaeffer_traite_1966} p. 271.}. Dans le cas des instruments de musique, on a cependant affaire à une source sonore d'un genre très particulier, dont la fonction est précisément de produire des sons (ce qui n'est pas spécifiquement le cas de la voiture). La programmabilité des \glspl{DMI} permet de conférer à une interface de multiples timbres, dont certains peuvent appartenir à des ``familles de sons'' déjà identifiées. C'est le cas par exemple lorsqu'on recourt à des banques de sons pré-enregistrés, dont le réalisme peut conduire l'auditeur à prendre une partition programmée pour une performance réelle d'instruments acoustiques\footnote{Cf. infra, note de bas de page \ref{fn:bacal}, l'exemple précédemment cité du Sacre du Printemps\index[people]{stravinsky@Stravinsky, Igor!sacreduprintemps@\textit{Le Sacre du Printemps}} programmé par Jay Bacal.}. Par ailleurs, l'inférence de la source s'étend au-delà de l'instrument lui-même et englobe l'acoustique des lieux, qu'il est possible de modéliser dans des algorithmes de réverbération et de spatialisation.

	\item \textbf{j) Inférence de l'instrument à partir de l'instrument}\\
	D'une manière similaire aux auto-inférences des gestes et du son précédemment évoquées, la perception partielle d'un instrument peut entraîner l'inférence de son fonctionnement global. Cette inférence est d'autant plus forte que des instruments iconiques se sont clairement établis au fil des siècles et ont été fabriqués en série. Si je vois donc un instrument ressemblant à un piano, j'en déduirai probablement qu'il se joue et sonne de manière similaire à ceux que j'ai déjà vus. Les instruments augmentés sont ainsi particulièrement aptes à subvertir cette relation, tel qu'un piano \gls{MIDI}, dont chaque touche peut délencher n'importe quel processus pré-programmé, sonore ou pas. En particulier, au delà de permettre de jouer d'autres sons que des sons de piano, il est possible de déclencher, depuis une touche, la ré-assignation dynamique de toutes les touches, c'est à dire de modifier le comportement de l'instrument en cours de jeu.

	\item \textbf{k) Inférence de l'instrument à partir de la cohérence geste/son}\\
	La perception des gestes de l'instrumentiste et/ou du résultat sonore sur un instrument dont on ne connait pas \textit{a priori} le fonctionnement entraîne une inférence de son fonctionnement. Les différents gestes ``d'accompagnement du son'', que nous avons déjà évoqués plus haut dans ce chapitre, peuvent ici être interprétés par le public comme des gestes ``effecteurs''. L'interprétation de la pièce \textit{Hangsimogato N°2} par György Kurtág Jr.\index[people]{kurtagjr@Kurtág, György, Jr.!hangsimogato@\textit{Hangsimogato N°2}}\footnote{Vidéo disponible sur \url{https://youtu.be/MJ8Z5skovLw}.} en est un exemple éloquent, qui laisse imaginer plusieurs scénarios possibles d'interaction instrumentale. Dans cette pièce, le développement musical se fait par avancement sur une partition pré-programmée par l'intermédiaire d'un capteur infra-rouge. Le capteur lui-même n'est pas capable de reconnaître quelle main, gauche ou droite, vient couper le rayon infra-rouge, ni sensible à son orientation, mais György Kurtág Jr. développe un vocabulaire gestuel qui établit des relations de correspondance avec les différents sons de la pièce. Ces relations de correspondance peuvent s'appuyer sur une similarité de morphologie énergétique, à l'instar de celle qui prévaut dans les instruments traditionnels, et renforcer la perception d'un système de valeur établi. Mais dans certains cas (e.g. geste de présentation des mains ouvertes vers le ciel, replis des bras en croix), ce système est subverti et ces gestes permettent d'étendre le fonctionnement imaginaire de l'instrument selon une dimension purement poétique.

	\item \textbf{l) Inférence du son à partir de la cohérence geste/instrument}\\
	Cette inférence se construit sur la base des gestes perçus et du fonctionnement supposé connu de l'instrument. C'est typiquement le cas de l'effet McGurk\footnote{Cf. supra, note \ref{fn:mcgurk}} : on voit les lèvres bouger, on suppose connaître le résultat produit par la voix lors d'un tel mouvement des lèvres et on infère le son entendu. Dans le domaine musical, c'est le résultat que peut produire l'interprétation de certaines pièces de György Kurtág comme par exemple \textit{Veszekedés}\index[people]{kurtag@Kurtág, György!veszekedes@\textit{Játékok: Veszekedés}}, dans lesquelles les doigts doivent \iquote{toucher la surface des touches très légèrement, sans les faire bouger}\footnote{György Kurtág évoque la pièce en ces termes : \iquote{Je voudrais vous présenter maintenant deux pièces, deux versions de Veszekedés (Dispute). La seconde a la même idée de caractère que la première, mais en forme de pantomime : le geste est très important, même au-delà du son (un geste pour le crescendo, un pour l’accelerando…), parce qu’il facilite la sensation physique.} \cite{kurtag_gyorgy_2018}} (figure \ref{fig:gesture:Kurtag-Jatekok}), ce qui pourra amener l'auditeur à une perception \textit{inframince}\footnote{C'est à dire si ténue qu'elle ne peut être qu'imaginée. Cf. supra, définition en note \ref{fn:inframince}.} d'un son, en deçà du \textit{pianississimo}.

%------------------ Figure : Kurtag Jatekok ---------------------
% \begin{figure}[H]
% 	\captionsetup{format=plain}%
% 	\includegraphics[width=\linewidth]{gfx/03_gesture/Kurtag-Jatekok.png}
% 	\caption[Exemple de gestes non-sonores dans les \textit{Játékok} de György Kurtág]{Exemple de gestes non-sonores dans les \textit{Játékok} de György Kurtág. L'indication en bas de page mentionne: \iquote{toucher la surface des touches très légèrement, sans les faire bouger}.}
% 	\label{fig:gesture:Kurtag-Jatekok}
% \end{figure}
\begin{figure}[!htbp]
	\begin{flushright}
		\begin{minipage}[t]{0.93\linewidth}
			\captionsetup{format=plain}%\centering
			\includegraphics[width=\linewidth]{gfx/03_gesture/Kurtag-Jatekok.png}
			\caption[Exemple de gestes non-sonores dans les \textit{Játékok} de György Kurtág]{Exemple de gestes non-sonores dans les \textit{Játékok} de György Kurtág. L'indication en bas de page mentionne: \iquote{Toucher la surface des touches très légèrement, sans les faire bouger}.}
			\label{fig:gesture:Kurtag-Jatekok}
		\end{minipage}
	\end{flushright}
\end{figure}
%------------------ Figure : Kurtag Jatekok ---------------------

	\item \textbf{m) Inférence des gestes à partir de la cohérence instrument/son}\\
	Le son perçu et le fonctionnement supposé de l'instrument peuvent enfin laisser supposer des gestes en décalage avec les gestes réellement effectués. Un exemple en serait l'écoute aveugle d'une pièce de Méta-Piano de Jean Haury\index[people]{haury@Haury, Jean!grandefugue@\textit{Grande Fugue Op. 133}}\footnote{Cf. par exemple l'interprétation de la \textit{Grande Fugue Op. 133} de Beethoven\index[people]{beethoven@Beethoven, Ludwig van!grandefugue@\textit{Grande Fugue Op. 133}}: \url{https://youtu.be/0hDlWxA6SlY}. On y notera également la manière dont Jean Haury invente de nouveaux gestes d'interprétation pour cette pièce, en traduisant gestuellement le rapport dialogique entre les parties musicales par l'alternance des mains.}.  Le Méta-Piano est un clavier à cinq touches, qui permet de jouer une pièce pré-séquencée, en laissant à l'instrumentiste le soin des nuances, du tempo, du phrasé et de l'agogique.  Ici, le son de piano et le fonctionnement supposé d'un tel instrument laisse imaginer des gestes qui ne correspondent évidemment pas à ceux réalisés, vue la réduction du clavier.
	%
\end{itemize}

\noindent Ces inférences sont évidemment présentées ici \textit{ex vivo}: une performance musicale réelle les superposera, les multipliera, les entrelacera, ainsi que d'autres (avec les autres instrumentistes, le public, le lieux de la performance, son contexte...) dans un tissu d'inférences bien plus complexe que ces cas pris séparément.


%%%%%%%%%%%%%%%%%%%%%%%%%%%%%%%
\todo{section sur gestalt, phénoménologie de la perception, structuralisme?}

\subsection{La subversion comme soudure de perceptions}

\todo{développer cette partie}\noindent Nous avons donc montré d'une part, que les stratégies expressives peuvent s'appuyer sur la visibilité et/ou l'audibilité relative de différentes composantes de la performance musicale, en particulier les gestes ``re-sonnants'' de l'instrumentiste, le fonctionnement de l'instrument (les ``gestes programmés''), et la perception du résultat sonore, et d'autre part que des inférences se manifestent à tous les niveaux entre ces différentes composantes.\\
\indent \iquote{C'est ainsi que la musique doit rendre sonores des forces insonores, et la peinture, visibles, des forces invisibles} disait Gilles Deleuze dans son Étude de la peinture de Francis Bacon\footnote{\cite{deleuze_francis_1981}, p. 57.}. La subversion de la relation instrumentale, qui tend ainsi à faire entendre des ``forces insonores'', émanant de gestes inattendus, sur des instruments improbables, peut s'appuyer sur ces toutes ces possibles inférences. Notre cognition commet facilement cette erreur de croire que les mêmes causes produisent les mêmes effets. Mais elle a le génie d'inventer ces relations quand elles viennent à manquer.\\ \todo{phrase longue et lourde}
\indent La réalisation de cette subversion passe par une \textit{soudure} entre deux espaces: celui physique des gestes et des sons de l'instrumentiste, et celui, psychique, de leur perception et de l'imaginaire qu'il stimulent, que l'on vient faire coïncider pour extrapoler les mouvements de l'un vers l'autre. L'invisibilité (et l'inaudibilité) de cette soudure, nécessaire à l'impression de continuité, implique qu'elle se glisse dans les synchronismes spatiaux et/ou temporels du jeu instrumental: synchronisation des mains, des pieds, ou du souffle sur les instruments acoustiques, à laquelle s'ajoute la synchronisation aux ``gestes programmés'' sur les \glspl{DMI}, qui ont leur mouvement propre.\\ 
%\indent \hl{En termes implémentationnels, cette soudure nécessite de pouvoir facilement réattribuer les paramètres de contrôle d'un modèle d'interaction à un autre. C'est notamment un des aspects qui a conduit au développement du protocole MP présenté au chapitre} \ref{ch:algorithms}.


%%%%%%%%%%%%%%%%%%%%%%%%%%%%%%%%%%%%%%%%% 
\section{Conclusion}
\label{sec:gesture:conclusion}

\noindent La relation entre le geste et l'instrument a été considérablement affectée par l'introduction de l'électricité et de l'enregistrement, mais plus encore du numérique, dans le corps de l'instrument. Ces relations se construisent sur un médium présentant des caractéristiques radicalement opposées à celles de l'instrument acoustique : l'absence de causalité et de \textit{continuum} énergétique entre le geste et le son, le métamorphisme du comportement de l'instrument, et l'absence possible de contact physique entre le geste et l'instrument y sont inhérentes et constituent des spécificités avec lesquelles il faut composer.\\
\indent Dans ce contexte, plusieurs voies sont possibles. La recherche d'une lisibilité de la relation geste/son peut s'appuyer sur le modèle instrumental acoustique dont la cohérence semble offrir un terrain propice à l'expressivité musicale car, comme le soulignait Jean-Claude Risset\index[people]{risset@Risset, Jean-Claude}, \iquote{notre ouïe a évolué dans un environnement d'objets vibrants: aussi la prise en considération des contraintes et des particularités des vibrations mécaniques est-elle importante pour comprendre les idiosyncrasies de la perception auditive et pour en tirer parti} \cite{risset_son_1992}.\\
\indent Ce modèle recréant artificiellement les modalités perdues de l'instrument acoustique ne saurait toutefois constituer une fin en soi, car comme le note encore Risset\index[people]{risset@Risset, Jean-Claude}, \iquote{la musique est aussi un art du mirage, de l’illusion} \cite{risset_propos_2010} et \iquote{les limites de l'acoustique numérique dépendent des capacités différentielles de perception davantage que des contraintes mécaniques} \cite{risset_son_1992}.\\
\indent À cette fin, l'étude des inférences perceptives entre le geste, le son et l'instrument, ainsi que la possible subversion de ces relations, nous fournit un ensemble d'éléments sur lesquels nous pouvons créer de nouvelles continuités imaginaires entre l'espace gestuel et l'espace sonore. Leur modélisation informatique, en complément de celle des lois de l'acoustique, de la psychoacoustique et de la théorie musicale contribue à un répertoire de \iquote{modèles intermédiaires}, à partir desquels se fabrique une sorte de lutherie généralisée. Celle-ci rassemble sur un même médium les différentes formes d'écriture de la relation gestuelle-sonore, préalablement distribuées dans le travail du luthier, du compositeur, de l'interprète et même de l'architecte responsable de l'acoustique d'une salle.\\
\indent La modélisation numérique de tous ces aspects, jusqu'à ceux-même de la perception auditive, ne se limite pas à la possibilité --~pour souhaitable qu'elle fût~-- de s'affranchir purement et simplement du réel, mais pose de manière critique la question des interactions musicales, des ``modes de résonance'' entre le son, le corps et l'esprit humain. C'est peut-être dans cette démarche que l'on peut entrevoir la manière dont les lutheries numériques poursuivent le travail de la lutherie classique et de la composition, qui prirent soin, elles aussi, d'inventer des formes de continuités artificielles entre les hauteurs, les timbres et les rythmes. Ces principes, formalisés dans les théories des la musique et du son, se matérialisent dans le corps de l'instrument. Le chapitre suivant proposera une analyse de cette interface entre humain et machine, où se joue cette interaction.

%%%%%%%%%%%%%%%%%%%%%%%%%%%%%%%%%%%%%%%%%%%%%%%%%%%%%%%%%%%%%%%%%%%%






% qui ont contribué à former les éléments d'expression de la musique.

% de la des  recréant artificiellement les conditions perdues de l'instrument acoustique, ou encore sur des relations basées sur la relation spectro-morphologiques entre mouvements du geste et mouvements du son.

% Ces gestes peuvent également s'appuyer sur une connaissance de la logique interne à un \textit{geste programmé}, auquel cas une relation de re-sonnance / résonance vient définir une modalité de relation instrumentale ou les mouvements gestuels ne sont pas nécessairement dans une relation mimétique, mais viennent s'articuler sur différents plans de jeu mêlant écoute, extrapolation imaginaire, [ré]agencements à la volée et anticipatifs, gestes muets dont l'action n'est perçue que de manière différée, etc.


% De même que l'on peut travailler une partition et devenir expert dans son interprétation, on peut travailler un instrument pour en devenir expert et travailler les différentes compositions pour cet instrument.On peut aujourd'hui travailler le geste (et l'écoute!) et en devenir expert pour jouer les différents instruments qui s'offrent à ces gestes.


% La possibilité de modéliser dans les \glspl{DMI} non seulement la traduction du geste en son (mapping), mais également la représentation de son influence (visualisation du mapping), ou encore les aspects métaphoriques auquel on l'associe (prodiction de son et d'image musicales) nécessite de prendre en compte la totalité de ses aspects phénoménologiques, sémiotiques, poétiques pour le design de l'interaction.
% \indent La synthèse des différentes approches ayant contribué jusque la à cette compréhension, ainsi que les propositions de nouvelles catégories et de nuances par rapport à ces modèles contribue, je l'espère, à l'avancement de sa compréhension dans cette perspective.


% Il est important de comprendre qu'il n'y a plus de relation fixe entre le geste.La relation se définit de manière contextuelle et fait partie d'un lutherie généralisée qui englobe à la fois des aspects compositionnels et des aspects de lutherie plus traditionnelle.


% => Comment ces aspects influencent le design de l’instrument ?
% De cette étude du geste instrumental, on peut retenir plusieurs éléments qui viennent orienter (todo, better word) le développement des briques de bases qui constituent les \glspl{DMI}.

% \vspace{-1em}
% \begin{itemize}[noitemsep]
% \item transgression des catégories (entre continu et discret, entre audio et non-audio, création de relation arbitraires entre paramètres orthogonaux)
% \item absence de limites arbitraires dans les représentations numériques (e.g.ambitus de pitch, polyphonie maximale)
% \end{itemize}

% \vspace{-1em}
% \begin{itemize}[noitemsep]
% \item \textbf{le format de données} : doit permettre le polymorphisme (cf.Zicarelli ``numbers without meaning'') entre les diverses formes de captation du geste (signal, événement, présence stable ou éphémère, etc.)
% \item \textbf{le mapping} entre variables est lui-même sujet à une reprogrammation dynamique durant le jeu
% \item \textbf{les différents gestes} de composition, de performance, d'écoute font partie intégrante des gestes de lutherie
% \end{itemize}


% \iquote{Part of the excitment in the domain of new digital musical instruments in the 21st century can be attributed to the fact this fact as the musical creativity goes beyond the sound itself and includes the system through which it is performed. A downside of this situation, however, is that the novelty and digital features if the instruments create a sense of discontinuity with tradition , alienation, and lack of understanding by the audience as to what the instrument or the performer is actually doing.}
% \cite{magnusson_sonic_2019}

% Un des risques de la possibilité de relations arbitraires entre le geste et ses effets est la perte de leur pouvoir de mystère, comme le remarquait le magicien Yann Frish dans une interview :

% \iquote{Si ça se trouve, cette notion que dans quelques années, ``tout sera possible avec la technologie'' fera que cela sera compliqué de créer un mystère entier et profond, parce que du coup les gens diront ``oui, j'en ai entendu parler, maintenant on peut faire ça''. J'ai un ami, Étienne Saglio, qui a fait voler un espèce de morceau de tulle au dessus des gens avec des principes mécaniques, et beaucoup de gens disaient ``ah oui, c'était incroyable mais je pense que c'était un drone'', alors que pas du tout.Mais je me suis dit, c'est vrai que d'ici quelques années, un objet qui vole tout seul en silence dans l'espace, n'aura plus le même pouvoir de mystère qu'il y a quelques années.} Yann Frish dans \url{https://youtu.be/5BqHXbQC36M}


% %%%%%%%%%%%%%%%%%%%%%%%%%%%%%%%%%%%%%%%%%%%%%%%%%%%%
% \section*{extra material}
% Notion de vivadi

% Subversion du geste : Kagel et le théâtre musical.
% \url{https://geste.hypotheses.org/gemme}

% \iquote{Dans le domaine du geste, les outils technologiques peuvent bien sûr jouer un rôle complice, démultipliant les perspectives, inversant les conséquences attendues, décelant l'infime ou captant par méthode statistique tel ou tel paramètre du jeu musical.} 
% \iquote{(...) s’approprier à la manière d’un mime les gestualités sonores qui, malgré les indications de la partition, ne peuvent être réellement considérées et donc interprétées que via le prisme de l’écoute.}
% P.Jodlowsky \cite{jodlowski_geste_2006}

% \noindent Jakobson (1960) :
% \vspace{-1em}
% \begin{itemize}[noitemsep]
% \item \textbf{expressive function}
% \item \textbf{representational function}
% \item \textbf{conative function}
% \item \textbf{phatic function}
% \item \textbf{metalingual function}
% \item \textbf{poetic function}
% \end{itemize}


% La mémoire et les gestes:
% Leroi Gourhan

% \iquote{Quant à l’action relayée (force motrice et transmission), elle domestique pour les utiliser des éléments qui étendent et complètent les effets techniques.Dans ce stade évolué, on n’est plus dans le faire mais dans le faire faire, engagé dans la voie techno-scientifique qui ne garde du geste humain initial que ses épures et en analyse indéfiniment les schèmes.} Michel Guérin, \cite{guerin_philosophie_2018}


% \iquote{J'appelle technique un acte traditionnel efficace (et vous voyez qu'en ceci il n'est pas différent de l'acte magique, religieux, symbolique).Il faut qu'il soit traditionnel et efficace.Il n'y a pas de technique et pas de transmission, s'il n'y a pas de tradition.} Marcel Mausse, les techniques du corps

% %%%%%%%%%%%%%%%%%%%%%%%%%%%%
% La notion ``d'image de son (i-son)'' de François Bayle exprime la mécanique psycho-poétique de construction de l'œuvre musicale acousmatique.Sa nomenclature ne se prête pas facilement à une application directe dans la lutherie (numérique ou non).
% En partant de la classification des fonctions de l'écoute proposée par Pierre Schaeffer et Michel Chion (\cite{chion_guide_1994}, p.26) (Insérer ici le tableau comprendre-écouter-entendre-ouïr), François Bayle retient notamment trois niveaux d'écoute attentive (en regroupant entendre et comprendre dans un seul niveau) qu'il fait correspondre à trois niveau d'intentionalité dans la mise en jeu des \textit{images-de-sons}:
% \vspace{-1em}
% \begin{itemize}[noitemsep]
% \item \textbf{\textit{im-son}}: l'image isomorphe, iconique, référentielle;
% \item \textbf{\textit{di-son}}: le diagramme, sélection de contours simplifiés, indiciels ;
% \item \textbf{\textit{mé-son}}: la métaphore ou métaforme, reliée à une généralité
% \end{itemize}

% Ces trois catégories font également écho au catégories de Delalande (gestes effecteurs, gestes accompagnateurs, gestes figurés) — sans qu'il y ait toutefois de relations causales triviales entre ces catégories du gestes et de l'écoute.

% %------------------ Bricout: g-son -------------------------
% Concept de \textit{g-son} proposé par Bricout \cite{bricout_les_2011}, à partir du concept d'i-son (\textit{image-son}) proposé par Bayle, comme ``dépassement de la suggestion de l'image par le son lui ajoutant de manière beaucoup plus évidente la suggestion du geste, de l'élan physique.''

% Romain Bricout : couple ``déclenchement/modulation'' (analogue à l'archétype ``percussion/voix'' Martin Laliberté) comme atomes gestuels constitutif de tout mouvement.=> NON tout l'espace gestuel avec toutes les connotations possibles (sémiotiques, mimétiques)

% Bricout :
% \iquote{Déclenchement et modulation représentent donc ces deux gestes primordiaux, à la base de de n'importe quel autre geste plus complexe.Par voie de conséquence, n'importe quel son renvoie lui-même à un geste producteur qui se rapprochera tantôt du déclenchement, tantôt de la modulation ou, par combinaison, des deux à la fois}
% => qu'en est il d'un field recording ?
% %------------------ END Bricout: g-son -------------------------


% Le travail de création des correspondances entre geste et son passe ainsi par trois étapes faisant écho à ces différents niveaux de perception/compréhension musicale :
% \vspace{-1em}
% \begin{itemize}[noitemsep]
% \item \textbf{coder la relation algorithmique}, c'est-à-dire concrètement la relation algorthmique qui s'opère entre les signaux captés par l'interface et le contrôle de la synthèse sonore, 
% \item\textbf{jouer la relation sensible}, c'est-à-dire pratiquer (chorégraphier) l'ensemble du mouvement gestuel dont une partie seulement sera captée par l'interface de jeu;
% \item\textbf{imaginer la relation poétique}, cette relation s'établit sur un ensemble plus complexe de valeurs esthétiques, de références culturelles impliquant de manière plus globale les questions de composition, de scénographie, de métaphores portée par les sons, etc.
% \end{itemize}




% \subsubsection{Les unités sémiotiques temporelles}

% La définition des UST est donnée dans \cite{timsit-berthier_les_2004}:
% \iquote{Les UST sont des segments musicaux, qui possèdent une signification temporelle en raison de leur organisation morphologique et cinétique.Elles peuvent êtres considérés comme des représentations iconiques qui entretiennent des rapports de ressemblance avec des modèles temporels naturels. L’UST ne traduit pas le phénomène musical à son niveau acoustique, mais cherche à y trouver en quelque sorte une intentionnalité.}



% %%%%%%%%%%%%%%%%%%%%%%%%%%%%%%%%%%%%%%%%%%%%%%%%%%%%%%%%%%%%
% %%%%%%%%%%%%%%%%%%%%%%%%%%%%%%%%%%%%%%%%%%%%%%%%%%%%%%%%%%%%
% %%%%%%%%%%%%%%%%%%%%%%%%%%%%%%%%%%%%%%%%%%%%%%%%%%%%%%%%%%%%
% %%%%%%%%%%%%%%%%%%%%%%%%%%%%%%%%%%%%%%%%%%%%%%%%%%%%%%%%%%%%
% %%%%%%%%%%%%%%%%%%%%%%%%%%%%%%%%%%%%%%%%%%%%%%%%%%%%%%%%%%%%
% %%%%%%%%%%%%%%%%%%%%%%%%%%%%%%%%%%%%%%%%%%%%%%%%%%%%%%%%%%%%
% %%%%%%%%%%%%%%%%%%%%%%%%%%%%%%%%%%
% \section*{Espace du geste musical}

% Anecdote De Laubier ` le haut parleur ne fonctionne pas'

% faisant écho à la trace latente que les sons et la musique ont déjà imprimée en nous.

% Là où la présence du musicien sur scène remplissait une nécessité acoustique pour l’écoute, la musique sur support, ou produite par des machines, déplace ce besoin au profit d’une autre fonction, à la fois de compréhension des gestes du musicien (mais est-ce là un jeu de dupes?) et d’un spectacle de l’ordre du funambulisme; le musicien prend des risques [celui de se tromper dans le cas de l’interprétation d’une partition] et la mise en question du corps, réagir au contexte (lieu et au public, ainsi qu’aux éventuels autre musiciens) d’une manière vivante.

% L’écoute nous plonge dans des flux sonores, et notre tendance à projeter des causes à ces sons (cf.gestalt) nous emmène sur les lieux — toujours en partie étrangers — de la production de ces flux.sitar indien, crissement de pneu, explosion, acoustique sous-marine ou ambiance de salle de café.
% Le musicien crée des passerelles et des agencements entre ces zones liminales.

% \Pierre{ si tu parles de gestalt, il faut développer mais c'est aussi une théorie très controversée, donc attention !}

% Si les gestes \textit{subversifs} peuvent être assimilés à des gestes accompagnateurs, la plupart des articles de la littérature semble ignorer cette part de subversion au profit de la lisibilité du geste et sa corrélation avec le son \cite{godoy_exploring_2006}.
% Cependant, la corrélation n'est pas nécessairement recherchée en tant que telle et si, comme le rappelle Risset, \iquote{la musique est aussi un art du mirage, de l’illusion} \cite{risset_propos_2010}, les œuvres sont nombreuses qui cherchent à dépasser le lien d'apparente causalité entre le geste et le son.
% Il serait alors plus juste d'utiliser le terme de \textit{geste accompagnant la musique}, si toutefois on 

% %---- Figure : Einarsson sculpture ---------
% \begin{figure}[!htbp]
% 	\includegraphics[width=\textwidth]{gfx/Einarson-SchumannSculpture}
% 	\caption{Einar Torfi Einarsson - Schumann-Sculpture (remnants + deracination)}
% 	\label{fig:gesture:einarsson}
% \end{figure}


% \iquote{Every music performance is a dramatic presentation for listeners and improvisers alike.In a sense, both groups play interactive roles as actors from their respective platforms.Just as the design of the hall, the stage and the lighting frames the band's activity for the audience's observation, it also frames the audience's activity for the band to observe.Performers and listeners form a communication loop in which the ction of each continuously affect the other.} Paul F.Berliner in \cite{berliner_thinking_2009}


% %%%%%%%%%%%%%%%%%%%%
% %-------------------------------------------
% \subsection*{Dans les instruments numériques}


% Les \glspl{DMI} ont souvent été analysés en tant qu'\gls{IHM}, et les conférences académiques qui leur sont consacré reflètent une culture dans laquelle l'interaction s'exprime via un cahier des charges préalablement identifié: une \gls{IHM} est utilisée dans le cas d'une tâche précise et sa qualité (ergonomie, précision, etc.) peut être mesurée de manière quantifiée.
% Dans le cas des instruments de musique cependant, cette tâche est plus complexe, car les enjeux de la création musicale dépassent par essence tout objectif identifié et mesurable au préalable.Par ailleurs, les \glspl{DMI} sont destinés à plusieurs types "d'utilisateurs" ayant un rôle différent : le musicien qui joue de l'instrument, mais également le public, qui bien qu'il ne joue pas de l'instrument est amené à en observer la performance.

% Low entry fee, high ceiling.

% La performance musicale est un "jeu" qui comporte une part de duplicité.Le public d'un concert est toujours le sujet d'une illusion.



% \cite{bin_show_2018}

% Une étude de Tsay \cite{tsay_sight_2013}, dans laquelle des amateurs et experts sont amenés à évaluer une performance musicale sur la seule base d'un enregistrement silencieux, met en évidence le rôle considérable du la part visuelle dans l'appréciation et l'évaluation de la performance.

% Carte et guide , frettage adaptatif (cite \cite{goudard_playing_2014})



% Dans en Echo de Manoury, ce sont les formants de la voir qui contrôlent la partie électronique, c'est-à-dire un geste invisible, sans contact.(extensible au suivi de partition)


% Pouvoir transformer tout type de donnée en geste programmé.


% L'interface sensible\footnote{Sur la notion d'interface sensible, cf.\ref{ch:interfaces}} doit pouvoir se prêter à des gestes sans intention, c'est-à-dire qu'elle doit permettre des gestes non-réfléchi, mal-contrôlé ou plutôt in-controlés, qui peuvent tomber en dehors de la zone prévue pour capter de le geste, où d'une manière inadéquate.Cela ne signifie pas nécessairement que l'instrument doit ``faire quelque chose'' de ces gestes: il peut les ignorer. Mais il est utile que l'intrument permette aux gestes de ``déborder'' du cadre prévu pour leur interaction (si toutefois ce cadre existe).


% %-------------------------------------------
% \subsection*{Tout ce qui bouge n'est pas geste - partie à revoir ou distribuer}

% Dans le domaine de la recherche musicale, les mouvements du corps sont associés à la notion de \textit{geste musical}, c'est-à-dire à un concept associant à la fois le \textit{mouvement} du corps et \textit{l'intention} et/ou \textit{la signification} de ce mouvement.

% Cela n'est pas nécessairement et systématiquement le cas et les mouvements de l'instrumentiste peuvent être envisagés et décrits avec d'autres perspectives que celle de leur potentielle intention.\todo{ref ou footnote ici vers des études en ce sens} La notion de \textit{geste musical} semble en effet implicitement suggérer un rapport hiérarchique entre le musicien et son instrument, dans lequel les gestes ne serait produits qu'intentionnellement, à l'initiative du musicien.Les instruments de musique, et en particulier les DMIs, sont envisagés plus récemment comme \textit{agents} qui opèrent dans un système de relations multi-directionnelles\todo{ref}, que Berliner décrit métaphoriquement par une \textit{conversation} dans \cite{berliner_thinking_2009}.\todo{attention,il parle de la relation musicien/public} 

% \Pierre{ je doute qu'il y ait beaucoup de gestes non-intentionnels chez le musicien !}


% Les gestes du musicien ne sont pas nécessairement remplis d'une intention ou d'une signification \textit{a priori}, ils peuvent s'apparenter aux gestes de la danse.




% L'instrument vibre et produit parfois du son sans qu'il soit explicitement déclenché ou controlé.Les mouvements du corps du musicien en témoignent et au dela des effets spectaculaires des DJs qui touchent aux potentiomètres de leurs interfaces comme s'ils étaient brûlants\footnote{Mark J.Butler apelle \iquote{passion of the knob} (\textit{la fièvre du potentiomètre}) ces moments qui surviennent \iquote{lorsqu'un musicien dirige une expressivité exceptionnellement intense vers un petit composante technique associée à l'ingénierie du son} \cite{butler_playing_2014} \url{https://www.youtube.com/watch?v=Nh9C7nQHmII}}, le corps est parcouru de mouvements qui ne sont pas uniquement des \textit{actions} mais des \textit{réactions} à ce qui est produit par l'instrument.

% \Pierre{ l'exemple choisi devrait être un peu plus analyser car les mouvements du DJs sont probablement totalement intentionnels - c-a-d ils font partie du spectacle car le DJ se sait regardé.}
% \Pierre{ je crois plus en la séparation geste-signe et geste-action}

% Si l'on considère la relation geste/instrument/musique comme un réseau multi-directionnel, le geste peut-être provoqué par la musique, via l'instrument lui même.Deux exemples caricaturaux viennent illuster cette possibilité : la performance \iquote{eletric stimulus to face — test} de l'artiste Daito Manabe\footnote{\url{http://www.daito.ws/work/electricstimulustoface_test.html}} ou dans le système de motorisation des doigts pour apprendre un instrument proposé récemment à la conférence NIME par \cite{zhang_adaptive_2019}.


% Notons enfin que les mouvements peuvent survenir également en interaction avec le public\footnote{This reveals that passion-of-the-knob moments and other actions are not interior to the musician’s world, but rather are intensely meaningful communications: they reverberate outward to the audience and then are reflected back to the stage as formative elements of a milieu whose participants seek to actively cultivate and sustain liveness.in \cite{butler_playing_2014}} 



% \subsection*{geste d'impression, geste d'expression}
% Si le geste peut ex-primer, c'est-à-dire ``faire sortir en pressant'', un mouvement intérieur et le faire exister dans la temporalité de la performance, il peut aussi im-primer (faire rentrer, en pressant) ce geste sur un support à même d'en accueillir la trace.

% geste du latin gero qui signifie ``porter''

% S'il y a différance (ajournement et différence) dans la grammatisation musicale (la composition, la lutherie, la programmation), il y a enfin le moment de sa performance, de sa répétition.

% Classification des controleurs gestuels dans \cite{wanderley_controgestuel_1999}:
% \vspace{-1em}
% \begin{itemize}[noitemsep]
% \item \textbf{Instrument-like controllers},where the input device design tends to reproduce each feature of an existing (acoustic) instrument in detail. Many examples can be cited, such as electronic keyboards, guitars, saxophones, marimbas, and so on.
% \item \textbf{Instrument-inspired controllers} that although largely inspired by the existing instrument’s design, are conceived for another use [62].Fig.3 presents one example of such controller, the SuperPolm violin developed by S.Goto, A.Terrier, and P.Pierrot [63], [64], where the input device is loosely based on a violin shape, but is used as a general device to control granular synthesis.=> emprunts variés de formes et de fonctions.
% \item \textbf{Extended instruments} are instruments augmented by the addition of extra sensors [58], [65].Commercial augmented instruments included the Yamaha Disklavier, used, for instance, in pieces by J.-C.Risset[66], [67].Other examples include the flute [68]–[70] and the trumpet [71]–[73], but any existing acoustic instrument may be extended to different degrees by the addition of sensors.
% \item \textbf{Alternate controllers} (see, e.g., Fig.4), whose design does not follow that of an established instrument.Some examples include the Hands [52], graphic drawing tablets [74] (cf.Fig.5), etc.For instance, an unorthodox gestural controller using the shape of the oral cavity has been proposed in [75].
% \end{itemize}

% These controllers can furthermore be classified into different categories.
% \begin{itemize}[noitemsep]
% \item \textbf{Touch, expanded range, or immersive} controllers [76], depending on the amount of physical contact required from the performer.Mulder also [76] separates immersive controllers into internal, external, and symbolic controllers according to the possibilities of visualization of the control surface.In a different approach, Piringer [77] classifies immmersive controllers into partial or completely immersive controllers.
% \item \textbf{Individual or collaborativecontrollers}[78],depending on whether the instrument is performed by one or multiple performers at one time.
% \item \textbf{Metaphorical} or \item{ad hoc} controllers, and so on.
% \end{itemize}


% Guerino Mazzola frozen gestures



% \iquote{Complétons tout d’abord la phrase : l’ordinateur n’est pas un instrument, mais une représentation d’instrument.Cette subtile nuance contient l’essentiel.Envisagé ainsi, l’ordinateur donne une nouvelle dimension au processus de création en y intégrant explicitement, en amont de l’acte instrumental, la construction d’une représentation du dispositif instrumental.Cette construction\/représentation offre une latitude nouvelle : la possibilité pour l’homme de se placer dans une relation ``de type instrumental'', une représentation de relation instrumentale où la liberté d’échapper aux contingences du réel lui permet de créer de nouveaux mondes imaginaires.
% Toutefois, le processus de création est également considérablement transformé par le fait que l’aller\/retour indispensable entre le réel et l’imaginaire, lui aussi, se déplace.Dans le cas de l’instrument réel, l’aller\/retour se fait in situ, dans la relation même avec l’instrument.Dans le cas de la représentation d’instrument, il se fait dans une boucle plus vaste : l’activité de représentation instrumentale, de jeu virtuel, de composition façonnent nos sens et notre intelligence d’une nouvelle manière qui sont alors en jeu dans une perception nouvelle du monde réel...à condition que nous y retournions, c’est à dire que nous ne finissions pas par substituer définitivement nos représentations à la réalité.} \cite{cadoz_musique_1999}, p99.

% Claude Cadoz défend l'hypothèse que les appareils électroniques ne sont pas des instruments, mais des ``représentations d'instrument''.

% L'apparence de causalité est fonction de
% \vspace{-1em}
% \begin{itemize}[noitemsep]
% 	\item congruence temporelle : synchronisation temporelle, morphologie rythmique similaire
% 	\item congruence spatiale : colocalisation spatiale, trajectoire similaires
% 	\item médiation par un objet intermédiaire dont on pense connaître le fonctionnement
% \end{itemize}

