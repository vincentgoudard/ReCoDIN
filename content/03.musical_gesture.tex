% !TEX root = ../thesis-example.tex
%
\chapter{Geste instumental}
\label{ch:transparency}

\cleanchapterquote{Any sufficiently advanced technology is indistinguishable from magic.}{Arthur C. Clarke}{Profiles of the Future (revised edition, 1973)}

\section{Revue des théories sur le geste instrumental}


Les instruments de musique numériques (DIM) présentent non seulement un découplage énergétique entre les gestes du musicien et le résultat sonore, mais également une partie computationnelle couplée à une mémoire qui permet un re-programmation complète de l'interaction, rendant leur fonctionnement cryptique. Cet aspect peut se révéler être aussi bien un inconvénient, dans la mesure où il prive le public d’une lecture possible d’une performance musicale. Cela peut cependant aussi s’avérer un avantage, dans la mesure où la performance musicale comporte une part scénographique dans laquelle l’illusion a toute sa place.
Je présente ici quelques aspects liés à l’interaction avec de tels instruments dans le contexte d’une performance audio-visuelle réalisée avec la plasticienne Gladys Brégeon. Cette performance implique différents types d’interfaces, et pour chacune de ces interfaces, des interactions diverses et hybrides qui joue en partie sur la lecture souhaitée ou fantasmée de ce qui s’opère sur scène.

Expliquer en quoi le découplage énergétique, qui a amené à "un sens de la discontinuité avec la tradition, aliénation et manque de compréhension par le public en ce qui concerne ce que l'instrumentiste ou l'instrument fait en réalité". (T Magnusson, sonic wrting pp. 61) a amené à une contre-réaction faisant passer la lisibilité de la relation geste/son comme un critère pertinent de design instrumental.

\section{Les limites d'une analyse des instruments en terme de HCI}
\label{sec:transparency:limitesHCI}

Plusieurs articles des NIME analyse les DMI comme HCI : 
Louange de la transparence : \cite{fels_mapping_2002}

Les instruments de musique numériques (DIM) ont un fonctionnement non-causal qui rend leur fonctionnement cryptique. Cet aspect peut se révéler être aussi bien un inconvénient, dans la mesure où il prive le public d’une lecture possible d’une performance musicale. Cela peut cependant aussi s’avérer un avantage, dans la mesure où la performance musicale comporte une part de scénographie dans laquelle l’illusion a toute sa place.
Le jeu musical joue en partie sur l’attente de l’audience (récompensée ou non) sur la base de règles d'harmonies, d’idiomes (e.g. cadences, résolutions, cycles rythmiques), de citations (e.g. via le sampling), etc..
Affordance des instruments ne peut être réduite aux objectifs d’affordance des HCI.


Kurtag Jr. Hangsimotato (video)
Jean Haury Meta-Piano
Applebaum Aphasia

\subsection{Les musiciens ne sont \emph{pas} des utilisateurs d'instruments}
\subsection{La scène et le laboratoire}


\section{Subversion sonore, subversion gestuelle}

\subsection{Continuités artificielles}

\subsection{Dans les instruments traditionnels}

\subsection{Dans les instruments numériques}


Les DMI ont souvent été analysés en tant qu'interface homme-machine, et les conférence académique qui leur sont consacrés reflètent une culture dans lequel l'interaction s'exprime via un cahier des charges préalablement identifié: une interface homme-machine est utilisée dans le cas d'une tâche précise et sa qualité (ergonomie, précision, etc.) peut être mesurée de manière quantifiée.
Dans le cas des instruments de musique cependant, cette tâche est plus complexe, car les enjeux de la création musicale dépassent par essence tout objectif identifié et mesurable au préalable. Par ailleurs, les DMI sont destinés plusieurs types "d'utilisateurs" ayant un rôle différent : le musicien qui joue de l'instrument, mais également le public, qui bien qu'il ne joue pas de l'instrument est amené à en observer la performance.

Low entry fee, high ceiling.

La performance musicale est un "jeu" qui comporte une part de duplicité. Le public d'un concert est toujours le sujet d'une illusion. 

Un des biais de la littérature sur l’affordance des instruments de musique numérique est qu’elle s’inspire souvent des objectifs de l’affordance des HCI en général, avec l’idée que l’instrument doit être compréhensible pour les “autres” utilisateurs potentiels que l’auteur de l’instrument. Pourtant, nombre d’instruments présentés dans la communauté NIME ne sont joués que par leurs auteurs (cf. [1], [2], [3]) et si le fait de vouloir transmettre son instrument aux autres est louable, il n’est pas gage de qualité en ce qui concerne la création qui sera faite avec cet instrument.     

\cite{bin_show_2018}

Carte et guide , frettage adaptatif (cite pitch)

\section{Morpho-dynamisme dans les DMI}

\section{Co-performance}


\section{Conclusion}
\label{sec:transparency:conclusion}
=> Comment ces aspects influencent le design de l’instrument ?


