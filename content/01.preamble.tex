% !TEX root = ../thesis-example.tex
%
\chapter{Introduction}
\label{ch:introduction}
%
\cleanchapterquote{L'instrument est un compromis instable entre des qualités non-convergentes.}{Bernard Sève}{(L'instrument de musique: une étude philosophique \cite{seve_instrument_2013})}

\Pierre{ l'introduction doit présenter le titre : représentation, contrôle, desgin interactif, instrument de musique et instrument de musique numérique (pas forçément dans cet ordre)}

\section{Préambule}
\label{sec:introduction:preamble}

\Pierre{ je pense que tu devrais ici repartir du début : tes motivations pour ce sujet et quels sont les notions que tu dois absolument présenter pour que l'on comprenne ta problématique. Typiquement, une intro de thèse = contexte -> problématique/hypothèses -> annonce du plan}

\noindent J'ai commencé à étudier un ``vrai'' instrument de musique --~le saxophone~-- en école de musique à l'âge de 12 ans, mais j'ai réalisé des années plus tard que le premier instrument de musique que j'avais pratiqué avait cinq touches et deux contrôleurs continus: REC, PLAY, STOP, RWD et FWD, un contrôle de volume et un sélecteur de fréquence \gls{FM}\footnote{Tous les acronymes utilisés sont définis dans le glossaire et clickables (dans la version numérique de ce document!) pour y accéder directement.}. Né l'année de l'invention du walkman, j'ai eu rapidement entre les mains cet objet que peu de monde appelerait ``instrument de musique'', mais qui permettaient pourtant de jouer des sons, des sons venant d'ailleurs ou des sons personnels, enregistrés ``à la main''.\\

\Pierre{21-08 Il faudrait mettre des espace insécables avant et après les —}

\indent Ce travail de recherche a commencé après plus de 15 ans passés à concevoir, fabriquer, programmer, pratiquer et écouter des instruments de musique à l'aide d'ordinateurs. J'ai créé durant ces années diverses sortes d'applications, d'instruments, d'outils, dans différents contextes : spectacle vivant, installations multimédia, ateliers pédagogiques, expositions muséographiques, émissions de radio, projets de recherche, etc. Il est sûrement vain de vouloir discriminer parmi tous ces objets lesquels constituent des instruments de musique et lesquels n'en sont pas, mais il me semble intéressant de constater que ces développements posent à chaque fois, sous différents angles, la question du rapport de musicalité avec la machine.
C'est ce constat qui m'a amené à réfléchir sur la notion d'instrument numérique pour essayer d'en cerner les contours, ou du moins, d'en percevoir les lignes de fuite.\\
\indent Poser la question de la représentation des instruments de musique, à une époque où l’objet a volé en éclat ainsi que les traditions musicales qu’il sous-tend, pose imanquablement la question des motivations pour lesquelles nous construisons des instruments, et par suite, des raisons pour lesquelle nous inventons, pratiquons, écoutons la musique. De cela découlent les multiples manières dont nous jouons avec le réel, avec les objets, avec les sons pour produire cet étrange --~et pourtant si familier~-- phénomène musical.\\

\Pierre{21-08 Je ne suis pas sûr qu'il faille dire que l'objet instrument a volé en éclat. L'instrument existe toujours, il a simplement intégré de nouveaux périmètres. Si tu penses qu'il a volé en éclat, il faudra bien le démontrer ensuite et proposer un concept de remplacement.}

\indent La notion d’instrument de musique numérique embrasse des problématiques complexes, sur le plan technique, mais également sur le plan esthétique et sociologique. La façon dont nous créons la musique et la manière dont nous l’écoutons a tellement changé en un siècle qu’il semble désuet de tenter de l’aborder sur un plan purement technique, tant celle-ci semble promise à bouleverser encore davantage nos usages dans l’avenir.

Pourquoi jouons nous de la musique ?


\section{Une thèse interdisciplinaire}

sciences exactes + sciences humaines + pratique musicale

\noindent Cette thèse a été l'objet d'un contrat doctoral soutenu par le Collegium Musicæ, dont la mission est de promouvoir l'interdisciplinarité entre différents acteurs institutionnels œuvrant dans le champ de la musique. En l'occurence, cette thèse codirigée par Pierre Couprie, musicologue à l'\gls{IReMus} et Jean-Dominique Polack et Hugues Genevois, chercheurs de l'équipe \gls{LAM}.
En se situant entre les domaines relativement distincts des sciences et de l'ingénierie d'une part et de la musicologie d'autre part, cette thèse est la tentative d'une étude des instruments de musique numériques prenant ces deux dimensions en compte.

Par ailleurs, ce travail de recherche théorique est soutenu par des développements techniques disponibles librement sur Internet et dont j'espère qu'ils pourront profiter à la communauté des musiciens intéressés par ces outils.
Enfin, cette thèse est soutenue par une pratique musicale utilisant ces développements pour confronter la théorie à la pratique tout autant que pour stimuler la théorie en partant de la pratique. (cf.``practice-based research'')

+ ajouter un mot sur le Collegium Musicæ.

ajouer (éventuellement?) un mot sur les usages, formats et conférences relativement différentes et séparées dans ces domaines. 
Evolution vers une interdisciplinarité nécessaire à la compréhension mutuelle. Cf. présence d'article musicologie dans NIME.


\section{Qu'entend-on par ...}

Il s'agit ici de donner quelques éléments permettant de préciser la signification donnée à un certain nombre de termes dans ce document.

\subsection*{``Représentation...}

Le terme ``représentation'' recouvre un très vaste champ sémantique. Si son étymologie évoque le fait de ``rendre présent'', ``de nouveau'', la représentation passe aussi par l'idée de l'image --~réelle ou mentale~--, distincte de l'originale représenté, que l'on se fait de quelque chose.

La question de la réprésentation (ou plutôt, des représentations) dans les lutheries numériques est envisagée sous différents angles, et présente ainsi des significations variables dans ce document : 
\vspace{-1em}
\begin{itemize}[noitemsep]
\item la représentation organologique des instruments de musique numériques, en particulier les manières variées dont il se présente comme agencements modulaires (ch. \ref{ch:ephemeral}) en contraste avec les représentations plus monolithiques de l'instrument classique;
\item la représentation de ce que l'on appelle communément ``le geste musical'' du point de vue des \gls{DMI}(ch. \ref{ch:gesture});
\item la représentation de l'instrument en tant qu'interface, en particulier son aspect visuel dynamique lié à l'usage de l'infographie (ch. \ref{ch:visual_representation});
\item la représentation proposée par le langage informatique dans lequel s'exprime le design de l'interaction musicale dans les \gls{DMI}, en particulier la manière dont s'articulent les messages venant représenter les données (gestuelles, audio, visuelles, etc.) à l'œuvre dans un DMI, sous forme de signaux continus ou d'événements discrets (ch. \ref{ch:algorithms});
\item la représentation de la musique électroacoustique jouée avec ces instruments numériques, à travers la notation (ch. \ref{ch:notation}), notamment en ce qu'elle permet de se référer collectivement et ``hors temps de jeu'' à une performance musicale passée ou à venir.
\end{itemize}

\Pierre{21-08 Attention à ne pas mélanger représentation et notation. Ce sont deux notions différentes même s'il peut y avoir des représentations dans la notation (par exemple dans les partitions d'œuvres mixtes). En général, la notation est plutôt prescriptive tandis que la représentation est descriptive (réalisée après coup).}

\subsection*{...et contrôle...}

Le contrôle est éminnement lié à la question de la représentation, ce couple formant deux versants complémentaires --~action et perception~-- du phénomènes d'interaction. En l'occurence, si dans le cas des instruments acoustiques, on agit sur l'instrument dans le but de contrôler le son qu'il produit, on agit dans le cas des \gls{DMI} sur une représentation du son encodé sous la forme de données numériques.

\subsection*{...dans le design interactif...}

Ces aspects de représentation et de contrôle sont étudiés dans la perspective de leur prise en compte concrète ``dans'' le travail de lutherie numérique. Le terme ``design interactif'' est un emprunt à l'anglais \iquote{interactive design}, généralement employé pour désigner le design de l'interaction, c'est-à-dire la conception et la réalisation des fonctions qui assurent l'aspect interactif de l'objet que l'on conçoit. Cependant, sa (semi-)traduction littérale laisse entendre l'aspect interactif du processus de design lui même, ce qui reflète dans une grande mesure la manière dont le développement d'un \gls{DMI} (et des instruments de musique, de manière plus générale) se passe : dans un jeu permanent entre la fabrication, la programmation et l'écoute, la pratique.

\subsection*{...des instruments...}

Une définition simple et efficace des instruments de musique serait la suivante : ``tout dispositif dont on se sert pour musiquer''.\\
Cette définition qui peut apparaître comme un truisme présente l'intérêt de ne pas définir les instruments en fonction de leur nature ou de leur caractéristiques techniques, mais en fonction de leur usage. En particulier, l'utilisation du terme ``jouer'' vient préciser que, si de nombreux objets techniques permettent depuis le XXème siècle de ``produire'' de la musique à partir d'enregistrement, il ne sont envisageable en tant ``d'instruments de musique'' que s'ils sont joués, et non simplement ``utilisés''. Ainsi, la platine vinyle est un instrument de reproduction sonore quand elle est utilisée par le mélomane dans son salon, mais le même objet sera un instrument de musique s'il est joué.

On pourrait alors poser la question des contours que recouvrent le terme jouer
Les instruments de musique ne sauraient donc être définis qu'à l'aune de la définition que l'on veut bien donner au terme musique. Une perspective intéressante du terme ``instrument'' cependant est sa double polarité d'objet qui sert à la fois à agir sur le monde (l'instrument-outil) et à le sentir (l'instrument de mesure).
Etymologie : \textit{struo} = construire, disposer, empiler, tramer.  \textit{instruo} : instruire, enseigner, former.

\iquote{Musiquer, c'est prendre part, quelqu'en soit notre capacité, à une performance musicale. Cela signifie non seulement jouer, mais aussi écouter, fournir des matériaux pour la performance --~ce que nous appelons composer~--, se préparer pour une performance --~ce que nous appelons répéter ou pratiquer~-- et tout autre activité connectée à la performance musicale. Nous devons certainement inclure le fait de danser, si quelqu'un danse, et nous pourrions même étendre la signification à l'occasion pour inclure ce que fait la dame qui prend les tickets à l'entrée, ou les hommes hefty todo?? qui déplacent le piano, ou les roadies qui préparent les instruments et portent le matériel de son, étant donné que leurs activités affectent également la nature de l'événement qu'est une performance musicale.}
\cite{small_musicking:_1998}

\Pierre{21-08 C'est super de parler de Small mais attention, tu mets un pied dans la sociologie et c'est bien dans ce périmètre qu'il conçoit le ‘musicking’. Je ne pense pas que cela corresponde à ce que tu nommes précédemment ‘musiquer’, ‘musicking’ est bien plus large.}

The digital has engendered a sense of novelty, curiosity, and originality in terms of performance, sound and music. The musical \textit{results} are strongly dependant on the instrument, and they often \textit{are} the instruments. Magnusson, Sonic Writing, p.61

Les instruments de musique sont des instrument pour jouer de la musique (ou pour musiquer, dirait Christopher Small \cite{small_musicking:_1998}). Cette apparente évidence est nécessaire pour signaler qu'il ne s'agit pas simplement de faire des notes, ou même du son. La musique implique également notre \textit{mémoire du son}, l'imagination que nous en avons, les aspects visuels qui se rattachent à la notion de musicalité, et d'autres dimensions esthétiques et culturelles. \todo{être plus précis}

La performance musicale a cela de particulier qu'elle ne possède pas de cahier des charges préalables (la partition ne saurait être considérée comme telle!) et que loin de se plier à la nécessité d'exécuter une tâche précise, comme il pourrait être le cas dans le design d'autres interfaces homme-machine, les instruments sont des objets techniques dont les musiciens abusent (plus qu'ils en usent), dont les artefacts peuvent être appréciables et souhaitables, dont la compréhension n'est pas un préalable requis pour leur utilisation, pas davantage que leur fiabilité n'est garante d'une performance musicale intéressante.

\Pierre{21-08 Ce paragraphe est une excellente definition de l'instrument numérique. C'est vraiment très bien dit.}

%
Le design des DMI, ainsi que le design des outils-mêmes du luthier numérique, doivent être informés de ces particularités propres à la création artistique si l'on souhaite qu'ils se prêtent à la création de musiques nouvelles et à l'exploration de territoires sonores inexplorés.


\subsection*{...de musique...} 
\subsubsection*{définition intrinsèque}
La musique est un concept difficile à définir et comme le dit le musicologue Pierre Billard en intoduction de la définition donnée par l'encyclopédie Universalis \iquote{plus notre connaissance de la musique est étendue et moins nous savons, en fin de compte, ce qu'elle est.}

Définir la musique sur la base de son contenu est probablement la perspective la moins consensuelle qui soit et dont l'intérêt principal est peut-être qu'elle qualifie surtout l'opinion de celui (ou du groupe) qui la définit ainsi : musique de son purs, musique de bruits, sons organisés de Varèse.
\subsubsection*{définition intrinsèque}

On peut définir la musique du point de vue de celui qui la produit (compositeur, instrumentiste) : « Tout corps sonore utilisé par le compositeur est un instrument de musique. » Berlioz, 1843, Grand Traité d'instrumentation et d'orchestration modernes

Définir la musique du point de vue de celui qui l'écoute : est musique tout ce que j'écoute et considère comme telle.

La musique est un art de la résonance, qu'elle soit d'ordre acoustique et physique, ou d'ordre plus intellectuelle et spirituelle. La musique fait écho.

\subsection*{...numériques ?} 
Numériques au pluriel car ce sont les instruments qui le sont, pas la musique.
C'est ici ce qu'amène le numérique qui m'intéresse, c'est-à-dire ( TODO) l'utilisation d'une forme symbolique automatisée et traitable en grande quantité par les ordinateurs. Ce que cela apporte aux instruments, ce qui unit ou sépare les différents types d'instruments numériques ensemble et par rapport aux lutheries classiques.


\section{Problématique}

\Pierre{ Je ne vois pas de problématique présentée. La problématique est la question que tu poses dans ta thèse ainsi que les questions annexes ou sous-questions.}

Nécessité de prendre en compte la part expérientielle de la performance musicale, notamment dans sa dimension subversive.

Cette thèse propose une étude de la création et la performance musicale avec des \gls{DMI}, pour essayer d'en re-définir les contours et en tirer des conséquences sur leur design.

Notamment, en envisageant le geste musical comme phénomène dépassant l'approche fonctionnelle qui lui est souvent conférée dans les études en \gls{IHM}, et en analysant les \glspl{DMI} en tant qu'assemblages et processus possédant des qualités propres et différentes de celles des instruments acoustiques, ce travail vise à étudier comment les différents enjeux qui se posent avec de tels instruments dans la performance musicale s'articulent au niveau de leur conception.

Quelques questions : 
\vspace{-1em}
\begin{itemize}[noitemsep]
\item Au delà des métaphores de la bureautique (menus, sliders, checkbox), quel vocabulaire graphique est envisageable pour le design des éléments d'interaction ?
\item Comment gérer les scénarios de déconnexion / reconnexion dynamiques intervenant dans le cours d'une performance ?
\item Comment noter la musique (électroacoustique) produite avec de telles interfaces pour le jeu collectif ?
\item Quelles interfaces seront pertinentes lors des différentes phases de conception, de composition, de répétition, ou de performance avec l'instrument ?
\item  Quelles représentations privilégieront, pour le contrôle de paramètres identiques, tantôt la virtuosité, la précision, l'étendue de de la palette sonore, la polyphonie gestuelle ou encore la structure temporelle ?
\end{itemize}

\section{Enjeux et hypothèses}

Enjeu de trouver des caractéristiques transversales dans les lutheries numériques malgré l'absence de tradition, de répertoire, de notation, de méthode d'apprentissage, etc.
Enjeu de confronter au réel des réalisations instrumentales et logicielles à travers une pratique musicale.

Ce travail tente de présenter l'ensemble d'une démarche de création d'un \gls{DMI}, comprenant sa conception, sa fabrication, sa programmation, sa pratique, et la composition avec cet instrument.
Ce travail de recherche s'offre donc comme une présentation ``en coupe'' d'un travail de lutherie, dans ce qu'il comporte de réflexions, de choix de matériaux, d'assemblages, de programmation, de notations, de pratiques et comment ces différents aspects interfèrent dans le cas particulier des instruments intégrant le numérique dans le design de leur interaction.

\subsubsection*{Hypothèse 1 : l'instrument atomisé, recomposé}

L'instrument est atomisé et se retrouve configuré comme un agencement modulaire évolutif qui se cristallise ponctuellement dans des instances contextuelles. \\
Pas d'instrument standard qui sorte du lot, si ce n'est pour imiter l'existant (le clavier, la guitare, etc.), mais des protocoles et modules qui deviennent standards et interconnectables.

\subsubsection*{Hypothèse 2 : l'instrument est subversif}

\Pierre{ le subversif est une très bonne idée mais c'est un terme tellement chargé de sens que tu dois en dire un peu plus et notamment dans quel sens tu le prends.}

La relation instrumentale n'est pas de même nature que la relation des HCI.
Les instrumentistes ne sont pas des \textit{interface users} mais plutôt des \textit{interface abusers}.
La relation entre le geste et le son n'est pas nécessairement faite pour être lisible et comprise du public, le musicien est un magicien.
L'œil augmente l'écoute (et la subvertit).


\subsubsection*{Hypothèse 3 : Le continu et le discret}

Si la continuité de la vibration physique semble être une donnée consitutive des instruments acoustique, qui recrééent artificiellement des espaces discrets (tels que les échelles harmoniques et rythmiques), le domaine du numérique part d'une certaine manière dans la direction opposée. Les instruments numériques sont caractérisé par la nature discrète (et même binaire) de l'encodage symbolique sous-jacent aux données traitées. Il s'agit donc davantage de pouvoir retrouver une continuité dans ce monde discret.\\
Cette bipolarité du continu et du discret traverse ainsi, à des degrés variés, les questions de design qui se présentent dans la conception des instruments numériques, que cela soit au niveau de l'encodage du geste capté ou de celui de la synthèse audio. Les développements présentés dans ce travail sont orientés par les possibilités de passage fluide du continu au discret et inversement, animés par la conviction qu'une partie du jeu musical se joue dans cette ambivalence, en soutenant l'aspect suversif du geste musical.


\section{Interviews}

Une caractéristique notable des lutheries numériques est leur diversité et le foisonnement d'approches, de propositions, de positions prises par ceux qui les inventent et les pratiquent. Un certain nombre d'entretiens ont été menées durant ce travail de thèse afin d'élargir le champ de la réflexion à différentes approches sur les instruments de musique numérique. Ces entretiens sont reproduits intégralement en annexe, accompagnés d'une brève biographie présentant les personnes ayant acceptés de présenter leur travail et leurs réflexions.

Ces interviews ont pris la forme de discussions libres, orientées par un certain nombre de questions, dont la première était invariablement : quelle a été la motivation originale qui vous a poussé à concevoir et utiliser des DMI ?. La suite de la discussion dépendait ensuite de l'interlocuteur, leurs projets étant relativement différents entre ceux d'entrepreneurs et ceux d'artistes. %On y trouve cependant quelques idées transversales qui ont contribuées à nourrir ma propre réflexion.

\todo{reproduire le guide d'interview en annexe}

Liste des personnes interviwées (et liens vers annexes) :

\vspace{-1em}
\begin{itemize}[noitemsep]
\item \textbf{\hyperref[appendix:bernier]{Nicolas Bernier}}, artiste canadien créant des installations et performances audio-visuelles, et enseigne la "musique numérique" à l'Université de Montréal;
\item \textbf{\hyperref[appendix:collins]{Nicolas Collins}}, compositeur, artiste sonore, professor au département son  à la School of the Art Institute de Chicago 1999 et auteur notable du livre "Handmade Electronic Music –The Art of Hardware Hacking";
\item \textbf{\hyperref[appendix:dumeaux]{François Dumeaux}}, musicien et compositeur de musiques électro-acoustiques;
\item \textbf{\hyperref[appendix:delaubier]{Serge De Laubier}}, musicien, inventeur du Méta-Instrument, directeur artistique de Puce Muse;
\item \textbf{\hyperref[appendix:fernandez]{Jose-Miguel Fernandez}}, compositeur 
%\item \textbf{\hyperref[appendix:kurtag]{György Kurtag Jr.}}, musicien improvisateur, compositeur, pédagogue...
\item \textbf{\hyperref[appendix:mamou-mani]{Adrien Mamou-Mani}}, chercheur et co-fondateur de HyVibes, startup créant des instruments augmentés tels la \textit{SmartGuitar};
\item \textbf{\hyperref[appendix:saint-denis]{Patrick Saint-Denis}}, compositeur, luthier numérique, enseigne à l'Université de Montréal.
\item \textbf{\hyperref[appendix:turchet]{Lucas Turchet}} (b. 1982), designer sonore, musicien, compositeur et écrivain, co-fondateur de Mind Music Labs, startup créant des instruments augmentés.
\item \textbf{\hyperref[appendix:zamborlin]{Bruno Zamborlin}} (b. 1984), fondateur et CEO de Mogees et HyperSurfaces. 
\end{itemize}



\section{Contributions de cette thèse}

Cette thèse propose plusieurs contributions théoriques dans le domaine de la recherche sur les \glspl{DMI}, ainsi que plusieurs contributions pratiques sous la forme de \glspl{LogicielLibre} et disponibles sur le web.

Les contributions théoriques concernent :
\vspace{-1em}
\begin{itemize}[noitemsep]
\item des perspectives sur la nature des \gls{DMI}, leur cycle de vie, la notion d'assemblage éphémère et ses conséquences sur leur design;
\item la caractérisation du geste musicale, en particulier sa part subversive et son inscription dans le design de l'instrument;
\end{itemize}

Les contributions pratiques sont les suivantes :
\vspace{-1em}
\begin{itemize}[noitemsep]
%\setlength\itemsep{-1.5em}
\item \textbf{LAM-lib} : un package pour le logiciel Max proposant une collection d'algorithmes utiles pour la lutherie numérique;
\item \textbf{MP} : un protocole de communication pour le contrôle de la synthèse, venant palier un certain nombre de limitations rencontrées dans le protocole MIDI, ainsi qu'un package Max rassemblant un certain nombre d'objets supportant ce protocole;
\item \textbf{mp.TUI} : un package pour Max permettant la création d'interfaces graphique tangibles personnalisables et polyphoniques, basées sur le protocole MP;
\item \textbf{sagrada} : un package Max de synthèse granulaire modulaire contrôlé par signal;
\item \textbf{John} : un logiciel pour la (semi-) composition et conduite d'improvisation électroacoustique, éditable collectivement;
\end{itemize}


\section{Structure de la thèse}
\label{sec:preamble:structure}

\textbf{Chapitre \ref{ch:introduction}} \\[0.2em]
Vous êtes ici.

\textbf{Chapitre \ref{ch:ephemeral}} \\[0.2em]
Le chapitre \ref{ch:ephemeral} présente un certain nombre de considérations sur les instruments de musique numériques et le contexte de leur utilisation. En particulier, la nature éphémère des assemblages modulaires, souvent ignorée, est ici considérée comme une des caractéristiques essentielles venant influencer leur design. Un distribution entre répertoire, musicien et contexte permet de re-définir la façon dont s'articulent ces différents pôles ainsi à l'œuvre dans la création et l'évolution des DMIs.

\textbf{Chapitre \ref{ch:gesture}} \\[0.2em]
Le chapitre \ref{ch:gesture} vient questionner la notion de geste musical dans le cas de la pratique avec des DMIs. En particulier, la lisibilité du geste et de sa relation à la synthèse sonore, souvent considérée comme un critère de design souhaitable, y est remise en question en prenant en compte les fins subversives de l'art musical. 

L'étude des artefacts qui en résultent et viennent bouleverser la perception de continuité(s) permet d'introduire la notion de morpho-dynamisme des DMIs, son intérêt dans la création et la pratique musicale et son intégration dans le corps de l'instrument.

\textbf{Chapitre \ref{ch:interfaces}} \\[0.2em]
Le chapitre \ref{ch:interfaces} présente une exemple particulier d'interface instrumentale et retrace l'histoire de son évolution à travers plusieurs générations, partant d'une interface standard et disponible dans le commerce (la tablette graphique) et évoluant vers une personnalisation et un enrichissement du dispositif. 
Seront discutées les raisons motivant l'ajout de capteurs, l'organisation de l'espace de jeu, la polyphonie des sources sonores, etc.

\textbf{Chapitre \ref{ch:algorithms}} \\[0.2em]
Le chapitre \ref{ch:algorithms} présente des développement réalisés pour la conception du ``mapping'' de l'instrument en tentant notamment de répondre aux problématiques soulevées dans les chapitres \ref{ch:ephemeral} et \ref{ch:gesture}. Sont présentés dans ce chapitre les concepts de modèle intermédiaire, ainsi qu'un protocole de contrôle expressif polyphonique, nommé ``MP'', permettant la communication entre interfaces, modules de transformation et de synthèse. Une extension des idées de MP dans le domaine du signal et appliqué à la synthèse granulaire, nommé Sagrada, est également présenté.

\textbf{Chapitre \ref{ch:visual_representation}} \\[0.2em]
Le chapitre \ref{ch:visual_representation} présente un système d'interface graphique tangible (TUI) basée sur le protocole présenté au chapitre  \ref{ch:algorithms}, afin de permettre notamment une reconfiguration dynamique de l'interface de jeu et une représentation graphique des processus utilisés pour la performance musicale. Ces interfaces graphiques permettent également d'intégrer des éléments de représentation musicale (forme d'onde, échelle, motifs rythmiques, etc.) ou non-musicale comme composants interactifs pour le contrôle expressif.

\textbf{Chapitre \ref{ch:notation}} \\[0.2em]
Le chapitre \ref{ch:notation} présente des travaux portant sur la notation musicale dans le domaine de la performance électroacoustique utilisant des DMI. Les questions de composition collective, d'édition collaborative et d'écologie de l'attention sont abordées et sont mises en œuvre dans ``John, the semi-conductor'', un logiciel permettant la génération automatique et l'édition collective de partitions minimales, utilisé dans l'ensemble d'improvisation électroacoustique ONE. 

\textbf{Chapitre \ref{ch:conclusion}} \\[0.2em]
Pistes de recherches à suivre...


\section*{extra material}
