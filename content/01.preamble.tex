% !TEX root = ../thesis-example.tex
%
\chapter{Introduction}
\label{ch:introduction}
%
\cleanchapterquote{L'instrument est un compromis instable entre des qualités non-convergentes.}{Bernard Sève}{(L'instrument de musique: une étude philosophique \cite{seve_instrument_2013})}

\Pierre{ l'introduction doit présenter le titre : représentation, contrôle, desgin interactif, instrument de musique et instrument de musique numérique (pas forçément dans cet ordre)}

\section{Préambule}

\Pierre{ je pense que tu devrais ici repartir du début : tes motivations pour ce sujet et quels sont les notions que tu dois absolument présenter pour que l'on comprenne ta problématique. Typiquement, une intro de thèse = contexte -> problématique/hypothèses -> annonce du plan}

Ce travail de recherche présente une forme quelque peu atypique et un contenu qui semblera probablement hétéroclite, d'un point de vue académique. Ce préambule donne quelques explications sur les motivations qui ont mené à ces résultats.

\Pierre{ le premier paragraphe est inutile : pas de commentaire sur ton propre texte.}

Poser la question de la représentation des instruments de musique, à une époque où l’objet a volé en éclat ainsi que les traditions musicales qu’il soutend, pose imanquablement la question des motivations pour lesquelles nous construisons des instruments, des raisons pour lesquelle nous inventons, pratiquons, écoutons la musique. De cela découlent les multiples manières dont nous jouons avec le réel, avec les objets, avec les sons pour produire cet étrange —et pourtant si familier— phénomène de musique.

La notion d’instrument de musique numérique embrasse des problématiques complexes, sur le plan technique, mais également sur le plan esthétique et sociologique. La façon dont nous créons la musique et la manière dont nous l’écoutons a tellement changé en un siècle qu’il semble désuet de tenter de l’aborder sur un plan purement technique, tant celle-ci semble promise à bouleverser encore davantage nos usages dans l’avenir.

Pourquoi jouons nous de la musique ?
Même s’il semble impossible d’apporter une réponse simple à cette question, il faut donc prendre en compte celle-ci 

\vspace{-1em}
\begin{itemize}[noitemsep]
\item voyager
\item expérience fragile et radicale de l’instant
\item partager
\end{itemize}

\section{Une thèse en science et musicologie}

En se situant entre les domaines relativement distincts des sciences et de l'ingénierie d'une part et de la musicologie d'autre part, cette thèse est la tentative d'une étude des instruments de musique numériques prenant ces deux dimensions en compte.
Usages, formats et conférences relativement différentes et séparées dans ces domaines. 
Evolution vers une interdisciplinarité nécessaire à la compréhension mutuelle.

+ ajouter un mot sur le Collegium Musicæ.

\section{Problématique}

\Pierre{ Je ne vois pas de problématique présentée. La problématique est la question que tu poses dans ta thèse ainsi que les questions annexes ou sous-questions.}

Les instruments de musique ont la particularité de pouvoir s'envisager sous ces deux aspects et bien qu'il soit possible de ne s'intéresser qu'à l'un des deux, nous croyons fortement qu'une étude des motivations qui poussent à leur design ne saurait faire abstraction des conditions particulières de leur inscription dans le domaine socio-culturel.

\Pierre{ "domaine socio-culturel" : es-tu sûr de vouloir te placer sur un niveau sociologique ? }

Les instruments de musique sont des instrument pour faire de la musique (ou pour musiquer, dirait Christopher Small \cite{small_musicking:_1998}). Cette apparente évidence est nécessaire pour signaler qu'il ne s'agit pas simplement de faire des notes, ou même du son. La musique implique également notre \textit{mémoire du son}, l'imagination que nous en avons, les aspects visuels qui se rattachent à la notion de musicalité, et d'autres dimensions esthétiques et culturelles. \todo{être plus précis}

La performance musicale a cela de particulier qu'elle ne possède pas de cahier des charges préalables (la partition ne saurait être considérée comme telle!) et que loin de se plier à la nécessité d'exécuter une tâche précise, comme il pourrait être le cas dans le design d'autres interfaces homme-machine, les instruments sont des objets techniques dont les musiciens abusent (plus qu'ils en usent), dont les artefacts peuvent être appréciables et souhaitables, dont la compréhension n'est pas un préalable requis pour leur utilisation, pas davantage que leur fiabilité n'est garante d'une performance musicale intéressante.
%
Le design des DMI, ainsi que le design des outils-mêmes du luthier numérique, doivent être informés de ces particularités propres à la création artistique si l'on souhaite qu'ils se prêtent à la création de musiques nouvelles et à l'exploration de territoires sonores inexplorés.

Nécessité de prendre en compte la part expérientielle de la performance musicale, notamment dans sa dimension subversive.

\section{Enjeux et hypothèses}

Enjeu de trouver des caractéristiques transversales dans les lutheries numériques malgré l'absence de tradition, de répertoire, de notation, de méthode d'apprentissage, etc.
Enjeu de confronter au réel des réalisations instrumentales et logicielles à travers une pratique musicale.

Ce travail de recherche s'offre donc comme une présentation "en coupe" d'un travail de lutherie, dans ce qu'il comporte de réflexions, de choix de matériaux, d'assemblages, de programmation, de notations et de pratiques. Il n'entend évidemment pas être exhaustif sur le sujet, ni généralisable à l'ensemble des lutheries numériques mais permettra, je l'espère, d'éclairer les personnes qui s'y intéressent à la lumière de cette approche transversale.

\Pierre{ "Il n'entend évidemment pas être exhaustif" -> ne pas le dire car le rôle d'une thèse est justement d'être exhaustif sur le sujet...}

\subsubsection*{Hypothèse 1 : l'instrument atomisé, recomposé}

L'instrument est atomisé et se retrouve configuré comme un agencement modulaire évolutif qui se cristallise ponctuellement dans des instances contextuelles. \\
Pas d'instrument standard qui sorte du lot, si ce n'est pour imiter l'existant (le clavier, la guitare, etc.), mais des protocoles et modules qui deviennent standards et interconnectables.

\subsubsection*{Hypothèse 2 : l'instrument est subversif}

\Pierre{ le subversif est une très bonne idée mais c'est un terme tellement chargé de sens que tu dois en dire un peu plus et notamment dans quel sens tu le prends.}

La relation instrumentale n'est pas de même nature que la relation des HCI.
Les instrumentistes ne sont pas des \textit{interface users} mais plutôt des \textit{interface abusers}.
La relation entre le geste et le son n'est pas nécessairement faite pour être lisible et comprise du public, le musicien est un magicien.
L'œil augmente l'écoute (et la subvertit).


\subsubsection*{Hypothèse 3 : Le continu et le discret}

Si la continuité de la vibration physique semble être une donnée consitutive des instruments acoustique, qui recrééent artificiellement des espaces discrets (tels que les échelles harmoniques et rythmiques), le domaine du numérique part d'une certaine manière dans la direction opposée. Les instruments numériques sont caractérisé par la nature discrète (et même binaire) de l'encodage symbolique sous-jacents aux données traitées. Il s'agit donc davantage de pouvoir retrouver une continuité dans ce monde discret.\\
Cette bipolarité du continu et du discret traverse ainsi, à des degrés variés, les questions de design qui se présentent dans la conception des instruments numériques, que cela soit au niveau de l'encodage du geste capté ou de celui de la synthèse audio. Les développements présentés dans ce travail sont orientés par les possibilités de passage fluide du continu au discret et inversement, animés par la conviction qu'une partie du jeu musical se joue dans cette ambivalence.


\section{Interviews}

Une caractéristique notable des lutheries numériques est leur diversité et le foisonnement d'approches, de propositions, de positions prises par ceux qui les inventent et les pratiquent. Un certain nombre d'entretiens ont été menées durant ce travail de thèse afin d'élargir le champ de la réflexion à différentes approches sur les instruments de musique numérique. Ces entretiens sont reproduits intégralement en annexe, accompagnés d'une brève biographie présentant les personnes ayant acceptés de présenter leur travail et leurs réflexions.

Ces interviews ont pris la forme de discussions libres, orientées par un certain nombre de questions, dont la première était invariablement : "quelle a été la motivation originale qui vous a poussé à concevoir et utiliser des DMI ?". La suite de la discussion dépendait ensuite de l'interlocuteur, leurs projets étant relativement différents entre ceux d'entrepreneurs et ceux d'artistes. %On y trouve cependant quelques idées transversales qui ont contribuées à nourrir ma propre réflexion.

\todo{reproduire le guide d'interview en annexe}

Liste des personnes interviwées (et liens vers annexes) :

\vspace{-1em}
\begin{itemize}[noitemsep]
\item \textbf{\hyperref[appendix:bernier]{Nicolas Bernier}}, artiste canadien créant des installations et performances audio-visuelles, et enseigne la "musique numérique" à l'Université de Montréal;
\item \textbf{\hyperref[appendix:collins]{Nicolas Collins}}, compositeur, artiste sonore, professor au département son  à la School of the Art Institute de Chicago 1999 et auteur notable du livre "Handmade Electronic Music –The Art of Hardware Hacking";
\item \textbf{\hyperref[appendix:dumeaux]{François Dumeaux}}, musicien et compositeur de musiques électro-acoustiques;
\item \textbf{\hyperref[appendix:delaubier]{Serge De Laubier}}, musicien, inventeur du Méta-Instrument, directeur artistique de Puce Muse;
\item \textbf{\hyperref[appendix:fernandez]{Jose-Miguel Fernandez}}, compositeur 
%\item \textbf{\hyperref[appendix:kurtag]{György Kurtag Jr.}}, musicien improvisateur, compositeur, pédagogue...
\item \textbf{\hyperref[appendix:mamou-mani]{Adrien Mamou-Mani}}, chercheur et co-fondateur de HyVibes, startup créant des instruments augmentés tels la \textit{SmartGuitar};
\item \textbf{\hyperref[appendix:saint-denis]{Patrick Saint-Denis}}, compositeur, luthier numérique, enseigne à l'Université de Montréal.
\item \textbf{\hyperref[appendix:turchet]{Lucas Turchet}} (b. 1982), designer sonore, musicien, compositeur et écrivain, co-fondateur de Mind Music Labs, startup créant des instruments augmentés.
\item \textbf{\hyperref[appendix:zamborlin]{Bruno Zamborlin}} (b. 1984), fondateur et CEO de Mogees et HyperSurfaces. 
\end{itemize}



\section{Contributions de cette thèse}

Cette thèse propose plusieurs contributions théoriques dans le domaine de la recherche sur les \glspl{DMI}, ainsi que plusieurs contributions pratiques sous la forme de \glspl{LogicielLibre} et disponibles sur le web.

Les contributions théoriques concernent :
\vspace{-1em}
\begin{itemize}[noitemsep]
\item des perspectives sur la nature des \gls{DMI}, leur cycle de vie, la notion d'assemblage éphémère et ses conséquences sur leur design;
\item la caractérisation du geste musicale, en particulier sa part subversive et son inscription dans le design de l'instrument;
\end{itemize}

Les contributions pratiques sont les suivantes :
\vspace{-1em}
\begin{itemize}[noitemsep]
%\setlength\itemsep{-1.5em}
\item \textbf{LAM-lib} : un package pour le logiciel Max proposant une collection d'algorithmes utiles pour la lutherie numérique;
\item \textbf{MP} : un protocole de communication pour le contrôle de la synthèse, venant palier un certain nombre de limitations rencontrées dans le protocole MIDI, ainsi qu'un package Max rassemblant un certain nombre d'objets supportant ce protocole;
\item \textbf{mp.TUI} : un package pour Max permettant la création d'interfaces graphique tangibles personnalisables et polyphoniques, basées sur le protocole MP;
\item \textbf{sagrada} : un package Max de synthèse granulaire modulaire contrôlé par signal;
\item \textbf{John} : un logiciel pour la (semi-) composition et conduite d'improvisation électroacoustique, éditable collectivement;
\end{itemize}


\section{Structure de la thèse}
\label{sec:preamble:structure}

\textbf{Chapitre \ref{ch:introduction}} \\[0.2em]
Vous êtes ici.

\textbf{Chapitre \ref{ch:ephemeral}} \\[0.2em]
Le chapitre \ref{ch:ephemeral} présente un certain nombre de considérations sur les instruments de musique numériques et le contexte de leur utilisation. En particulier, la nature éphémère des assemblages modulaires, souvent ignorée, est ici considérée comme une des caractéristiques essentielles venant influencer leur design. Un distribution entre répertoire, musicien et contexte permet de re-définir la façon dont s'articulent ces différents pôles ainsi à l'œuvre dans la création et l'évolution des DMIs.

\textbf{Chapitre \ref{ch:transparency}} \\[0.2em]
Le chapitre \ref{ch:transparency} vient questionner la notion de geste musical dans le cas de la pratique avec des DMIs. En particulier, la lisibilité du geste et de sa relation à la synthèse sonore, souvent considérée comme un critère de design souhaitable, y est remise en question en prenant en compte les fins subversives de l'art musical. 

L'étude des artefacts qui en résultent et viennent bouleverser la perception de continuité(s) permet d'introduire la notion de morpho-dynamisme des DMIs, son intérêt dans la création et la pratique musicale et son intégration dans le corps de l'instrument.

\textbf{Chapitre \ref{ch:interfaces}} \\[0.2em]
Le chapitre \ref{ch:interfaces} présente une exemple particulier d'interface instrumentale et retrace l'histoire de son évolution à travers plusieurs générations, partant d'une interface standard et disponible dans le commerce (la tablette graphique) et évoluant vers une personnalisation et un enrichissement du dispositif. 
Seront discutées les raisons motivant l'ajout de capteurs, l'organisation de l'espace de jeu, la polyphonie des sources sonores, etc.

\textbf{Chapitre \ref{ch:algorithms}} \\[0.2em]
Le chapitre \ref{ch:algorithms} présente des développement réalisés pour la conception du "mapping" de l'instrument en tentant notamment de répondre aux problématiques soulevées dans les chapitres \ref{ch:ephemeral} et \ref{ch:transparency}. Sont présentés dans ce chapitre les concepts de modèle intermédiaire, ainsi qu'un protocole de contrôle expressif polyphonique, nommé "MP", permettant la communication entre interfaces, modules de transformation et de synthèse. Une extension des idées de MP dans le domaine du signal et appliqué à la synthèse granulaire, nommé Sagrada, est également présenté.

\textbf{Chapitre \ref{ch:visual_representation}} \\[0.2em]
Le chapitre \ref{ch:visual_representation} présente un système d'interface graphique tangible (TUI) basée sur le protocole présenté au chapitre  \ref{ch:algorithms}, afin de permettre notamment une reconfiguration dynamique de l'interface de jeu et une représentation graphique des processus utilisés pour la performance musicale. Ces interfaces graphiques permettent également d'intégrer des éléments de représentation musicale (forme d'onde, échelle, motifs rythmiques, etc.) ou non-musicale comme composants interactifs pour le contrôle expressif.

\textbf{Chapitre \ref{ch:notation}} \\[0.2em]
Le chapitre \ref{ch:notation} présente des travaux portant sur la notation musicale dans le domaine de la performance électroacoustique utilisant des DMI. Les questions de composition collective, d'édition collaborative et d'écologie de l'attention sont abordées et sont mises en œuvre dans "John, the semi-conductor", un logiciel permettant la génération automatique et l'édition collective de partitions minimales, utilisé dans l'ensemble d'improvisation électroacoustique ONE. 

\textbf{Chapitre \ref{ch:conclusion}} \\[0.2em]
Pistes de recherches à suivre...


\section*{extra material}
