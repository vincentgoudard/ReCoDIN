\chapter{Interview : Nicolas Collins}
\label{appendix:collins}

\section*{Biographie}

\noindent Né ét élevé à New York, Nicolas Collins a passé la majeure partie des années 1990 en Europe, où il a été directeur artistique invité du \gls{STEIM} (Amsterdam), et compositeur en résidence du DAAD à Berlin. Professeur à l'École de l'Institut des Beaux-Arts de Chicago depuis 1999 et chercheur à l'Institut Orpheus (Gand) depuis 2016, il a été, de 1997 à 2017, rédacteur en chef du Leonardo Music Journal. Un des premiers à adopter les micro-ordinateurs pour la performance musicale, Collins utilise également des circuits électroniques faits maison et des instruments acoustiques conventionnels. Son livre, ``Handmade Electronic Music - The Art of Hardware Hacking'' (Routledge), a influencé la musique électronique émergente dans le monde entier. 

\noindent Site web : \url{www.nicolascollins.com}

\section*{Transcript}

\noindent Nicolas Collins, interview du 10/11/2017 à l'IRCAM, en marge du colloque ``Music and Hacking''.

VG — okay so I was saying that I make these interviews out of interest and curiosity for the fact that people making digital musical instruments or  playing them often have a very personal way of doing them, there's no not much tradition as compared to cello or whatever instrument... so the first question I ask usually is what in the first place led you to build your own instrument rather than using existing ones? what drove you to this weird activity?

NC — well that's a good question... This goes back before digital I have to say because I actually started out in music doing electronic music from very early on when I was still in high school, and by the time I got to university which was in 1972, there was a kind of a movement in America of homemade and handmade circuitry for music and the reason was that it was primarily economic which is that the electronic music equipment of the time, synthesizers, were too expensive for a person to buy. In other words studios bought them and pop stars bought them but a high school student couldn't buy a synthesizer. It wasn't like now where you can get a Casio, Yamaha synthesizer for, you know, less than you would pay for a trumpet, right? That was not the case in the 70s, so a lot of us started learning how to make circuits just for the reason of economy. But then a kind of a movement started about a kind of an alternative electronic music that was based not so much on using electronic sound to realize an existing vision but as David Tudor called it : "composing inside electronics" which is that we would make a circuit and then sort of figure out what the circuit did well and have the circuit as it were embed a certain amount of the score and structure of a piece as much as it would be the sound. So from a relatively young age for me building electronic devices was not a question of making an instrument like a violin but it was also about making a composition that the composition or the or the rules for the improvisation were built into that. So it was a little bit different from instrument building because the things you built were dedicated to a particular piece. As I say they essentially contain the essence of a work... so that is a little different from most of the history of musical instrument design because most musical instruments have been designed to be broadly useful, otherwise if you are in the business of making musical instruments, you will not succeed if you make an instrument that can only play one composition... yeah? Because only one person may buy it, right?... but if you're building your own instruments there was that character as I say that that was a... that was a sound device but it was also... let's call it a composition, for want of better term so that tradition was in my blood from as I say relatively young age like probably by the time I was 20 I was... I was thinking that way. So sometimes it's very impractical because it meant that, you know, you had to design one set of circuits to do one piece and something else to do another and if you're a traveling musician it means you have to travel with all this equipment instead of just having a flute and playing repertoire, right? So what happened is as you build up this sort of collection you begin to discover that... that, you know, this object and this object interact in a way that makes a third thing... You know, this was designed for one piece of music, this was designed for another and when I use the two together I get a third piece, you know? So sometimes we got a little bit... more than one application out of a given circuit or a given system and gradually I began to design things that were somewhat more like instruments in that I could use them in different pieces but it was kind of a backwards thing for me it started out with with circuits that were devoted to one piece and then they kind of broadened out a bit. So I, in the 80s, I basically designed two or three sort of things that you could think of as instruments it could be adapted to perform different pieces and the first were these instruments I called "backwards electric guitars" where they are guitars that have electro- magnetic resonators under each string so that you can play sound into the strings to vibrate the strings so that instead of strumming the guitar you used it as a signal processor so you could talk into it or sing into it you could play sounds into it and then by using your left hand you could change the filtering and the resonating of it... okay? Kind of thing you can do with digital filters these days but in those days it was a very unusual sound a little like shouting into a piano with the sustain pedal man but with able to change it... And I used those in several pieces both solos and small ensembles of these instruments... and then in the early 80s I got very interested in live sampling and signal processing doing live transformation of found sound material either coming from other musicians or I used radio very often, live radio, and I built a few systems and then I ended up building something that really was like a musical instrument and it was probably for your purposes the most relevant because it was a digital musical instrument and it was around 87 that I made this...1987... and I took a digital reverb an early digital reverb and I hacked into the operating system of the reverb so that I could drive various algorithms and processes that it did from a micro computer that I put inside the box actually a Commodore 64 of all things, and to control this I wanted something that was bigger than little knobs and sliders because it was at a time in the music scene in New York where kind of post punk music was a big thing and there was a lot of very visual action on stage, a lot of violent action, so I wanted something that was bigger, something that would make me visible and I was going around my loft saying: "I need a really big slide pot, I need to slide pot that's like this big" and I thought "oh a trombone trombone is a big slide pot" stupid idea but um so what I did was I took what's called a optical shaft encoder it's like the data wheel on a rack mount device — it's half a mouse basically half a mouse — and I coupled it to the movement of the slide with a retractable dog leash. This was a very mechanical system. So that as I move to slide the knob turned and this was read by the computer so you always knew where you were. And then I put a small keypad on the slide with like 20 key-buttons that I could press. And I wrote a program. It was basically like a graphic interface without graphics where I would press this switch and I could click and drag the pitch, or the length of the sample, or the frequency of the cutoff filter and it was just like clicking and dragging on a computer screen only I didn't have to look at anything because I knew that this switch was always the button for pitch and this was always for something else yeah? 

VG — Very direct mapping ... 

NC — Very direct mapping of keystroke and mouse. That was it. It was essentially, you know this was early Macintosh days I was basically taking another road, another branch on the Macintosh interface. And very nice because of course you didn't have to look at a computer screen to perform so you could be free. And then what I did was I put a loudspeaker on the mouthpiece of the trombone, so that the sound could play back through the instrument. And what this meant was that I could aim the sound anywhere I wanted. Right? In other words it wasn't like fixed coming out of a speaker I could walk around anything. I had an acoustic presence on the stage, so if I was playing with another musician we had this acoustic identity together, even though it was an electronic instrument. When I moved the slide of course it filtered the sound, because it's like a resonant tube that we're changing the center frequency and I could use a mute to do kind of like a wah-wah filtering as well. So it had a very vivid acoustic presence. And what I did with the instrument was live sampling of other musicians and it was very fast I mean was instantaneous to go "pshht" and make a loop and do stuff, okay? So whereas I developed this like all of my hardware and software, I developed it for one specific composition, okay? I can send you the URL for it it's called "Tobabo Fonio" and it's processing of recordings from the peruvian altiplano of brass bands players. But I found that it was a very flexible instrument for working with improvising musicians. The thing is that this was a time when there was very little being done actually with live electronics and improvised music. I mean it was, yes amplifiers and maybe a few effect pedals, but very little and I would go out on stage with another musician and I would grab the first few sounds they made, maybe they were just tuning or, you know, checking the instrument. I would grab it and then I would do variations on it for three minutes. Speeding it up, slowing it down, make it go backwards, all the standard vocabulary of sampling — but live. And instantaneous. And it was very flexible, very playable and it had this quality of fitting in, in the context of acoustic instruments, where a lot of computer music and electronic music, where you try to combine acoustic instruments with electronics, there is always this very clear distinction between, you know, this electronic sound coming out of the speaker and there's this beautiful beautiful cello on the stage and they never quite get together. As it turned out, I also had a line output from this instrument, you know, that when I wanted to go loud, when I wanted something to be very Hi-Fi, I could, you know, move a button and send it to the speaker system. But the charm of it was, you know, doing essentially acoustic duets with people doing computer transform, which is a very odd idea and still is; almost nobody doing it. There was a guy talking at the conference yesterday about a piece using the same idea, but it's... you know, he had no idea I did it 30 years ago [laughter]. You know, he's just stumbled in right now. But it's rare to have that acoustic presence of computer sounds onstage. So that was... I think for your purposes, that was, you know, kind of a landmark computer music instrument — digital musical instrument, because it was, there was nothing like it industrially available. In other words if I had gone shopping and said I want an instrument for live sampling and signal processing I wouldn't have been able to buy one and much less would I have been able to buy one that had, you know, this sort of combination of being almost a musical instrument... So... I built it. 

VG — And do you think you would  have been able to sell it ? 

NC — One time, I worked very closely with the engineers of the company that designed the digital reverb — "Ursa Major" and they were a small company, I represented them through a job I had in New York I was friendly with the engineers when I ran into a problem I'd call them and they'd make a suggestion, they gave me documentation... One day they, a letter arrived in the mail from them, and they had passed on a letter that had arrived at their factory by some... it was sent to them by some kid in middle-school saying: "I've seen pictures of this digital trombone that was built using your circuitry, I'm a trombone player in a middle school band and I'd really like to get one of these instruments. Is it for sale?" So I have had *one* inquiry for a sale. 

VG — It's really spontaneous. 

NC — Yeah. 

VG — Without advertising for it?  

NC — Yeah, and it was written in hand on lined paper in a pencil it was like, litterally, the kid  was probably ten years old, right? So now, it ... er... again I mean maybe that's the flip side of my saying that you know these early instruments in my community were not just instruments they were compositions. And what that meant was that they were kind of limited, that, you know, I would build a circuit no one else would want this circuit the way somebody wanted a Moog, you know or wanted another a Theremin because the circuit was really not usable for lots of things. It was usable for what I wanted to do. This instrument, for me, it was important because I could do lots of different things with it and at one point I considered commissioning other composers to write pieces for it but it was really... still, it was kind of too personal... it would never have been hugely popular. The electronic end of it was, in the sense that this was like a precursor of all of the looping pedals that were developed, say, I don't know, during the late 90s and noughts. Looping pedals are everywhere now. Well this was sort of like the first live looping-pedal, yeah?  and it was way before any of those things. There was only one thing that was at all like it, which I had worked with, which was this famous pedal of Electro-Harmonix called the "16 second delay" which was never meant as a looping pedal but people figured out how to use it as one, and that was... I bought one of the very first ones that was ever made and it was after working with that, that I went on to design this system, so... But yeah, I could have taken the practical part of this project and developed a commercial instrument, but I think I would have been way too early in the development of the aesthetic of electronics in popular music to have had many sales. And as a friend of mine once said "it never pays to be too early" and I've always been too early, you know, I've always been too early and you never get rich being too early, unless you file fundamental patents and, you know, what artist has ever had the time to do that? So yeah... so that was, as I say the closest thing to a ... to an instrument. I can give you the URL, I wrote a paper on this, that you can look at. And then, um ... around the same time I started hacking CD-players to turn CD-players into sort of sample manipulation devices. And I used those in several compositions... manipulating CD recordings of music as a way to stretch it out. So you know, that was instrumental in the sense that it, you know, it did this particular thing but I could use it in a number of pieces rather than the one piece. But it never had the same flexibility of application as the trombone instrument did. 

VG — What do you mean by "CD... "  er... how did you use it ? 

NC — How did I do it? Right... What I did was... I had this idea... Yasunao Tone, very important Japanese composer working in New York at the time had done these things where he damaged the CDs with scotch tape and he made this very beautiful glitch music. His work was based on manipulating the recording, the CD itself. And using a stock player. I was interested in modifying the player to do ... what I wanted to do was I wanted to do DJing with CDs before there were DJ players... all right... this was 87-88 and I wanted to be able to scratch CDs and it took me years to figure out how to do that but I did figure out how to do this thing where you could put it in pause and continue to hear the sound and it would be sort of a suspension of the sound and a loop... you hear it when you're when your disc gets dirty and sometimes it'll get caught in a loop and I got so I could control it so that I could sort of suspend a little loop and then move forward a little bit, and move forward a little bit, forward a little bit and one of the classic clichés of the avant-garde is "slow it down" you know... That's one of the things we do all the time, we slow things down. And so I got very interested in taking recordings primarily of early music like late Renaissance and early Baroque music by period ensembles and stretching out these performances and using that as a backdrop for live performance by instruments. And you have to understand that within five years, this kind of stuff you could do with... well, not within five years, maybe within eight years... you could do it on a computer but at that time you needed a dedicated DSP to do this kind of stuff or a sampler you couldn't do it with the CPU on your computer. So the CD manipulation was sort of halfway between making a piece for instrument and fixed media, like what we used to call "instrument and tape", you know, where you have a tape playing and the instrument play, and true interactive music like we would have by the end of the 90s where the computer would listen and respond and do stuff... before the computers became powerful enough to do that, this was a way of sort of having a manipulatable backing tape where you could kind of change how often it would change and everything like that... 

VG — There is something with what you describe for the trombone that strikes me is that the mapping that you made is really direct and simple... Well, simple — I don't know but very direct. And it is something that is very often said in the litterature about digital musical instruments that er... direct mappings don't work so well... 

NC — Ahaa... 

VG — That you can't map one fader to one sine wave or something. And... this is said more often than the opposite, at least. 

NC — Well, it has to do with the nature of the interface. In other words the problem is that a mouse or a trackpad is really only good for you know one parameter at a time. Maybe two. It's very difficult to do multi-access control with a standard computer interface. So it means first of all you have to use an external interface. And thinking commercially, if you're making software for a computer you want it to run on any computer you don't want it to have to have a custom interface to work with because that will limit the number of sales you have because, you know... If they have a choice between two recording and editing softwares for a computer, and one will work immediately when you put it in and the other one says oh you have to spend 300 euro on this external controller or it won't work at all *but!* once you have it, then you have direct control. You would think that the direct control would be a better deal but people will always buy the cheap one, yeah? And I think that the development of computer tools has generally been for non-real-time  production... yeah? In other words, hard-disk based editing and workstations is what's called "nonlinear editing". In other words you don't have a strip of tape that's in sequence, you're jumping around you're doing everything. And if you don't care, if it's not really that important how fast you work, you don't need to have 28 faders for the 28 tracks because 90 percent of the time, you're only moving one fader at a time, so why pay for all those faders? So you know you run a mix and you say "oh you know I think that second track should have been a little quieter from 30 seconds to 33 seconds" ... So you go back and you fix it. In other words they have developed tools to get around the problem of direct control and save people a lot of money and investment in alternative technology to interface to the system. But performing music is different. Performing music is a real-time event. Yeah. You can't ask the audience to sit there while you go back and edit something. Do you know what I mean? And performance traditionally... most musical instruments are about direct control... right? And I think that .. you know, I was watching these videos yesterday of two projects here at IRCAM where, you know, they're using a graphics tablet whose advantage is that you can have multiple controllers on it and the other thing was with something working with a Kinect for visual tracking, which meant that you could you know track two points multiple axes, you know, these things are things that musicians are thinking about... and it's just that ... em... yeah again maybe I was just very early with the idea that what I wanted was to have instant access to all the parameters of the DSP without having to say navigate through menus or anything else like that. And, you know, for example your typical rackmount effect processor, like an Eventide harmonizer, you have one wheel and you have a few buttons and you have a hierarchy of menus, and it's designed so that the things you need to do most are button click and the wheel. And then the things that you don't have to do quite so often, you have to do two button clicks go to another menu and then use the wheel and they design it so that the deeper you go it's the stuff that's less often accessed. If they had 28 buttons on the front of the thing, a) it would cost more, from a hardware standpoint and b) it would be more confusing for most people because they don't want to control all those things at once. But I don't think you want to tell a pianist who's coming out on stage: "All right, look, we're gonna cover up a bunch of the keys because in this piece you're not gonna play them and then we'll take the covers off for the next piece and put them on something else. You won't need those keys, will you?" That isn't the way instruments work! 

VG — That's a good comparison. [laughter] But... er... I'm not sure it was meant this way in the litterature I was talking about... I was thinking of articles that were published at the NIME conference, that is dedicated to mostly live instruments, and for example, there are a few papers actually that say that if the mapping is too simple, the result will not be rich enough to be interesting... 

NC — Right, are these players ... writing these ? I mean you know it's... I deal with this all the time there's what we call a "sweet spot" in terms of the nature of mapping and curving and everything else like that and... in most conventional instruments it's been worked out over centuries, you know, of like, exactly what is the tuning like, what is the response, what is the touch everything else... We're in the, you know, when you're designing a new instrument, you know, you don't have that history, you don't have that big database of users who have said, you know, this works better than this, you know, I mean look at the evolution of say like, I don't know, the flute or the clarinet, you know, and you see all these like little alleys people went down... the "klappen-trumpet", you know, evolving into the trumpet, I mean it's like, all this weird stuff, but um... were you at the talk that I did yesterday morning? 

VG — Yes 

NC — OK, so I showed a picture of the trombone and I also showed a picture of a new instrument that I've made based on a trumpet and I have these... unlike the trombone which had something like 28 switches on it, there's just, I think, there's just six buttons on this. And I went through a few different ideas about how to make it work as an instrument. And the first ideas were always too complicated because you know one button would change what the other buttons did, like a function button and this and that not... and it was... you have to kind of like think to make your way through it in a non-intuitive way and I ended up bringing it down to a much much simpler, a much much simpler set of mappings but I could only do that because I changed very much my idea about what the sort of sound property and the kind of nature of the interaction was. I decided to pull back and have less direct control, have more nuance of its own... yeah? and the other thing is that I was working with these sensors on the valves so that I could have basically three axes of continuous control at the same time. So with a little thinking I was able to have a much more direct mapping where there was less steering by buttons. And it just became a somewhat more fluid instrument. I don't ... it's a difficult thing to articulate but I think that if you get... if you give somebody too many choices of things to do, it will not be an expressive musical instrument because there's gonna be too much thinking. If there's too little, it'll be like a snare drum — or worse, a drum machine snare drum you know, which is all you can do is press the button and it always comes out exactly the same. You have to find... you know, you have to sort of find a spot in between... yeah? I mean think about,  for example, um...  think about if for an electric guitar instead of just plucking and fingering you had to move the fret to wherever you wanted it to be for that note, you know, like on a sitar, or you know, you had to put the chord into three different jacks depending on whether you wanted to send it to this amplifier, this amplifier, or the PA... You know, these things would slow you down ... you know ... 

VG — Yeah, that's one of the drawback of digital stuffs...  one of the main drawback is this non-directness of tools ...

NC — Right 

VG — ... that you have to start the computer, launch the program, open you patch, recall the settings ...

NC — I wrote a paper ... 

VG — That reminds me of a friend's quote about how to make a good digital instrument design is when you are able to play it drunk... 

NC — Yeah! ... No, exactly... exactly... um I wrote a paper in the early 90s called "Exploded view" that was about how MIDI had done, was it had exploded the musical instrument that it used to be, that everything was integrated, in other words that there was a string that was making sound and there was a finger board that was determining pitch and it was all combined into this one ... you know, like, like self-contained system and now with MIDI you know we have one instrument that's just a controller and the media is coming out and it's going to a thing that's generating sound and it gives you tremendous power on the one hand, you know,  because it means that if you have technique on one instrument you can play all these other sounds but at the same time there was none of the same feedback that you would have from a conventional instrument and I mean, again, you know, with the electric guitar ... the electric guitar lives in close proximity to the amplifier and every good electric guitarist knows how the guitar and the amplifier in the body interact to form a complete system... well... if you break it up into more separate pieces you know, a fingerboard, string-sensors, a string-synthesizer, a DSP, an amplifier... you tend to lose that clear-cut feedback network and you have to start designing haptic feedback into the system ...

VG — that makes me think of another question that I usually ask to people about ... I don't know how much it applies in your case, in your music but... many people using digital musical instruments, when they make a concert, they have several tracks or songs or whatever pieces, and ... um ... with those digital instruments, you have the ability with one click, to completely change sounds and mapping between the sensors, so you have to figure out your own way with this issue... and there is not only one way, I think, to go from one end to end of a concert, so... is it relevant in your case ? From what you told me, I tend to realize that you mostly use one self-contained object with... 

NC — yeah and very often my instruments have... have a very... em... have a very limited sound palette, yeah? In other words or they have what we might call a limited instrumental palette which is to say they produce a particular type of sound or a particular type of process and they don't do anything else so, for example, this...  this hacked CD-player only does one thing which is it does this sort of looping and drawing out of sound... now, if I put a baroque music CD in, obviously it's gonna sound different than if I put a heavy-metal CD in but the process will be the same, you know, the nature of the transformation... with the trombone it had a finite vocabulary of what it could do you know, it could could make a loop, slow it down, it could speed it up do various sort of multi-tap processes which were very beautiful in terms of changing the sound but it couldn't, for example... I don't know, break the signal down into constituent sine waves or, you know, do a Fourier transformation or vocoding or you know, track the pitches of the sample that I made and map them to an accordion sample, you know... no... there were lots of things it couldn't do but as a result as a performance instrument, it was very reliable, it was the kind of thing that you knew what you were gonna get, at a given moment. 

VG — That's somewhat of a design decision 

NC — That *is* a design decision... so for example, if you... 

VG — you could have make this instrument evolve, could you? 

NC — um, well... at that point uh... I was up against the very limit of what you could do with DSP. I don't think I could have gotten many more types of processing because nothing was available... so it was self limited but on the other hand it made for a very performable instrument you know, it's a little bit like if you're a guitar player and you're playing a six string guitar and you press a button and suddenly three more strings are added to the side of the neck, you have to think before you go, right? now if you're playing a keyboard and one time you have a sample of the piano, yeah?, and the next time it's a sample of a hammond organ... that's less... of a break but at the same time pianists have a very different touch on a keyboard then an organ player has... and to play a Hammond organ and to play a piano require thinking about your physical interaction with the keyboard and what kind of gesture will become what ...  and switching between the two ... musicians would talk about this in the old days when you'd have you know like a piano in an organ, or a piano in a synth or something, you know... they talk about, if you've got them talking about it, how they had to kind of like just think a little bit about the performance styles differences, even though it's still a keyboard, right?, so you have a MIDI keyboard connected to a number of different things and you know if you're playing, say, you know, a piano and the next you're trying to do a bass line and it's a different kind of articulation because you're trying to get pop bass or you're doing percussion samples for a piece, you're gonna have to bring a different technique to bear... So I think that there is the problem when an instrument gets too flexible, that you lose the ability to be virtuosic you know that it can stand in the way of attainment... Now you know, one of the people that I'm sure you've looked at in terms of digital musical instruments is Michel Waisvisz... and Michael developed the first version of his "Hands" I think in the mid-1980s maybe 84-85, something like that, and continued to work with, you know, subtle variations on it until his death... and after a few years, people said : "So, when are you going to design another instrument ?" and he said "I don't want to design another instrument,  I want to get better at playing this one." yeah? ... um ... that was  an instrument that had a lot of sophisticated mapping and variable mapping because, you know, he didn't have a keyboard he could go all over, he had to limit his buttons to what he could fit within two hands on ping-pong paddle type devices... he had these other factors of the tipping of the thing and how far apart they were spread but you know, he spent a long time thinking about what will I map to what, how much control do I want, how sensitive do I want it to be, you know... how far do I have to move it to effect a change... and then he spent years practicing and making pieces for it that, you know, each piece probably involved changing the instrument slightly, but always keeping a core structure that he was familiar with. 

VG — One of the feeling that I have regarding  this question of versatility of... and ephemerality of the thing that you have below your fingers is that it somehow... em... displace the things that you have  to learn and remember. I remember discussing this issue with people about why there is no or very few instruments in this digital realm that last in time more than 10 years apart from the keyboard... which has this piano-cousin... there are many, many (instruments) coming out on the market which fail to succeed ... and... so this question of what you learn when you learn  an instrument which is a set of many things, you learn gestures, you learn the timbre of your instrument the sensibility of your sensors and I was discussing this with another friend who is musician and he uses many different types of interfaces, he was telling me that, somehow, processes that he used under the various interfaces, he got to know them, that ... like if you play with FM-synthesis you can play with a joystick, or with a FSR sensor or whatever, that would make lot of changes,  and some things you will have to re-learn, but there is a common thing to them... that you know that the FM-synthesis will be that shaky at some settings and that stable at some other settings, so you kind of know the sound  generator as an object even if its a purely virtual algorithm... this kind of mental switch  between virtual instruments that you may use in the time of the performance and ... I don't know at all how Michel Waisvisz used... because he had several instruments, like, he switches between instruments... 

NC — towards the end of his life he developed... well, he developed this... a variation on the Hands that was a conducting instrument that was sort of a stripped-down version of the Hands;  fewer direct controls and ... then he did something that was based on the idea of a spiderweb with four sensors on the threads, it looked a little bit like a funny harp and... I don't think he ever made terribly effective use of that as an instrument... um... there's an expression in American-English maybe even English-English, from the music industry called "the second album syndrome" that for most pop bands the second album is always a disastrously bad record and the first one can be a magnificent success and  the second one is inevitably terrible... and it's called the curse of the second album  and it's like everybody has to get through it and if the band is lucky they go on to make a third record which is a little bit better but it takes them years to be as good as their first and I think when you design instruments, I think most inventors will probably tell you the same thing, you know, whether they're designing toasters or you know something else, that their second invention is probably a dog, you know, and I think that Michael stuck to his idea about becoming a virtuoso on this instrument but then at a certain point said "okay, you know, I now know how to play this instrument, I'm really good, I'd like to do some more experiments you know he was at STEIM when all these crazy controllers were being made so he sort of would look at what's going on and said "you know that's kind of interesting, I don't have that in the Hands, that would be something interesting"... so you know that's the motivation... why the instruments don't survive is a very interesting question, okay, and it will be interesting to look in a hundred years to see, for example, are people still playing the Theremin? which is probably the oldest electronic instrument that's still in production and has a base of users. It's not as popular as the saxophone but the average you know, educated music person when you say Theremin they'll say "oh yeah that thing you wave your hands around" whereas if you say "Nick's, you know, trombone propelled electronics" they'll say "what?" you know, it's like... no!... it's like a completely different world... you know, I think there have been these drum pad controllers that I think are kind of a sort of like, they are to drums what a MIDI keyboard is to the piano, you know, in other words they're so closely related it's you can barely call them separate instruments and a lot of the drum controllers now basically use drums and just put sensors on the heads you know, because people don't want to play some weird rubbery thing... The guitar controllers have all been miserable failures, you know, none of them seem to work, even though potentially it's a huge market... huge market, you know, if you could get one that worked well, you get a lot of people buying it because electric guitars are still you know the hugely popular instrument wind controllers, you know, brass and wind controllers... very small market for them and again they are to the brass instruments and the wind instruments what the MIDI keyboard is to the piano, that is to say, they only succeed because they're very close to the instrument they model, but you know, I mean you've probably looked at the instruments that Don Buchla invented, the Lightning which is his conducting instrument, the Thunder which was a sort of a very very touch sensitive control pad system... I have no idea how many of those he sold... not a lot... they were you know, the conducting instrument bore a resemblance to conducting but was really you know, quite different; the control surface was very different from any existing keyboard or drum pad... not a huge market for it I think that the ... kind of proliferation of affordable electronic instruments, synthesizers, in the 80s was followed so quickly by sequencing and ... digital audio workstation softwares that I think that we now have a split in a sort of ... economy of music where, in certain worlds most notably classical music there is still a lot of performance taking place, you know, symphony orchestra still exists, opera companies exist, string quartets exist and tour, pianists tour... more and more pop music is a studio based practice and when you're producing music in a studio you don't worry so much about the instruments, you know, in other words there's a lot of non-linear music being produced now where you know, the drummer is not drumming, you know... they're not putting down their drum tracks by hitting buttons in real time, you know, everything is being done in an editing, oriented fashion... um... an awful lot of the work is removed from the traditional idea of performance and I think what it meant was that just at the time that you might have seen a proliferation of new electronic instruments, the need for instruments sort of went down, you know, because ... from a practical standpoint it's ... it's easier to perfect that music in a non real-time way, you know,  it's easier to do those drum tracks with editing than it is by hitting drums unless you're a great drummer... yeah? ... even vocal performances now as you know are so highly edited and processed that you know we're really only this far off from having it be essentially a sample based technology and techno it's been that way for years...  it's just little shouts and hits you know... 

VG — aligned text-to-speech.... 

NC — Yeah, or it's somebody singing once. 

VG — There's a whole scene in Japan ...

NC — Oh yeah I know ...

VG — ... with stars and fans, so yeah, it exists but you're right that there is a real cut between  electronic, or not even electronic, music production and ... "instruments" or so called instruments 

NC — oh yeah and so for example I think IRCAM is an interesting place in which to be asking these questions because historically it's had a very high level of institutional support for extremely serious live performance of music incorporating electronics and looking to get beyond the kind of excellent European model of live instrumental performance against a backing recording and you know, starting with the 4X, clearly a huge amount of effort has taken place in IRCAM to get this stuff working. Now as it turns out, and I'm sorry to say this from within the hallowed halls the advances that have been made in the commercial music industry and other research facilities like STEIM for example have taken place much much faster than the work here at IRCAM. In other words, outside IRCAM we're miles ahead of what's being done here but it's being done for rather different communities and I think that the sort of the community of kind a more academic classical -contemporary classical- composition is going a little more slowly and perhaps more methodically  at looking at this 

VG — more conservative ...

NC — yeah ... I think it's a question of ... people want a kind of a certain reliability and proof to stuff I think there's less interested in experimenting with the technology per se they want the end product whereas outside you have two things : number one in commercial pop music the economic rewards are so high because the sheer number of buyers out there that you know stuff is being pushed out much faster... the development of these things just takes place much faster ... um ... and then you have a bigger user base that shows you all the variations and all the bugs and everything else like that so stuff gets very very ... 

VG — there's a real community 

NC — yeah... it is a lot of community support, I mean, you read you know reviews of looping pedals and within four months another company has come out with a pedal that responds to the critique of the previous one, you know, I mean, this is very very quick moving but .... you know it's... I think one of the questions that you have to face doing your work is what the role of the instrument is in different musical communities because it's very different here inside IRCAM for very similar technology as it would be in a techno studio in Cologne, on the stage of CBGB in New York if it were still alive, you know, at the Kitchen in New York in other words, find like six different venues around the world that are presenting music of some sort using electronics and think about what is the consistency of how we view a musical instrument in those different in those different places and as I say I think non-linear music production has just changed the equation hugely... 

VG — yeah, that's for sure... to me this is obviously a big topic in my research, that the frontiers, the borders, of what we call, what we may call, a digital musical instrument are very blurred,  and should we call an iPod... to paraphrase John Cage asking if a truck passing in the street is more of a musical event if it is listened in a conservatory of music... is downloading MP3 a musical activity ?... so the number of situations we are facing, not necessarily listening or participating, but dealing with music has increased so much, the number of devices which allow us to interact with music has also increased so much that there is a diversity of situations which is just non-quantifiable ... 

NC — and also there are very different set of problems emerging, so for example, I'll give you a few examples, I started teaching for the first time in my life in 99, I'd been out of university for 20 years and I took this job at this art school, School of the Art Institute, in Chicago.  These were artists, art students, not music students but they were working with sound, a lot of them chose to come to Chicago to go to school because Chicago is a very important city for music, for a large number of communities. It's the birthplace of a lot of african-american music forms in America every post rural blues music form in the african-american music culture is essentially originated in Chicago, or has a strong connection to Chicago from electric blues through various stages of so called jazz music... 

VG — to techno... 

NC — ... to techno and that's the other thing very big, especially at the end of the 90s for techno house garage and this funny sort of indie pop music the band "Tortoise" was a sort of classic example of record label called Thrill Jockey... "Clicks and cuts" was a Chicago invention I mean all of these these things... so... the kids come there because they like music even if they're only amateur musicians I have this seminar, undergraduate composition seminar, and there are people working on soundtracks for films, there are people doing sound sculpture installation and soundtracks for performance, some are doing essentially contemporary music composition though they don't call it that and there's a bunch of house producers ... and what's interesting is that most of them can produce a reasonably competent backing track. That is their rhythm construction and their basic bass lines and jabs are solid. They have no problem. Why? Because the software for doing that is by that time very very good, you know, your looping and editing software for doing rhythm based metronomic dance music has reached the point that you don't really have to have a lot of skill to do it you have to use the same editing skills you would use for editing a video to do this you know we know that has nothing to do with music per se or needless to say it has nothing to do with soul ... right? On the other hand,  not one of them could write a hook for the singer. Not one of them could write a three note melody that you would want to listen to more than once. And I went nuts because after like listening to 20 of these things I finally just got... I yelled at them I said "if you cannot write an interesting melody, don't put one into the song" and it wasn't like I was expecting them to be Schubert I would play them techno that had a three note hook that worked and I would play the mere track that had a three note hook that didn't work and I'd say "why can't you do that?" you know what they said ? "we need a class on how to write hooks" so I had to hire someone to write a class on how to write hooks and the reason was that that was like an old-fashioned musical idea that was not part of their background, you know, if I was doing the same class at a conservatory, maybe they would be rhythmically incompetent but would be able to write the hook, I don't know! All I know is that there was a displacement between the way the technology could speed them up in certain aspects of what they were doing and not help in other ones. You know what I mean ? 

VG — too specialized... 

NC — yeah... and then it was interesting because of course then we would look deeper into what they would do and you know they would play their track and I'd say "bring in your favorite track at the moment" and then we'd sit down and we'd analyze the two and I'd say "I want you, in your track, to tell me where the cowbell falls" and they'd say "it's always on this beat and  this beat" and I'd say "all right now listen to the track you love and tell me where the cowbell falls" and they say "oh! it moves..." aaah...it moves...  In other words you start to rely upon the mechanism of these tools and your music goes in a particular direction because it answers questions for you, it makes decisions for you. So,you know, each one of these advances brings one in the same time kind of liberation and power and the other hand it kind of suppresses certain decision-making processes. 

VG — yeah, thats one global issue about those digital tools, about technology... Stiegler, the philosopher, deals with that issue that we are ... being handicapped... we are becoming handicapped and the fun thing with disability we are all facing is ... I've been working on a number of projects which were dedicated to the accessibility of blind or deaf people  for music and the solutions that we find to improve the access they have to music also work for non-disabled people... You can always ease the work that you have to do to to reach a goal, like for example, I remember this MIT team which has developed glasses which would transform color into sound for color-blind people but actually the sensors are able to detect ultraviolet and infrared so that this person now sees more than non-disabled people can see... so the limit... the difference between enhancing the body and the... where is the limit to a normal body ? oh I wanted to say something about that but ... what I wanted to say had to do with with... yeah ... I wanted say that you develop things to reach a particular goal — yeah — these tools but the thing with music is that the goal is always somewhere else somehow... you said it somehow relate to sex in this irrational thing that there is no such goal for perfect music or perfect sex,  and it always move to somewhere else to a new place, to unexplored places... So all the tools that we might develop to ease the task of making music is always failing at producing innovative... 

NC — well, yes and no, I mean I think that kind of post post electronic music has made very interesting misuse of Technology you know, that that is sort of, I mean you know, everything from the 808 in techno to a certain type of samplers and synthesizers by being pushed to the extremes of their behavior have become signature sounds in particular styles of music ...um... the guitar amp in an over-driven state is not something that the engineers who designed the early guitar amps wanted to have happen and yet it became the signature sound of electric guitar. Look at the prepared piano in classical music you know the early developers of the of the pianoforte would never have anticipated anybody doing anything like that to the instrument, it's you know, I think a lot of people still think it's horrible misuse of the instrument and yet you know, those sort of adaptations have been powerful liberating forces and in some cases have led to an entirely, you know, it's an entire new genre of music yeah right, at the same time you know there there are drawbacks a lot of musicians will talk about auto-tune and you know what it's done to pop singing you know, as it generates an artifact you know, it generates an artifact there's no question, and yet people will accept that with the interest of sort of having better pitch there's beautiful research has been done on the impact of recording on performance style and how you know, you have pianists who had a very long career and their career spanned from before recording to two or three stages of recording and you have them heard on recordings that were made, say, 20 years apart from each other over a 60-year period and there's been this comment made that they became more conservative in their performance style after the advent of recording because of course before recording nobody remembered your mistakes if you made a mistake in a concert maybe three people in the audience would know and the critic might comment on it and there are lots of pianists for example it were famous for hitting wrong notes all the time in performance you can't get away with that in a recording studio because it's gonna keep coming back people will hear that note again and again and again ... so what happened is that you had less rubato, you had passages being played slower than they would be on stage and this and that... adapting to the technology ... so yeah you know, at the same time this is the technology that brought music to the masses and brought profit to the musician you know, so it's not a... it's it's not a good thing or a bad thing it's a mixed thing and I mean this is a little bit off the topic of what we were talking about, I mean initially we were you know, trying to adjust sort of the more overt technical issues of what makes a digital instrument, what it is, you know, how do you limit it, how do you facilitate performance and how to be flexible but you know in the end, I suppose you have to see all of these parts in terms of the bigger picture and the bigger picture is you know, sadly it's the economy of music, and that can happen in a number of ways, number of levels it can simply be you know literally how many records will you sell if you use this technology versus that technology, yeah, or by using this technology will my music change in a way that defines a big thing that people love or something that people hate, yeah, in other words every bifurcation in the evolution of dance music in the last 20 or 30 years you know, you see one spin-off that dies and another one that flourishes okay, and then in non-commercial music it has to do with you know, the way you value your own work in other words I, as a composer, I work at a piece, I perform a piece some pieces bring me greater pleasure the act of presenting them than others and I look at it and I say well why is that ? you know, is it the sound world ? is it the nature of the performance experience ? is it the nature of the audience response ? sometimes my public and me are in agreement about what is good yeah, and sometimes we're not sometimes I'm convinced that this is really a good piece and the audience just doesn't get it okay, and other times it's the opposite I'm thinking "that's the one they liked??" you know, that's kind of the stupidest thing I did all night and that's the one they remember... everything seems to have that sort of like economic trade-off in the in the larger sense of the term 

VG — Yeah... maybe I should ask one last question, I think we are late already, yeah 30min late, so, I usually ask people a tricky question about making a pronostic or a vision of... 

NC — ... the future ? [sigh] 

VG — ... but you made it the whole time  during the discussion so... 

NC — I think that... yeah..  I really do think... that... we're in a very odd position at the moment in terms of this sort of the evolution of music technology, which is on the one hand I'm quite serious when I say that you know more and more music's going to become non performance-based and therefore all of this obsession about instruments is sort of gonna disappear in a large amount of the community, people aren't gonna worry about it, stuff's gonna become very generic, yeah, in other words it's like, pop bands in the 1960s didn't worry about the instruments; yeah, guitar, bass, drums, keyboard, yeah, boom! that's fine, we can do it! synthesizers in the 80s everybody was "oh are you using this using this, using this..." I think that's gonna kind of disappear, except for you know, a very small specialized sector of the population that's interested more in live performance but live performance is going to be less and less and less of our music world... I think that the advances in technology are not going to be in instrumental performance but in non-linear work those tools that improve and speed up the production of music on a computer, for distribution as a sound file and whatever it takes to make that work better just the way people are working on making better search engines, or you know, better word processor, better spreadsheet programs, it's gonna be in that market... it's not going to be in the market of Stradivarius and and Leo Fender... yeah? at the same time when I was at STEIM we had the sensor lab and this was essentially the first Arduino, yeah? and it cost three thousand guilders, that's sixteen hundred euro and you had to kind of work with the STEIM engineer for a few weeks to figure out how to make an interface with it... yeah? very robust, very reliable, but ridiculously expensive. Arduino comes out, not only is it cheap, you don't even have to know how to program to use it. Why? you know I have an art student,  never programmed before gets an Arduino says "I want to control the speed of a motor" they type "motor speed control Arduino" into the search engine, they copy, they paste, they download, it works. And maybe you know, they have to change the speed and they post a question "how do I change speed on Arduino?" they get an answer, they do it. In other words you have this very very easy entry point. This place [IRCAM] is a little odd because it still follows this old model of the engineer does that not the composer you know, but... [whispering: In the rest of the world the composer does it!] ok? and what it means is that at the same time if you are doing live performance if you do need a specialized instrument it's ... it's almost more like cooking than it is like building a musical instrument you know, everybody cooks ! you don't think "oh did you go to chef school ? you made me a spaghetti bolognese... did you go to the cordon bleu school? - no man I just cook you know, I got to eat, I got cook."  Then I think that you're gonna have much more sort of low level,  low pain construction of alternate interfaces and some of them may be  very simple it may just be a button, a pressure pad... some people  may build up something you know, like that looks like a damn sitar you know, but it's a very fluid thing and it doesn't necessitate an institution. It doesn't need IRCAM and it doesn't even need STEIM anymore, you know, STEIM was always like the budget-IRCAM; we did 60 projects a year when I was artistic director... you know, you don't do that at IRCAM. We did that at STEIM. No, it's not really necessary you don't need to come and work with engineers to make this stuff happen, okay? but here's the other thing I was gonna mention when I talked about the problem of my techno producers not being able to write a hook; you know, if like, where did ... where are these weird problems coming from ? You know my students tell me the biggest problem they have in music production is finding their samples on their hard drive. They say "I have so many kick drums on my Drive, that I've downloaded from so many places, that I can never find the one I want. Now that's really weird. That's like the first time you ever look at a harp and you say "how do you know which string is which ?" you know, you got so many strings, right?... Guitar! That's why people like guitar and bass... six strings, four strings, I can deal with that. Harp?... Phew! You know... who would have thought that that would become the biggest musical problem ? ... not staying in tune, yeah, so now... 

VG — it's the library of Jorge-Luis Borges,  where people are in this infinite library containing all the possible  books in the world... 

NC — Exactly!... Exactly, it's the digital version, it's Borges' light, you know which is that... who'd have thought that, you know,  now maybe it means that you don't need Stradivarius you need a database programmer, that's the most important thing in your life would be like, a really brilliant database that would allow you to intuitively retrieve whatever sample you wanted... yeah?... but that's not an instrumental  idea right? In other words that's in such a different domain but it may be, as I say, the single most important thing for a composer working today ... Funny idea, huh? [tacet] 

VG — Okay. Should we go to the opening ... 

NC — Go hack ? Let's go see what's there! 
