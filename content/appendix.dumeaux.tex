\chapter{Interview : François Dumeaux}
\label{appendix:dumeaux}

\section*{Biographie}


\section*{Transcript}

François Dumeaux, interview du 30/07/2017, dans son studio, Cuzorn, France.



 séquence musicale 

VG —  C'était très bien de commencer par une séquence musicale car cela me permet de te poser tout de suite la question de décrire un petit peu cet instrument, ces instruments, je ne sais pas comment tu les appeles, s'ils ont un nom 

FD —  alors celui là, je ne fais pas preuve d'originalité, je l'appelle le synthétiseur modulaire … mais j'y adjoins cet instrument qui est un instrument de musique traditionnel qui est un tambourin à cordes que j'ai détourné complètement de sa fonction de départ. Normalement, ça se joue avec une flute harmonique et ça sert à faire un bourdon rythmique, et moi je m'en sers pour faire de la matière… Mais pour moi, c'est comme s'il y avait un arrière plan imaginaire de tout ce que ça draine d'histoire, de projections… même si au final le son que j'en sors est assez éloigné de ce pourquoi c'est fabriqué au départ 

VG —  dans le son et dans l'objet, tu veux dire ? 

FD —  Oui, voilà tout ce que ça peut représenter, pour moi hein, ça nourrit mon imaginaire. Je ne sais pas, c'est comme s'il y avait un lien avec qqch qui fait vraiment partie de ma pratique musicale au même titre que la musique expérimentale, il  y a la musique trad et j'ai toujours fait les deux, mélanger un peu, et plus ça va plus je mélange les deux en fait...Je fais aussi des trucs purement l'un ou purement l'autre mais les deux m'intéresse, en fait. Plus ça va et plus j'ai tendance à effacer les frontières entre mes pratiques. Donc là, ce que j'ai utilisé là c'est ce que j'utilise le plus souvent. Tu vois, tous les modules ne sont pas patchés, il y en a dont je ne me suis pas servi. Et ça, c'est ce dont j'aime bien me servir quand je fais de l'improvisation… donc c'est des oscillateurs, il y en a des analogiques, il y en a des numériques… et c'est un peu les deux opposés, c'est à dire, il y en a des très simples, des sinus, et il y a des oscillateurs à table d'onde… 

VG —  dans lesquels tu charges des samples, tu veux dire ? 

FD —  non, il a toute une banque de … je sais plus combien il en a, 64 je crois…. (les faisant écouter une à une) on balaie …  

VG —  ça interpole entre les tables d'onde ? 

FD —  voilà c'est ça… il fait du morphing entre les tables d'onde… voilà et puis après il y a des trucs de modulation, de l'aléatoire, des cycles très longs, et puis… euh… et puis j'utilise celui-ci (joignant le geste à la parole) qui … euh… un truc qui marche avec l'électricité statique (petite séquence de jeu, sons très bruitistes, corrosifs, à changements rapide)… voilà… ça je l'utilise beaucoup en improvisation… et je sais pas si tu as fait gaffe, il y avait un paysage sonore…de vent…  

VG —  oui j'ai entendu 

FD —  et alors c'est ce module là, qui est un truc qui était un kit… que j'ai soudé moi, ça s'appelle le « radio-musique » et c'est basé sur la pièce de John Cage… c'est l'idée, tu sais, de composer une pièce où tu … où dans la partition, tu avais « ouvrir la radio à telle fréquence, à tel moment» … et alors comme ça a beaucoup changé la radio et même que bientôt ça risque de plus exister, la radio… analogique, ils se sont dit, on va faire le même  truc mais avec nos propres banques de son donc tu coup, tu as… tu peux avoir plein de canaux différents (séquence de jeu) là par exemple, c'est une banque, c'est que des paysages sonores et je joue le tuner, je vais passer sur un autre paysage sonore… et pendant ce temps, l'autre il continue comme si c'était une vraie radio, donc si je reviens sur  l'autre, ça reprend pas la lecture où c'était, ça a continué entre-temps de compter…des cycles… (faisant la démo, sons d'oiseaux) …  

VG —  c'est radio-John… 

FD —  ouais… (rires) … alors évidemment, c'est de trucs que tu peux moduler donc tu peux … déclencher… un changement de fréquence .. et moi je m'en sers surtout comme d'une banque de sons pour… euh… de la même façon que quand j'utilise un ordinateur pour faire des citations parce que j'ai envie sur le moment… je l'ai sans avoir besoin de trainer un ordinateur … et je disais il y a plusieurs banques.. là c'est la banque paysages...(extraits)… bon là j'ai ça.. (bleeps)… là c'est des instru acoustiques avec pas mal d'instru traditionnels… ça c'est de la flûte… des séquences de jeu que j'ai enregistrées… du stick, une cornemuse landaise… pas tout à fait jouée comme prévu mais… voilà… tu vois le truc quoi … euh… et, oui, je fais passer le son du tambourin à corde dans un filtre résonant...euh… ici… (démo) et j'ai \hl{de l'aléatoire} (l'aléatoire est une matière, NDT) qui change la coupure du filtre … (extraits sonores)…. Bon là j'ai des réglages un peu exagéré mais… voilà… cet espèce de chassé-croisé comme ça… et... en fait, c'est euh… je me suis rendu compte en réfléchissant à cette interview qu'on devait faire que j'utilisais en fait exactement les mêmes choses, euh… j'étais allé vers les mêmes matériaux que du temps où je travaillais avec des patchs Max, çàd majoritairement des sinus, de la synthèse type forme d'onde, donc FM, modulation d'amplitude… et des samples... et du geste !… 

VG —  paysages sonores et des sons produits en direct… 

FD —  ouais… 

VG —  ensemble.. 

FD —  ouais… et après, ce que ça a fait … moi presque pendant 20 ans j'ai utilisé quasiment que l'ordinateur… c'était aussi pour raisons financières, hein… j'avais ça et ça me permettait d'avoir un maximum de synthètéiseurs sans avoir à les acheter en fait… et ce qui m'a fait basculer,en fait,  c'est l'arrivé du MiniBrute (de Arturia, NdE) qui est un petit synthétiseur analogique… au début je l'ai pris plus pour m'amuser… et en fait ça m'a vraiment accroché...pour retrouver le geste...et puis quelques mois après j'ai commencé le synthé modulaire… 

VG —  De retrouver le geste, parce sur l'ordinateur tu étais plutôt avec la souris ? 

FD —  eh ben, non j'ai beaucoup utilisé les interfaces … mais… il me semble qu'il y a toujours euh… un moment où t'as un peu la flemme… où tu vas pas aller au bout, où tu dis 'ah ce serait pas mal, là de faire une automation de volume mais bon, je vais plutôt le faire à la souris... » … quand tu composes par exemple une pièce, combien de fois je me suis retrouvé à aligner des belles courbes… c'était pas avec les oreilles que je faisais ma courbe de volume, c'était avec mes yeux. Voilà… donc c'est un peu… tu te retrouves à faire des trucs absurdes...d'être tatillon… (avec le ton tatillon, NdE) « ah non c'était à 127 » alors qu'en fait on entend pas la différence… par exemple… mais du coup ouais, quand j'ai commencé le synthé modulaire, ça m'a….fait prendre un chemin que j'avais pas prévu au départ …. pas à ce point, disons. Moi, comme je te disais, j'ai toujours mélangé musique expérimentale et musique trad mais auparavant j'amenais plutôt mon instrumentarium électronique dans des formations traditionnelles… et où je venais un peu jouer le trouble-fête...mais je chantais pas, je faisais pas d'instrument acoustique ni rien...et en fait… donc au début quand j'ai commencé le synthé modulaire c'était plus dans une idée d'avoir des beaux matériaux, que j'allais assembler, continuer mes compositions pareil dans l'ordinateur, sur plusieurs pistes et tout ça… et en fait je suis retrouvé très vite à mordre un autre hameçon qui est le jeu en direct … et pendant 2 ou 3 ans, ma pratique musicale c'était que ça, j'étais devant mon modulaire, je faisais des patchs, je les enregistrais quand ça me plaisait, c'était très éphémère, et…. 

VG —  tu enregistrais … le son ? 

FD —  j'enregistrais ce qui sortait oui… et parfois je faisais des patchs génératifs que je laissais tourner pendant des jours ...et que je changeais un petit micro-poil.. alors là on était plus proche de l'installation par exemple… mais il y a eu beaucoup un truc de « jouage »… de vraiment jouer en direct des trucs… de plus en plus, de plus en plus, jusqu'à arriver à un point où le …. le logiciel que j'utilisais d'habitude pour du montage se retrouvait juste un magnétophone multi-pistes… et je jouais mes pièces, de A à Z, quoi. Bon maintenant je fais… 

VG —  plus trop un outil pour composer, mais plutôt pour le cas où tu veuilles enregistrer ou… 

FD —  en tout cas la composition ne se faisait plus, euh… de manière formelle, en pensant à « tiens je vais rajouter ça... », c'était plus préparer un … euh… un réservoir de jeu, et en jouer. Donc un truc entre composition et improvisation un peu, on pourrait dire…bon maintenant, je fais encore des compositions plus… classiques… voire aussi des compositions sans utiliser du tout le synthé modulaire, mais j'ai eu une espèce de période comme ça un peu exclusive où je ne faisais plus que ça, pendant 2 ou 3 ans. Mais donc ça m'a amené vers d'autres choses, parce que je me suis dit,, tiens, ce serait chouette d'y adjoindre un tambourin à cordes… donc il y a un copain qui me l'a fabriqué, et je l'ai aidé … c''est Romain Colautti… et à partir de là, ça a été encore plus la fuite en avant, je me suis mis à chanté et j'ai chanté de plus en plus et j'ai même été après, du coup, jusqu'à avoir envie de m'inscrire dans un DEM de musique traditionnelle, et voilà j'ai fait ça… et alors c'est marrant parce que du coup, ça a encore relativité l'outil… Donc comme je te disais j'ai eu une période exclusive où il n'y avait plus que ça, et là il est revenu à sa place d'outil… et maintenant par dessus je fais du tambourin à corde, je chante, je commence même à apprendre le violon, etc. et lui je m'en sert pour, euh, ça fait partie de mon ... instrumentarium, on va dire, maintenant… et par exemple avant tout ça j'ai fait beaucoup de prises de son, de paysages sonores, de trucs comme ça... et pareil quand j'ai eu ma période très obscessionnelle du modulaire, ja faisais plus du tout de prise de son… et là maintenant j'ai repris, tu vois, donc y'a un espèce de truc qui s'est ré-équilibré, ça a généré tout un tas de trucs, ça m'a fait bifurqué dans plein de direction que j'avais pas prévues, donc ça c'est vraiment pas mal, et maintenant ça a repris sa place… parmi d'autres… 

VG —  ok tu as fait un peu l'interview dans le sens inverse de ce que j'avais prévu (rires), parce que je voulais te demander comment ça avait commencé et on est parti, pas des derniers, mais des presque derniers outils que tu utilises pour créer tes enregistrements… tu parlais d'enregistrement je ne sais pas s'il faut commencer par ça mais la question que je voulais te poser, alors tu y as déjà répondu en partie mais si je reprends le fil dans l'autre sens, une question par laquelle je voulais commencer, c'est qu'est ce qui t'a poussé à la base à faire de la musique avec des outils comme ça plutôt qu'avec des instruments pré-..., avec une histoire, des cours, une pédagogie, pourquoi prendre un instrument bizarroïde et mal fini… 

FD —  ouais, ouais, ouais… euuuuh…. En fait au début, je pense qu'il y a plein de raisons, et une des raisons c'est que j'avais envie de faire de la musique de manière… euh… avec du plaisir en fait… moi depuis gamin je voulais faire de la musique, et … bon, d'ailleurs j'avais envie de construire des synthés modulaires quand j'étais petit, parce que j'avais vu Jean-Michel Jarre à la télé, et j'avais dit « ah ouais, ça c'est super » … mais bon sans savoir du tout à quoi ça correspondait mais… et donc j'étais allé à la bibliothèque de mon village et j'avais demandé à la bibliothécaire « est-ce que vous avez des livres pour construire ses propres synthétiseurs ? » et elle devait se dire « qu'est ce qu'il a lui ? » (rires) et du coup elle disait « ah non, on n'a pas ça, désolé » et du coup j'attendais un mois et je revenais « et maintenant vous en avez ?» … donc au bout d'un moment elle a dit « non mais ça n'existe pas en fait » (rires)… voilà 

VG —  tu avais quel âge ? 

FD —  là, j'avais, euh, je sais pas, je devais avoir entre 7 et 12 ans, un truc comme ça… 

VG —  et tu jouais d'un instrument de musique ? 

FD —  non, alors au départ non, et alors du coup j'enquiquinais mes parents pour apprendre à jouer du synthétiseur, tout ça… et ils me disaient que c'était pas possible et gnagnagna … et puis en fait, euh… du coup j'ai essayé de me dire mais comment je peux faire pour faire quelque chose qui ressemble à ça donc j'ai cherché, cherché, et dans mon cerveau d'enfant, de ce que je voyais à la télé ou quoi, ce qui se rapprochait de ce que j'imaginais du synthétiseur, c'était la guitare électrique. Donc j'ai dit à mes parents que je voulais faire de la guiatre électrique. Alors ils m'ont dit «  mais il n'y a pas de cours de guitare électrique, faut que tu fasses des cours de guitare classique » … et donc me voilà à prendre des cours de guitare classique, c'était assez éloigné de ce que je voulais faire au départ… 

VG —  … pour faire du synthétiseur… 

FD —  et évidemment la déception a été grande, d'autant que le mec dont je ne citerai pas le nom était un piètre pédagogue, donc il m'a fait jouer les trois mêmes morceaux pendant trois ans, non, pendant deux ans… Pendant deux ans les mêmes morceaux en boucle, ça le dérangeait pas… il faisait ses heures et… et voilà au bout d'un moment j'ai dit que je voulais arrêter parce que c'était décevant justement … et donc là mes parents ont décidé que je n'avais pas vraiment envie de faire de la musique … donc ça aurait pu s'arrêter là… et ça s'est pas arrêté là… et donc après, à l'adolescence, découverte du rap, de la techno, des musiques électroniques en général, et j'ai commencé à bidouiller des machins… je les ai tannés assez longtemps pour qu'ils me filent un peu d'argent pour acheter mes premiers trucs, et j'ai commencé avec un 4 pistes à K7, un synthé … parce qu'entre temps ils avaient quand même acheté un synthé à ma sœur — qui n'avait jamais demandé ça… mais qui jouait du piano, donc elle avait le droit de faire du synthé analogique parental… et elle ne s'en servait pas du tout, et d'ailleurs je l'ai là … il est là, derrière… (rires)… et c'était un synthé FM, figures toi… un truc pour enfants, mais qui avait un moteur de synthèse FM rudimentaire...donc là j'ai vraiment beaucoup passé de temps à jouer sur le spectre et tout ça… donc il y a un espèce de début comme ça… j'achetais Keyboard magazine et je bavais sur de trucs qui étaient moins intéressants en fait, mais qui avaient l'air mieux parce que c'était « professionnel »…  mais je faisais mes armes là-dessus… après j'ai acheté des boites à rythmes et tout ce genre de trucs… et ensuite … 

VG —  c'était un synthé avec des boutons du coup ? 

FD —  Euh, il y a des tirettes… 

VG —  c'est celui à gauche là ? 

FD —  c'est pas celui là, il est derrière, là, attends je te le sors… c'est vraiment le synthé-jouet, de la gamme PSS de Yamaha, c'est des trucs, c'était pour enfant quoi, c'est des petites touches… (me le montrant) Donc là, il y avait le spectre, l'intensité de modulation, et puis l'enveloppe, attack, decay, release, le vibrato et un volume… c'est rudimentaire mais ça commence à … 

VG —  tu peux sculpter tes sons… faire un peu de cuisine… 

FD —  ouais… et tu pouvais transformer la banque de sons que tu avais à l'intérieur… tu partais de sons qui étaient déjà plus ou moins complexes et tu pouvais… ouais.. je l'ai ré-utilisé, j'ai fait des pièces électroacoustiques avec ce synthé, il n'y a pas si longtemps… en fait, il est très très bien … (rires)… et voilà et… bon après voilà j'ai fait de la musique électronique plus ou moins techno et tout ça… et par la rencontre d'un copain, je suis rentré dans la classe d'électroacoustique de Bordeaux, et alors qu'au début je pensais juste venir chercher des techniques, faire une espère d'espionnage industriel, encore une fois, tout ce qu'on fait transforme nos pratiques, et là beaucoup plus que ce que j'aurais imaginé parce que ça m'a vraiment beaucoup plus plût que ce que je croyais au départ… j'arrivais avec des a priori, de trucs élitistes, chiants… un peu les clichés, quoi… et en fait, notamment la musique de Bernard Parmeggiani, où j'ai complètement plongé dedans, et dans les cours, dans le cursus, il y avait une initiation à Max... et alors là ça a été super parce que tout à coup je pouvais vraiment aller dans la matière, beaucoup plus qu'avec des synthés fabriqués par d'autres… donc là il y a eu une période, là aussi encore une fois bien intensive, et  bien obsessionnelle… 

VG —  que tu utilisais pour te créer tes propres sons du coup ? 

FD —  ouais, je faisais mes patchs… 

VG —  … ou tu t'en servais pour du live ?  

FD —  alors il y avait les deux… il y avait les trucs de matières que j'utilisais comme réservoir pour mes pièces composées et il y a eu aussi l'improvisation très très vite… parce que l'improvisation ça a fait partie assez vite de ma pratique musicale… ouais, dès que j'ai commencé à faire de la musique vraiment sérieusement, j'ai composé des trucs « et » fait de l'improvisation … les deux m'intéressent… pour des raisons différentes… et donc ouais, il y avait des patchs de matière, il y avait un patch que j'avais commencé où il y avait des couches et des couches … je sais pas, j'ai du l'améliorer sur 7 ans un truc comme ça… je pourrais en rejouer… 

VG —  le même patch, le même noyau qui s'est …. 

FD —  ouais la même base qui s'est enrichie, complexifié… qui au début était un truc très simple, juste un jeu de lire un sample à des vitesses différentes de manière aléatoire… et puis petit à petit… et là je l'utilisais avec une interface de contrôle UC33 (de la marque XXX TODO, NdT)  

VG —  c'est des faders, non ? 

FD —  oui, il y avait 8 faders et dessus t'avais 8 fois 3 potentiomètres rotatifs…que tu pouvais assigner à ce que tu voulais… et c'est quand même une interface vachement bien… 

VG —  ouais… qui a eu ses grandes heures de gloire… 

FD —  ouais… et que j'ai trouvé en fait souvent plus pertinente que des choses qui voulaient aller plus loin, du type Méta-Instrument (inventé par Serge de Laubier, NdE) et tout ça... qui, pour moi, il y a un côté ou presque on se perd dans le contrôle et on oublie presque le sonore… alors il y a des gens qui en font des choses super mais… quelque part le côté hyper-simple de « un fader », moi ça me plait plus… d'ailleurs je me retrouve encore avec un fader ici (en montrant les faders de la petite table de mixage à côté de son modulaire interface modulaire) voilà… une espèce de robinet à son…  

VG —  ok… parce que sur un fader d'UC33 tu peux y mettre plein d'autres choses qu'un volume sonore… 

FD —  oui… c'est vrai… c'est vrai… mais je m'en servais, dans mon patch principal d'improvisation, c'était exactement ça… c'était des canaux, que je dosais … donc voilà, mais donc oui après quelques années, je sais pas, j'ai du faire presque sept ans de Max, et quand j'ai basculé là dessu (le modulaire NdE) voilà… ça m'a rendu un peu triste, Max… je me retrouvais devant un écran (en prenant un dos vouté, NdE) à faire des trucs… là j'avais… j'ai presque retrouvé une sensation que j'avais au début en découvrant Max, de (mimant des gestes de connections) « ah et si je fais ça qu'est ce que ça fait ? »  

VG —  de patcher ? 

FD —  de patcher, et en même temps comme je te disais, avec un truc de corps, un truc direct, et un truc éphémère qui a quand même sa valeur… tu peux pas rappeler un patch en un clic, alors ça peut être un inconvénient mais ça peut aussi être un avantage parce que … euh… je sais très bien comment refaire le même type de truc que j'ai fait une autre fois mais je ne vais jamais le faire exactement pareil, donc ce sera « le truc de ce jour là », quoi… un côté comme ça qui me plait pas mal… 

VG —  et tu retrouves des chemins là-dedans quand même ? 

FD —  ouais, j'ai mes espèces de routines, de logiques, de programmation… je sais très bien que si je veux faire par exemple un truc très … des nappes dans la durée, je vais partir là-dessus, si je veux faire quelque chose de très dynamique, je vais patcher autrement… 

VG —  et le patch, tu le fais en amont de ton jeu, ou bien tu patches en cours de jeu des fois ? 

FD —  ça dépend… en concert, j'aime bien ne rien avoir prévu et juste je démarre avec du sinus et on va voir ce qui se passe… 

VG —  et rajouter … 

FD —  … je rajoute, je patche en jouant,  

VG —  .. partir quasiment sans cap tu veux dire, ou bien avec un truc très simple … 

FD —  ouais, juste, ouais, très simple, et je rajoute, là  j'avais prévu quelques trucs mais c'était aussi dû à la contrainte de faire une improvisation courte, donc, si je dois commencer à patcher, ça prend du temps quoi… ouais de longues improvisation de 1 heure, 2 heures, 3 heures... 

VG —  il y a un truc dont tu parlais de rappeler un preset en un clic, qui n'est pas forcément qu'un avantage, et donc je parlais de chemin tout à l'heure, parce que si tu veux passer d'une config à une autre, tu peux pas physiquement dé-pluguer tout et re-pluguer en un clic justement, et donc, enfin moi je ne pratique pas le synthé modulaire, mais ça me fait penser à cette caractéristique à laquelle je n'avais pas pensé avant, qui est que tu prends des chemins que tu tricotes et dé-tricotes, et d'une certaine manière, quand tu veux aller quelque part, tu … (34:03) 

FD —  ouais… ben par exemple, mettons, j'ai ça.. (début de séquence musicale) si lui par exemple je veux le transformer, je sais pas, euh, j'aurais envie de le faire interagir avec lui, ce que je vais faire, je vais faire une espèce de tour de passe-passe, je vais venir faire un autre événement donc pendant ce temps je fais ça, donc il y a un truc musical qui est en train de se passer… je débranche mon truc… je prends un câble… et… je branche… je vais pouvoir ré-introduire mon son … (fin de la démo) et par exemple des fois je peux…. Ça peut m'arriver d'être moins dans le jonglage que ça et patcher alors qu'on entend le son… mais… c'est plus périlleux… faut être un peu joueur quoi, des fois ça ne fait pas du tout ce que tu avais prévu… 

VG —  ouais.. (rires) 

FD —  hehe… donc là faut jouer avec, quoi … mais c'est des trucs que j'aime bien en improvisation justement, les accidents, les surprises… des fois ça m'est arrivé en improvisant de… d'être surpris par le résultat et de trouver que c'était un peu moche, que c'était presque de mauvais goût, quoi, par rapport à mon goût personnel… donc là du coup, démerdes toi avec ça !… (rires) 

VG —  et… t'as une sorte de pavage… c'est une autre chose qui m'intéressait qui est aussi caractéristiques des DMI de pouvoir changer la config en un clic et de ne pas se retrouver avec les mêmes choses sous les doigts, c'est à dire qu'un même bouton peut changer une fonction…  

FD —  Ouais … et bien là ça peut être aussi le cas… enfin pardon, je te coupe… 

VG —  il y a à la fois cette question là, et il y a aussi la question de la construction musicale là dessus où … en fait dans ces instruments qui sont un peu des méta-instruments, que tu ré-assembles, que ça soit en un clic ou via un patch, tu as une sorte de programmation plus ou moins en live, avec ton instruments…  

FD —  ouais 

VG —  et sur des longs set, sur la durée d'un concert, du coup tes stratégies pour passer d'un… enfin, est ce que tu fonctionnes plutôt avec un set continu ou est ce que tu as des… 

FD —  ça dépend… 

VG —  parce que quand tu as des pistes, des morceaux séparés, comment tu passes d'un morceau à l'autre, alors que dans les logiciels de type (Ableton) Live, les gens rappelle le presets du deuxième morceau… 

FD —  non, alors je ne l'utilise pas du tout pour faire des morceaux précis que je rejoue… ça je ne fais pas du tout ça...en fait ce serait hyper compliqué de faire ça, d'ailleurs… c'est pas du tout prévu pour… je dis pas que c'est impossible mais à mon avis faut beaucoup s'entrainer pour un résultat pas forcément génial… par contre ça m'est déjà arrivé de faire un set en plusieurs fois, et au milieu de couper le son, débrancher, et repartir de zéro… comme si c'était un autre morceau, ou une autre pièce… enfin, on met les mots qu'on veut mais… et par contre ça m'arrive de l'utiliser dans des groupes, par exemple, où là il y a des choses qui sont prévues mais c'est plus de l'ordre de … à tel moment, à tel tableaux, je sais que je fais des vagues de sons modulés en FM… et c'est pas plus prévu que ça… 

VG —  c'est pas forcément une config qui est prévue mais plutôt un résultat sonore que tu atteins d'une manière ou d'une autre ? 

FD —  alors, si… en général quand je fais des trucs avec des groupes, s'il y a des tableaux vraiment prévus, là je note des patchs de manière très succinte, c'est à dire, telle modulation sort de tel connecteur et va à tel connecteur, et en gros les positions des potars… donc on n'a pas exactement le même résultat mais on a quelque chose approchant, quoi… et de toute façon ça a vocation à jouer, j'en joue quoi… les positions de potars sont des positions moyennes quoi, après j'en change.. peut-être pas tous, mais… 

VG —  et ça me fait penser à une autre question qui est la question de l'ergonomie de l'instrument, parce que dans ces DMI, il y a une ergonomie de l'instrument qui n'est, a priori, pas dictée par des critères de facture acoustique, à l'inverse des instruments acoustiques… et du coup c'est souvent plus défini, à la fois par des contraintes techniques, du fait que tel module a besoin de tel bouton là, tout ça, et aussi par la contrainte personnelle, enfin les désirs ou les contraintes liée à ta propre pratique, à ton propre corps, à comment tu organises les choses que tu as sous les doigts… 

FD —  ouais, aussi ta façon de le voir, de l'imaginer en amont... 

VG —  oui… et par exemple pour ce dispositif là (le modulaire dont on parle, NdE) qu'est ce qui fait que tu as organisé les modules de cette manière là plutôt que d'une autre ? 

FD —  j'ai essayé d'être au maximum logique… j'ai mes sources en haut, tout ce qui est modulation… donc les oscillateurs, et puis le sampleur dont je te parlais qui fait la radio, là…  là, j'ai toute les sources qui vont venir transformer, envoyer du signal ou de l'aléatoire, de la commande un peu quoi… j'ai par exemple ici un mini séquenceur euclidien… 

VG —  pour piloter les sources ? 

FD —  euh… qui envoie des impulsions, quoi 

VG —  c'est à dire, ceux du haut c'est plutôt des générations horizontales de … 

FD —  c'est plutôt eux qui vont me générer du son, et eux vont venir faire des variations de voltage, donc selon où je le branche ça fait quelque chose ou quelque chose… 

VG —  des enveloppes ? 

FD —  ça peut faire des enveloppes, ça peut faire des LFO, ça peut faire des… ouais. 

VG —  qui vont venir sculpter tes sources… 

FD —  ça peut venir changer la fréquence de ma radio… 

VG —  d'accord 

FD —  où ouvrir le filtre qui vient filtrer mon tambourin à corde 

VG —    

FD —  ouais… je sais pas...euh… 

VG —  c'est à dire quand tu as un LFO, je dis générateur de geste parce que on pourrait le moduler … 

FD —  oui, quelque part c'est comme si on déléguait à quelqu'un de faire … (geste d'oscillation de la main) 

VG —  … sur le potar… 

FD —  oui voilà c'est ça… et après en bas, c'est plutôt les trucs de geste, en fait là c'est bêtement parce que \hl{c'est là où je met les mains…} là sur le patch que j'ai montré tout à l'heure, je ne me suis pas servi de ça mais celui là je m'en sers pas mal en impro… 

VG —  Sur es capteurs électrostatiques aussi non ? 

FD —  Voilà … qui lui, sert par exemple à … c'est comme des banques de mémoire de voltage… donc là j'ai branché, y'a trois lignes, j'ai branché la sortie de la première ligne sur le pitch de ce premier oscillateur… (démo de l'effet) mais si je le branche sur la fréquence de coupure de mon filtre, je vais faire résonner le filtre (du tambour, NdE) 

VG —  c'est une sorte de clavier que tu peux assigner à n'importe quel paramètre 

FD —  voila, tout le truc qui est super avec ce système, c'est que tu as des voltages et selon où tu le branches, c'est des voltages qui vont contrôler une hauteur, un volume, une intensité, une vitesse, enfin bon comme dans Max en fait, quand tu as des trucs qui vont de 0 à 127, c'est pas 127dB, c'est juste à 127  

VG —  c'est une course en 0-5V 

FD —  oui, je sais plus, c'est plus ou moins 12V je crois 

VG —  ton patch orange c'est ton patch de modulation ? 

FD —  non j'ai pas de codage de couleur, mes câbles oranges c'est juste des jack-jack simples 

VG —  mais tu as une différenciation entre ce qui est signal audio et modulation ? 

FD —  non pas sur ceux là, sur certain oui, sur les Buchla il y a carrément un format différent de câble pour ce qui est signal et ce qui est euh… 

VG —  là c'est tout au même niveau 

FD —  oui 

VG —  c'est à dire tu peux prendre un signal audio et le mettre comme une modulation d'autres trucs… 

FD —  oui, et d'ailleurs je trouve que c'est un avantage parce que du coup tu peux faire des trucs qui n'étaient pas prévus et moduler un filtre en audio, par exemple, c'est un truc que j'adore faire… par exemple (faisant la démo) là j'ai une sortie audio de ce générateur d'aléatoire et là comme tu entends ça … 

VG —  c'est la fréquence de coupure que tu modules avec le signal audio ? 

FD —  oui… ou ça pourrait être aller récupérer un sinus… donc tu déconstruis un peu des trucs de…  « ah les filtres, c'est ... »  

VG —  ça existe aussi dans Max entre ce qui est signal et ce qui est message de contrôle 

FD —  oui, tu peux faire la traduction et du coup c'est hyper bien … 

VG —  tu peux la faire, mais quand même elle est présente comme qui n'est pas … 

FD —  elle est marquée oui c'est vrai… 

VG —  certains paramètres que tu voudrais pouvoir contrôler directement en audio et tu te retrouves à devoir les convertir en message avec ce que tu perds… 

FD —  oui c'est ça, du coup tu as une fréquence d'échantillonnage un peu 

VG —  oui.. 

FD —  après qui peut générer une autre esthétique mais… 

VG —  et… j'ai perdu le fil de ce que je disais avant, je parlais d'ergonomie et… 

FD —  oui, mais ça c'est marrant, je vois aussi des points commun entre quand je faisais du Max et ça c'est, euh, déjà c'est évolutif… c'est pas figé… et… euh, il y a quelque chose que j'ai pu trouver hyper pratique à un moment me semble absurde, peut-être un an après, et je vais changer des trucs… j'ai pas mal discuté avec des gens qui pratiquaient ça, le synthé modulaire, et il y en a certains qui ré-organisent de manière totalement aléatoire leur modules et ils disent que ça les fait  faire des choses différentes qu'ils ne faisaient pas du tout avant… du coup ça leur change leur logique, ça leur fait faire… ré-essayer des trucs qu'ils ne faisaient plus… ou qu'ils n'avaient jamais essayé… du coup c'est rigolo… 

VG —  ah, tu as un truc qui n'est pas propre au son électronique, mais qui est fortement encouragé par ce genre de dispositif où tu es dans l'exploration sonore beaucoup, et c'est parfois la critique inverse qui est entendue à propos du fait que les instruments électroniques sont des instruments sur lesquels il est difficile d'avoir une pratique comme celle d'un violoncelliste, qui joue de son violoncelle qui ne bouge pas … 

FD —  oui, qui va rejouer exactement la même chose oui… 

VG —  oui, ou alors explorer son instrument, mais elle est dans rechercher des choses fines et inaccessibles avec son instrument … 

FD —  ben pour moi en fait c'est différent…  