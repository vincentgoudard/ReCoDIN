\chapter{Interview : François Dumeaux}
\label{appendix:dumeaux}

\section*{Biographie}
\noindent François Dumeaux (Rodez, 1978), compositeur et improvisateur, a obtenu en 2006 un D.E.M. et le prix SACEM, dans la classe de composition électroacoustique de Christian Eloy et Christophe Havel, au C.N.R. de Bordeaux. Finaliste du concours de composition acousmatique Métamorphoses 2008, il a été à cette occasion édité sur la compilation éponyme. Pour le 80ème anniversaire de Bernard Parmegiani, il a composé SRA à la demande de Annette Vande Gorne, pour Une guirlande pour Bernard Parmegiani, aux côtés de 23 autres compositeurs. Sa musique a été jouée aux États-Unis, au Canada, au Japon, en Belgique, en Espagne et en France. Il a enseigné la composition électroacoustique au C.R.R. de Bordeaux entre 2011 et 2015.

\section*{Transcript}

\noindent François Dumeaux, interview du 30/07/2017, dans son studio, Cuzorn, France.

\noindent [séquence musicale sur un dispositif comprenant un synthétiseur modulaire, et un tambourin repris par microphone]

VG — C'était très bien de commencer par une séquence musicale car cela me permet de te poser tout de suite la question de décrire un petit peu cet instrument, ces instruments, je ne sais pas comment tu les appeles, s'ils ont un nom 

FD — alors celui là, je ne fais pas preuve d'originalité, je l'appelle le synthétiseur modulaire ... mais j'y adjoins cet instrument qui est un instrument de musique traditionnel qui est \hl{un tambourin à cordes que j'ai détourné complètement de sa fonction de départ}. Normalement, ça se joue avec une flûte harmonique et ça sert à faire un bourdon rythmique, et moi \hl{je m'en sers pour faire de la matière}... Mais pour moi, \hl{c'est comme s'il y avait un arrière plan imaginaire de tout ce que ça draine d'histoire, de projections}... même si au final le son que j'en sors est assez éloigné de ce pourquoi c'est fabriqué au départ 

VG — dans le son et dans l'objet, tu veux dire ? 

FD — Oui, voilà tout ce que ça peut représenter, pour moi hein, ça nourrit mon imaginaire. Je ne sais pas, c'est comme s'il y avait un lien avec quelque chose qui fait vraiment partie de ma pratique musicale au même titre que la musique expérimentale, il  y a la musique trad et j'ai toujours fait les deux, mélanger un peu, et plus ça va plus je mélange les deux en fait... Je fais aussi des trucs purement l'un ou purement l'autre mais les deux m'intéressent, en fait. \hl{Plus ça va et plus j'ai tendance à effacer les frontières entre mes pratiques}. Donc là, ce que j'ai utilisé là, c'est ce que j'utilise le plus souvent. Tu vois, tous les modules ne sont pas patchés, il y en a dont je ne me suis pas servi. Et ça, c'est ce dont j'aime bien me servir quand je fais de l'improvisation... donc c'est des oscillateurs, il y en a des analogiques, il y en a des numériques... et \hl{c'est un peu les deux opposés}, c'est à dire, il y en a des très simples, des sinus, et il y a des oscillateurs à table d'onde... 

VG — dans lesquels tu charges des samples, tu veux dire ? 

FD — non, il a toute une banque de ... je sais plus combien il en a, 64 je crois... (les faisant écouter une à une) on balaie ... 

VG — ça interpole entre les tables d'onde ? 

FD — voilà c'est ça... il fait du morphing entre les tables d'onde... voilà et puis après il y a des trucs de modulation, de l'aléatoire, des cycles très longs, et puis... euh... et puis j'utilise celui-ci (joignant le geste à la parole) qui ... euh... un truc qui marche avec l'électricité statique (petite séquence de jeu, sons très bruitistes, corrosifs, à changements rapide)... voilà...  ça je l'utilise beaucoup en improvisation... et je sais pas si tu as fait gaffe, il y avait un paysage sonore... de vent... 

VG — oui j'ai entendu 

FD — et alors c'est ce module là, qui est un truc qui était un kit... que j'ai soudé moi, ça s'appelle le ``radio-musique'' et c'est basé sur la pièce de John Cage... c'est l'idée, tu sais, de composer une pièce où tu ... où dans la partition, tu avais ``ouvrir la radio à telle fréquence, à tel moment'' ... et alors comme ça a beaucoup changé la radio et même que bientôt ça risque de plus exister, la radio... analogique, ils se sont dits, on va faire le même  truc mais avec nos propres banques de son donc du coup, tu as... tu peux avoir plein de canaux différents (séquence de jeu) là par exemple, c'est une banque, c'est que des paysages sonores et je joue le tuner, je vais passer sur un autre paysage sonore... et pendant ce temps, l'autre il continue comme si c'était une vraie radio, donc si je reviens sur  l'autre, ça reprend pas la lecture où c'était, ça a continué entre-temps de compter... des cycles... (faisant la démo, sons d'oiseaux) ... 

VG — c'est radio-John... 

FD — ouais... (rires) ... alors évidemment, c'est des trucs que tu peux moduler donc tu peux ... déclencher... un changement de fréquence ... et moi je m'en sers surtout comme d'une banque de sons pour... euh... de la même façon que quand j'utilise un ordinateur pour faire des citations parce que j'ai envie sur le moment... je l'ai sans avoir besoin de trainer un ordinateur ... et je disais il y a plusieurs banques... là c'est la banque paysages... (extraits)... bon là j'ai ça... (bleeps)... là c'est des instru acoustiques avec pas mal d'instru traditionnels... ça c'est de la flûte... des séquences de jeu que j'ai enregistrées... du stick, une cornemuse landaise... pas tout à fait jouée comme prévu mais... voilà... tu vois le truc quoi ... euh... et, oui, je fais passer le son du tambourin à corde dans un filtre résonant... euh... ici... (démo) et j'ai de l'aléatoire qui change la coupure du filtre ... (extraits sonores)... Bon là j'ai des réglages un peu exagérés mais... voilà... cet espèce de chassé-croisé comme ça... et... en fait, c'est euh... je me suis rendu compte en réfléchissant à cette interview qu'on devait faire, que j'utilisais en fait exactement les mêmes choses, euh... j'étais allé vers les mêmes matériaux que du temps où je travaillais avec des patchs Max, c'est à dire majoritairement des sinus, de la synthèse type forme d'onde, donc FM, modulation d'amplitude... et des samples... et du geste !... 

VG — paysages sonores et des sons produits en direct... 

FD — ouais... 

VG — ensemble... 

FD — ouais... et après, ce que ça a fait ... moi presque pendant 20 ans j'ai utilisé quasiment que l'ordinateur... c'était aussi pour raisons financières, hein... j'avais ça et ça me permettait d'avoir un maximum de synthétiseurs sans avoir à les acheter en fait... et ce qui m'a fait basculer, en fait,  c'est l'arrivé du MiniBrute (de Arturia, NdE) qui est un petit synthétiseur analogique... au début je l'ai pris plus pour m'amuser... et en fait ça m'a vraiment accroché... pour retrouver le geste... et puis quelques mois après j'ai commencé le synthé modulaire... 

VG — De retrouver le geste, parce sur l'ordinateur tu étais plutôt avec la souris ? 

FD — eh ben, non j'ai beaucoup utilisé les interfaces ... mais... il me semble qu'il y a toujours euh... un moment où t'as un peu la flemme... où tu vas pas aller au bout, où tu dis ``ah ce serait pas mal, là de faire une automation de volume mais bon, je vais plutôt le faire à la souris'' ... quand tu composes par exemple une pièce, combien de fois je me suis retrouvé à aligner des belles courbes... c'était pas avec les oreilles que je faisais ma courbe de volume, c'était avec mes yeux. Voilà... donc c'est un peu... tu te retrouves à faire des trucs absurdes... d'être tatillon... ``ah non c'était à 127''  alors qu'en fait on entend pas la différence... par exemple... mais du coup ouais, quand j'ai commencé le synthé modulaire, ça m'a... fait prendre un chemin que j'avais pas prévu au départ ... pas à ce point, disons. Moi, comme je te disais, j'ai toujours mélangé musique expérimentale et musique trad mais auparavant j'amenais plutôt mon instrumentarium électronique dans des formations traditionnelles... et où je venais un peu jouer le trouble-fête... mais je chantais pas, je faisais pas d'instrument acoustique ni rien... et en fait... donc au début quand j'ai commencé le synthé modulaire c'était plus dans une idée d'avoir des beaux matériaux, que j'allais assembler, continuer mes compositions pareil dans l'ordinateur, sur plusieurs pistes et tout ça... et en fait je suis retrouvé très vite à mordre un autre hameçon qui est le jeu en direct ... et pendant 2 ou 3 ans, ma pratique musicale c'était que ça, j'étais devant mon modulaire, je faisais des patchs, je les enregistrais quand ça me plaisait, c'était très éphémère, et... 

VG — tu enregistrais ... le son ? 

FD — j'enregistrais ce qui sortait oui... et parfois je faisais des patchs génératifs que je laissais tourner pendant des jours ... et que je changeais un petit micro-poil... alors là on était plus proche de l'installation par exemple... mais il y a eu beaucoup un truc de ``jouage'' ... de vraiment jouer en direct des trucs... de plus en plus, de plus en plus, jusqu'à arriver à un point où le ... le logiciel que j'utilisais d'habitude pour du montage se retrouvait juste un magnétophone multi-pistes... et je jouais mes pièces, de A à Z, quoi. Bon maintenant je fais... 

VG — plus trop un outil pour composer, mais plutôt pour le cas où tu veuilles enregistrer ou... 

FD — en tout cas la composition ne se faisait plus, euh... de manière formelle, en pensant à ``tiens je vais rajouter ça...'', c'était plus préparer un ... euh... un réservoir de jeu, et en jouer. Donc un truc entre composition et improvisation un peu, on pourrait dire... bon maintenant, je fais encore des compositions plus... classiques... voire aussi des compositions sans utiliser du tout le synthé modulaire, mais j'ai eu une espèce de période comme ça un peu exclusive où je ne faisais plus que ça, pendant 2 ou 3 ans. Mais donc ça m'a amené vers d'autres choses, parce que je me suis dit,, tiens, ce serait chouette d'y adjoindre un tambourin à cordes... donc il y a un copain qui me l'a fabriqué, et je l'ai aidé ... c''est Romain Colautti... et à partir de là, ça a été encore plus la fuite en avant, je me suis mis à chanté et j'ai chanté de plus en plus et j'ai même été après, du coup, jusqu'à avoir envie de m'inscrire dans un DEM de musique traditionnelle, et voilà j'ai fait ça... et alors c'est marrant parce que du coup, ça a encore relativité l'outil... Donc comme je te disais j'ai eu une période exclusive où il n'y avait plus que ça, et là il est revenu à sa place d'outil... et maintenant par dessus je fais du tambourin à corde, je chante, je commence même à apprendre le violon, etc. et lui je m'en sert pour, euh, ça fait partie de mon ... instrumentarium, on va dire, maintenant... et par exemple avant tout ça j'ai fait beaucoup de prises de son, de paysages sonores, de trucs comme ça... et pareil quand j'ai eu ma période très obscessionnelle du modulaire, ja faisais plus du tout de prise de son... et là maintenant j'ai repris, tu vois, donc y'a un espèce de truc qui s'est ré-équilibré, ça a généré tout un tas de trucs, ça m'a fait bifurqué dans plein de direction que j'avais pas prévues, donc ça c'est vraiment pas mal, et maintenant ça a repris sa place... parmi d'autres... 

VG — ok tu as fait un peu l'interview dans le sens inverse de ce que j'avais prévu (rires), parce que je voulais te demander comment ça avait commencé et on est partis, pas des derniers, mais des presque derniers outils que tu utilises pour créer tes enregistrements... tu parlais d'enregistrement je ne sais pas s'il faut commencer par ça mais la question que je voulais te poser, alors tu y as déjà répondu en partie mais si je reprends le fil dans l'autre sens, une question par laquelle je voulais commencer, c'est qu'est ce qui t'a poussé à la base à faire de la musique avec des outils comme ça plutôt qu'avec des instruments pré-... , avec une histoire, des cours, une pédagogie, pourquoi prendre un instrument bizarroïde et mal fini... 

FD — ouais, ouais, ouais... euuuuh... En fait au début, je pense qu'il y a plein de raisons, et une des raisons c'est que j'avais envie de faire de la musique de manière... euh... avec du plaisir en fait... moi depuis gamin je voulais faire de la musique, et ... bon, d'ailleurs j'avais envie de construire des synthés modulaires quand j'étais petit, parce que j'avais vu Jean-Michel Jarre à la télé, et j'avais dit ``ah ouais, ça c'est super'' ... mais bon sans savoir du tout à quoi ça correspondait mais... et donc j'étais allé à la bibliothèque de mon village et j'avais demandé à la bibliothécaire ``est-ce que vous avez des livres pour construire ses propres synthétiseurs ?'' et elle devait se dire ``qu'est ce qu'il a lui ?'' (rires) et du coup elle disait ``ah non, on n'a pas ça, désolé'' et du coup j'attendais un mois et je revenais ``et maintenant vous en avez ?'' ... donc au bout d'un moment elle a dit ``non mais ça n'existe pas en fait'' (rires)... voilà 

VG — tu avais quel âge ? 

FD — là, j'avais, euh, je sais pas, je devais avoir entre 7 et 12 ans, un truc comme ça... 

VG — et tu jouais d'un instrument de musique ? 

FD — non, alors au départ non, et alors du coup j'enquiquinais mes parents pour apprendre à jouer du synthétiseur, tout ça... et ils me disaient que c'était pas possible et gnagnagna ... et puis en fait, euh... du coup j'ai essayé de me dire mais comment je peux faire pour faire quelque chose qui ressemble à ça donc j'ai cherché, cherché, et dans mon cerveau d'enfant, de ce que je voyais à la télé ou quoi, ce qui se rapprochait de ce que j'imaginais du synthétiseur, c'était la guitare électrique. Donc j'ai dit à mes parents que je voulais faire de la guiatre électrique. Alors ils m'ont dit ``mais il n'y a pas de cours de guitare électrique, faut que tu fasses des cours de guitare classique'' ... et donc me voilà à prendre des cours de guitare classique, c'était assez éloigné de ce que je voulais faire au départ... 

VG — ... pour faire du synthétiseur... 

FD — et évidemment la déception a été grande, d'autant que le mec dont je ne citerai pas le nom était un piètre pédagogue, donc il m'a fait jouer les trois mêmes morceaux pendant trois ans, non, pendant deux ans... Pendant deux ans les mêmes morceaux en boucle, ça le dérangeait pas... il faisait ses heures et... et voilà au bout d'un moment j'ai dit que je voulais arrêter parce que c'était décevant justement ... et donc là mes parents ont décidé que je n'avais pas vraiment envie de faire de la musique ... donc ça aurait pu s'arrêter là... et ça s'est pas arrêté là... et donc après, à l'adolescence, découverte du rap, de la techno, des musiques électroniques en général, et j'ai commencé à bidouiller des machins... je les ai tannés assez longtemps pour qu'ils me filent un peu d'argent pour acheter mes premiers trucs, et j'ai commencé avec un 4 pistes à K7, un synthé ... parce qu'entre temps ils avaient quand même acheté un synthé à ma sœur — qui n'avait jamais demandé ça... mais qui jouait du piano, donc elle avait le droit de faire du synthé analogique parental... et elle ne s'en servait pas du tout, et d'ailleurs je l'ai là ... il est là, derrière... (rires)... et c'était un synthé FM, figures toi... un truc pour enfants, mais qui avait un moteur de synthèse FM rudimentaire... donc là j'ai vraiment beaucoup passé de temps à jouer sur le spectre et tout ça... donc il y a un espèce de début comme ça... j'achetais Keyboard magazine et je bavais sur de trucs qui étaient moins intéressants en fait, mais qui avaient l'air mieux parce que c'était ``professionnel''... mais je faisais mes armes là-dessus... après j'ai acheté des boites à rythmes et tout ce genre de trucs... et ensuite ... 

VG — c'était un synthé avec des boutons du coup ? 

FD — Euh, il y a des tirettes... 

VG — c'est celui à gauche là ? 

FD — c'est pas celui là, il est derrière, là, attends je te le sors... c'est vraiment le synthé-jouet, de la gamme PSS de Yamaha, c'est des trucs, c'était pour enfant quoi, c'est des petites touches... (me le montrant) Donc là, il y avait le spectre, l'intensité de modulation, et puis l'enveloppe, attack, decay, release, le vibrato et un volume... c'est rudimentaire mais ça commence à ... 

VG — tu peux sculpter tes sons... faire un peu de cuisine... 

FD — ouais... et tu pouvais transformer la banque de sons que tu avais à l'intérieur... tu partais de sons qui étaient déjà plus ou moins complexes et tu pouvais... ouais... je l'ai ré-utilisé, j'ai fait des pièces électroacoustiques avec ce synthé, il n'y a pas si longtemps... en fait, il est très très bien ... (rires)... et voilà et... bon après voilà j'ai fait de la musique électronique plus ou moins techno et tout ça... et par la rencontre d'un copain, je suis rentré dans la classe d'électroacoustique de Bordeaux, et alors qu'au début je pensais juste venir chercher des techniques, faire une espère d'espionnage industriel, encore une fois, tout ce qu'on fait transforme nos pratiques, et là beaucoup plus que ce que j'aurais imaginé parce que ça m'a vraiment beaucoup plus plût que ce que je croyais au départ... j'arrivais avec des a priori, de trucs élitistes, chiants... un peu les clichés, quoi... et en fait, notamment la musique de Bernard Parmeggiani, où j'ai complètement plongé dedans, et dans les cours, dans le cursus, il y avait une initiation à Max... et alors là ça a été super parce que tout à coup je pouvais vraiment aller dans la matière, beaucoup plus qu'avec des synthés fabriqués par d'autres... donc là il y a eu une période, là aussi encore une fois bien intensive, et  bien obsessionnelle... 

VG — que tu utilisais pour te créer tes propres sons du coup ? 

FD — ouais, je faisais mes patchs... 

VG — ... ou tu t'en servais pour du live ?  

FD — alors il y avait les deux... il y avait les trucs de matières que j'utilisais comme réservoir pour mes pièces composées et il y a eu aussi l'improvisation très très vite... parce que l'improvisation ça a fait partie assez vite de ma pratique musicale... ouais, dès que j'ai commencé à faire de la musique vraiment sérieusement, j'ai composé des trucs \emph{et} fait de l'improvisation ... les deux m'intéressent... pour des raisons différentes... et donc ouais, il y avait des patchs de matière, il y avait un patch que j'avais commencé où il y avait des couches et des couches ... je sais pas, j'ai du l'améliorer sur 7 ans un truc comme ça... je pourrais en rejouer... 

VG — le même patch, le même noyau qui s'est ... 

FD — ouais la même base qui s'est enrichie, complexifié... qui au début était un truc très simple, juste un jeu de lire un sample à des vitesses différentes de manière aléatoire... et puis petit à petit... et là je l'utilisais avec une interface de contrôle UC33 (de la marque Evolution, NdT)  

VG — c'est des faders, non ? 

FD — oui, il y avait 8 faders et dessus t'avais 8 fois 3 potentiomètres rotatifs... que tu pouvais assigner à ce que tu voulais... et c'est quand même une interface vachement bien... 

VG — oui... qui a eu ses grandes heures de gloire... 

FD — oui... et que j'ai trouvé en fait souvent plus pertinente que des choses qui voulaient aller plus loin, du type Méta-Instrument (inventé par Serge de Laubier, NdE) et tout ça... qui, pour moi, il y a un côté ou presque on se perd dans le contrôle et on oublie presque le sonore... alors il y a des gens qui en font des choses super mais... quelque part le côté hyper-simple de ``un fader'', moi ça me plait plus... d'ailleurs je me retrouve encore avec un fader ici (en montrant les faders de la petite table de mixage à côté de son modulaire interface modulaire) voilà... une espèce de robinet à son... 

VG — ok... parce que sur un fader d'UC33 tu peux y mettre plein d'autres choses qu'un volume sonore... 

FD — oui... c'est vrai... c'est vrai... mais je m'en servais, dans mon patch principal d'improvisation, c'était exactement ça... c'était des canaux, que je dosais ... donc voilà, mais donc oui après quelques années, je sais pas, j'ai du faire presque sept ans de Max, et quand j'ai basculé là dessu (le modulaire NdE) voilà... ça m'a rendu un peu triste, Max... je me retrouvais devant un écran (en prenant un dos vouté, NdE) à faire des trucs... là j'avais... j'ai presque retrouvé une sensation que j'avais au début en découvrant Max, de (mimant des gestes de connections) ``ah et si je fais ça qu'est ce que ça fait ? '' 

VG — de patcher ? 

FD — de patcher, et en même temps comme je te disais, avec un truc de corps, un truc direct, et un truc éphémère qui a quand même sa valeur... tu peux pas rappeler un patch en un clic, alors ça peut être un inconvénient mais ça peut aussi être un avantage parce que ... euh... je sais très bien comment refaire le même type de truc que j'ai fait une autre fois mais je ne vais jamais le faire exactement pareil, donc ce sera ``le truc de ce jour là'' , quoi... un côté comme ça qui me plait pas mal... 

VG — et tu retrouves des chemins là-dedans quand même ? 

FD — ouais, j'ai mes espèces de routines, de logiques, de programmation... je sais très bien que si je veux faire par exemple un truc très ... des nappes dans la durée, je vais partir là-dessus, si je veux faire quelque chose de très dynamique, je vais patcher autrement... 

VG — et le patch, tu le fais en amont de ton jeu, ou bien tu patches en cours de jeu des fois ? 

FD — ça dépend... en concert, j'aime bien ne rien avoir prévu et juste je démarre avec du sinus et on va voir ce qui se passe... 

VG — et rajouter ... 

FD — ... je rajoute, je patche en jouant,  

VG — ... partir quasiment sans cap tu veux dire, ou bien avec un truc très simple ... 

FD — ouais, juste, ouais, très simple, et je rajoute, là  j'avais prévu quelques trucs mais c'était aussi dû à la contrainte de faire une improvisation courte, donc, si je dois commencer à patcher, ça prend du temps quoi... ouais de longues improvisation de 1 heure, 2 heures, 3 heures... 

VG — il y a un truc dont tu parlais de rappeler un preset en un clic, qui n'est pas forcément qu'un avantage, et donc je parlais de chemin tout à l'heure, parce que si tu veux passer d'une config à une autre, tu peux pas physiquement dé-pluguer tout et re-pluguer en un clic justement, et donc, enfin moi je ne pratique pas le synthé modulaire, mais ça me fait penser à cette caractéristique à laquelle je n'avais pas pensé avant, qui est que tu prends des chemins que tu tricotes et dé-tricotes, et d'une certaine manière, quand tu veux aller quelque part, tu ... (34:03) 

FD — ouais... ben par exemple, mettons, j'ai ça... (début de séquence musicale) si lui par exemple je veux le transformer, je sais pas, euh, j'aurais envie de le faire interagir avec lui, ce que je vais faire, je vais faire une espèce de tour de passe-passe, je vais venir faire un autre événement donc pendant ce temps je fais ça, donc il y a un truc musical qui est en train de se passer... je débranche mon truc... je prends un câble... et... je branche... je vais pouvoir ré-introduire mon son ... (fin de la démo) et par exemple des fois je peux... Ça peut m'arriver d'être moins dans le jonglage que ça et patcher alors qu'on entend le son... mais... c'est plus périlleux... faut être un peu joueur quoi, des fois ça ne fait pas du tout ce que tu avais prévu... 

VG — ouais... (rires) 

FD — hehe... donc là faut jouer avec, quoi ... mais c'est des trucs que j'aime bien en improvisation justement, les accidents, les surprises... des fois ça m'est arrivé en improvisant de... d'être surpris par le résultat et de trouver que c'était un peu moche, que c'était presque de mauvais goût, quoi, par rapport à mon goût personnel... donc là du coup, démerdes toi avec ça !... (rires) 

VG — et... t'as une sorte de pavage... c'est une autre chose qui m'intéressait qui est aussi caractéristiques des DMI de pouvoir changer la config en un clic et de ne pas se retrouver avec les mêmes choses sous les doigts, c'est à dire qu'un même bouton peut changer une fonction... 

FD — Ouais ... et bien là ça peut être aussi le cas... enfin pardon, je te coupe... 

VG — il y a à la fois cette question là, et il y a aussi la question de la construction musicale là dessus où ... en fait dans ces instruments qui sont un peu des méta-instruments, que tu ré-assembles, que ça soit en un clic ou via un patch, tu as une sorte de programmation plus ou moins en live, avec ton instruments... 

FD — ouais 

VG — et sur des longs set, sur la durée d'un concert, du coup tes stratégies pour passer d'un... enfin, est ce que tu fonctionnes plutôt avec un set continu ou est ce que tu as des... 

FD — ça dépend... 

VG — parce que quand tu as des pistes, des morceaux séparés, comment tu passes d'un morceau à l'autre, alors que dans les logiciels de type (Ableton) Live, les gens rappelle le presets du deuxième morceau... 

FD — non, alors je ne l'utilise pas du tout pour faire des morceaux précis que je rejoue... ça je ne fais pas du tout ça... en fait ce serait hyper compliqué de faire ça, d'ailleurs... c'est pas du tout prévu pour... je dis pas que c'est impossible mais à mon avis faut beaucoup s'entrainer pour un résultat pas forcément génial... par contre ça m'est déjà arrivé de faire un set en plusieurs fois, et au milieu de couper le son, débrancher, et repartir de zéro... comme si c'était un autre morceau, ou une autre pièce... enfin, on met les mots qu'on veut mais... et par contre ça m'arrive de l'utiliser dans des groupes, par exemple, où là il y a des choses qui sont prévues mais c'est plus de l'ordre de ... à tel moment, à tel tableaux, je sais que je fais des vagues de sons modulés en FM... et c'est pas plus prévu que ça... 

VG — c'est pas forcément une config qui est prévue mais plutôt un résultat sonore que tu atteins d'une manière ou d'une autre ? 

FD — alors, si... en général quand je fais des trucs avec des groupes, s'il y a des tableaux vraiment prévus, là je note des patchs de manière très succinte, c'est à dire, telle modulation sort de tel connecteur et va à tel connecteur, et en gros les positions des potars... donc on n'a pas exactement le même résultat mais on a quelque chose approchant, quoi... et de toute façon ça a vocation à jouer, j'en joue quoi... les positions de potars sont des positions moyennes quoi, après j'en change... peut-être pas tous, mais... 

VG — et ça me fait penser à une autre question qui est la question de l'ergonomie de l'instrument, parce que dans ces DMI, il y a une ergonomie de l'instrument qui n'est, a priori, pas dictée par des critères de facture acoustique, à l'inverse des instruments acoustiques... et du coup c'est souvent plus défini, à la fois par des contraintes techniques, du fait que tel module a besoin de tel bouton là, tout ça, et aussi par la contrainte personnelle, enfin les désirs ou les contraintes liée à ta propre pratique, à ton propre corps, à comment tu organises les choses que tu as sous les doigts... 

FD — ouais, aussi ta façon de le voir, de l'imaginer en amont... 

VG — oui... et par exemple pour ce dispositif là (le modulaire dont on parle, NdE) qu'est ce qui fait que tu as organisé les modules de cette manière là plutôt que d'une autre ? 

FD — j'ai essayé d'être au maximum logique... j'ai mes sources en haut, tout ce qui est modulation... donc les oscillateurs, et puis le sampleur dont je te parlais qui fait la radio, là... là, j'ai toute les sources qui vont venir transformer, envoyer du signal ou de l'aléatoire, de la commande un peu quoi... j'ai par exemple ici un mini séquenceur euclidien... 

VG — pour piloter les sources ? 

FD — euh... qui envoie des impulsions, quoi 

VG — c'est à dire, ceux du haut c'est plutôt des générations horizontales de ... 

FD — c'est plutôt eux qui vont me générer du son, et eux vont venir faire des variations de voltage, donc selon où je le branche ça fait quelque chose ou quelque chose... 

VG — des enveloppes ? 

FD — ça peut faire des enveloppes, ça peut faire des LFO, ça peut faire des... ouais. 

VG — qui vont venir sculpter tes sources... 

FD — ça peut venir changer la fréquence de ma radio... 

VG — d'accord 

FD — où ouvrir le filtre qui vient filtrer mon tambourin à corde 

VG —   

FD — ouais... je sais pas... euh... 

VG — c'est à dire quand tu as un LFO, je dis générateur de geste parce que on pourrait le moduler ... 

FD — oui, quelque part c'est comme si on déléguait à quelqu'un de faire ... (geste d'oscillation de la main) 

VG — ... sur le potar... 

FD — oui voilà c'est ça... et après en bas, c'est plutôt les trucs de geste, en fait là c'est bêtement parce que \hl{c'est là où je met les mains... } là sur le patch que j'ai montré tout à l'heure, je ne me suis pas servi de ça mais celui là je m'en sers pas mal en impro... 

VG — Sur es capteurs électrostatiques aussi non ? 

FD — Voilà ... qui lui, sert par exemple à ... c'est comme des banques de mémoire de voltage... donc là j'ai branché, y'a trois lignes, j'ai branché la sortie de la première ligne sur le pitch de ce premier oscillateur... (démo de l'effet) mais si je le branche sur la fréquence de coupure de mon filtre, je vais faire résonner le filtre (du tambour, NdE) 

VG — c'est une sorte de clavier que tu peux assigner à n'importe quel paramètre 

FD — voila, tout le truc qui est super avec ce système, c'est que tu as des voltages et selon où tu le branches, c'est des voltages qui vont contrôler une hauteur, un volume, une intensité, une vitesse, enfin bon comme dans Max en fait, quand tu as des trucs qui vont de 0 à 127, c'est pas 127dB, c'est juste à 127  

VG — c'est une course en 0-5V 

FD — oui, je sais plus, c'est plus ou moins 12V je crois 

VG — ton patch orange c'est ton patch de modulation ? 

FD — non j'ai pas de codage de couleur, mes câbles oranges c'est juste des jack-jack simples 

VG — mais tu as une différenciation entre ce qui est signal audio et modulation ? 

FD — non pas sur ceux là, sur certain oui, sur les Buchla il y a carrément un format différent de câble pour ce qui est signal et ce qui est euh... 

VG — là c'est tout au même niveau 

FD — oui 

VG — c'est à dire tu peux prendre un signal audio et le mettre comme une modulation d'autres trucs... 

FD — oui, et d'ailleurs je trouve que c'est un avantage parce que du coup tu peux faire des trucs qui n'étaient pas prévus et moduler un filtre en audio, par exemple, c'est un truc que j'adore faire... par exemple (faisant la démo) là j'ai une sortie audio de ce générateur d'aléatoire et là comme tu entends ça ... 

VG — c'est la fréquence de coupure que tu modules avec le signal audio ? 

FD — oui... ou ça pourrait être aller récupérer un sinus... donc tu déconstruis un peu des trucs de... ``ah les filtres, c'est ...''   

VG — ça existe aussi dans Max entre ce qui est signal et ce qui est message de contrôle 

FD — oui, tu peux faire la traduction et du coup c'est hyper bien ... 

VG — tu peux la faire, mais quand même elle est présente comme qui n'est pas ... 

FD — elle est marquée oui c'est vrai... 

VG — certains paramètres que tu voudrais pouvoir contrôler directement en audio et tu te retrouves à devoir les convertir en message avec ce que tu perds... 

FD — oui c'est ça, du coup tu as une fréquence d'échantillonnage un peu 

VG — oui... 

FD — après qui peut générer une autre esthétique mais... 

VG — et... j'ai perdu le fil de ce que je disais avant, je parlais d'ergonomie et... 

FD — oui, mais ça c'est marrant, je vois aussi des points commun entre quand je faisais du Max et ça c'est, euh, déjà c'est évolutif... c'est pas figé... et... euh, il y a quelque chose que j'ai pu trouver hyper pratique à un moment me semble absurde, peut-être un an après, et je vais changer des trucs... j'ai pas mal discuté avec des gens qui pratiquaient ça, le synthé modulaire, et il y en a certains qui ré-organisent de manière totalement aléatoire leur modules et ils disent que ça les fait  faire des choses différentes qu'ils ne faisaient pas du tout avant... du coup ça leur change leur logique, ça leur fait faire... ré-essayer des trucs qu'ils ne faisaient plus... ou qu'ils n'avaient jamais essayé... du coup c'est rigolo... 

VG — ah, tu as un truc qui n'est pas propre au son électronique, mais qui est fortement encouragé par ce genre de dispositif où tu es dans l'exploration sonore beaucoup, et c'est parfois la critique inverse qui est entendue à propos du fait que les instruments électroniques sont des instruments sur lesquels il est difficile d'avoir une pratique comme celle d'un violoncelliste, qui joue de son violoncelle qui ne bouge pas ... 

FD — oui, qui va rejouer exactement la même chose oui... 

VG — oui, ou alors explorer son instrument, mais elle est dans rechercher des choses fines et inaccessibles avec son instrument ... 

FD — ben pour moi en fait c'est différent... mais c'est ça qui est intéressant... c'est presque un autre pays, je sais pas, c'est euh... les règles ne sont pas les mêmes, les actions ne sont pas les mêmes, les résultats ne sont pas les mêmes ... et en fait c'est justement ça la richesse... là du coup, de me mettre au violon, je vois très bien ce que ça peut donner effectivement de ta ... tu l'attrapes, tu fais un trucs et voilà ça y est... alors tu le fais plus ou moins bien selon comment tu es aguérri mais ... il y a un truc très direct... mais là, dans ce monde là il y a aussi quelque chose de ... tu peux lancer des espèces de processus, de micro-vies qui se développent... tout à coup tu peux déclencher un énorme... alors par exemple, ça peut être vrai avec un orgue, tu peux aussi déclencher un orage, juste avec ton orgue mais... 

VG — il y a aussi quelque chose de génératif là dedans qui fait que tu déploies un univers dont tu ne perçois pas forcément tous les contours, un peu... 

FD — ... et puis il y a quelque chose de... il y a aussi les bons côté du machinique, tout à l'heure je disais, alors ça peut paraître péjoratif, je parlais de robinet à son, d'ailleurs ça peut être... un côté négatif du synthé, notamment en improvisation si on ne fait plus gaffe et qu'on laisse pisser un son, ça peut vite devenir pénible... mais il y a aussi l'avantage de,  tu peux laisser un truc exister dans un coin et tu fais quelque chose ailleurs, tu mets en route un autre truc et par touche tu créé une forêt, alors qu'avec ton violon ou ton violoncelle, tu joues maintenant, un truc... c'est chouette aussi, mais c'est carrément autre chose... voilà... 

VG — je pensais, notamment par rapport au fait que tu chantes, je ne suis pas sûr que cela ait un intérêt de savoir si on appelle ça un instrument ou pas, mais dans ce qui caractérise les instruments, j'ai l'impression qu'il y a quelque chose qui passe par le fait de pouvoir chanter, avec des gros guillemets, ce qu'on joue ... ça veut dire notamment dans l'improvisation, où tu as une part d'aléatoire qui existe avec ces instruments — peut-être avec des instruments acoustiques aussi, mais est-ce que tu as l'impression de ... comment est-ce que tu arrives à anticiper ce que tu fais justement, pour pouvoir improviser sans faire forcément juste ta sauce... 

FD — ouais... et bien déjà je connais bien mon instrument... et j'ai choisi les éléments que j'y ai mis par rapport à ce que j'avais envie d'y mettre, c'était des choses que j'utilisais déjà avant par ailleurs, donc je sais à peu près où je vais ... après des fois je suis surpris comme je te disais, ça fait pas exactement ce que j'avais prévu, ça ça arrive, mais quand même la plupart du temps ça fait grosso-modo ce que je pensais, et ce que j'aime beaucoup par exemple en improvisation c'est, il y a plein de fois où j'ai un tilt de me dire ``ah mais si je faisais ça'' et je fais le branchement et ça fait ce que j'avais imaginé, ou des fois c'est encore mieux ... tu vois un espèce de... 

VG — tu as une espèce de cartographie des recettes de cuisine qui permettent de faire ci ou ça ... 

FD — ouais... et puis des association d'idées, de ``ah mais j'ai jamais pensé de faire ça avec ça, tiens ce serait peut-être bien...''  

VG — quand on discutait avec Serge de Laubier, il me parlait du fait qu'on développe quand même malgré les différences ... il parlait de ça par rapport au fait que, quand a été créé la méta-mallette, qui est un dispositif utilisant des joysticks, il avait lancé ce projet avec l'idée que des amateurs puissent commencer sur le joystick une pratique musicale électro-acoustique et continuer après sur une interface plus experte comme le Méta-Instrument qu'il a construit... et en même temps je lui faisait remarquer qu'en terme d'ergonomie, cela n'a pas grand-chose à voir ... comment tu considères que c'est un peu le même instrument que tu joues quand tu passes d'un joystick de jeu-vidéo à un Méta-Instrument... et il me parlait de ça, du fait que tu développes une culture de ... par exemple, quand tu fais de la synthèse FM tu sais que lorsque tu es autour de tel ratio, ça sonne un peu comme une cloche, quand tu es autour de trois c'est plus cuivré... tu développes un peu une sorte de ... 

FD — tu peux pas vraiment dire ce qui va se passer mais ... 

VG — et quelque part que tu fasses ça avec Max ou avec un modulaire tu retrouves des chemins qui... 

FD — oui c'est net... et entre temps les quelques fois où je suis revenu sur Max pour X ou Y raisons, il y avait des cheminements que j'avais développé sur cet instrument (le modulaire) que je retraduisais dans Max... là pour ça, il y a vraiment un cousinage direct... c'est la même pensée... après c'est évidemment beaucoup plus souple Max parce que tu as autant d'oscillateurs que tu veux... c'est ... \hl{beaucoup plus infini} 

VG — je pensais aussi à ça aussi par rapport au fait que, notamment toi qui a donné des cours d'électroacoustique, comme tu enseignes des instruments comme ça, par rapport à une classe d'instrument classique, qu'est ce que t'apprends à des gens qui veulent faire de la musique avec ce genre d'outils ? 

FD — et bien effectivement cela va plutôt être, alors je ne sais pas comment il faudrait nommer ça, on pourrait dire des stratégies, ou des recettes de cuisine si on veut... une espèce de boite à outils... il y aurait la modulation de fréquence, la modulation d'amplitude, tout ce genre de trucs, l'aléatoire... 

VG — apprendre à se familiariser avec des processus en fait, plutôt que des objets ? 

FD — oui c'est ça... voilà oui... c'est un peu plus conceptuel, t'as raison... il faut être plus abstrait... 

VG — avec le concret du son quand même... 

FD — mais d'ailleurs, c'est marrant, en cours je voyais très bien qu'il y avait un truc comme ça presque ``mais qu'est ce qu'il nous dit ?'' à un moment tu as le type qui touche du doigt ce dont tu veux parler, et à partir du moment où tu as capté ça, c'est bon ... 

VG — parce que ce que tu disais était abstrait ? 

FD — ben oui, donc il y a un moment où il faut (faisant un signe d'ajustement avec les mains, NdE)... alors dans la pédagogie ça passe par l'exemple simplifié, faire un dessin au tableau, faire écouter des exemples... et puis vas-y... pratique... et ce moment où tu as le déclic, voilà... 

VG — tu as une partie de l'instrument qui est beaucoup dans les oreilles, dans les représentations que tu te fais de ces outils... 

FD — oui, d'ailleurs, on en parle pas tellement dans les gens qui font ça on ne parle pas tellement de nos cartographies mentales, qu'est ce qu'on se représente... 

VG — c'est marrant, j'ai l'impression qu'il y a quelque chose de ce qu'on pourrait peut-être, parce que je ne suis pas sûr que cela soit ce qu'on fasse réellement, apprendre en formation musicale, et d'autant plus que ces outils technologiques intègrent et réifient quelque part beaucoup de règles qui sont des règles abstraites, des gammes... la quantification de tout un tas de choses qui sont de l'ordre du solfège, qui sont transférées dans un dispositif technique qui les absorbent, il y a pas besoin par exemple de, si tu utilises un arpeggiateur, quelque part tu lui dit juste, je veux que ça soit calé sur du Ré mineur et il va faire ça vie sur du Ré mineur, et il a intégré le fait que le Ré mineur c'était tout cet ensemble de ratios entre les fréquences... 

FD — alors ça bon, je pense que c'est des outils hyper pratiques, mais je pense que ça a de gros travers, parce qu'effectivement, c'est un peu pareil que l'histoire du GPS... moi j'utilises de plus en plus le GPS et du coup il y a des endroits où je ne suis allé qu'avec le GPS et je ne saurais pas y aller sans mon GPS... alors que si je fais la démarche d'aller regarder la carte, de me noter le nom des villes et tout ça, je vais avoir des erreurs de parcours, et je vais peut-être me tromper, mais au final j'y vais deux fois et toute ma vie je saurai aller à cet endroit... alors qu'avec la technologie on externalise un truc, on s'en n'occupe pas et du coup l'inconvénient c'est qu'on ne fait pas ce travail là... mais... moi ce n'est pas pour cette raison que je n'en n'ai pas mis de quantiseur dans mon système, mais c'est plutôt qu'en fait, dans ce que j'en fait moi, je suis plutôt dans de la micro-tonalité, ou de l'atonalité... ce qui m'intéresse ça va être les battements, les phénomènes acoustiques... du coup je n'utilises pas ... mais c'est pas complètement vrai, je te disais que j'avais un séquenceur euclidien, il tient comme un métronome des rythmes hyper-complexes que je ne saurais pas jouer, donc je ne suis pas dans une radicalité de ...  ``il faut absolument apprendre à faire le truc'' mais je pense que ... c'est plus que dans ma pratique ça n'était pas pertinent d'avoir un quantiseur... 

VG — oui, je prenais l'exemple du quantiseur, mais ce n'était pas forcément le meilleur exemple par rapport à ta musique, mais tu as tout un tas de choses comme ça qui sont des outils qui... 

FD — oui par exemple tu n'as pas besoin d'apprendre le souffle continu, quand tu mets un son continu, tu l'allumes et tant qu'il y a de l'électricité, il est là quoi... 

VG — tu délègues l'énergie mais il y a aussi une part de la mémoire... je prenais l'exemple du quantiseur parce que pour le coup il y a une mémoire de quelles notes font partie de la gamme de Ré mineur, mais quelque part cette mémoire elle existe aussi quand tu intègres un sample, un enregistrement que tu as fait un autre moment, voilà... 

FD — oui tout à fait... oui 

VG — Bon, pour le coup ce serait assez difficile techniquement d'avoir un micro branché à l'océan pour faire ton sample mais ... tu as des bouts comme ça que la machine retient à ta place quoi... 

FD — et après moi, je réagissais à ce truc du quantiseur, parce que pour moi, autant Max que ces instruments, les synthétiseurs modulaires, justement, ça m'affranchit un peu des règles du solfège ... j'ai pas un clavier avec des touches noires et blanches qui me guident dans un truc en demi-tons, à tempérament égal... tout à l'heure comme tu as vu, j'accordais mon oscillateur (sur le clavier à mémoires, NdE) sur une gamme qui n'existe pas du tout... alors je peux le faire, ça m'est déjà arrivé de rentrer une gamme, je sais pas, diatonique, voilà on peut, mais on n'est pas limité à ça et c'est ça qui m'intéresse ... voilà tu vois, je vais avoir mes deux oscillateurs, un des trucs que j'aime beaucoup faire, j'ai mes deux oscillateurs et je vais essayer de les accorder l'un par rapport à l'autre ... il est où le ... mince... voilà... quelque chose que j'utilises beaucoup en improvisation... je vais m'accorder et me désaccorder légèrement et tous ces jeux de battements... (fait entendre les battements entre oscillateurs, NdE) on est entre les touches quoi ... 

VG — oui, et quelque part, tu utilises ça pour un jeu qui est presque percussif en fait... 

FD — oui, ça peut devenir rythmique, tout à fait ... 

VG — un rythme qui s'installe en partant d'un clavier mal-tempéré 

FD — oui c'est ça (rires) un clavier pas tempéré du tout !... 

VG — Peut-être une dernière question piège ... qu'est ce qui te semble te limiter dans ces outils là ou pour poser la question de manière moins négative, quelles sont les pistes où tu sens qu'il y a des choses à améliorer... est-ce que c'est rajouter des modules pour avoir plus de modules ? Est-ce que c'est d'avoir autre chose que des boutons rotatifs pour plus d'ergonomie ? Qu'est ce qui te pré-occupe sur ces instruments ? Ou peut-être des choses qui ne sont pas forcément aussi techniques que celles que je viens de citer... 

FD — je crois que paradoxalement, ce qui nous limite le plus dans ces instruments, qu'ils soient virtuels ou physiques, en fait c'est la profusion de possibles ... ce qui m'intéresse pas mal c'est de réduire et de me mettre des contraintes, et là ça devient souvent plus intéressant que ``waouh on peut tout faire, allons-y'' ... d'ailleurs je réfléchis à passer sur un rack plus petit de deux rangées qui serait un poil plus large, mais donc j'aurais moins de modules... pour plein de raisons... \hl{des choses aussi prosaïque que, quand je le trimballe, c'est lourd}, et tout ça... et au final, ce qui est ma pratique principale avec ces instruments c'est l'improvisation en temps-réel ou le jeu en temps réel, et dans ces cas là, il y a plein de modules que je n'utilises jamais. Ça m'arrive d'utiliser tous les modules et tous les câbles... c'est plutôt quand je fais des trucs génératifs, mais bon, c'est des choses qui m'amusent mais c'est pas du tout le truc principal de ma pratique musicale... 

VG — tu t'en sers plutôt comme d'un instrument... enfin, c'est à dire ``live'', j'entends...  

FD — Oui !... (rires) ... ``quand j'en joue''... 

VG — ``quand tu en joues'' ... oublions ce terme d'instrument... quand tu en joues tu préfères avoir moins de boutons et plus de réactivité ... 

FD — oui oui... systématiquement, quand je fais de l'impro, là ce que j'ai fait tout à l'heure c'est presque maximal, par rapport à ce que je fais d'habitude... il n'y a pas longtemps j'ai joué à Bordeaux pour une nuit sur le thème du sommeil, et j'ai quasiment utilisé que deux oscillateurs sinus... avec après des jeux de modulation dessus et tout ça, mais... ma base c'était ça... allez un moment j'ai utilisé trois oscillateurs et j'ai rajouté un petit paysage sonore, mais d'ailleurs à la ré-écoute j'ai trouvé que c'était superflu... (rires)... oui, il y a un truc que j'aime bien dans la limitation ... je pense que dans ce qui nous limite beaucoup c'est la profusion... en fait il faut faire un effort de limiter... je ne sais pas si tu as vu ce documentaire sur Eliane Radigue où elle est dans son studio face à son synthé... et elle parle d'une musique ``du bout des doigts''  ... j'aime vachement ça, j'ai montré cette émission aux étudiants parce que,  je trouve ça super juste, il y a un truc dans le ``trois fois rien''  des fois beaucoup plus efficace que d'avoir des trucs qui tournent dans tous les sens et on comprend plus rien... 

VG — oui... quelque chose qui se joue un peu sur un fil... 

FD — oui... ça c'est d'ailleurs un des travers, je ne sais pas si tu as déjà un peu fouillé sur les forums de synthé modulaires, il y a une grande partie des gens qui font cette pratique musicale qui ... il y a un passage un peu obligé, j'ai l'impression, de sur-accumulation ... et t'écoutes les trucs, c'est de la bouillie, ça dégouline de partout, il y a des sons partout mais tu n'as aucune forme, aucune intention, c'est presque de l'aléatoire complet, quoi ... et l'aléatoire je l'utilises beaucoup, ça m'intéresse vachement, ce n'est pas un jugement de valeur de l'aléatoire hein ... il y a un moment où... mais ça fait peut-être partie de l'apprentissage, on a besoin de... gribouiller et après on affine... 

VG — tu parles des forums, et... ce sont aussi des instruments où ... qui étant technologiques aussi, se prêtent beaucoup à la formation de groupes, de mailing-list, de forums, tout ça... toi, tu as l'impression que c'est quelque chose qui t'a beaucoup servi ou... ? 

FD — ah oui oui oui... donc avant de me lancer là-dedans, j'avais utilisé des synthétiseurs pendant au moins une vingtaine d'années, peut-être un peu moins mais... si c'est ça en gros... j'avais l'impression de très bien connaître la synthèse sonore. Déjà, quand j'ai commencé Max, j'ai un peu revu ma copie en disant ouh la la mais il y a plein de paramètres que j'avais délégué à la machine, justement, je tournais le filtre, mais je n'avais pas forcément idée de tout ce qu'il y avait derrière... Un exemple tout simple, euh... une enveloppe... j'avais l'impression que c'était l'enveloppe qui me découpait mon son, alors qu'en fait l'enveloppe, elle ne faisait que diriger mon amplificateur... et ça, tant que tu ne l'as pas fait, et bien tu crois que l'enveloppe, c'est un truc qui fait changer le son... c'est vrai ... (rires)... Et donc après quand je suis passé là-dessus (le modulaire, NdE), il y a eu encore un nouveau choc de trucs que j'avais... qui paradoxalement étaient un peu simplifiés dans Max... alors c'est vrai et c'est pas vrai parce qu'il y a des modules qui sont très complexes déjà et je ne regarde pas dedans... mais d'ailleurs, ça me fait penser à un autre truc auquel j'avais réfléchi en pensant à cette interview... c'est que, autant dans Max que dans cette pratique, il y a un moment où je me suis arrêté. C'est à dire que moi, mon but a toujours été musical, d'être musicien en fait, et j'ai jamais voulu, euh... j'ai très bien vu le potentiel de devenir programmeur et je vois très bien le plaisir qu'on peut avoir à ça, mais moi je n'ai pas voulu commencer à faire des études hyper-poussées d'ingénierie, par exemple... quitte à rester sur des recettes un peu simple, mais mon but était juste d'avoir des outils pour servir ma musique et j'ai commencé, de la même façon, dans le synthé modulaire, à acheter des modules en kit que j'ai soudé et tout ça... et j'ai vu qu'il pouvait y avoir la bascule de commencer à réfléchir à faire mes propres modules pour faire \emph{exactement} le truc que j'avais envie ... et dans une autre vie, ça m'intéresserait en fait... mais j'ai trop envie de faire de la musique, donc oui, il y a le moment où je m'arrête et il y en a d'autres qui le font et c'est génial je suis hyper content qu'ils le fassent... c'est super, mais moi je veux faire de la musique (rires) et je vois très bien le monde que c'est, pour l'avoir effleuré... 

VG — oui... et si les luthiers sont toujours un peu musiciens, avec ces outils technologiques, il y a une projection de plein choses dedans qui font que les frontières ne sont pas toujours clairement dessinées entre qui fabrique des instruments, qui en joue, qui compose, qui... 

FD — oui, c'est vrai... oui tout est brouillé, c'est plus aussi compartimenté que ça... c'est vrai... 

VG — et du coup, savoir à qui tu délègues l'écriture d'une pièce, ou d'un instrument, ou la performance d'une composition que tu as écrite... 

FD — oui, c'est vrai... 

VG — ce qui est étonnant, c'est qu'à la fois tu as ce mélange des ... 

FD — des fonctions, presque... 

VG — ... des fonctions... et en même temps, sur des outils qui je sont vraiment que très partiellement interchangeables en fait... c'est ça qui est étonnant... tu n'as pas deux synthés, deux instruments numériques ou analogiques pareils, quoi 

FD — Oui, ce qui pose de problèmes... 

VG — oui, de transmission... et que ce soit en hard ou en soft, parce que, oui les logiciels ne sont pas forcément compatibles tous entre eux, chacun à son... 

FD — oui, en même temps, la musique classique, on ne la joue aujourd'hui pas du tout pareil qu'à l'époque, et pas du tout pareil aujoud'hui qu'en 1960... pourquoi pas... le fantasme de l'éternité... 

VG — oui, mais au-delà du problème de l'éternité, ou plutôt en deça, il y a la question du répertoire... est ce qu'il y a un répertoire de musique électroacoustique, comment tu le joues, comment tu l'interprètes... par quoi cela passe, par quelle transmission cela passe... 

FD — pfff... il y a plein de réponses... différentes possibles... par exemple, la pratique de l'acousmatique sur support, c'est un truc qui m'a beaucoup intéressé et qui m'intéresse encore... j'aime bien le terme de ``cinéma pour l'oreille''  parce que... on diffuse quelque chose qui est déjà... l'œuvre est déjà là, elle est sur le support et on ... en gros c'est comme si on la sortait de sa boite quoi... et c'est super parce qu'avec ça tu peux faire des choses que tu pourras jamais faire en live, ne serait-ce que parce que tu as fait super gaffes à tous les équilibres, aux temporalités, à la forme globale... des choses faites avec du recul... un travail de montage, de plan et tout ça... vraiment cinéma, en fait... et de l'autre côté, l'improvisation en direct, ce qui est vachement important c'est ce qui se passe là, aujourd'hui... et si tu enregistres cette improvisation et que tu l'écoutes dans un autre contexte, ça aura presque perdu la moitié de son sens, parce qu'il n'y aura pas les gens en train de réagir, il n'y a pas l'acoustique de la pièce où c'était, le taux d'humidité qu'il y avait à ce moment là, les interactions entre les gens qu'il y avait à ce moment là... donc pour moi il y a certaines improvisations qui sont bien, qui tiennent la route à l'enregistrement, mais c'est hyper-rare... c'est presque pas l'objet, en fait... si on veut filer la métaphore, l'acousmatique c'est le cinéma et l'improvisation, ce serait presque comme une performance de... ouais, d''art contemporain... l'important, c'est là, c'est ce qui advient là... si on a une trace, c'est une trace, mais c'est tout... et du coup qu'est ce qu'on transmet de ça... bonne question... et comment on le transmet... je ne sais pas ... en allant à ces concerts, en participant à des concerts ... 

VG — Oui, pour toute la vie qui va avec en fait... plutôt que dans un objet sacré qui représente bien l'œuvre... 

FD — oui, voilà... 

VG — ... que tu peux donner à quelqu'un... 

FD — oui... c'est des trucs complètement différents... et d'ailleurs ça me fait penser que j'ai un duo d'improvisation que j'aime beaucoup avec un ami qui s'appelle François Bessac, lui son dispositif, il est ultra-simple, c'est de la récup, du détournement, c'est des petites platines vinyle, et à la place des vinyles, il va mettre de la moquette, ou il remplace la tête de lecture par des ressorts, il fait rebondir son truc, et il met des cymbales, il fait du larsen, enfin tout ce genre de trucs... et on a fait des session d'improvisation et ça nous est arrivé d'improviser ensemble pendant plusieurs jours... et au début d'être très contents, et il y a eu un moment où il me dit ``ouais mais toi ton dispositif il est vachement mieux... tu as tous tes sons au quart de poil... c'est hyper fin... c'est hyper...''... et moi j'étais en train de me dire dans ma tête, à ce moment là, j'étais en train de me dire ``ouais, mais c'est vachement plus malin son truc, c'est fait de bric et de broc, ça réagit de manière presque magique, il est là avec presque rien et c'est hyper beau'' ... (rires) ... donc voilà... l'important c'est presque pas l'outil, c'est ce qu'on en fait... 

VG — où on l'emmène... 

FD — oui... et d'ailleurs, là il vient à partir de demain, on se fait une session d'improvisation de deux ou trois jours, et là il m'a proposé qu'on ne fasse qu'avec des petits objets acoustiques, zéro amplification, zéro électricité... et ça m'intéresse pas mal, je n'ai jamais trop fait ça, sauf des séquences de jeu pour des pièces électroacoustiques mais c'est pas pareil, là tu fais du réservoir... là c'est c'est comment tu tiens un discours musical avec juste un bout de papier, une pince à linge... 

VG — ça fait travailler d'autres choses 

FD — oui... mais pour moi c'est pas en opposition, on est dans la continuité de ce qu'on fait en fait... au final on pourrait se retrouver lui avec un piano moi avec une contrebasse, dont on ne sait pas jouer ni l'un ni l'autre, et on ferait quelque chose... j'ai assez confiance en ça... 

VG — oui, c'est assez marrant d'ailleurs, je ne sais pas si l'improvisation libre est née en même temps que la musique électronique et électroacoustique, mais c'est un peu concomitant quand-même 

FD — ah oui, moi je vois vraiment des ponts entre les deux, alors il y a des gens qui sont très à cheval, qui sont d'un bord et qui conspuent l'autre, mais c'est pas trop le truc qui m'intéresse, je préfère voir les ponts qu'il y a, oui... 

VG — il y a quelque chose de rigolo avec ces instruments électroniques qui ne pouvaient pas forcément être joués et qui peut-être ne peuvent toujours pas être joué comme des instruments acoustiques avec des choses qui sont parfois ... enfin déjà avec une lecture du geste qui n'est pas la même, tu peux tourner un bouton et déclencher tout un paysage et du coup les musiciens en face n'ont pas une base classique et tu dois t'adapter avec ce que tu as dans l'oreille, et c'est marrant comme les deux ont contribué à se nourrir l'un l'autre, l'improvisation libre et les musique électroacoustiques 

FD — oui... et d'ailleurs, un des gros points communs que je vois, c'est dans la pédagogie, c'est que tu as un rapport au sonore et au musical qui est beaucoup plus direct, et après tu peux élaborer ... mais ... il y a quelque chose de beaucoup plus primaire, qui me plait vachement... et qui me fait me rendre compte, j'y avais déjà pensé mais, d'avoir cette discussion me renforce encore cette impression, que toutes les musiques que j'ai pratiquées depuis que je fais de la musique — si on accepte les deux ans de guitare classique qui étaient fort peu fructueux, en fait ce qui m'a vraiment intéressé c'est ça, musiques expérimentales, noise, improvisations, paysages sonores, improvisation libre, et musiques traditionnelles ... et dans tout ça, le point commun, c'est le timbre, la recherche, ... la prépondérance du timbre dans tout ça... et le rapport direct et ... en tout cas c'est cette façon là de l'attraper  qui m'intéresse, c'est le rapport un peu décomplexé au sonore... où il faut oublier un peu justement, les gammes, le sport musical et ... oui, d'abord commencer par la pratique et si tu veux aller plus loin, tu vas attraper d'autres bagages qui vont t'amener plus loin, mais pas commencer par te dire qu'il faut d'abord que tu fasses trois ans de solfège et après, peut-être,  tu auras le droit de faire un Do... et encore... 

VG — j'ai l'impression qu'il y a tout un langage musical qui, faute d'avoir été formalisé sous forme de solfège, avec des noms de notes, des symboles qui représentent les notes, et en musique électroacoustique comme en musique improvisé, on peine à représenter un peu ça, à le noter sur papier, à l'exprimer par des mots, et ... je vois avec le septet de musique électroacoustique improvisé où on a tous des instruments différents, c'était une des questions de trouver comment noter, si tu notes, comment au-delà des répétitions tu essaies d'avancer, et on a commencé à essayer d'établir une sorte de vocabulaire, une sorte de lexique qui fonctionne par image un peu... mais oui, ce que tu dis ça me fait vraiment penser à ça ... à tout ce langage musical qui est là mais qui n'est pas formalisé et qui du coup, n'est pas forcément reconnu à sa juste valeur... peut-être 

FD — ouais, mais il y a aussi je pense une espèce de syndrome, comment dire, de complexe d'infériorité par rapport au classique... je crois qu'il y a un truc comme ça... on est toujours dans un ``par rapport à'' ... et on veut prouver qu'on est des grands musiciens ... pour moi c'est une connerie en fait... je crois que notre civilisation, on a jeté l'oralité un peu vite... l'oralité, elle a de grandes vertues... Le truc de vouloir absolument — et ça rejoint ce qu'on disait tout à l'heure, un fantasme d'éternité, on croit qu'on va graver dans le marbre pour toujours une œuvre impérissable, mais on sait que c'est faux... le jour où le soleil s'arrête c'est fini tout ça... donc on a embrayé sur une civilisation de l'écrit qui a été mise sur un piédestal, comme l'accomplissement de tout, en jetant un peu vite l'oralité... Je ne sais plus quelle personne avait dit ça, je crois que c'est un conteur qui disait ``on dit les paroles s'en vont et les écrits restent, en fait c'est l'inverse'' Les écrits s'envolent et les paroles restent... en fait, des choses de mémoire orale qui perdurent de manière très prégnante alors qu'on aurait tendance à se dire que ça va disparaître et ... ça a été fait plein de fois, des historiens qui arrivaient dans des archives à recouper des trucs et ``tiens ça correspond à ce qu'on raconte entre les gens depuis longtemps'' 

VG — oui, quelque part, une fois que c'est écrit on peut l'oublier... 

FD — oui, c'est ça, on le délègue, c'est le syndrome du GPS ! (rires) et ça me fait penser à un autre truc qui m'a vraiment marqué... Je ne sais pas si tu as déjà entendu parler d'Angélique Fullin... 

VG — non... 

FD — ... qui est une pédagogue de la musique... La première fois que je l'ai vue, c'est dans un documentaire, ``Ecoute'' (de Miroslov Sebestik, NdE)

VG — ah oui, alors je l'ai vue mais je n'avais pas retenu son nom... 

FD — alors tu te rappelles, c'est cette femme qui fait un atelier dans une classe avec des enfants, et elle les fait se ballader, et leur demande de s'imprégner du paysage et de retenir la musique du monde quoi... et arrivés en classe il faut qu'ils reproduisent ce dont ils se souviennent, et on les voit faire des mini-trucs juste avec une feuille et un papier, des trucs hyper-fins, et elle dit justement que l'apprentissage musical, il y a une espèce d'absurdité pour elle, dans le système des conservatoires, par exemple, c'est qu'on commence par appendre le solfège et elle dit, la musique c'est un langage, c'est une sorte de langage, même si on peut pas exactement mettre un égal mais bon... tu commences l'apprentissage de la parole, en commençant par l'imitation, le jeu, tu expérimentes, et une fois que tu sais très bien parler —parce que les enfants savent très bien parler avant de savoir lire, tu commence seulement à apprendre les lettres, et apprends tu apprends à lire et écrire, et après tu apprends les règles grammaticales... tu ne commences pas par la grammaire... alors que là, dans le système classique le plus répandu de l'apprentissage de la musique, tu commences par apprendre la grammaire... Mais bien sûr que les gens ne font pas de la musique après !... Bien sûr que les gens te demandent comment tu fais pour improviser... t'es trop fort, moi j'ai 20 ans de conservatoire, je ne sais pas improviser... ben oui... tu as commencé à l'envers... 

VG — oui... et généralement en plus, tu commences ton instrument tout seul... au lieu de faire une pratique de groupe où tu écoutes les autres pour éventuellement devenir un jour un fameux soliste, tu commences comme  un fameux soliste qui ne sait rien faire... 

FD — oui oui... et la notion de plaisir elle est où aussi ? De jeu ? ... il n'y en a pas, c'est austère, c'est l'école, c'est ``faut pas se tromper, sinon t'as faux...''  ... les fausses notes... 

VG — ça fait penser à Musicking de Christopher Small, qui parle du verbe ``musiquer'' pluôt que de ``faire de la musique'' pour mettre l'emphase sur l'action plutôt que sur un objet... 

FD — ah je ne connais pas, mais ça m'intéresse d'avoir la référence... et donc je te parlais d'Angélique Fullin, et ce documentaire ``Ecoutes'', je l'ai vu plusieurs fois et chaque fois, ça m'a marqué cette séquence avec cette femme... et j'en parlais à Jean-François Tisner, qui est un musicien qui lui-aussi a une casquette de musicien traditionnel et compositeur électroacoustique et qui mélange les deux, il me dit ``ah mais oui, Angélique Fullin, je la connais très bien, je l'ai croisée plein de fois...''  et après, plus tard, en cherchant des trucs, je suis tombé sur une discussion sur la pédagogie de la musique où on essaie de voir où est ce qu'on va, qu'est ce qu'on fait... tout ce dont on est en train de parler... pourquoi on fait tout ça... et dans les gens qui étaient en train de discuter avec elle, il y avait Alain Savouret, qui est un des compositeur qui m'a le plus marqué en électroacoustique, une liberté, un jeu, tout ça... entre parenthèse, j'ai appris qu'il avait fait des trucs avec Jean-François Tisner, Christian Vieussens, des gens qui font de la bourrée en Auvergne, donc tout un truc qui m'importe beaucoup... et donc il y avait Angélique Fullin, Alain Savouret, et Xavier Vidal, qui est le mec qui a dirigé la classe de musique traditionnelle dans laquelle je suis en DEM... un espèce de truc où tout s'assemble et en fait je ne suis pas juste un mec absurde qui fait des collages étranges, pour moi il y a une unité logique dans tout ça, qui s'est manifestée dans ces signes... 

VG — qui sont une manière d'attraper les musiques qui se fichent un peu des étiquettes 

FD — ouais, et qui n'a pas peur de rigoler, qui n'est pas forcément la plus carriériste... et qui n'est pas en train de délimiter un petit pré carré à grand coup de théorie de ``nous avons raison, les autres sont des imbéciles et n'entendent rien à la musique'' (rires) 
