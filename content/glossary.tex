% !TEX root = ../thesis-example.tex
%

% usage : \gls{IRCAM} 

\newacronym{DMI}{DMI}{Instrument de musique numérique (\textit{Digital Musical Instrument})}

\newacronym{GRM}{GRM}{Groupe de recherches musicales}

\newacronym{IRCAM}{IRCAM}{Institut de Recherche et Coordination Acoustique/Musique}

\newacronym{MIDI}{MIDI}{\textit{Musical Instrument Digital Interface}}

\newacronym{OSC}{OSC}{\textit{Open Sound Control}}

\newacronym{MPE}{MPE}{\textit{Multidimensional Polyphonic Expression}}

\newacronym{rtpMIDI}{rtpMIDI}{\textit{Real-time Protocol MIDI}:  (également connu sous le nom « AppleMIDI ») est un protocole destiné au transport de messages MIDI dans des messages RTP (Real-Time Protocol), sur des réseaux Ethernet et Wi-Fi.}

\newacronym{NIME}{NIME}{\textit{New Interfaces for Musical Expression}}

\newacronym{mapping}{mapping}{\textit{c'est le mapping}}

\newacronym{HCI}{HCI}{Interface Homme-Machine (\textit{Human Computer Interface})}

\newacronym{IHM}{IHM}{Interface Homme-Machine}

\newacronym{CNMAT}{CNMAT}{Center for New Music and Audio Technologies: multidisciplinary research center within University of California, Berkeley Department of Music}

\newacronym{GDIF}{GDIF}{Gesture Description Interchange Format: un format développé pour le streaming et l'enresgistrement de données de mouvement en relation à la musique}

\newacronym{UDP}{UDP}{User Datagram Protocol: protocole asynchrone de transmission de données sur réseau, utilisant un mode de transmission sans accusé de réception}

\newacronym{MMA}{MMA}{MIDI Manufacturer Association:  organisation professionnelle sans profit, composée principalement de développeur de solutions matérielles et logicielles et travaillant sur les standards liés à la norme MIDI.}

\newacronym{LFO}{LFO}{Low Frequency Oscillator: circuit électronique (ou simulé en numérique) permettant la modulation d'un signal à des fréquences inférieures aux fréquences audio, permettant notamment des effets tel que le vibrato.}