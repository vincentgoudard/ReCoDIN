% !TEX root = ../thesis-example.tex
%
% usage : \gls{IRCAM} 

% \newglossaryentry{API_g}{name={API}, description={An Application Programming Interface (API) is a particular set of rules and specifications that a software program can follow to access and make use of the services and resources provided by another particular software program that implements that API}}

% \newglossaryentry{API}{type=\acronymtype, name={API}, description={Application
% Programming Interface}, first={Application
% Programming Interface (API)\glsadd{API_g}}, see=[Glossaire:]{apig}}


% TODO : enelver les éventuels point à la fin des description, ils sont automatiquement ajoutés par le formatage.
% query/replace dans le document des terme du glossaire
 

\newglossaryentry{ACROE}
{
    name={ACROE},
    description={\textit{\glsentrylong{ACROE}}: créée en 1976 par Claude Cadoz, Jean-Loup Florens et Annie Luciani à l’INP de Grenoble, avec le soutien du ministère de la Culture, l'ACROE a mené des travaux scientifiques, technologiques, artistiques et pédagogiques, dans le secteur des arts multisensoriels numériques. Elle a élaboré plusieurs outils de création artistique dont le paradigme central est la situation instrumentale, notamment le système \gls{CORDIS-ANIMA}},
    long={Association pour la Création et la Recherche sur les Outils d'Expression},
    first={\glsentrylong{ACROE} (\glsentryname{ACROE})}
}
\newglossaryentry{ADSR}
{
    name={ADSR},
    description={\textit{\glsentrylong{ADSR}}: acronyme utilisé pour désigner une enveloppe temporelle, typiquement utilisée pour moduler l'amplitude d'un son produit par un synthétiseur et spécifiée par ces quatre paramètres : temps d'attaque (A), de décroissence (D), d'extinction (R), ainsi que la valeur de maintien (S)},
    long={Attack Decay Sustain Release},
    first={\glsentrylong{ADSR} (\glsentryname{ADSR})}
}
\newglossaryentry{AFIM}
{
    name={AFIM},
    description={\textit{\glsentrylong{AFIM}}: Fondée en 2002 avec le soutien de la Direction de la Musique de la Danse, du Théâtre et des spectacles, l’AFIM est issue de la fusion de deux associations : l’ADERIM et la SFIM, dont elle reprend les objectifs et les actions, en particulier le pilotage depuis 1994 des Journées d'Informatique Musicale (JIM)},
    long={Association Francophone d'Informatique Musicale},
    first={\glsentrylong{AFIM} (\glsentryname{AFIM})}
}
\newglossaryentry{AHRS}
{
    name={AHRS},
    description={\textit{\glsentrylong{AHRS}}: ensemble de capteurs (3 gyroscopes, 3 accéléromètres, et une boussole à vanne de flux 3 axes) permettant de mesurer la position et l'orientation dans l'espace, grâce aux accélérations et aux champs magnétiques qu'ils subissent},
    long={Attitude and Heading Reference System},
    first={\glsentryname{AHRS}}
}
\newglossaryentry{AIMI}
{
    name={AIMI},
    description={\textit{\glsentrylong{AIMI}}: Association italienne fondée en 1981, ayant pour objectif de coordonner et de favoriser le développement de l'informatique musicale et des activités liées à l'interaction entre la technologie et le son. Elle organise en particulier la conférence CIM (Colloqui di Informatica Musicale), qui existe depuis 1976},
    long={Association Italienne d'Informatique Musicale},
    first={\glsentrylong{AIMI} (\glsentryname{AIMI})}
}
\newglossaryentry{ANR}
{
    name={ANR},
    description={\textit{\glsentrylong{ANR}}: agence de moyens créée en 2005, qui finance la recherche publique et la recherche partenariale en France},
    long={Agence Nationale de la Recherche},
    first={\glsentrylong{ANR} (\glsentryname{ANR})},
    firstplural={\glsentrylong{ANR}\glspluralsuffix\ (\glsentryname{ANR}\glspluralsuffix)}
}
\newglossaryentry{APO33}
{
    name={APO33},
    description={Association créée en 1997 à Nantes, se définissant comme ``un laboratoire artistique, technologique et théorique transdisciplinaire qui développe des projets collectifs divers alliant recherche, expérimentation et intervention dans l’espace social.'' APO33 a réalisé de multiples projets dans le domaine des lutheries numériques, notamment la distribution logicielle APODIO, ou encore le Grand Orchestre d'Ordinateurs (GOO)}
}
\newglossaryentry{API}
{
    name={API},
    description={\textit{\glsentrylong{API}}: une interface de programmation d’application est un ensemble normalisé de classes, de méthodes ou de fonctions qui sert de façade par laquelle un logiciel offre des services à d'autres logiciels},
    long={Application Programming Interface},
    first={\glsentrylong{API} (\glsentryname{API})},
    firstplural={\glsentrylong{API}\glspluralsuffix\ (\glsentryname{API}\glspluralsuffix)}
}
\newglossaryentry{ASP}
{
    name={ASP},
    description={\textit{\glsentrylong{ASP}}: un ensemble de logiciels développés par Microsoft et utilisés dans la programmation Web, développé entre les années 1996 et 2000},
    long={Active Server Pages},
    first={\glsentrylong{ASP} (\glsentryname{API})}
}
\newglossaryentry{CD}
{
    name={CD},
    description={\textit{\glsentrylong{CD}}: disque optique développé par Sony et Philips, et introduit en 1982, utilisé pour stocker des données sous forme numérique. Notamment dans sa version CD-Audio, le CD permet d'enregistrer 74min (voire 80min) de son sur deux canaux encodés en 16bits},
    long={Compact Disc},
    first={\glsentrylong{CD} (\glsentryname{CD})}
}
\newglossaryentry{circuit-bending}
{
    name={circuit-bending},
    description={Activité consistant à détourner expérimentalement des appareils électroniques à des fins créatives, pour créer de nouveaux générateurs de sons ou d'images. Reed Ghazala est reconnu comme l'un des pionners dans cette démarche entreprise à la fin des années 1960}
}
\newglossaryentry{CNC}
{
    name={CNC},
    description={\textit{\glsentrylong{CNC}}: machine-outil à commande numérique},
    long={Computer Numerical Control},
    first={machine-outil à commande numérique (\glsentryname{CNC})}
}
\newglossaryentry{CNMAT}
{
    name={CNMAT},
    description={\textit{\glsentrylong{CNMAT}}: centre de recherche multidisciplinaire du Departement de Musique de l'Université de Californie, Berkeley, USA},
    long={Center for New Music and Audio Technologies},
    first={\glsentrylong{CNMAT} (\glsentryname{CNMAT})}
}
\newglossaryentry{CCRMA}
{
    name={CCRMA},
    description={\textit{\glsentrylong{CCRMA}}: centre de recherche multidisciplinaire en musique et acoustique de l'université Stanford, San Francisco, USA},
    long={Center for Computer Research in Music and Acoustics},
    first={\glsentrylong{CCRMA} (\glsentryname{CCRMA})}
}
\newglossaryentry{CEMAMu}
{
    name={CEMAMu},
    description={\textit{\glsentrylong{CEMAMu}}: institut de recherche et centre en musique contemporaine fondé en 1966 par le compositeur Iannis Xenakis. A l'origine nommé EMAMu, il devient par la suite le CEMAMu en 1972, puis Les Ateliers UPIC en 1985, puis le CCMIX en 2000},
    long={Centre d'Études de Mathématique et Automatique Musicale},
    first={CEMAMu}
}
\newglossaryentry{CORDIS-ANIMA}
{
    name={CORDIS-ANIMA},
    description={CORDIS - ANIMA est à la fois un formalisme générique et un langage. Il permet de concevoir et de décrire tout objet physique ou tout comportement dynamique que l'on souhaite modéliser et simuler à l'aide de l'ordinateur sous la forme d'un réseau de composants discrets élémentaires interconnectés}
}
\newglossaryentry{CLI}
{
    name={CLI},
    description={\textit{Command Line Interface} interface humain-machine dans laquelle la communication entre l'utilisateur et l'ordinateur s'effectue en mode texte},
    first={\glsentrylong{CLI} (\glsentryname{CLI})},
    firstplural={interfaces en ligne de commande(\glsentryname{API}\glspluralsuffix)}
}
\newglossaryentry{CPU}
{
    name={CPU},
    description={\textit{Central processing Unit} (micro)processeur central d'un ordinateur},
    long={Processeur Central},
    first={\glsentrylong{CPU} (\glsentryname{CPU})}
}
\newglossaryentry{CAN}
{
    name={CAN},
    description={\textit{\glsentrylong{CAN}}: (ou DAC pour \textit{Digital to Analog Converter}) composant électronique dont la fonction est de transformer une valeur numérique (codée sur plusieurs bits) en une valeur analogique proportionnelle à la valeur numérique codée, généralement la tension électrique},
    long={Convertisseur Analogique-Numérique},
    first={\glsentrylong{CAN} (\glsentryname{CAN})}
}
\newglossaryentry{DIY}
{
    name={DIY},
    description={\textit{\glsentrylong{DIY}}: pratiques et mouvement culturel, visant à créer et réparer soi-même des objets souvent techniques, sans nécessairement disposer d'une formation académique pour ce faire. La notion de DIY est apparue au début du \siecle{20}~siècle en Amérique du Nord à travers un certain nombre de magazines grand public, avant de connaître un essor dans les années 1970 auprès du public étudiant, en écho à la révolution socio-culturelle de cette époque. Un second essor au \siecle{21}~siècle est lié à la disponibilité de ressources sur Internet, tels que des tutoriels vidéos et des forums en ligne},
    long={Do-It-Yourself},
    first={\glsentrylong{DIY} (\glsentryname{DIY})}
}
\newglossaryentry{DJ}
{
    name={DJ},
    description={\textit{\glsentrylong{DJ}}: musicien·ne dont la pratique consiste en la diffusion, le mixage d'enregistrements audio (à l'origine sur platine vinyle, mais d'autres supports existent depuis) ainsi que l'utilisation de différentes techniques permettant d'ajouter des effets et de redécouper le morceau dans le temps. L'apparition du DJ en tant que personne chargée de l'enchainement de disque date des années 1930, mais l'expressivité musicale des DJs s'est réellement développée à la fin des années 1970, avec la naissance du break, requérant une performance technique conséquente et donnant lieu à des musiques dans lesquelle l'identité des morceaux originaux, utilisés de manière fragmentée, passe au second plan par rapport au travail du DJ lui-même},
    long={Disc Jockey},
    first={\glsentrylong{DJ} (\glsentryname{DJ})},
    firstplural={\glsentrylong{DJ}\glspluralsuffix\ (\glsentryname{DJ}\glspluralsuffix)}
}
\newglossaryentry{DMI}
{
    name={DMI},
    description={\textit{\glsentrylong{DMI}}: instrument de musique numérique},
    long={Digital Musical Instrument},
    first={Instrument de Musique Numérique (DMI)},
    firstplural={Instruments de Musique Numériques (DMIs)}
}
\newglossaryentry{DSP}
{
    name={DSP},
    description={\textit{\glsentrylong{DSP}}: Processeur de signal numérique},
    long={Digital Signal Processor},
    first={\glsentrylong{DSP} (\glsentryname{DSP})},
    firstplural={\glsentrylong{DSP}\glspluralsuffix\ (\glsentryname{DSP}\glspluralsuffix)}
}
\newglossaryentry{e-bow}
{
    name={e-bow},
    description={(littéralement "archet électronique") accessoire pour guitare inventé par Greg Heet en 1969, et commercialisé par \textit{Heet Sound Products}, fonctionnant comme un excitateur électromagnétique portatif permettant d'obtenir des sons entretenus rappelant ceux produits par un archet sur des cordes},
    long={e-bow}
}
\newglossaryentry{EDM}
{
    name={EDM},
    description={\textit{\glsentrylong{EDM}}: Terme regroupant les différents genre de musiques électroniques dansantes, jouées principalement dans les clubs et discothèques},
    first={\glsentrylong{EDM} (\glsentryname{EDM})},
    long={Electronic Dance Music}
}
\newglossaryentry{EMG}
{
    name={EMG},
    description={un capteur EMG mesure l'activité des muscles},
    long={électromyographie}
}
\newglossaryentry{fab-lab}
{
    name={fab-lab},
    description={contraction de l'anglais fabrication laboratory, « laboratoire de fabrication ». Tiers-lieu de type \gls{makerspace} cadré par le \gls{MIT} et la FabFoundation},
    long={fab lab},
    first={fab-labs},
    firstplural={fab-labs}
}
\newglossaryentry{FAUST}
{
    name={FAUST},
    description={\textit{\glsentrylong{FAUST}}: langage de programmation fonctionnel développé par le \gls{GRAME}, dédié à l'implémentation d'algorithmes de traitement du signal sous forme de bibliothèques, de plug-ins audio ou d'applications autonomes. \url{http://faust.grame.fr}},
    long={Functional AUdio STream},
    first={\glsentrylong{FAUST} (\glsentryname{FAUST})}
}
\newglossaryentry{firewire}
{
    name={firewire},
    description={nom commercial donné par Apple à une interface série multiplexée, aussi connue sous la norme IEEE 1394 et également connue sous le nom d'interface i.LINK, nom commercial utilisé par Sony. Il s'agit d'un bus informatique véhiculant à la fois des données et des signaux de commandes des différents appareils qu'il relie},
    long={\textit{firewire}}
}
\newglossaryentry{FM}
{
    name={FM},
    description={\textit{Frequency Modulation}: Technique de traitement de signal consistant à moduler un signal basse fréquence par un autre de fréquence plus élevée (appelé onde porteuse), généralement utilisée à des fins de télécommunication, comme par exemple dans le cas de la bande radiophonique ``FM''. Cette même technique est à la base d'une synthèse sonore inventée par John Chowning en 1973 au \gls{CCRMA}, notamment utilisée dans le célèbre synthétiseur DX7 de Yamaha},
    long={modulation de fréquence},
    first={\glsentryname{FM}}
}
\newglossaryentry{FSR}
{
    name={FSR},
    description={\textit{\glsentrylong{FSR}}: Capteur de force résistif, produisant plus ou moins de résistance en fonction de la pression qui lui est appliquée},
    long={Force Sensing Resistor},
    first={\glsentrylong{FSR} (\glsentryname{FSR})}
}
\newglossaryentry{FTS}
{
    name={FTS},
    description={\textit{\glsentrylong{FTS} : FTS est le nom du logiciel développé en 1989 par l'IRCAM pour le traitement et la synthèse audio en temps-réel. Il tournait à l'origine sur l'\gls{ISPW} avant d'être intégré directement dans le logiciel Max}},
    long={Faster Than Sound},
    first={\glsentrylong{FTS} (\glsentryname{FTS})}
}
\newglossaryentry{GALS}
{
    name={GALS},
    description={\textit{\glsentrylong{GALS}}: architecture pour la conception de circuits électroniques ou de programmes informatiques, qui répond au problème de la sécurité et de la fiabilité du transfert de données entre domaines d'horloge indépendants. Ce modèle de calcul apparu dans les années 1980 permet de concevoir des systèmes informatiques composés de plusieurs modules synchrones (en utilisant une programmation synchrone pour chacun de ces modules) interagissant avec d'autres modules en communication asynchrone},
    long={Globalement Asynchrone, Localement Synchrone},
    first={\glsentrylong{GALS} (\glsentryname{GALS})}
}
\newglossaryentry{GDIF}
{
    name={GDIF},
    description={\textit{\glsentrylong{GDIF}}: format développé pour le streaming et l'enresgistrement de données de mouvement en relation à la musique},
    long={Gesture Description Interchange Format},
    first={\glsentrylong{GDIF} (\glsentryname{GDIF})}
}
\newglossaryentry{glitch}
{
    name={glitch},
    description={genre de musique électronique ayant émergé dans les années 1990, dans laquelle l'usage délibéré de disfonctionnements comme source de matériaux sonores occupe une place centrale},
    long={\textit{glitch}}
}
\newglossaryentry{GMU}
{
    name={GMU},
    description={\textit{\glsentrylong{GMU}}: environnement de syntèse granulaire pour Max, développé au \gls{GMEM} par Laurent Pottier et Charles Bascou en 2003},
    long={GMEM Microsound Universe},
    first={\glsentrylong{GMU} (\glsentryname{GMU})}
}
\newglossaryentry{GRAME}
{
    name={GRAME},
    description={\textit{\glsentrylong{GRAME}}: Centre national de création musicale, créé à Lyon en 1982 par Pierre Alain Jaffrennou et James Giroudon},
    long={Groupe de réalisation et de recherche appliquée en Musique Électroacoustique},
    first={\glsentrylong{GRAME} (\glsentryname{GRAME})}
}
\newglossaryentry{GMEM}
{
    name={GMEM},
    description={\textit{\glsentrylong{GMEM}}: Centre national de création musicale, créé à Marseille en 1972 par un collectif de compositeurs dont Georges Boeuf, Michel Redolfi et Marcel Frémiot},
    long={Groupe de Musique Expérimentale de Marseille},
    first={\glsentrylong{GMEM} (\glsentryname{GMEM})}
}
\newglossaryentry{GRM}
{
    name={GRM},
    description={\textit{\glsentrylong{GRM}}: Centre de recherche musicale dans le domaine du son et des musiques électroacoustiques fondé par Pierre Schaeffer en 1958 à Paris, France},
    long={Groupe de recherches musicales},
    first={\glsentrylong{GRM} (\glsentryname{GRM})}
}
\newglossaryentry{GPU}
{
    name={GPU},
    description={\textit{Graphics Processing Unit} (micro)processeur principal d'une carte graphique, assurant les fonctions de calcul de l'affichage, optimisé pour le calcul d'image 2D et 3D par une structure de traitement parallèle de blocs de données telles que des matrices},
    long={processeur graphique},
    first={\glsentrylong{GPU} (\glsentryname{GPU})}
}
\newglossaryentry{GUI}
{
    name={GUI},
    description={\textit{Graphical User Interface}: dispositif de dialogue humain-machine, dans lequel les objets à manipuler sont dessinés sous forme de pictogrammes à l'écran, de sorte que l'usager peut utiliser en imitant la manipulation physique de ces objets avec un dispositif de pointage, le plus souvent une souris. Ce type d'interface est apparu dans les années 1970 pour remplacer notamment les interfaces \gls{CLI}, moins facile d'utilisation},
    long={interface graphique},
    first={\glsentrylong{GUI} (\glsentryname{GUI})}
}
\newglossaryentry{HCI}
{
    name={HCI},
    description={\textit{\glsentrylong{HCI}}: Interface Humain-Machine},
    long={Human-Computer Interface},
    first={\glsentrylong{HCI} (\glsentryname{HCI})},
    firstplural={\glsentrylong{HCI}\glspluralsuffix\ (\glsentryname{HCI}\glspluralsuffix)}
}
\newglossaryentry{HTML5}
{
    name={HTML5},
    description={\textit{\glsentrylong{HTML5}}: est la dernière révision majeure du HTML datant de 2014, définissant la syntaxe permettant de représenter les pages web. Cette version intègre notamment les langages CSS et Javascript},
    long={HyperText Markup Language 5},
    first={\glsentryname{HTML5}}
}
\newglossaryentry{IA}
{
    name={IA},
    description={\textit{\glsentrylong{IA} : désigne l'intelligence des machines, par opposition à l'intelligence ``naturelle'' des humains. Ce terme est utilisé pour décrire l'ensemble des théories et des techniques (généralement informatiques) qui imitent les fonctions cognitives associées à l'esprit humain, telles que l'apprentissage et la résolution de problèmes}},
    long={Intelligence Artificielle},
    first={\glsentrylong{IA} (\glsentryname{IA})}
}
\newglossaryentry{ICLC}
{
    name={ICLC},
    description={\textit{\glsentrylong{ICLC}} : Conférence Internationale sur le Live-Coding, conférence annuelle dont la première édition a eu lieu en 2015},
    long={International Conference on Live-Coding},
    first={\glsentrylong{ICLC} (\glsentryname{ICLC})}
}
\newglossaryentry{ICLI}
{
    name={ICLI},
    description={\textit{\glsentrylong{ICLI}} : conférence interdisciplinaire biennale axée sur le rôle des interfaces dans toutes les activités de performance artistique impliquant leur utilisation en direct},
    long={International Conference on Live Interfaces},
    first={\glsentrylong{ICLI} (\glsentryname{ICLI})}
}
\newglossaryentry{ICMA}
{
    name={ICMA},
    description={\textit{\glsentrylong{ICMA}} : affiliation internationale d'individus et d'institutions impliqués dans les aspects techniques, créatifs et de performance de la musique par ordinateur, dont la mission principale est l'organisation de la conférence \gls{ICMC}},
    long={International Computer Music Association},
    first={\glsentrylong{ICMA} (\glsentryname{ICMA})}
}
\newglossaryentry{ICMC}
{
    name={ICMC},
    description={\textit{\glsentrylong{ICMC}} : Conférence Internationale sur la Musique par Ordinateur, organisée annuellement par l'\gls{ICMA} et dont la première édition a eu lieu en 1974},
    long={International Computer Music Conference},
    first={\glsentrylong{ICMC} (\glsentryname{ICMC})}
}
\newglossaryentry{IDM}
{
    name={IDM},
    description={\textit{\glsentrylong{IDM}} : genre de musique électronique construite sur les même bases que le \gls{glitch} en terme de matériau sonore, et sur l'usage de structures rythmiques produites algorithmiquement, leur conférant une ésthétique à la fois machinique et destructurée (TODO: améliorer cette définition)},
    long={\textit{Intelligent Dance Music}},
    first={\glsentrylong{IDM} (\glsentryname{IDM})}
}
\newglossaryentry{IDMIL}
{
    name={IDMIL},
    description={\textit{\glsentrylong{IDMIL}}: Laboratoire de recherche en \gls{IHM} appliquée à l'interaction musicale, affilié au département de Musique de l'Université McGill (Montréal, Canada)},
    long={Input Devices and Music Interaction Laboratory},
    first={\glsentrylong{IDMIL} (\glsentryname{IDMIL})}
}
\newglossaryentry{IHM}
{
    name={IHM},
    description={\textit{\glsentrylong{IHM}}},
    long={Interface Humain-Machine},
    first={\glsentrylong{IHM} (\glsentryname{IHM})},
    firstplural={\glsentrylong{IHM}\glspluralsuffix\ (\glsentryname{IHM}\glspluralsuffix)}
}
\newglossaryentry{IMU}
{
    name={IMU},
    description={\textit{\glsentrylong{IMU}} (centrale inertielle) : instrument capable d'intégrer les mouvements d'un mobile (accélération et vitesse angulaire) pour estimer son orientation (angles de roulis, de tangage et de cap), sa vitesse linéaire et sa position},
    long={Inertial Measurement Unit},
    first={\glsentrylong{IMU} (\glsentryname{IMU})},
    firstplural={\glsentrylong{IMU}\glspluralsuffix\ (\glsentryname{IMU}\glspluralsuffix)}
}
\newglossaryentry{IRCAM}
{
    name={IRCAM},
    description={\textit{\glsentrylong{IRCAM}}: Centre français de recherche scientifique, d'innovation technologique et de création musicale, fondé par Pierre Boulez en 1970 à Paris, France},
    long={Institut de Recherche et Coordination Acoustique/Musique},
    first={\glsentrylong{IRCAM} (\glsentryname{IRCAM})},
    firstplural={\glsentrylong{IRCAM}\glspluralsuffix\ (\glsentryname{IRCAM}\glspluralsuffix)}
}
\newglossaryentry{IReMus}
{
    name={IReMus},
    description={\textit{\glsentrylong{IReMus}}: créé en 2014 du regroupement de trois équipes : l'OMF (Observatoire Musical Français) et PLM (Patrimoine et Langages Musicaux) de l’Université Paris-Sorbonne et l’IRPMF (Institut de recherche sur le patrimoine musical en France). \url{http://www.iremus.cnrs.fr}},
    long={Institut de Recherche en Musicologie},
    first={\glsentrylong{IReMus} (\glsentryname{IReMus})}
}
\newglossaryentry{ISGS}
{
    name={ISGS},
    description={\textit{\glsentrylong{ISGS}} : association internationale créée en 2002 et dédiée à l'étude du geste. \url{http://gesturestudies.com}},
    long={International Society for Gesture Studies},
    first={\glsentrylong{ISGS} (\glsentryname{ISGS})}
}
\newglossaryentry{ISPW}
{
    name={ISPW},
    description={\textit{\glsentrylong{ISPW}} : L'ISPW, développée en 1989 à l'\gls{IRCAM}, était une plate-forme matérielle \gls{DSP} faisant tourner un serveur de calcul audio temps-réel, pilotable par le logiciel Max},
    long={Ircam Signal Processing Workstation},
    first={\glsentrylong{ISPW} (\glsentryname{ISPW})}
}
\newglossaryentry{JIM}
{
    name={JIM},
    description={\textit{\glsentrylong{JIM}} : conférence annuelle organisée par l'\gls{AFIM} où se réunissent sur plusieurs jours, des chercheurs en Informatique Musicale, des scientifiques et différents acteurs de la vie musicale utilisant l'informatique comme moyen d'expression, comme aide à la composition ou comme outil pédagogique},
    long={Journées d'Informatique Musicale},
    first={\glsentrylong{JIM} (\glsentryname{JIM})}
}
\newglossaryentry{JSON}
{
    name={JSON},
    description={\textit{\glsentrylong{JSON}} : format de données textuelles proposé par Douglas Crockford en 2002, dérivé de la notation des objets du langage JavaScript et permettant de représenter de l’information structurée en paires clé / valeur},
    long={JavaScript Object Notation},
    first={\glsentrylong{JSON} (\glsentryname{JSON})}
}
\newglossaryentry{KarplusStrong}
{
    name={Karplus-Strong},
    description={méthode de synthèse audio-numérique par modèle physique, nommmée d'après ses inventeurs Alexander Strong et Kevin Karplus. Elle modélise un guide d'onde à l'aide d'une ligne à retard filtrée, pour simuler le son d'une corde pincée ou de certains types de percussions et présente l'intérêt d'être très économe en terme de calculs},
    long={Karplus-Strong},
    first={\glsentrylong{KarplusStrong}}
}
\newglossaryentry{LaBRI}
{
    name={LaBRI},
    description={\textit{\glsentrylong{LaBRI}}: unité mixte de recherche du CNRS (UMR 5800) installée sur le domaine universitaire de Talence-Pessac-Gradignan. Rattaché à l'Université Bordeaux I et à l'ENSEIRB, il dépend du département des sciences et technologies de l'information et ingénierie (ST2I). Le \gls{SCRIME} lui est rattaché },
    %first={\glsentrylong{LaBRI} (\glsentryname{LaBRI})}
}
\newglossaryentry{LAM}
{
    name={LAM},
    description={\textit{\glsentrylong{LAM}}: Equipe de recherche rattachée à l'Institut Jean-Le Rond d'Alembert de la faculté des Sciences de Sorbonne Université, et fondé en 1963 par Émile Leipp},
    long={Lutherie, Acoustique, Musique},
    first={\glsentrylong{LAM} (\glsentryname{LAM})}
}
\newglossaryentry{LFO}
{
    name={LFO},
    description={\textit{\glsentrylong{LFO}}: circuit électronique (ou simulé en numérique) permettant la modulation d'un signal à des fréquences inférieures aux fréquences audio, permettant notamment des effets tel que le vibrato},
    long={Low Frequency Oscillator},
    first={\glsentrylong{LFO} (\glsentryname{LFO})},
    firstplural={\glsentrylong{LFO}\glspluralsuffix\ (\glsentryname{LFO}\glspluralsuffix)}
}
\newglossaryentry{LIMSI}
{
    name={LIMSI},
    description={\textit{\glsentrylong{LIMSI}}: unité propre de recherche du Centre national de la recherche scientifique, associé à l'université Paris-Sud et situé à Orsay. Le groupe Audio-Acoustique y a notamment développé des dispositifs de synthèse de voix parlée et chantée controlés de manière expressive en temps-réel},
    long={Laboratoire d'Informatique pour la Mécanique et les Sciences de l'Ingénieur},
    first={\glsentrylong{LIMSI} (\glsentryname{LIMSI})}
}
\newglossaryentry{LLVM}
{
    name={LLVM},
    description={anciennement appelé Low Level Virtual Machine, infrastructure de compilateur conçue pour l'optimisation du code à la compilation et permettant notamment une compilation à la volée (\textit{Just In Time})}
}
\newglossaryentry{LMA}
{
    name={LMA},
    description={\textit{\glsentrylong{LMA}}: unité mixte de recherche sous tutelle Aix-Marseille Université, CNRS et Centrale Marseille. L'équipe ``Son'' s'intéresse notamment aux instruments de musique},
    long={Laboratoire de Mécanique et d'Acoustique de Marseille},
    first={\glsentrylong{LMA} (\glsentryname{LMA})}
}
\newglossaryentry{LogicielLibre}
{
    name={Logiciel Libre},
    description={Logiciel dont le code source est disponible et libre (i.e. non-soumis à un brevet propriétaire), permettant sa redistribution et la création de travaux dérivés},
    long={Logiciel Libre},
    plural={Logiciels Libres}
}
\newglossaryentry{makerspace}
{
    name={makerspace},
    description={atelier de fabrication numérique, ouvert au public et mettant à disposition des machines-outils à commande numérique habituellement réservées à des professionnels dans un but de prototypage rapide ou de production à petite échelle}
}
\newglossaryentry{MAO}
{
    name={MAO},
    description={\textit{\glsentrylong{MAO}}: Ensemble des pratiques utilisant l'informatique à des fins de composition et d'interprétation musicale},
    long={Musique Assistée par Ordinateur},
    first={\glsentrylong{MAO} (\glsentryname{MAO})}
}
\newglossaryentry{mapping}
{
    name={mapping},
    description={Terme utilisé pour désigner la mise en relation de variables d'entrées (e.g. des données issues de capteurs) à des variables de sortie (e.g. des paramètres de synthèse audio). Cf. chapitre \ref{ch:algorithms}},
    long={mapping}
}
\newglossaryentry{MARG}
{
    name={MARG},
    description={\textit{\glsentrylong{MARG}} : acronyme utilisé pour désigner un type de capteur permettant la mesure de l'orientation dans les trois dimensions de l'espace},
    long={Magnetic, Angular Rate, and Gravity},
    first={MARG},
    firstplural={MARG}
}
\newglossaryentry{Max}
{
    name={Max},
    description={\textit{\glsentrylong{Max}}, aussi connu sous le nom Max/MSP/Jitter, est un environnement de programmation multimédia, développé par la société Cycling'74. Développé dans les années 1980 à l'\gls{IRCAM} par Miller Puckette, il est devenu l'un des logiciels les plus utilisés pour la programmation audio, tant dans le domaine de la création artistique, que dans celui de la recherche et de l'industrie},
    long={Max},
    first={Max},
    firstplural={Max}
}
\newglossaryentry{Meta-Instrument}
{
    name={Meta-Instrument},
    description={\textit{\glsentrylong{Meta-Instrument}}: Instrument de musique numérique inventé par Serge de Laubier en 1993},
    long={Méta-Instrument},
    first={\glsentrylong{Meta-Instrument} (\glsentryname{Meta-Instrument})},
    firstplural={\glsentrylong{Meta-Instrument}\glspluralsuffix\ (\glsentryname{Meta-Instrument}\glspluralsuffix)}
}
\newglossaryentry{MIDI}
{
    name={MIDI},
    description={\textit{\glsentrylong{MIDI}}: protocole de communication et format de fichier créés en 1984, dédiés à la musique, et utilisés pour la communication entre instruments électroniques, contrôleurs, séquenceurs, et logiciels de musique},
    long={Musical Instrument Digital Interface},
   	first={\glsentrylong{MIDI} (\glsentryname{MIDI})},
    firstplural={\glsentrylong{MIDI}\glspluralsuffix\ (\glsentryname{MIDI}\glspluralsuffix)}
}
\newglossaryentry{MIM}
{
    name={MIM},
    description={\textit{\glsentrylong{MIM}}: groupe de recherche pluridisciplinaire (compositeurs, plasticiens, vidéastes, scientifiques), fondé en 1984 par Marcel Frémiot et portant sa réflexion sur les pratiques musicales contemporaines},
    long={Laboratoire Musique et Informatique de Marseille},
    first={\glsentrylong{MIM} (\glsentryname{MIM})}
}
\newglossaryentry{MIR}
{
    name={MIR},
    description={\textit{\glsentrylong{MIR}}: science interdisciplinaire, recourant notamment à l'informatique, l'acoustique, la musicologie, le traitement du signal, l'intelligence artificielle, etc. se focalisant sur la recherche (et l'extraction) d'informations musicales, en particulier à l'aide de méthodes automatisés et informatiques. Ce domaine est soutenu par l'International Society for Music Information Retrieval (ISMIR) créé en 2008 organisant la conférence éponyme qui existe depuis l'an 2000},
    long={Music Information Retrieval},
    first={\glsentrylong{MIR} (\glsentryname{MIR})}
}
\newglossaryentry{MIT}
{
    name={MIT},
    description={\textit{\glsentrylong{MIT}}: université américaine situés à Cambridge, Massachusetts, spécialisée dans les domaines de la science et de la technologie. Son \textit{Media Lab} est un lieu important d'innovation dans de multiples domaines de l'interaction humain-machine},
    long={Massachusetts Institute of Technology},
    first={\glsentrylong{MIT} (\glsentryname{MIT})}
}
\newglossaryentry{MMA}
{
    name={MMA},
    description={\textit{\glsentrylong{MMA}}: organisation professionnelle sans profit, composée principalement de développeur de solutions matérielles et logicielles et travaillant sur les standards liés à la norme \gls{MIDI}},
    long={MIDI Manufacturer Association},
    first={\glsentrylong{MMA} (\glsentryname{MMA})},
    firstplural={\glsentrylong{MMA}\glspluralsuffix\ (\glsentryname{MMA}\glspluralsuffix)}
}
\newglossaryentry{MPE}
{
    name={MPE},
    description={\textit{\glsentrylong{MPE}}: méthode d'utilisation du protocole MIDI qui permet d'ajuster de manière continue et polyphonique le pitch bend ainsi que d'autres dimensions de contrôle expressif},
    long={Multidimensional Polyphonic Expression},
   	first={\glsentrylong{MPE} (\glsentryname{MPE})},
    firstplural={\glsentrylong{MPE}\glspluralsuffix\ (\glsentryname{MPE}\glspluralsuffix)}
}
\newglossaryentry{MSP}
{
    name={MSP},
    description={\textit{\glsentrylong{MSP}}: extension \gls{DSP}, basée sur \gls{FTS}, ajoutée à Max en 1997 et permettant le traitement du signal en temps réel directement dans Max},
    long={Max Signal Processing}
}
\newglossaryentry{MUSIC-N}
{
    name={MUSIC-N},
    description={MUSIC-N fait référence à une famille de programmes de musique par ordinateur et de langages de programmation issus ou influencés par le programme MUSIC, écrit par Max Mathews en 1957 aux Bell Labs. MUSIC a été le premier programme informatique à générer des formes d'ondes audionumériques par synthèse directe}
}
\newglossaryentry{NIME}
{
    name={NIME},
    description={\textit{\glsentrylong{NIME}}: Nouvelles Interfaces pour l'Expression Musicale, désignant aussi bien ces interfaces que la conférence éponyme qui leur est consacrée},
    long={New Interfaces for Musical Expression},
    first={\glsentrylong{NIME} (\glsentryname{NIME})},
    firstplural={\glsentrylong{NIME}\glspluralsuffix\ (\glsentryname{NIME}\glspluralsuffix)}
}
\newglossaryentry{OS}
{
    name={OS},
    description={\textit{\glsentrylong{OS}} (système d'exploitation): ensemble de programmes qui dirige l'utilisation des ressources d'un ordinateur par des logiciels applicatifs. Les principaux sont MacOS, Windows, GNU/Linux ainsi qu'iOS et Android sur mobile},
    long={Operating System},
    first={\glsentrylong{OS} (\glsentryname{OS})}
}
\newglossaryentry{OSC}
{
    name={OSC},
    description={\textit{\glsentrylong{OSC}}: format de transmission de données entre ordinateurs, synthétiseurs, robots ou tout autre matériel ou logiciel compatible, conçu pour le contrôle en temps réel. Il utilise le réseau au travers des protocoles \gls{UDP} ou \gls{TCP} et apporte des améliorations en termes de rapidité et flexibilité par rapport à l'ancienne norme MIDI},
    long={Open Sound Control},
    first={\glsentrylong{OSC} (\glsentryname{OSC})},
    firstplural={\glsentrylong{OSC}\glspluralsuffix\ (\glsentryname{OSC}\glspluralsuffix)}
}
\newglossaryentry{OSI}
{
    name={OSI},
    description={\textit{\glsentrylong{OSC}}: norme de communication, en réseau, de tous les systèmes informatiques, proposé en 1978 par l'International Organization for Standardization (ISO)},
    long={Open Systems Interconnection},
    first={\glsentrylong{OSI} (\glsentryname{OSI})}
}
\newglossaryentry{PMMA}
{
    name={PMMA},
    description={\textit{\glsentrylong{PMMA}}: polymère thermoplastique transparent, connu sous le nom commercial ``Plexiglass''},
    long={Polyméthacrylate de méthyle},
    first={\glsentrylong{PMMA} (\glsentryname{PMMA})}
}
\newglossaryentry{RIM}
{
    name={RIM},
    description={\textit{\glsentrylong{RIM}}: (anciennement ``tuteur'', puis ``Assistant Musical''), personne chargée d'aider un compositeur dans la partie technique, en particulier informatique, d'une œuvre musicale comprenant une part d'électronique. Ce statut est historiquement lié à l'\gls{IRCAM} et ses sont contours encore sujet à discussion cf. \cite{zattra_les_2013}. Il existe cependant depuis quelques années des cursus de formation destinés au métier de RIM},
    long={Réalisateur en Informatique Musicale},
    first={\glsentrylong{RIM} (\glsentryname{RIM})},
    firstplural={Réalisateurs en Informatique Musicale (\glsentryname{RIM}\glspluralsuffix)}
}
\newglossaryentry{RTP}
{
    name={RTP},
    description={\textit{\glsentrylong{RTP}}: protocole de communication informatique permettant le transport de données soumises à des contraintes de temps réel, tels que des flux média audio ou vidéo},
    long={Real-Time Transport Protocol},
    first={\glsentrylong{RTP} (\glsentryname{RTP})}
}
\newglossaryentry{rtpMIDI}
{
    name={rtpMIDI},
    description={\textit{\glsentrylong{rtpMIDI}}: (également connu sous le nom `'AppleMIDI') est un protocole destiné au transport de messages MIDI dans des messages \gls{RTP}, sur des réseaux Ethernet et Wi-Fi},
    long={Real-time Protocol MIDI},
    first={\glsentrylong{rtpMIDI} (\glsentryname{rtpMIDI})}
}
\newglossaryentry{SCRIME}
{
    name={SCRIME},
    description={\textit{\glsentrylong{SCRIME}}: Groupement d’Intérêt Scientifique (GIS) et Artistique constitué de l’Université de Bordeaux, du CNRS, de Bordeaux INP, du Ministère de la Culture et de la Communication, de la Ville de Bordeaux et de la Région Aquitaine. Il est administré par le \gls{LaBRI} et a pour objectifs principaux le développement de collaborations entre acteurs scientifiques et artistes à des fins de recherche et de création},
    long={Studio de Création et de Recherche en Informatique et Musiques Expérimentales},
    %first={\glsentrylong{SCRIME} (\glsentryname{SCRIME})}
}
\newglossaryentry{SMC}
{
    name={SMC},
    description={\textit{\glsentrylong{SMC}}: conférence annuelle créée en 2004 à l'initiative conjointe de l'\gls{AFIM} et de l'\gls{AIMI}, consacrée à l'étude interdisciplinaire de la musique et du son par des méthodes computationnelles},
    long={Sound and Music Computing},
    first={\glsentrylong{SMC} (\glsentryname{SMC})},
    firstplural={\glsentrylong{SMC}\glspluralsuffix\ (\glsentryname{SMC}\glspluralsuffix)}
}
\newglossaryentry{SQL}
{
    name={SQL},
    description={\textit{\glsentrylong{SQL}}: langage informatique servant à exploiter des bases de données relationnelles créé en 1974},
    long={Structured Query Language},
    first={\glsentrylong{SQL} (\glsentryname{SQL})}
}
\newglossaryentry{STEIM}
{
    name={STEIM},
    description={\textit{\glsentrylong{STEIM}}: association créée en 1969 à Amsterdam, Pays-Bas, par les compositeurs Misha Mengelberg, Louis Andriessen, Peter Schat, Dick Raaymakers, Jan van Vlijmen, Reinbert de Leeuw, et Konrad Boehmer. Le STEIM est un centre de recherche et de développement sur les nouvelles lutheries. Michel Waisvisz en fût le principal directeur artistique et y inventera le dispositif ``The Hands''},
    long={STudio for Electro Instrumental Music},
    first={\glsentryname{STEIM}}
}
\newglossaryentry{STK}
{
    name={STK},
    description={\textit{\glsentrylong{STK}}: \gls{API} pour la synthèse audio en temps réel, orientée en particulier vers la synthèse par modèle physique. Ecrite en C++ et maintenue depuis 1995 par Perry R. Cook et Gary Scavone, elle contient à la fois des classes de synthèse et de traitement du signal de bas niveau (oscillateurs, filtres, etc.) et des classes d'instruments de niveau supérieur qui contiennent des exemples de la plupart des algorithmes de modélisation physique actuellement disponibles en usage. STK est un \gls{LogicielLibre}, mais un certain nombre de ses classes, en particulier certains algorithmes de modélisation physique, sont couverts par des brevets détenus par l'Université de Stanford et Yamaha},
    long={Synthesis ToolKit},
    first={\glsentrylong{STK} (\glsentryname{STK})}
}

\newglossaryentry{TCP}
{
    name={TCP},
    description={\textit{\glsentrylong{TCP}}: protocole asynchrone de transmission de données sur réseau, utilisant un mode de transmission avec accusé de réception},
    long={Transmission Control Protocol},
    first={\glsentrylong{TCP} (\glsentryname{TCP})}
}
\newglossaryentry{TENOR}
{
    name={TENOR},
    description={\textit{\glsentrylong{TENOR}}: conférence annuelle créée en 2015 à la suite d'un groupe de travail de l'\gls{AFIM} sur la question de la notation musicale},
    long= {Conference International sur les Technologies de la Notation et de la Représentation Musicale},
    first={\glsentrylong{TENOR} (\glsentryname{TENOR})}
}
\newglossaryentry{TUI}
{
    name={TUI},
    description={\textit{Tangible User Interface}: interface utilisateur dans laquelle l'utilisateur interagit avec l'information numérique en agissant directement sur une représentation visuelle de celle-ci. L'objectif de développement des interfaces utilisateur tangibles est d'encourager la collaboration, l'éducation et le design (conception) en donnant à l'information digitale une forme physique, profitant ainsi des capacités humaines de saisir et de manipuler des objets physiques et des matériaux},
    long={interface utilisateur tangible},
    first={\glsentrylong{TUI} (\glsentryname{TUI})}
}
\newglossaryentry{TUIO}
{
    name={TUIO},
    description={contraction de \gls{TUI} et IO (\textit{Input/Output}): protocole de communication conçu pour le développement de la ReacTable, permettant de récupérer les coordonnées d'une multitude de pointeurs, tels que des doigts ou des objets étiquetés},
    first={\glsentryname{TUIO}}
}
\newglossaryentry{UDP}
{
    name={UDP},
    description={\textit{\glsentrylong{UDP}}: protocole asynchrone de transmission de données sur réseau, utilisant un mode de transmission sans accusé de réception},
    long={User Datagram Protocol},
    first={\glsentrylong{UDP} (\glsentryname{UDP})}
}
\newglossaryentry{UPIC}
{
    name={UPIC},
    description={\textit{\glsentrylong{UPIC}}: outil de composition musicale assisté par ordinateur, inventé par le compositeur Iannis Xenakis et développé au \gls{CEMAMu}, à Paris, entre 1975 et 1977. Il se présente comme une grande tablette graphique permettant de dessiner des formes ensuite interprétées comme son. L'axe des abscisses représente le temps et celui des ordonnées les fréquences, mais les échelles de ces axes sont modifiables. Xénakis s'en servira notamment pour composer \textit{Mycènes Alpha}},
    long={Unité Polyagogique Informatique du CEMAMu},
    first={\glsentryname{UPIC}}
}
\newglossaryentry{USB}
{
    name={USB},
    description={\textit{\glsentrylong{USB}}: norme de bus informatique et de connecteurs associés, servant typiquement à relier des périphériques informatiques à un ordinateur},
    long={Universal Serial Bus},
    first={\glsentryname{USB}}
}
\newglossaryentry{UST}
{
    name={UST},
    description={\textit{\glsentrylong{UST}}: éléments d'analyse musicologique proposé par le \gls{MIM} et définis comme des ``fragments sonores qui, même hors de leur contexte musical, possèdent une signification temporelle due à leur organisation''},
    long={Unités Sémiotiques Temporelles},
    first={\glsentrylong{UST} (\glsentryname{UST})}
}
\newglossaryentry{WFS}
{
    name={WFS},
    description={\textit{\glsentrylong{WFS}}: technique de rendu audio spatial basée sur la production de fronts d'ondes artificiels, synthétisés par un grand nombre de haut-parleurs à commande individuelle. De tels fronts d'onde semblent provenir d'un point de départ virtuel, la source virtuelle ou source fictive. Contrairement aux techniques traditionnelles de spatialisation telles que le son stéréo ou surround, la localisation des sources virtuelles dans WFS ne dépend pas de la position de l'auditeur et ne change pas avec elle},
    long={Wave Field Synthesis},
    first={\glsentrylong{WFS} (\glsentryname{WFS})}
}
\newglossaryentry{ZIPI}
{
    name={ZIPI},
    description={\textit{\glsentrylong{ZIPI}}: projet de recherche initié par Zeta Instruments et le \gls{CNMAT} en 1994, conçu comme le protocole de transport de nouvelle génération pour les instruments de musique numériques, en conformité avec le modèle \gls{OSI}},
    long={Zeta Instrument Processor Interface},
    first={\glsentrylong{ZIPI} (\glsentryname{ZIPI})}
}


%%%%%%%%%%%%%%%%%%%%%%%%%%%%%%%%%%%%%%%%%%%%%%%%

% EXEMPLE POUR FAIRE UN INDEX DES NOMS PROPRES
% \newglossaryentry{ppl:Bayle}{
%     type=people,
%     name={François Bayle},
%     description={---}
% }
%%%%%%%%
%
% EXEMPLE DANS LE DOCUMENT :
% Le livre "musique acousmatique" de \gls{ppl:Bayle}
