% !TEX root = ../thesis-example.tex
%
% usage : \gls{IRCAM} 

% \newglossaryentry{API_g}{name={API}, description={An Application Programming Interface (API) is a particular set of rules and specifications that a software program can follow to access and make use of the services and resources provided by another particular software program that implements that API}}

% \newglossaryentry{API}{type=\acronymtype, name={API}, description={Application
% Programming Interface}, first={Application
% Programming Interface (API)\glsadd{API_g}}, see=[Glossaire:]{apig}}


% TODO : enelver les éventuels point à la fin des description, ils sont automatiquement ajoutés par le formatage.

\newglossaryentry{ANR}
{
    name={ANR},
    description={\textit{\glsentrylong{ANR}}: agence de moyens créée en 2005, qui finance la recherche publique et la recherche partenariale en France},
    long={Agence Nationale de la Recherche},
    first={\glsentrylong{ANR}} (\glsentryname{ANR}),
    firstplural={\glsentrylong{ANR}\glspluralsuffix\ (\glsentryname{ANR}\glspluralsuffix)}
}
\newglossaryentry{API}
{
    name={API},
    description={\textit{\glsentrylong{API}}: une interface de programmation d’application est un ensemble normalisé de classes, de méthodes ou de fonctions qui sert de façade par laquelle un logiciel offre des services à d'autres logiciels},
    long={Application Programming Interface},
    first={\glsentrylong{API}} (\glsentryname{API}),
    firstplural={\glsentrylong{API}\glspluralsuffix\ (\glsentryname{API}\glspluralsuffix)}
}
\newglossaryentry{CNMAT}
{
    name={CNMAT},
    description={\textit{\glsentrylong{CNMAT}}: Center for New Music and Audio Technologies (CNMAT) : un centre de recherche multidisciplinaire du Departement de Musique de l'Université de Californie, Berkeley, USA},
    long={Center for New Music and Audio Technologies},
    first={\glsentrylong{CNMAT}} (\glsentryname{CNMAT}),
}
\newglossaryentry{CLI}
{
    name={CLI},
    description={\textit{Command Line Interface} interface homme-machine dans laquelle la communication entre l'utilisateur et l'ordinateur s'effectue en mode texte},
    first={\glsentrylong{CLI}} (\glsentryname{CLI}),
    firstplural={interfaces en ligne de commande(\glsentryname{API}\glspluralsuffix)}
}
\newglossaryentry{CPU}
{
    name={CPU},
    description={\textit{Central processing Unit} (micro)processeur central d'un ordinateur},
    long={Processeur Central},
    first={\glsentrylong{CPU}} (\glsentryname{CPU}),
}
\newglossaryentry{DIY}
{
    name={DIY},
    description={\textit{\glsentrylong{DIY}}: pratiques et mouvement culturel, visant à créer et réparer soi-même des objets souvent techniques, sans nécessairement disposer d'une formation académique pour ce faire. La notion de DIY est apparue au début du XXème siècle en Amérique du Nord à travers un certain nombre de magazines grand public, avant de connaître un essor dans les années 1970 auprès du public étudiant, en écho à la révolution socio-culturelle de cette époque. Un second essor au XXIè siècle est lié à la disponibilité de ressources sur Internet, tels que des tutoriels vidéos et des forums en ligne},
    long={Do It Yourself},
    first={\glsentrylong{DIY}} (\glsentryname{DIY}),
}
\newglossaryentry{DMI}
{
    name={DMI},
    description={\textit{\glsentrylong{DMI}}: instrument de musique numérique},
    long={Instrument de Musique Numérique},
    first={\glsentrylong{DMI}} (\glsentryname{DMI}),
    firstplural={Instruments de Musique Numériques (\glsentryname{DMI}\glspluralsuffix)}
}
\newglossaryentry{DSP}
{
    name={DSP},
    description={\textit{\glsentrylong{DSP}}: Processeur de signal numérique},
    long={Digital Signal Processor},
    first={\glsentrylong{DSP}} (\glsentryname{DSP}),
    firstplural={\glsentrylong{DSP}\glspluralsuffix\ (\glsentryname{DSP}\glspluralsuffix)}
}
\newglossaryentry{e-bow}
{
    name={e-bow},
    description={\textit{\glsentrylong{e-bow}}:(littéralement "archet électronique") accessoire pour guitare inventé par Greg Heet en 1969, et commercialisé par \textit{Heet Sound Products}, fonctionnant comme un excitateur électromagnétique portatif permettant d'obtenir des sons entretenus rappelant ceux produits par un archet sur des cordes},
    long={e-bow}
}
\newglossaryentry{firewire}
{
    name={firewire},
    description={nom commercial donné par Apple à une interface série multiplexée, aussi connue sous la norme IEEE 1394 et également connue sous le nom d'interface i.LINK, nom commercial utilisé par Sony. Il s'agit d'un bus informatique véhiculant à la fois des données et des signaux de commandes des différents appareils qu'il relie},
    long={\textit{firewire}}
}
\newglossaryentry{GDIF}
{
    name={GDIF},
    description={\textit{\glsentrylong{GDIF}}: format développé pour le streaming et l'enresgistrement de données de mouvement en relation à la musique},
    long={Gesture Description Interchange Format},
    first={\glsentrylong{GDIF}} (\glsentryname{GDIF})
}
\newglossaryentry{GRM}
{
    name={GRM},
    description={\textit{\glsentrylong{GRM}}: Centre de recherche musicale dans le domaine du son et des musiques électroacoustiques fondé par Pierre Schaeffer en 1958 à Paris, France},
    long={Groupe de recherches musicales},
    first={\glsentrylong{GRM}} (\glsentryname{GRM})
}
\newglossaryentry{GPU}
{
    name={GPU},
    description={\textit{Graphics Processing Unit} (micro)processeur principal d'une carte graphique, assurant les fonctions de calcul de l'affichage, optimisé pour le calcul d'image 2D et 3D par une structure de traitement parallèle de blocs de données telles que des matrices},
    long={processeur graphique},
    first={\glsentrylong{GPU}} (\glsentryname{GPU}),
}
\newglossaryentry{GUI}
{
    name={GUI},
    description={\textit{Graphical User Interface}: dispositif de dialogue homme-machine, dans lequel les objets à manipuler sont dessinés sous forme de pictogrammes à l'écran, de sorte que l'usager peut utiliser en imitant la manipulation physique de ces objets avec un dispositif de pointage, le plus souvent une souris. Ce type d'interface est apparu dans les années 1970 pour remplacer notamment les interfaces \gls{CLI}, moins facile d'utilisation},
    long={interface graphique},
    first={\glsentrylong{GUI}} (\glsentryname{GUI}),
}
\newglossaryentry{HCI}
{
    name={HCI},
    description={\textit{\glsentrylong{HCI}}: Interface Homme-Machine},
    long={Human-Computer Interface},
    first={\glsentrylong{HCI}} (\glsentryname{HCI}),
    firstplural={\glsentrylong{HCI}\glspluralsuffix\ (\glsentryname{HCI}\glspluralsuffix)}
}
\newglossaryentry{ICLC}
{
    name={ICLC},
    description={\textit{\glsentrylong{ICLC}} : Conférence Internationale sur le Live-Coding, conférence annuelle dont la première édition a eu lieu en 2015},
    long={International Conference on Live-Coding},
    first={\glsentrylong{ICLC}} (\glsentryname{ICLC}),
}
\newglossaryentry{ICMA}
{
    name={ICMA},
    description={\textit{\glsentrylong{ICMA}} : affiliation internationale d'individus et d'institutions impliqués dans les aspects techniques, créatifs et de performance de la musique par ordinateur, dont la mission principale est l'organisation de la conférence \gls{ICMC}},
    long={International Computer Music Association},
    first={\glsentrylong{ICMA}} (\glsentryname{ICMA}),
}
\newglossaryentry{ICMC}
{
    name={ICMC},
    description={\textit{\glsentrylong{ICMC}} : Conférence Internationale sur la Musique par Ordinateur, organisée annuellement par l'\gls{ICMA} et dont la première édition a eu lieu en 1974},
    long={International Computer Music Conference},
    first={\glsentrylong{ICMC}} (\glsentryname{ICMC}),
}
\newglossaryentry{IHM}
{
    name={IHM},
    description={\textit{\glsentrylong{IHM}}},
    long={Interface Homme-Machine},
    first={\glsentrylong{IHM}} (\glsentryname{IHM}),
    firstplural={\glsentrylong{IHM}\glspluralsuffix\ (\glsentryname{IHM}\glspluralsuffix)}
}
\newglossaryentry{IRCAM}
{
    name={IRCAM},
    description={\textit{\glsentrylong{IRCAM}}: Centre français de recherche scientifique, d'innovation technologique et de création musicale, fondé par Pierre Boulez en 1970 à Paris, France},
    long={Institut de Recherche et Coordination Acoustique/Musique},
    first={\glsentrylong{IRCAM}} (\glsentryname{IRCAM}),
    firstplural={\glsentrylong{IRCAM}\glspluralsuffix\ (\glsentryname{IRCAM}\glspluralsuffix)}
}
\newglossaryentry{IReMus}
{
    name={IReMus},
    description={\textit{\glsentrylong{IReMus}}: créé en TODO par la réunion des équipes xxxx TODO},
    long={Institut de Recherche en Musicologie},
    first={\glsentrylong{IReMus}} (\glsentryname{IReMus}),
}
\newglossaryentry{KarplusStrong}
{
    name={Karplus-Strong},
    description={\textit{\glsentrylong{KarplusStrong}}: méthode de synthèse de modélisation physique qui boucle une forme d'onde courte à travers une ligne à retard filtrée pour simuler le son d'une corde martelée ou pincé ou de certains types de percussions},
    long={Karplus-Strong},
    first={\glsentrylong{KarplusStrong}} (\glsentryname{KarplusStrong}),
}
\newglossaryentry{LAM}
{
    name={LAM},
    description={\textit{\glsentrylong{LAM}}: Equipe de recherche rattachée à l'Institut Jean-Le Rond d'Alembert de la faculté des Sciences de Sorbonne Université, et crée TODO en par Emile Leipp},
    long={Lutherie, Acoustique, Musique},
    first={\glsentrylong{LAM}} (\glsentryname{LAM}),
}
\newglossaryentry{LFO}
{
    name={LFO},
    description={\textit{\glsentrylong{LFO}}: circuit électronique (ou simulé en numérique) permettant la modulation d'un signal à des fréquences inférieures aux fréquences audio, permettant notamment des effets tel que le vibrato},
    long={Low Frequency Oscillator},
    first={\glsentrylong{LFO}} (\glsentryname{LFO}),
    firstplural={\glsentrylong{LFO}\glspluralsuffix\ (\glsentryname{LFO}\glspluralsuffix)}
}
\newglossaryentry{LogicielLibre}
{
    name={Logiciel Libre},
    description={Logiciel dont le code source est disponible et libre (i.e. non-soumis à un brevet propriétaire), permettant sa redistribution et la création de travaux dérivés},
    long={Logiciel Libre},
    plural={Logiciels Libres}
}
\newglossaryentry{mapping}
{
    name={mapping},
    description={Terme utilisé pour désigner la mise en relation de variables d'entrées (e.g. des données issues de capteurs) à des variables de sortie (e.g. des paramètres de synthèse audio)},
    long={mapping}
}
\newglossaryentry{Meta-Instrument}
{
    name={Meta-Instrument},
    description={\textit{\glsentrylong{Meta-Instrument}}: Instrument de musique numérique inventé par Serge de Laubier en 1993},
    long={Méta-Instrument},
    first={\glsentrylong{Meta-Instrument}} (\glsentryname{Meta-Instrument}),
    firstplural={\glsentrylong{Meta-Instrument}\glspluralsuffix\ (\glsentryname{Meta-Instrument}\glspluralsuffix)}
}
\newglossaryentry{MIDI}
{
    name={MIDI},
    description={\textit{\glsentrylong{MIDI}}: protocole de communication et format de fichier créés en 1984, dédiés à la musique, et utilisés pour la communication entre instruments électroniques, contrôleurs, séquenceurs, et logiciels de musique},
    long={Musical Instrument Digital Interface},
   	first={\glsentrylong{MIDI}} (\glsentryname{MIDI}),
    firstplural={\glsentrylong{MIDI}\glspluralsuffix\ (\glsentryname{MIDI}\glspluralsuffix)}
}
\newglossaryentry{MMA}
{
    name={MMA},
    description={\textit{\glsentrylong{MMA}}: organisation professionnelle sans profit, composée principalement de développeur de solutions matérielles et logicielles et travaillant sur les standards liés à la norme MIDI},
    long={MIDI Manufacturer Association},
    first={\glsentrylong{MMA}} (\glsentryname{MMA}),
    firstplural={\glsentrylong{MMA}\glspluralsuffix\ (\glsentryname{MMA}\glspluralsuffix)}
}
\newglossaryentry{MPE}
{
    name={MPE},
    description={\textit{\glsentrylong{MPE}}: méthode d'utilisation du protocole MIDI qui permet d'ajuster de manière continue et polyphonique le pitch bend ainsi que d'autres dimensions de contrôle expressif},
    long={Multidimensional Polyphonic Expression},
   	first={\glsentrylong{MPE}} (\glsentryname{MPE}),
    firstplural={\glsentrylong{MPE}\glspluralsuffix\ (\glsentryname{MPE}\glspluralsuffix)}
}
\newglossaryentry{NIME}
{
    name={NIME},
    description={\textit{\glsentrylong{NIME}}: Nouvelles Interfaces pour l'Expression Musicale, désignant aussi bien ces interfaces que la conférence éponyme qui leur est consacrée},
    long={New Interfaces for Musical Expression},
    first={\glsentrylong{NIME}} (\glsentryname{NIME}),
    firstplural={\glsentrylong{NIME}\glspluralsuffix\ (\glsentryname{NIME}\glspluralsuffix)}
}
\newglossaryentry{OSC}
{
    name={OSC},
    description={\textit{\glsentrylong{OSC}}: format de transmission de données entre ordinateurs, synthétiseurs, robots ou tout autre matériel ou logiciel compatible, conçu pour le contrôle en temps réel. Il utilise le réseau au travers des protocoles \gls{UDP} ou \gls{TCP} et apporte des améliorations en termes de rapidité et flexibilité par rapport à l'ancienne norme MIDI},
    long={Open Sound Control},
    first={\glsentrylong{OSC}} (\glsentryname{OSC}),
    firstplural={\glsentrylong{OSC}\glspluralsuffix\ (\glsentryname{OSC}\glspluralsuffix)}
}
\newglossaryentry{RIM}
{
    name={RIM},
    description={Réalisateur en Informatique Musicale: (anciennement \iquote{tuteur}, puis /iquote{Assistant Musical}),personne chargée d'aider un compositeur dans la partie technique, en particulier informatique, d'une œuvre musicale comprenant une part d'électronique. Ce statut est historiquement lié à l'IRCAM et ses contours encore flou cf. \cite{zattra_les_2013}. Il existe cependant depuis quelques années des cursus de formation en tant que RIM},
    long={Réalisateur en Informatique Musicale},
    first={\glsentrylong{RIM}} (\glsentryname{RIM}),
    firstplural={Réalisateurs en Informatique Musicale (\glsentryname{RIM}\glspluralsuffix)}
}
\newglossaryentry{rtpMIDI}
{
    name={rtpMIDI},
    description={\textit{\glsentrylong{rtpMIDI}}: (également connu sous le nom `'AppleMIDI') est un protocole destiné au transport de messages MIDI dans des messages RTP (Real-Time Protocol), sur des réseaux Ethernet et Wi-Fi},
    long={Real-time Protocol MIDI},
    first={\glsentrylong{rtpMIDI}} (\glsentryname{rtpMIDI}),
    firstplural={\glsentrylong{rtpMIDI}\glspluralsuffix\ (\glsentryname{rtpMIDI}\glspluralsuffix)}
}
\newglossaryentry{TCP}
{
    name={TCP},
    description={\textit{\glsentrylong{TCP}}: protocole asynchrone de transmission de données sur réseau, utilisant un mode de transmission avec accusé de réception},
    long={Transmission Control Protocol},
    first={\glsentrylong{TCP}} (\glsentryname{TCP})
}
\newglossaryentry{TUI}
{
    name={TUI},
    description={\textit{Tangible User Interface}: interface utilisateur dans laquelle l'utilisateur interagit avec l'information numérique en agissant directement sur une représentation visuelle de celle-ci. L'objectif de développement des interfaces utilisateur tangibles est d'encourager la collaboration, l'éducation et le design (conception) en donnant à l'information digitale une forme physique, profitant ainsi des capacités humaines de saisir et de manipuler des objets physiques et des matériaux},
    long={interface utilisateur tangible},
    first={\glsentrylong{TUI}} (\glsentryname{TUI})
}
\newglossaryentry{TUIO}
{
    name={TUIO},
    description={contraction de \gls{TUI} et IO (\textit{Input/Output}): protocole de communication conçu pour le développement de la ReacTable, permettant de récupérer les coordonnées d'une multitude de pointeurs, tels que des doigts ou des objets étiquetés},
    first={\glsentryname{TUIO}}
}
\newglossaryentry{UDP}
{
    name={UDP},
    description={\textit{\glsentrylong{UDP}}: protocole asynchrone de transmission de données sur réseau, utilisant un mode de transmission sans accusé de réception},
    long={User Datagram Protocol},
    first={\glsentrylong{UDP}} (\glsentryname{UDP})
}
