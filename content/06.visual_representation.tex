% !TEX root = ../thesis-example.tex
%
\chapter{Représentation visuelle}
\label{ch:visual_representation}

\cleanchapterquote{This is a first quote.}{Some guy}{(some source)}

\cleanchapterquote{This is a 2nd quote.}{Some other guy}{(some other source)}

\section{le cockpit du musicien?}
\label{sec:visual_representation:sec1}

\section{Guides, carte, fretting}

\section{représentation de l'objet/processus virtuel}

\section{intégration de la partition dans l'interface}

\section{aspects esthétiques / instruments visuels}


citations :

The  keyboards  were  always  there...  for  some  reason  or  other  it  looks  good  if  you’re playing a keyboard. People understand then you’re making music.” Robert Moog in Trevor Pinch, “Why You Go to a Piano Store to Buy a Synthesizer: Path Dependence and the Social Construction of Technology,” in Path Dependence and Creation



\begin{quote}
Visual language is one of the oldest forms of knowledge representation and predates conventional written language by almost 25,000 years
\end{quote}
\cite{tufte_visual_2001}

\cite{moody_physics_2009}


\begin{quotation}
If you ask the man on the stree "What's a synthesizer?" He will reply "A synthesizer is a keyboard instrument"... If you go in a retail store ad say I want to see some electronic instruments, they'll send you to the keyboard department, because a synthesizer is a keyboard instrument by default. — Don Buchla, synthesizer pioneer (interview)
\end{quotation}
\cite{pinch_why_2001}