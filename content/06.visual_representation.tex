% !TEX root = ../thesis-example.tex
%
\chapter{Représentations visuelles}
\label{ch:visual_representation}

\cleanchapterquote{Ces petits morceaux d’espace visuels,\\
dont la connexion n’est pas donnée d’avance, \\
par quoi voulez vous qu’ils soient connectés, \\
sinon par la main?}{Gilles Deleuze}{\textit{Qu'est ce que l'acte de création?}\\ conférence donnée à la FEMIS \cite{deleuze_deux_2003}}

voir mp.TUI.key pour les images à inclure

%%%%%%%%%%%%%%%%%%%%%%%%%%%%%%%%%%
\section{Aspects visuels du design d'instrument}

\noindent La conception d'un instrument englobe plusieurs aspects qui affectent son allure visuelle. Je passerai ici en revue quelques-uns de ces aspects, en m'appuyant en partie sur des exemples d'instruments acoustiques, pour souligner certaines continuités avec les développements entrepris ici avec les \glspl{DMI}.

\subsection{Adaptation à la production du son}

\noindent La conception des instruments est notamment orientée par la recherche d'une certaine qualité de son. Bien que cela soit particulièrement évident pour les instruments acoustiques dont les formes ont des conséquences directes sur le rendement sonore (comme le montre la figure \ref{fig:visual_representation:fhole}), les profils particuliers issues de la lutherie traditionnelle ont également donné naissance à un certain nombre d'éléments iconiques (par exemple, les ouïes du violons ou le clavier du piano) et de facteurs de forme (par exemple, une taille plus grande donne un ton grave) associés à l'idée d'un instrument, comme le rappelle Trevor Pinch dans son interview de Robert Moog \cite{pinch_why_2001}: \iquote{Les claviers étaient toujours là, et chaque fois que quelqu'un voulait prendre une photo, pour une raison ou pour une autre, c'est bien si vous jouez du clavier. Les gens comprennent alors tu fais de la musique. (...) Cette pose [prenant la pose, bras gauche tendu tandis que la main droite joue du clavier] associe graphiquement la musique et la technologie.}\footnote{``The keyboards were always there, and whenever someone wanted to take a picture, for some reason or other it looks good if you’re playing a keyboard. People understand that then you’re making music. (...) This pose here [acts out the pose of the left arm extended while the right hand plays a keyboard] graphically ties in the music and the technology.''}.\\
\indent De plus, les \glspl{DMI} peuvent intégrer des transducteurs acoustiques, tels que des microphones piézoélectriques ou des haut-parleurs tactiles (comme nous l'avons vu à la section \ref{sec:interfaces:part_acoustique}), qui influencent la conception acoustique de leurs composants matériels et donc les facteurs de forme évoqués précédemment.

%------------ Figure : f-hole et boehm -----------
\begin{figure}[!htbp]
	\captionsetup{format=plain}%
	\centering
	\begin{minipage}[t]{0.48\textwidth}
		\includegraphics[width=\linewidth]{gfx/06_visual_representation/f-hole.png}
		\caption{évolution de la forme des ouïes du violon d'après \cite{nia_evolution_2015}}
		\label{fig:visual_representation:fhole}
	\end{minipage}
	\hspace{.02\linewidth}
	\begin{minipage}[t]{0.48\textwidth}
	    \includegraphics[width=\linewidth]{gfx/06_visual_representation/Julliot_patent.png}
		\caption{Extrait du brevet de J. Djalma sur \iquote{l'amélioration du clétage des flûtes de Boehm}, 1908.}
		\label{fig:visual_representation:boehm}
	\end{minipage}
\end{figure}


\subsection{Adaptations ergonomiques}

\todo{éliminer les redites du chapitre algorithms}

\noindent L'instrument s'adapte également au corps. Un exemple intéressant est l'évolution du traverso vers la flûte de concert occidentale, à l'aide du système Boehm dans les années 1840 (cf. figure \ref{fig:visual_representation:boehm}). Ce système de clétage découple la topologie gestuelle de la topologie du flux d'air et de la topologie de résonance. En utilisant des manches et des platines, il permet d'acroître la puissance sonore par l'élargissement des trous et leur déplacement à des endroits adéquats pour la résonance, tandis que les touches peuvent être placées à des endroits adaptés à la position des doigts du flûtiste.\\
\indent Le système Boehm peut être qualifié de ``modèle intermédiaire'' entre le geste et la production sonore, fait d'un système mécanique dans ce cas. La plupart des instruments combinent divers ``modèles intermédiaires'' pour amplifier, enrichir, déplacer, focaliser, multiplier les gestes des interprètes et générer des mouvements hors du champ des possibilités du corps humain : pédales de grosse caisse, marteaux et amortisseurs pour piano, archets et plectras, etc.

\subsection{Intégration de la théorie musicale}

%------------ Figure : keyboard et TUI - scale -----------
\begin{figure}[!htbp]
	\captionsetup{format=plain}%
	\centering
	\begin{minipage}[t]{0.48\textwidth}
		\includegraphics[width=\linewidth]{gfx/06_visual_representation/Mersenne_clavier.png}
		\caption{Twenty-seven-steps keyboard invented by Mersenne (1636)}
		\label{fig:visual_representation:MersenneKeyboard}
	\end{minipage}
	\hspace{.02\linewidth}
	\begin{minipage}[t]{0.48\textwidth}
	    \includegraphics[width=\linewidth]{gfx/06_visual_representation/mpTUI_pitchgrid_72dpi.png}
		\caption{Une grille de hauteur avec une représentation micro-tonale réalisée avec mp.TUI. La luminosité des lignes verticales varie en fonction de la quantité de quantification.}
		\label{fig:visual_representation:pitch_grid}
	\end{minipage}
\end{figure}

\noindent Les instruments de musique intègrent également des éléments de théorie musicale. Par exemple, la partie supérieure d'un clavier (touches noires et blanches) représente la gamme chromatique, tandis que la partie inférieure (touches blanches seulement) représente la gamme diatonique de Do majeur. Le dimensionnement et le positionnement de ces touches est un compromis intéressant entre les contraintes mécaniques du système de marteaux et une représentation uniforme des échelles diatonique et chromatique. De plus, la largeur de l'octave est telle qu'elle tient sous une main tendue, ce qui permet de jouer n'importe quel intervalle à l'intérieur d'une octave avec une seule main, réifiant en quelque sorte la notion d'équivalence des octaves. Les claviers ont fait l'objet de nombreux développements expérimentaux avec des dispositions de notes utilisant des grilles hexagonales ou plusieurs couches de touches (figure \ref{fig:visual_representation:MersenneKeyboard}).\\
\indent En tant que système symbolique, la théorie musicale peut être facilement encodée dans les ordinateurs. Les logiciels de production musicale contiennent tellement de fonctions et de règles basées sur la théorie musicale qu'il en est difficile de toutes les représenter sur l'interface. Thor Magnusson parle ainsi ``d'outils épistémiques'' pour décrire les \glspl{DMI}, affirmant qu'il sont conçus avec ``un tel degré de pertinence symbolique qu'ils deviennent un système de connaissance et de pensée dans leurs propres termes'' \cite{todo Magnusson, 2009}. Pour le \textit{musicien numérique}, ce ``système de connaissances'' est un paysage imaginaire à explorer, un territoire sonore pour lequel l'interface de l'instrument peut métaphoriquement prendre le rôle d'une carte géographique, ou d'un cockpit de pilote \cite{vertegaal_towards_1996}.

\subsection{Adaptations au contexte de performance}

\noindent Si l'on considère les instrument de musique comme des ``instruments pour musiquer'' en reprenant la définition de Christopher Small (cf. section \ref{sec:introduction:preamble}), alors les partitions, les salles de concert, le public et plus généralement, le contexte de la performance participent aussi et influencent la conception et la représentatin des instruments. Les partitions orientées (figure \ref{fig:visual_representation:table_music}) sont un exemple d'adaptation de la partition au contexte de la ``musique de table'', permettant dans ce cas aux musiciens de lire la partition lorsqu'ils sont assis autour d'une table. De même, les \glspl{DMI} collectifs (cf. \ref{sec:ephemeral:origins:collectiveDMIs}) peuvent adapter leur représentation au nombre d'interprètes en présentant à chacun d'eux un groupe d'éléments d'interface utilisateur orientés vers eux (figure \ref{fig:visual_representation:multi_orientation}).\\
\indent Comme examples d'influence du lieu de concert sur le design visuel de l'instrument, on peut notamment évoquer la modéliser de son espace acoustique pour venir contrôler la spatialisation du son (cf. figure \ref{fig:visual_representation:spat}) ou --~à l'inverse~-- la projection sur le lieu d'un \textit{mapping vidéo} en correspondance avec la musique\footnote{ou une ``musique visuelle'' comme l'appelle Serge de Laubier} (cf. \ref{fig:visual_representation:pucemuse-monument}).

%------------ Figure : keyboard et TUI - scale -----------
\begin{figure}[!htbp]
	\captionsetup{format=plain}%
	\centering
	\begin{minipage}[t]{0.48\textwidth}
		\includegraphics[width=\linewidth]{gfx/06_visual_representation/Dowland-firstBookOfSonges.png}
		\caption[Partition ``de table'' à plusieurs voix]{Partition ``de table'' à plusieurs voix (John Dowland - First Booke of Songes or Ayres. Édition Peter Short, London, 1597)}
		\label{fig:visual_representation:table_music}
	\end{minipage}
	\hspace{.02\linewidth}
	\begin{minipage}[t]{0.48\textwidth}
	    \includegraphics[width=\linewidth]{gfx/06_visual_representation/mpTUI_multi-orientation.png}
		\caption[Instrument simple adapté pour 6 joueurs]{Instrument simple adaptée pour 6 joueurs située autour d'une interface de jeu commune.}
		\label{fig:visual_representation:multi_orientation}
	\end{minipage}
\end{figure}
%------------ Figure : keyboard et TUI - scale -----------

%------------ Figure : spat et PM -----------
\begin{figure}[!htbp]
	\captionsetup{format=plain}%
	\centering
	\begin{minipage}[t]{0.48\textwidth}
		\includegraphics[width=\linewidth]{gfx/06_visual_representation/IRCAM-spat.jpg}
		\caption[IRCAM Spat Revolution]{L'interface du logiciel ``Spat Revolution'' de l'IRCAM modélise l'espace de projection acoustique du lieu de concert.}
		\label{fig:visual_representation:spat}
	\end{minipage}
	\hspace{.02\linewidth}
	\begin{minipage}[t]{0.48\textwidth}
	    \includegraphics[width=\linewidth]{gfx/06_visual_representation/PuceMuse-Facade.jpg}
		\caption[Projection monumentale, Puce Muse]{Projections monumentales de ``musique visuelle'', controlée de manière synchrone à la ``musique sonore''. Photographie © Puce Muse.}
		\label{fig:visual_representation:pucemuse-monument}
	\end{minipage}
\end{figure}
%------------ Figure : spat et PM -----------

\subsection{Adaptations à l'expérimentation}

\noindent En fin de compte, le processus de conception des instruments de musique contient une grande part de travail empirique. Le processus d'ajustement des réglages d'un instrument nécessitera souvent une rétroaction directe pour les réglages fins faits à la main, jusqu'à ce qu'il sonne et se prête au jeu.

\subsection{Aspects esthétiques : l'instrument œuvre d'art}

\noindent Le design de l'interface visuelle de l'instrument se résume rarement à ses aspects fonctionnels. Les instruments de musique sont souvent l'œuvre d'un artisanat très pointu, voire des objets d'art. La finesse des détails et le souci de leur aspect esthétique jouent ainsi une part éminente dans le design final. Dans les interfaces visuelles des logiciels audio, on retrouve souvent ce soin esthétique, que cela soit sous forme d'un skeuomorphisme\footnote{le terme de skeuomorphisme est utilisé pour définir un élément de design dont la forme n'est pas directement liée à la fonction, mais qui reproduit de manière ornementale un élément qui était nécessaire dans l'objet d'origine.} (figure \ref{fig:visual_representation:skeuomorphisme}) cherchant à imiter le bois des instruments acoustiques, le métal et l'aspect physique des boutons sur les équipements hardware, ou d'un design plus épuré (figure \ref{fig:visual_representation:apparatum}).
%------------ Figure : skeuomorphisme et soin du design -----------
\begin{figure}[!htbp]
	\captionsetup{format=plain}%
	\centering
	\begin{minipage}[t]{0.48\textwidth}
		\includegraphics[width=\linewidth]{gfx/06_visual_representation/Redstair_GEARcompressor.png}
		\caption[Skeuomorphisme dans les logiciels audio]{Le skeuomorphisme dans les logiciels audio témoigne de l'importance accordée à l'esthétique, au delà des fonctionnalités de l'interface.}
		\label{fig:visual_representation:skeuomorphisme}
	\end{minipage}
	\hspace{.02\linewidth}
	\begin{minipage}[t]{0.48\textwidth}
	    \includegraphics[width=\linewidth]{gfx/06_visual_representation/2018_06_26_PAN_GENERATOR_APPARATUM0396.jpg}
		\caption[Apparatum, par panGenerator]{Le design minimal et soigné de l'instrument Apparatum. © panGenerator}
		\label{fig:visual_representation:apparatum}
	\end{minipage}
\end{figure}
%------------ Figure : skeuomorphisme et soin du design -----------

\noindent En tant qu'élément scénographique, la représentation visuelle joue un rôle essentiel dans l'esthétique et la poésie de l'instrument.
Exemple dans FIB\_R : toute la performance visuelle re-projetée sur l'écran correspond à ce que les instrumentistes ont face à eux sur leur tablette. A la fois espace d'interaction et création visuelle esthétique/poétique.



%%%%%%%%%%%%%%%%%%%%%%%%%%%%%%%%%%%%%%%%%
\section{Le visuel comme interface}

\subsection{L'écran, cockpit du musicien?}

\noindent En tant qu'outils épistémiques, les \glspl{DMI} mettent en jeu un nombre extrêmement élevés de flux de données et de processus qui interagissent entre eux et avec l'interface. Pouvoir contrôler le bon fonctionnement de ces processus\footnote{car reconnaissons le, le bon fonctionnement d'un programme informatique n'est jamais garanti}, leurs dynamiques, pouvoir sélectionner des réglages dans une banque de données (presets, samples, etc.) font parties des tâches qui peuvent être facilitées par un contrôle visuel. Les écrans permettent l'affichage d'un grand nombre d'éléments dynamiques qui peuvent également être support d'interaction, que cela soit avec le clavier, la souris, une interface de contrôle \gls{MIDI} ou encore l'écran lui-même quand il est tactile.


Un premier intérêt de l'écran comme interface dynamique est la possibilité d'avoir des informations et/ou des éléments de contrôle contextuel

``For including realtime interactive visualizations and, at the same time, overcoming
mouse limitations without adding indirections, interfaces should be able to reflect their own states and behaviors. They should integrate, like the abacus, both representation and control.''
\cite{jorda_digital_2005}

Interfaces héritées de l'ingénierie sonore avec leur quantité de commandes indépendantes et ordonnées.

Ces éléments virtuels permettent :
\vspace{-1em}
\begin{itemize}[noitemsep]
	\item d'étendre considérablement le champ des opérations possibles, en proposant un espace virtuel beaucoup plus grand, grâce à des systèmes d'onglets, de fenêtres multiples, de menus condensant différentes options;
	\item d'afficher des éléments dynamiques tels que l'état d'avancement d'un séquenceur;
	\item d'adapter l'ergonomie de l'interface à un contexte particulier, e.g. en duplicant des éléments de GUI pour permettre un jeu à plusieurs sur une même interface;
\end{itemize}


%%%%%%%%%%%%%%%%%%%%%%%%%%%%%%%%%%%%%%%%%
\subsection{Guides, carte, fretting : topologie de l'information}

Apprendre indépendamment de la transposition
\iquote{I love the piano sound but not the difficulty of learning the variations from one key to another.  The LinnStrument with its 4th tuning avoids all of those issues.} Jeff Moen about the linnstrument (\url{http://jeffmoen.com/how_i_got_here.html})

Passage du "savoir le contenu" à "savoir que ce contenu existe" et "savoir le chercher".
Savoir cartographique.

%%%%%%%%%%%%%%%%%%%%%%%%%%%%%%%%%%%%%%%%%
\subsection{représentation de l'objet/processus virtuel : chronolgique}
La musique recourt parfois à l'utilisation de processus pour le développement musical. 
\subsubsection{visualisation pour le public}
cf. De Laubier: Bach déploiement de la forme et cadence rendues visibles.

\subsubsection{interaction avec la visualisation}
cf. Goudard: FIB\_R

%%%%%%%%%%%%%%%%%%%%%%%%%%%%%%%%%%%%%%%%%
\subsubsection{intégration de la partition dans l'interface}
Screen scores, patterns de séquenceur, cues de déclenchement, presets



%%%%%%%%%%%%%%%%%%%%%%%%%%%%%%%%%%%%%%%%%
\section{La librairie mp.TUI pour Max}

\subsection{Motivations}

\noindent Il n'existait pas de système ouvert permettant l'interaction polyphonique avec des éléments de GUI intégré à un environnement audio. Parmi les réalisations, on peut noter le développement de :
\vspace{-1em}
\begin{itemize}[noitemsep]
	\item{\textbf{le Lemur}} : une interface \textit{multitouch} personnalisable développé par la société JazzMutant entre 2005 et 2011, d'abord sur une interface hardware dédiée avant d'être distribuée sous la forme d'un logiciel pour tablettes et smartphones\footnote{distribuée par la société Liine \url{https://liine.net/en/products/lemur/}}. Cette interface était pionnière à la fois à une époque om le \textit{multitouch} grand public était naissant, et par la possibilité de personnaliser l'interface à l'aide d'un éditeur tournant sur un ordinateur standard. La personnalisation était cependant limitée au fait de choisir parmi un ensemble de composant alignables sur une grille;

	\item{\textbf{la librairie MMF}} : ``Max Multitouch Framework''\footnote{Vidéo présentant MMF: \url{https://www.youtube.com/watch?v=EEkj85GU_is}} développé par Mathieu Chamagne dans le cadre du projet \gls{ANR} Virage (2008-2010), qui permettait de contrôler certains éléments de GUI de Max.

	\item{\textbf{TouchOSC}} : proposé en 2008\footnote{\url{https://hexler.net/products/touchosc}}, touchOSC est basé sur les principes du Lemur mais développé comme une application pour tablette et smartPhones, permettant de renvoyer des données \gls{MIDI} ou \gls{OSC}.

	\item{\textbf{Mira}} : En 2013, Sam Tarakajian développe Mira\footnote{dont la vidéo de présentaion est intéressante à plus d'un titre \url{https://vimeo.com/63846055}}, qui sera distribué ensuite par Cycling'74, la société développant Max. Mira simplifie grandement la connection entre un iPad et un patch Max dans la mesure où les éléments \gls{GUI} d'un patch Max sont directement transposés sur l'interface \textit{multitouch}, sans qu'il y ait besoin de recréer des connections manuellement entre deux interfaces différentes, comme cela pouvait être le cas sur Lemur ou TouchOSC.

	\item{\textbf{Max multitouch} par ailleurs, Cycling'74 a considérablement améliorer la mécanique de patching depuis la version 8, et a commencé fin 2018 l'implémentation native du \textit{multitouch} dans l'éditeur de patch\footnote{pour l'instant limitée à un certain nombre d'objet et à la version MS Windows.}}
\end{itemize}

\noindent \par{\textbf{de la nécessité de réinventer la roue}} A la vue de ces développements, on peut légitimement se demander quel est l'intérêt de développer une librairie graphique pour le \textit{multitouch} dans Max.
La réponse est essentiellement liée à la conviction que les interfaces graphiques, tout comme le domaine du mapping, ne sont pas des simples ustensiles fonctionnels mais un champ ouvert à la créativité pour développer de nouvelles relations.\\
\indent Si l'on prend l'exemple d'un objet aussi simple et courant qu'un \textit{slider}, on s'aperçoit rapidement que de multiples scénarios d'interactions sont possibles, et que les implémentations de cette interaction dans la plupart des logiciels de bureautique ne sont pas forcément les plus adaptées au contrôle musicale. Ainsi, un musicien pourra apprécier jouer à plusieurs doigts sur une telle interface linéaire pour y contrôler des intervalles mélodiques, usage pour lequel la mémoire de ces intervalles s'avère très ergonomique. Il suffit de tester le fonctionnement des sliders sur les tablettes dans la plupart des applications pour constater que ce genre d'interaction n'y est, sans surprise, pas prévu.

Ainsi, la prise en charge native du \textit{multitouch} dans l'interface de Max ne saurait être satisfaisante en terme créatif si elle ne laissait pas les relations entre les objets graphiques et le geste ouvertes à la reprogrammation. 


ré-inventer la route (ou le slider) : les interactions prévues dans les éléments de GUI de la bureautique ne sont pas conçues pour l'interaction musicale. Adaptation ad-hoc du comportement de GUI. Exemple d'un slider avec mémoire du dernier doigt. 

\todo{inclure les figures de la présentation mp.TUI.key}

\subsection{Utilisation de MP pour le contrôle \textit{multitouch} de GUI}

\noindent La bibliothèque mp.TUI\footnote{Sources disponible sur \url{https://github.com/LAM-IJLRA/ModularPolyphony-TUI/}.} est construite sur le protocole MP (cf. \ref{sec:algorithms:MP}). Elle fournit un cadre basé sur les logiques de patch de Max pour créer de nouveaux composants d'interface utilisateur \textit{multitouch} dans un contexte OpenGL et surmonter certaines limitations de l'interface graphique native de l'environnement de patching de Max. Par exemple, les interfaces graphiques sont généralement orientées sur une disposition horizontale/verticale avec une orientation de lecture du haut vers le bas alors qu'on peut souhaiter avoir plusieurs orientations, comme dans la situation présentée sur la figure \ref{fig:visual_representation:multi_orientation}. La superposition de divers composants peut nécessiter des couleurs et des transparents personnalisés, et l'on peut souhaiter inclure des interfaces visuelles plus complexes que les curseurs et les boutons, par exemple des particules, des vidéos, des modèles 3d, des shaders (cf. figure \ref{fig:visual_representation:phonetogramme}), etc.

\noindent Les composants de la bibliothèque sont d'un ensemble d'abstractions de trois types :
\vspace{-1em}
\begin{itemize}[noitemsep]
	\item \textbf{des composants système}, qui implémentent les fonctions essentielles pour envelopper les graphiques dans un élément cliquable. Cela inclut le "mp.TUI.hub" qui récupère les données de l'interaction de la souris sur la fenêtre OpenGL ainsi que les messages TUIO reçus par UDP et les envoie aux composants graphiques sélectionnés.;
	\item \textbf{des éléments de GUI}, qui sont des instances prêtes à l'emploi de widgets courants ou moins courants tels que curseurs, claviers, graphes, etc.;
	\item \textbf{des outils}, un ensemble d'abstractions qui permettent de créer facilement de nouveaux composants en proposant des fonctions utiles pour la conception d'interaction (transformation de la visualisation, gestion de la polyphonie sur un élément, interaction tels que pinch-zoom, calcul de dérivées, etc.).
\end{itemize}

\subsection{Les composants de la librairie mp.TUI}

\noindent Les composants utilisent des transformations géométriques hiérarchiques\footnote{à l'aide de l'objet jit.anim.node de Max}, qui permet d'obtenir des coordonnées relatives au monde ou à l'objet indépendamment de la position, de l'échelle et de l'orientation du composant de l'interface utilisateur. Cela permet également de créer des groupes de composants, comme on le ferait dans n'importe quel logiciel de CAO. Suivant la nature empirique de la lutherie numérique revendiquée ci-dessus, un "mode édition" est également disponible pour manipuler rapidement à la main la position, l'échelle et l'orientation des composants de l'interface utilisateur (figure \ref{fig:visual_representation:groups_patch} et \ref{fig:visual_representation:groups}).

%-------------------------- Figure : phonetogramme ----------------------------------
\begin{figure}[!htbp]
	\includegraphics[width=\textwidth]{gfx/06_visual_representation/Phonetogramme.png}
	\caption{``Le phonetogramme'', une application muséographique conçue pour la Cité des Sciences, dont la GUI est réalisée avec la librairie mp.TUI.}
	\label{fig:visual_representation:phonetogramme}
\end{figure}

\noindent La possibilité de concevoir des objets audiovisuels en Max en étroite relation avec la programmation de l'interaction entre le geste, l'audio et le visuel permet de les intégrer dans des scénarios dynamiques personnalisés : histoires narratives pour des ateliers éducatifs avec des enfants, scénarios réactifs, visualisations personnalisées pour les malvoyants, expositions muséographiques avec chartes graphiques spécifiques, adaptation réactive aux formats d'écran, graphismes expérimentaux pour l'esthétique des performances artistiques live, etc.


%------------------ Figure : mp.TUI : simple slider ---------------------
\begin{figure}[!htbp]
	\makebox[\linewidth][c]{%
		\begin{subfigure}[b]{.5\textwidth}
			\centering
			\includegraphics[width=.95\textwidth]{gfx/06_visual_representation/mpTUI_slider-patcher.png}
			\caption{L'objet Max créant un slider}
		\end{subfigure}%
		\begin{subfigure}[b]{.5\textwidth}
			\centering
			\includegraphics[width=.95\textwidth]{gfx/06_visual_representation/mpTUI_slider-onscreen.png}
			\caption{Rendu du slider dans une fenêtre OpenGL}
		\end{subfigure}%
	}
	\caption{Un simple slider dans la librairie mp.TUI}
\end{figure}


\subsection{Outils pour l'interaction multitouch}


%-------------------------- Figure : mp.TUI overview ------------------------
\begin{figure}[!htbp]
	\includegraphics[width=\textwidth]{gfx/mpTUI/mp-TUI-preview.png}
	\caption{Aperçu de quelques composants graphiques de la librairie mp.TUI}
	\label{fig:visual_representation:mp.TUI}
\end{figure}

% \begin{figure}
% 	\captionsetup{format=plain}%
% 	\centering
% 	\begin{minipage}[t]{0.48\textwidth}
% 		\includegraphics[width=\linewidth]{gfx/mpTUI/mp-TUI-preview.png}
% 		\captionof{figure}{Exemples de composants}	
% 		\label{fig:visual_representation:overview}
% 	\end{minipage}%
% 	\hspace{.02\linewidth}	
% 	\begin{minipage}[t]{0.48\textwidth}
% 	    \includegraphics[width=\linewidth]{gfx/mpTUI/mp-TUI-voronoi.png}
% 		\caption{Modèle de Voronoi}
% 		\label{fig:visual_representation:voronoi}
% 	\end{minipage}
% \end{figure}


\subsection{Groupement d'objets graphique}

\noindent L'objet mp.TUI.groups permet de rattacher différents éléments de GUI à un groupe, comme on le fait dans la plupart des logiciels d'édition vectorielle, afin de gérer leur position, échelle et orientation de manière globale. Cela permet de définir un ensemble de composant dans des coordonnées relative, puis de venir ajuster l'emplacement d'un seul et unique bloc. Il est possible de procéder à l'ajustement de ces coordonnées par de valeurs envoyées explicitement, ou bien par une manipulation directe du groupe, selon le même mode opératoire que pour les objets usuels.\\
\indent Un exemple est présenté sur la figure \ref{fig:visual_representation:groups_patch} et son rendu graphique figure \ref{fig:visual_representation:groups}, associant des curseurs et un texte affichant leurs coordonnées dans une zone précise définie par le canvas.

%-------------------------- Figure : groups ----------------------
\begin{figure}[!htbp]
	\captionsetup{format=plain}%
	\centering
	\begin{minipage}[t]{0.48\textwidth}
		\includegraphics[width=\linewidth]{gfx/06_visual_representation/mpTUI_groups_patcher.png}
		\caption{Patch Max présentant des objets groupés}
		\label{fig:visual_representation:groups_patch}
	\end{minipage}
	\hspace{.02\linewidth}
	\begin{minipage}[t]{0.48\textwidth}
	    \includegraphics[width=\linewidth]{gfx/06_visual_representation/mpTUI_groups.png}
		\caption{Groupement d'objets et édition ``à la main''}
		\label{fig:visual_representation:groups}
	\end{minipage}
\end{figure}
%-------------------------- Figure : groups ----------------------

\subsection{GUI composites}

\noindent Il peut s'avérer nécessaire de coordonner plusieurs éléments d'interaction graphique dans un seul et même ensemble, afin qu'un élément puisse réagir à une interaction sur un autre élément. L'objet mp.TUI.canvas est destiné à cette fin. C'est un objet graphique vide, qui définit simplement une zone ajustable en position, échelle et orientation. Comme tous les autres objets de la librairie mp.TUI, il emet en sortie les \textit{MP-events} qui lui arrivent, ce qui permet de les renvoyer sur les entrées MP d'autres composants, même si ceux-ci ne sont pas directement touchés par les curseurs de position.\\
\indent Un exemple simple de \gls{GUI} composite est présenté sur la figure \ref{fig:visual_representation:canvas}, associant des curseurs et un texte affichant leurs coordonnées dans une zone précise définie par le canvas.


%------------------ Figure : canvas ---------------------
\begin{figure}[!htbp]
	\includegraphics[width=\textwidth]{gfx/06_visual_representation/mpTUI_canvas.pdf}
	\caption[Exemple de GUI composite avec mp.TUI.canvas]{Exemple de GUI composite avec mp.TUI.canvas: les curseurs ne sont pris en compte que dans la zone définie par le canvas. Patch Max en haut, rendu en bas}
	\label{fig:visual_representation:canvas}
\end{figure}
%------------------ Figure : canvas ---------------------

%------------------ Figure : canvas ---------------------
% \begin{figure}[!htbp]
% 	\captionsetup{format=plain}%
% 	\centering
% 	\begin{minipage}[t]{0.38\textwidth}
% 		\includegraphics[width=\linewidth]{gfx/06_visual_representation/mpTUI_composition-canvas.png}
% 		\caption[Exemple de GUI composite avec mp.TUI.canvas]{Exemple de GUI composite avec mp.TUI.canvas}
% 		\label{fig:visual_representation:canvas-patch}
% 	\end{minipage}
% 	\hspace{.01\linewidth}
% 	\begin{minipage}[t]{0.58\textwidth}
% 	  	\includegraphics[width=\linewidth]{gfx/06_visual_representation/mpTUI_canvas_window.png}
% 		\caption[Exemple de GUI composite : rendu graphique]{Exemple de GUI composite : rendu graphique}
% 		\label{fig:visual_representation:canvas-window}
% 	\end{minipage}
% \end{figure}
%------------------ Figure : canvas ---------------------


\subsection{Instanciations dynamiques}

Utilisation de la librairie MP pour créer dynamiquement des instances éphémères de composant UI.
Exemple avec mp.TUI.canvas ou l'on vient ``piocher'' un slider créé via un objet mp.TUID.slider.
Considération pour la gestion de la destruction de ces instances.


\subsection{Performances}

\noindent La bibliothèque mp.TUI est entièrement développée avec des objets natifs de la distribution Max. Cette approche, bien que plus coûteuse en charge \gls{CPU} que des objets compilés, a l'avantage de permettre à tout utilisateur de Max de modifier facilement les composants et de les adapter à ses besoins. De plus, les composants mp.TUI s'appuient essentiellement sur OpenGL, de sorte que la majeure partie de la charge de calcul est laissée au \gls{GPU}. L'interaction tangible avec les objets de la \gls{GUI} se fait à l'aide du moteur Bullet-Physics\footnote{\url{http://bulletphysics.org/}} intégré dans Max. Bien que cela puisse être plus coûteux pour certaines formes simples, cela nous permet de concevoir des composants \gls{GUI} de n'importe quelle forme et orientation, comme des \textit{sliders} courbes ou des formes creuses, et de les animer potentiellement, comme dans l'exemple des "balles rebondissantes" où plusieurs curseurs 2D peuvent être déplacés et lancés dans une zone délimitées.

%%%%%%%%%%%%%%%%%%%%%%%%%%%%%%%%%%%%%%%%%
\section*{miscellanées (temporaire à supprimer)}
citations :

The  keyboards  were  always  there...  for  some  reason  or  other  it  looks  good  if  you’re playing a keyboard. People understand then you’re making music.” Robert Moog in Trevor Pinch, “Why You Go to a Piano Store to Buy a Synthesizer: Path Dependence and the Social Construction of Technology,” in Path Dependence and Creation


NIME 2019 : ``From Mondrian to Modular Synth: Rendering NIME using Generative Adversarial Networks''

Chamagne

\begin{quote}
Visual language is one of the oldest forms of knowledge representation and predates conventional written language by almost 25,000 years
\end{quote}
\cite{tufte_visual_2001}

\cite{moody_physics_2009}


\begin{quotation}
If you ask the man on the stree "What's a synthesizer?" He will reply "A synthesizer is a keyboard instrument"... If you go in a retail store ad say I want to see some electronic instruments, they'll send you to the keyboard department, because a synthesizer is a keyboard instrument by default. — Don Buchla, synthesizer pioneer (interview)
\end{quotation}
\cite{pinch_why_2001}

Roel Vertegaal, Tamas Ungvary et Michael Kieslinger utilisait ainsi le terme de ``\textit{musician's cockpit}'' dans un article de 1996, une métaphore qui laisse imaginer l'instrument comme un véhicule que l'on pilote.