\chapter{Interview : Lucas Turchet}

\section*{Biographie}

\section*{Transcript}

LT — To answer to your question basically I...  what is the motivation for a person to to build these instruments that are so unique and peculiar and why play them, right? and simply because I find I had this motivation because I wanted to explore something new it was a need for me to express myself in other ways by using an interface  that I knew already a lot but I was not satisfied only with the possibilities of theacoustic sound itself and I was not satisfied with the possibilities of using external equipment such as a conventional foot pedal, stompboxes thatare used to to affect the sound process the sound and I also wanted to addfurther sounds not contemplated in the typical set of commercially availableproducts and I think you can't play something typically a trigger drummachine or a sequencer or a synthesizer andabove all control it in real time fromthe instrument in your hand and not with yourfoot or pressing a button on the computerso these things contributed to me, to my visionand my needs were such thatsuch that... so great that I really wanted tobuild something that was unique customizedfor me, for my hands, for my playing styleso in short this was just a research toprimarily respond to a need that I hadas a composer and as a musicianand also not only this was the primary needmotivation but the second motivationalso was thatI wanted to give this instrument oneday in the hands of someone elseso composers could avail themselves of this instruments, this new interfaces and use I mean to composeI mean I didn't want to be the only oneto use this instrument and I wanted ...in my ideals this instrument should be used byso this would be a more complete goal I would say

VG — so that it extends the availibility of works for this instrument ?

LT — Exactly a repertoire including moreperformances so when we build something newyou would also to give a little bit of dignity and I mean I'm happy that that I play itbut it would benice not to be the only oneit would be nice for instance to have an orchestra of mandolin, or mandolin an cellosmind sciences and technology and dosomething together...why not?

VG — and see what other people do with the same instrument

LT — yeah exactly, see also how people are reactingfor instance with the same interfacesame technology, with the different ideas andsure that the ideas that they have explored arenot the only ones that are availablethat are possible certainly when yougive an instrument to someone elsehe will have his own mental models, his own needs,expression needs and so on that willbring him or her to discover another set of gesturesprogram certain types of algorithms and so on that will lead to different resultsthan those than I have achieved

VG — When you say that you wanted to explore something new, new ways to expresswere you thinking more of sound, or gesture, or a little bit of both ?

LT — the two things go hand in hand,but I think that the very beginning was mostlya matter of control. So okay I might behappy with the delay, but I would like tohave the control at the note level,meaning that I play, I want amusical sentence and in the, I don't know, twenty notes there are in musical sentencesonly on three of them I want toapply a the delay effect.With the common foot pedals you can't achieve this level of detail, of note level control.So the thing was that I could play, I press a button, a button or a sensor,in the exact moment which this noteis going to happen and I release itwhen this note is elapsed but theeffect is applied, the next note to comehas no effects and this will avoidme to do strange tip-tap movementsthat can't be done of coursein front of an audience.

VG — So the design that you made was... did you make a study of ... as you play the mandolin I guess you knowwhat kind of movements do exist 

LT — exactly

VG — but the location where you put the buttons and sensorsyou chose them according to existing mandolin gesturesor on the opposite, not to interfere too much ?because it can be a drawback that if you do a gesture witha conventional instrument youwould trigger things without willing

LT — Absolutely. The second finger can be also exploited as a compositionalparameter by the way, but of course ingeneral no you want to add thecontroller and applying effect when youwant okay sofirst of all, my investigation has started by a very simple consideration that issome mandolin players in particular myself when we play, we pose the little fingeron the part on the bottom of the strings tomake stability on the wrist in some occasions okayso it was already easy topress something to activate somethingand it was with no effort. So from thatis easy to extend this concept toother pressure sensors than that are nearby and of course with a set ofergonomical studies to understand where the sensors were better placedand a lot of researching was doneto discover this, precisely thisand they found this configuration then in the very end is also applied in the startup companyin the smart guitar that is produced by my startup company

VG — Another thing that I am interested in is the fact that in digital musical instrumentsunder any sensor,you can map any sensor to any controlI don't know if you have various configurations, various presets...can you change or is it fixed ?

LT — no it's not fixed at all, and moreoverthere are also even different layoutsphysical layouts of the sensors. I havedevelopped different layouts but in thevery end, I typically used mostly one I feelmore comfortable or that the responseis most much better to some needsfor a set of pieces that are composed so far.It is not fixed though and theproblem is that when you change theassignment of a parameter, or an effector a drum-machine,sampler or whatever synthesizerto a particular sensor and in the next pieceyou change or in the next part even ofthe same piece you change this mappingplacing, I don't know, a phaser rateI don't know, that parameter instead of the pitch of a whammy pitch-shifterof course you have to re-learn theinstrument because the instrument changes.It's beautiful because you have a power in that moment, at the cost of one click,that your instrument will changetotally timbre, and we switch between itvery very quickly, but in the other handyou have to re-learn the instrumentand sometimes it takes a lot of practice also,especially if you do a concertwith different pieces that arerather different between each otherbecause typically a musician want to vary.So you have to remember, you have to studyit is not an invented instrument, but it is an instrument that has "software" (?)something is not the (...)the same problems of any instrument,so you have to study, you change piece,you have to practice, study, rehearse,learn and be perfect... (laughs)VG — Well, digital musical instruments are morequick to change that's maybe...once you have this kind of stable behaviorof an acoustic instrumentand the constant evolutingbehaviour on the digital musical instrumentBut then what you say is that you would choose to rely only on memory for the kind ofbehavior modification. I mean you could havefor example screens telling you "this is preset 2"or whatever, or do not have presets...

LT — no, no, I do. I have a setting interface,I have two interfacesone is for the expressive controlthe other is for the settingsand they have were six buttons, they are capacitive sensors with LEDs which allowsme to understand in which bank I amand I can navigate between the bankswith up and down and there are fourpresets within each bank

VG — do you have like a visual feedback?

LT — no because at the momentI have only LEDs that give you this,but I don't have a small displaythis is something if they want to have

VG — but those LEDs are of visual feedback giving youinformation about which bank you're you said?

LT — no I don't have it on the mandolinI have and I don't havebecause I have a two lightstwo buttons with its own LEDsI know that if I press one time I will havea LED that will blink if I press two times the LED will not blinkand I would be in the secondand I typically use two banks, four banksin general : two navigating upwards andtwo navigating backwards, in total fourbut the fact is also that I amhaving a computer near meso the visual feedback is there I mean I can always check it in which piece I amwhere I am because I have a second screen over thaton the table near me

VG — but you'd preferably skip this screenand have an embedded solution somehow ?

LT — this is precisely the reason because I am building these ...(interruption)this is precisely the reason because I am... I have developed this novelfamily of instruments that are called"smart instruments".I don't know if you are familiar with them?

VG — a little bit

LT — ok the first examplar is the "smart guitar" by mind music labsthis company that I mentioned before and that I co-foundedso just in a nutshell what are smart instrumentsthey are instruments that have anembedded computational unitso they have intelligence and thisintelligence is responsible primarilyfor the sensors processing so it is an augmented instrument with sensors andactuators or loudspeakers, these in many casesplaced where the sound source isdirectly in the instrument itselfand they have the featureof having a multi-directionalwireless connectivityin this context if you have an embedded computational unit that powerfulit is easy to connect a touch display withvisual information and more importantlyyou do not need an externalcomputational unit such as a laptoptraditionally, typically the mosttraditional setup for augmented instrumentis where there is the sensors augmentationplace on the instrument itselfbut then the computational unit is placed outside

VG — so you mean this connectivity is also a way I guess to updatethe instrument bank... 

LT — for instancethis is a precisely the new line of researchand I'm going to present a poster now(at audio mostly conference  2017)out the manifold interactionpossibilities that are enabled by aninstrument that has a sound engineinside, a system to deliver electronicallygenerated sounds placed on theinstrument itselfthis connectivity feature that allows people to jointogether with this instrument forinstance and people for instance with asmartphone or a tablet can or even on awireless keyboard can connect to theinstrument play together with theinstrument players that is playing the guitaror whatever other smart instrumentand all the sounds are mixed and generated by the guitar itselfso, this is a simple application

VG — maybe we're running out of time butmaybe one last questionI usually ask people ...a pronostic on what's important tothem, what they feel is important to themin the field of digital musicalinstrument for the next ten yearsand how do they think this could changethe way we make music in general

LT — oh this is a one million dollar questionin the sense that ... mmm ...musicians in general arethe most conservative peopleof all categories of human beings okay

LT — you think so ?

LT — for instance guitarists want the the sound ofthe Stratocaster of the fifty trees and...and that's it. They are not open to anything else for instancemost a musician nowadays are like thatthen there are the experimental musicianswithin those we should do ananalysis but I I do believe however thatmany possibilities of producing soundsand spatialization and augmentedinstruments are somehow already beingdiscovered there is a lot of research todo I agree but the core concept is notgroundbreaking any more on that sidein my humble opinion. What is the future inmy opinion is the possibility ofconnecting things together okay so it'snot by chance that we are living in thisInternet of Things world now. This domain is more and moreimportant in any level and also music isdefinitely affecting the by thesetools, technologies but also behaviors peoplewant to be connected anytime everywhereand with anyone andmusicians are people and they, in my opinionthey will feel more and more this need. okay?yeah, that sounds goodhope that I've answered to your questions

VG — yes, sure, thanks you.