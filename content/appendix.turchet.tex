\chapter{Interview : Lucas Turchet}
\label{appendix:turchet}

\section*{Biographie}
\noindent Né à Vérone, Italie en 1982, Lucas Turchet obtenu une maîtrise en informatique de l'Université de Vérone en 2006, une maîtrise en guitare classique et composition du Conservatoire de musique de Vérone, en 2007 et 2009, un doctorat en technologie des médias de l'Université Aalborg, Copenhague, Danemark, en 2013, et une maîtrise en musique électronique du Royal College of Music de Stockholm, Suède, en 2015. Il est actuellement boursier Marie-Curie au Centre of Digital Music, Queen Mary University of London, Londres, Royaume-Uni. Il est également professeur adjoint au Département de génie de l'information et d'informatique de l'Université de Trente, Italie. Ses recherches scientifiques, artistiques et entrepreneuriales ont été soutenues par de nombreuses subventions de différents organismes de financement, dont la Commission européenne, le ministre italien des Affaires étrangères et le Conseil danois de la recherche. Ses principaux centres d'intérêt de recherche sont la technologie musicale, l'interaction homme-machine et la perception multimodale. Il fait partie des fondateurs de Mind Music Labs, une entreprise basée à Stockholm fabriquant des instruments augmentés.

\noindent Site web : \url{http://www.lucaturchet.it}

\section*{Transcript}

\noindent Lucas Turchet, interview du 23/08/2017, à Queen Mary University, en marge de la conférence AudioMostly'17, Londres.

LT — To answer to your question basically I... what is the motivation for a person to to build these instruments that are so unique and peculiar and why play them, right? and simply because I find I had this motivation because I wanted to explore something new it was a need for me to express myself in other ways by using an interface that I knew already a lot but I was not satisfied only with the possibilities of the acoustic sound itself and I was not satisfied with the possibilities of using external equipment such as a conventional foot pedal, stompboxes that are used to to affect the sound process the sound and I also wanted to add further sounds not contemplated in the typical set of commercially available products and I think you can't play something typically a trigger drummachine or a sequencer or a synthesizer and above all control it in real time fromthe instrument in your hand and not with your foot or pressing a button on the computer so these things contributed to me, to my vision and my needs were such that such that... so great that I really wanted tobuild something that was unique customized for me, for my hands, for my playing style so in short this was just a research to primarily respond to a need that I had as a composer and as a musician and also not only this was the primary need motivation but the second motivation also was that I wanted to give this instrument one day in the hands of someone elseso composers could avail themselves of this instruments, this new interfaces and use I mean to composeI mean I didn't want to be the only one to use this instrument and I wanted ...in my ideals this instrument should be used by so this would be a more complete goal I would say

VG — so that it extends the availibility of works for this instrument ?

LT — Exactly a repertoire including more performances so when we build something new you would also to give a little bit of dignity and I mean I'm happy that I play it but it would be nice not to be the only one it would be nice for instance to have an orchestra of mandolin, or mandolin an cellos, mind sciences and technology and do something together... why not?

VG — and see what other people do with the same instrument

LT — yeah exactly, see also how people are reacting for instance with the same interface same technology, with the different ideas and sure that the ideas that they have explored arenot the only ones that are available that are possible certainly when you give an instrument to someone else he will have his own mental models, his own needs,expression needs and so on that will bring him or her to discover another set of gestures program certain types of algorithms and so on that will lead to different results than those than I have achieved

VG — When you say that you wanted to explore something new, new ways to expresswere you thinking more of sound, or gesture, or a little bit of both ?

LT — the two things go hand in hand, but I think that the very beginning was mostly a matter of control. So okay I might be happy with the delay, but I would like tohave the control at the note level, meaning that I play, I want a musical sentence and in the, I don't know, twenty notes there are in musical sentences only on three of them I want to apply a the delay effect. With the common foot pedals you can't achieve this level of detail, of note level control. So the thing was that I could play, I press a button, a button or a sensor,in the exact moment which this noteis going to happen and I release it when this note is elapsed but the effect is applied, the next note to comehas no effects and this will avoid me to do strange tip-tap movements that can't be done of course in front of an audience.

VG — So the design that you made was... did you make a study of ... as you play the mandolin I guess you knowwhat kind of movements do exist 

LT — exactly

VG — but the location where you put the buttons and sensors you chose them according to existing mandolin gestures or on the opposite, not to interfere too much ? because it can be a drawback that if you do a gesture witha conventional instrument you would trigger things without willing to do so

LT — Absolutely. The second finger can be also exploited as a compositional parameter by the way, but of course in general no you want to add the controller and applying effect when you want... okay so first of all, my investigation has started by a very simple consideration that is some mandolin players in particular myself when we play, we pose the little finger on the part on the bottom of the strings to make stability on the wrist in some occasions okay ... so it was already easy to press something to activate something and it was with no effort. So from that is easy to extend this concept too the pressure sensors that are nearby and of course with a set of ergonomical studies to understand where the sensors were better placed and a lot of researching was done to discover this, precisely this and they found this configuration then in the very end is also applied in the startup company, in the smart guitar that is produced by my startup company

VG — Another thing that I am interested in is the fact that in digital musical instruments under any sensor, you can map any sensor to any control. I don't know if you have various configurations, various presets... can you change or is it fixed ?

LT — no it's not fixed at all, and moreover there are also even different layouts, physical layouts of the sensors. I have developped different layouts but in the very end, I typically used mostly one I feel more comfortable or that the response is most much better to some needs for a set of pieces that are composed so far. It is not fixed though and the problem is that when you change the assignment of a parameter, or an effector a drum-machine, sampler or whatever synthesizer to a particular sensor and in the next piece you change or in the next part even of the same piece you change this mapping placing, I don't know, a phaser rate, I don't know, that parameter instead of the pitch of a whammy pitch-shifter of course you have to re-learn the instrument because the instrument changes. It's beautiful because you have a power in that moment, at the cost of one click, that your instrument will change totally timbre, and we switch between it very very quickly, but in the other hand you have to re-learn the instrumentand sometimes it takes a lot of practice also, especially if you do a concert with different pieces that are rather different between each other, because typically a musician want to vary. So you have to remember, you have to study it is not an invented instrument, but it is an instrument that has ``software'' (?) something is not the (...) the same problems of any instrument,so you have to study, you change piece,you have to practice, study, rehearse, learn and be perfect... (laughs)

VG — Well, digital musical instruments are more quick to change that's maybe... once you have this kind of stable behavior of an acoustic instrument and the constant evoluting behaviour on the digital musical instrument. But then what you say is that you would choose to rely only on memory for the kind of behavior modification. I mean you could have for example screens telling you ``this is preset 2'' or whatever, or do not have presets...

LT — no, no, I do. I have a setting interface, I have two interfaces, one is for the expressive control, the other is for the settings and they have were six buttons, they are capacitive sensors with LEDs which allows me to understand in which bank I am and I can navigate between the banks with up and down and there are four presets within each bank

VG — do you have like a visual feedback?

LT — no because at the moment I have only LEDs that give you this, but I don't have a small display, this is something if they want to have.

VG — but those LEDs are a visual feedback giving you information about which bank you're using, you said?

LT — no I don't have it on the mandolin I have and I don't have because I have a two lights two buttons with its own LEDs I know that if I press one time I will have a LED that will blink if I press two times the LED will not blink and I would be in the second and I typically use two banks, four banks in general : two navigating upwards and two navigating backwards, in total four but the fact is also that I am having a computer near meso the visual feedback is there I mean I can always check it in which piece I amwhere I am because I have a second screen over that on the table near me

VG — but you'd preferably skip this screenand have an embedded solution somehow ?

LT — this is precisely the reason because I am building these ...(interruption) this is precisely the reason because I am... I have developed this novel family of instruments that are called ``smart instruments''. I don't know if you are familiar with them?

VG — a little bit

LT — ok the first examplar is the ``smart guitar'' by mind music labs this company that I mentioned before and that I co-founded so just in a nutshell what are smart instruments, they are instruments that have an embedded computational unit so they have intelligence and this intelligence is responsible primarily for the sensors processing so it is an augmented instrument with sensors and actuators or loudspeakers, these in many cases placed where the sound source is directly in the instrument itself and they have the feature of having a multi-directional wireless connectivity in this context if you have an embedded computational unit that powerfulit is easy to connect a touch display with visual information and more importantly you do not need an external computational unit such as a laptop traditionally, typically the most traditional setup for augmented instrument is where there is the sensors augmentation place on the instrument itself but then the computational unit is placed outside

VG — so you mean this connectivity is also a way I guess to update the instrument bank... 

LT — for instance this is a precisely the new line of research and I'm going to present a poster now( at audio mostly conference 2017, NdE) out the manifold interaction possibilities that are enabled by an instrument that has a sound engine inside, a system to deliver electronically generated sounds placed on the instrument itself this connectivity feature that allows people to join together with this instrument for instance and people for instance with a smartphone or a tablet can or even on a wireless keyboard can connect to the instrument play together with the instrument players that is playing the guitaror whatever other smart instrument and all the sounds are mixed and generated by the guitar itself so, this is a simple application

VG — maybe we're running out of time but maybe one last question I usually ask people ... a pronostic on what's important to them, what they feel is important to them in the field of digital musical instrument for the next ten years and how do they think this could change the way we make music in general

LT — oh this is a one million dollar question in the sense that ... mmm ... musicians in general are the most conservative people of all categories of human beings okay

LT — you think so ?

LT — for instance guitarists want the the sound of the Stratocaster of the fiftees and... and that's it. They are not open to anything else for instance most a musician nowadays are like that then there are the experimental musicians within those we should do analysis but I do believe however that many possibilities of producing sounds and spatialization and augmented instruments are somehow already being discovered there is a lot of research to do I agree but the core concept is not groundbreaking any more on that side in my humble opinion. What is the future in my opinion is the possibility of connecting things together okay so it's not by chance that we are living in this Internet of Things world now. This domain is more and more important in any level and also music is definitely affecting the by these tools, technologies but also behaviors people want to be connected anytime everywhere and with anyone and musicians are people and they, in my opinion they will feel more and more this need. okay? yeah, that sounds good, hope that I've answered to your questions

VG — yes, sure, thanks you.