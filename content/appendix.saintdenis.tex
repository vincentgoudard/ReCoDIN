\chapter{Interview : Patrick Saint-Denis}

\section*{Biographie}


\section*{Transcript}

PSD — Je me suis rendu compte que je faisais des IM slash scénographie, et puis là je suis un peu plus conscient que c'est ça que je fais, mais au début c'était pas dans le but de faire... c'était vraiment de la scénographiqe pour de la musique instrumentale ... donc j'ai ... 

VG — instruments acoustiques tu veux dire ? 

PSD — comment ? 

VG — instruments acoustiques ? 

PSD — ouais, instruments acoustiques, donc j'avais fait des trucs c'était 2008, 2009, c'était Processing, OpenFrameworks... de l'image audio-réactive...puis j'ai commencé à jouer avec des plumes d'oiseaux, qui étaient montées sur des moteurs et qui tournaient en fonction des amplitudes qui faisaient tourner les moteurs... et puis... que j'amplifiais visuellement avec une caméra, je traitais ... donc une espèce de workflow qui était celui de la musique mixte, si tu veux, prendre le son d'un instrument puis le modifier puis l'envoyer dans les HP, mais je faisais la même chose visuellement, avec des éléments qui bougeaient en même temps que le son ; je commence à apprendre l'arduino... donc je commence à développer des scénos avec des objets et de la vidéo qui réagissait au son... et puis j'ai tout de suite vu que ... qu'il y avait quelque chose de plus dans les objets que dans l'image vidéo... quelque chose qui attire plus, qui... euh... plus surprenant... avec un objet, à l'époque c'était une plume d'oiseau, c'était tout petit comme ça, et puis ça out-stageai (sic) à peu près les projections architecturales, tout ça... donc j'ai décidé d'aller là dedans.. poursuivre ça, poursuivre l'idée des objets animés par le son. En fait là j'ai fait un gros écran physique, qui est derrière, just de l'autre côté [du mur de l'atelier dans lequel nous sommes NDR]  

VG — avec les feuilles de papier là ? 

PSD — ça prend de la place.. et là c'est juste des petits ventilateurs... [me montrant le dispositif] (incompréhensible) donc c'est un écran physique... 

VG — et qui fait du son ? 

PSD — ben oui c'est ça, qui fait du son, et que tu peux toucher aussi... pour le spectateur, tu sens la draught d'air là..   

VG — oui 

PSD — il y a 192 petits ventilateurs qui font ffffffffffoooooo... y'a quand même quelque chose de fun là.... puis ce fait qu'là j'ai transposé dans le monde physique l'interaction de type audiovisuel... c'est comme ça que je l'imaginais, et puis là j'ai fait d'autres dispositifs avec des HP, des robots, et puis des ... encore des genres de transpositions là... des trucs vidéos dans de la... dans le monde physique et puis, et là j'ai commencé à travailler aussi avec la danse, et à utiliser ces dispositifs là vraiment comme de la scénographiqe et puis de fil en aiguille ces machines là, qui se voulaient de la scéno, sont aussi des instruments de musique, par exemple celui-là [montrant], donc cette série là d'accordéons-robots qui est vraiment... ça a été pensé comme étant de la scéno, mais tout le son vient de la scénographie et puis donc la scéno, la scène est un instrument. Et puis c'est un peu... c'est ça que je désire poursuivre maintenant, en tout cas c'est ça que je ... conçois mon travail mais tu sais j'aime beaucoup garder une bonne part de mystère dans ce que je fais, donc souvent je fais des choses et puis je découvre plus tard... souvent mon travail devance un petit peu ce que je pense de mon travail... ce qui fait que les projets qui s'en viennent sont des projets qui sont plus pensés au début comme étant des instruments scénographiques. 

VG — et... qu'est ce qui ... une question que je pose souvent quand je commence une interview... tu réponds en partie à ça mais qu'est ce qui... t'a amené à utiliser des instruments ou des outils numériques pour ton travail plutôt que d'utiliser des objets acoustiques ou classiques... 

PSD — ouais... je sais pas t'as quel âge toi, moi j'ai 42 ... 

VG — Moi 37.. 

PSD — ... quand je suis arrivé dans la composition... instrumentale... je suis rentré au conservatoire, j'avais 19ans, à Québec... clairement j'avais l'impression de rentrer dans un bateau qui coule... celui de la musique contemporaine... le discours... encore avec le poids des vieilles avant-gardes... et puis un discours assez... défaitistes là... “ fallait être là dans l'temps... les grandes œuvres sont faites... “ ... puis vraiment c'était ça le discours ambiant et très défaitistes... et puis évidemment t'as pas envie, toi, d'arriver dans un bateau qui coule... Et puis d'autant plus que culturellement cette musique là n'a jamais pris non plus dans la culture là... au Québec c'est encore assez colonial pour la musique... et puis là est arrivé... tu sais y'a un bateau qui coule et puis un espèce de bateau avec des moteurs, qui coule pas du tout, qui va vite, y'a des gens dessus qui font l'party [la fête, en Québécois, NDT] c'est celui des arts numériques, et puis ça m'intéressait beaucoup le fait que ... le fait que je me sois intéressé aux instruments c'est un peu... par cette mouvant là.. 

VG — l'effevescence qu'il y avait autour... 

PSD — ouais, les arts numériques et ... voilà... Aujourd'hui, c'est différent, on a tout l'intérêt avec les instruments tout le temps un peu  ... sachant se renouveller... par exemple je travaille beaucoup avec des danseurs pour jouer avec des instruments que je fais... assez peu avec des musiciens... et puis il y a quelque chose avec la technologie, que le corps disparaît... notre corps disparaît... devient invisible, à travers les écrans, ou même la musique éléctronique, le corps est quand même complètement... sorti de la boucle. C'est à dire que l'énergie, en musique électronique, l'énergie ne vient pas du corps.. le corps contrôle peut-être le son ? avec l'énergie pour faire le son qui vient de l'électricité... c'est pas une mauvais chose en soi... et puis des fois le corps est tellement gardé en dehors de ça que on se met à danser et puis on essaie de communiquer la musique avec le corps... toute la culture DJ par exemple, où on essaie de se garder occupé là, à tourner des pots (“ potentiomètres rotatifs des interfaces ”, NDT) ... y'a une autre façon d'engager le corps que ça.. et puis probablement la pire façon d'engager le corps c'est l'interface... [sortant son laptop et montrant le clavier] cette interface là qui est faite pour faire de la bureautique... écrire des courriels, faire des  fichiers excels... et puis c'est ce fait qu'en travaillant avec des danseurs, le fait c'est que tu as un mouvement, le mouvement du corps mais un mouvement qui n'est pas nécessairement instrumental, qui est comme une autre façon d'arranger ... de faire participer le corps avec le son... 

VG — il n'y a peut être pas tant de volonté de contrôle, tu veux dire ? ... par rapport à un instrumentiste qui contrôle son instrument ? 

PSD — ouais, c'est ça, ben c'est un autre genre de ... c'est un autre type de contrôle effectivement qui est moins fin [faisant un geste avec le bout de ses doigts] ... effectivement qui peut-être plus conceptuel... par exemple quand je prends des... par exemple avec les accordéons, avec le chandail qui permet d'aller la respiration de l'interprète et puis de le transférer aux accordéons... donc là y'a un transfert “ Homme-Machine ” [ajoutant des guillemets à l'expression avec ses mains, NDT] ... un anthropomorphise de... projections du corps sur le robots. Donc on n'est oas dans un contrôle d'interprétation de type instrumental. 

VG — Je pensais au fait que le danseur, dans sa pratique, est plutôt dans une pratique où le corps s'exprime en tant que tel, pour lui même, sans avoir besoin d'un instrument, alors que l'instrumentiste a besoin d'un instrument ... 

PSD — oui...effectivement, y'a ... et puis il y a plein d'autres façon par exemple avec la vision par ordinateur d'aller chercher le corps en mouvement, avec OpenCV, la Kinect... et puis de ... de transposer ces gestes là au son.. ç m'intéressait beaucoup.. peut-être un peu moins maintenant.. je sais pas, peut-être qu'on le voit trop...euh... mais oui, donc cette gestuelle là, sur scène et puis aussi toute le workflow, toutes les méthodes de travail qu'il y a en danse ça m'intéresse beaucoup... le temps qu'ils vont passer ensemble...parce qu'il n'y a pas vraiment de notation pour la danse... y'en a mais mais elles sont pas pratiquées hein, y'a personne qui prend la notation Laban et puis qui fait ah.... [mimant le fait de lire et comprendre une partition Laban] ... donc ils sont obligés de se parler dans le processus de création, et puis d'arriver comme ça et puis de.... de.. et puis là émergent des discussions de ... en laboratoire... ce qu'on fait très très très peu en musique ... en musique faut que ça soit efficace [tapant des main pour indique le rythme].. un show faut qu'ça se monte en trois répèts...etc... Passer seize semaines à monter un show c'est ... [mimant le fait que c'est impensable avec ses mains]... très rare... les chanteurs le font un petit peu plus... se pointer à une répétition sans savoir ce qu'on va faire...  

VG — ce serait plutôt l'équivalent de faire un album, ou un truc comme ça, presque ? 

PSD — oui, tout à fait...avec un groupe... oui, oui... dans les musiques populaires... tout à fait c'est plus près de ça... ce monde là que j'ai trouvé dans la danse, qui était réceptif à ce genre de travail que je faisais, qui était prêt à passer du temps, à développer une gestuelle qui... parce que c'est sûr qu'eux vont développer... si j'arrive avec une “ scéno slash instrument “ , eux vont aussi développer une gestuelle qui fait sens en terme d'interaction, par exemple, donc les deux se nourrissent... et voilà... c'est un peu l'espère de filon dans lequel je travaille en ce moment... 

VG — et par rapport à cette idée d'instrument, une autre différence que j'imagine entre ce qui est scénographique et l'instrument, c'est que l'instrument... acoustique en tout cas, l'instrument traditionnel il est réutilisé, pratiqué, pendant toute une vie, pour plein d'instrumentistes... alors que là j'imagine que ce sont des dispositifs qui sont plus éphémères.... ou.. comment tu travailles par rapport à ça ? 

PSD — absolument... ben la question de... premièrement la séparation entre l'instrument et l'œuvre, donc l'instrumentiste peut jouer, va jouer dans sa vie plein d'œuvres... etc...  moi je conçois plutôt que l'instrument c'est l'œuvre... et puis, bien sûr que je peux jouer plein de musiques différentes avec.. bon,.les accordéons, surtout... mais ça ferait pas sens. Y'a ... ouais, c'est pas quelque chose qui m'intéresse... moi ce qui m'intéresse c'est de développer... de faire un instrument avec lequel je vais peut-être aller chercher, allez, deux, trois performances maximum, la plupart du temps, une, si je fais un instrument qui est très limité, donc c'est un “ œuvre -instrument ”, que je vais performer. Donc ça c'est un volet. Après ça, la question de la pérennité des œuvres... euh... c'est sûr que la musique instrumentale est habituée à... [début de raillerie] ben on écrit une partition et là elle peut partir pour les siècles des siècles (rires)  quelqu'un dans 600 ans dans une autre culture complètement différente va pouvoir reprendre ton œuvre et la rejouer... [fin de raillerie] ... ça ça ne m'intéresse pas du tout, du tout, c'est pas comme ça que je veux contribuer à la culture, pas du tout. Durer... ben sur Youtube... je vais durer par l'archive visuelle de mes œuvres, qui est donc très importante.. documenter son travail c'est ... donc une façon de contribuer à la culture, comme ça... mais pas dans une optique de re-performance. Donc moi, je veux faire mes trucs, je veux m'amuser avec mes amis, on va faire des spectacles, les spectacles vont avoir une durée de vie, qui va être la leur, qui va trouver sa résonance dans le milieu ici... Et puis après ça, ça se retrouve sur Youtube, sur Facebook, et puis ça influencera peut-être quelqu'un d'autre, et puis ça aura la résonance que ça va avoir... donc ... c'est sûr c'est un autre euh... le théâtre c'est comme ça, le théâtre fallait... fallait être là ! Quand c'est arrivé. Le spectacle de danse c'est comme ça. Ou il y a des danses qui sont reprises, etc, mais ... 

VG — des pièces aussi de théâtre ... (non?) 

PSD — oui, mais le metteur... la mise en scène est scène est tellement importante...  

VG — oui 

PSD — par exemple, je sais pas... le cirque.... fallait aller au spectacle de cirque, pour voir le truc...si on n'était pas là, on n'était pas là, voilà, c'est pas grave, y'en a un autre (qui vient après) donc c'est un peu comme ça que je perçois mon travail...  

VG — donc, pour cet instrument là (les écrans de feuilles), tu vas te cantonner à une esthétique, comme tu disais tu pourrais faire plein de choses avec, mais tu veux faire une œuvre musicale d'un choix de jeux, de performances qui vont être spécifiques ... 

PSD — ouais, si tu veux ça c'est mon genre de devoir, que j'essaie de faire, c'est à dire de ... une fois que je me rends compte de ce que j'ai fait, comme instrument, tu sais quand c'est un projet, tu avances dans ton projet, et puis là tu as des itérations, et puis à un moment donné oups, le projet devient un instrument, et puis c'est là que tu prends un peu conscience de “ oh ! J'ai fait ça?! ” et puis là j'essaie déjà d'écouter, ce que j'ai fait, d'aller trouver c'est quoi les affordances spécifiques à la patente ["truc", "bidule", "machin" au Québec] que j'ai fait. Et puis par exemple dans le cas des accordéons, le fait que j'en ai 5, le fait qu'ils sont sur roues, avec des batteries de Skidoo... je sais pas si tu sais c'est quoi un Skidoo ? C'est un moto-neige... donc tu peux avoir les accordéons en mouvement dans l'espace sur un plateau scénique, donc ça vient influencer le jeu aussi, le fait que ça peut jouer vraiment plus vite qu'un être humain... donc il y a là une zone... c'est la zone que je recherche, la zone qui est unique à ça... et puis le fait qu'il y en ait 5... par exemple dans le spectacle de flamenco si tu veux, le spectacle commence ...(attrapant des accessoires) ... puis tu vois j'étais une peu habillé en marin, et puis ici j'ai une lumière de marin .... avec une petite centrale inertielle là dessus... et puis je la fait tourner comme ça (mimant un geste de tournoiement au dessus de sa tête)... pendant longtemps... et puis à un moment donné, quand le truc passe devant un accordéon, ça fait “ prrrrr ” et puis l'accordéon se met à faire des notes... et puis si y'a, dépendemment de la position de l'accordéon, [mime les bruits de déclenchement de chaque accodéon à différentes positions de l'espace] ... ça c'est une chose qui m'intéresse, à développer pour l'instrument... mais juste envoyer des notes pour envoyer des notes [non, de la tête].. ça c'est pas l'instrument... par exemple, comme le truc des instruments-robots qu'y en Europe...c'est en Allemagne qu'il ya  ... [cherchant] ... un truc de fou..  

VG — le projet de Squarepusher ? 

PSD — Pat Metheny.... [son projet Orchestion, NDT] 

VG — Pat Metheny a fait ça aussi ? ... Y'a pas mal de musiciens qui ont fait des trucs comme ça... Stephan Eicher a fait des trucs du genre.. 

PSD — Oh mon dieu ! ... Stephan Eicher... je suis allé cherché dans les années 90s 

VG — Oui, il fait un truc avec Max/MSP... 

PSD — ah oui ? 

VG — oui, je l'ai entendu interviewé à la radio, j'étais impressionné de ses connaissances là dessus... 

PSD — oh ! Par exemple (sortant son laptop), je vais te montrer, j'avais fait une patente à Mutek, où j'ai fait des “ compilations ” pour parler de ce que je viens de te parler... avec le mur de papier... [montrant une vidéo de XXX] ça c'est 30min c'était avec une chanteuse mais elle chante pas, elle fait juste souffler sursa main, sur  la membrane du micro. C'est vraiment de l'image vidéo, mais audio-réactive avec juste le souffle.. c'est 30 min, assez exigentes, évidemment il n'y a pas de partition... on a fait un “ trajet ” ... y'a des sons, y'a des traitements, à un moment il y a des seuils d'entrée ... et puis il;y a aussi toute une gestuelle, à un moment elle se met à bouger à spatialiser un son gauche/droite, et puis après ça elle va contrôler le mur de papier... et puis il y a toute une gestuelle performative plus proche de la danse... qu'on a développé ensemble... et puis bon, moi j'ai fait le mur avant, mais il y a beaucoup d'interactions qu'on a fait en atelier... en travaillant, en collaborant... je ne me sens pas comme un compositeur “ top -down “  qui viendrait donner... c'est plutôt “ bottom-up ”... donc le bottom étant la scénographie quand j'arrive, qui est tout le temps à peu près à 50\% finie... j'arrive avec quelque chose en input... et puis après ça émerge 

VG — en fonction de la personne ? 

PSD — ouais, en fonction de la personne, de ce qu'on est capable d'aller chercher... souvent on travaille avec — truc que j'ai appris de la danse — des “ œil extérieurs ” qu'on a assez peu en musique... voire jamais... Ca c'est un autre instrument [montrant une vidéo du projet XXX] c'est une Leslie, en fait c'est 16 Leslies. Donc là il y a un mesh en 3D, chaque point ferme ton point ferme ton mesh... imagine que chaque point serait un haut-parleur Leslie, qui tourne, qui monte et qui descend. Ce qui se passe, si tu en as 16 ...haut Parleurs Leslie qu'on a fait... et ils tournent, montent et descendent sur à peu près 2 mè!res... ce qui est en jaune, ce qui est en blanc, c'est juste la lumière qui est attachée aux Leslie... et puis moi je contrôle une Leslie virtuelle qui est rouge, avec un simulateur d'interface... un joystick... et puis un peu comme un “ flocking ”, une nuée d'oiseaux, là c'est une nuée de speakers, donc le HP rouge que je déplace est maître et les autres le suivent avec une certaine distance, et puis le son est en fonction de la position du HP... mais dans la vraie vie.... ce qui fait que là tu vois on a un “ acousmonium slash instrument ” joué en live ...  

VG — ça fait quelque taille là ? 

PSD — ça fait 16 par 16 par 16...  

VG — mètres ? 

PSD — Pieds ... euh... 3 par 3 par 3m... ça c'est un gros projet (montrant un autre projet) 

VG — ah.. j'aurais bien aimé entendre ça .. [le projet Leslie] 

PSD — on l'a fait que 2 fois ! (rires) 

VG — c'est pour ça que c'est bien les choses pérennes des fois.. (rires)... sur Youtube tu n'as pas... 

PSD — on n'a pas bien documenté 

VG — c'est difficile de transposer ça sur YouTube quand même 

PSD — ouais... je suis d'accord... 

VG — l'expérience sonore est... 

PSD — ça traine encore dans le sous-sol de l'université... faudrait le refaire...on était sensés le faire à la SATT, mais bon ce ne sera pas possible 

VG — et il n'y a pas de gens qui utilisent des acousmoniums qui seraient intéressés par utiliser le truc ? 

PSD — euh... faut que je vois comment...euh... quel serait l'intérêt des gens, avec une scéno... 

VG — ...qui soit un peu sur ce créneau là ? 

PSD — ... ouais c'est ça... en même temps... je suis un peu passé à d'autres choses...[montrant un autre projet]  tu vois j'en ai fait comme un autre... mais c'est un peu au garage... faut que ... je veux améliorer la vitesse de réaction des ... ici c'est un peu comme un Leslie  mais qui a un tilt/pan dans le fond... qui tourne comme ça [montrant les sens de rotation] qui monte et descend... qui est attaché avec la centrale inertielle de la danseuse... de la chanteuse/performeuse... ce qui fait que quand elle, elle va bouger, ça va contrôler les HP... chacun sont à 2m de hauteur... et puis le traitement du signal dépend aussi du... de la position... il y a un filtre qui va ouvrir en fonction de [mouvement avec les bras]... 

VG — c'est très beau... 

PSD — ça c'est une scénographie que je veux développer... on a fait une résidence qui... en fin de compte on n'a pas fait grand chose d'autre que mettre des pads [nappes de sons] et puis d'avoir un filtre, et puis déjà d'avoir fait ça,ça a pris un mois, on a été surpris... tu vois, il y a des itérations, et puis ce projet là il est dans des itérations assez ... 

VG — tu sais qu'il y a quelque chose... 

PSD — je sais qu'il y a quelque chose à découvrir et puis faudrait que je trouve soit une compagnie... mais tu vois en ce moment, je suis en train de penser, que cet été si je mets pas un peu d'effort pour me donner une structure artistique qui me permettrait d'explorer moi-même les objets que je travaille et puis d'inviter des danseurs, des gens avec qui travailler à l'intérieur de ma structure artistique plutôt que... c'est très difficile quand tu arrives avec un chorégraphe qui lui aussi a son idée de show... avec le mur [de feuillles], j'avais fait un truc avec une chorégraphe et puis ça avait bien fonctionné, mais... ça prend un ou une chorégraphe... tu sais la fameuse hiérarchie entre la technologie et l'art... que souvent des gens voient comme des choses opposées quand les gens connaissent pas beaucoup les arts numériques... t'apprend une chorégraphe qui a une culture des arts numériques pour ... voilà... pouvoir se sentir tassé [poussé, prendre la place, en québécois]  par les ... 

VG — par l'utilisation qui est faite de ce que tu as ... 

PSD — ouais c'est ça... c'est sûr que des fois il peut y avoir une technologie qui est contraignante, en terme de mouvement... les musiciens c'est facile... les musiciens, ils ne bougent pas, ils sont là comme ça, et puis  tu peux leur mettre un capteur... les danseurs, ça bouge, ils se frottent, donc tu mets un capteur, le capteur tombe... et puis donc, si tu veux travailler avec une technologie, et que cette technologie t'apporte des contraintes en terme de mouvement, et bien ça peut être bien... si c'est c'est l'affordance de ton truc qui va faire émerger une danse unique avec la patente que t'as là... mais pour ça, ça prend une bonne culture des arts numériques, c'est pas tout le monde... donc oui, je suis en train de me développer tranquillement... euh... 

VG — pour créer tes propres formes... 

PSD — ben, créer et faire du spectacle vivant ... faire du spectacle vivant avec juste une structure pour aller chercher assez de sous, dans le fond, pour ... pouvoir travailler avec les gens avec qui ça me tente de travailler pendant à peu près ... un spectacle c'est à peu près 16 semaines de travail sur deux ans... à peu près... à partir de la 1ère rencontre... “ bon ok, on a des cônes blanc, avec des HP, qu'est ce qu'on fait avec ça ? ” et puis de faire émerger un spectacle, peut-être de quoi... 50 min, 1h ... 16 semaines... à peu près... tu fais deux semaines là,  et puis six mois plus tard, deux semaines là... et puis ouais.. j'aimerais ça avoir ma propre structure... “ un jour peut-être ”... Il y a beaucoup de compositeurs qui font ça... Georges Aperghis, Heiner Goebbels... 

VG — oui, et dans les arts numériques, comment il s'appelle déjà... un mec qui a pas mal tourné en France... qui a commencé sur la base de jonglage... leur compagnie s'appelle ... c'est leurs initialles... euh... elle c'est Claire Bardenne et ... Adrien Mondot... 

PSD — je les connais pas 

VG — c'est un type qui a commencé par du jonglage un peu  augmenté par de la vidéo et en fait il a beaucoup développé le... 

PSD — Julien Mondot tu dis ? 

VG — Adrien Mondot  

PSD — [cherchant sur le web] ... 

VG — je pense qu'il a du venir à Elektra ou ... c'est un truc assez grand public... 

PSD — OK, je vais regarder... 

VG — et je disais parce qu'il a créé sa compagnie un peu pour ces raisons là en fait... il créait des objets et voulait pouvoir ... quelque part... assumer la mise en scène de ces objets du début à la fin sans forcément que cela soit intégré dans un spectacle où tu n'as plus trop le contrôle de .. 

PSD — ouais, c'est ça... ton choix de mots est bon... “ intégré ”... “ l'intégration de technologies ”,  à chaque fois que j'entends ça , ça me fait des maux de ventre... quand est-ce que quelqu'un intègre... je sais pas... “ on va  intégrer ... intégrer une flûte ”... ben non tu peux prendre la flute, la flute ça fait partie de ta ... c'est comme si la techno , les gens qui faisaient... 

VG — comme si c'était un truc que tu achetais... 

PSD — ouais, voilà .... “ ce sont surtout pas des artistes, les codeurs, tout ça  ”... c'est dommage, mais c'est la peur... essentiellement... et puis c'est un espèce d'état qui vient aussi des générations passées ... que je comprends... quand les outils n'étaient pas disponibles non plus... tu sais il y a une grosse grosse démocratisation des outils pour faire de la création numérique depuis... internet... qui fait que avant ça... surtout le modèle français, à l'IRCAM, qu'on a eu aussi ici... tu sais le “ réalisateur en informatique musicale ” qu'est comme le ... [mimant un geste de mise de côté] et puis là le compositeur qui lui a les idées, la création.... 

VG — et puis le réalisateur qui fait tout le boulot ... 

PSD — oui, qui fait tout le boulot, mais tu sens qu'il peut pas faire grand chose... parce que ... et puis ça c'est le modèle, tu sais IRCAMien de la musique mixte, etc... et puis c'est très très hiérarchique comme musique, tu sais, c'était la musique instrumental qui est tout au sommet, et t'as en “ en périphérie ”... en “ reverb ” essentiellement... très sophistiquée parfois... comme un enrobage... et puis bon c'est sûr quand t'as un être humain sur scène, t'as tendance à lui donner le rôle principal et puis que leur son est comme environnant à ça... en même temps y'a aussi dans la collaboration le compositeur qui connait pas grand chose et la personne qui fait tout le reste... 

VG — et même au dela de ça, tu vois j'étais même étonné de voir que... ça change petit à petit mais vraiment très doucement... que le RIM n'était même pas sur scène, qu'il fallait toujours qu'il soit caché... 

PSD — ben ya aussi la perspective de pouvoir faire une meilleure balance de son aussi... 

VG — oui, mais c'est quand même un peu plus que ça... 

PSD — il y a des œuvres aussi ... je pense par exemple à Light Music de Thierry de Mey...euh... c'est le RIM qui a fait ça... au complet... et puis c'est une œuvre qui en ce moment est dans un état de presque impossibilité de performance, tu sais ça prend un Mac 2 avec telle version d'OpenCV... tu sais, c'est pas la fin du monde... c'est du beau travail, c'est bien fait, mais on parle quand même juste deux blobs avec les mains... je sais pas si tu connais cette pièce là... 

VG — des sons qui sont déclenchés par les mains... 

PSD — ouais c'est ça... 

VG — ... en mettant les mains dans un fil de lumière c'est ça ? 

PSD — ouais, avec une caméra, et puis ... c'est quand même pas la fin du monde, là... mais les artistes qui sont sensés avoir l'autorat de cette œuvre là, ben ils ne peuvent pas mettre à jour cette œuvre là... c'est à dire il y a comme une tension entre la musique instrumentale qui veut la re-performance etc.. et puis des instruments.. qui ... tu sais l'obsolescence qui fait partie intégrée du truc, tu sais... 

VG — c'était la grosse question à la fin de TENOR [conférence TENOR]... qui revient à chaque fois qu'il y a des conférences sur les technologies de la musique... qui est que ... là ils veulent faire un réseau ... Sandeep a trouvé des sous pour qu'il y ait un réseau sur ces questions de technologies de la notation, et ils demandaient aux gens qui étaient là à la conférence, ce à quoi pourrait servir ces sous, quels projets ils trouvaient important, et quelqu'un a dit ... Daniele Ghisi en l'occurence, qui est le créateur de Bach, la librairie pour Max... qui dit “ la préservation des œuvres ”.. “ le problème est qu'aujourd'hui, une œuvre est créée tel jour et ne peut pas être rejouée trois ans après, c'est un gros gros problème ”... et alors là, c'était la boite de ... 

PSD — mais pourquoi c'est un problème ? 

VG — — ben pour lui, c'est un problème — 

PSD — mais voilà, parce qu'on est attaché à ... 

VG — ... la boite de Pandore où chacun avait son idée différente... 

PSD — moi ce que j'ai remarqué, c'est que en général, c'est un problème des... les artistes... un artiste, c'est dans la création, ça veut faire un truc, et puis là “ oh c'est fini”... hop on passe à la prochaine création, et puis on s'en fout, je veux dire qui vraiment là veut contribuer à la culture... je les trouve suspects ces gens là... enfin voilà, y'a plein... la danse c'est comme ça... on fait la danse où les gens collabore beaucoup plus qu'en musique, et puis là ils se font des heures, et ça empêche pas que la pratique avance elle aussi... mais c'est pas nécessairement des “ monuments ”... mais c'est vrai qu'il y a quelque chose d'intéressant dans l'archéologie de l'informatique si tu veux, il y a vraiment quelque chose d'intéressant là, d'aller fouiller dans l'histoire, et il commence quand même à y avoir plusieurs couches là ... ça c'est très intéressant mais... je peux comprendre qu'un musicologue se pose cette question là 

VG — pour les créateurs, enfin en tout cas à titre personnel,  je trouve qu'il y a aussi un intérêt par rapport à son propre boulot... c'est juste que c'est chiant à titre personnel si tu dois ré-ouvrir un truc pour le refaire fonctionner parce qu'on te fait une commande pour dire, tiens le truc que tu avais fait il y a trois ans là, on voudrait le programmer là, et puis t'es là... oh putain... le système a changé, le capteur il marche plus... 

PSD — ouais... mais ça c'est ta maintenance, la maintenance de tes œuvres, ce qui est correct là... faut vivre avec, et puis il y a des œuvres qui sont non-reperformables, ça devient une archive et puis... une autre façon ce serait d'arriver avec des standards, de faire en sorte que les... 

VG — oui, c'était une des réponses qui étaient évoquées... 

PSD — oui, mais c'est pas connaître les artistes pour imaginer [joignant les mains docilement] “ oh, on va tous fonctionner avec le même standard ”... ben non... on est punks... 

VG — dans la nature tu vas chercher ce qui est après le standard... 

PSD — “ vous allez faire ça parfait ”... c'est une vue de l'esprit qu'on va le mettre en cage et qu'on va régler le problème... ben non c'est pas un problème, c'est un trait caractéristique de la création aujourd'hui 

VG — d'aller gratter des zones qui sont en dehors des ... 

PSD — c'est ça... 

VG — et une question, par rapport à ces scénographies que tu peux faire... ce qui m'intéresse dans ma recherche sur ces DMI, c'est vraiment ce qui est spécifique au numérique.. alors toi peut-être que tu prends pas le numérique comme... quand j'en parlais à Nicolas [Bernier] hier, il me disait “ mais moi le numérique je m'en fous, c'est juste un moyen ” et son point de vue est complètement OK... mais ce qui m'intéresse de voir c'est ce qui est spécifique et notamment le fait que quand tu fais de la scéno, tu peux avoir les mêmes objets physiques, les mêmes capteurs, les mêmes synthèses FM ou que sais-je et la relation, le mapping qui change complètement entre le début et la fin... après t'as différentes stratégies possibles avec ça... fonctionner par scène, par tableau, ou par métamorphose... c'est quoi tes manières de faire par rapport à ça toi ? 

PSD — par rapport au mapping ? 

VG — par rapport au fait que tout soit constamment permutable 

PSD — ouais 

VG — est ce que tu prends ça comme un truc qui te ... 

PSD — bonne question, je l'ai jamais vraiment réfléchie... c'est une bonne question... je sais pas... je sais pas quoi te répondre... 

VG — par exemple, dans l'exemple de “ Ways ” où tu fais une performance avec.. est ce que tu utilises le même mapping du début à la fin, est ce que c'est un truc qui change...si tu dois faire un concert d'une heure, comment tu vas t'y prendre,  par exemple, pour structurer une heure de temps  

PSD — ben c'est un art, j'ai pas de recettes...mais toutes les possibilités sont là... je peux fonctionner soit fonctionner par cues... fonctionner... dans le fond j'ai toute la liberté que je veux... je programme ... tu sais la plupart du temps en C++ dans OpenFrameworks ou dans SuperCollider, c'est les deux outils que j'ai, et ces deux outils là me permettent la flexibilité, c'est souvent aussi... donc oui, par exemple je vais prendre un set de mapping qui va durer... par exemple dans la pièce Ways, j'ai deux instruments [montre la vidéo]... ça c'est un instrument, c'est pas compliqué, c'est de la PWM avec des gros sons... avec des trigs... et puis après ça j'ai un second instrument, qui est ... ça commence c'est cet instrument là qu'entame, puis après ça j'ai un instrument de granulaire qui granule ma voix ... 

VG — par exemple, comment tu passes de l'un à l'autre ? 

PSD — j'ai un footswitch... 

VG — par scène, par tableaux 

PSD — ouais, c'est ça j'ai deux tableaux 

VG — quand tu créé un objet comme ça, t'as ... enfin j'imagine que les deux cas de figure sont possibles, t'as plus ou moins un scénario en tête où tu te dis “ je veux raconter ça et je vais développer les objets qui me permettent de raconter ça ”... ou dans l'autre sens, tu développes un objet et tu te dis “ je vais juste garder ce mapping, rien faire d'autre, pas le faire évoluer et je vais tirer tout ce que je peux de ce truc là“  

PSD — oui, mais c'est plus compliqué que ça.... je sais qu'il y a beaucoup d'importance qui est donné au mapping, dans le design d'instruments de musique... mais ça revient à une vision qui est tellement structuraliste de d'une œuvre... c'est plus que ton mapping, ton œuvre... moi je privilégie vraiment une approche... 

VG — en fait je dis “ mapping ” mais je n'aime pas ce terme du tout... 

PSD — ouais...  

VG — c'est plutôt du design d'interaction... parce que mapping on a l'impression que c'est juste relier une variable à une autre alors que c'est tout sauf ça... 

PSD — Peut-être... 

VG — tu as l'écriture du temps, tu as la scénographie, tu as l'interaction de quel geste va être associé à quoi... 

PSD — tu sais, ça vient... je suis pas arrivé à cette œuvre là du premier coup... il y a eu plusieurs itérations... et là maintenant je vais en faire une autre... 

VG — mais quand tu fais des œuvres comme celle là, où comme la matrice de feuilles, ou les accordéons, tu pars généralement plutôt d'une idée de scénario, ou je vais faire un truc sur une heure, ou... 

PSD — je pars de l'objet... ouais, c'est intéressant... je pars de mon design... qui est souvent physique, mais qui peut être comme une composante logicielle ... 

VG — plus proche du plasticien où tu explores le matériau 

PSD — ouais exactement... par la matière... il y a assez peu ... je suis assez peu intéressé par le symbolisme... en général ... je sais qu'il y en a beaucoup... tu sais “ l'homme-machine ”,  l'extension de l'être humain par l'interaction... mais c'est pas quelque chose que je développe... c'est peut-être mon passé de musique instrumentale... c'est tellement abstrait la musique instrumentale, c'est [ ???] ... il n'y a pas beaucoup de discours... que tu peux faire avec ça... alors que tu peux en faire beaucoup avec des objets... quand je faisais des trucs avec des plumes d'oiseaux migrateurs, alors t'as plein de symbolisme là, qui peut sortir... et puis ça m'intéresse assez peu... de faire des œuvres qui s'adresse... 

VG — de raconter ? 

PSD — ouais.. de faire des œuvres qui s'adressent à la raison, ce genre de sens là ... et puis étonnamment c'est quelque chose que je me suis battu contre... dans la musique instrumentale... mais finalement je suis quand même attaché, quand je fais de la vidéo c'est quand même très abstrait... j'aime ça qu'il y ait des éléments de sens qui gravitent là autour de mes œuvres, mais mon travail n'est pas là dessus... mon travail est dans le faire... 

(more transcript to come)