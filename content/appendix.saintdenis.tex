\chapter{Interview : Patrick Saint-Denis}

\section*{Biographie}


\section*{Transcript}


PSD — Je me suis rendu compte que je faisais des IM slash scénographie, et puis là je suis un peu plus conscient que c'est ça que je fais, mais au début c'était pas dans le but de faire... c'était vraiment de la scénographiqe pour de la musique instrumentale … donc j'ai … 

VG —  instruments acoustiques tu veux dire ? 

PSD — comment ? 

VG —  instruments acoustiques ? 

PSD — ouais, instruments acoustiques, donc j'avais fait des trucs c'était 2008, 2009, c'était Processing, OpenFrameworks... de l'image audio-réactive...puis j'ai commencé à jouer avec des plumes d'oiseaux, qui étaient montées sur des moteurs et qui tournaient en fonction des amplitudes qui faisaient tourner les moteurs... et puis... que j'amplifiais visuellement avec une caméra, je traitais … donc une espèce de workflow qui était celui de la musique mixte, si tu veux, prendre le son d'un instrument puis le modifier puis l'envoyer dans les HP, mais je faisais la même chose visuellement, avec des éléments qui bougeaient en même temps que le son ; je commence à apprendre l'arduino... donc je commence à développer des scénos avec des objets et de la vidéo qui réagissait au son... et puis j'ai tout de suite vu que … qu'il y avait quelque chose de plus dans les objets que dans l'image vidéo... quelque chose qui attire plus, qui... euh... plus surprenant... avec un objet, à l'époque c'était une plume d'oiseau, c'était tout petit comme ça, et puis ça out-stageai (sic) à peu près les projections architecturales, tout ça... donc j'ai décidé d'aller là dedans.. poursuivre ça, poursuivre l'idée des objets animés par le son. En fait là j'ai fait un gros écran physique, qui est derrière, just de l'autre côté [du mur de l'atelier dans lequel nous sommes NDR]  

VG —  avec les feuilles de papier là ? 

PSD — ça prend de la place.. et là c'est juste des petits ventilateurs... [me montrant le dispositif] (incompréhensible) donc c'est un écran physique... 

VG —  et qui fait du son ? 

PSD — ben oui c'est ça, qui fait du son, et que tu peux toucher aussi... pour le spectateur, tu sens la draught d'air là..   

VG —  oui 

PSD — il y a 192 petits ventilateurs qui font ffffffffffoooooo... y'a quand même quelque chose de fun là.... puis ce fait qu'là j'ai transposé dans le monde physique l'interaction de type audiovisuel... c'est comme ça que je l'imaginais, et puis là j'ai fait d'autres dispositifs avec des HP, des robots, et puis des … encore des genres de transpositions là... des trucs vidéos dans de la... dans le monde physique et puis, et là j'ai commencé à travailler aussi avec la danse, et à utiliser ces dispositifs là vraiment comme de la scénographiqe et puis de fil en aiguille ces machines là, qui se voulaient de la scéno, sont aussi des instruments de musique, par exemple celui-là [montrant], donc cette série là d'accordéons-robots qui est vraiment... ça a été pensé comme étant de la scéno, mais tout le son vient de la scénographie et puis donc la scéno, la scène est un instrument. Et puis c'est un peu... c'est ça que je désire poursuivre maintenant, en tout cas c'est ça que je … conçois mon travail mais tu sais j'aime beaucoup garder une bonne part de mystère dans ce que je fais, donc souvent je fais des choses et puis je découvre plus tard... souvent mon travail devance un petit peu ce que je pense de mon travail... ce qui fait que les projets qui s'en viennent sont des projets qui sont plus pensés au début comme étant des instruments scénographiques. 

VG —  et... qu'est ce qui … une question que je pose souvent quand je commence une interview... tu réponds en partie à ça mais qu'est ce qui... t'a amené à utiliser des instruments ou des outils numériques pour ton travail plutôt que d'utiliser des objets acoustiques ou classiques... 

PSD — ouais... je sais pas t'as quel âge toi, moi j'ai 42 … 

VG —  Moi 37.. 

PSD — … quand je suis arrivé dans la composition... instrumentale... je suis rentré au conservatoire, j'avais 19ans, à Québec... clairement j'avais l'impression de rentrer dans un bateau qui coule... celui de la musique contemporaine... le discours... encore avec le poids des vieilles avant-gardes... et puis un discours assez... défaitistes là... “ fallait être là dans l'temps... les grandes œuvres sont faites... “ … puis vraiment c'était ça le discours ambiant et très défaitistes... et puis évidemment t'as pas envie, toi, d'arriver dans un bateau qui coule... Et puis d'autant plus que culturellement cette musique là n'a jamais pris non plus dans la culture là... au Québec c'est encore assez colonial pour la musique... et puis là est arrivé... tu sais y'a un bateau qui coule et puis un espèce de bateau avec des moteurs, qui coule pas du tout, qui va vite, y'a des gens dessus qui font l'party [la fête, en Québécois, NDT] c'est celui des arts numériques, et puis ça m'intéressait beaucoup le fait que … le fait que je me sois intéressé aux instruments c'est un peu... par cette mouvant là.. 

VG —  l'effevescence qu'il y avait autour... 

PSD — ouais, les arts numériques et … voilà... Aujourd'hui, c'est différent, on a tout l'intérêt avec les instruments tout le temps un peu  … sachant se renouveller... par exemple je travaille beaucoup avec des danseurs pour jouer avec des instruments que je fais... assez peu avec des musiciens... et puis il y a quelque chose avec la technologie, que le corps disparaît... notre corps disparaît... devient invisible, à travers les écrans, ou même la musique éléctronique, le corps est quand même complètement... sorti de la boucle. C'est à dire que l'énergie, en musique électronique, l'énergie ne vient pas du corps.. le corps contrôle peut-être le son ? avec l'énergie pour faire le son qui vient de l'électricité... c'est pas une mauvais chose en soi... et puis des fois le corps est tellement gardé en dehors de ça que on se met à danser et puis on essaie de communiquer la musique avec le corps... toute la culture DJ par exemple, où on essaie de se garder occupé là, à tourner des pots (“ potentiomètres rotatifs des interfaces ”, NDT) … y'a une autre façon d'engager le corps que ça.. et puis probablement la pire façon d'engager le corps c'est l'interface... [sortant son laptop et montrant le clavier] cette interface là qui est faite pour faire de la bureautique... écrire des courriels, faire des  fichiers excels... et puis c'est ce fait qu'en travaillant avec des danseurs, le fait c'est que tu as un mouvement, le mouvement du corps mais un mouvement qui n'est pas nécessairement instrumental, qui est comme une autre façon d'arranger … de faire participer le corps avec le son... 

VG —  il n'y a peut être pas tant de volonté de contrôle, tu veux dire ? … par rapport à un instrumentiste qui contrôle son instrument ? 

PSD — ouais, c'est ça, ben c'est un autre genre de … c'est un autre type de contrôle effectivement qui est moins fin [faisant un geste avec le bout de ses doigts] … effectivement qui peut-être plus conceptuel... par exemple quand je prends des... par exemple avec les accordéons, avec le chandail qui permet d'aller la respiration de l'interprète et puis de le transférer aux accordéons... donc là y'a un transfert “ Homme-Machine ” [ajoutant des guillemets à l'expression avec ses mains, NDT] … un anthropomorphise de... projections du corps sur le robots. Donc on n'est oas dans un contrôle d'interprétation de type instrumental. 

VG —  Je pensais au fait que le danseur, dans sa pratique, est plutôt dans une pratique où le corps s'exprime en tant que tel, pour lui même, sans avoir besoin d'un instrument, alors que l'instrumentiste a besoin d'un instrument … 

PSD — oui...effectivement, y'a … et puis il y a plein d'autres façon par exemple avec la vision par ordinateur d'aller chercher le corps en mouvement, avec OpenCV, la Kinect... et puis de … de transposer ces gestes là au son.. ç m'intéressait beaucoup.. peut-être un peu moins maintenant.. je sais pas, peut-être qu'on le voit trop...euh... mais oui, donc cette gestuelle là, sur scène et puis aussi toute le workflow, toutes les méthodes de travail qu'il y a en danse ça m'intéresse beaucoup... le temps qu'ils vont passer ensemble...parce qu'il n'y a pas vraiment de notation pour la danse... y'en a mais mais elles sont pas pratiquées hein, y'a personne qui prend la notation Laban et puis qui fait ah.... [mimant le fait de lire et comprendre une partition Laban] … donc ils sont obligés de se parler dans le processus de création, et puis d'arriver comme ça et puis de.... de.. et puis là émergent des discussions de … en laboratoire... ce qu'on fait très très très peu en musique … en musique faut que ça soit efficace [tapant des main pour indique le rythme].. un show faut qu'ça se monte en trois répèts...etc... Passer seize semaines à monter un show c'est … [mimant le fait que c'est impensable avec ses mains]... très rare... les chanteurs le font un petit peu plus... se pointer à une répétition sans savoir ce qu'on va faire...  

VG —  ce serait plutôt l'équivalent de faire un album, ou un truc comme ça, presque ? 

PSD — oui, tout à fait...avec un groupe... oui, oui... dans les musiques populaires... tout à fait c'est plus près de ça... ce monde là que j'ai trouvé dans la danse, qui était réceptif à ce genre de travail que je faisais, qui était prêt à passer du temps, à développer une gestuelle qui... parce que c'est sûr qu'eux vont développer... si j'arrive avec une “ scéno slash instrument “ , eux vont aussi développer une gestuelle qui fait sens en terme d'interaction, par exemple, donc les deux se nourrissent... et voilà... c'est un peu l'espère de filon dans lequel je travaille en ce moment... 

VG —  et par rapport à cette idée d'instrument, une autre différence que j'imagine entre ce qui est scénographique et l'instrument, c'est que l'instrument... acoustique en tout cas, l'instrument traditionnel il est réutilisé, pratiqué, pendant toute une vie, pour plein d'instrumentistes... alors que là j'imagine que ce sont des dispositifs qui sont plus éphémères.... ou.. comment tu travailles par rapport à ça ? 

PSD — absolument... ben la question de... premièrement la séparation entre l'instrument et l'œuvre, donc l'instrumentiste peut jouer, va jouer dans sa vie plein d'œuvres... etc...  moi je conçois plutôt que l'instrument c'est l'œuvre... et puis, bien sûr que je peux jouer plein de musiques différentes avec.. bon,.les accordéons, surtout... mais ça ferait pas sens. Y'a … ouais, c'est pas quelque chose qui m'intéresse... moi ce qui m'intéresse c'est de développer... de faire un instrument avec lequel je vais peut-être aller chercher, allez, deux, trois performances maximum, la plupart du temps, une, si je fais un instrument qui est très limité, donc c'est un “ œuvre -instrument ”, que je vais performer. Donc ça c'est un volet. Après ça, la question de la pérennité des œuvres... euh... c'est sûr que la musique instrumentale est habituée à... [début de raillerie] ben on écrit une partition et là elle peut partir pour les siècles des siècles (rires)  quelqu'un dans 600 ans dans une autre culture complètement différente va pouvoir reprendre ton œuvre et la rejouer... [fin de raillerie] … ça ça ne m'intéresse pas du tout, du tout, c'est pas comme ça que je veux contribuer à la culture, pas du tout. Durer... ben sur Youtube... je vais durer par l'archive visuelle de mes œuvres, qui est donc très importante.. documenter son travail c'est … donc une façon de contribuer à la culture, comme ça... mais pas dans une optique de re-performance. Donc moi, je veux faire mes trucs, je veux m'amuser avec mes amis, on va faire des spectacles, les spectacles vont avoir une durée de vie, qui va être la leur, qui va trouver sa résonance dans le milieu ici... Et puis après ça, ça se retrouve sur Youtube, sur Facebook, et puis ça influencera peut-être quelqu'un d'autre, et puis ça aura la résonance que ça va avoir... donc … c'est sûr c'est un autre euh... le théâtre c'est comme ça, le théâtre fallait... fallait être là ! Quand c'est arrivé. Le spectacle de danse c'est comme ça. Ou il y a des danses qui sont reprises, etc, mais … 

VG —  des pièces aussi de théâtre … (non?) 

PSD — oui, mais le metteur... la mise en scène est scène est tellement importante...  

VG —  oui 

PSD — par exemple, je sais pas... le cirque.... fallait aller au spectacle de cirque, pour voir le truc...si on n'était pas là, on n'était pas là, voilà, c'est pas grave, y'en a un autre (qui vient après) donc c'est un peu comme ça que je perçois mon travail...  

VG —  donc, pour cet instrument là (les écrans de feuilles), tu vas te cantonner à une esthétique, comme tu disais tu pourrais faire plein de choses avec, mais tu veux faire une œuvre musicale d'un choix de jeux, de performances qui vont être spécifiques … 

PSD — ouais, si tu veux ça c'est mon genre de devoir, que j'essaie de faire, c'est à dire de … une fois que je me rends compte de ce que j'ai fait, comme instrument, tu sais quand c'est un projet, tu avances dans ton projet, et puis là tu as des itérations, et puis à un moment donné oups, le projet devient un instrument, et puis c'est là que tu prends un peu conscience de “ oh ! J'ai fait ça?! ” et puis là j'essaie déjà d'écouter, ce que j'ai fait, d'aller trouver c'est quoi les affordances spécifiques à la patente ["truc", "bidule", "machin" au Québec] que j'ai fait. Et puis par exemple dans le cas des accordéons, le fait que j'en ai 5, le fait qu'ils sont sur roues, avec des batteries de Skidoo... je sais pas si tu sais c'est quoi un Skidoo ? C'est un moto-neige... donc tu peux avoir les accordéons en mouvement dans l'espace sur un plateau scénique, donc ça vient influencer le jeu aussi, le fait que ça peut jouer vraiment plus vite qu'un être humain... donc il y a là une zone... c'est la zone que je recherche, la zone qui est unique à ça... et puis le fait qu'il y en ait 5... par exemple dans le spectacle de flamenco si tu veux, le spectacle commence ...(attrapant des accessoires) ... puis tu vois j'étais une peu habillé en marin, et puis ici j'ai une lumière de marin …. avec une petite centrale inertielle là dessus... et puis je la fait tourner comme ça (mimant un geste de tournoiement au dessus de sa tête)... pendant longtemps... et puis à un moment donné, quand le truc passe devant un accordéon, ça fait “ prrrrr ” et puis l'accordéon se met à faire des notes... et puis si y'a, dépendemment de la position de l'accordéon, [mime les bruits de déclenchement de chaque accodéon à différentes positions de l'espace] … ça c'est une chose qui m'intéresse, à développer pour l'instrument... mais juste envoyer des notes pour envoyer des notes [non, de la tête].. ça c'est pas l'instrument... par exemple, comme le truc des instruments-robots qu'y en Europe...c'est en Allemagne qu'il ya  … [cherchant] … un truc de fou..  

VG —  le projet de Squarepusher ? 

PSD — Pat Metheny.... [son projet Orchestion, NDT] 

VG —  Pat Metheny a fait ça aussi ? … Y'a pas mal de musiciens qui ont fait des trucs comme ça... Stephan Eicher a fait des trucs du genre.. 

PSD — Oh mon dieu ! … Stephan Eicher... je suis allé cherché dans les années 90s 

VG —  Oui, il fait un truc avec Max/MSP... 

PSD — ah oui ? 

VG —  oui, je l'ai entendu interviewé à la radio, j'étais impressionné de ses connaissances là dessus... 

PSD — oh ! Par exemple (sortant son laptop), je vais te montrer, j'avais fait une patente à Mutek, où j'ai fait des “ compilations ” pour parler de ce que je viens de te parler... 

VG ...
(more transcript to come)