% !TEX root = ../thesis-example.tex
%
\chapter{Préambule}
\label{ch:preamble}

\cleanchapterquote{Il vaut toujours mieux préambuler avant de déambuler.}{John Doe}{(Marcheur matinal)}

Ce travail de recherche présente une forme quelque peu atypique et un contenu qui semblera probablement hétéroclite. Ce préambule donne quelques explications sur les motivations qui ont amené à ces résultats.

\section{Une thèse en science et musicologie}

En se situant entre les domaines relativement distincts des sciences et de l'ingénierie d'une part, et de la musicologie d'autre part, il est la tentative d'une étude des instruments de musique numérique prenant ces deux dimensions en compte. En effet, les instruments de musique ont la particularité de pouvoir s'envisager sous ces deux aspects et bien qu'il soit possible de ne s'intéresser qu'à l'un des deux, nous croyons fortement qu'une étude des motivations qui poussent à leur design ne saurait faire abstraction des conditions particulières de leur inscription dans le domaine socio-culturel.
En particulier, la performance musicale a cela de particulier qu'elle ne possède pas de cahier des charges préalables (la partition ne saurait être considérée comme telle!) et que loin de se plier à la nécessité d'exécuter une tâche précise, comme il pourrait être le cas dans le design d'autre interface homme-machine, les instruments sont des objets techniques dont les musiciens abusent (plus qu'ils en usent), dont les artefacts peuvent être appréciables et souhaitables, dont la compréhension n'est pas un préalable requis pour leur utilisation pas davantage que leur fiabilité n'est garante d'une bonne performance musicale. 
Obscur objet du désir, l'instrument est un compromis instable entre des qualités non-convergentes comme le souligne Bernard Sève \cite{seve2013a}.

Le design des DMI, ainsi que le design des outils-mêmes du luthier numérique, doivent être informés de ces particularités propres à la création artistique si l'on souhaite qu'ils se prêtent à la création de musiques nouvelles et à l'exploration de territoires sonores inexplorés.

Nécessité de prendre en compte la part expérientielle de la performance musicale, notamment dans sa dimension subversive.

Cette thèse s'offre donc comme une présentation "en coupe" d'un travail de lutherie, dans ce qu'il comporte de réflexions, de choix de matériaux, d'assemblage, de programmation, de notations et de pratiques. Il n'entend évidemment pas être exhaustif sur le sujet, ni généralisable à l'ensemble des lutheries numériques, mais permettra je l'espère d'éclairer les personnes qui s'intéressent à ce sujet.


\section{Structure de la thèse}
\label{sec:preamble:structure}

\textbf{Chapitre \ref{ch:preamble}} \\[0.2em]
Vous êtes ici.

\textbf{Chapitre \ref{ch:introduction}} \\[0.2em]
Le chapitre 2 présente un certain nombre de considérations sur les instruments de musique numériques. En particulier, la nature
éphémère de leur assemblages et les fins subversives de leur utilisation seront prises comme caractéristiques trop souvent ignorées venant influencer leur design.

\textbf{Chapitre \ref{ch:interfaces}} \\[0.2em]
Le chapitre 3 présente une interface instrumentale et retrace l'histoire de son évolution à travers plusieurs générations, partant d'un objet standard et disponible dans le commerce (la tablette graphique) et allant vers une personnalisation et un enrichissement du dispositif. 
Seront discutées les raisons président à l'ajout de tels capteurs, à l'organisation de l'espace de jeu, à la polyphonie des sources sonores, etc.

\textbf{Chapitre \ref{ch:mapping}} \\[0.2em]
Le chapitre 4 présente des développement réalisés pour la conception du "mapping" de l'instrument en tentant notamment de répondre aux problématiques soulevées dans le chapitre \ref{ch:introduction}.

\textbf{Chapitre \ref{ch:synthesis}} \\[0.2em]
Le chapitre 5 présente des développement réalisés pour la conception de la synthèse qui tentent un continuation de la logique modulaire dans le domaine audio, à travers une librairie de synthèse granulaire modulaire nommée Sagrada.

\textbf{Chapitre \ref{ch:visual_representation}} \\[0.2em]
Le chapitre 6 présente un système d'interface graphique tangible (TUI) basée sur le protocole présenté au chapitre 4, afin de permettre notamment une reconfiguration dynamique de l'interface de jeu et une représentation graphique des processus utilisés pour la performance musicale.

\textbf{Chapitre \ref{ch:notation}} \\[0.2em]
Le chapitre 7 présente des travaux portant sur les questions de notation musicale dans le domaine de la musique électroacoustique utilisant des DMI. En particulier, y est présenté "John, the semi-conductor", un logiciel permettant la génération automatique et l'édition collective de partitions minimales, utilisé dans l'ensemble d'improvisation électroacoustique ONE.

\textbf{Chapitre \ref{ch:conclusion}} \\[0.2em]
Des pistes de recherches à suivre sont enfin exposées dans la conclusion.