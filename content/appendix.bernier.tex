\chapter{Interview : Nicolas Bernier}
\label{appendix:bernier}

\section*{Biographie}

\noindent Nicolas Bernier (né en 1977) crée des performances et des installations audiovisuelles visant à sculpter un dialogue entre le son et la matière tangible. Formé par son travail dans les domaines du cinéma, de la littérature, de la danse et du théâtre, son propre langage mêle des éléments de musique, de photographie, de design, de science, d'art vidéo, d'architecture, de lumière et de scénographie. Au milieu de cet éclectisme, ses préoccupations artistiques restent constantes : l'équilibre entre le cérébral et le sensuel, entre les sources organiques et le traitement numérique.\\
\indent Lauréat du prestigieux Golden Nica au Prix Ars Electronica 2013 (Autriche), son œuvre est largement reconnue et présentée dans le monde entier : SONAR (Espagne), Mutek (Canada), Elektra (Canada), ZKM (Allemagne), Transmediale (Allemagne) et LABoral (Espagne) pour n'en citer que quelques-uns. Ses compositions sonores sont largement publiées sur les labels de musique électronique : 901 Editions (Italie), LINE (États-Unis), leerraum (Suisse), Entr'acte (Royaume-Uni) et empreintes DIGITALes (Québec).\\
\indent Il est titulaire d'un doctorat en arts sonores de l'Université de Huddersfield (Royaume-Uni). Il est membre des centres de recherche et développement en arts médiatiques Perte de signal, CIRMMMT et Hexagram basés à Montréal. Il enseigne dans le cadre du programme de musique numérique de l'Université de Montréal.

\noindent Site web : \url{http://nicolasbernier.com}

\section*{Transcript}

\noindent Nicolas Bernier, entretien du 28/05/2018, dans un café à coté l'Université de Montréal, Canada. Les termes quebécois sont traduit en français au fil du texte, à leur première occurence.
 
\noindent [Les entretiens ne sont pas disponibles publiquement.]


% VG — maintenant tout ce qu'on dit est enregistrée 

% NB — ``tout ce que vous dites peut être retenu contre vous'' 

% VG — ... avec votre accord 

% NB — que puis-je ?

% VG — alors... je vais te poser des questions assez générales, mais elles n'appellent pas du tout à une réflexion générale, c'est vraiment ton approche qui m'intéresse ... 

% NB — voyons voir si j'ai quelque chose à dire 

% VG — en premier lieu, qu'est ce qui t'a amené aux instrument numériques? est-ce que tu avais une pratique musicale acoustique avant de t'intéresser au son digital ?

% NB — oui, ça on peut dire mais pas... je viens du rock \textit{grosso-modo}... 

% VG — et qu'est ce qui t'a amené à utiliser les technologies numériques plutôt que de faire de la guitare électrique ou du piano ? c'était quoi la motivation ?

% NB — ouais, c'est quand même une bonne question ... ça remonte à quelques années quand même... tu sais faut que tu te demandes à cette époque là quand j'ai commencé, qu'est ce qui m'a amené ... je peux pas donner une réponse comme récente là... mais je pense que c'est juste \textit{l'infini des possibles} qui ... avec l'instrumental puis avec ma ... c'est une question de capacité moi je venais de la musique pop hein... donc trois accords et quelques rythmes différents mais tandis qu'avec les sons... dans le fond c'est pas tant le numérique... c'est ça, ça c'est une bonne réponse quand même, le numérique je m'en fous un peu mais les sons, les autres sont intéressants, puis quand on se met à pouvoir faire de la musique avec tous les sons... voici... ben c'est sûr que ça ouvre... tout à coup il ya plus de limites donc ça je pense c'est une des grandes motivations... 

% VG — pour autant, enfin du peu que je connais ton travail, c'est très électronique ce que tu fais... ou bien tu fais du field recording, des sons concrets je veux dire... quand tu parles de "tous les sons"

% NB — quand je parles de tous les sons, je parle de tout ce qui est pas nécessairement instrumental, puis après est-ce qu'il a des ... quel type de son j'utilise, ben justement j'utilise tous les sons, je travaille avec des sons d'instruments, j'ai travaillé du field recording, j'ai travaillé avec l'enregistrement de non-instrumentale en studio, j'ai travaille avec des sons électroniques, mais ça c'est plus récent en fait... mettons, la synthèse c'est vraiment récent, j'ai eu vraiment ... en bon enfant de l'école Schaefferienne, pour moi c'était... ce qui est tout à fait irrationnel de toute façon c'est une façon romantique de vendre la musique concrète qui serait basée sur l'enregistrement acoustique... mais dans le fond c'est juste notre façon de traduire la musique concrète parce que la musique concrète ça voulait pas dire... ça n'était pas anti-synthèse mais moi j'étais anti-synthèse... c'était un \textit{statement} ... tu sais j'étais pas de l'école allemande... Stockhausen ... j'étais de l'école ...française ...donc c'est ça, mais récemment j'ai pu me débarrasser de mes démons et puis...

% VG —  qu'est-ce qui faisait que tu étais anti-synthèse ?

% NB — ben c'est ça, je pense qu'il y a quand même ... la synthèse j'associe ça à ... et puis encore une fois c'est toujours ça, on a des biais qui sont plus ou moins justes, mais j'associais ça peut-être au contrôle absolu sur tous les paramètres. Tandis qu'avec l'enregistrement acoustique, j'ai l'impression que ça dévoilait, par transformation même simple, j'ai l'impression que ça dévoilait tout le temps dès aspects inouïs qu'on n'aurait jamais pu imaginer dans son et sur lesquels j'aurai pas nécessairement de contrôle, en tout cas...

% VG — une part d'imprévisible ?

% NB — ouais... donc voilà ... mais ... puis peut-être, je reviens à ta question, ta première question, c'est quoi ta première question ? ah oui, vers le numériques c'est ça...  donc c'est ça c'est pas tant le numérique, et moi à la rigueur faire de la musique de bande, j'aurais vraiment aimé ça, j'en ai fait une... un désastre total... (rires) ... mais j'aurais aimé ça travailler à la bande 

% VG — à la bande ... magnétique ?

% NB — avec un couteau, oui, un couteau et du papier collant... mais tu vois la raison, une autre des raisons en fait c'est ça c'est drôle parce que, moi initialement je m'intéressais à la musique contemporaine en général, sans éducation musicale, tu sais dans le fond moi c'est ça, je viens de... je vais essayer de faire l'histoire courte là... mais disons t'es adolescent, tu fais du rock, après t'arrives à Montréal, tu viens d'une région ... de banlieue, il n'y a pas grand chose qui existe, t'arrives à Montréal, tu découvres un peu l'impro, tu découvre qu'il ya d'autres sortes de rock ou ça chante pas, donc là tu penses au post-rock et puis là finalement t'incorpore un peu de jazz, t'incorpores l'impro, là tu te rends compte qu'il y a la musique répétitive qui existe, puis là tu te rends compte que la musique contemporaine existe et puis moi dans le fond, je m'intéressait *aux* musiques, au pluriel, contemporaines, et puis je voulais plus me diriger vers la musicologie, sauf qu'à un moment donné je me suis rendu compte que 1) que pour rentrer à la faculté ici, donc moi j'avais pas d'étude en musique, donc pour rentrer à l'université en électroacoustique, on n'avait pas nécessairement besoin d'un fort bagage en théorie musicale, ça prend un minimum mais c'est tout... puis ensuite je me suis dit plutôt que d'étudier la musique, cette musique là m'intriguait beaucoup, l'électroacoustique, l'acousmatique, je comprenais pas ... je comprenais pas vraiment ... tu sais, c'est un des moments marquants, quand même, de ma vie le concert acousmatique où on s'assoit, il y a personne sur la scène, t'es encore là tu viens du rock, de ta région, et puis t'arrives à Montréal et tu t'assois dans l'concert, personne sur la scène, du son partout partout, et puis le concert finit, les gens applaudissent,  tu te demandes vraiment... \textit{what the fuck} ?  Donc je me suis dit à la place de l'étudier, je vais la jouer, je vais l'apprendre et c'est comme ça que je suis rentré un peu ... donc c'est une combinaison de... ça me semblait être plus ouvert parce qu'après ça, tu sais avec tous les groupes de musique pop ben c'est sûr que si tu fais du reggae, tu fais du reggae, et puis tu fais du reggae longtemps, et puis si tu fais du ... métal, tu fais du métal longtemps, et puis si tu...  c'était difficile de sortir des carcans ... de la musique électro-quelque-chose m'a semblé plus... plus encline à faire ce qu'on veut... 

% VG — des mélanges... des collaborations?

% NB — ouais, et puis qui interdit pas la récupération d'idiomes pop ou rock non plus ... voilà, ce qui fait que ça, plus combiné au fait que c'était relativement facile d'intégrer le milieu ... c'est un peu ça qui a fait que je me suis ramassé là dedans ("se ramasser" : au Québec, "se retrouver dans un endroit sans l'avoir prévu ni voulu", NDT)...

% VG — et quand tu parlais de bandes, et de montages que tu as travaillé au ciseau tout ça, il y a un côté cinématographique là-dedans ?

% NB — Non... pas... en tout cas pas avec ... non pas tu tout en fait...  je veux dire dans mon travail il y a un côté cinématographique, j'ai commencé avec la vidéo en fait c'est plus... c'est des choses qui se sont oubliées un peu mais dans le fond toutes mes premières œuvres c'était vidéo et puis j'ai un background en design graphique en fait donc j'ai toujours été très visuel, ça a toujours été  assez important ... mais quand j'ai travaillé avec la bande, et puis même si je travaillais encore aujourd'hui avec la bande, je pense que il n'y a pas de relation avec le film en tant que tel... je penserai "sonore"...

% VG — pas de "cinéma pour l'oreille"... 

% NB — ouais, non c'est ça... non... le cinéma pour l'oreille... 

% VG — ça ne te parles pas plus que ça 

% NB — non ... mais tu sais, je trouve ça très correct, les parallèles sont super intéressants... mais après, moi je vais pas m'asseoir je sais pas trop où... quand je me mets à travailler, je me dis pas que je vais faire du cinéma pour l'oreille, je me dis pas que je vais faire rien en fait... S'inscrire dans un courant, là, je sais pas trop de quoi tu vas me parler...  J'en parlais tantôt avec avec quelqu'un, je lui disais c'est drôle cette conférence là, TENOR (conférence sur les technologie de la notation et de la représentation, NDR), où t'as des gens qui vont comme, revendiquer leur "appartenance" au monde de la partition graphique et puis ... là je me rends compte que moi je me suis ramassé, j'ai un ensemble, je sais pas si t'es au courant mais j'ai un ensemble ici sur des vieux oscillateurs des années 50... par défaut on s'est ramassés dans la partition graphique parce qu'il faut bien qu'on trouve des façons de jouer ensemble, puis de lire, puis de transmettre, puis... j'ai jamais été là-dedans mais là tout à coup avec un groupe, faut qu'on se structure un peu... donc là, tout à coup je me ramasse un peu à mon insu dans ce milieu là, il ya plein de gens qui me contactent et puis qui me disent (ton emprunté) "ah oui, toi aussi tu travailles sur la partition graphique, tu peux tu me donner... c'est quoi qui t'intéresse, et puis tu travailles sur quoi... " ... mais, moi c'est juste un moyen parce que bon faut que l'on fasse des musiques, mais ce n'est pas une fin en soi, c'est pas un intérêt plus qu'il faut... c'est juste que c'est un peu... une obligation (rire) en quelque sorte...

% VG — c'est des outils dont tu as besoin pour arriver à une finalité..

% NB — ben pour, ouais, pour faire de l'art... ce qui m'intéresse c'est l'art, c'est la seule chose qui m'intéresse ... \textit{quote} : "la seule chose qui m'intéresse c'est l'art. Nicolas Bernier, en face de l'église, 2018" (rires) 

% VG —  dans les choses qui m'intéressent dans le numérique, ce qu'il y a de particulier notamment, c'est le fait que par rapport à des instruments classiques tu as une possibilité de disruption très forte, tu peux faire des ruptures toutes les nanosecondes si tu veux, toutes les millisecondes, on va dire... et ça change un peu le rapport... 

% NB — Quoique la disruption... Là je pense en temps réel mais ... je ne sais pas si c'est un...  parce que tu sais au début, excuse moi je te laisse même pas finir ta phrase, je renchéris déjà, mais allons-y ... disruption...  parce que tu dis, bon, disruption, ça "permet" la disruption ... le numérique... bon, on dit numérique mais l'analogique le permettait déjà, de 1, donc déjà quand on utilise le numérique faut faire attention avec l'utilisation du terme, et puis de deux, je me dis ouais mais les instruments acoustiques aussi permettaient la disruption, et puis là tout à coup, sauf que tu dis ouais mais l'on peut à la milli-seconde ou à la nanoseconde, et là je me dis ah ouais ok c'est vrai, on peut peut-être pas faire ça avec des instruments acoustiques...  sauf que là si on fait des disruptions à la nanoseconde... tout à coup c'est peut-être plus de la disruption, en fait parce que pour qu'il y ait une disruption faut qu'il y ait un certain temps, ça devient, tu sais pas comme la granulation, on pourrait dire que c'est de la disruption à la nano-seconde mais dans le fond c'est plus de la disruption, c'est de la création de masses, qui elles, pour être rompues, devrait avoir un... donc en tout cas "\textit{food for thoughts}" ... "nourriture à réfléchir" peut être... je te laisse continuer... 

% VG —  peut-être que ce n'était pas un exemple très bien choisi, même si effectivement ça permet de faire ça à des fréquences qu'on ne pouvait pas faire avant, mais ce n'est pas uniquement au niveau sonore que je pense à ça mais au niveau de la relation entre le geste éventuel qui va générer un son, ou d'autres choses d'ailleurs, les outils qui permettent de contrôler le son, la lumière, la vidéo, ont tendance à fusionner un peu, dans des logiciels comme Max ou on manipule des données... les relations que tu établies du coup entre la personne ou la machine qui contrôle la musique, qui produit la musique entre le geste et le résultat, tu as une diruption possible, tu peux changer tout le mapping n'importe quand ...

% NB — ça oui, on est d'accord ... (mimant sur la table un geste très doux, et faisant subitement un bruit très saturé avec la bouche) petit côté théâtral... 

% VG — du coup ça a des conséquences...

% NB — ... sur la transmission, oui

% VG — il n'y a pas une tradition... si tu donnes des cours à la fac de musique sur les technologies numériques, il y a un contexte assez différent entre le fait d'enseigner le piano ou le violon où tu as des traditions et des techniques qui sont pérennes, en tout cas plus ou moins établies depuis des dizaines ou des centaines d'années, et des outils où il faut ré-inventer les choses à chaque fois, à la fois parce qu'elles "permettent" des relations qu'il faut re-définir à chaque fois et puis les technologies eux-même sont moins stables que le bois et le cuivre, et sujets à des mises à jour de système et des choses comme ça. C'est à la fois des contraintes et des possibilités de création, mais qu'est ce qui t'amène à utiliser des outils qui ne sont pas forcément plus stables et facile à utiliser ...

% NB — ouais, sauf tu sais, pour quelqu'un comme moi... enfin ça dépend de ta culture aussi ... mais pour moi c'est beaucoup plus facile de manipuler un outil disons informatique que ... d'apprendre le violon avant vingts ans avant de peut-être pouvoir jouer un peu correctement ... il y a une certaine facilité quand même qui vient avec...  on peut s'adapter, on peut changer, il y a pas le fardeau de centaines d'années de musique tonale... encore là... toutes ces questions là... je suis comme ça moi... je me dis qu'il n'y a pas d'absolu

% VG — et tu parlais tout à l'heure du fait de toi tu te fichais, on a évoqué ça tout à l'heure (avant l'interview NDR) de comment transmettre les oeuvres, comment faire qu'elles durent, qu'elles puissent être rejoué dans dix ans, tu disais que tu t'en fichais complètement 

% NB - ouais... voilà, c'est dit ... officiellement... (rires) non mais c'est parce que moi j'ai... c'est ça j'ai peut-être aussi parce que justement ça c'est, justement, un réflexe qui vient de la Musique, de la Musique avec un grand M, d'instruments de bois et de métal, la pérennité donc moi je suis tellement pas là-dedans, surtout que quand ... ça dépend... moi je fais surtout bon, de la performance, des installations... la performance, je l'écris pour moi, pour jouer, pour ... c'est moi qui joue... si c'est pas moi, ça se peut là qu'un jour j'écrive des performances pour d'autres personnes mais... ce n'est pas le premier réflexe en tout cas...  j'ai envie d'être sur scène..  ça fait partie de ... je suis compositeur, mais  en même temps je suis musicien pop aussi donc ... donc pour moi la performance vient avec, quand je meurs, la performance meurt avec moi... je vois pas vraiment l'intérêt que mon oeuvre soit jouée encore, interprétée par quelqu'un d'autre ... et puis ensuite les installations, ça c'est peut-être plus intéressant, mais là c'est comme un autre domaine c'est plus proche des arts visuels, tout ça... et puis j'ai une installation qui a été achetée, qui est dans une collection permanente, en France justement, et puis je me dis que quand je vais mourir ben y'a quelqu'un que lui c'est son travail dans la vie c'est de faire en sorte que cette installation là existe indépendamment de mon être et c'est \textit{correct} (en Québecquois: ``vraiment bien'', NDT).

% VG — c'est quelle installation ?

% NB — c'est tout petit, ça s'appelle "Frequencies (a / friction)"  c'est un oscillateur sur une table lumineuse, un oscillateur, un diapason... l'oscillateur est à 438Hz et le diapason est à 440 ... l'oscillateur est constant, et le diapason qui tape dessus à interval aléatoire et donc quand le diapason est activé on entend le batement... c'est tout simple ... c'est tout simple mais je l'aime beaucoup...puis voilà c'est quand même chouette ... mais non, moi j'ai pas d'ego d'artistes, il y a des artistes qui se disent je veux laisser ma marque sur la terre, et on va se rappeler, on va se souvenir de mon nom ... j'ai pas cette prétention là, je fais mon petit truc et voilà ...

% VG — mais tu donnes des cours...

% NB — ouais

% VG — ... donc d'une certaine manière tu es dans la transmission à ce niveau là... qu'est ce que tu estimes utiles ou intéressant de transmettre dans une pratique comme ça ?

% VG — oui, dans un cours où tu as beaucoup à inventer, 

% NB —  ouais - les cours que je donne c'est grosso modo des cours de composition, mais au sens élargi ... je fais des cours au bac on appelle ça des "cours-projet", c'est à dire que ton projet ça peut être la composition stricto-sensu, de la composition musicale, mais ça peut être aussi construire un instruments pour faire une performance, ça peut être de l'installation. Voilà, une des choses — c'est un peu abstrait, l'une des choses qui m'intéressent, c'est la cohérence du propos artistique, qui fait que peu importe que les outils, les méthodes, les traitements... que tout ce que tu va utiliser pour faire de l'art soit en phase avec ... un processus... des intention qui auront peut-être changé en cours de route, c'est essayer de garder un peu la cohérence dans tout ... ça c'est la première chose qui m'intéresse de transmettre, c'est un peu abstrait peut-être ?...  pas tant que ça ...

% VG — du coup, c'est pas forcément... c'est plutôt un travail de guide, d'encadrement? 

% NB — oui oui c'est ça... c'est à peu près tout ce que je fais, l'encadrement. Il y a le cours d'ensemble, ça c'est un peu différent, je me ramasse un peu avec le chapeau, un peu chef d'orchestre, en même temps pas trop chef non plus parce que toutes les partitions sont écrites, je suis plus comme un ... si je veux le dire de façon pas glamour, je suis plus comme un "coordinateur", ces trucs là, "directeur artistique"... mais sinon même mes cours, mon grand groupe, "grand" est toujours relatif mais en musique j'ai un groupe de 20 personnes à peu près, ça c'est un "grand groupe",  mais c'est quand même un cours d'initiation à la composition que j'essaie quand même d'inculquer ce dont je viens de parler, sauf que là c'est des étudiants qui entrent en  première année, qui ont vraiment peu d'expérience ou pas du tout, donc l'autre chose que j'essaie de transmettre c'est juste d'avoir... une conscience du développement du temps... c'est un peu... un classique du "compositeur" ... je pense que c'est quelque chose que je maîtrise relativement ... pas pire...(rires) la conscience du développement du temps ou de l'évolution d'énergie dans le temps... et puis je le dis comme ça parce qu'encore une fois l'évolution de l'énergie dans le temps, le développement du temps, ça s'applique pas juste à la musique, ça s'applique... à tout... ça s'applique à la façon dont je suis en train de te parler et puis je vais mettre l'emphase à un moment donné, je vais prendre le temps quelque part d'autre, ça c'est quelque chose qui m'intéresse, que je transmets... qui m'intéresse mais cela dit, sur laquelle j'ai jamais réfléchi, ou j'ai jamais été... si tu me demandes "oui  mais c'est quoi ta conception de..." j'en ai aucune idée, c'est quelque chose que ... que je sens... grosso modo... puis après ça bon c'est sûr qu'à un niveau, euh... ce cours là disons de groupe de composition son, c'est sûr qu'il y a quand même des choses techniques c'est la partie qui m'intéresse pas en fait... mais on va parler d'espace, on va parler de montage, on va parler de filtrage, on va parler de toutes ces choses qui vont aider éventuellement à développer ton temps comme tu veux... mais c'est pas ...

% VG — tu n'attaches pas plus d'importance à ça...

% NB — non

% VG — une question par rapport aux objets techniques, est ce qu'il y a des outils que tu utilises de manière récurrente, et s'il y en a pourquoi, qu'est ce qui t'intéresse dans ces outils et qui fait qu'ils reviennent dans ton travail ?

% NB — mouais, mon petit côté "baveux" (arrogant, méprisant en Québecois, NdT) aimerait répondre "non" à cette question ... mais bon forcément on utilise... mais pour vrai, moi j'utilise le moins d'outils possible ... je te disais tantôt que j'ai un passé en design graphique un peu, mais j'ai un passé aussi en programmation pour le web, c'est fin des années 90... à l'époque c'était ASP (cf. \gls{ASP}) qu'on faisait... ASP et SQL (cf. \gls{SQL}) ... et puis donc quand j'ai commencé à travailler sur le marché professionnel à 17 ou 18 ans dans un bureau avec un salaire... et puis là, ) un moment donné tu te dis bon, t'as 22 ans ça fait déjà cinq ans que tu travailles, tu te dis je vais pas faire ce jusqu'à ma mort... ça n'a pas de bon sens ... donc làj'ai décidé de retourner en musique, faire de l'art... pourquoi je dis ça?... ah oui! par rapport aux outils... qui qu'on programme beaucoup en musique maintenant ou dans les arts numériques, et puis moi c'est une chose que j'ai pas particulièrement le goût de faire, j'ai quitté un milieu pas pour... bon ça m'a aidé dans mes études j'avais de l'aisance là dedans plus que d'autres gens, sauf que je suis pas là pour ça tu sais... comme je disais tantôt ce qui m'intéresse, c'est la finalité qui m'intéresse, c'est pour ça que je vais engager des gens pour faire les choses, c'est une culture qu'on n'a pas beaucoup en tout cas à Montréal mais en France plus, tout l'assistanat musical là, nous autres ici c'est quelque chose qui n'existe pas vraiment, on est vraiment une culture \gls{DIY}, on fait tout ... il y a quelque chose d'un peu macho là dedans, tu sais, si t'as pas fait tout, si t'as pas ...  c'est pas "authentique", tu sais... mais moi je suis pas vraiment là dedans parce que au final moi, j'aimes ça jouer... j'aime ça entendre les choses mais je ne tiens pas à avoir fait la mécanique en arrière, bon en même temps j'en fais une bonne partie quand même parce que mes moyens sont limités ... tout ça pour dire que... pouvez-vous répéter la question monsieur? (rires)

% VG — je te demandais s'il y avait des outils qui revenait de manière récurrent...

% NB — ah ouais, c'est ça...  donc longue histoire pour dire que j'utilise le moins d'outils possible ... puis moi depuis plusieurs années ma plateforme c'est Live (Ableton Live, NDR), pourquoi Live? entre autres parce qu'il y a Max for Live dedans puis moi avec ces deux choses là, écoute, j'ai pas besoin de beaucoup plus... après... 

% VG — c'est la rapidité du fait d'arriver à tes fins qui est la motivation principale ?

% NB — ouais, puis de pouvoir faire un maximum de choses en changeant le moins, en ne changeant pas d'environnement constamment... ça évite des bugs je pense, tu sais, le fait de pas avoir untel qui communique avec untel, qui communique avec untel, avec untel qui revient à untel et puis... tout est dans la même fenêtre, tout est... ce qui fait que moi, ça me convient vraiment ... puis après, bon ça c'est un peu  les outils de base disons, mais après c'est sûr que chaque projet est quand même différent donc il va toujours y avoir... mais tu sais, je vais te donner un exemple où je n'ai pas fait... où je me suis pas écouté dans un projet récent, un gros projet, bon j'ai travaillé beaucoup son et lumière ces dernières années, et puis je travaille avec un microcontrôleur, que tu peux même pas dire le nom, je sais même pas si ça a un nom,  je suis tombé là dessus par hasard un moment donné dans mes recherches il y a dix ans, un gars dans un sous-sol aux États-Unis qui fait des petites cartes... à l'époque j'étais en train de travailler avec des dimmer-packs... je dis dimmer-pack, tu vois c'est quoi ? je ne sais pas le terme en français ...  les trucs pour les gradateurs de lumière, normalement t'as des \textit{shovel} où tu peux mettre quatre ampoules puis tu peux envoyer du DMX et contrôler ton éclairage... tu vois c'est quoi cette boite là, ce qu'il y a dans les théâtre pour contrôler l'éclairage ... tout ça pour dire que là j'étais avec ça et puis je me suis rendu compte, ça faisait pas très longtemps qu'on travaille avec les LEDS et puis à un moment donné, je tombe sur ce micro-contrôleur là, qui à la place de peser 10 livres, faire 4 canaux c'est juste une carte ça fait trente deux canaux, puis ça pèse rien, puis donc j'ai commencé à travailler avec ça, puis là tous mes projets sont construits sur cette carte là que je connais, avec un objet ... puis là c'est l'autre affaire, tu sais ya un objet dans Max qui communique avec machin et puis là est-ce que l'interface DMX, USB-DMX qui parle à l' objet qui ... (soupirs) ... là j'ai une formule puis bon il ya d'autres choses qui rentrent en ligne de compte moi j'ai besoin d'une rapidité à toute épreuve et puis les interfaces commerciales, style ENTTEC, qui sont assez connues dans le monde de l'éclairage semi-professionnel, ben ça va pas assez vite il y a des dropped-frames tu sais ça perd des images, et ça me convient pas, je suis tombé un peu par chance sur cette formule là, qui fonctionne super bien pour mon usage, donc la plupart de mes projets sont construits là dessus. Quand je pars sur un gros projet j'ai pas besoin de me dire/il y a beaucoup de choses qui/chaque projet est différent, moi la partie compliquée c'est la partie mécanique, tu sais la partie, euh...  j'ai besoin d'un parasol, telle couleur, tel matériau ... ça je trouve ça compliqué et puis la partie design industriel ... j'apprends, je connais rien là dedans et puis j'ai quand même pas le choix de ... la partie technique, technologique un peu, qui se renouvelle tout le temps c'est plus ça tu vois... la partie "outil" parce que je construis mes outils... c'est ça ... qu'est ce qu'on entend par "outil"... mes outils c'est toujours grosso-modo les mêmes, mais mon "dispositif" il est jamais pareil ...  ce qui fait que là je réutilise, je sais pas si tu me suis dans mon histoire, je réutilise tout le temps le même truc sauf que le dernier projet, gros projet, je me fais convaincre par des jeunes... trop ambitieux... que je dois changer tous mes outils... parce que ... X Y raisons ... ``ok ... ok ... on va tout changer'' ... mais là au final il ya des bugs, ya des machins qui communiquent pas avec d'autres machins, puis d'autres machins qui communiquent pas, puis là y'a trop de data, et puis on perd du data, puis là gna gna gna... Au final on finit par réussir à faire un prototype de peine et de misère qui annoncerait un projet somme toute quand même assez intéressant, sauf qu'on en parlait tantôt, faut que j'aille chercher les sous pour aller... puisqu'on a réinventé les outils, l'argent est parti dans les outils et pendant l'art ... parce que je te dis, moi ce qui m'intéresse c'est l'art...  puis là je me ramasse avec un prototype... pas d'art ...puis faudrait que j'aille chercher des sous encore pour faire l'art, puis je réussi pas, je sais pas pourquoi, j'ai pas le don, les gens n'aiment pas le projet, je sais pas trop...


% VG — peut-être parce que tu ne demandes pas des sous pour du matériel, peut-être

% NB — ouais c'est ça... ben oui, voilà, c'est ça... exactement... tu as mis le doigt sur quelque chose...

% VG — un problème récurrent oui...

% NB — donc ce projet là, ben ... poubelle ... et puis j'ai travaillé fort longtemps et comme je te dis, c'est pour avoir un prototype quand même assez chouette et puis là finalement, j'ai juste fait comme... arf ... poubelle... et puis tous ces processus sont assez longs... tu fais une demande de subventions, tu reçois la réponse un an après ... le projet démarre six mois après... tu travail pendant un an ... ce qui fait que ça veut dire qu'une oeuvre, s'il faut que tu fasses un protoype, ça prend deux ans et demi, ça veut dire que t'es cinq ans sur le même truc... moi je peux pas travailler comme cela... parce que je suis rendu après cinq ans ça ne m'intéresse plus en fait, mon projet m'intéresse plus me ... donc ça répond un peu la question, peut-être ? 

% VG — oui, d'une manière tout à fait intéressante...

% NB — ok tant mieux ....

% VG — c'est moi qui ait perdu le fil de mes questions du coup ... 

% NB — mais je peux peut-être te relancer en fait, je serai peut-être curieux, parce que par rapport à ce que je te dit, tu sais moi dans le fond les outils c'est pas tant ça la question, qui est plus ... dans le fond moi, c'est comme la scénographie, quelque part... moi c'est ça, je dis ``scénographie'' mais le dispositif, l'objet ...  parce que je crée des objets, je ne créé pas des instruments ... je crée des objets, c'est plus comme de... je pense plus proche de la sculpture que de la musique... 

% VG — c'est des objets dont tu joues quand même un peu en live...tu fais des performances avec...

% NB — ben c'est des objets, ouais c'est ça, ça va être ....  je sais pas c'est quelque chose comme .... un dispositif, mettons pour simplifier, de visualisation ou ... c'est ça, ça peut se rapprocher un peu d'un décor de théâtre, des fois ça peut être un instrument de musique, des fois ça peut être un peu tout ça...  mais pour moi c'est ça qui est compliqué... c'est à dire il faut que j'habite une scène, que j'habite un espace... moi j'ai une pratique grosso modo solo, j'ai pas une équipe de concepteurs qui vont me proposer ``ah! voici on a pensé travailler l'aluminium avec euh... ''. Il faut que j'essaye d'imaginer tout ça moi-même et c'est vraiment pas évident. C'est faire, tu sais aller voir les designers industriels, faire des plans, dire ``ok j'ai besoin d'eux un tube d'aluminium de cette grandeur, avec tel genre d'ancrage, qui va prendre tel genre de boulons, qui va se ranger dans telle genre de caisse, qui va pas peser plus que tant de livres, ou de  machin, ou de...'' ce qui fait que tout ça... et puis en plus, faut que je fasse de l'art avec ça... ça devient compliqué... j'ai un bel objet mais là qu'est-ce-que je fais avec ?... Ça m'a coûté cher à faire, il faut bien que je trouve quelque chose à faire avec ça. Je me suis mis une hypothèse qui me disait que je ne trouverai quelque chose à faire... Est ce que ça va fonctionner?... Je ne sais pas tant que l'objet existe pas. Et puis comme j'ai pas des budgets de recherche et développement qui font que je peux flamber de l'argent et puis dire ``ah oui j'essaye des choses''  mais c'est... ça passe ou ça casse tout le temps... 

% VG — tu fais beaucoup les choses toi même ?

% NB —  ben comment dire... je les imagine moi-même... après je coupe pas l'aluminium moi-même et je fais pas de travail manuel moi-même. La programmation, je vais en faire partie quand c'est trop ... moi d'habitude la programmation c'est vraiment plus, euh .... pour moi c'est très utilitaire... je fais pas des ``créations programmatiques'' tu sais, j'ai besoin que ça fasse ``ça'', et puis ça, ça va être une étape quand même assez simple, ok je vais le faire. Quand ça deviennt plus complexe, je m'embarque plus là dedans... ça m'intéresse pas, et puis j'ai pu le temps de toute façon. Ce qui fait que ça je vais engager quelqu'un pour le faire... J'ai pas engagé tant de monde que ça pour faire la programmation de ce projet là que j'ai mis à la poubelle... Il y en a un autre que j'ai mis à là poubelle. Et puis il y en a un que j'utilise ... cette application pour, justement, de suivi partitions graphiques. Ça je l'ai faite faire. C'est à peu près tout. Le reste, tu sais j'ai besoin de déclencher des choses, de lire telle donnée...ça je m'arrange, grosso-modo...

% VG — par rapport à ce que tu disais tout à l'heure, du fait qu'avec la programmation, le numérique, tous ces outils hardware ou software où tu vas, enfin ou tu peux potentiellement interconnecter plein de choses pour faire...  et qui donnent des projets qui sont lourds à monter... parce que du coup, en voyant ça d'une manière optimiste et naïve on peut se dire que justement ça permet de faire une écriture de la métamorphose, parce que tu peux avoir un processus qui se transforme complètement durant le cours de l'installation, mais quelque part le fait que ce soit ... plus lourd ... en fait cette complexité et cette lourdeur de la programmation du coup, est-ce que tu as l'impression que ça influe la manière dont tu vas concevoir tes œuvres? 
% Je veux dire, est ce que tu fais une œuvre très directe par rapport au fait que c'est justement, lourd techniquement, au fait que ça prend du temps

% NB — ouais je pense que ... je pense qu'il n'y a pas nécessairement de corrélat à faire, enfin ça va dépendre de chaque projet... des fois oui, des fois tu dis bon ben là... mais des fois ... comme là, tu viens de parler, je sais pas comment t'as dit, t'as parlé d'écriture de la métamorphose? comme un peu une écriture algorithmiques? c'est ça que tu entends un peu par là ?  une forme qui va se développer, euh...? 

% VG — on peut faire des choses interactive qui aient un scénario qui se développe dans le temps 

% NB — ouais, imprévu tu veux dire ?

% VG — qu'elles soient prévues ou scénarisées 

% NB — ok 

% VG — mais avec ces outils qui permettent, dans un ordinateur au sens large du terme, tu peux mettre beaucoup de mémoire et du coup ça permet potentiellement de faire des choses qui vont avoir une durée de performance où les choses se renouvellent changent et... ça permet potentiellement de le faire, mais peut-être que de le faire, ça génère des projets qui peuvent être plus lourds techniquement et longs à monter... et pour garder une certaine fraîcheur et que tu puisse arriver à ton idée avant que cinq ans soient passés, est-ce que ça a influencé ta manière de travailler ?

% NB —  je sais pas, je n'ai... ça ne m'a jamais vraiment posé de problème mais peut-être que je ...  mais effectivement je vais essayer de forcer toujours, de toute façon, dans tout les projets, une certaine simplicité. Tu sais, j'essaie de ... Je veux dire, de toute façon un projet ça fonctionne tout le temps pareil, il y a tout le temps, bon t'as plein de belles idées, et puis il y a tout le temps un moment donné où tu te rends compte que tout ça n'est pas réalisable, et puis là faut tout refaire... et cette prise de conscience de ... ou cette façon de travailler, cette zone là, où il faut réorganiser les idées de départ, il y a des gens qui vont pousser la frontière le plus loin possible, tu sais ils vont essayer de faire le truc impossible jusqu'à la dernière seconde, et puis peut-être qu'à la toute fin ils vont réussir à faire ce qui était impossible ou peut-être qu'ils vont juste revenir à la case départ ... moi j'ai tendance à rapidement au plus simple. Tu vois j'ai plein d'idées et puis à un moment, qui fait que peut-être j'ai jamais eu, euh... Je me suis jamais rendu à cette complexité là, où je me suis senti un peu menotté par la complexité du projet, mais par exemple un projet avec Martin Messier, qui est un genre de... on peut appeler ça une machine ... une structure en bois, c'est assez haut, je sais pas si tu vois, c'est seize pieds de haut, seize pieds de large... je peux pas te dire en mettre mais... (me montrant dans l'espace ce que cela représente) c'est un instrument assez imposant, qu'on ne peut pas sortir dans notre chambre à coucher et puis se mettre à jouer ... donc un projet comme ça, c'est compliqué mais c'est pas tant au niveau des outils qu'au niveau de... si on veut répéter ce spectacle là, il faut louer une salle, idéalement faut qu'il y ait de l'éclairage donc idéalement il faut qu'il y ait un technicien... donc là si tu veux, et puis nous on n'a pas les moyens d'une compagnie de théâtre, on a les moyens/on partage... on est dans la ``catégorie musiciens'' donc un musicien, ou un artiste disons numérique, ça s'organise tout seul, tu sais...  nos budgets sont pas cinq fois plus gros parce qu'on a un instrument qui demande ça... donc ça c'est compliqué, et puis ça je pense que c'est un frein... c'est un projet duquel je suis plus ou moins content artistiquement d'ailleurs ... et puis d'après moi c'est, entre autres, parce qu'il était tellement complexe à gérer et puis là tout seul, faut que ça ce range. Tu sais ça va dans des caisses, des grosses caisses qu'il faut que t'entrepose, et puis quand tu l'entrepose ça coûte de l'argent ... Ça c'est un frein à l'art. Et puis tu vois, Martin, lui ç'en est un, c'est un peu à lui que je pensais, lui va pousser la limite le plus loin possible. Dans ce projet là que j'ai fait avec lui, ça fait longtemps que j'aurais fait comme ``ok, regarde, notre machine, elle va être deux fois plus petite et puis on va pouvoir la faire dans mon petit local, puis au moins, ça va être moins imposant visuellement, mais au moins on va pouvoir plus travailler avec, et puis apprendre à jouer notre instrument, et puis travailler le mapping, et puis travailler la composition, et puis travailler la... '' Mais lui... mais c'est un projet collaboratif ...  par définition on fait des personnes pas heureuses... (rires) Ce qui fait que lui a du mettre de l'eau dans son vin, et puis moi aussi... pour découvrir quelque chose qui est ni lui ni moi... Ce qui fait que c'est peut-être plus à ce niveau là, je dirais que je sens où la complexité des projets, à un moment donné, va avoir un frein, mais... en fait je dis ça, j'ai donné cet exemple là parce que c'est un peu le plus... pour moi, je te montrerai une photo tantôt, c'est le plus flagrant... mais tous mes projets c'est comme ça en fait... Chaque projet, comme là, tous mes projets, c'est tout des dispositifs sur lesquels je peux jamais répéter... qui sont tous un peu gros, ou tous entreposés euh... parce que j'ai pas les moyens, ça coûte cher à envoyer outre-mer...  donc la plupart de mes projets, je les fais une fois à Montréal, après je l'envoie en Europe, puis ça reste en Europe... j'essaie de... je le fais entreposer à gauche à droite pour pas avoir à payer le voyage du retour. Donc ça qu'est ce que ça veut dire, ça veut dire que je peux pas travailler, puis là évidemment, quand je le fais, disons la première fois à Montréal mais c'est tout le temps... c'est une première, puis une première, ben c'est jamais une version finale, ce qui fait que là ça veut dire que je peux pas répéter avec mes dispositifs... ça pour moi c'est un gros gros gros problème en fait... 

% VG — qui conditionne la manière dont tu travailles plus que ce dont je parlais avant ?

% NB — ouais ...  je sais pas si ça conditionne la manière dont je travaille mais en tout cas ça... ça met des freins dans le processus. Et puis à l'achèvement de mon travail. Contrairement à une compagnie de danse ou de théâtre, qui ont des budgets pour avoir un lieu, des techniciens... Martin Messier, lui a sa compagnie maintenant et puis il fait ça, quand il travaille sur un projet, il peut se permettre de louer une salle et puis de vraiment ... d'expérimenter son projet en salle  et puis de faire le plan d'éclairage, et puis de savoir comment ça va se passer pour de vrai. Mon dernier projet, c'est drôle j'en parlais avec un diffuseur, en espagne récemment, et puis qui me disait ``mais comment...'' —parce tu sais moi j'ai de l'éclairage un peu— et il me disait ``mais comme tu fais pour faire ça ??'' parce que lui fait de la musique électronique, il a son studio mais y'a pas nécessairement de positif physique qui va avec son travail... et je lui disais, écoute —comme le projet que j'ai fait dans son festival— je lui disais je l'ai monté, c'est comme trois murs à peu près de cette largeur là (me montrant avec ses mains), qui sont l'un en arrière de l'autre au début, et puis dans la performance, je vais ouvrir les murs comme ça, ce qui fait qu'à à la fin ils vont être tous placés comme sur une ligne, ce qui fait qu'au début, c'est juste large comme ça, mais à la fin c'est quand même 20 pieds de large, donc ça prend une certaine largeur, que j'ai pas dans mon appartement... donc moi je croise les doigts en me disant que ça va fonctionner, j'espère que ça fonctionne une fois sur place... Ça c'est une chose, et puis après, une fois sur place je sais que je suis dans des théâtres, je me dis ah, bon, donc je vais essayer de mettre une lumière, tel type, là, je parle avec les techniciens, et tel type de couleurs, peux tu m'essayer un spot ici et un spot là, et puis ... ce qui fait qu'à chaque représentation j'améliore des choses, j'ajoute une lumière, j'en ferme une là, je change un peu la disposition ... et puis à un moment donné je finis par avoir quelque chose de satisfaisant mais je le travaille en live dans mes répétitions, juste avant que le public entre en salle et puis c'est comme ça que je finis par concevoir mes chose parce que j'ai pas le luxe d'avoir un lieu... 

% VG — je te pose une dernière question, en totale disruption, qui est peut-être une boîte de pandore aussi qui est la question d'Internet, du fait que ... j'ai interviewé Nick Collins qui me disait qu'avec l'accessibilité des outils numériques sur Internet, faire de la lutherie numérique ressemblait davantage à faire de la cuisine. Il prenait l'exemple d'un de ses étudiants qui tape sur google ``asservir un moteur'', qui télécharge le fichier arduino, et ça marche... Comment considères tu ça, et notamment le fait qu'on peut travailler très facilement en faisant du copié/collé, où tu récupères le travail de quelqu'un d'autre qui n'est pas forcément ton idée originale, donc on travailles avec quelque chose qui a été pensé, orienté d'une certaine manière par quelqu'un d'autre... est ce que ça ré-affecte le résultat de ce que tu fais ?

% NB — c'est sûr que ça affecte le résultat mais en même temps, c'est l'histoire de l'humanité ... On ne fait que ça, du copier-coller

% VG — et de s'influencer les uns les autres...

% NB — ben oui ... tant mieux si c'est plus facile ... après est ce que c'est garant d'un intérêt de ton œuvre, évidemment non. Si tu prends une pièce pour guitare de tel compositeur célèbre et puis que tu la recopies... tant mieux, on peut progresser là-dedans...  est ce que c'est que garant de l'intérêt de l'œuvre pour guitare, ben non, ben c'est la même chose avec l'électronique.  Après c'est au public, ou aux pairs, ou un mélange de ces deux choses là de dire, de voir s'il y a un intérêt là-dedans, d'une part... Ou sinon, si la personne qui le fait grandi là-dedans, tant mieux ... mais je sais pas... Nick Collins voyait ça d'un mauvais oeil ? 

% VG — c'était pas forcément un jugement positif ou négatif

% NB — plutôt un constat ?

% VG — c'était plutôt positif, en tout cas lui vient d'une période où il n'avait pas toute la documentation qu'il y a et il a du créer des choses en faisant du reverse engineering...

% NB — ouais ça m'aurait étonné, parce que c'est sûr il ya des gens qui pourraient avoir un discours un peu péjoratif par rapport à ça... mais Nick Collins, il me semble, c'est son but justement que ça devienne accessible, l'électronique, et je me dis il devrait être content, il devrait se réjouir...

% VG — oui, c'était plutôt positif

% NB — l'autre truc positif, c'est que si on n'a pas à apprendre la mécanique, si la mécanique se fait toute seul en tapant dans google, mais ça veut dire qu'on peut faire de l'art peut-être plus rapidement, et ça veut dire qu'on peut peut-être passer moins de temps à parler des outils, et puis qu'on peut passer peut-être plus de temps à parler de qu'est ce que tu aurais fait avec ces outils là, de forme, de temps, de matières, des choix, des intentions, qui pour moi — c'est personnel—  je dis ça parce que je me dis peut-être que lui il se dit ben là ça m'intéresse pas tes affaires, de forme et puis de matières... mais je pense que c'est une bonne nouvelle si on peut parler plus de l'art plutôt que de demander ``mais comment c'est fait?...'' Je suis pas obligé de te demander comment c'est fait, je comprends tout maintenant, je sais, c'est pas compliqué...  t'as branché un moteur dans un arduino avec une ligne de code qui fait \textit{random} sur le temps, machin... Mais qu'est ce que tu voulais dire avec ça? Ç'est la discussion qui me semble plus enrichissante. 

% VG — Pour faire l'avocat du diable, j'aurais tendance à dire qu'Internet a des bons côtés et des mauvais côté, mais je ne sais pas si elle favorise un esprit geek, mais elle ne l'empêche pas en tout cas... c'était aussi par rapport à ce biais là...

% NB — ouais... favorise un truc geek ..?...

% VG — il y a plus de discussions peut-être autour des objets techniques et des plateformes...

% NB — ouais, je sais pas, j'ai 40 ans, j'ai vécu 40 ans sur cette planète, est ce que ça n'a pas toujours été là les discussions techniques? Je veux dire, on se retrouve autour d'une auto puis on parle de la mécanique, et puis on part du moteur, et puis on parle du...  on est un peu fait comme ça je pense... A défaut de parler de l'esthétique... ``Pourquoi as tu mis cet aileron? Pourquoi ce type d'aileron?'' (ton emphatique) ``— c'est pour la métaphore de... (rires)''

% il y a une facilité des outils qui est là avec Internet... Ça rend la recette facile.. C'est un peu la même chose que dire tout le monde peut faire de la musique, tout le monde peut faire de la vidéo, tout le monde peut faire de la photo...  Est ce que ça veut dire que parce que tout le monde peut faire de la photo avec son téléphone, que toutes les photos qui se font ont la même valeur? ... Pas vraiment...

% VG — oui... il y en a plus en quantité...

% NB — L'histoire va connaître celles qui ont vraiment de l'intérêt...

% VG — Et si je peux te demander un pronostic, sur ces instruments numériques qui n'ont pas l'équivalent dans leur durée historique des instruments acoustiques, est ce que tu penses, souhaites, ou ne souhaites pas que les instruments numériques vont se cristalliser dans des formes particulières, comme pour les instruments acoustiques qui se sont établis de manière assez stable ...

% NB — ouais... est-ce ce que je souhaite... je pense pas que je le souhaite... mais tu sais... j'en ai rien à cirer....  il va sûrement avoir des instrument, c'est sûr qu'il va y avoir des instruments numériques qui vont se standardiser... C'est aussi bête que les controleurs à boutons... qu'est ce qu'un contrôleur, qu'est ce qu'un instrument .... un contrôleur avec un ordinateur, ça va se standardiser, si on prend ça pour acquis, on le considère même pas comme un instrument...  mais quelque part ... j'ai des boutons rotatifs, des push buttons puis c'est relié à des paramètres, si ça c'est standardisé, tant mieux ... c'est sûr qu'il va en avoir, et puis il y aura sûrement des trucs un peu fou quand même un jour qui vont être inventés et puis qui vont êtres normaux et puis qu'on va jouer ... ouais, c'est cool... tant mieux....  mais j'espère en fait qu'il va continuer à y avoir tout ce pan là d'inventions qui vont être éphémères peut-être... Encore une fois, je ne vais pas dans le sens du développement d'un instrument qui va être utlisé, ma pensée va vers créer un objet d'art, et pour créer cet objet d'art là, j'ai peut-être besoin d'instruments qui existe mais j'ai peut-être besoin d'instruments qui n'existe pas puis le projet d'après ben j'aurai sûrement pas le goût de réutiliser le même instrument parce que ça va être une autre œuvre d'art, donc ça veut dire que ça va être des idées différentes, avec un objectif différent, et puis ça j'espère que ça va continuer à exister... Quelque part c'est un peu... euh... un peu moche la période du synthétiseur où tout le monde, si tu faisais de la musique électronique, tu faisais du synthétiseur. C'était la seule option, oui y'avait plein de sortes de synthétiseurs, mais ... c'est cool en ce moment qu tu te dis je peux faire ce que je veux... et puis en même temps cette époque là on l'a déjà vécu, la révolution industrielle, ils inventaient des instruments... sans arrêt... du côté de la musique on en inventait... tu sais je suis un peu là dedans...les vieux livres, le matériel scientifique, ils inventaient plein de... c'est fou, ça ressemblait quand même drôlement notre époque d'aujourd'hui où on avait l'impression qu'on pouvait tout inventer...  ça je trouve ça cool... on voit que la boucle, ça revient au début de la conversation quand tu me demandais pourquoi je faisais ça... parce que tout est possible... Il me semble que j'aimerais garder ça... garder cet aspect là... de sentir que tu tout est possible parce que si ça se perd et puis on commercialise, et que tous les instruments sont pareils ... ``not my cup of tea''... 

% VG — La biodiversité de la technologie ...

% NB — ouais...tu vois quand tant de forme —tantôt je te parlais de forme— l'infini des possibles... là c'est \textit{basic}, on finit par le plus ennuyant de la forme, mais dans une conversation c'est peut-être un peu moins ennuyant que musicalement... Je suis bon là dedans à créer de la forme...(rires) quand on retourne au point de départ... voilà... y a quelque chose à faire avec ça ?



