\chapter{Interview : Nicolas Bernier}
\label{appendix:bernier}

\section*{Biographie}

Nicolas Bernier creates audiovisual performances and installations aiming to carve a dialogue between sound and tangible matter. Shaped by his work within the fields of cinema, literature, dance and theatre companies, his own language blend together elements of music, photography, design, science, video art, architecture, light design and scenography. In the midst of this eclecticism, his artistic concerns remain constant: the balance between the cerebral and the sensual, and between organic sources and digital processing.

Awardee of the prestigious Golden Nica at Prix Ars Electronica 2013 (Austria), his work widely recognize, presented all over the world: SONAR (Spain), Mutek (Canada), Elektra (Canada), ZKM (Germany), Transmediale (Germany) and LABoral (Spain) to name a few. His sound compositions are widely published on electronic music labels: 901 Editions (Italy), LINE (US), leerraum (Switzerland), Entr’acte (UK) and empreintes DIGITALes (Québec).

He holds a PhD in sonic arts from the University of Huddersfield (UK). He is a member of Perte de signal, CIRMMT and Hexagram media arts research and development centres based in Montreal. He is teaching in the Digital Music program of the Université de Montréal. 


\section*{Transcript}

Nicolas Bernier, interview du 28/05/2018, à l'Université de Montréal, Canada.
Les termes quebécois sont traduit en français au fil du texte, à leur première occurence.
 
VG — maintenant tout ce qu'on dit est enregistrée 

NB — ``tout ce que vous dites peut être retenu contre vous'' 

VG — ... avec votre accord 

NB — que puis-je ?

VG — alors... je vais te poser des questions assez générales, mais elles n'appellent pas du tout à une réflexion générale, c'est vraiment ton approche qui m'intéresse ... mais tu peux partir sur des théories générales sur la musique ...

NB — voyons voir si j'ai quelque chose à dire 

VG — en premier lieu, qu'est ce qui t'a amené aux instrument numériques? est-ce que tu avais une pratique musicale acoustique avant de t'intéresser au son digital ?

NB — oui, ça on peut dire mais pas... je viens du rock grosso modo... 

VG — et qu'est ce qui t'a amené à utiliser les technologies numériques plutôt que de faire de la guitare électrique ou du piano ? c'était quoi la motivation ?

NB — ouais, c'est quand même une bonne question ... on va laisser passer les voitures pour qu'on entende la réponse ... c'est une bonne question, ça remonte à quelques années quand même... tu sais faut que tu te demandes à cette époque là quand j'ai commencé, qu'est ce qui m'a amené ... je peux pas donner une réponse comme récente là... mais je pense que c'est juste l'infini des possibles qui ... avec l'instrumental puis avec ma ... c'est une question de capacité moi je venais de la musique pop hein... donc trois accords et quelques rythmes différents mais tandis qu'avec les sons... dans le fond c'est pas tant le numérique... c'est ça, ça c'est une bonne réponse quand même, le numérique je m'en fous un peu mais les sons, les autres sont intéressants, puis quand on se met à pouvoir faire de la musique avec tous les sons... voici... ben c'est sûr que ça ouvre... tout à coup il ya plus de limites donc ça je pense c'est une des grandes motivations... 

VG — pour autant, enfin du peu que je connais ton travail, c'est très électronique ce que tu fais... ou bien tu fais du field recording, des sons concrets je veux dire... quand tu parles de "tous les sons"

NB — quand je parles de tous les sons, je parle de tout ce qui est pas nécessairement instrumental, puis après est-ce qu'il a des ... quel type de son j'utilise, ben justement j'utilise tous les sons, je travaille avec des sons d'instruments, j'ai travaillé du field recording, j'ai travaillé avec l'enregistrement de non-instrumentale en studio, j'ai travaille avec des sons électroniques, mais ça c'est plus récent en fait... mettons, la synthèse c'est vraiment récent, j'ai eu vraiment ... en bon enfant de l'école Schaefferienne, pour moi c'était... ce qui est tout à fait irrationnel de toute façon c'est une façon romantique de vendre la musique concrète qui serait basée sur l'enregistrement acoustique... mais dans le fond c'est juste notre façon de traduire la musique concrète parce que la musique concrète ça voulait pas dire... ça n'était pas anti-synthèse mais moi j'étais anti-synthèse... c'était un statement ... tu sais j'étais pas de l'école allemande... Stockhausen ... j'étais de l'école ...française ...donc c'est ça, mais récemment j'ai pu me débarrasser de mes démons et puis...

VG —  qu'est-ce qui faisait que tu étais anti-synthèse ?

NB — ben c'est ça, je pense qu'il y a quand même ... la synthèse j'associe ça à ... et puis encore une fois c'est toujours ça, on a des biais qui sont plus ou moins justes, mais j'associais ça peut-être au contrôle absolu sur tous les paramètres. Tandis qu'avec l'enregistrement acoustique, j'ai l'impression que ça dévoilait, par transformation même simple, j'ai l'impression que ça dévoilait tout le temps dès aspects inouïs qu'on n'aurait jamais pu imaginer dans son et sur lesquels j'aurai pas nécessairement de contrôle, en tout cas...

VG — une part d'imprévisible ?

NB — ouais... donc voilà ... mais ... puis peut-être, je reviens à ta question, ta première question, c'est quoi ta première question ? ah oui, vers le numériques c'est ça...  donc c'est ça c'est pas tant le numérique, et moi à la rigueur faire de la musique de bande, j'aurais vraiment aimé ça, j'en ai fait une... un désastre total... (rires) ... mais j'aurais aimé ça travailler à la bande 

VG — à la ... ? avec un groupe ? à la bande ... magnétique ?

NB — avec un couteau, oui, un couteau et du papier collant... mais tu vois la raison, une autre des raisons en fait c'est ça c'est drôle parce que, moi initialement je m'intéressais à la musique contemporaine en général, sans éducation musicale, t'sais dans le fond moi c'est ça, je viens de... je vais essayer de faire l'histoire courte là... mais disons t'es adolescent, tu fais du rock, après t'arrives à Montréal, tu viens d'une région ... de banlieue, il n'y a pas grand chose qui existe, t'arrives à Montréal, tu découvres un peu l'impro, tu découvre qu'il ya d'autres sortes de rock ou ça chante pas, donc là tu penses au post-rock et puis là finalement t'incorpore un peu de jazz, t'incorpores l'impro, là tu te rends compte qu'il y a la musique répétitive qui existe, puis là tu te rends compte que la musique contemporaine existe et puis moi dans le fond, je m'intéressait *aux* musiques, au pluriel, contemporaines, et puis je voulais plus me diriger vers la musicologie, sauf qu'à un moment donné je me suis rendu compte que 1) que pour rentrer à la faculté ici, donc moi j'avais pas d'étude en musique, donc pour rentrer à l'université en électroacoustique, on n'avait pas nécessairement besoin d'un fort bagage en théorie musicale, ça prend un minimum mais c'est tout... puis ensuite je me suis dit plutôt que d'étudier la musique, cette musique là m'intriguait beaucoup, l'électroacoustique, l'acousmatique, je comprenais pas ... je comprenais pas vraiment ... tu sais, c'est un des moments marquants, quand même, de ma vie le concert acousmatique où on s'assoit, il y a personne sur la scène, t'es encore là tu viens du rock, de ta région, et puis t'arrives à Montréal et tu t'assois dans l'concert, personnes sur la scène, du son partout partout, et puis le concert finit, les gens applaudissent,  tu te demandes vraiment... what the fuck ?  Donc je me suis dit à la place de l'étudier, je vais la jouer, je vais l'apprendre et c'est comme ça que je suis rentré un peu ... donc c'est une combinaison de... ça me semblait être plus ouvert parce qu'après ça, tu sais avec tous les groupes de musique pop ben c'est sûr que si tu fais du reggae, tu fais du reggae, et puis tu fais du reggae longtemps, et puis si tu fais du ... métal, tu fais du métal longtemps, et puis si tu...  c'était difficile de sortir des carcans ... de la musique électro-quelque-chose m'a semblé plus... plus encline à faire ce qu'on veut... 

VG — des mélanges... des collaborations?

NB — ouais, et puis qui interdit pas la récupération d'idiomes pop ou rock non plus ... voilà, ce qui fait que ça, plus combiné au fait que c'était relativement facile d'intégrer le milieu ... c'est un peu ça qui a fait que je me suis ramassé là dedans ("se ramasser" : au Québec, "se retrouver dans un endroit sans l'avoir prévu ni voulu", NDT)...

VG — et quand tu parlais de bandes, et de montages que tu as travaillé au ciseau tout ça, il y a un côté cinématographique là-dedans ?

NB — Non... pas... en tout cas pas avec ... non pas tu tout en fait...  je veux dire dans mon travail il y a un côté cinématographique, j'ai commencé avec la vidéo en fait c'est plus... c'est des choses qui se sont oubliées un peu mais dans le fond toutes mes premières œuvres c'était vidéo et puis j'ai un background en design graphique en fait donc j'ai toujours été très visuel, ça a toujours été  assez important ... mais quand j'ai travaillé avec la bande, et puis même si je travaillais encore aujourd'hui avec la bande, je pense que il n'y a pas de relation avec le film en tant que tel... je penserai "sonore"...

VG — pas de "cinéma pour l'oreille"... 

NB — ouais, non c'est ça... non... le cinéma pour l'oreille... 

VG — ça ne te parles pas plus que ça 

NB — non ... mais tu sais, je trouve ça très correct, les parallèles sont super intéressants... mais après, moi je vais pas m'asseoir je sais pas trop où... quand je me mets à travailler, je me dis pas que je vais faire du cinéma pour l'oreille, je me dis pas que je vais faire rien en fait... s'inscrire dans un courant là, je sais pas trop de quoi tu vas me parler...  j'en parlais tantôt avec avec quelqu'un, je lui disais c'est drôle cette conférence là, TENOR, où t'as des gens qui vont comme, revendiquer leur "appartenance" au monde de la partition graphique et puis ... là je me rends compte que moi je me suis ramassé, j'ai un ensemble, je sais pas si t'es au courant mais j'ai un ensemble ici sur des vieux oscillateurs des années 50... par défaut on s'est ramassés dans la partition graphique parce qu'il faut bien qu'on trouve des façons de jouer ensemble, puis de lire, puis de transmettre, puis... j'ai jamais été là dedans mais là tout à coup avec un groupe, faut qu'on se structure un peu... donc là, tout à coup je me ramasse un peu à mon insu dans ce milieu là, il ya plein de gens qui me contactent et puis qui me disent (ton emprunté) "ah oui, toi aussi tu travailles sur la partition graphique, tu peux tu me donner... c'est quoi qui t'intéresse, et puis tu travailles sur quoi... " ... mais, moi c'est juste un moyen parce que bon faut que l'on fasse des musiques, mais ce n'est pas une fin en soi, c'est pas un intérêt plus qu'il faut... c'est juste que c'est un peu... une obligation (rire) en quelque sorte...

VG — c'est des outils dont tu as besoin pour arriver à une finalité..

NB — ben pour, ouais, pour faire de l'art... ce qui m'intéresse c'est l'art, c'est la seule chose qui m'intéresse ... quote : "la seule chose qui m'intéresse c'est l'art, Nicolas Bernier, en face de l'église, 2018" (rires) 

VG — et alors, si je tire un peu la couverture du côté du sujet qui m'intéresse... dans les choses qui m'intéressent dans le numérique, c'est le.. enfin que j'essaie de comprendre en tout cas, parce que je ne suis pas sûr, ou plutôt je suis sûr de ne pas en comprendre tous les contours, mais dans les choses que je trouve intéressantes là dedans, ce qu'il y a de particulier notamment, le fait que dans les instruments qui sont utilisés par rapport à des instruments classiques tu as une possibilité de disruption très forte, tu peux faire des ruptures toutes les nanosecondes si tu veux, toutes les millisecondes, on va dire... et ça change un peu ... et tu utilises des capteurs qui vont contrôler je sais pas quoi derrière ...

NB — quoique la disruption... là je pense en temps réel mais ... je ne sais pas si c'est un...  parce que tu sais au début, excuse moi je te laisse même pas finir ta phrase, je renchéris déjà, mais allons-y ... disruption...  parce que tu dis, bon, disruption, ça "permet" la disruption ... numérique... bon, on dit numérique mais l'analogique le permettait déjà, de 1, donc déjà quand on utilise le numérique faut faire attention avec l'utilisation du terme, et puis de deux, je me dis ouais mais les instruments acoustiques aussi permettaient la disruption, et puis là tout à coup, sauf que tu dis ouais mais l'on peut à la milli-seconde ou à la nanoseconde, et là je me dis ah ouais ok c'est vrai, on peut peut-être pas faire ça avec des instruments acoustiques...  sauf que là si on fait des disruptions à la nanoseconde... tout à coup c'est peut-être plus de la disruption, en fait parce que pour qu'il y ait une disruption faut qu'il y ait un certain temps, ça devient, tu sais pas comme la granulation, on pourrait dire que c'est de la disruption à la nano-seconde mais dans le fond c'est plus de la disruption, c'est de la création de masses, qui elles, pour être rompues, devrait avoir un... donc en tout cas "food for thoughts" ... "nourriture à réfléchir" peut être... je te laisse continuer... 

VG —  peut-être que ce n'était pas un exemple très bien choisi, même si effectivement ça permet de faire ça à des fréquences qu'on pouvait pas faire avant, mais c'est pas uniquement au niveau sonore que je pense à ça mais au niveau de la relation entre le geste éventuel qui va générer un son, ou d'autres choses d'ailleurs, les outils qui permettent de contrôler le son, la lumière, la vidéo, ont tendance à fusionner un peu, dans des logiciels comme Max ou on manipule des données... les relations que tu établies du coup entre la personne ou la machine qui contrôle la musique, qui produit la musique entre le geste et le résultat, tu as une diruption possible, tu peux changer tout le mapping n'importe quand ...

NB — ça oui, on est d'accord ... (mimant sur la table un geste très doux, et faisant subitement un bruit très saturé avec la bouche)

VG — du coup ça a des conséquences...

NB — petit côté théâtral... sur la transmission oui

VG — il n'y a pas une tradition... si tu donnes des cours à la fac de musique sur les technologies numériques, il y a un contexte assez différent entre le fait d'enseigner le piano ou le violon où tu as des traditions et des techniques qui sont pérennes, en tout cas plus ou moins établies depuis des dizaines ou des centaines d'années, et des outils où il faut ré-inventer les choses à chaque fois, à la fois parce qu'elles "permettent" des relations qu'il faut re-définir à chaque fois et puis les technologies eux-même sont moins stables que le bois et le cuivre, et sujets à des mises à jour de système et des choses comme ça. C'est à la fois des contraintes et des possibilités de création, mais qu'est ce qui t'amène à utiliser des outils qui ne sont pas forcément plus stables et facile à utiliser ...

NB — ouais, sauf tu sais, pour quelqu'un comme moi... enfin ça dépend de ta culture aussi ... mais pour moi c'est beaucoup plus facile de manipuler un outil disons informatique que ... d'apprendre le violon avant vingts ans avant de peut-être pouvoir jouer un peu correctement ... il y a une certaine facilité quand même qui vient avec...  on peut s'adapter, on peut changer, il y a pas le fardeau de centaines d'années de musique tonale... encore là... toutes ces questions là... je suis comme ça moi... je me dis qu'il n'y a pas d'absolu

VG — et tu parlais tout à l'heure du fait de toi tu te fichais, on a évoqué ça tout à l'heure (avant l'interview NDR) de comment transmettre les oeuvres, comment faire qu'elles durent, qu'elles puissent être rejoué dans dix ans, tu disais que tu t'en fichais complètement 

NB - ouais... voilà, c'est dit ... officiellement... (rires) non mais c'est parce que moi j'ai... c'est ça j'ai peut-être aussi parce que justement ça c'est ... ok voilà faut que je mette l'argent dans le parc-mètre. je suis désolé si mon téléphone il fait du bruit, d'habitude je ferme le son,  mais là je veux être sûr de ne pas avoir les contraventions ... en fait c'est ce qui est drôle parce que ... là je vais remettre en une demi heure puis prochaine fois que ça sonne faut que notre conversation se termine... dans 10 minutes ça va sonner dans vingt minutes... ça justement c'est un réflexe qui vient de la Musique, justement, de la Musique avec un grand M, d'instruments de bois et de métal, la pérennité donc moi je suis tellement pas là-dedans, surtout que quand ... ça dépend... moi je fais surtout bon, de la performance, des installations... la performance, je l'écris pour moi, pour jouer, pour ... c'est moi qui joue... si c'est pas moi, ça se peut là qu'un jour j'écrive des performances pour d'autres personnes mais... ce n'est pas le premier réflexe en tout cas...  j'ai envie d'être sur scène..  ça fait partie de ... je suis compositeur, mais  en même temps je suis musicien pop aussi donc ... donc pour moi la performance vient avec, quand je meurs, la performance meurt avec moi... aucun... je vois pas vraiment l'intérêt que mon oeuvre soit jouée encore, interprétée par quelqu'un d'autre ... et puis ensuite les installations, ça c'est peut-être plus intéressant comme, tu vois, mais là c'est comme un autre domaine c'est plus proche des arts visuels, tout ça... et puis j'ai une installation qui a été achetée, qui est dans une collection permanente, en France justement, et puis je me dis que quand je vais mourir ben y'a quelqu'un que lui c'est son travail dans la vie c'est de faire en sorte que cette installation là existe indépendamment de mon être et c'est correct 

VG — c'est quelle installation ?

NB — c'est tout petit, ça s'appelle "Frequencies (a / friction)"  c'est un oscillateur sur une table lumineuse, un oscillateur, un diapason... l'oscillateur est à 438Hz et le diapason est à 440 ... l'oscillateur est constant, et le diapason qui tape dessus à interval aléatoire et donc quand le diapason est activé on entend le batement... c'est tout simple ... c'est tout simple mais je l'aime beaucoup...puis voilà c'est quand même chouette ... mais non moi j'ai pas de... j'ai pas d'ego d'artistes, il y a des artistes qui se disent je veux laisser ma marque sur la terre, et on va se rappeler, on va se souvenir de mon nom ... j'ai pas cette prétention là, je fais mon petit truc et voilà ...

VG — mais tu donnes des cours...

NB — ouais

VG — ... donc d'une certaine manière tu es dans la transmission à ce niveau là... qu'est ce que tu estimes utiles ou intéressant de transmettre dans une pratique comme ça ?

NB — (regardant un skateur passer dans la rue, puis excusant son absence d'attention :) parce que je fais du skate ... quand t'entends des skates, tu peux pas t'empêcher de regarder un peu ... c'est quel genre de roues... oh, lui a failli tomber là...  qu'est-ce-qui ... ? qu'est ce que ce que je transmets ? qu'est ce qui m'intéresse de transmettre ?

VG — oui, dans un cours où tu as beaucoup à inventer, 

NB —  ouais - les cours que je donne c'est grosso modo des cours de composition, mais au sens élargi ... je fais des cours au bac on appelle ça des "cours-projet", c'est à dire que ton projet ça peut être la composition stricto-sensu, de la composition musicale, mais ça peut être aussi construire un instruments pour faire une performance, ça peut être de l'installation , ça peut être... q'est-ce que ...(attiré par les skateurs) ça sent le beuh... c'est des vrais skateurs ça... ils sont meilleurs que moi, c'est clair.. voilà une des choses, c'est un peu abstrait, l'une des choses qui m'intéressent, c'est la cohérence du propos artistique, qui fait que peu importe que les outils, les méthodes, les traitements... que tout ce que tu va utiliser pour faire de l'art soit en phase avec ... un processus... des intention qui auront peut-être changé en cours de route, c'est essayer de garder un peu la cohérence dans tout ... ça c'est la première chose qui m'intéresse de transmettre, c'est un peu abstrait peut-être ?...  pas tant que ça ...

VG — du coup, c'est pas forcément... c'est plutôt un travail de guide, d'encadrement? 

NB — oui oui c'est ça... c'est à peu près tout ce que je fais, l'encadrement. Il y a le cours d'ensemble, ça c'est un peu différent, je me ramasse un peu avec le chapeau, un peu chef d'orchestre, en même temps pas trop chef non plus parce que toutes les partitions sont écrites, je suis plus comme un ... si je veux le dire de façon pas glamour, je suis plus comme un "coordinateur", ces trucs là, "directeur artistique"... mais sinon même mes cours, mon grand groupe, "grand" est toujours relatif mais en musique j'ai un groupe de 20 personnes à peu près, ça c'est un "grand groupe",  mais c'est quand même un cours d'initiation à la composition que j'essaie quand même d'inculquer ce dont je viens de parler, sauf que là c'est des étudiants qui entrent en  première année, qui ont vraiment peu d'expérience ou pas du tout, donc l'autre chose que j'essaie de transmettre c'est juste d'avoir... une conscience du développement du temps... c'est un peu... un classique du "compositeur" ... je pense que c'est quelque chose que je maîtrise relativement ... pas pire...(rires) la conscience du développement du temps ou de l'évolution d'énergie dans le temps... et puis je le dis comme ça parce qu'encore une fois l'évolution de l'énergie dans le temps, le développement du temps, ça s'applique pas juste à la musique, ça s'applique... à tout... ça s'applique à la façon dont je suis en train de te parler et puis je vais mettre l'emphase à un moment donné, je vais prendre le temps quelque part d'autre, ça c'est quelque chose qui m'intéresse, que je transmets... qui m'intéresse mais cela dit, sur laquelle j'ai jamais réfléchi, ou j'ai jamais été... si tu me demandes "oui  mais c'est quoi ta conception de..." j'en ai aucune idée, c'est quelque chose que ... que je sens... grosso modo... puis après ça bon c'est sûr qu'à un niveau, euh... ce cours là disons de groupe de composition son, c'est sûr qu'il y a quand même des choses techniques c'est la parti qui m'intéresse pas en fait... mais on va parler d'espace, on va parler de montage, on va parler de  filtrage, on va parler de toutes ces choses qui vont aider éventuellement à développer ton temps comme tu veux... mais c'est pas ...

VG — tu n'attaches pas plus d'importance à ça...

NB — non

VG — une question par rapport aux objets techniques, est ce qu'il y a des outils que tu utilises de manière récurrente, et s'il y en a pourquoi, qu'est ce qui t'intéresse dans ces outils et qui fait qu'ils reviennent dans ton travail ?

NB — mouais, mon petit côté "baveux" (arrogant, méprisant en Québecois, NdT) aimerait répondre "non" à cette question ... mais bon forcément on utilise... mais pour vrai, moi j'utilise le moins d'outils possible ... je te disais tantôt que j'ai un passé en design graphique un peu, mais j'ai un passé aussi en programmation pour le web, c'est fin des années 90... à l'époque c'était \gls{ASP} qu'on faisait... ASP et SQL ... et puis donc quand j'ai commencé à travailler sur le marché professionnel à 17 ou 18 ans dans un bureau avec un salaire... et puis là, ) un moment donné tu te dis bon, t'as 22 ans ça fait déjà cinq ans que tu travailles, tu te dis je vais pas faire ce jusqu'à ma mort... ça n'a pas de bon sens ... donc làj'ai décidé de retourner en musique, faire de l'art... pourquoi je dis ça?... ah oui! par rapport aux outils... qui qu'on programme beaucoup en musique maintenant ou dans les arts numériques, et puis moi c'est une chose que j'ai pas particulièrement le goût de faire, j'ai quitté un milieu pas pour... bon ça m'a aidé dans mes études j'avais de l'aisance là dedans plus que d'autres gens, sauf que je suis pas là pour ça tu sais... comme je disais tantôt ce qui m'intéresse, c'est la finalité qui m'intéresse, c'est pour ça que je vais engager des gens pour faire les choses, c'est une culture qu'on n'a pas beaucoup en tout cas à Montréal mais en France plus, tout l'assistanat musical là, nous autres ici c'est quelque chose qui n'existe pas vraiment, on est vraiment une culture \gls{DIY}, on fait tout ... il y a quelque chose d'un peu macho là dedans, tu sais, si t'as pas fait tout, si t'as pas ...  c'est pas "authentique", t'sais... mais moi je suis pas vraiment là dedans parce que au final moi, j'aimes ça jouer... j'aime ça entendre les choses mais je ne tiens pas à avoir fait la mécanique en arrière, bon en même temps j'en fais une bonne partie quand même parce que mes moyens sont limités ... tout ça pour dire que... pouvez-vous répéter la question monsieur? (rires)

VG — je te demandais s'il y avait des outils qui revenait de manière récurrent...

NB — ah ouais, c'est ça...  donc longue histoire pour dire que j'utilise le moins d'outils possible ... puis moi depuis plusieurs années ma plateforme c'est Live (Ableton Live, NDR), pourquoi Live? entre autres parce qu'il y a Max for Live dedans puis moi avec ces deux choses là, écoute, j'ai pas besoin de beaucoup plus... après... 

VG — c'est la rapidité du fait d'arriver à tes fins qui est la motivation principale ?

NB — ouais, puis de pouvoir faire un maximum de choses en changeant le moins, en ne changeant pas d'environnement constamment... ça évite des bugs je pense, tu sais, le fait de pas avoir untel qui communique avec untel, qui communique avec untel, avec untel qui revient à untel et puis... tout est dans la même fenêtre, tout est... ce qui fait que moi, ça me convient vraiment ... puis après, bon ça c'est un peu  les outils de base disons, mais après c'est sûr que chaque projet est quand même différent donc il va toujours y avoir... mais tu sais, je vais te donner un exemple où je n'ai pas fait, où je me suis pas écouté dans un projet récent, un gros projet, bon j'ai travaillé beaucoup son et lumière ces dernières années, et puis je travaille avec un microcontrôleur, que tu peux même pas dire le nom, je sais même pas si ça a un nom,  je suis tombé là dessus par hasard un moment donné dans mes recherches il y a dix ans, un gars dans un sous-sol aux états unis qui fait des petites cartes... à l'époque j'étais en train de travailler avec des dimmer-packs... je dis dimmer-pack, tu vois c'est quoi ? je ne sais pas le terme en français ...  les trucs pour les gradateurs de lumière, normalement t'as des showvel (TODO ???) où tu peux mettre quatre ampoules puis tu peux envoyer du DMX et contrôler ton éclairage... tu vois c'est quoi cette boite là, ce qu'il y a dans les théâtre pour contrôler l'éclairage ... tout ça pour dire que là j'étais avec ça et puis je me suis rendu compte, ça faisait pas très longtemps qu'on travaille avec les LEDS et puis à un moment donné, je tombe sur ce micro-contrôleur là, qui à la place de peser 10 livres, faire 4 canaux c'est juste une carte ça fait trente deux canaux, puis ça pèse rien, puis donc j'ai commencé à travailler avec ça, puis là tous mes projets sont construits sur cette carte là que je connais, avec un objet ... puis là c'est l'autre affaire, tu sais ya un objet dans Max qui communique avec machin et puis là est-ce que l'interface DMX, USB-DMX qui parle à l' objet qui ... (soupirs) ... là j'ai une formule puis bon il ya d'autres choses qui rentrent en ligne de compte moi j'ai besoin d'une rapidité à toute épreuve et puis les interfaces commerciales, style ENTTEC, qui sont assez connues dans le monde de l'éclairage semi-professionnel, ben ça va pas assez vite il y a des dropped-frames tu sais ça perd des images, et ça me convient pas, je suis tombé un peu par chance sur cette formule là, qui fonctionne super bien pour mon usage, donc la plupart de mes projets sont construits là dessus. Quand je pars sur un gros projet j'ai pas besoin de me dire/il y a beaucoup de choses qui/chaque projet est différent, moi la partie compliquée c'est la partie mécanique, tu sais la partie, euh...  j'ai besoin d'un parasol, telle couleur, tel matériau ... ça je trouve ça compliqué et puis la partie design industriel ... j'apprends, je connais rien là dedans et puis j'ai quand même pas le choix de ... la partie technique, technologique un peu, qui se renouvelle tout le temps c'est plus ça tu vois... la partie "outil" parce que je construis mes outils... c'est ça ... qu'est ce qu'on entend par "outil"... mes outils c'est toujours grosso-modo les mêmes, mais mon "dispositif" il est jamais pareil ...  ce qui fait que là je réutilise, je sais pas si tu me suis dans mon histoire, je réutilise tout le temps le même truc sauf que le dernier projet, gros projet, je me fais convaincre par des jeunes... trop ambitieux... que je dois changer tous mes outils... parce que ... X Y raisons ... ok ... ok ... on va tout changer ... mais là au final il ya des bugs, ya des machins qui communiquent pas avec d'autres machins, puis d'autres machins qui communiquent pas, puis là  ya trop de data, et puis on perd du data, puis là gna gna gna... Au final on finit par réussir à faire un prototype de peine et de misère qui annoncerait un projet somme toute quand même assez intéressant, sauf qu'on en parlait tantôt, faut que j'aille chercher les sous pour aller... puisqu'on a réinventé les outils, l'argent est parti dans les outils et pendant l'art ... parce que je te dis, moi ce qui m'intéresse c'est l'art...  puis là je me ramasse avec un prototype... pas d'art ...puis faudrait que j'aille chercher des sous encore pour faire l'art, puis je réussi pas, je sais pas pourquoi, j'ai pas le don, les gens n'aiment pas le projet...


VG — peut-être parce que tu ne demandes pas des sous pour du matériel, peut-être

NB — ouais c'est ça... ben oui, voilà, c'est ça... exactement... tu as mis le doigt sur quelque chose...

VG — un problème récurrent oui...

NB — donc ce projet là, ben ... poubelle ... et puis j'ai travaillé fort longtemps et comme je te dis, c'est pour avoir un prototype quand même assez chouette et puis là finalement, j'ai juste fait comme... arf ... poubelle... et puis tous ces processus sont assez longs... tu fais une demande de subventions, tu reçois la réponse un an après ... le projet démarre six mois après... tu travail pendant un an ... ce qui fait que ça veut dire qu'une oeuvre s'il faut qu'il fasse beau taxi ça fait deux ans et demi heure avant deux ans et demi ça veut dire quatre cinq ans sur le même truc moi je peux pas travailler comme cela c'est parce que je suis rendu après cinq ans charia ne s'intéressent plus en fait mon projet me donne explique donc ça répond un peu la question ouais ok tant mieux moi tout perdu [Applaudissements] mais par un peu curieux parce que par abbas que je t'ai dit c'est moi dans le fond les petits spots en sol la question qui est plus en forme c'est comme la scénographie quelque part c'est pour moi c'est seul jeu scénographie mais le dispositif l'objet que parce que se crée des objets discret pas des instruments je crée des objets plus camden c'est ce que je pense plus pro jeunesse c'est un mec pas peut-être plus proche de la sculpture que cas de la musique je com j'ai fait des blessés des objets voix c'est sûr ça va être fait passer quelque chose comme bâtir un dispositif de maintenant pour simplifie la visualisation c'est ça ça peut se rapprocher un peu dingue des cartes de théa des fois ça peut être un instrument de musée des fois ça peut être un peu de tout seul mais pour moi c'est ce qui compte j'habite j'habite une scène j'habite un espace d'une pratique grosso modo solo j'ai pas une équipe de concepteurs et qu'ils ont même proposé à voici on a pensé travailler l'aluminium faut que j'essaye d'imaginer tout seul - marine qui c'est vraiment un pas évident c'est faire c'est ça les voileux désengluer industriel fab est plein dur qui j'ai besoin d'eux un tube d'aluminium de cette grandeur avec tel jean d'ancrage jean d'ancrage qui va prendre tel jean de boulons qui faut se ranger dentelle gens de 15 qui va pas peser plus que temps de livres de machines fait tout seul qui en plus faut que je fasse d'alors avec ça devient compliqué nous un bel objet mais le sait ce que je fais ça m'a coûté cher a faim il faut bien que je trouve quelque chose à faire avec ça je me suis mis ni patins ce qui me disait que je ne trouverai quelque chose à faire est ce que ça va fonctionner je ne sais pas tant que l'objet de cette époque comme j'ai pas des budgets de recherche et développement qui fait que l'outil flambé de l'argent tu verras oui j'essaye des choses mais c'est ça passe ou ça casse tout le temps tu fais beaucoup les choses toi même ben comment dire je les imagine moi même après je coupe à l'aluminium moi même que je sois pas de travail manuel mois maintenant la programmation je veux en faire partie quand c'est trop mou d'habitude la programmation c'est vraiment plus pour moi c'est très utile quand je fais pas de création programme à succès j'ai besoin que ça fasse ça ça va être une étape quand même assez sans évoquer leurs enfants car jusqu'à ça devient plus complexe le sache je m'embarque culotte ça m'intéresse que j'ai pu le temps de toute façon fait ça je vais engager quelqu'un nippone j'ai pas engagé tant manqué ça pour ça la première phase de ce projet l'objet mis à la poubelle il y en a un autre que j'ai mis à des poubelles il y en a un que j'utilise cette application pour justement de suivi partitions graphiques surface c'est à peu près tout le reste c'était j'ai besoin de déclencher des shows de l'est elle donne pour moi de voir ce que tu disais tout à l'heure de fait compte qu'avec la programmation numérique tous ces outils de hardware ou software où tout va un terrain tu peux potentiellement interconnecté plein de chocs et kiki qui donne des projets qui sont son lourd à monter parce que du coup en voyant ça d'une manière optimiste et et naïve on peut se dire que justement ça permet de faire et d'être vu comme ça de l'admettre à bordeaux avoir un bon centre durant un cours de l'installation mais quelque part le fait que ce soit le lourd face à cette complexité et cette lourdeur de la programmation du coup est-ce que tu as l'impression que ça influe la manière dont tu vas recevoir teaser tu penses que c'est toute façon fait que ce soit vraiment streamline tu n as qu est très direct par rapport au fait que c'est justement techniquement les agissements du temps c ouais je pense que je pense que prenait sous serment qu aurélien chantent ils dansent pas ça va dépendre de chaque fois j'ai ça dépend des fois oui des fois tu dis quand même mais des fois mais c'est ce comme le tout signé par les deux commentait à parler de deux d'écriture de la métamorphose donne quand même un peu de mon écriture algorithmiques ne peut pas par là une forme qui va se développer on peut faire des choses interactif qui est un scénario qui se développe d'autant moins d'un imprévu qu'elle soit prévue ou scénarisé ok mais avec ces outils qui permettent dans un ordinateur au sens large du terme son statut peut mettre beaucoup de mémoire d'un coup ça permet potentiellement dangereux lorsqu'ils à voir de performance les choses se renouvellent chan avait un je dormais potosi manque de fer mais peut-être que de le faire aussi ça génère des projets qui est plus lourd et donc s'attacher ça que par le fait que ce soit regardez une certaine fraîcheur et que tu puisse arriver at on idée avant que cinq ans soit passé je sais pas je n'ai jamais vraiment posé de problème mais peut-être que je me mets effectivement je vais essayer de forcer toujours de toute façon dans tout les projets une certaine simplicité fait j'essaie de dire de toute façon un projet ça fonctionne pourtant pareil du tout un bon top bip le mandat ni tranquille sur hypo real ça le football le dans cette prise de conscience de cette façon de travailler cette zone là où il faut réorganiser les cités de départ et des gens qui vont qui vont pousser la frontière le plus loin possible ses conseillers enfin le truc impossible jusqu'à la dernière seconde qui t'as qu'à l'être un tas qui vont réussir à face qui est impossible ou presque qui vont jusqu'à revenir à la case départ moi je moi j'ai tendance à rapidement un livre tout seul je me suis peut-être tel que j'ai jamais jamais eu je me rendis à cette complexité jouer ça ici un peu menotté par la complexité du projet mais par exemple un projet avec marseille mais ici que nos gens de parler c'est une machine couture en bois s'est classé au chocolat avec comme ça et ce qui demande souvent aux pieds de large le papa de dirhams ait jamais reçue delajoux hélas house band l'instrument s'imposant on peut pas sortir dans notre chambre à coucher qui se met à jouer dans un projet comme ça c'est compliqué mais ce matin au niveau des des outils qu au niveau 2 sans veut répéter sans me répéter ce spectacle est faut louer une salle idéalement cocaïne l'éclairage donc finalement faute et un technicien donc le si tu veux qu'ils nous on n'a pas les moyens d'une compagnie perd on a les moyens on partage on est dans la catégorie musiciens donc empressé par un musicien ça s'organise ou un artiste disons numérique ça s'organise seul fait que fais tu là on n'a pas des ''j'ai son pote cinq fois plus gros parce que on a un instrument qui demande ça donc ça c'est compliqué ça je pense que c'est un frein c'est un projet qu'elle juge plus ou moins conteste d'ailleurs qui d'après moi c'est entre autres parce que il était tellement complexe l'algérie pleut tout sauf que ça ce range si ça va dans des cas sinon ceux des grosses qu'un est ce que les fonds qu'ils entreposent le campion entrepôts ça coûte de l'argent c'est un frein c'est un frein à lens ou même tuer martin lui ça n'est pas simple du tout pensé lui va pousser la lumière de plus loin possible dans ce projet que j'ai fait avec lui tout ça fait longtemps que j'aurais fait quand entier regarde notre machine à rêves deux fois plus petit qu'on va pouvoir la faire dans mon petit local puis au moins ça va être moins imposants visuellement mais au moins on va pouvoir plus travailler avec qui apprendre à jouer d'un instrument en pis travailler le matin travaillé la composition a dream en mai et lui mais c'est un projet collaboratif par définition on fait d'aider des personnes pas heureuse qui lui met de l'eau dans son gréement aussi pour découvrir quelque chose qui ni lui ni moi même qui fait que c'est peut-être plus à ce niveau là je dirais que je sens où la complexité des projets en demandait nous va avoir un frein à main tout mais en fait j'ai donné l'exemple parce que c'est un peu devenu pour moi c'est justement crittenton une photo c'est le plus flagrant mais toutes mes projets c'est comme ça chaque projet comme la toutes mes projets c'est tout des dispositifs que sur les cages que jamais répétées sont un peu gros tout centre poser parce que j'ai pas les moyens ça coûte cher à envoyer ou trop mal donc mais la plupart de mes projets je lui fais une fois à montréal après je n'en vois en europe puis ça reste en europe je sais que je le fais entreposés à gauche à droite pour pas avoir à payer le voyage du retour donc ce qu'est ce que ça veut dire ça veut dire que je peux pas travailler évidemment beloeil quand je le fais disons la première fois à montréal mais c'est tout le temps c'est une première puis une première s'est jamais une version finale la clé ça veut dire que je peux pas répéter avec mais les dispositifs comme ça pour moi c'est un gros gros gros pas mal listen to la manière dont tu travailles ouais des accents que je cite cinq ans je sais pas si ça conditionne la manière travail mais en tout cas ça c'est si ça me ferait à ça mais des freins dans le processus qui à l'achèvement de mon travail contrairement une compagnie danse théâtre qui ont des budgets pour avoir un lieu des techniciens des espaces qui martin messier demi à sa campagne maintenant puisque c'est lui qui travaille sur un projet on peut se permettre de de nuit une salle vide vraiment d'expérimenter son projet en salle d'avoir de faire le clan d'éclairage de savoir comment ça va se passer mon dernier projet c'est drôle j'ai parlé avec un diffuseur en allemand en espagne le fameux dixième mais comment par ce projet de l'éclairage un truc parce qu'elle lui fait de la musique électronique hier son studio mais y'a pas nécessairement de positif physique avec son travail reste pas assez laid les camions de magna qui sait ça j écoute fait comme le projet que j'ai faite non son festival de mode les montées c'est comme trois murs enregistre c'est comme 3,3 meur à peu près cette affaire là en arrière de l'autre au début puis dans la performance jour toujours ouvrir début comme ça qui à la fin ils vont être de se placer comme sur une ligne faite au début c'est le seul comme ça mais à la fin c'est c'est quand même dingue et blanche on sent une certaine l'argent que j'ai pas dans mon dans mon appartement donc vous en dit on a croisé dois-je me disant ça va fonctionner j'espère que ça fonctionne une fois sur place qui fixe la seule chose qui après 20 heures une fois sur place sexy dans des termes ce médian dont je vais essayer de mettre une lumière tel titre le schiste parle avec les techniciens etc tel type de couleurs que tu n'essayes est un sport qui leur à chaque représentation jamais lire des choses ajoute une lumière gens ferment une loge mieux je ressens un peu la disposition qui à un moment donné j'ai fini par avoir quelque chose de satisfaisant mais je le travaille la sève dans mes répétitions juste avec le public entre en salle c'est comme ceux que je finis par concevoir mais mais chose parce que j'ai pas lieu d'avoir on la question que total d'inscriptions nazies peut-être une boîte de pandore aussi question d'internet du fait que j'ai interviewé nick collins nouvelle qui nous les outils numériques actuels qui sont très accessibles d'internet que faire de la lutherie numérique c'était plus comme faire de la cuisine prenez l'exemple de ces étudiants qui et donne à servir un moteur et qui avait jamais fait d'armé notre excellent les charlie sheen a maintenant doublé tél à quel point tu me comment tu considères sa façon plus générale en particulier le fait que du coup à le travail est fait facilement copier coller de fichiers comme ça tu dois comprendre le truc de quelqu'un d'autre était pas forcément connus deux jeunes hommes je travaille avec le un truc a été pensé au regard d une certaine manière pour tous est-ce que ça réaffecte derrière le résultat de ce c'est sûr que ça fait que le résultat mais en même temps je de ce fils c'est l'histoire de l'humanité on ne fait que ça du copier coller ben oui tant mieux si c'est plus facile après est ce que ces gars 1 d'un intérêt de ton heure dan évidemment non le souverain pièces pour du temps tél compositeur celle ci la recopie est tant mieux c'est tant mieux peut progresser là dedans est ce que c'est que ces garants l'intérêt de putain nombre c'est la même chose avec l'électronique qui après bien sûr vos publics les cours au pérou en mêlant le fait de choses à deux ainsi de vos coups d'avancé un intérêt lapin d'une part aussi norbert la personne qu'il fait grandi là dedans tant mieux mais je sais pas nés quand l'un d'eux me voyais à sion d'un mauvais oeil j'étais pas forcément un jugement positif ouais c'est ça tout cas lui vient d'une période où il n'avait pas toute la documentation c'est sage arrêter tout ça m'étonnerait mais pas c'est sûr il ya des gens qui pourraient avoir un discours un peu quand je vois comment du pg artistes par rapport à sony mais qu'en l'absence ils se sont vus jusque là m'a dit que ça devienne accessible l'électronique et il devrait être content tu devrais jouer lorsque - sackey mais pipi l'autre consulter sexy section n'a pas à apprendre la mécanique si la mécanique se fait seul puis en tapant dans google mais ça je dirais qu'on peut faire de la retape plus rapidement que ça veut dire qu'on peut peut-être passer moins de temps à parler des outils concrets passé plus de temps à parler d'eux qu'est ce que tu as fait avec style de forme de teint de matière 2 des choix des intentions qui pour moi ces personnes ça parce que je m'étais peut-être que miss mais le mou ça m'intéresse poté afin de leçon mais une bonne nouvelle c'est en fait parler plus d'allant tous cas me demander mais comment c'est fait je te demander comment s'est fait dire je comprends tout maintenant je suis pas compliqué d'abord en chantant dans une lignes de code qui fait windham save latin qu'est ce que tu voulais dire ça c'est la discussion qui m'a formé l'avocat du diable nous a peut-être que un bon côté à venir c j'ai passé elle favorise un esprit geek mais s'il vous plaît là comme les laisse loin sabrine truc geek le plus de discussions peut être autour des objectifs des panneaux à ossature binger s'est protégée dont j'ai vu que j'ai 40 ans je vais cutter en dessous cette langue de sable c'est ce que fait pas toujours été chaud sur les discussions techniques jeudi on se retrouve autour d'une auto puis on parle l'ami kane et on part du matin de campagne on était un peu fait comme ça jusqu'en mai à défaut de parler de l'esthétique pourquoi azumi s'est tellement cette soupe d'ailerons pour la métaphore mais c'est sa meilleure moyenne fascine c'est ça une facilité de avec les outils qui est là avec internet sarah la recette facile c'est un peu la même chose qu'eux c'est une honte de faire la mise et qui n'ont que faire de auto est ce que ça veut des cas parce que tout le monde peut faire des photos avec son téléphone mettre des photos qui sont là ont la même valeur dans la main de plus on compte c'était un très bon on va reconnaître c'est ça l'histoire sera va connaître celles qui voudraient abandonner demander un 3 il ces instruments numériques mais n'existe pas dans la y a pas l'équivalent dans la durée historique est ce que tu te penses est ce que tu souhaites faire ne souhaite pas vous les instruments numériques vont se cristalliser vers des formes particulières comme comme ça a été le cas avec les instruments acoustiques qui se sont établis en forme assez stable ouais ce que je souhaite je pense je le souhaite je pense pas qu'ils le souhaitent motif jamais - mm non jamais je venais juste rien à cirer j'étais en fait c'est qu'ils vivent il va sûrement avoir des instruments 6 66 soit tu vas voir des instruments numériques non standard [Musique] c'est aussi un bail que les camps pour l'instant dans un concours la halle qu'est ce qu'un instrument tout en contrôlant avec un ordinateur je veux ça s'est standardisée on prend ça pour acquis au continent même pas quand même un instrument mais à quelque part c'est embêtant dans ma tête il bouge pas te relire des paramètres l'instrument ça s'est standardisée est tant mieux c'est ce qui va en avoir qui aura sûrement des instrumentistes licence qui va sûrement avoir des trucs un peu un peu fou quand même un jeu qui n'ont rien à dire qu'ils vont êtres normaux on voit jouer à six coups tant mieux mais mais j'espère en fait qui va continuer à voir tout ce temps l'aude d'inventions qui vont être les faits mais la planète et qui sont si sommes pas encore une fois moi je pense pas ma pensée repas dansant du développement d'un instrument qui va être ma pensée va aux vins créer un objet d'art christophe j'adore le bain j'ai peut-être besoin d'instruments qui excite mais j'ai peut-être besoin d'instruments qui n'existe pas puis le projet d'après ben j'avais sûrement pau le goût de réutiliser le même instrument parce que ça va être une longue robe noire long terme a dit que sa maîtrise des différents avec un objectif différent qui j'espère que ça va continuer à quelques pas ça veut simplement faire période c'est un cotisant au tout le monde c'est de la musique électronique réside synthétisant de la seule option oui avec l'unc sorte de synthétiser ses chronos mon truc fasse que france que je veux pas rester petit nuage devait déjà vécu révolution industrielle l'inventeur des instruments sens aux aguets l'acte est pas du côté de la musique on a même grandi touchant peu la demande si dans les vieux lisent le matériel scientifique l'inventaire plein plaindre ce qui ressemblait quand même drôlement notre époque d'aujourd'hui où on avait l'impression qu'on pouvait tout inventé bon sur ce coup de serviette en boucle à gauche fermé au début de la conversation pourquoi je faisais ce parce que tout est possible il me semble que j'aimerais ça garderie hier des sols de sentir que tu t'es passé parce que si ça se perde on commercialise c'est que tous les instruments sont pareils not make up victimes la biodiversité tu vois quand tant de formes tantôt ce faire les deux femmes s'étaient dits choix payant dans la forme si tu es le seul fait baiser qu'ils veulent en finir le plus ennuyant la forme mais dans une conversation est un peu moins fait un peu moins bien qu'eux les moustaches mais bon la planète et de la faune je ne tiens on retourne on pouvait pas voilà il ya quelque chose à faire avec ça



