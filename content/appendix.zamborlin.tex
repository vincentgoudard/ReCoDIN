\chapter{Interview : Bruno Zamborlin}
\label{appendix:zamborlin}

\section*{Biographie}

\noindent Bruno Zamborlin (né en 1984) a effectué une thèse entre Goldsmith Univerity et l'IRCAM sur le sujet de l'appropriation des interfaces numériques pour la musique, avant de créer en 2013 une startup pour commercialiser les Mogees, un dispositif captant les vibrations pour ``transformer tout objet physique en instruments de musique''. En 2018, la startup Mogees est devenue ``HyperSurfaces'' et étend la technologie des Mogees à d'autres domaines que la musique, pour convertir des objets usuels en interfaces Homme-Machine via le traitement de leur vibration.

\noindent Site web HyperSurfaces : \url{https://www.hypersurfaces.com}

\section*{Transcript}

\noindent Bruno Zamborlin, entretien du 23/08/2017, dans les bureaux de Mogees, Londres.

\noindent [Les entretiens ne sont pas disponibles publiquement.]

% VG — first I wanted you to tell me about what, in the first plac, gave you the will to design your own instrument rather than choosing the piano or other ready-made instruments 

% BZ — So I think the first time I thought of Mogees was... the idea that I had was really trying to overcome some of the limitations of music creation and specifically electronic music creations, something I've been always very jealous I was see my friends playing acoustic or electric instruments and being able to you know just improvise and just go to the beach or to the middle of the street and just improvise and just jam together, you know, very easy, very spontaneous, very visual, very powerful... and then I was the guy you know behind the laptop [laughing] so was the kind that always had to prepare for ages actually to to you know to load my samples and load my setup and actually always had to think about what to play beforehand, I never had that spontaneity that they had. So I wanted to come up with some tool that actually at the very beginning was for myself. Some tool that could actually enabled me to be so spontaneous with electronic music as they were with a guitar or a drum kit. Is that one minute yes? 

% VG — So basically you wanted mobility? Is that what you mean ?

% BZ — I wanted spontaneous... yes I wanted something that was simple for me to carry on, that I could fit in my pocket and at the same time I wanted something that was flexible that allowed me that a variety of different sounds so MIDI was very good because I you know this idea of actually controlling multiple sound sources was was very interesting for me and that, you know, I experimented with cameras and with other types of sensors but I never managed actually to find something it was actually easy to set up and plug and play... then I had this idea of actually using the sound of physical objects so actually using existing physical objects as a ... as an interface and physical objects are great because they are surprising because you don't carry them with you just find things around you and after you try actually playing Mogees with different objects more and more you you find that there are actually commonalities, common patterns... so every time you play on a new table or a new tree or a new bike or a thing, you're always like, you know, looking at easier and easier actually, and I really tried to design Mogees so it could be as customizable as possible. So you know there is more than 15,000 pieces, like there are 15,000 people that use it in a completely different way one from the other and I love this concept of reusing the skills that someone has so... if, you know, dancers for example that use Mogees, they... they use like the skills of dancing, you know, for example they stick it on the table they jump on a table and they starts actually you know stepping on the table with their feet and there and they are triggering different sounds... I really like this idea of like okay I am a dancer and I'm free using the skills that I already acquired what something completely different to make music. The same thing happened with all the guitar players that actually stick a Mogees on the body of their guitar and then you know they used it they kind of extend the capabilities of the guitars, they're still playing the guitar but they add some new gestures... for example to trigger a pattern or whatever and they are reusing that skills to play Mogees as well ... I love that variety, you know, that flexibility. And you can see that with artists but you can see that even with with kids so every time I see kids like you know playing it I am amazed by the curiosity and the ideas that they have, like, one time, like, they embedded like a Mogees inside a basketball like a spongy ball, with the phone inside and then it started actually play the ball or they would come up with any crazy idea, using toys or LEGO construction to create their own live-set, they do amazing things and like not be constrained by what you learned...

% VG — You mentionned that you were ``the guy behind a laptop'', so I guess you were essentially using keyboard and mouse and maybe some MIDI interfaces to play music ?

% BZ — Yes exactly, and I started more more to move to a mobile setup where I had like a couple of Mogees, a couple of phones, connected to a speaker and ... yes, that's it... that's so much more portable ... 

% VG — but still that's a pretty radical shift ... you just abandonned all the buttons and knobs-style interfaces which are still the most common interfaces for playing electronic music ...

% BZ — yes you're right, it was a shift yes... so at the beginning, I did it for me, you're right... and it was really like a big jump to letting go the more classic MIDI interfaces... I mean I still use them, of course, you know, they give extra much control, they're super useful in many situation but on the other side, for more... uh... in other musical situations, I just like to have nothing and just improvise completely and not having anything prepared beforehand ... so yes I didi it for me, and I did this jump and I started playing with a few Mogees... and then I think that was the first video that I made that got popular and many people asked me if they could actually get a Mogees and I said well, no I don't have any... it was just, you know, just for me ... and then I did the very first kickstarter project and that's when basically I raised the funds to make the first 2000 maybe or something... that was back in ... like few years ago... and that's when actually people started using it and that's when, of course like it always happens, I realized that that the instrument was simple for me but was not simple at all for other people... so I had to improve a lot the usability, do a lot of tests... and theses guys, the very first backers, they did help so much in ensuring that actually the instrument was usable by anyone ... and then, after that, we improved so much the usability 

% VG — you mean the software interface ? 

% BZ — both ... yes, well mostly software, mostly software ... but also ...

% VG — because it looks to me like the object still looks like it was in the first place 

% BZ —  yes...  the hardware yes ... we did like very minor tweaks ... but yes, hardware-wise yes it's pretty much exactly the same, yes, and all the work has been about the apps ... 

% VG — can you describe a little bit how Mogees works ?

% BZ — so the way it works is it's a piezo transducer that you stick on to the object you want to play and like every piezo transducer basically it transforms all the vibrations that you make when you touch this physical object into an electric signal ... like a microphone... and then this signal goes to your phone ... and in this phone, our app is running and our app has these algorithms that basically analyzes the signal and transform ... and try to understand in real-time how we are interacting with the physical object ... and it lets you program this physical objects so as to react to the sound you want, with the gestures that you want ... So you can train, say, this table and you can train this table so that it learns when you tap with your thumb or with your other thumb, or when you scratch... any gesture you want and then that's the machine learning, that actually learns all these gestures and then allows you basically to trigger different sounds every time you do those gestures ... It's like, if you want it's like if you were programming these physical objects so as to react exactly the way you want ... So it's one of the first instruments really that allows you to, you know, that kind of learns from you, so that doesn't impose you any particular behavior, you know, if you have a MIDI controller, you'll have to move that fader or that knob or push that button, you know... So the gesture you do are always the same...  so you can, I mean, if you want you can push the button with your nose, if you want to, but of course it's much easier to push it with your finger, and MIDI controller has been designed for your fingers ... you know ... While this table hasn't been designed for me to make music out of it, you know, so it's really like a hacking, in a way, it's like I am actually seeing this table and I think wow! ... what kind of music, what kind of interaction can I make with this table ? And it's up to me to make it happen, the app is just a tool to fulfill what I have in mind. So that's why, for me, what we made is not a musical instrument at all... It's a tool to allow you to make your own instruments 

% VG — mm... that's a good pitch... One of the specifics of DMIs is the fact that they are programmable, and that's a very interesting point that you raised, that you somehow program the objects ... which is, in a funny way, a kind of twist on the Internet of things somehow... because we usually see digital objects replacing the usual everyday ones and this turns the usual object into a digital interface... 

% BZ — yes!

% VG — with this instrument, you offer people the possibility to plug anything on anything somehow ... which sounds like softwares that we use, like Max or whatever, that allow you to plug anything on anything... and there are a lot of different strategies for the mapping between interfaces and synthesis ... and so, if I understood well, the app is containing instruments that you can load and the choose your synthesis ?

% BZ — in terms of sound, you mean ? So .. the way you can see it... in technical terms, you know in Mogees, you have the analysis and the synthesis, right ? So the analysis is on the part that actually learns your gestures and it triggers a... a MIDI event, if you want... then we have our own internal synthesizers, that we made, so there are couples that can triggers old-school drum machine sounds, then we have a lot of physical model syntheses, that are our famous ones that you may have seen on our videos, thoses are really the more ... uh... peculiar ones, because you can actually use them only with Mogees, because the idea of these physical models is a physical model that is actually excited by the real sound that you make when you interact with the physical objects ... You have really continuous interaction, like a scratch, like a physical scratch will actually sound like a scratch ... like you're scratching a string of a piano, or the string of a violin... but yes, you can program it so you can associate for example the model of a, I mean, you know, it's not a real physical model, but you can imagine, like, if you were to associate, like, the string of violin to this table, so this table would sound like a violin... 

% VG — this acts like a filter, in the general sense of it...

% BZ — yes... it's a synthesizer... 
 
% VG — since you can plug anything on anything, and you have this particular thing with digital musical instrument that the same gesture, depending on what virtual stuff you loaded on the computer, your phone, or whatever, the same gesture will have totally different behavior on sound. I mean, not only sound but but also the behavior .... maybe that's a little less true for Mogees, since you are very close to the material, but I mean if you take a MIDI interface, turning a button could as much change the frequency of cutoff filter, as scrolling in a sound Bank for example, which are just mentally totally unrelated actions...

% BZ — yes, there is no connection between the controller and the effect that the controller is having on the system, yes. In Mogees, so... it's up to you as well, so you define the mapping, so it's similar in that way but the analysis and the synthesis, if you want, like the gesture, and the effect that it has on the system are more tied together than in a normal MIDI controller, because... all these synthesizers that we made, that are excited by the real sound of the object, are definitely connected with the gesture that you do ... So, you know, we allow this freedom to users, the user can decide to just trigger a sample that has nothing to do with this table, so if you want to disconnect completely the gesture and the sound, he can do that. It's very useful sometimes, you know. But ... you're right when saying that actually you also have this possibility to trigger sounds that are strictly dependent on the gesture they make. 

% VG — on the gesture energy somehow ...

% BZ — Yes, the sound really, yes ... it's like, you know, we are exciting the physical model with real sound that is picked up by the microphone. 

% VG — another thing that is specific to digital musical instruments is that you can embed memory, knowledge, into the body of the instrument itself. This could be sound samples but this could be also rules of musical composition like scales or whatever ... and how do you see, as far as Mogees is concerned, this part of musical science that you can embed into the instrument ? 

% BZ — well, a little bit less than in other instruments because I expose the mapping to the user, so a little bit less than, you know, a ReacTable, or other instruments that actually have a behavior that is very musical, in a way. In Mogees, I like the user to define this behavior. But... there are course constraints like for example you have to tap on object, right? So that's already like a very specific behavior that I'm imposing ... so it is a percussion, a percussive instrument in a way, I'm actually constraining the platter of actions that you do, because you have to tap on objects. So the kind of way I am implicitly pushing someone to use it is percussive... Anyway... we can define that as a behavior or a constraint, if you want... In terms of musical composition, no, because you trigger the sounds you want... So there's no.... it's quite free, I think

% VG — yes... I guess that is more about sound design, somehow, that you embed in Mogees example and the composition is...

% BZ — ... is up to you... I'm not defining any scale, for example, or any ...

% VG — I had in mind this video where you tap the table and the note would change automatically ...

% BZ — yes, we also made an app for kids where you preload a MIDI file, and then every time you tap, you just step through a MIDI sequence... right? I would... struggle... to call that a musical instrument... because for me it's more like a musical tool to explore the sensation of playing ... a piano piece ... you know? ... I find it fascinating from a pedagogical point of view, in the sense that, you know, teachers are really excited about that app, because it allows basically a kid  is creating music for the first time or is playing music first time, actually, to focus on the aspects, like the tempo and the accents, that ... usually you focus on a very ... like, after a few years you're actually playing piano... and at the same time, they know this is always right, because it's actually pre-recorded in the app... so you don't have to learn other more difficult skills, like, you know, how to sit, and the coordination between the two hands and where all the notes are... So it's a way actually to stimulating at you ... stimulating the creativity of the kids, in terms of tempo and accent and personalization of the playing music. For me is really like a an exercise of active listening, between listening to music and playing a piece. It is a little bit in between of these two worlds. I don't know if people call it a musical instrument or not, I'm usually not interested in definitions, anyone has his own definitions but it's ... yes... I'm not into that...

% VG — among the ... 5000 users, you said ?

% BZ — 16000 users

% VG — 16000 users, oh my god

% BZ — between ``pro'' and ``play'', we have two products, like, ``pro'' is more targeting musicians and artists and ``play'' is more for kids and gamers... 

% VG — and how much do you estimate the part of ``pro'' versus ``play'' users? 

% BZ — uh... half and half roughly... very roughly...

% VG — and amateur playing, is it somehow more related to music education ?

% BZ —  yes, well we have four apps. If you buy Mogees Pro, you have all the apps, if you buy Mogees Play, you have the three games, but you don't have the music creation app which is the more complex one

% VG — which lets you define the mappings...

% BZ — yes... exactly...

% VG — and regarding these two poles of amateur versus pro practices, one of the girl from the Bela-platform team, was mentionning that they try to bridge the gap between amateurs and experts, well, in the quite expert field of programming your own embedded low-latency device...

% BZ — yes, this one is quite different that mine, leaning toward the experts, yes (rires) and I couldn't be onto [Bela] (rires)

% VG —  still, you find people that would like to make their instruments and are interested in the musical side of it and would not want to learn assembly code... 

% BZ — of course, it's very important...

% VG — one need that we encouter with digital tools nowadays, is to be able to start using a tool as an amateur and explore further on later... keep the same technological tool and reach expert practice after practicing the amateur version... starting from learning the basics, but still being able to discover more advanced functions 

% BZ — hacking the system you mean ?... with something simple that has constraints but that can be appropriated by the user... right ?

% VG — that an amateur instrument can become an expert instrument, in short.

% BZ — yes, oh absolutely...  yes so that's what I think... yes there's like a study, you know like one of the guys behind Bela was also my PhD examiner, Andrew McPherson, and Michael Gurevitch from the NIME community, like they started this very interesting concept that is the concept of constraints in musical instruments.... it's like, you know, a project you may have heard of is this one-button instrument ... so they said, okay let's try to design the instrument that has the highest level of constraint ever : button... just a button that has no velocity, makes just one simple sound, one beep, that's it. Let's give it to musicians and see what they come up with. And they notice that of course, because the instrument is so constrained, everyone was actually inventing their own techniques of playingn, with very very different ... to hands, to feet, to nose... and they tried to find ways to make something special, you know...  and I think it's a little bit what you say, it's... it's a very interesting approach, this classic, you know, ``low entry fee with no ceiling to virtuosity'', a motto where you try to make something simple that everybody can use but that also allows for some customization and someone to appropriate a little bit ... a little bit of rules and try to make it his own instrument. That's a fascinating concept, yes. With Mogees, it is almost like forced ... because, like, the instrument has no shape, has no rules, is just a sensor, yes? And you ... like the first action I ask everybody to do is : find an object, stick it (the Mogees, NdE) to that object and then, think about how you can play that object, you know, so in a way I kind of push in this behavior. Because I'm not telling anyone how to how to play it, you know, while if you have a piano, you know that you have to sit down and you got to play it in a certain way... so I kind of opened this out completely, you know...  the first time this can be quite terrifying, you know, like ... as a commercial product it can give a lot of promise because users are so used to be guided... used to being told what to do and I'm giving you like a blank sheet... I say okay you can do whatever you want and they're like ``for example what ?'' ... you know... like, they wanna know ... So we had to make a lot of videos and a lot of, like, ideas just to stimulate them a little bit because, you know, of course the experts ones have no problem, they were just uh yes walk in the streets finding crazy objects and start playing them, but others are more, like ... the need for some stimulus, some guidelines first... 

% VG — yes... I was saying that another specific thing of this digital instrument making activity of the 21st century, is the existence of communities which are helping both instrument makers and instrument players.... I wanted to know, in your case, how do you feel that it helped you, in the first place, to design your instrument 

% BZ — sure ... I think... music instrument designers always need to talk to the musicians and to actually have this constant feedback from the users, otherwise their job will be impossible. So it has always been the case, 100 years ago... now it just became easier, if you want, because we can read all the comments on the socials and we actually know what they think, in a more unbiased way, because we can actually watch that talking about your instrument between them... So that it's a definitely easier now ... In my case, Kickstarter helped so much, because you know, you have like a very direct way of actually engage with your community... But this is a thing that has been the case for ages, like, think about the first people who started actually appropriating the turntable, to use it as a musical instrument ... I thinkg that all the pioneer, all the manufactors of turntables were really interested, and saying ``oh my god, I'm seeing a new market there... let's try to make a turntable that was actually designed to make scratches, so we can sell more''. So like, all this feedback is crucial, we are always looking for endorsers that can actually become virtuoso of our instruments and show how amazing our instruments are, to the rest of the world.... We're also looking for very honest feedback from the musicians that actually try to do something with our instrument, they have suggestions and criticisms...

% VG — you mean it's just the scale that is changing ?

% BZ — the scale is changing, yes... well, you know, it depends... it depends... I mean, you ... in the past, you know, there were fewer manufacturers and these manufacturers were selling very huge numbers ... now there are many small, very small manufacturers, that have like more specific communities, if you want... so I don't know if the scale changes actually, I don't know ...   there's always been a lot of musicians, it's more like the now is easier to gather information, that's what I think... it's more like, musicians have always be keen to buy new things and talk about it, but we couldn't hear them... and now we can...  

% VG — I was thinking of the community of musician but also of the community of makers. I mean, this technological objects need a lot of competence, knowledges from from various fields, like electronics, computer science, design ... So, how did you find your way to gather all of this knowledge and to come up with an instrument that is at the same time brand-new and ready-made ? Like the time-span between the idea that you have and the commercial product is pretty short, if you compare it to, say, the time needed for a luthier to produce one violin, a hundred years ago ...

% BZ — sure, yes .... now is much quicker, yes...

% VG — how did you connect with all these people who have this knowledge of design, of electronics ... Or did you do it on your own ? 

% BZ — No, definitely not... In my case, yes, I started a startup, with a Kickstarter to raise funds ... Kickstarter is an amazing way of doing this, because you basically sell it first, you're raising the funds and then you can actually pay experts that can help you in all the areas that you have no idea... Like, I'm not a designer, I am not a graphist, I am not a manufacturer... so you need all these people to work with you... and you need funds for that and that's where the crowd-funding really helps... 

% VG — but did you have enough knowledge to make the protoype by yourself ?

% BZ — no I had enough knowledge to know ... that I couldn't (rires) ... 

% VG — even during your PhD at IRCAM ?

% BZ — yes, first prototype, yes... of course, yes... I made it myself, yes... that was basically, with Max/MSP and a few contact mics, you know, custom made contact mics... but then you know, from a proof of concept to a product that you can give to people, there are years of work... so much ... it's huge

% VG — maybe one last question to conclude this interview, I'd like to ask you a pronostic about something that you feel will be important in the next 10 years, not necessarily ``The'' future, but something that you feel is developping and interesting at some point, in terms of instrument making, or that you feel related to instrument making...

% BZ — musical instruments are more and more defining music genders... I think... sorry... let me correct it, what I just said... I think new interfaces for musical expression are more and more versatile and can enable someone to make very diverse music genders... the opposite of, you know, 100 years ago, the piano plays piano-music, the flute plays flute-music and now this is so different... I think there's gonna be more and more people that will want to act of music creation by music creation ... I guess these boundaries will blur with other fields like gaming, education... I can see a huge disruption coming in the world of education towards a much more personnalized, ad-hoc, free, gamified way of teaching to kids ... much less standardized like it is now where we are just, you know, the huge classrooms and we all have to do the same thing, at the same pace, with the same program... That is going to disappear... And because of that, because of that freedom, maybe we will enable the next generation to enhance their creativity much more and I hope music will be a very good test-bed to stimulate our creativity, that can then be applied to many many other fields in our lives ...