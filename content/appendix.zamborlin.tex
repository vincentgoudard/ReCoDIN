\chapter{Interview : Bruno Zamborlin}
\label{appendix:zamborlin}

\section*{Biographie}
\noindent Bruno Zamborlin a effectué une thèse entre Goldsmith Univerity et l'IRCAM sur le sujet de l'appropriation des interfaces numériques pour la musique, avant de créer en 2013 une startup pour commercialiser les Mogees, un dispositif captant les vibrations pour ``transformer tout objet physique en instruments de musique''. En 2018, la startup Mogees est devenue ``HyperSurfaces'' et étend la technologie des Mogees à d'autres domaines que la musique, pour convertir des objets usuels en interfaces Homme-Machine via le traitement de leur vibration.

\section*{Transcript}

\noindent Bruno Zamborlin, interview du 23/08/2017, dans les bureaux de Mogees, Londres.

VG — first I wanted you to tell me about what's in the first place give you the will to design your own instrument rather than choosing the piano of ready-made instruments 

BZ — So I think the first time I thought of Mogees was... the idea that I had was really trying to overcome some of the limitations of music creation and specifically electronic music creations, something I've been always very jealous I was see my friends playing acoustic or electric instruments and being able to you know just improvise and just go to the beach or to the middle of the street and just improvise and just jam together, you know, very easy, very spontaneous, very visual, very powerful... and then I was the guy you know behind the laptop [laughing] so was the kind that always had to prepare for ages actually to to you know to load my samples and load my setup and actually always had to think about what to play beforehand, I never had that spontaneity that they had. So I wanted to come up with some tool that actually at the very beginning was for myself. Some tool that could actually enabled me to be so spontaneous with electronic music as they were with a guitar or a drum kit. Is that one minute yeah? 

VG — So basically you wanted mobility? Is that what you mean ?

BZ — I wanted spontaneous... yes I wanted something that was simple for me to carry on, that I could fit in my pocket and at the same time I wanted something that was flexible that allowed me that a variety of different sounds so MIDI was very good because I you know this idea of actually controlling multiple sound sources was was very interesting for me and that, you know, I experimented with cameras and with other types of sensors but I never managed actually to find something it was actually easy to set up and plug and play... then I had this idea of actually using the sound of physical objects so actually using existing physical objects as a ... as an interface and physical objects are great because they are surprising because you don't carry them with you just find things around you and after you try actually playing Mogees with different objects more and more you you find that there are actually commonalities, common patterns... so every time you play on a new table or a new tree or a new bike or a thing, you're always like, you know, looking at easier and easier actually, and I really tried to design Mogees so it could be as customizable as possible. So you know there is more than 15,000 pieces, like there are 15,000 people that use it in a completely different way one from the other and I love this concept of reusing the skills that someone has so... if, you know, dancers for example that use Mogees, they... they use like the skills of dancing, you know, for example they stick it on the table they jump on a table and they starts actually you know stepping on the table with their feet and there and they are triggering different sounds... I really like this idea of like okay I am a dancer and I'm free using the skills that I already acquired what something completely different to make music. The same thing happened with all the guitar players that actually stick a Mogees on the body of their guitar and then you know they used it they kind of extend the capabilities of the guitars, they're still playing the guitar but they add some new gestures... for example to trigger a pattern or whatever and they are reusing that skills to play Mogees as well ... I love that variety, you know, that flexibility. And you can see that with artists but you can see that even with with kids so every time I see kids like you know playing it I am amazed by the curiosity and the ideas that they have, like, one time, like, they embedded like a Mogees inside a basketball like a spongy ball, with the phone inside and then it started actually play the ball or they would come up with any crazy idea, using toys or LEGO construction to create their own live-set, they do amazing things and like not be constrained by what you learned...

VG — At some point, you mentioned the guy behind the laptop so we're essentially yes exactly yes and I started more more to move my stop where I just have a couple of more just couple of phones connected to that speaker yeah that's it was like so much more portable I mean you just abandon all the buttons and no style interface which are the most most interfaces for paying what you are made up of buttons and faders and stuff like that so there are two things you you dropped all these and at the same time I feel that you mentioned about all people who so we are using Mogees and able to be used I guess from that that you have in mind that all the people that you could use the instrument that you that yes you're right it was it was a shift yes at the beginning I did it for me you're right and it was not correct like a big job like to you know to letting go like all the classic MIDI interfaces I mean I still use them of course I even to give you looks much control over there super useful situations but at the other side you know for more yeah I mean in other musical situations like I just like to have nothing and just improvise completely and not having a defeat prepared beforehand so yeah so at the beginning it was really for me and I did this jump you know when I started playing a few Mogees and then I am that I think like I was when I it was that first video that I made that got popular than many people asked me if they could actually get Mogees and I said well no you know it was just that's for me and then I did at least the very first Kickstarter project and that's when basically like I did you know I raised the funds to make first mm maybe was one thing that was back a few years ago and that's when actually people started using a and that's of course where that always happens like I really realized that an instrument a simple me but was not simple tools for other people so I had to you know to improve a lot usability a lot of tests and these guys like the very first backers like the helped so much ensuring that actually the instrument was usable and then after that basically you know like that the interface so now I think oh yeah I mean I was a software mostly software but also looks like it was in the first place yes the hardware yeah put it like very minor tweaks but yeah like hard hardware wise yes it's pretty much exactly the same yeah and all the work is being about I mean it's been about the apps so we can do this one minute explanation about how much it works to like you know I have no plan for the description so hi works yeah so the way it works is it's a kisser transducer that you stick on to the objective on a play and like every piece reducer basically transforms all the vibrations that you make when you touch this physical object into an electric signal and then this signal goes to your phone and in this form our app is running and our app has designer with them to basically analyzes signal and transform and you try to to understand in real time how we are interacting with the physical object any lets you program this physical objects or stream yet to the sound you want with the gestures that you want so you can train say this table and it can train this table so as in there's your top with your tongue your target another town your scratch you know any just do you want and then that's the machine learning the ABS actually learns all these gestures and then he allows you basically to trigger different sounds every time you do those gestures it's like if you want it's like if you were programming these physical objects us to react exactly the way you want so it's not the first instruments really that allows you to you know that kind of learns from you so that doesn't impose you any particular behavior you know if you have a MIDI controller you have to move that failure that number that push that button you know so the gesture didn't use it was always the same so you can I mean if you want I mean you can push the button with your nose if you want to but of course it's much easier and we should your fingers so that MIDI controller has to be designed for your fingers you know what this day well hasn't been designed for me to make music out of it you know so it's really like a hockey in a way it's like I'm I am actually seeing in this table and I think wow what kind of music what kind of interaction can I make with this table and it's up to me to make it happen in their the app is just a tool to feel what I have in mind so that's why for me what we made itself it's not a musical instrument to allow you to make your own instruments one of the specific specifics of sessions is that they are programmable very interesting that she raised that you somehow program the objects and yeah which is in a funny way the kind of that's a funny in version of and that leads me to question about how with this instrument they offer people the possibility to plug anything on anything somehow what that's like the software that we use at max or whatever work they are you to plug anything or anything so there are a lot of different choices on the way on the mapping so called nothing between interfaces and synthesis and so I thought I know how much is worth it it's not containing instruments that you can load this way and then you choosing it a veteran synthesizes it meant so the way you can see in technical terms you you know which is you have the analysis and synthesis so the analysis is of the part that actually learns your gestures any triggers are and then we have our own internal synthesizers that are that we made so there are couples that capture your old-school vintage drum machine sounds then we have a lot of physical models synthesis the more famous ones seen many videos so those are really the more cool ear months because you can actually use them all even Mochis because the idea of these physical models is that there's a physical model that is actually excited by the real sound that you make when you interact with the physical objects you have really continuous interaction is like a scratch like a physical scratch will actually sound like a scratch like you're scratching us-31 or this train the body looking but yes now you can program it so you can associate for example the model Allah I mean you know it's not a real physical model but in a week you can imagine like if you were to associate like the string of violin - you can plug anything on anything and you have this touch block in the digital musical instrument that changes depending on what virtual stuffability whatever these are you the same gesture will have totally different different behavior even sound I mean not only sound but but also the behavior maybe that's a little less informative since you are very close to the material but I mean if you take media interface turning a button could as much change the frequency of cutoff frequency as scrolling in a sound Bank which are just actually there is no connection between the controller and the fact that the controller is having on the system yes Mogees so it's up to you as well so you you to find the mapping so it's similar in that that way but the analysis and the synthesis if you want like that the gesture and the fact that he has on the system are more tied together MIDI controller because all this synthesizer that we made there are excited by the real sound of the of the object are definitely connected but adjuster didn't you do so you know we allow this freedom to users through the user can decide to just figure out a sample that has nothing to do with this table so if you want to disconnect completely but just you and the sound she can do that it's very useful sometimes you know but you're right saying that actually you was a hot possibility to meet you - to trigger sounds that are strictly dependent on the on the gesture they make yes the sound really yeah it's like you know we are exciting the physical model real sound that is beat up by the microphone the other thing another thing that is specific to digital music events for money that you can embed memory knowledge into the body of the instrument itself but this could be some sample but this could be also rules of musical composition like scape survivor and how do you see as far as modulus concerned this is part of musical tires that you well a little bit less than in every instruments because I expose them up painting to the user so a little bit less than you know at a restore my stucco I can we have table muster actually have a behavior that is it's very musical in a way Jesus like I let the user define this behavior but there are course constraints like for example you have to stop on object right so that's already like a very specific behavior that I'm imposing so it is a Prakash percussive instrument is why I'm actually constraining the platter of actions that you can do because you have to tap on objects so they the kind of way I am implicitly pushing someone to use it is percussive anyway we can define that behavior or constraining in terms of musical composition now because you sound to you you want for example and the note will change that you have yes we also may be made like enough for kids with this we know what I mean I mean fire and then every time you talk you just step through a mini sequence from that struggle to call them Israelis to man because for me it's more like a musical to explore the sensation of playing a piano piece no I find it fascinating from a pedagogical point of view in the sense that you know teachers are really excited about that up because he allows basically a kid that is creating music for the first time or is playing is for the first time actually to focus on on the the aspects like the tempo and the accents that you usually you focus and very likely after a few years ago in Direction playing piano and at the same time they know this always right because recorded in yeah so you don't have to learn other more difficult skills like you know how to sit quite collision between the two hands and where where all the notes are so it's a way actually to stimulating at you still meeting the creativity of the kids in terms of tampon accent and personalization of the play music for me is really like a mystery an exercise of active listening switch between listen music and playing piece is it a little bit in the between of these two worlds I don't know if people call it a musical instrument or not I know I'm usually not interested in definitions anyone else's over the finish but it's a between point place where two two products like Pro it's more targeting musicians and artists [Music] yeah for apps if you buy will just throw out all the apps if you buy them a display you have the three games but you don't have the music creation up there isn't more yes yesterday saying that with the same technological to it should be easy to discover and learn basics and to find your way with it and you should be able to go further with more expert [Music] Hawking tea this is to meet you mrs. Wang some would say that all that has constraints but the dad can be appropriated by the user yeah right yeah instrument can become the experiment yes yes how absolutely yes so that's what I think - yeah there's like a study you know like one of the guys behind Bela was with some entities examiner and remembered that person like he and Michael guru each from the non community like they started this very interesting concept that is the concept of constraints instruments it's like oh you know that was virtually my hair that is the bottom instrument so they say okay so let's try to design the issue that has more like they the highest level of constraint every button is just a button has no velocity makes just one simple sound this one beep that's it let's give it to musicians and see what they come up with and they notice that of course like because the instrument is so it's trying to be then everyone was actually invented its own technique of playing with very very different - hi Liz my the lecture right ways to maybe something special you know and I think it's a little bit what you say is like is like it's it's very interesting approach this this classic you know low entry fee is we know sealing to your toe city motto where you try to make something simple that everybody can use but that also allows for some customization and someone to appropriated a little bit and a little bit of rules and try to make it his own instrument a fascinating concept yeah when whoa Jesus is almost like forced because like the instrument has no shape as no rules is just a sensor yeah and you like the first first action I guess everybody to do is find an object stick it to the Dobson and then think about how you can play that object you know it's in a way I can appreciate this behavior because I'm not telling anyone how to how to play it you know where if you have Catalan you know that you have to sit down and you got to play today you know in a certain way yeah so I kind of opened this out you know first time this can be quite terrifying you know like as a commercial product again gifts give a lot of promise because users are so used to be guided I'm used in negative to all yeah what what to do when I and I'm giving you like a blank sheet I say okay you can do what I really want and then like for example what you know like they wanna know so we had to make a lot of videos and a lot of like ideas just to stimulate them a little bit because you know of course the experts wants have no problem they were just uh yeah walk in the stratum finding crazy objects and start playing it but others are more like the need so I was saying that another specific thing of this activity of the 21st century the existence of communities which are helping both instrument makers and instrument players I would to know in your case how do you feel that it had it helped you in the first place being able to design an instrument I sure I think [Music] music instrument designers always need to talk to the musicians and to actually have this constant feedback from the users otherwise their job will be impossible so it has always been the case 100 years ago now it just became easier if you want because we can read all the comments social and we actually know what they think we can actually watch that talking about your instrument between there so that it's a little bit easier now mark my case Kickstarter helped so much because you you have like very direct way of actually engaging your community despina has been case for ages like think about first people started actually appropriating the dirt able to say oh my god I'm seeing a new market there's let's try to make an inter table that was actually designed to make scratches we got more today so like what is this feedback as crucial we are always looking for not only for door search that can actually become your tools or about rings to mention show how amazing our instruments are to the rest of the world we're also looking for very honest feedback from the musicians but actually trying to do something suggesting the skin is changing ya know I wear the pants in the pants I mean you in the past you know there's always be a lot of musicians it's more like the nouns is easier to gather information that's why it's more like musicians have always be by new things and I was thinking of the community of musician but also of the community I mean this technological objects the competence knowledge is from from various fields like electronics computer science design so how how to define a way to gather all of all this knowledge and to come up with arrangement at least the same time brand-new and and I ready-made like the time span between the item that you have and the commercial product is pretty short if you contrary to I mean even the time routine needs to produce one dialing a hundred years ago has no it's totally relate sure sure so yeah how do you connect with all these people who have this knowledge of design of electronics of did you do it on your own or in my case yeah really start to raise funds except there is an amazing way of doing this because you basically salad first erasing the pants and then you can actually a experts that can help you in all the areas that you profits factor I don't know so it's you need all these people you know to to to work with you but you had enough knowledge to make the prototype by yourself no I had enough knowledge to know but that you know from that from proof of concept and but maybe just one last question I'd like to finish something that you feel will be important in the next 10 years what what you feel about is not necessarily the future but something that you feel music in terms of digital instrument making oh let's say instrument making what you feel mr. storm is a more horrifying music genres I think sorry so we corrected it what it I said I think musical a new interfaces for musical expressions are more more their supplies music generous so different it's definitely I think they got me like more and more people they will want to skip a step after music creation but music creation I can see its boundaries feels like gaming indication I can see like a huge disruption coming in the world towards much more personalized at all free game defied way of teaching kids much less standardized like is now huge classrooms we all have to do the same the same pace and same program that that's balance appear and cause of death of that freedom maybe we will enable the next generation 2 X 2 2 2 2 2 gangster music very good that's that to stimulate our creativity that but can that be applied to many many other fields of our lives.