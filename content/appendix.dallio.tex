la réflexion c'était comment peut on inventer quelque chose qui permet à la fois de garder cette pratique du clavier, qui pour moi est importante, mais aussi pour contrôler l'ordinateur, puisque cet instrument est simplement une extension de l'ordinateur, pour changer un paramètre


Olivier :

J'ai l'habiture de récupérer des objets qui me parlent, je ne sais pas pourquoi ils me parlent, mais ils me parlent. Et j'ai pu associer des mouvements aux formes des objets. Je ne sais pas si c'est clair, mais c'est comme ça que ça se passe dans ma tête.
Le mouvement m'inspirait l'objet, ou l'objet était en corrélation avec le mouvement.

c'est aussi une de mes habitudes de récupérer et de savoir ce que j'ai dans mes affaires et de pouvoir l'intégrer à quelque chose d'utile et de pratique, qui n'est pas forcément son but d'origine.



La recherche, elle est dans comment trouver le son qu'on a envie d'entendre à un moment X, comment avec tous ces paramètres on peut arriver à un son qui convient au moment, à l'intensité musicale ou une réponse à un partenaire de jeu.