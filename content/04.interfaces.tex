% !TEX root = ../thesis-example.tex
%
\chapter{Interface sensible / hardware}
\label{ch:interfaces}

\cleanchapterquote{Komponieren heißt: über die Mittel nachdenken.\\
Komponieren heißt: ein Instrument bauen.\\
Komponieren heißt: nicht sich gehen, sondern sich kommen lassen.\\
.}{Helmut Lachenmann}{1986}

\cleanchapterquote{Fingers are not to be 
despised: they are great inspirers, and, in contact with a 
musical instrument, often give birth to subconscious ideas 
which might otherwise never come to life.}{Igor Stravinsky}{1936}


\vspace{-1em}
\begin{itemize}[noitemsep]
\item Faire évoluer une interface en la raffinant (De Laubier, VG).
\item Faire évoluer une interface en rajoutant des choses (
\item 
\item 

\end{itemize}



Pourquoi parler d’interface “sensorielle” plutôt que gestuelle : différence entre le geste et l’information captée par l’ordinateur. Exemple critique de la caméra. Data sonification => data comme geste ?

Filigramophone : évolution depuis la tablette graphique simple, tablette augmentée, écran multitouch augmenté, intégration de Bela…

\section{Interface pour composer/interpréter/improviser en live}
Articulation entre grandes formes et immédiateté : la gestion du temps de l’interaction à différentes échelles.
La performance musicale avec un DMI permet d’articuler une partie compositionnelle (échelle de temps longue) et une partie interprétation/improvisation (immédiateté de l’instant). 
Quelle design d’interface permet de répondre à l’articulation de ces 2 pôles ? 

\section{Qualité ergodynamiques}
\subsection{Ergodynamisme}
cf. définition de Thor Magnusson
choix des capteurs, latence
agencement de l’interface, représentation visuelle, repères tactile, frettage

\subsection{facteur de forme et topologie: héritage et transpositions}

\subsection{retour audio-tactile}
Intégration de HP tactile
Spatialisation ou diffusion ad-hoc
Travail avec Pascale Criton à la maison des aveugles


\section{Généalogie d’une interface de DMI}
\label{sec:interfaces:sec1}

De la simple tablette graphique à l’écran multitouch augmenté de capteurs, histoire de l’évolution d’une interface pour la performance électroacoustique.
La conception d’une nouvelle interface pour la performance musicale est une tâche complexe, nécessitant de nombreux aller-retours entre conception, fabrication et pratique musicale. Le filigramophone est une interface qui a connu plusieurs versions, suffisamment différentes pour avoir envie de leur donner un nouveau nom à chaque fois et suffisamment similaire pour y voir la continuité d’un seul et même instrument.

\subsection{origine : la tablette graphique}
La tablette graphique (nommément un modèle Sapphire de Wacom) a été l’interface originelle qui a servi de base au filigramophone. J’ai commencé à l’utiliser suite à son utilisation dans le cadre de la Méta-Mallette\footnote{Logiciel pour la pratique collective de musique par ordinateur développé par l’association Puce Muse.}. La raison de ce choix est que la tablette graphique offre une interface relativement bon marché (donc déployable en nombre) qui permet un contrôle assez fin de la synthèse sonore.
Un certain nombre de musiciens, compositeurs et concepteurs de NIME l’ont adopté pour leur projet \cite{Zby07}, et Nicolas d’Alessandro a consacré une partie de son travail de thèse \cite{Ale09} à ce sujet.

La tablette graphique permet de bénéficier de l’expertise du geste d’écriture et de dessin.

\begin{quote}
"Dufourt suggests that contemporary music highlights what was rejected in the Greek world : it rather captures the evanescent, the ephemeral, the ambivalent, the Erebus, it favors the endless metamorphosis of qualities and forms; as Nietzsche proclaimed, western music tends toward the liberation of the dyonisiac dimension and the acceptance of the inacceptable part of myths. » Risset, Sound and Music Computing Meets Philosophy, ICMC proc. 2014
\end{quote}

\subsection{augmentation de la tablette graphique}
ajout de piezo et HP tactile pour une réponse immédiate
ajout de MPD pour le changement de comportement

\subsection{de la tablette à l'écran graphique multitouch}


\section{Conclusion}
\label{sec:interfaces:conclusion}
