% !TEX root = ../thesis-example.tex
%
\chapter{Conclusion}
\label{ch:conclusion}

\cleanchapterquote{
\textnormal{MUSIQUE.} Coup de baguette et récapitulation des musiques précédentes ou musique source seule.\\ 
Un temps.\\ 
\textnormal{PAROLES. — Encore. (}Un temps. Implorant.\textnormal{) Encore ! \\
MUSIQUE.} Répète dernière musique telle quelle ou à peine variée.\\ 
Un temps.\\ 
\textnormal{PAROLES.} \textit{Profond soupir.}
}
{Samuel Beckett}{(dans: paroles et musique. 1962.)}

Maurice Conti, TED talk the incredible inventions of intuitive ai:
\vspace{-1em}
\begin{itemize}[noitemsep]
	\item Things fabricated => things farmed
	\item constructed => grown
	\item isolated => connected
	\item extraction => aggregation
	\item obedience => autonomy
\end{itemize}


Le mathématicien et poète Jacques Roubaud confiait, ``La poésie dit ce qu’elle dit en le disant. Ce qui exclut la paraphrase. On peut exposer ce que la poésie raconte, mais, si on le fait, on perd ce quelque chose d’essentiel qui est la poésie. À l’extrême opposé, la mathématique ne se développe qu’en se paraphrasant, qu’en se redisant d’une manière différente.''
On court ainsi toujours le risque, lorsqu'on cherche à expliquer les motivations sous-jacentes aux outils de création, souvent fortement entrelacées avec celles des œuvres elles-mêmes, de refermer l'horizon qu'elles ouvrent.

La musique contient le futur des instruments.


Après 30 ans de NIME, il faut prendre garde à la calcification de certains concepts en des fondations solides. 

Quel intérêt à construire des instruments éphémère (cf. Cage) ?
Même que la construction de fusée à usage unique pour l'exploration de l'espace?
Pas écologiques .