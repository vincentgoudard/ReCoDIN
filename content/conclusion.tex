% !TEX root = ../thesis-example.tex
%
\chapter{Conclusion}
\label{ch:conclusion}

\cleanchapterquote{
\textnormal{MUSIQUE.} Coup de baguette et récapitulation des musiques précédentes ou musique source seule.\\ 
Un temps.\\ 
\textnormal{PAROLES. — Encore. (}Un temps. Implorant.\textnormal{) Encore ! \\
MUSIQUE.} Répète dernière musique telle quelle ou à peine variée.\\ 
Un temps.\\ 
\textnormal{PAROLES.} \textit{Profond soupir.}
}
{Samuel Beckett}{(dans: paroles et musique. 1962.)}


\noindent La motivation initiale de ce travail était de cerner les contours ou du moins, d'en dessiner les lignes de fuite. Au terme de cette étude, on se rend compte que les contours de la notion d'instrument de musique en général sont mouvants, que son histoire est marquée par des apports et des évolutions permanentes et que sa pérennité ne s'établit que de manière relative, à l'aune d'un répertoire et d'une pratique associée.

Les \gslpl{DMI} ajoute à cette instabilité l'atomisation du corps instrumental dû à la modularité intrinsèque au fonctionnement de l'informatique. Les reconfigurations possibles de l'instrument, envisagé comme un agencement de fonctions instrumentales qui se cristallise dans une configuration contextuelle lui ont, davantage encore que l'instrument acoustique, fait perdre la stabilité relative que présentent les instruments acoustiques.

Cette reconfiguration générale de l'agencement possible d'un instrument permet d'écrire sur un medium commun, à la fois les relations gestuelles-sonores autrefois séparées dans les domaines de la lutherie, de la composition, et d'autre part les mécanismes cognitifs propres à la perception de ces relations.


Ce champ de possibilités, qui constitue, comme l'ont montré les interviews réalisées, une des motivations majeures pour l'utilisation des \glspl{DMI}, implique des choix de design dont la singularité est une marque manifeste. 

La singularité déroutante de ces nouveaux instruments (insatisfaisante pour certains), peut amener à tenter de définir le socle définissant


Maurice Conti, TED talk the incredible inventions of intuitive ai:
\vspace{-1em}
\begin{itemize}[noitemsep]
	\item Things fabricated => things farmed
	\item constructed => grown
	\item isolated => connected
	\item extraction => aggregation
	\item obedience => autonomy
\end{itemize}


Le mathématicien et poète Jacques Roubaud confiait, ``La poésie dit ce qu’elle dit en le disant. Ce qui exclut la paraphrase. On peut exposer ce que la poésie raconte, mais, si on le fait, on perd ce quelque chose d’essentiel qui est la poésie. À l’extrême opposé, la mathématique ne se développe qu’en se paraphrasant, qu’en se redisant d’une manière différente.''
On court ainsi toujours le risque, lorsqu'on cherche à expliquer les motivations sous-jacentes aux outils de création, souvent fortement entrelacées avec celles des œuvres elles-mêmes, de refermer l'horizon qu'elles ouvrent.

La musique contient le futur des instruments.


Après 30 ans de NIME, il faut prendre garde à la calcification de certains concepts en des fondations solides. 

Quel intérêt à construire des instruments éphémère (cf. Cage) ?
Même que la construction de fusée à usage unique pour l'exploration de l'espace?
Pas écologiques .