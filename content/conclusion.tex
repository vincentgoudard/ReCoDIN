% !TEX root = ../thesis-example.tex
%
\chapter{Conclusion}
\label{ch:conclusion}

\cleanchapterquote{
\textnormal{MUSIQUE.} Coup de baguette et récapitulation des musiques précédentes ou musique source seule.\\ 
Un temps.\\ 
\textnormal{PAROLES. — Encore. (}Un temps. Implorant.\textnormal{) Encore ! \\
MUSIQUE.} Répète dernière musique telle quelle ou à peine variée.\\ 
Un temps.\\ 
\textnormal{PAROLES.} \textit{Profond soupir.}
}
{Samuel Beckett}{\textit{paroles et musique}, Pièce radiophonique, 1962. \cite{beckett_comeet_2014}}

TODO

\noindent La motivation initiale de ce travail était l'étude transversale des aspects de contrôle et de représentation dans le design des \gls{DMI}, à travers les différents contexte qu'il implique, conception, fabrication, programmation et pratique musicale. Au terme de cette étude, on se rend compte que les contours de la notion d'instrument de musique en général sont mouvants, que son histoire est marquée par des apports et des évolutions permanentes et que sa pérennité ne s'établit que de manière relative, à l'aune d'un répertoire et d'une pratique associée.

Les \glspl{DMI} ajoute à cette instabilité l'atomisation du corps instrumental dû à la modularité intrinsèque au fonctionnement de l'informatique. Les reconfigurations possibles de l'instrument, envisagé comme un agencement de fonctions instrumentales qui se cristallise dans une configuration contextuelle lui ont, davantage encore que l'instrument acoustique, fait perdre la stabilité relative que présentent les instruments acoustiques.

Cette reconfiguration générale de l'agencement possible d'un instrument permet d'écrire sur un medium commun, à la fois les relations gestuelles-sonores autrefois séparées dans les domaines de la lutherie, de la composition, et d'autre part les mécanismes cognitifs propres à la perception de ces relations.


Ce champ de possibilités, qui constitue, comme l'ont montré les interviews réalisées, une des motivations majeures pour l'utilisation des \glspl{DMI}, implique des choix de design dont la singularité est une marque manifeste. 

La singularité déroutante de ces nouveaux instruments (insatisfaisante pour certains), peut amener à tenter de définir le socle définissant



Plusieurs aspects ont été soulignés qui se prête à l'invention : 
\vspace{-1em}
\begin{itemize}[noitemsep]
	\item la programmation, dont l'étude au chapitre \ref{ch:algorithms} a montré qu'au delà de l'inter-connection entre des capteurs et des paramètres de synthèse, il s'agit de l'endroit principal où se définit le comportement de l'instrument. Nous avons proposé le concept de ``modèle intermédiaire'' comme objet conceptuel représentant une unité de traitement matérialisant de manière algorithmique une transformation du signal, qu'il soit audio, gestuel, pour désigner des modules de traitement pouvant impliquer 
\end{itemize}

Maurice Conti, TED talk the incredible inventions of intuitive ai:
\vspace{-1em}
\begin{itemize}[noitemsep]
	\item Things fabricated => things farmed
	\item constructed => grown
	\item isolated => connected
	\item extraction => aggregation
	\item obedience => autonomy
\end{itemize}


Le mathématicien et poète Jacques Roubaud confiait, ``La poésie dit ce qu’elle dit en le disant. Ce qui exclut la paraphrase. On peut exposer ce que la poésie raconte, mais, si on le fait, on perd ce quelque chose d’essentiel qui est la poésie. À l’extrême opposé, la mathématique ne se développe qu’en se paraphrasant, qu’en se redisant d’une manière différente.''
On court ainsi toujours le risque, lorsqu'on cherche à expliquer les motivations sous-jacentes aux outils de création, souvent fortement entrelacées avec celles des œuvres elles-mêmes, de refermer l'horizon qu'elles ouvrent.

La musique contient le futur des instruments.


Après 30 ans de NIME, il faut prendre garde à la calcification de certains concepts en des fondations solides. 

Quel intérêt à construire des instruments éphémère (cf. Cage) ?
Même que la construction de fusée à usage unique pour l'exploration de l'espace?
Pas écologiques.

L'inspiration instrumentale

L'analyse présentée dans ce travail, ainsi que les différents concepts proposés pour décrire l'organologie des \glspl{DMI} contribueront je l'espère à la considération des caractéristiques qui lui son propre et à favoriser son épanouissement, en évitant de le contraindre trop rapidement dans des catégories se conformant au modèle de l'instrument acoustique. 

Cette perspective, qui suppose de se défaire d'un certain nombre d'a priori sur le rôle d'un instrument de musique, pose immanquablement des questions d'ordre esthétique voire philosophique sur les motivations musicales, qui dépassent la portée de ce travail. Il a ici été tenté d'en inclure quelques unes qui ont semblé essentielles : 
\begin{itemize}[noitemsep]
	\item l'aspect subversif de la création musicale (et artistique en général) qui vient remettre en question la notion de transparence régulièrement invoquée comme une qualité à poursuivre;
	\item l'éphémérité de la performance musicale, qui vient remettre en question les questions de stabilité et de pérénnité des objets également invoquées comme des objectifs nécessaires;
	\item la (re)distribution, lors de la conception des \glspl{DMI}, des fonctions de composition, d'interprétation, de performance, de production du son, de direction, d'écoute, et leur importance relative qui peuvent être confiées à l'humain comme à la machine, en prenant en compte les différences qu'une attribution à l'une ou à l'autre entraine, en termes de jeu pour le musicien comme en terme de perception par le public;
	\item l'inspiration instrumentale, qui peut être d'origine instrumentale (i.e. s'inspirer d'instruments existants), mais également extra-instrumentale, et s'appuyer sur le son lui-même, sur les gestes de performance,
	\item cette redistribution des rôles implique une inspiration instrumentale, qui puisse être, justement, extra-instrumentale, par exemple
\end{itemize}


Le magicien Yann Frisch évoquait la perte de mystère possible que pourrait entrainer l'omnipotence de la technologie,

un organe dont on connait les limites, par expérience, par empathie, est le corps humain. 
Dans un tel contexte d'omnipotence de la technologie, avec son immatérialité, la fulgurance de sa vitesse, son insensibilité à la fatigue, à la douleur, autant qu'à au plaisir et à la joie, pose la question du corps : ses limitées, sa fragilité, mais aussi ses ressources inconnues, expriment plus que jamais son rôle crucial dans la performance ``live'' avec les technologies numériques.