% !TEX root = ../thesis-example.tex
%
\chapter{Conclusion}
\label{ch:conclusion}

\cleanchapterquote{
\textnormal{MUSIQUE.} Coup de baguette et récapitulation des musiques précédentes ou musique source seule.\\ 
Un temps.\\ 
\textnormal{PAROLES. — Encore. (}Un temps. Implorant.\textnormal{) Encore ! \\
MUSIQUE.} Répète dernière musique telle quelle ou à peine variée.\\ 
Un temps.\\ 
\textnormal{PAROLES.} \textit{Profond soupir.}
}
{Samuel Beckett}{\textit{paroles et musique}, Pièce radiophonique, 1962. \cite{beckett_comeet_2014}}
\index[people]{becket@Beckett, Samuel!parolesetmusique@\textit{paroles et musique}}

ECRITURE DU SON
ECRITURE DU GESTE
ECRITURE DE L'INSTRUMENT
ECRITURE DE L'ESPACE
ECRITURE DE LA PERCEPTION AUDITIVE
ECRITURE DE LA PERCEPTION TACTILE

\noindent La motivation initiale de cette recherche était l'étude transversale des aspects de contrôle et de représentation à l'œuvre dans le design des \glspl{DMI}, à travers les différents contextes que ce design implique: conception, fabrication, programmation et pratique musicale. Au terme de cette étude, on se rend compte que les contours de la notion d'instrument de musique, de manière générale, sont assez mouvants. Leur histoire est marquée par des apports et des évolutions permanentes et leur pérennité ne s'établit que de manière relative, à l'aune d'un répertoire et d'une pratique associée. Mais si les instruments acoustiques présentent des évolutions plus ou moins rapides durant leur longue histoire, l'introduction de la computation numérique dans le fonctionnement même de l'instrument représente un tournant majeur, qui entraine un bouleversement fulgurant des pratiques qui lui sont associées [de la lutherie, à la performance en passant par la composition — en rendant leur frontières poreuse, sinon caduque].

\noindent En premier lieu, nous avons vu que différents facteurs concourent à l'instabilité et l'éphémérité des \glspl{DMI} (chapitre 2), ce qui nous amène à constater que, si certains de ces facteurs sont indésirables, telles que l'obsolescence --~parfois programmée~-- des outils technologiques, d'autres apparaissent comme des éléments inhérents à la performance musicale, et peuvent ainsi être recherchés. En effet, la musique, qui se déploie dans le temps en mouvements fugaces peut embrasser cette impermanence de l'instrument; la pratique de l'improvisation libre fournit des exemples évidents d'une telle instabilité, artistiquement fertile.


Les \glspl{DMI} ajoutent à cette instabilité \textit{l'atomisation} du corps instrumental --~c'est-à-dire la miniaturisation à l'extrême des composants qui le compose~--, causée par le design modulaire intrinsèque au fonctionnement de l'informatique. Les reconfigurations possibles de l'instrument numérique, envisagé comme un agencement de \textit{fonctions instrumentales}, qui se cristallisent dans une configuration contextuelle, lui ont fait perdre la relative stabilité qu'offrent les instruments acoustiques. À cette instabilité des \glspl{DMI} semble s'opposer l'émergence de composants ``recyclables'', qui dessinent en filigrane un \textit{répertoire} embryonnaire de fonctions instrumentales. L'évolution récente des logiciels de programmation audio semble y faire écho, en proposant des librairies de telles fonctions, exportables vers différentes plateformes, selon le contexte dans lequel l'instrument est amené à se matéraliser.\\
\indent Sur un terrain aussi mouvant, l'élaboration et la pratique d'un \gls{DMI} prennent la forme d'un processus co-dynamique, dans lequel s'opèrent des aller-retours permanents entre fabrication, programmation, pratique et écoute musicienne.
%Musiques écrites sur le sable mouvant de logiciels constamment mis à jour, phonographiées sur un sable-ciment des formats de distribution audio.
%La dissemblance des états de concentration psychique que requiert ces différentes activités implique toutefois de laisser à chacune le temps de se développer. 
Les artéfacts, aussi imprévus que fertiles (TODO: répétition fertile + haut), émergeant des usages et mésusages des \glspl{DMI} bénéficient d'une pratique active de la sérendipité, car bien souvent --~et cela s'applique d'autant plus à ceux usant des algorithmes génératifs de l'IA~-- les luthier·e·s numériques méconnaissent l'étendue de la palette sonore de leur instruments.

% TRANSITION VERS GESTE
L'immensité vertigineuse des musiques accessibles aujourd'hui d'un simple clic, et la possibilité récente de générer des morceaux complexes d'un simple ``prompt'', pose d'autant plus radicalement, pour les instrumentistes, la question du rôle du corps dans la performance musicale. Car malgré cette ``décorporalisation'' de l'interaction musicale, la musique et le rythme sont profondément, viscéralement inscrites dans notre chair. Cela parait évident pour qui rentre dans une boite de nuit, mais cela reste aussi vrai lorsque nous sommes sur un siège lors d'un concert de musique classique --~les recherches en neurosciences ont mis en évidence l'activation des zones du cerveau liées au contrôle moteur lors de la perception de motifs rythmiques par des auditeurs (TODO: forme inclusive) parfaitement immobiles.\footnote{TODO:ref?}.\\


\noindent L’étude croisée de l’instrument, du geste instrumental et de leur perception relative par le public nous a amenés, au troisième chapitre, à remettre en question la notion de ``transparence'' de l’instrument. Celle-ci, souvent invoquée comme une qualité nécessaire pour que le ou la musicienne puisse pleinement s’exprimer, nous est apparue inadéquate, sinon incomplète, pour décrire les intentions sous-jacentes à l’interaction instrumentale. La relation généralement causale et cohérente que présente l’instrument de musique acoustique apporte, de toute évidence, certaines preuves que la transparence de son fonctionnement peut, en effet, servir de modèle pour le design de l’interaction. Cependant, nous avons montré, en nous appuyant sur plusieurs exemples musicaux (dont certains basés sur l’instrument acoustique), que la création musicale s’appuie largement sur une subversion --~davantage qu’une transparence~-- de la relation gestuelle-sonore. En étudiant les différentes inférences à l’œuvre entre le geste, l’instrument et le résultat sonore, nous avons dessiné quelques-unes des stratégies possibles de subversion, qui s’appuient sur des soudures artificielles invisibles entre des inférences contradictoires. Le découplage possible entre geste et résultat sonore permet ce que Michel Chion appelait au cinéma une ``synchrèse'', c'est à dire une ``synthèse du synchronisme'' indépendante de toute logique rationnelle\cite{chion_audio-vision:_2013}. Cette spécificité des \glspl{DMI} ouvre, plus encore qu'avec les instruments acoustique, la possibilité, voire le besoin, d'une écriture du geste possédant une certaine autonomie, en contrepoint de la production sonore à laquelle elle est associée.

\noindent 

La physicalité du rapport entre le geste et l'instrument s'incrit ainsi dans une scénographie mettant en jeu le corps, l'espace de la performance et de potentiels objects concrets, supports de l'interaction musicale. L'analyse, au chapitre 4, des différents matériaux constituant le corps physique des \glspl{DMI} nous amène à constater qu'ils ne sont plus nécessairement des ``objets que l'on touche'', et qu'ils procèdent de différents héritages --~lutheries traditionnelles, design technico-industriel, schémas corporels, scénographie poétique de l'objet~-- qui s'agencent dans des proportions variables et propres au projet musical. Il en résulte un éventail très large de configurations instrumentales, de l'objet physique massif à sa disparition la plus complète, de sa référence à l'instrument traditionnel à sa prise de distance la plus manifeste, de sa fonctionnalité la plus technique à sa présence essentiellement scénographique et poétique.\\
\indent Les constituants matériels d'un \gls{DMI} étant très souvent, au moins partiellement, issus d'une fabrication industrielle dont les objectifs diffèrent de ceux de la création musicale, la conception de l'interface de jeu implique une part de détournement et de ``re-programmation'' de l'usage de ces objets.
%L'analyse de la conception du \textit{Filigramophone} et du \textit{Xypre} a clairement mis en évidence ces aspects. 
Les pratiques ``artisanales'' du \glslink{DIY}{``\textit{Do It Yourself}''} ou du \glslink{DIWO}{``\textit{Do It With Others}''} des lutheries numériques vient affirmer la singularité des instruments et constitue le versant complémentaire à la fabrication industrielle en série des objets techniques qui implique, à l'inverse, des millions d'autres personnes et des usages génériques.

Il est bien difficile de dire ce qui constitue la musique, qui ne saurait être réduite à la production d'une suite de notes, comme pourrait le faire un algorithme. clairement pas une suite de notes
%constituent pour autant des vecteurs polarisant 

Tous ces champs de la création musicale rendus 

%Ces soudures nécessitent notamment la possibilité de reconnecter des modèles entre eux (cf. MP)

Ce rapport distancié entre le geste, l'interface de jeu (c'est à dire, l'instrument, dépourvu de sa logique de production sonore) et le résultat sonore, nous amène à cette aire la plus sujette à l'écriture ad-hoc: la logique algorithmique de l'instrument. Celle-ci intègre tout autant les relations de correspondance geste-son (le ``mapping''),  la synthèse sonore et sa spatialisation, la représentation des données pour l'instrumentiste et/ou le public, ainsi qu'une partie compositionnelle de matériaux pré-enregistrés et ``gestes programmés''. 

En tant que données symboliques, le code de cette part de l'instrument est sujette à toutes ces ``tâches'' que l'on peut attendre de la part de la machine: la rapidité d'éxécution, l'optimisation, la génération automatique, voire, nous promet-on aujourd'hui, l'invention pure et simple à l'aide des ``grands modèles de langage''\footnote{``Large language models'' en anglais, tels qu'utilisés par l'agent conversationel chatGPT}. Mais si une vertu qui découle des pouvoirs de la machine est la possibilité d'explorer des recoins jusque là inaccessibles ou d'éclairer des zones inconnues, ``l'inadéquation de l'ordinateur [en tant qu'instrument] a été évidente depuis le début'', comme le notait Joel Ryan \cite{ryan_remarks_1991}, étant donné que les musicien.n.es ne recherchent pas l'absence d'effort que propose la plupart des programmes. La réponse donnée par le chanteur Nick Cave à un fan lui demandant ce qu'il pense d'une chanson écrite par ChatGPT ``dans le style de Nick Cave'' est éclairante sur cette question:

\begin{quotation}
Les chansons naissent de la souffrance, je veux dire par cela une qu'elles sont fondées sur la lutte interne et complexe des humains dans l'acte de création et, pour autant que je sache, les algorithmes ne ressentent rien. Les données ne souffrent pas. ChatGPT n'a pas d'être intérieur, il n'a été nulle part, il n'a rien enduré, il n'a pas eu l'audace d'aller au-delà de ses limites et, par conséquent, il n'a pas la capacité de partager une expérience transcendante, puisqu'il n'a pas de limites à transcender.\footnote{Nick Cave, The Red Hand Files TODO: rajouter Nick Cave dans la liste des artistes cités}
\end{quotation}

Sans nécessairement en passer par la souffrance chrétienne chère à Nick Cave, la création artistique requiert un cheminement personnel qu'il serait vain d'imaginer court-circuiter en le réduisant à de simples choix sur un catalogue de proposition ready-made.



\noindent L'étude des matériaux algorithmiques au chapitre 5 a fait ressortir deux composantes essentielles venant articuler le fonctionnement des \glspl{DMI}. D'une part la notion de ``modèle intermédiaire'', comme unité de transformation de signaux, permet de définir les relation désirées entre les gestes de l'instrumentiste, les gestes programmés dans l'instrument et la production sonore. D'autre part, l'interconnexion entre ces modèles intermédiaires et les entrées/sorties de l'interface instrumentale nécessite de définir une syntaxe des flux circulants dans ce graphe relationnel. Les notions de signal synchrone et asynchrone reflètent la distinction perceptive entre le continu et le discret et se traduisent généralement dans des protocoles de communication distincts. 

L'élaboration de tels protocoles relève autant de la science informatique que de la musicologie et de la recherche en création. 
En effet, la création artistique se satisfait rarement des contraintes imposées par des protocoles trop rigides et la conception des instruments, comme des œuvres, ne saurait être automatisée, contrairement à ce que prétendent certains discours sur l'IA

%On entend fréquemment ces dernières années que l'IA pourrait générer des œuvres et ainsi remplacer les créateurs. Celles et ceux qui croient ces discours ou les clament confondent, il me semble, la capacité des machines à suivre une règle et générer un résultat respectant ces règles, et les fonctions de l'art qui consiste à s'affranchir de le norme, d'une part, et à s'adresser à cette part de nous qui connait la  écho à la fragilité


Nous avons proposé un protocole (MP) permettant d'articuler ces deux aspects, de manière à permettre la création de soudures entre les modèles intermédiaires, évoquées lors de l'analyse de la relation geste/instrument/son. Le dessein principal de ce protocole est de permettre des soudures suffisamment fortes entre les modèles (en particulier, polyphoniques et continues) pour que puisse y circuler toute la richesse des modulations souhaitées, et suffisamment souples pour permettre des ré-affectations dynamiques à l'envie.

% cf "low coupling / high cohesion"

% De la même manière que la photographie et le cinéma ont renouvellé les motivations sous-jacentes à la peinture, qui s'est départie de la recherche de réalisme au profit l'exploration de nouvelles formes de rapport entre les couleurs, les formes, nos sensations et notre esprit, la musique a elle aussi poursuivi de telles métamorphoses.


\noindent Sur la base de ce protocole de communication, nous avons ensuite élaboré un système de représentation graphique (mp.TUI) qui tire parti de la technologie des écrans tactiles \textit{multitouch}, pour co-articuler contrôle et représentation dans des objets graphiques interactifs. Les possibilités de personnalisation de ces représentations sont au centre de la conception de cette librairie logicielle et proposent une réponse concrète aux besoins spécifiques et contextuels, qui viennent affirmer l'identité visuelle de l'instrument, au-delà du simple aspect fonctionnel de son contrôle et de son \textit{monitoring}. La dimension esthétique du design de l'interface graphique permet de nouveaux modes d'interaction et de représentation avec des modèles complexes, en même temps qu'elle constitue un support d'expression visuelle à l'attention du public.

\noindent La pratique musicale, notamment au sein de l'ensemble ONE, a permis de confronter les différents outils conceptuels et logiciels développés durant ce travail de recherche et d'évaluer continûment leur pertinence (ou non) dans le présent du jeu musical. L'absence d'idiomes musicaux sur lequel s'appuyer pour l'improvisation collective nous a amenés à effectuer une \iquote{entomologie musicale} et constituer un répertoire de \iquote{karmas}, unités sémiotiques singulières extraites de notre pratique d'ensemble. L'élaboration de ``John'', un outil collaboratif pour la \textit{comprovisation} est également venu concrétiser un certain nombre d'aspects liés à la manière dont l'interaction musicale se construit, non seulement \textit{entre l'instrumentiste et son instrument}, mais également \textit{entre les instruments eux-mêmes} --~qui peuvent s'échanger des données, et surtout, \textit{entre les musiciens} qui s'emparent, dans une anticipation permanente, des gestes et des sons de leurs partenaires.


\noindent Cette reconfiguration générale de l'agencement possible d'un instrument permet d'écrire (de programmer) sur un médium commun (l'ordinateur), à la fois les relations gestuelles-sonores autrefois séparées dans les domaines distincts de la lutherie et de la composition, ainsi que de modéliser les mécanismes cognitifs propres à la perception de ces relations, mais encore de composer spécifiquement la représentation visuelle des ces instruments et leur inscription scénographique. 


Les conséquences de cette \iquote{notation instrumentale} pourraient s'apparenter à celles provoquées, il y a un millénaire, par l'émergence de la notation musicale. Notamment, si cette dernière a engendré une profusion de compositions --~dont toutes n'ont pas traversé le temps~-- ainsi que le développement d'une théorie musicale riche et fertile sur les manières d'organiser les sons, la \textit{notation instrumentale} engendre une profusion de nouveaux instruments et de nouvelles interactions musicales dont on ne peut qu'imaginer l'ampleur des développements possibles dans le futur.


Les \glspl{DMI} permettent tout cela. Ecrire l'instrument, écrire le geste, écrire l'espace, écrire la représentation de manière nouvelle à chaque performance. Cela ne signifie pas pour autant que toutes ces écritures soient nécessaires ou souhaitables à chaque fois. Dans notre société contemporaine saturée d'images et de sons, le minimalisme et la simplicité du dispositif instrumental permet tout autant de richesse dans la performance. La saturation médiatique ``horizontale'' à laquelle nous sommes exposés se fait souvent au dépens d'une investigation ``verticale'', en profondeur. Qui aujourd'hui prend le temps d'écouter un album en entier sur les plateformes de streaming ? Qui prend le temps d'une ``écoute profonde'', comme le prônait Pauline Oliveros, de toute la richesse sonore qui se déploie dans des sons très simples?

Les instruments de musique sont des outils de connaissance de notre corps et de notre esprit, et la puissance des technologies numériques donne accès à une incroyable quantité d'information, à qui veut bien les chercher. Mais ces outils numériques, connectés comme jamais au rythme de l'économie du monde, peuvent aussi être, paradoxalement, des outils d'aliénation de notre attention et de notre conscience, des outils d'isolement social, des outils excluant d'autant plus violemment celles y ceux qui n'y ont pas accès; toutes choses dont nous commençons à mesurer les effets délétères. Les composants utilisés des technologies numériques --~terres rares, métaux lourds~-- sont également une source de destruction de notre environnement réellement alarmante, qu'il est impensable d'ignorer aujourd'hui.

%Il fût un temps lointain où la seule et unique manière de transmettre une œuvre musicale, ou les savoir-faire liés à la pratique d'un instrument était la tradition orale/aurale, de maître à élève. Si cette tradition est loin d'avoir totalement disparue de nos jours, on peut constater que l'élaboration de systèmes de notation musicale, à partir de l'antiquité, a progressivement permis de transmettre des œuvres ainsi que des savoir-faire, au moins en partie, sans avoir recours à la performance musicale elle-même.

% Ce champ de possibilités, qui constitue, d'après les interviews réalisées, une des motivations majeures pour l'utilisation des \glspl{DMI}, implique des choix de design dont la singularité est une marque manifeste. 


\noindent Le poète et mathématicien Jacques Roubaud confiait: \iquote{La poésie dit ce qu’elle dit en le disant. Ce qui exclut la paraphrase. On peut exposer ce que la poésie raconte, mais, si on le fait, on perd ce quelque chose d’essentiel qui est la poésie. À l’extrême opposé, la mathématique ne se développe qu’en se paraphrasant, qu’en se redisant d’une manière différente}. On court ainsi toujours le risque, lorsqu'on cherche à expliquer les motivations sous-jacentes aux outils de création, souvent fortement entrelacées avec celles des œuvres elles-mêmes, de refermer l'horizon qu'elles ouvrent. Au terme des quelques décennies d'existence des instruments de musique numériques, il reste nécessaire de prendre garde à la calcification précoce de concepts développés lors de l'étude de ces instruments nouveaux, qui pourraient donner lieu à des fondations solides mais stériles. J'espère aussi que les idées, les catégories et les modèles proposés dans ce travail ne seront pas pris pour des règles strictes, mais contribueront au contraire à nourrir le champ des questions, davantage que celui des réponses, ayant trait à l'organologie des \glspl{DMI}, ainsi qu'à l'épanouissement des possibilités créatives qu'ils offrent. Ils ont été, du reste, une manière pour moi de \iquote{travailler mon instrument}.


Qu'est ce que l'instrument peut nous apprendre de nous même? Que peut-il nous apprendre du monde dans lequel nous vivons? Que nous apprend il de l'Histoire et quel futur imaginable peut il nous offrir dans le présent de la performance ?


En se numérisant, les organes de l'instrument de musique deviennent sujet à écriture

% Cette perspective, qui suppose de se défaire d'un certain nombre d'a priori sur le rôle d'un instrument de musique, pose immanquablement des questions d'ordre esthétique voire philosophique sur les motivations musicales, qui dépassent la portée de ce travail. Il a ici été tenté d'en inclure quelques unes qui ont semblé essentielles : 
% \begin{itemize}[noitemsep]
% 	\item l'aspect subversif de la création musicale (et artistique en général) qui vient remettre en question la notion de transparence régulièrement invoquée comme une qualité à poursuivre;
% 	\item l'éphémérité de la performance musicale, qui vient remettre en question les notions de stabilité et de pérénnité des objets également invoquées comme des qualités nécessaires;
% 	\item la (re)distribution, lors de la conception des \glspl{DMI} de l'importance relative des fonctions de composition, d'interprétation, de performance, de production du son, de direction, d'écoute, qui peuvent être confiées à l'humain comme à la machine, en prenant en compte les différences qu'une attribution à l'une ou à l'autre entraîne, en termes de jeu pour le musicien comme en terme de perception par le public;
% 	\item l'inspiration instrumentale, qui peut être d'origine instrumentale (i.e. s'inspirer d'instruments existants), mais également extra-instrumentale, et s'appuyer sur le son lui-même, sur les gestes de performance,
% 	\item cette redistribution des rôles implique une inspiration instrumentale, qui puisse être, justement, extra-instrumentale, par exemple
% \end{itemize}


%Le magicien Yann Frisch évoquait la perte de mystère possible que pourrait entraîner l'omnipotence de la technologie,

%un organe dont on connait les limites, par expérience, par empathie, est le corps humain. 
%Dans un tel contexte d'omnipotence de la technologie, avec son immatérialité, la fulgurance de sa vitesse, son insensibilité à la fatigue, à la douleur, autant qu'à au plaisir et à la joie, pose la question du corps : ses limitées, sa fragilité, mais aussi ses ressources inconnues, expriment plus que jamais son rôle crucial dans la performance ``live'' avec les technologies numériques.

