% !TEX root = ../thesis-example.tex
%
\chapter{Conclusion}
\label{ch:conclusion}

\cleanchapterquote{
\textnormal{MUSIQUE.} Coup de baguette et récapitulation des musiques précédentes ou musique source seule.\\ 
Un temps.\\ 
\textnormal{PAROLES. — Encore. (}Un temps. Implorant.\textnormal{) Encore ! \\
MUSIQUE.} Répète dernière musique telle quelle ou à peine variée.\\ 
Un temps.\\ 
\textnormal{PAROLES.} \textit{Profond soupir.}
}
{Samuel Beckett}{\textit{paroles et musique}, Pièce radiophonique, 1962. \cite{beckett_comeet_2014}}


\noindent La motivation initiale de ce travail était l'étude transversale des aspects de contrôle et de représentation à l'œuvre dans le design des \glspl{DMI}, à travers les différents contextes que ce design implique: conception, fabrication, programmation et pratique musicale. Au terme de cette étude, on se rend compte que les contours de la notion d'instrument de musique en général sont mouvants, que son histoire est marquée par des apports et des évolutions permanentes et que sa pérennité ne s'établit que de manière relative, à l'aune d'un répertoire et d'une pratique associée.

\noindent Au second chapitre, nous avons étudié les différentes causes concourant à l'instabilité et l'éphémérité des \glspl{DMI}. Certaines sont indésirables, telles que l'obsolescence --~parfois programmée~-- des outils technologiques, mais d'autres apparaissent comme un élément intrinsèque à la performance musicale, et peuvent être souhaitées en tant que polarité artistiquement fertile. Les \glspl{DMI} ajoutent à cette instabilité l'atomisation --~c'est-à-dire la modularisation à l'extrême~-- du corps instrumental dû à la modularité intrinsèque au fonctionnement de l'informatique. Les reconfigurations possibles de l'instrument numérique, envisagé comme un agencement de fonctions instrumentales, qui se cristallisent dans une configuration contextuelle, lui ont fait perdre la stabilité relative qu'offrent les instruments acoustiques. À cette méta-instabilité semble s'opposer une émergences de composants recyclables qui dessinent en filigrane un répertoire embryonnaire de fonctions instrumentales. L'évolution récente des logiciels de programmation audio semble y faire écho, en proposant des librairies de modèles exportables contextuellement vers différentes plateformes.\\
\indent Sur ce terrain aussi instable que des sables mouvants, l'élaboration et la pratique d'un \gls{DMI} prennent la forme d'un processus co-dynamique, dans lequel s'opèrent des aller-retours permanents entre fabrication, programmation, pratique et écoute musicienne. La dissemblance des états de concentration psychique que requiert ces différentes activités implique toutefois de laisser à chacune le temps de se développer. Les artéfacts aussi imprévus que fertiles émergeant de l'usage, voire de l'abus, des \glspl{DMI} bénéficie d'une pratique active de la sérendipité.

\noindent L'étude croisée de l'instrument, du geste instrumental et de leur perception relative par le public nous a amené, au troisième chapitre, à remettre en question la notion de ``transparence'' de l'instrument. Celle-ci, souvent invoquée comme une qualité nécessaire pour que le musicien puisse pleinement s'exprimer, nous est apparue inadéquate, sinon incomplète, pour décrire les intentions sous-jacentes à l'interaction instrumentale. La relation généralement causale et cohérente que présente l'instrument de musique acoustique apporte, de toute évidence, certaines preuves que la transparence de son fonctionnement peut, en effet, servir de modèle pour le design de l'interaction. Cependant, nous avons montré, en nous appuyant sur plusieurs exemples musicaux (dont certains basés sur l'instrument acoustique), que la création musicale s'appuie largement sur une subversion --~davantage qu'une transparence~-- de la relation gestuelle-sonore. En étudiant les différentes inférences à l'œuvre entre le geste, l'instrument et le résultat sonore, nous avons dessiné quelques unes des stratégies possibles de subversion, qui s'appuient sur des soudures artificielles invisibles entre des inférences contradictoires.

\noindent L'analyse, au chapitre 4, des différents matériaux constituant le corps physique des \glspl{DMI} a mis en évidence qu'ils ne sont plus nécessairement des ``objets que l'on touche'', et qu'ils procèdent de différents héritages --~lutheries traditionnelles, design technico-industriel, schémas corporels, scénographie poétique de l'objet~-- qui s'agencent dans des proportions variables et propres au projet musical. Il en résulte un éventail très large de configurations instrumentales, de l'objet physique massif à sa disparition la plus complète, de sa référence à l'instrument traditionnel à sa prise de distance la plus manifeste, de sa fonctionnalité la plus technique à sa présence essentiellement scénographique et poétique.\\
\indent Les constituants matériels d'un \gls{DMI} étant très souvent, au moins partiellement, issus d'une fabrication industrielle dont les objectifs diffèrent de ceux de la création musicale, la conception de l'interface de jeu implique fréquemment une part de détournement et de ``re-programmation'' de l'usage de ces objets. L'analyse de la conception du \textit{Filigramophone} et du \textit{Xypre} a clairement mis en évidence ces aspects. La pratique \glslink{DIY}{``\textit{Do It Yourself}''} des lutheries numériques vient affirmer la singularité des instruments et constitue le versant complémentaire à la fabrication industrielle en série des objets techniques qui implique, au contraire, des millions d'autres personnes et des usages génériques.

%constituent pour autant des vecteurs polarisant 

%Ces soudures nécessitent notamment la possibilité de reconnecter des modèles entre eux (cf. MP)

\noindent L'étude des matériaux algorithmiques au chapitre 5 a fait ressortir deux composantes essentielles venant articuler le fonctionnement des \glspl{DMI}. D'une part la notion de ``modèle intermédiaire'', comme unité de transformation de signaux, permet notamment de définir les relation désirées entre les gestes de l'instrumentiste, les gestes programmés dans l'instrument et la production sonore. D'autre part, l'interconnexion entre ces modèles intermédiaires et les entrées/sorties de l'interface instrumentale nécessite de définir une syntaxe des flux circulants dans ce graphe relationnel. Les notions de signal synchrone et asynchrone reflètent la distinction perceptive entre le continu et le discret et se traduisent généralement dans des protocoles de communication distincts. Nous avons proposé un protocole (MP) permettant d'articuler ces deux aspects, de manière à permettre la création de soudures entre les modèles intermédiaires, évoquées lors de l'analyse de la relation geste/instrument/son. Le dessein principal de ce protocole est de permettre des soudures suffisamment fortes entre les modèles (en particulier, polyphoniques et continues) pour que puisse y circuler toute la richesse des modulations souhaitées, et suffisamment souples pour permettre des ré-affectations dynamiques à l'envie.

\noindent Sur la base de ce protocole de communication, nous avons ensuite élaboré un système de représentation graphique (mp.TUI) qui tire parti de la technologie des écrans tactiles \textit{multitouch}, pour co-articuler contrôle et représentation dans des objets graphiques interactifs. Les possibilités de personnalisation de ces représentations sont au centre de la conception de cette librairie logicielle et proposent une réponse concrète aux besoins spécifiques et contextuels, qui viennent affirmer l'identité visuelle de l'instrument, au-delà du simple aspect fonctionnel de son contrôle et de son \textit{monitoring}. La dimension esthétique du design de l'interface graphique permet de nouveaux modes d'interaction et de représentation avec des modèles complexes, en même temps qu'elle constitue un support d'expression visuelle à l'attention du spectateur/auditeur.

\noindent La pratique musicale, notamment au sein de l'ensemble ONE, a permis de confronter les différents outils conceptuels et logiciels développés durant ce travail de recherche et d'évaluer continûment leur pertinence (ou non) dans le présent du jeu musical. L'absence d'idiomes musicaux sur lequel s'appuyer pour l'improvisation collective nous a amenés à effectuer une \iquote{entomologie musicale} et constituer un répertoire de \iquote{karmas}, unités sémiotiques singulières extraites de notre pratique d'ensemble. L'élaboration de ``John'', un outil collaboratif pour la \textit{comprovisation} est également venu concrétiser un certain nombre d'aspects liés à la manière dont l'interaction musicale se construit, non seulement \textit{entre le musicien et son instrument}, mais également \textit{entre les instruments eux-mêmes} --~qui peuvent s'échanger des données, et surtout, \textit{entre les musiciens} qui s'emparent, dans une anticipation permanente, des gestes et des sons de leurs partenaires.


\noindent Cette reconfiguration générale de l'agencement possible d'un instrument permet d'écrire (de programmer) sur un médium commun (l'ordinateur), à la fois les relations gestuelles-sonores autrefois séparées dans les domaines distincts de la lutherie et de la composition, ainsi que de modéliser les mécanismes cognitifs propres à la perception de ces relations. Les conséquences de cette \iquote{notation instrumentale} pourraient s'apparenter à celles provoquées, il y a un millénaire, par l'émergence de la notation musicale. Notamment, si cette dernière a engendré une profusion de compositions --~dont toutes n'ont pas traversé le temps~-- ainsi que le développement d'une théorie musicale riche et fertile sur les manières d'organiser les sons, la \textit{notation instrumentale} engendre une profusion de nouveaux instruments et de nouvelles interactions musicales dont on ne peut qu'imaginer l'ampleur des développements possibles dans le futur.

% Si comme le dit John Cage, ``nous n'avons pas besoin de tradition si nous nous affranchissons de la mémoire''
% En retournant la formule de John Cage, si nous nous affranchissons de la mémoire, nous pouvons inventer de nouvelles formes

% La notation musicale a permis à la composition de s'affranchir, sans toutefois l'oublier, de la musique traditionnelle pour inventer des formes nouvelles, que l'histoire a transformé en traditions nouvelles. 

% Il s'avère pareillement nécessaire pour la notation instrumentale de s'affranchir, sans toutefois l'oublier, de la lutherie traditionnelle.

% de nouvelles organisations de l'interaction musicale

% pourrait engendrer, si l'on s'affranchit sans l'oublier pour autant, de nouvelles organisations de l'interaction musicale.


% Ce champ de possibilités, qui constitue, d'après les interviews réalisées, une des motivations majeures pour l'utilisation des \glspl{DMI}, implique des choix de design dont la singularité est une marque manifeste. 


\noindent Le poète et mathématicien Jacques Roubaud confiait, \iquote{La poésie dit ce qu’elle dit en le disant. Ce qui exclut la paraphrase. On peut exposer ce que la poésie raconte, mais, si on le fait, on perd ce quelque chose d’essentiel qui est la poésie. À l’extrême opposé, la mathématique ne se développe qu’en se paraphrasant, qu’en se redisant d’une manière différente}.\\
\indent On court ainsi toujours le risque, lorsqu'on cherche à expliquer les motivations sous-jacentes aux outils de création, souvent fortement entrelacées avec celles des œuvres elles-mêmes, de refermer l'horizon qu'elles ouvrent. Au terme des quelques décennies d'existence des instruments de musique numériques, il reste nécessaire de  prendre garde à la calcification précoce de concepts développés lors de l'étude de ces instruments nouveaux, qui pourraient donner lieu à des fondations solides mais stériles.\\
\indent J'espère aussi que les concepts, les catégories et les modèles proposés dans ce travail contribueront à nourrir le champ des questions, peut-être davantage que celui des réponses, ayant trait à l'organologie des \glspl{DMI}, ainsi qu'à l'épanouissement des possibilités créatives qu'ils offrent. Ils ont été, du reste, une manière pour moi de \iquote{travailler mon instrument}.



% Cette perspective, qui suppose de se défaire d'un certain nombre d'a priori sur le rôle d'un instrument de musique, pose immanquablement des questions d'ordre esthétique voire philosophique sur les motivations musicales, qui dépassent la portée de ce travail. Il a ici été tenté d'en inclure quelques unes qui ont semblé essentielles : 
% \begin{itemize}[noitemsep]
% 	\item l'aspect subversif de la création musicale (et artistique en général) qui vient remettre en question la notion de transparence régulièrement invoquée comme une qualité à poursuivre;
% 	\item l'éphémérité de la performance musicale, qui vient remettre en question les notions de stabilité et de pérénnité des objets également invoquées comme des qualités nécessaires;
% 	\item la (re)distribution, lors de la conception des \glspl{DMI} de l'importance relative des fonctions de composition, d'interprétation, de performance, de production du son, de direction, d'écoute, qui peuvent être confiées à l'humain comme à la machine, en prenant en compte les différences qu'une attribution à l'une ou à l'autre entraîne, en termes de jeu pour le musicien comme en terme de perception par le public;
% 	\item l'inspiration instrumentale, qui peut être d'origine instrumentale (i.e. s'inspirer d'instruments existants), mais également extra-instrumentale, et s'appuyer sur le son lui-même, sur les gestes de performance,
% 	\item cette redistribution des rôles implique une inspiration instrumentale, qui puisse être, justement, extra-instrumentale, par exemple
% \end{itemize}


%Le magicien Yann Frisch évoquait la perte de mystère possible que pourrait entraîner l'omnipotence de la technologie,

%un organe dont on connait les limites, par expérience, par empathie, est le corps humain. 
%Dans un tel contexte d'omnipotence de la technologie, avec son immatérialité, la fulgurance de sa vitesse, son insensibilité à la fatigue, à la douleur, autant qu'à au plaisir et à la joie, pose la question du corps : ses limitées, sa fragilité, mais aussi ses ressources inconnues, expriment plus que jamais son rôle crucial dans la performance ``live'' avec les technologies numériques.

