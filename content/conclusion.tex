% !TEX root = ../thesis-example.tex
%
\chapter{Conclusion}
\label{ch:conclusion}

\cleanchapterquote{
\textnormal{MUSIQUE.} Coup de baguette et récapitulation des musiques précédentes ou musique source seule.\\ 
Un temps.\\ 
\textnormal{PAROLES. — Encore. (}Un temps. Implorant.\textnormal{) Encore ! \\
MUSIQUE.} Répète dernière musique telle quelle ou à peine variée.\\ 
Un temps.\\ 
\textnormal{PAROLES.} \textit{Profond soupir.}
}
{Samuel Beckett}{\textit{paroles et musique}, Pièce radiophonique, 1962. \cite{beckett_comeet_2014}}

TODO

\noindent La motivation initiale de ce travail était l'étude transversale des aspects de contrôle et de représentation dans le design des \gls{DMI}, à travers les différents contextes qu'il implique: conception, fabrication, programmation et pratique musicale. Au terme de cette étude, on se rend compte que les contours de la notion d'instrument de musique en général sont mouvants, que son histoire est marquée par des apports et des évolutions permanentes et que sa pérennité ne s'établit que de manière relative, à l'aune d'un répertoire et d'une pratique associée.\\

\indent Les \glspl{DMI} ajoutent à cette instabilité l'atomisation[, c'est-à-dire la modularisation à l'extrême,] du corps instrumental dû à la modularité intrinsèque au fonctionnement de l'informatique. Les reconfigurations possibles de l'instrument numérique, envisagé comme un agencement de fonctions instrumentales qui se cristallisent dans une configuration contextuelle, lui ont fait perdre la stabilité relative que présentent les instruments acoustiques. A cette méta-instabilité semble s'opposer, en filigrane, une cristalisation de sous-éléments qui dessinent un répertoire naissant de fonctions instrumentales. 
L'évolution récente des logiciels de programmation audio semble y faire écho, en proposant des librairies de modèles exportables vers diverses plateformes.


Dans le second chapitre, nous avons étudié les différentes causes concourant à l'instabilité et l'éphémérité des \glspl{DMI}, (que nous avons considéré comme une spécificité intrinsèque à la fois au  medium numérique naissant, mais aussi au fait musicale lui-même) mais nous l'avons envisagé comme une modalité quasi intrinsèque à l'agencement contextuel invoqué par la perforance musicale.

L'étude croisée de l'instrument, du geste instrumental et de leur perception par le public nous a amené, au 3ème chapitre, à remettre en question la notion de ``transparence'' de l'instrument. Celle-ci, souvent invoquée comme une qualité nécessaire pour que le musicien puisse pleinement s'exprimer, nous est apparue incomplète. La relation généralement causale et cohérente que présente l'instrument de musique acoustique apporte certaines preuves que la transparence de son fonctionnement permet l'expression musicale, et peut en effet servir de modèle pour le design des interactions musicales. Cependant, nous avons montré, en nous appuyant sur plusieurs exemples musicaux (dont certains ``extra-numériques''), que la création musicale s'appuie largement sur une subversion --~davantage qu'une transparence~-- de la relation gestuelle-sonore. En étudiant les différentes inférences ayant lieu entre le geste, l'instrument et le résultat sonore, nous avons dessiné quelques unes des stratégies possibles de subversion, envisagées comme soudures invisibles entre des inférences contradictoires.


Le chapitre 4 nous a permis de voir les différentes composantes physiques constituant les \glspl{DMI}, qui ne sont plus nécessairement des ``objets que l'on touche'', mais procèdent de différents héritages --~lutheries traditionelles, design technico-industriels, scénographie poétique de l'objet~-- qui s'agencent dans des proportions variables et propres au projet musical. Il en résulte un éventail très large de configurations instrumentales, de l'objet physique massif à sa disparition la plus complète, de sa référence à l'instrument traditionnel à sa prise de distance la plus manifeste, de sa fonctionnalité la plus technique à sa présence essentiellement scénographique et poétique.
Les constituants matériels d'un \glspl{DMI} étant très souvent, au moins partiellement, issus d'une fabrication industrielle à vocation non-instrumentale, la conception de l'interface de jeu implique fréquemment une part de détournement, de ``reprogrammation'' de l'usage de ces objets. La description de la conception du filigramophone et du Xypre a mis en évidence ces aspects. La pratique \glslink{DIY}{``\textit{Do It Yourself}''} des lutheries numériques constitue le versant complémentaire à la fabrication industrielle en série des objets techniques qui, elle, implique ``\textit{des millions d'autres personnes}''.

constituent pour autant des vecteurs polarisant 

Ces soudures nécessitent notamment la possibilité de reconnecter des modèles entre eux (cf. MP)


Dans ce contexte d'atomisation extrême des fonctions de l'instrument, un facteur essentiel à leur assemblage est la possibilité de créer des soudures suffisament fortes entre ces fonctions, pour que puisse y circuler toute la richesse des modulations, et suffisamment souples pour permettre des ré-affectations créatives à l'envie. C'est ce dessein qui a guidé la conception du protocole MP et des modules de synthèse sonore et graphique qui l'utilise.

un besoin est de trouver les modes de resonances possible entre l'expressivité des gestes de l'instrumentiste et l'expressivité des gestes programmés de l'instrument. Cette résonance peut s'établir à l'aide de protocoles de communication 

Cette reconfiguration générale de l'agencement possible d'un instrument permet d'écrire sur un medium commun, à la fois les relations gestuelles-sonores autrefois séparées dans les domaines distincts de la lutherie, de la composition, et d'autre part les mécanismes cognitifs propres à la perception de ces relations.


Ce champ de possibilités, qui constitue, d'après les interviews réalisées, une des motivations majeures pour l'utilisation des \glspl{DMI}, implique des choix de design dont la singularité est une marque manifeste. 

La singularité déroutante de ces nouveaux instruments (insatisfaisante pour certains), peut amener à tenter de définir le socle définissant



Plusieurs aspects ont été soulignés qui se prête à l'invention : 
\vspace{-1em}
\begin{itemize}[noitemsep]
	\item la programmation, dont l'étude au chapitre \ref{ch:algorithms} a montré qu'au delà de l'inter-connection entre des capteurs et des paramètres de synthèse, il s'agit de l'endroit principal où se définit le comportement de l'instrument. Nous avons proposé le concept de ``modèle intermédiaire'' comme objet conceptuel représentant une unité de traitement matérialisant de manière algorithmique une transformation du signal, qu'il soit audio, gestuel, pour désigner des modules de traitement pouvant impliquer 
\end{itemize}

Maurice Conti, TED talk the incredible inventions of intuitive ai:
\vspace{-1em}
\begin{itemize}[noitemsep]
	\item Things fabricated => things farmed
	\item constructed => grown
	\item isolated => connected
	\item extraction => aggregation
	\item obedience => autonomy
\end{itemize}


Le mathématicien et poète Jacques Roubaud confiait, ``La poésie dit ce qu’elle dit en le disant. Ce qui exclut la paraphrase. On peut exposer ce que la poésie raconte, mais, si on le fait, on perd ce quelque chose d’essentiel qui est la poésie. À l’extrême opposé, la mathématique ne se développe qu’en se paraphrasant, qu’en se redisant d’une manière différente.''
On court ainsi toujours le risque, lorsqu'on cherche à expliquer les motivations sous-jacentes aux outils de création, souvent fortement entrelacées avec celles des œuvres elles-mêmes, de refermer l'horizon qu'elles ouvrent.

La musique contient le futur des instruments.


Après 30 ans de NIME, il faut prendre garde à la calcification de certains concepts en des fondations solides. 

Quel intérêt à construire des instruments éphémère (cf. Cage) ?
Même que la construction de fusée à usage unique pour l'exploration de l'espace?
Pas écologiques.

L'inspiration instrumentale

L'analyse présentée dans ce travail, ainsi que les différents concepts proposés pour décrire l'organologie des \glspl{DMI} contribueront je l'espère à la considération des caractéristiques qui lui son propre et à favoriser son épanouissement, en évitant de le contraindre trop rapidement dans des catégories se conformant au modèle de l'instrument acoustique. 

Cette perspective, qui suppose de se défaire d'un certain nombre d'a priori sur le rôle d'un instrument de musique, pose immanquablement des questions d'ordre esthétique voire philosophique sur les motivations musicales, qui dépassent la portée de ce travail. Il a ici été tenté d'en inclure quelques unes qui ont semblé essentielles : 
\begin{itemize}[noitemsep]
	\item l'aspect subversif de la création musicale (et artistique en général) qui vient remettre en question la notion de transparence régulièrement invoquée comme une qualité à poursuivre;
	\item l'éphémérité de la performance musicale, qui vient remettre en question les notions de stabilité et de pérénnité des objets également invoquées comme des qualités nécessaires;
	\item la (re)distribution, lors de la conception des \glspl{DMI} de l'importance relative des fonctions de composition, d'interprétation, de performance, de production du son, de direction, d'écoute, qui peuvent être confiées à l'humain comme à la machine, en prenant en compte les différences qu'une attribution à l'une ou à l'autre entraine, en termes de jeu pour le musicien comme en terme de perception par le public;
	\item l'inspiration instrumentale, qui peut être d'origine instrumentale (i.e. s'inspirer d'instruments existants), mais également extra-instrumentale, et s'appuyer sur le son lui-même, sur les gestes de performance,
	\item cette redistribution des rôles implique une inspiration instrumentale, qui puisse être, justement, extra-instrumentale, par exemple
\end{itemize}


Le magicien Yann Frisch évoquait la perte de mystère possible que pourrait entrainer l'omnipotence de la technologie,

un organe dont on connait les limites, par expérience, par empathie, est le corps humain. 
Dans un tel contexte d'omnipotence de la technologie, avec son immatérialité, la fulgurance de sa vitesse, son insensibilité à la fatigue, à la douleur, autant qu'à au plaisir et à la joie, pose la question du corps : ses limitées, sa fragilité, mais aussi ses ressources inconnues, expriment plus que jamais son rôle crucial dans la performance ``live'' avec les technologies numériques.




Dans le second chapitre, nous avons étudié les différentes causes concourant à l'instabilité et l'éphémérité des \glspl{DMI}, (que nous avons considéré comme une spécificité intrinsèque à la fois au  medium numérique naissant, mais aussi au fait musicale lui-même) mais nous l'avons envisagé comme une modalité quasi intrinsèque à l'agencement contextuel invoqué par la perforance musicale.