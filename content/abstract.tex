% !TEX root = ../thesis-example.tex
%
\pdfbookmark[0]{Abstract}{Abstract}
\chapter*{Resumé}
\label{sec:abstract}
\vspace*{-10mm}

Les instruments de musique numériques se présentent comme des objets complexes, qui se situent à la fois dans une continuité historique avec l'histoire de la lutherie tout en étant marqués par une rupture forte provoquée par le numérique et ses conséquences en terme de possibilités sonores, de relations entre le geste et le son, de situations d'écoute, de re-configurabilité des instruments, etc. Ce travail de doctorat tente de décrire les caractéristiques émanant de l'intégration du numérique dans les instruments de musique, en s'appuyant notamment sur une réflexion musicologique, sur des développements logiciels et matériels et sur pratique musicale, ainsi que sur de nombreux échanges avec d'autres musiciens, luthiers, compositeurs et chercheurs.

\vspace*{20mm}

{\usekomafont{chapter}Abstract}\label{sec:abstract-diff} \\

Les instruments de musique numériques se présentent comme des objets complexes, qui se situent à la fois dans une continuité historique avec l'histoire de la lutherie tout en étant marqués par une rupture forte provoquée par le numérique et ses conséquences en terme de possibilités sonores, de relations entre le geste et le son, de situations d'écoute, de re-configurabilité des instruments, etc. Ce travail de doctorat tente de décrire les caractéristiques émanant de l'intégration du numérique dans les instruments de musique, en s'appuyant notamment sur une réflexion musicologique, sur des développements logiciels et matériels et sur pratique musicale, ainsi que sur de nombreux échanges avec d'autres musiciens, luthiers, compositeurs et chercheurs.