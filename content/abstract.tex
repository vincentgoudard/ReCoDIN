% !TEX root = ../thesis-example.tex
%
\pdfbookmark[0]{Resumé}{Abstract}
\chapter*{Resumé}
\label{sec:abstract}
\vspace*{-10mm}

\noindent Les instruments de musique numériques se présentent comme des objets complexes, qui se situent à la fois dans une continuité historique avec l'histoire de la lutherie tout en étant marqués par une rupture forte provoquée par le numérique et ses conséquences en terme de possibilités sonores, de relations entre le geste et le son, de situations d'écoute, de re-configurabilité des instruments, etc. Ce travail de doctorat propose une analyse des caractéristiques émanant de l'intégration du numérique dans les instruments de musique, en s'appuyant notamment sur une réflexion musicologique, sur des développements logiciels et matériels et sur pratique musicale, ainsi que sur des échanges avec d'autres musiciens, luthiers, compositeurs et chercheurs.

\vspace*{20mm}

{\usekomafont{chapter}Abstract}\label{sec:abstract-diff} \\

\noindent Digital musical instruments come as complex objects, which are both in historical continuity with the history of lutherie while being marked by a strong rupture caused by digital technology and its consequences in terms of sound possibilities, relations between gesture and sound, listening situations, re-configurability of instruments, etc. This thesis proposes an analysis of the characteristics resulting from the integration of digital technology into musical instruments, supported by musicological considerations, software and hardware developments and musical practice, as well as exchanges with other musicians, luthiers, composers and researchers.