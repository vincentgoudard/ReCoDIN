% !TEX root = ../thesis-example.tex
%
\pdfbookmark[0]{Abstract}{Abstract}
\chapter*{Resumé}
\label{sec:abstract}
\vspace*{-10mm}

Les instruments de musique numérique se présentent comme des objets complexes, qui se situent à la fois dans une continuité historique avec l'histoire de la lutherie tout en étant marqué par une rupture forte provoquée par le numérique et ses conséquences en terme de possibilités sonores, de relation entre le geste et le son, de situations d'écoute, de re-configurabilité des instruments, etc. Ce travail de doctorat tente de caractériser les caractéristiques émanant de l'intégration du numérique dans les instruments de musique, en s'appuyant notamment sur mon travail de recherche, de développement et de pratique musicale avec de tels instruments.

\vspace*{20mm}

{\usekomafont{chapter}Abstract}\label{sec:abstract-diff} \\

Les instruments de musique numérique se présentent comme des objets complexes, qui se situent à la fois dans une continuité historique avec l'histoire de la lutherie tout en étant marqué par une rupture forte provoquée par le numérique et ses conséquences en terme de possibilités sonores, de relation entre le geste et le son, de situations d'écoute, de re-configurabilité des instruments, etc. Ce travail de doctorat tente de caractériser les caractéristiques émanant de l'intégration du numérique dans les instruments de musique, en s'appuyant notamment sur mon travail de recherche, de développement et de pratique musicale avec de tels instruments.