% !TEX root = ../thesis-example.tex
%
\pdfbookmark[0]{Resumé}{Abstract}
\chapter*{Résumé}
\label{sec:abstract}
\vspace*{-10mm}

\noindent Les instruments de musique numériques se présentent comme des objets complexes, qui se situent à la fois dans une continuité historique avec l'histoire de la lutherie tout en étant marqués par une rupture forte provoquée par le numérique et ses conséquences en terme de possibilités sonores, de relations entre le geste et le son, de situations d'écoute, de re-configurabilité des instruments, etc. Ce travail de doctorat propose une analyse des caractéristiques émanant de l'intégration du numérique dans les instruments de musique, en s'appuyant notamment sur une réflexion musicologique, sur des développements logiciels et matériels et sur pratique musicale, ainsi que sur des échanges avec d'autres musiciens, luthiers, compositeurs et chercheurs.

\vspace*{20mm}

\noindent {\usekomafont{chapter}Abstract}\label{sec:abstract-diff} \\

\noindent Digital musical instruments appear as complex objects, being positioned in a continuum with the history of lutherie as well as marked with a strong disruption provoked by the digital technology and its consequences in terms of sonic possibilities, relations between gesture and sound, listening situations, reconfigurability of instruments and so on. This doctoral work tries to describe the characteristics originating from the integration of digital technology into musical instruments, drawing notably on a musicological reflection, on softwares and hardwares development, on musical practice, as well as a number of interactions with other musicians, instruments makers, composers and researchers.
