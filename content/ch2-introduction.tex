% !TEX root = ../thesis-example.tex
%
\chapter{Instruments décomposés, instruments recomposés}
\label{ch:introduction}

\cleanchapterquote{Komponieren heißt: über die Mittel nachdenken.\\
Komponieren heißt: ein Instrument bauen.\\
Komponieren heißt: nicht sich gehen, sondern sich kommen lassen.\\
.}{Helmut Lachenmann}{1986}


\section{Atomisation}
\label{sec:introduction:atomisation}
L'instrument de musique au XXème siècle 

Le terme instrument de musique électronique sous-entend un préalable consensus sur la définition d’un tel objet, là où il commence, là ou il s’arrête. Or, il a maintes fois été montré combien l’essence d’un instrument de musique, qui pouvait jusqu’alors se définir comme un objet physique (« hardware » ) transformant les gestes en son d’une manière directe, a été remise en question par les découplages successifs (temporel, spatial, energétique, symbolique) qu’ont introduit l’enregistrement, l’électricité, la computation.

Parmi la quantité d’instruments électroniques produits en ce dernier siècle, peu répondent à l’agencement des instruments classiques (« acoustiques »). Ceux qui y répondent sont souvent les premiers réalisés (martenot, theremin) et/ou ceux qui imitent un instrument classique (synthétiseurs à clavier). Une exception notable est la platine vinyle, dont le succès est probablement du à la conjonction de trois facteurs : sa fabrication industrielle en grande quantité d’une part, l’existence d’un « répertoire » (tous les disques vinyles produits dans le but originel d’être écoutés, plutôt que mixés et scratchés) et l’existence de courants musicaux populaires qui se sont spécifiquement développés autour de cet instrument (hip-hop et techno). 

Plus récemment on observe un phénomène similaire autour des interfaces à « pads » (MPD, ableton push, etc) qui a donné lieu, via un phénomène de battle sur les plateformes de video en ligne, à un courant de musique électro-pop où la virtuosité est mise en avant.

Dans un musée des instruments de musique, on trouve des instruments « complets », c’est à dire avec le « corps » entier de l’instrument. On ne présente pas un bec de saxophone tout seul, ou bien une pédale de grosse caisse, ou alors c’est pour montrer l’intérieur de ce qu’on appelle un instrument. Est-ce qu’un musée des instruments de musique qui présente une organologie ne deviendra t il pas un musée des « organes » qui compose le corps d’un instrument ?



\section{Quels axes de recomposition?}
\label{sec:introduction:axes}

Cette atomisation de l'instrument de musique appelle à sa recomposition, son ré-agencement. 