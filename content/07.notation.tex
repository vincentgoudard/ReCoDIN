% !TEX root = ../thesis-example.tex
%
\chapter{Notation}
\label{ch:notation}

\cleanchapterquote{Innovation distinguishes between a leader and a follower.}{Steve Jobs}{(CEO Apple Inc.)}

This article presents “John”, an open-source software designed to help collective free improvisation. It provides generated screen-scores running on distributed, reactive web-browsers. The musicians can then concurrently edit the scores in their own browser. John is used by ONE, a septet playing improvised electro-acoustic music with digital musical instruments (DMI). One of the original features of John is that its design takes care of leaving the
musician's attention as free as possible.
Firstly, a quick review of the context of screen-based
scores will help situate this research in the history of contemporary music notation. Then I will trace back how improvisation sessions led to John's particular “notational perspective”. A brief description of the software will precede a discussion about the various aspects guiding its
design.


\section{Notes sur la partition}
\subsection{Des partitions traditionnelles aux partitions graphiques}
\subsection{Screen scores}
\subsection{Improvisation dans l'ensemble ONE – naissance d'une notation}

\section{A propos de John}
\subsection{Générateur de partitions}
\subsection{Visualisation interactive}
\subsection{Implementation}

\section{John en pratique}
\subsection{Composition générative}
\subsection{Distribution de la participation}
\subsection{Synchronisation}
\subsection{Support visuel pour des repères musicaux}
\subsection{Une écologie de l'attention}
\subsection{Partitions pour humains \emph{et} machines}
\subsection{Montrer la partition ?}

\section{Perspectives}
