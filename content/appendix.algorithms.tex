\chapter{Algorithms}
\label{appendix:algorithms}

\section*{}
\subsection*{}
\subsubsection*{frettage audio-tactile}

Algorithme pour déclencher des impulsions audio pour donner l'impression de frettes virtuelles.
Valeurs normalisé entre 0 et 1.

\vspace{-1em}
\begin{itemize}[noitemsep]
\item  S = série ordonnée des valeur de position de frettes
\item  x[n] = position x du stylet, sous forme de signal synchrone, au temps n;
\item  p[n] = pression du stylet, sous forme de signal synchrone, au temps n.
\end{itemize}

Détecter la frette courante :
 $$zone[n] = \sum_{i=1}^{length(S)} x[n]>S(i) $$ 

Détecter un changement de frette :
 $$absdelta[n] = abs(zone[n] - zone[n-1]) $$ 

Créer une enveloppe à partir de cette impulsion :
 $$pulse[n] = pulse[n-1] + \frac{(absdelta[n] - pulse[n-1])}{slide} $$ 

Modulante d'impulsion de fréquence $$f_mod$$ :
$$cycle130[n] = sin(2\pi\frac{f_m}{SR})$$

Signal audio de frettage, avec P[n] la pression :
$$fretsignal[n] = P[n] * cycle130[n] * pulse[n]$$

\noindent
\begin{minipage}{.5\linewidth}
	\begin{equation}
   		Z[n] = \sum_{i=1}^{length(S)} x[n]>S(i)
	\end{equation}
\end{minipage}%
\begin{minipage}{.5\linewidth}
	\begin{equation}
  		P[n] = \lVert zone[n] - zone[n-1] \rVert > 0.5
	\end{equation}
\end{minipage}
