\chapter{Interview : Adrien Mamou-Mani}
\label{appendix:mamou-mani}

\section*{Biographie}

\noindent Adrien Mamou-Mani (né en 1980) a d'abord fait une thèse sur la vibration des tables d'harmonie, puis travaillé sur l'augmentation des instruments par contrôle actif des modes de résonance, avant de créer une startup développant une ``smart guitar'', guitare acoustique augmentée à l'aide d'un \gls{DSP} contrôlant la résonance de la guitare et permettant autant de diffuser de l'audio directement via le corps de l'instrument, que de transformer en temps réel le timbre de la guitare et d'y ajouter des effets.

\noindent Site web de Hyvibes: \url{https://www.hyvibe.audio}


\section*{Transcript}

\noindent Adrien Mamou-Mani, interview du 20/12/2017 dans les bureaux de HyVibe, Paris.

VG — Qu'est ce qui à l'origine a motivé cette idée qu'on pourrait dire farfelue de vouloir faire de la musique avec des instruments avec des outils numériques et des ordinateurs plutôt que de prendre un instrument existants... qu'est ce qui a motivé sa en premier lieu ?


AM — il y a deux aspects complètement différents qui m'ont motivés. Le premier aspect est quelque chose qui vient de la recherche, c'est peut-être un peu original, cela n'a pas été au départ une problématique musicale, mais née de mon histoire de chercheur sur la physique des instruments de musique. Ma thèse était sur la vibration des tables d'harmonie et j'arrivai bien à analyser comment ça vibre et comment le geste de lutherie (des luthiers) vient influencer certaines particularités vibratoires. J'ai pas mal travaillé avec des luthiers sur la relation entre leurs techniques de fabrication et le résultat vibratoire des instruments.
Et donc le premier aspect qui m'a motivé ensuite à aller commencer à concevoir des choses comme ça c'était un peu dire "est ce que ces choses là qu'eux font par leur savoir faire, est que par de l'électronique on est capable de la même chose. c'est-à-dire que moi, sans être luthier, est ce que je suis capable de faire comme eux. Ça a été une motivation importante dans la conception. Et en fait il y avait tout un champ, il y avait Charles Besnainou qui avait déjà commencer à travailler la dessus, Steven Griffin à CalTech. 
Je me suis rendu compte qu'il y avait des gens qui avaient déjà essayé de faire ça avec, d'avoir une approche disons d'acousticien qui veut essayer de comprendre comment la chose fonctionne mais qui va aborder son sujet d'étude par un autre biais qui était de l'électronique au lieu d'être de la mécanique si tu veux. 
Donc ça, ça a été un truc qui m'a amené à faire ça. Et deuxième aspect... 

VG — Juste pour le premier aspect, pour poursuivre ce que tu disais, quand tu dis "aborder ce point de vue de mode résonance de table d'harmonie avec de l'électronique, c'est (ou ce n'est pas?) une envie de modéliser la table d'harmonie via des méthodes numériques

AM — Non non non, moi ce qui m'intéressait dans les savoir-faire des fabricants, c'est comment  est-ce que eux, quand ils déformaient les choses, ça changeait la qualité de l'instrument. Parce que moi je regardais les résonances et comment ça les modifiait mais... et donc je me suis dit est-ce que c'est possible de changer les qualités, mais non pas mécaniquement comme eux ils font, mais électroniquement.

VG — en les pilotant avec des vibreurs... 

AM — en les pilotant, voilà, c'est ça... c'est euh... un peu arriver au... au départ c'était ça du moins, arriver au même résultat qu'eux, et d'ailleurs moi, pendant mes recherches c'était intéressant de comprendre aussi ce qui fait la qualité, mais par quelque chose de complètement empirique, direct, sur l'instrument, moi je règle comme eux ils font mais par de l' électronique, donc j'ai une maîtrise, je sais ce que je fais par l'électronique.
Ça, ça a été quelque chose d'important, euh... ouais, de quelque chose qui vient disons de problématiques de recherche, disons... on pourrait presque dire c'est presque opportuniste, c'est-à-dire comment le... une thématique de recherche de l'époque m'a fait sentir qu'il y avait des choses qui par mes connaissances à moi, par mes compétences dans un domaine spécifique pouvait m'amener à faire comme les luthiers, enfin je sais pas comment te dire tu vois... 

VG  : oui, oui... 

AM — c'était vraiment un peu en temps que chercheur qui d'un coup bascule en recherche appliquée ou disons développement par rapport à ma curiosité à la base de chercheur. Donc ça ça a été un des truc mais ça n'a pas été ça qui a été le déclencheur vraiment.
Ce qui a été le déclencheur, après je suis pas mon propre psy mais... (rire) ya.. euh... enfin c'est très précisément en 2010, euh, ma femme était enceinte de ma fille, ça a été je pense  quelque chose de très important pour moi, c'est-à-dire elle elle était, enfin on était en train de construire quelque chose quoi, son ventre qui grossissait et je sentais que je servais à rien, donc j'ai eu envie de faire quelque chose avec mes doigts, bon... 
et ça a été au moment où, donc là je travaillais à Londres à ce moment là, je travaillais sur les hautbois, donc des choses qui n'avaient rien à voir et à ce moment là il y avait de plus en plus d'applis qui sortaient sur iPhone, il y avait, tu sais l'ocarina là des gens de Stanford, comment ça s'appelle ? Smule... il y avait, il comment à y avoir plein d'applis qui sortaient et je me souviens très bien car il y a mon frère qui m'a appelé et qui m'a dit "bon Adrien, il y a plein de trucs qui sortent sur les iPhones, toi, avec tout ce que tu connais des instruments de musique t'es pas capable de nous faire une petite appli là, un truc sympa ?"... et ça ça a été un déclencheur en fait.
Donc il y avait ce truc là, j'avais mon projet qui m'intéressait à Londres mais ça arrivait à, enfin j'avais envie d'avancer... ma femme (fait un geste montrant son ventre enceinte)... mon frère qui m'dit ça... . et... et dans le labo il y avait quelqu'un qui utilisait... tiens c'est la première fois que je le dis ça... 

VG — faut faire un bébé en fait... 

AM — ouais c'est ça... d'ailleurs la 1ère guitare je l'ai appelé Annabelle comme ma fille... enfin bon tout ça était un peu mélangé... 
et dans mon labo il y avait... et en fait moi je me suis dit bon mon frère c'est marrant ce qu'il dit mais moi je veux pas faire un truc que je considère comme un gadget, parce que justement pour moi tout ce qui était numérique, toi c'est ça qui t'intéresse, mais moi à cette époque là j'ai complètement ça, la musique numérique c'était pas du  tout mon truc à cette époque là, j'ai eu des phases, quoi... Et quand il m'a dit ça je me suis dit non je vais pas encore faire un gadget,  mais je me suis dit par contre est ce qu'il n'y a pas moyen de connecter les deux mondes, moi ce que je fais sur l'acoustique et ce qu'on est capable de faire avec du numérique et... et donc c'est là que mes recherches aussi sont intervenues et j'ai repensé à ce qui m'a intéressé dans mes recherches, etc.

et dans mon labo il y avait quelqu'un d'autre, ça c'est la première fois que je le dis, mais il y avait Roland qui était un post doc dans un labo en Angleterre, qui avait bossé avec des boîtes en aéronautique, je ne sais plus avec qui, un équipementier d'Airbus ou de Boeing qui utilisait des petits actionnaires électrodynamique NXT, des choses qui maintenant sont vachement répandues, mais moi en 2010 je ne connaissais pas trop, les premiers brevets là-dessus c'est en 2000 à peu près. Et donc moi je connaissais pas, je lui disais “j'aimerais bien trouver un truc qui vibre“, j'avais trouvé des choses dans les boutiques à Londres mais qui étaient des gros trucs qui n'allaient pas et il m'a dit “ ah ouais, mais moi j'avais utilisé pour les avions, on avait utilisé des petits trucs, regarde la marque NXT qui est une marque anglaise justement, et donc j'ai pris un iPhone, j'ai pris un actionneur NXT, je l'ai collé sur une guitare à deux balles que j'avais acheté et j'ai commencé à balancer des trucs dans la guitare. 

Ça c'était mi-2010 ou 3ème trimestre 2010 je pense. J'ai fait ce truc là dans la guitare à la maison. Et là, dans ma tête ça a explosé et ça a tout fait exploser en fait. Parce qu'elle quand j'ai fait cette expérience là, je me suis rendu compte qu'avec quelque chose d'assez peu invasif où je pouvais continuer à jouer de la guitare acoustique, j'avais en plus du numérique qu'était dedans. Je me suis rendu compte que tout ce que j'avais fait avant sur comment ça résonne, le couplage avec les cordes, les propriétés vibratoires de la caisse, tout ça avait un impact sur... enfin tout ça me servait à comprendre ce qui se passait là... donc voilà j'ai repris, j'ai rappelé Charles Besnainou, je lui ai dit “ toi tu avais fait des boucles de feedback dans les guitares, racontes moi ce que tu avais fait ” et là je monte un projet de recherche avec l'ANR, aussi grâce à une super rencontre avec Baptiste Chomette à l'université Pierre et Marie Curie, qui était lui spécialiste du contrôle vibratoire pour l'aéronautique et pour autre chose et pouf, cette série de choses entre 2010 et début 2011, j'ai monté le projet ANR, ça a marché et après donc j'ai créé ce projet “ smart instruments ” à l'IRCAM. 

Donc voilà, ce qui m'a motivé allait commencé à jouer avec ces technologies là. Voilà c'était tout bête c'était juste ça je pense. 

VG — Quand tu commençais à jouer avec ces technologies, tu l'as dit au fil de la discussions, tu jouais, tu joues de la guitare toi ? 

AM — moi, je joue de la... j'ai joué de la guitare, euh... mais vraiment amateur dans un groupe de rock ; je jouais de la basse et je chantais dans un groupe de rock. Des choses un peu guitare acoustique. J'ai commencé à gratouiller à 12 ans, à faire des chansons, ça j'ai toujours bien aimé, sinon moi je joue plutôt du violoncelle ; disons que je fais du violoncelle classique depuis petit, et ensuite j'ai fait une année de Méta-Instrument avec Serge de Laubier, toi tu n'étais pas encore là je croise (à Puce Muse, où j'ai travaillé entre 2005 et 2008, NdE), enfin on s'était croisés après ou à la fin.

VG — il y a eu plusieurs rencontres croisées... 

AM — voilà, il y a eu ça, avec Pierre Leveau on avait commencé à faire aussi un peu de musique électronique à la maison... Voilà... il ya plusieurs choses comme ça entre les instruments classiques et électroniques et moi sinon surtout ce qu'il ya eu, c'est que moi tout ce qui m'a motivé depuis j'ai 19 ou 20 ans c'était la musique concrète, Pierre Schaeffer, j'ai tout lu, j'étais un grand fan de de Pierre Schaeffer, donc en fait c'est ça surtout, qui... au bout du compte ce que je voulais, c'était réussir à trouver un équivalent de la musique concrète aujourd'hui. C'est un peu naïf mais en tout cas c'était ça qui m'a motivé à la base dans mon histoire musicale, disons c'est c'est ça qui a joué. Instruments classiques, musique concrète et lectures de Pierre Schaeffer... 

VG — un désir sonore ? enfin quelque chose de l'ordre du désir sonore qui peut être stimulé par la musique concrète qui n'a pas vraiment ses instruments à proprement parler ?

AM — qu'est ce que tu veux dire par désir sonore ?

VG — dans les motivations qui t'ont poussé à créer des objets comme ça 

AM — euh, je sais pas... non c'était pas désir, c'était plutôt essayer de trouver... qu'est ce que c'est... qu'est ce que c'est la musique aujourd'hui... (rires) excuses moi je dis des banalités... (rires)

VG — non non, Pierre Boulez a écrit un bouquin donc c'est une question valable (rires)... 

AM — exactement (rires) donc voilà je me suis juste dit si Pierre Schaeffer était là aujourd'hui, avec ce que lui avait fait à l'époque, cette rencontre entre technologie, perception et sciences, disons, qu'est ce qu'il aurait fait aujourd'hui, qu'est ce qui serait un peu le... la petite goutte qui noise de notre de notre temps... si je devais faire quelque chose qui représenterait mon époque, musicalement, un objet musical disons, je sais pas comment dire, au début je pensais plutôt d'un point de vue composition quand j'avais 18 ans, après j'ai pensé à la musique électronique après l'acoustique, enfin bon ça pouvait être composition, ça pouvait être un instrument mais je voulais essayer de trouver quelque chose qui représentait pour moi notre époque. Et avec ça je me suis embarqué dans un truc où j'avais l'impression que j'avais débloqué quelque chose, où ça y est, ça a donné un sens à ces trucs là, tout ce que je cogitais sur Pierre Schaeffer, je me suis bon finalement peut-être que ce truc là qui est la rencontre entre le monde physique et non numérique c'était ça notre époque, pour te résumer... voilà

VG — alors par rapport à ça, j'ai plutôt interviewé récemment des gens qui pratiquaient, donc mes questions étaient plutôt orientées sur comment les musiciens qui utilisent ce genre d'instrument gèrent un certain nombre de propriétés parfois contraignante du numérique, le fait que ça marche ou ça marche pas —que c'est soit ON ou OFF, le fait que qu'on peut passer de manière brutale d'un contexte à un autre qu'il y a une continuité... et par rapport peut-être à que tu fabriques, toi, cette même question ressurgit je pense et tu parlais de connecter les deux mondes entre l'acoustique et le numérique, peut-être une manière de formuler de manière provocante, ça serait de te demander si c'est pas introduire un peu le loup dans la bergerie que de mettre du numérique dans l'acoustique 

AM — dans quel sens ? Le numérique vers l'acoustique, c'est le loup dans la bergerie le numérique dans l'acoustique ? La bergerie c'est l'acoustique ?

VG — je sais pas, je pense que ça peut jouer dans les deux sens... 

AM — oui... 

VG — notamment, le numérique a l'avantage d'être répétable, donc on a quelque chose de... si ça marche une fois, ça marche de la même manière à chaque fois... 

AM — (sourire)

VG —... ce qui fait peut-être aussi partie de ses limites et à l'inverse, une table d'harmonie c'est du bois, donc on n'en a pas deux pareilles. Je repensais à une conférence de Nic Collins, qui est un chercheur et musicien qui expérimente beaucoup de manière très empirique avec l'électronique et qui a fait une présentation où il est détaille un peu les différences entre hardware et software... 

AM — oui 

VG —... notamment le fait que software, c'est constamment dans le présent, c'est constamment mis à jour, que ça marche ou ça ne marche pas, alors que le hardware ça peut marcher même en étant un peu cassé

AM — oui

VG — donc comment tu vois un peu cette jonction en terme de... 

AM — oui, ok... alors moi ce que je vois c'est... euh... c'est une bonne question... comme ça le premier truc qui me viendrait, ça serait te dire que... mon approche, je veux dire en une phrase, mon approche c'est d'avoir une acoustique programmable. C'est tout simple, c'est juste ça. c'est-à-dire le numérique est au service de l'acoustique. Ce qui m'épate le plus c'est quand tu ne sens même pas qu'il ya du numérique, c'est ça qui m'intéresse le plus. c'est-à-dire parfois on s'amuse des musiciens, on leur fait des traitements, et eux ont juste l'impression que c'est une autre qualité acoustique mais ils ont pas l'impression qu'il y a du numérique qui vient faire ça. C'est ça qui m'intéresse le plus, c'est numérique le service de l'acoustique, pour avoir... pour donner au monde physique les capacités de programmation du monde numérique. c'est-à-dire, une versatilité, les anglais disent versatility, je n'ai toujours pas trouvé en français exactement comment traduire ça, voilà, le côté polyvalent, modifiable, qui va avoir une vie, voilà te dire simplement il aurait sûrement plein d'autres trucs mais en tout cas ce qui est vrai pour moi, la vérité c'est que je vois les choses comme ça : utiliser la technologie au service de l'acoustique pour pouvoir la programmer.

VG — par rapport à cette métaphore peut-être maladroite du loup dans la bergerie, tu verrais plutôt le loup d'un côté ou de l'autre du coup?

AM — j'ai eu la bonne surprise de voir que pour les luthiers, quand donc les vrais luthiers purs et durs de l'association de luthiers-violon, quand je leur faisais des démonstrations, eux ils me disaient ben t'es comme nous, tu es juste un luthier ; nous on fait de la lutherie pour du Mozart et toi pour que ta lutherie soit intéressante il faut qu'elles servent la musique de maintenant, ou  de demain, mais de me voir faire des traitements sur des violons et changer leurs timbre en temps-réel pendant qu'eux jouaient dessus, ça ne les a pas choqué. Donc il n'y a pas vraiment eu ce truc de “ ha... qu'est ce qu'il nous amène, pourquoi il nous met un ordinateur dans notre violons ”. Il n'y a pas eu ça. J'ai pas rencontré de ça. 

Les seules choses... alors peut-être que c'est plus côté musicien, alors ça rejoint ce que tu disais toi, parce que quand tu vois des gens qui pratiquent plutôt, les seules choses qu'on a eu, pas sur les guitares qu'on fabrique maintenant mais sur les prototypes d'avant, en particulier sur les clarinettes par exemple, et sur les violons aussi, c'est qu'en fait, il ya eu certains musiciens qui ont senti que on amenait une énergie concurrente... à ce que... à eux en fait... et ça, ça les a...  il y en a certains qui ont été gênés. Dire que normalement un instrument, il est figé. Un  instruments acoustiques, un instrument classique, il est figé, et c'est eux ensuite qui vont le transcender, faire émerger des chose, le maîtriser en connaissant bien, en apprenant bien tout son timbres, ses différentes notes et à l'exploiter au maximum. Et nous quand on arrivait avec nos traitement d'un coup l'instrument c'était plus le même donc eux ils disent “quoi ? moi je peux pas faire ça, j'ai travaillé le truc et maintenant en fait ça sonne plus du tout pareil ” et bon pour eux ça devient gênant d'avoir un instrument qui bouge  puisque d'habitude c'est eux qui donnent ce côté souvent dynamique à l'instrument par leur façon de jouer. 

Ça, ça a été un problème et un deuxième aspect, ça a été des gens qui... voilà, eux mettent de l'énergie dans l'instrument et d'avoir une énergie concurrente qui vient corriger, ils ont pas l'habitude, c'est comme s'il y avait quelqu'un d'autre qui jouait avec eux quoi, et donc ça ça peut être gênant aussi. 

Donc par rapport au loup dans la bergerie, c'est plutôt côté interprète, côté musiciens que ça ça a pu, dans certains prototypes pour certaines pièces qui avaient été écrites que ça a peu gêner. 
Aujourd'hui on n'a plus tout ce problème là avec nos produits, vraiment guitare, parce qu'en termes fonctionnels, c'est des choses qui se faisait déjà. Ce qu'on propose là, c'est des choses qui se faisaient déjà avec du traitement électronique mais que nous juste on a intégrer dans la guitare, donc pour eux il y a un côté transparent de ça, déjà connu, c'est pas un nouvel instrument pour eux. Là on est dans un cadre où... un peu comme ce que je te disais avant, c'est le numérique au service de la guitare, de ce que eux imaginait déjà, que ça soit des pédales analogiques, que ça soit des timbres acoustiques qu'on change, nous on leur fait revivre ça, mais avec une expérience plus simple et plus directe quoi. 
Donc là il n'y a pas ce côté loup dans la bergerie parce que... parce qu'en termes fonctionnels à mon avis, c'est juste quelque chose qu'ils se représentent déjà. 

Mais il y a un côté high-tech qui est un peu problématique, là plus coté lutherie quand on parle avec des grandes marques et de leur dire qu'on va intégrer ça dans leurs instruments, voilà ils n'ont jamais encore vu ça.. 
T'as pas dans les instruments acoustiques ce genre d'ordinateur, avec une technologie très particulière que nous on met en place, justement parce qu'on fait du numérique mais avec du hardware pour du numérique très spécifique, pour avoir des latences très faibles. 
Donc quand tu parlais de cette distinction hardware-software, pour moi dans le numérique je mets aussi la partie hardware, disons qu'on a besoin de faire un hardware pour pouvoir faire nos traitements numériques et pour pouvoir coupler ça et ça avec notre monde physique, et donc là par rapport à eux, les luthiers aujourd'hui installer ces trucs là dans des guitares, c'est quand même, euh... c'est pas loup dans la bergerie, c'est plutôt “ whoo..comment on va faire ? ”... 

VG — par rapport à ça, à cette réaction que pouvait avoir certains musiciens, j'ai l'impression que la smart guitar, elle rassure en grande partie à cause du fait qu'elle est liée à un objet qui est connu, qui est identifié et qui a fait l'histoire, et qu'il y a un minimalisme de l'interface numérique qui vient dessus... 

AM — exactement... tout est pensé comme ça oui

VG — qui donne une place assez discrète à ça. Et des vidéos de démonstration que j'ai vu où il y avait les vibratos, chorus, reverb,etc. on reste sur des effets qui reviennent un peu à incorporer ce qu'on pourra voir avec une pédale d'effet dans le corps de l'instrument... 

AM — c'est ça, c'est ce que je te disais oui... 

VG —  mais j'imagine qu'il est possible de faire des choses que, pour présenter au début, et pour rassurer tout le monde, c'est probablement bien de commencer par ce genre de choses... 

AM — ahah... t'as capté ma stratégie... 

VG — mais j'imagine plein d'autres choses possibles et par rapport à ça et il y a aussi une question qui se pose par rapport à ces instruments... 

AM — je peux t'en citer un si tu veux... 

VG —... qui sont toujours en mouvement, c'est l'apprentissage. Ce que tu disais par rapport avec violonistes qui disent 'mais moi je ne peux pas faire ça'... Et certains auraient tendance à prétendre qu'on ne peut pas les apprendre parce que l'objet change sans arrêt... comment vois tu ça toi ?

AM — oui... ok... alors par rapport à ce que tu disais au départ, donc en effet ça c'était le plus magique dans les tests qu'on a fait parce que ça me rappelait les trucs au LAM justement, la façon de faire des tests et tous les protocoles parce qu'il faut toujours rester très ouvert parce que tu as des surprises. Et nous, les premières surprises qu'on a eu, ça a été donc au début on s'était très fixées sur des effets connus et on s'est rendu compte que dès que tu arrives à un certain niveau de musiciens, les effets connus ils disaient “ ok mais bon je les ai déjà et moi j'ai mon matos qu'il faut, j'ai mes bonnes pédales et tout, donc j'en ai pas besoin ” donc ce qui est ressorti, c'est que des trucs que nous, on pouvait considérer comme des défauts, c'est ça qui les a intéressés. Ils ont essayé de tirer au maximum sur tous les trucs qu'ils n'avaient jamais entendus, en particulier le truc qui marche le mieux c'est tout ce qui est sustain ; donc tu sais un peu les \gls{e-bow} là, tu vois ce genre de système. Donc nous on a des sortes de e-bow dynamiques et qui pètent bien sur toutes les cordes et que tu peux mettre différemment sur chaque note, et tout ça et eux ils ont vachement exploité ça et c'était le truc qui justement était à la marge et qui ne ressemblait à rien et finalement dès qu'on est arrivé à un certain niveau de musiciens, c'est ça qui les intéressent. Donc, tu disais “ pour rassurer les gens ”, oui mais c'est pas ça qui va faire que les pro vont l'acheter parce que eux ils veulent justement le truc unique qui d'habitude est impossible à faire. Donc il ya eu ce côté sustain et aussi des côtés très originaux c'est la transformation de l'acoustique de l'instrument. Donc ça c'est particulier aussi. Donc on change le timbre. J'ai pas voulu mettre l'accent là dessus au départ parce que c'est une niche et c'est très subtil de booster un peu ton son acoustique et de le transformer... et donc voilà j'ai pas mis en avant mais donc en effet, il ya ces aspects là qui sont d'ailleurs beaucoup plus révélateurs de nos technologies que les choses que l'on a mis en avant aujourd'hui, où la technologie est importante mais plus implicite là dedans, dans le contrôle du feedback. Mais donc en effet, il ya tous ces aspects là et on s'est dit que pour... c'est exactement ce que tu as dit, juste pour vous rassurer pour commencer à attirer les gens il fallait déjà... il fallait pas les envoyer sur des choses qui étaient complètement étrangères, pour qu'ils puissent commencer à s'approprier leur interfaces pour qu'ils puissent garder leurs références à la guitare, tout en... enfin c'est vraiment de la guitare augmentés, quoi, tout en ayant des capacités en plus. Donc ça c'est la première partie de ce que tu disais. Donc en effet il ya ça. 

Ensuite sur le côté, si j'ai bien compris, quand tu parlais de comment essayer d'avoir quelque chose qui est stable et qui va pouvoir s'ancrer au fur et à mesure parce qu'avec ces technologies là ça bouge en permanence et donc il y a toujours le risque, enfin, ou en tout cas est-ce-que dans sa nature même, est ce que ces trucs-là restent sur des choses éphémères et qu'il faut toujours renouveler, renouveler... c'est bien ça tu voulais dire dans ta question ?

VG —  un peu... ce que je voulais dire c'était que si je dois ramener ça à une chose concrète, je parlais de la pédagogie, dans le sens où si tu vas dans un conservatoire, tu peux apprendre la guitare et on t'explique un certain nombre de techniques de jeu, tout ça... avec ces nouveaux instruments, enfin les instruments électroniques de manière générale, la question qui se pose vraiment franchement, c'est comment tu enseignes l'électroacoustique dans le cas d'un instrument purement numérique, vraiment purement la-dedans, donc il ya un enseignement électro-acoustique qui existe mais qui est une école assez particulière dans le vaste monde de la musique électronique ; avec ce genre d'instrument est ce qu'il ya une pédagogie ?

AM — OK, j'ai compris... alors je crois que je vais te décevoir, parce que... en tout cas aujourd'hui, avec ce premier produit là, tout ce qu'on fait c'est basique, c'est juste des choses qu'ils ont dans leur pédales ou avec leurs enceintes bluetooth, ils l'ont dans leurs guitares. Donc il n'y a pas... on cherche pas, dans cette smart guitar, on cherche pas à exploiter les spécificités du monde numériques pour essayer ensuite de les apprendre, les transmettent et tout. Non, nous on veut... pour eux, au lieu d'avoir leur pédalier, ils ont leur pédalier qui est dans leur écran et c'est tout. C'est ça. Le produit c'est ça. 

VG — on se concentre sur le timbre, quoi... 

AM —  Voilà. C'est un travail du son qu'ils font déjà en plus et que les guitaristes ont déjà l'habitude de faire par deux dispositifs que nous on intègre dans des guitares acoustiques, pour profiter des questions de... profiter des propriétés des caisses, des questions de qualité sonore, des questions pratiques qui va sortir de la guitare, et une question de connectivité pour qu'eux, ils puissent échanger entre eux et que quand ils écoutent un cours en ligne au lieu d'avoir un sale son qui sort d'un ordinateur, il vont avoir le son d'une guitare avec le maximum de qualité possible. Donc il y a la question de qualité sonore, de connectivité et de simplicité du dispositif. Donc pour ce produit là, on n'est pas dans, disons un produit pour aider un processus créatif que nous on viendrait proposer pour faire des choses vraiment nouvelles et qu'on va essayer ensuite de faire apprendre et voir comment partager ça. On n'est pas là dedans, on est plutôt sur quelque chose pratique, appuyer sur un bouton et voilà. Alors après on pourrait discuter sur ce que je pense que j'ai fait avant, ou bien sur des aspects plus créatifs. Là, il y a d'autres sujets qui arrivent, je sais pas si tu veux qu'on parle de ça ou qu'on se concentre sur ce produit là. 

VG — pas forcément, mais peut-être la question que j'ai à te poser a trait à ça... C'est dans le design de la smart guitare... enfin... ces nouveaux apprentissages sont aussi des fois liés à des nouveaux gestes... 

AM — oui, exact. La on a un gros sujet déjà avec ça... 

VG — et la smart guitar que vous avez designé... 

AM —... l'interface est problématique pour ça. 

VG — j'imagine que c'est un choix qui a des raisons, bonnes ou mauvaises, ou sans parler du fait qu'elles sont bonnes ou mauvaises, mais vous auriez pu, j'imagine, facilement insérer des capteurs gyroscopiques, des capteurs de pression, des glissières et toutes sortes de choses. Je veux dire, la table complète pourrait être farcie de boutons et ce n'est pas le cas. 

AM — exact

VG — donc j'imagine qu'il y a quand même un choix assez fort de ne pas le faire, de ce minimalisme ...

AM — exact 

VG — et qu'est ce qui... 

AM — c'est important ouais... j'ai pas encore... euh... explicité ça... [silence réflexif] tu es le premier qui me pose cette question là... parce que dans le monde des guitaristes, eux ils ont une vision très normale de ça... mais tu as raison, il y a rien de normal quand on vient plutôt du monde des technologies, où il y avait... au départ on avait plein plein plein de possibilités justement... 

VG — et où les interfaces ont plutôt tendance à être couvertes de boutons... 

AM — c'est super intéressant et d'ailleurs il y en a certains qui nous poussent qui disent “ ah j'aimerais bien plus si ça et ça ” et là moi je suis “ niet ”... pour l'instant en tout cas pas dans ce produit là. Alors, ben ça rejoint ce que je te disais avant. En fait, c'est disons une philosophie de réalité augmentée disons de guitare augmentée... moi mon idéal, c'est le numérique au service de l'acoustique et personne se rend compte de rien en fait. C'est ça l'idéal que j'ai en tête, c'est un numérique intégré en fait dans notre monde physique, parce que... moi j'ai toujours été très frustré ou critique sur... face à toute la complexité du monde physique qu'on a et qu'on a encore du mal à analyser, comme l'acoustique des instruments, quand après finalement on appuie sur un clavier, sur une touche, qu'on a du MIDI... j'ai jamais cru à... sur du long terme qu'on tendrait vers quelque chose avec ça... qu'on arriverait à la complexité du monde physique. Moi j'ai toujours vu les choses dans l'autre sens. Partir déjà, de notre monde à nous et je trouve que ça rentre dans une philosophie plus actuelles de développement durable et de choses... voilà les gens ils s'embêtent à apprendre la guitare pendant dix ans, qu'on a on a tout un répertoire, on a toute une richesse... je veux que quand ils prennent ça ils se disent qu'ils ont le potentiel d'utiliser toute cette richesse qu'ils ont déjà. Donc c'est dans cette logique là de s'appuyer vraiment sur l'existant dans une logique de réalité augmentée, où paf, on va juste rajouter autre chose derrière qui va venir compléter cette chose là et qui doit être le moins invasif possible, le moins intrusif possible et qui rejoint ce qu'on appelle aujourd'hui les objets connectés, l'Internet des objets... Par rapport à ce qu'on disait précédemment sur qu'est-ce que c'est notre époque, moi j'essaie de le voir comme ça... Quelque chose comme ça, où finalement les ordinateurs, il va en avoir de toute façon partout, ça va être intégré partout et qu'il n'y a plus de distinction de ces deux mondes-là, du monde physique et du monde de l'ordinateur. Et ça (la smart guitar, NDR) c'est un démonstrateur de ça, je le vis comme ça. J'aimerais que ça soit un démonstrateur de ça, où après avoir fait des objets dédiés dont les ordinateurs après avoir commencé à porter ça dans des smartphones et quelques objets, ben maintenant on est rentré dans le truc où ça commence à être de plus en plus partout. Ça commence à être de moins en moins intrusif, donc on peut avoir un idéal comme ça, où on peut s'appuyer sur, non pas le savoir faire qui vient du monde numérique, mais le savoir-faire qui vient du monde physique, et en l'occurrence des instruments acoustiques, et... ou industriel. C'est tout des trucs, l'industrie “ trois point zéro ”  “quatre point zéro ” je sais pas quoi, c'est un peu le même idéal un peu comme ça, où l'industrie existante va juste intégrer ça, et ça va être complètement intégré. Tu vois c'est un idéal complètement de fusion des choses et tout est parfait... haha.. de symbiose... un idéal symbiotique

VG — mmm... “ l'homme symbiotique ”... pour reprendre le titre du livre de Joël de Rosnay... 

AM — ouais c'est ça je crois aussi... que j'ai pas lu... mais voilà c'est ça.

VG — d'accord... je vois qu'on a déjà passé presque une heure... j'aurais deux autres questions... la première question est liée au contexte actuel de production musicale, dans lequel un très grand pourcentage de la musique, je ne saurais pas dire précisément, mais qui a tendance à grossir encore, de la musique qui est produite, enregistrés sur CD, est enregistrée en studio à l'aide de logiciels de montage qui permettent de tout recomposer et bien souvent d'ailleurs elle est programmée à la main, c'est-à-dire que plutôt que de faire jouer un batteur qu'on va enregistrer avec la difficulté qu'on va voir après )à remettre son jeu en place, à ré-ajuster pour des besoins de production, à les programmer directement avec des algorithmes qui permettent d'ajouter du rubato ou que sais-je... donc il y a tout un pan de la musique produite sur les enregistrements qui est faite par des méthodes “ offline ” qui en fait ne sont plus du “ jeu ”.  Et par rapport à ça, dans la fabrication d'un instrument, il y a une sorte de pari quelque part, en fait, de miser sur l'instrument en disant “ l'instrument n'est pas mort ” 

AM — oui, il y a un engagement, un pari et un engagement... 

VG — comment tu vois la smart guitar ou la position de manière plus générale des instruments dans cette production musicale où le numérique a tendance à assister de plus en plus la production musicale, que ce soit par des méthodes hors-ligne comme celles d'édition

AM — alors c'est un nouveau pavé mais je n'y connais pas grand chose donc je vais dire juste très simplement comment ça se passe un peu aujourd'hui, quand on parle du son avec des gens qui sont justement côté studio et quand ils voient ce qu'on fait. Eux ce qu'ils nous disent globalement, c'est... moi le maximum que je peut choper à la source je le chope... sur l'instrument... parce que tout ce qui se fait off-line derrière sur les modifications du timbre et tout, pour eux ils nous disent “ si on peut éviter on préfèrerait ”. Donc eux ils sont contents quand on leur montre ça, pour l'instant on n'a jamais travaillé — on n'a jamais été au bout d'un projet comme ça, mais en tout cas au premier abord quand on parle avec des gens plutôt côté studio, eux ce qu'ils nous disent c'est “ ah oui ça m'intéresse parce que moi le son piézo c'est toujours un problème, donc si vous pouvez me faire quelque chose de programmable qui fait que moi à la source j'ai pas ce côté piézo ça m'arrange ”... “ moi j'aime pas trop rajouter les traitements sur une guitare acoustique a posteriori, si vous vous les avez en amont et que moi j'ai juste à mettre un micro devant avec déjà le truc qui est bien mixé comme il faut à l'entrée, ben ça m'arrange aussi ”. Donc eux... là pour l'instant... moi je suis suis pas du tout spécialiste mais tout ce que j'entends c'est plutôt que par rapport à un instrument, la prise prise de son, disons d'un instrument, donc ça ne recouvre pas tout ce que tu as dit avec tous les aspects aussi de jeu instrumental, mais disons, prise de son, si on parle de timbre, de son de l'instrument, voilà, plus on part d'une bonne source mieux c'est, donc si ils ont ça en plus, ils sont plus contents. Voila... mais on est en amont quand même... 

VG —  side note, je faisais volontairement l'avocat du diable là dessus et je conçois très bien que si on avait rajouté l'effet wah-wah sur les guitares de Jimi Hendrix en studio, ça n'aurait pas du tout marché... 

AM — exactement ouais... oui oui mais comme tu dis il ya tout un pan de la musique qui se fait quand même comme ça maintenant et... après tu peux dire qu'il y a aussi tout un retour à la musique live et aux concerts et tout, maintenant que la monétisation par le studio est de plus en plus complexe aussi, donc bon... je sais pas. 

En tout cas, côté guitare acoustique, ça oui, ce que je peux te dire quand même c'est que j'ai découvert un monde de gens qui se retrouvent dans dans des salons à faire de la guitare acoustique, il y a des associations de gens qui se retrouvent comme ça en semaine, en soirée pour aller présenter leurs nouvelles compos en petit comité comme ça, dans des appartements... Il n'ya plein plein plein plein de choses qui se montent dans le monde comme ça, donc par rapport à la guitare, je pense que pour ce produit leur particulier quoi, ça colle quand même avec — alors c'est peut-être pas la grosse industrie je ne sais pas aujourd'hui— mais en tout cas ça colle avec une pratique musicale qui est très très très très très courante encore voilà juste prendre une guitare et faire de la musiue, ça reste un truc énorme.

VG — c'est vrai que la production automatisée de musique se voit surtout, enfin là où elle est le plus flagrante, c'est dans la production typiquement de musiques de films ou économiquement de faire venir un orchestre par rapport à acheter une banque de son de Vienna Library, il y a un rapport économique là dedans, et la guitare de ce point de vue là est un instrument assez populaire que tu peux prendre et jouer de manière très simple, donc elle est moins sujette à ça peut-être.

AM — peut-être, après tout ce qui est sample de guitares, ça marche bien, il y a pas mal de choses mais peut-être on est dans un contexte un peu différent... 

VG —... et ma dernière question à dix mille dollars, comme on a commencé par ton passé, ton parcours, on va terminer par l'avenir... la question est très ouverte, c'est qu'est ce qui selon toi, enfin lié quand même à ce sujet des instruments de musique numériques, qu'est-ce qui selon toi est la voie vers laquelle on va où tu voudrais aller, la manière dont ça va transformer notre monde et notre réalités perceptive... 

AM — alors on déjà si on arrive à faire un truc où on est capable d'avoir une acoustique programmable, j'aurai l'impression d'avoir fait quelque chose... et pas simplement sur les instruments, on discute avec d'autres secteurs, à savoir le côté design sonore par programmation numérique, le design sonore des objets par programmation numérique. On discute avec l'automobile avec d'autres secteurs qui veulent transformer les timbres des objets mais non pas mécaniquement comme ils le font d'habitude en changeant des choses mais par des traitements numériques. Donc ça pour moi, voila si on est capable d'avoir des choses en particulier, si on arrive à avoir des choses comme je disais où tu n'as pas... un “ son numérique ”, je sais pas comment dire, peut-être que toi tu connais ça mieux que moi... une identité sonore qu'on associe aujourd'hui à une technologie numérique. Si on arrive à faire ça, à avoir plutôt vraiment quelque chose avec les objets qui résonnent plus, qui résonnent moins,où on a plus d'aiguë, plus de grave... et que tout ça soit programmables ça déjà c'est pas mal... 

VG — appliquer le design sonore, ce qu'on appelle aujourd'hui le design sonore, vraiment à la matérialité des objets qui nous entourent?

AM — c'est ça, exactement. Donc ça, ça ça m'excite bien déjà et et je pense qu'il y a un enjeu dans nos sociétés où les objets ils sont... ils peuvent avoir des fonctions différentes, ils bougent et donc tout ça, c'est bien d'avoir quelque chose qui peut s'adapter, pour qu'ils s'adaptent et dont la sonorité s'adapte. Ça peut servir. Donc ça c'est un des aspects, c'est pas exactement ta question sur les instruments numériques. Alors moi sur les autres instruments ...

VG — sur les instruments au sens large... 

AM — ouais ok donc si aussi une voiture ça peut être un instrument pour toi... voilà donc tu as là dessus donner cette flexibilité, cette versatilité aux objets, à les rendre des instruments d'une certaine manière, ça ça m'intéresse bien. Sinon après pour la la suite de ça, ça serait de ne pas se limiter aux sons mais à intégrer les choses plus... . aller plus vers de l'interprétation ou aller vers... par exemple on parlait de Philadephie (pour la conférence ImproTech Paris – Philly (ikPP) rassemble des universitaires, des technologues, des musiciens, des créateurs, autour de l’idée de l’improvisation musicale en interaction avec intelligence numérique, NDR) quand j'y étais la semaine dernière, là c'était parce que j'aimerais bien, dans la boucle, rajouter peut-être un accompagnement qui va s'adapter automatiquement à ce que tu fais, ou qui va aller te chercher dans des bases de données telle musique que tu as envie de jouer, donc on va aller te mettre directement des accompagnements. Donc coupler  ça avec des technologies qui sont pas sonore mais plus de la musique au sens large, du symbolique musical, si tu peux avoir en même temps ta partition pendant que toi tu es en train de jouer et qu'il reconnait ce que tu fais, qu'il va te faire les effets correspondant... voilà bon, lier tout ça ensemble... 

VG — avec les collègues d'Antescofo à côté (startup hébergée dans le même bâtiment, proposant un logiciel de suivi de partition et d'accompagnement automatique, NDR)... 

AM — voilà, donc eux c'est le genre de choses qu'ils font... on peut imaginer, ils sont au dessus de nous Antescofo, on pourrait imaginer une collaboration... mais il n'y a pas que Antescofo, hier il y a des gens de Make-music, qui étaient là, qui était incubés ici avant aussi Weezic qui était là... peut-être pourquoi pas discuter avec les gens de guitar-pro, j'ai vu que c'était des français en fait mais qu'ils font un peu l'outil parfait pour guitaristes où tu as ta partition, ta tablatures, les pédales qui correspondent avec et tout ça, et nous nous intégrer là dedans en disant que, ben en fait tout ça, ça peut être tout directement ta guitare sans être branché sur rien et voilà... Donc du point de vue musical, intégrer ce qu'on fait nous dans le son intégré à des problématiques plus générales qui pour l'apprentissage tu disais, et à la pratique, la composition et voilà et proposer un outil quasiment alternatif on pourrait dire à des outils type écran-clavier, mais qui sont l'instrument lui-même. Et tout ça, ça passe direct dans l'instrument, c'est pas ça qui fera que tu verras ta partition, on n'a pas encore prévu que l'instrument t'affiche la partition mais en tout cas voilà, avoir l'instrument comme interface aussi bien pour l'entrée que pour la sortie, de tout ton monde, toute ta vie digitale en tant que musicien. Ça, ça m'intéresserait bien et ça de manière connectée surtout. 
c'est-à-dire que l'un des enjeux qu'on voit derrière ça, c'est que par exemple, l'un des copains qui a une guitare avec ses propriétés, avec toutes ses données propres que tu peut charger dans une autre guitare, donc qui va changer de timbre, qui va avoir le même timbre que l'autre, que ta musique à toi tu vas pouvoir la partager plus facilement... tu vois, une une sorte de twitter de la musique où petit à petit t'es besoin de rien et tout de suite la communication, elle se fait très vite entre les gens et le partage se fait facilement. Ça c'est sur le long terme parce qu'il ya plein plein plein de problèmes techniques liés à ça encore aujourd'hui... mais je pense que ces habitudes là, elles vont pas partir. Le fait que les gens commencent à avoir le plus en plus l'habitude d'être en contact en permanence avec du savoir, avec les autres grâce au numérique je pense que ça, ça va rentrer de plus en plus dans la pratique musicale et qu'on va vouloir avoir —pouf ! instantanément je fais ça et je vais pas me prendre la tête et comme avec mon smartphone j'appuie sur un bouton et j'ai tout de suite le résultat que je veux, ben dans nos pratiques musicales tout ce qu'on imagine, on va vouloir l'avoir matériellement. Donc c'est une vision un peu cognitive de la pratique et du numérique, dans la pratique d'un instrument où... le rêve pour moi, le truc ultime de cette guitare ou d'un autre instrument c'est que t'as un truc en tête, et ben juste ça te le fait... C'est aussi simple que ça. 
Et quand tu apprends, t'as différents morceaux, ou souvent c'est la musique pop, jazz ou tu vas apprendre toi-même des choses, ben t'imagines des choses et d'ailleurs la construction mentale elle est intéressante en soi, mais qu'en même temps tu puisses l'avoir matériellement, directement... je sais pas, t'es en train de travailler le solo de Jimmy Page de Stairway to Heaven comme tout le monde, tous les guitaristes l'on fait à un moment... et ben tout le monde fait ça avec sa guitare acoustique parce que bon bah, tu as a sous la main et t'es en train de travailler que tu as fait le début avant qui lui, est complètement avec un son clair et que t'as pas envie de te brancher, ben que tu puisses avoir “ ah oui tiens, là il est passé en disto, ben t'as juste ta disto et puis voilà quoi ” C'est ça que j'imagine sur le long terme que tout ce que tu imagines, ben juste que tu puisse l'avoir matériellement... c'est pas la grande... (rire) peut-être que tu vois beaucoup plus inspiré moi... je suis très... 

VG — non c'est très bien et c'est d'autant mieux que tu as répondu au passage à une question dont je me rends compte que j'avais oublié de te poser concernant l'importance de la communauté dans le monde du numérique de manière plus générale et dans les instruments comme ça... 

AM — oui, à mon avis ça va être le gros gros enjeu, comme dans le reste quoi... C'est juste que les gens ne sont pas différents quand ils vont sur facebook discuter avec leurs copains et quand ils font de la guitare. Ça reste les même gens, donc voilà... 

VG — une pervasion du numérique où ces notions là de communautés, de plateformes sont très présentes

AM — voilà peut-être que ça va partir mais moi je pense pas. Je pense que c'est quelque chose de très humain. Ça s'accélère avec les technologies numériques mais qu'on a ce besoin là de (se regrouper, mimé avec les mains, NDR). On est de plus en plus on est là dedans et même pour une efficacité... je veux dire mais même en terme d'efficacité technique on pourrait presque dire. Parce que dans ta pratique musicale, tu as besoin. Tu as besoin dans ton apprentissage, tu as besoin d'échanger un maximum, quand t'as appris ton nouveau morceau, si tu peux pas tout de suite le jouer à tes potes et ben tu sais pas si tu... enfin, je sais pas, après c'est ma vision peut-être à moi des choses, mais en général je pense que quel que soit le truc, quel que soit les nouvelles connaissances, tu as besoin ensuite de la mettre en œuvre, juste pour avancer quoi, même pour apprendre le morceau d'après, ben t'as besoin d'avoir présenté en audition ou avec tes potes à la maison, ton nouveau morceau voir comment c'est reçu, échanger avec quelqu'un qui va jouer une autre partie du truc et le fait de le faire en live, tu vas le ressentir différemment, ça va te donner des nouvelles idées, c'est par cette interaction en permanence. Même en termes d'efficacité mais pour toi, pour avancer, t'as besoin d'être connecté, pas en permanence permanence, parce que t'as besoin de travailler individuellement aussi mais très vite il faut que ces choses là se mettent en place et les technologies maintenant elle permettent de faire ça de manière complètement folle. Tous les youtubeurs et tout ça c'est incroyable... 

Et je pense que nous, notre réponse à nous par rapport à ça c'est que la qualité sonore elle est rarement au rendez-vous dans tout ça en fait. Elle n'a pas été... elle n'a pas pris la vague du numérique en fait... la qualité sonore, voilà, les youtubeurs, on a des super trucs tu as des supers mecs qui font des compos super intéressante sur youtube, mais souvent le son est pourri. C'est con mais c'est complètement dépendant de ces technologies là... Ou bien pour avoir un bon son, il faut un matériel très sophistiqué mais c'est pas intégré dans le... dans ces nouveaux objets là et nous on aimerait que ce qu'on propose ça soit le truc qui permet ça. Parce que si tu as composé ton morceau avec notre guitare et que tu joues, et que le youtubeur il joue chez lui, il fait sa vidéo, ben après si tu charges sa musique que lui il a écrit, tout son cover de je ne sais pas quel morceau connu de guitare, ben nous tu le charges dans la guitare et après tu l'écoute ben t'as ta guitare qui joue ce que le mec a joué. Donc en termes de qualité sonore c'est incomparable avec ce que tu as... et lui il a eu besoin de rien s'il a la même guitare chez lui, tu vois ce que je dire, si lui il a la guitare, et qu'il s'est enregistré avec ça, toi tu charges ce qu'il a fait, tu l'écoutes dans ta guitare, t'as une super qualité sonore. Alors peut-être qu'aujourd'hui c'est encore un peu long à mettre en œuvre, donc il va falloir qu'on arrive à faire ça vite et il y a problèmes techniques, donc c'est sur le long terme, mais je crois beaucoup plus à ça que de rester complètement avec des matériels qui sont indépendants dans la communauté, tu vois ce que je veux dire ? Si tu as une unité dans l'instrument et déjà la base, dans les gens qui partagent les choses, ça va simplifier énormément ça. Il y a un côté normatif si tu veux de types de captation, types de réécoute, types d'enregistrements que tu fait et après qu'ils partagent, et les autres sont dans les mêmes conditions que toi. Ça je pense que ça peut accélérer... comme des outils numériques voilà des plateformes, sauf que là, la plateforme elle passe par un hardware spécifique... 

VG — qui est une sorte de contre-argument à la modularité qui est caractéristique de tous les objets électronico-numériques... 

AM — voilà t'as compris que là, en ce moment, je suis à l'opposé de ça... 

VG — dans la synchrèse... 

AM — la quoi ?

VG — c'est un terme de Michel Chion qui parle du fait qu'image et le son fusionnent, il parlait de ça pour le cinéma, qui sont a priori deux choses indépendantes

AM — ah, je connaissais pas ce terme là...

VG — est ce que tu as une minute pour me faire une démo ?

AM — ben oui, quand même attend, t'es venu jusque là... alors cette guitare là...  je vais te la présenter comme on la présente au musiciens... tu vas voir justement le type de façon de partager sur la technologie... donc voilà une guitare acoustique, tout à fait correcte du coup ... 

[demo] 
