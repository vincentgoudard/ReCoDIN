\chapter{Interview : Adrien Mamou-Mani}

\section*{Biographie}

\section*{Transcript}

VG — Qu'est ce qui à l'origine a motivé cette idée qu'on pourrait dire farfelue de vouloir faire de la musique avec des instruments avec des outils numériques et des ordinateurs plutôt que de prendre un instrument existants... qu'est ce qui a motivé sa en premier lieu ?


AM : il y a 2 aspects complètement différents qui m'ont motivés. Le 1er aspect est qqch qui vient de la recherche, c'est peut etre un peu original, cela n'a pas été au départ une problématique musicale, mais née de mon histoire de chercheur sur la physqiue des instruments de musique. Ma thèse était sur la vibration des tables d'harmonie et j'arrivai bien à analayser comment ça vibre et comment le geste de lutherie (des luthiers) vient influencer certaines particularités vibratoires. J'ai pas mal travaillé avec des luthiers sur la relation entre leurs techniques de fabrication et le résultat vibratoire des instruments.
Et donc le 1er aspect qui m'a motivé ensuite à aller commencer à convevoir des choses comme ça c'était un peu dire "est ce que ces choses là qu'eux font par leur savoir faire, est que par de l'électronique on est capable de la même chose. C'est à dire que moi, sans être luthier, est ce que je suis capable de faire comme eux. Ca ça a été une motivation importante dans la conception. Et en fait il y avait tout un champ, il y avait Charles Besnainou qui avait déjà commencer à travailler la dessus, Steven Grifin (?) à CalTech. 
Je me suis rendu compte qu'il y avait des gens qui avaient déjà essayé de faire ça avec, d'avoir une approche dison d'accousticien qui veut essayer de comprendre comment la chose fonctionne mais qui va aborder son sujet d'étude par un autre biais qui était de l'électronique au lieu d'être de la mécanique si tu veux. 
Donc ça, ça a été un truc qui m'a amené à faire ça. Et deuxième aspect ...

VG : Juste pour le 1er aspect, pour poursuivre ce que tu disais, quand tu dis "aborder ce point de vue de mode résonance de table d'harmonie avec de l'électronique, c'est (ou ce n'est pas?) une envie de modéliser la table d'harmonie via des méthodes numériques

AM : Non non non, moi ce qu'il m'intéressait dans les savoir-faire des fabricants, c'est comment  est-ce que eux, quand ils déformaient les choses ça changeait la qualité de l'instrument. Parce que moi je regardais les rasonnances et comment ça les modifiait mais ... et donc je me suis dit est-ce que c'est possible de changer les qualités, mais non pas mécaniquement comme eux ils font, mais électroniquement.

VG : en les pilotant avec des vibreurs ...

AM : en les pilotant, voilà, c'est ça ... c'est euh... un peu arriver au ... au départ c'était ça du moins, arriver au mêle résultat qu'eux, et d'ailleurs moi, pdt mes recherches c'était intéressant de comprendre aussi ce qui fait la qualité, mais par qqch de complètement empirique, direct, sur l'instrument, moi je règle comme eux ils font mais par de l' électronique, donc j'ai une maitrise, je sais ce que je fais par l'électronique.
Ça, ça a été qqch d'important, euh... ouais, de qqch qui vient disons de pronlématiques de recherche, disons ... on pourrait presque dire c'est presque opportuniste, c'est à dire comment le ... une thématique de recherche de l'époque m'a fait sentir qu'il y avait des choses qui par mes connaissances à moi, par mes compétences dans un domaine spécifique pouvait m'amener à faire comme les luthiers, enfin je sais pas comment te dire tu vois...

VG  : oui, oui ...

AM : c'était vraiment un peu en temps que chercheur qui d'un coup bascule en recherche appliqué"e ou disons développement par rappprot à ma curiosité à la base de chercheur. Donc ça ça a été un des truc mais ça n'a pas été ça qui a été le déclencheur vraiment.
Ce qui a été le déclencheur, après je suis pas mon propre psy mais... (rire) ya.. euh... enfin c'est très précisément en 2010, euh, ma femme était enceinte de ma fille, ça a été  je pense  qqch de très important pour moi, càd elle elle était, enfin on était en train de construire qqchose quoi, son ventre qui grossissait et je sentais que je servais à rien, donc j'ai eu envie de faire qqchose avec mes doigts, bon...
et ça a été au moment où, donc là je travaillais à Londres à ce moment là, je travaillais sur les hautbois, donc des choses qui n'avaient rien à voir et à ce moment là il y avait de plus en plus d'appli qui sortaient sur iphone, il y avait, tu sais l'ocarina là des gens de Stanford, comment ça s'appelle ? Smule... il y avait, il comment à y avoir plein d'appli qui sortaient et je me souviens très bien car il y a mon frère qui m'a appelé et qui m'a dit "bon Adrien, il y a plein de trucs qui sortent sur les iphones, toi, avec tout ce que tu connais des instruments de musique t'es pas capable de nous faire une petite appli là, un truc sympa ?" ... et ça ça a été un déclencheur en fait.
Donc il y avait ce truc là, j'avais mon projet qui m'intéressait à Londres mais ça arraivait à, enfin j'avais envie d'avancer... ma femme (fait un geste montrant son ventre enceinte)... mon frère qui m'dit ça .... et ... et dans le labo il y avait qqn qui utilisait... tiens c'est la 1ère fois que je le dis ça...

VG : faut faire un bébé en fait ...

AM : ouais c'est ça... d'ailleurs la 1ère guitare je l'ai appelé Annabelle comme ma fille ... enfin bon tout ça était un peu mélangé...
et dans mon labo il y avait ... et en fait moi je me suis dit bon mon frère c'est marrant ce qu'il dit mais moi je veux pas faire un truc que je considère comme un gadget, parce que justement pour moi tout ce qui était numérique, toi c'est ça qui t'intéresse, mais moi à cette époque là j'ai complètement ça, la musique nuémrique c'était pas du  tout mon truc à cette époque là, j'ai eu des phases tu vois.
et quand il m'a dit ça je me suis sdit non je vais pas encore fair eun gadget,  mais je me suis dit par conrtre est ce qu'il n'y a pas moyen de connecter les 2 mondes, moi ce que je fais sur l'acoustique et ce qu'on est capable de faire avec du numérique et... 
et donc c'est là que mes recherches  sont intervenues et c'

 ------- unparsed

revenu j'ai pensé à ce que l'intéressé est dans mon labo il y avait quelqu'un d'autre ça parfois que le jaudy médial des rôles ans qui était un post doc d'angleterre qui avait bossé avec des boîtes en aéronautique je sais que l'équipementier de d'airbus ou de boeing qui utilisé des petits actionnaires électrodynamique nxt les choses qui maintenant sont vachement répondu même en 2010 tout ne s'est pas trop débourser 2000 à peu près des premiers noms de là dessus et donc moi je connaissais pas donc je moi je lui disais j'aimerais bien trouver un truc qui vibre j'allais trouver des choses dans les boutiques à londres cité des gros truc qui n'allait pas et il m'a dit avec moi j'avais utilisé dans pour les avions d'avoir utilisé des petits trucs regarde la marque est mixte et qu'il marque anglaise lui ce moment et donc j'ai pris un iphone j'ai pris un actionneur et 10 des jeux les culés sur une guitare à deux balles de match et j'ai commencé à balancer des trucs dans la guitare ça c'était mi 2010 ou 3ème trimestre 2010 france j'ai fait ce truc là dans la guitare la maison et il a ça dans ma tête ça explose et ça a tout fait exploser parce qu elle quand j'ai fait cette expérience là je me suis rendu compte qu'il y avait quelque chose d'assez peu invasif où je pouvais continuer à jouer de la guitare acoustique j'avais en plus du numérique allait dont je me suis rendu compte que tout ce que j'avais fait avant sur comment ça résonne au couplage avec les cordes les propriétés vibratoire de la caisse tout ça avait un impact sur enfin tout ça me servait à comprendre ce qui se passait là donc voilà j'ai repris j'ai rappelé charles et ses mots jeudi total et fait des beaucoup de feedback et dans l'est dans les guitares pour moi ce que tu avais fait et apparaît là je monte un projet de recherche avec la nr aussi grâce à une super rencontre avec baptiste se met à nu st pierre et marie curie était lui spécialiste du contrôle vibratoire pour les remercient pour autre chose et pour vous cette série de choses entre 2010 et début 2011 j'ai monté le projet nerfs ça a marché et après donc j'ai créé ce truc smart instruments se produisant un siècle donc voilà ce qui m'a motivé allait commencer à jouer avec les technologies là c'est voilà c'était tout bête c'était juste je pense quand j'ai commencé à jouer avec cette technologie là c'est tout l'adieu au fil de nos discussions que j'ai joué de la toujours de la guitare mois joudelat j'ai joué de la guitare mais vraiment amateur dans un groupe de rock je lui toute la basse et chanteuse de rock les choses un petit tapis stick j'ai commencé à rappeler enfin des chansons c'est toujours bien mais si moi je veux plutôt du violoncelle ils ont jeté dans cette classique petit outil et ensuite j'ai fait une année n'est qu'un instrument avec serge de l'aubier les tentes étaient pas encore là mais je crois qu'en cd croisée après soit fin à la faim et à fait plaisante pas avec les voilà il ya eu ça avec pierre le goux on allait commencer à faire aussi un peu de musique électronique à la maison voilà google il ya plusieurs choses comme ça entre d'instruments classiques électronique et moi sinon sur tout ce qu'il ya eu c'est que moi tout ce qui m'a motivé depuis 
j'ai 19 20 ans c'était la musique concrète pierre schaeffer j'étais un grand fan de de pierre schaeffer donc en fait c'est ça surtout qui au bout du compte ce que je voulais c'était réussir à trouver les clients un équivalent de la musique concrète aujourd'hui c'est un peu naïf et entrer c'était ça qui m'a motivé à la base dans mon histoire musicale disons c'est c'est ça qui a joué souvent plastiques musique concrète lecture de pierre schaeffer et désireuse sommer quelque chose de l'heure des airs sonore qui peut être cernée par la musique concrète qui n'a pas vraiment ses instruments comme on parlait des cimetières par des versements dans les désirs dans les motivations qui ont poussé à hama à créer des objets comme ça enfin je sais pas non c'était pas désir c'était plutôt essayer trouvé que ce qu'est ce que c'est qu'est ce que c'est 
la musique aujourd'hui c'est tout ça c'est pas parce que j'ai des banalités vous reste valable voilà c'est fait les grecs de bon et donc voilà je me suis juste dit si pierre schaeffer il était là aujourd'hui avec ce que lui évoque cette rencontre entre technologie perception et et sciences ils ont qu'est ce qu'il aurait fait aujourd'hui qu'est ce qui serait un peu le la petite boule guînoise de notre de notre temps si je devais faire quelque chose qui représenterait mon époque musicalement ça va faire un objet musical disons comment dire au début je pensais plutôt d'un point de vue composition quand j'avais 18 ans après j'ai pensé à musique électronique après le poussif un monde mais ça pouvait être composition ça pouvait être un instrument mais je voulais essayer de trouver quelque chose qui pour moi représente et qui représentait pour moi notre époque et avec sa fille jade voici embarqués sont très cools avait l'impression que j'avais débloquer quelque chose que vous 
alliez quoi ça a donné un sens à ces trucs là tout ce que je cogite et sur pierre schaeffer je me suis bon finalement peut-être que ce truc là qui est la rencontre entre le monde physique et non numérique c'était ça notre pour tout résumer et voilà alors par rapport à l'habitude de changer j'ai interviewé récemment nous emmener des gens qui pratiquaient donc mes questions étaient toutes orientées sur comment le musicien qui est une ce genre d'instrument oui gère un certain nombre de propriétés parfois contraignante du numérique le fait que ça marche ou ça marche pas c'est oui ou off le fait que qu'on peut passer de manière brutale d'un contexte à un autre qu'il ya une continuité par rapport à peut-être que tu favori que toi sauf à cette même question ressurgit chance et tu parlais de connecter les deux mondes entre l'acoustique lumière et demi c'était une manière de promo de formuler de manière chronique ans ça serait de te demander si c'est pas introduire un peu la bergerie km du numérique dans l'acoustique dans la torre renaissance numérique dans l'acoustique du numérique 
dans le coup si la bergerie c'est la critique je ne sais pas je pense notamment le nick a peut-être al'avantage d'être répétable donc on a quelque chose si ça marche une fois ça marche de la même manière à chaque fois c'est parfois de aussi partie de ces lignes est l inverse une table d'harmonie c'est les moyens de par elle il faudrait il a je repensais notamment à une conférence de données collins qui a cherché les musiciens qui espéraient tout moment de beaucoup de manières très empirique avec l'électronique et qui a fait une présentation où il est allé un peu les différences entre hardware et software oui notamment le fait que software s'est constamment dans le présent c'est constamment mis à jour que ça marche ou ça ne marche pas alors qu'elle va vers ça que le marché même en étant un peu cassé pour wii donc comment comment tu vois un petit peu cette fonction terme dans le dos de l indice que huawei hockey comment est ce que je vois c'est demandent ensemble alors une question comme ça le premier truc qui me viendrait ça serait te dire que tu perdes approche je veux dire en une phrase en approche c'est avoir une acoustique programme c'est tout simple c'est à dire le numérique est au service de l'acoustique ce qui m'épate le plus c'est qu'en fait tu te sens même pas qu'il ya du numérique c'est ça qui m'intéresse le plus c'est bien parfois enflés on s'amuse des cliniciens on leur fait des traitements et lui ils ont juste 
l'impression que c'est une autre qualité acoustique maison pendant l'impression que du numérique ça c'est ça qui m'intéresse le plus c'est numérique le service il pourra voir pour donner au monde physique la cap des capacités de programmation du numérique c'est à dire dans les diverses attitudes de ne pas trouver en france exactement comment traduire ça à côté voilà polyvalent au modifiable qui veulent avoir une vie voilà te dire simplement il aurait sûrement plein de trucs mais en tout cas ce qui est vrai pour moi la vérité c'est que je le vois les choses comme ça utiliser la technologie au service de la russie pour pouvoir le programme pour cette métaphore peut-être mal droite de loudes radin je vais plutôt d'un côté j'ai eu la bonne surprise de voir que pour les luthiers quand donc les productifs et dur de l'association lui qui est violement quand je leur faisais des démonstrations il ya eu il visitait comme nous tu es juste un luthier où on fait de la lutherie pour du mozart et toi pour que tu as dû trier sont intéressantes il faut qu'elles servent la musique de maintenance de demain mais de me voir faire des traitements sur des violons échanger leurs talents pour en temps réel pendant curieuse louis dessus ça les a pas choqué donc ya pas vraiment lu ce truc de hacker ce qui nous amène pour me nourrir l'ordinateur dans l'audio ya pas eu ça j'ai pas rencontré de salé seule chose alors peut-être que c'est plus que tu es musicien alain rolland ce que tu dis toi parce que des gens qui pratiquent les seules choses qu on a eu pas sur les guitares pop et maintenant les sur les prototypes d'avant en particulier sur les clarinettes par exemple et 
sur les violons aussi c'est qu'en fait il ya eu certains musiciens qui ont senti que on a mené une énergie concurrence à ce que le en fait elle et ça ça les à dire à certains qui ont été gênés dire que normalement un instrument et les filles et d'instruments acoustiques et la cdefi ge cee ensuite qui vont de 300 des faire émerger des choses maîtrisé en connaissant bien à propos en reprenant bien tous sont un peu différentes notes à l'exploiter au maximum et un nom quand on arrivait avec le traitement d'un coup l'instrument ce n'est plus le même dix mois je peux pas faire ça j'ai travaillé le truc est maintenant en fait ça sonne plutôt pas et bon pour ça devient gênant d'avoir lancement refuge puisque d'habitude c'est eux qui donne ce côté souvent dynamique dira seulement la façon de jouer ça c'est un problème et un deuxième aspect ça a été des gens qui ouais il met de l'énergie dans l'instrument est d'avoir une énergie concurrence qui vient corriger ils ont perdu 5,6 avec quelqu'un d'autre qui joue avec legault et donc ça ça peut donner aussi donc pour lui dans la bergerie qui pensait plutôt côté interprète côté musiciens que ça ça a pu dans certains prototypes pour certains certaines pièces qui avaient été écrits de s'appeler génie aujourd'hui on n'a plus tout ce problème là avec nos produits a vraiment dit tard parce que en termes fonctionnels c'est des choses qui se faisait déjà ce qu'on propose là c'est des choses qui se faisait déjà avec du traitement électronique mais que non vu son enterrement l'hectare donc pour rien côté a encore été transparent de saoû déjà connue entre ici c'est pas un nouvel instrument pour eux la rendait dans un cadre où un peu comme ce que je visais annoncé le numérique au service de la guitare de ce que peut imaginer déjà que ça soit des pédales analogique que ça soit des tomes acoustique qu'on change nous on leur fait revivre ça mais avec une expérience plus simple et plus direct mois donc là on n'est pas ce côté ou dans la bergerie parce que parce qu'en tant fonctionnels à mon avis juste c'est quelque chose qui se représentent déjà mais il ya un côté high-tech qui est un peu problématique la plus cotée lutherie quand on parle avec des grandes marques et de leur dire qu'on va intégrer ça dans la revanche seulement voilà ils n'ont jamais encore plus à tapas
 dans les instruments acoustiques ce genre de l'ordinateur avec une technologie très particulière que nous on est en place justement parce qu'on fait du numérique mais avec du hardware pour du numérique très spécifique par des latences donc il ya quand tu parlais de cette distinction annuelle sa victoire pour moi dans le numérique jeu mais aussi la partie hardware disons qu'on a besoin de pouvoir faire dans le traitement numérique et pour par couple et ça avec lui est donc là par rapport à eux les luthiers aujourd'hui installé ces trucs là dans delta c'est quand même c'est pas loup dans la bergerie c'est plutôt au bois ou comment on va faire certains musiciens j'ai l'impression de l'habitarelle elle rassure en grande partie à cause du fait qu'elle est liée à un objet qui est connue qui est identifié et qui a fait l'histoire et qu'il y à un minimalisme de l'interface numérique qui vient de sujets comment tout est pensé qui donne une place assez discrète à sa demande et vidéos de démonstration que j'ai vu où il y avait les vibratos chorus reverb on reste sur des effets qui reviennent un peu incorporer ce qu'on pourra voir avec une pédale d'effet dans ce sens que je suis j'imagine qu'il est possible de faire des choses que pour présenter au début et pour rassurer tout le monde c'est probablement vient de commencer car son heure de feu d'attaqués nas trapu imagine pas autre chose possible oui et par rapport à ça et là aussi une question qui se pose par rapport aux équipes je veux tu mets ton site est en situation qui sont 
 toujours en mouvement de la question de l'apprentissage de ce que tu disais avec les violonistes oui mais moi je le fais et certains auraient tendance à prétendre que on ne peut pas les apprendre parce que l'objet ne change sans arrêt oui ok comment tu fais alors par rapport à ce que tu disais au départ donc en effet à ça c'était une magique dans les tests qu'on a fait ça me rappelait les trucs là où justement la façon de faire des tests et tous les protocoles mais il faut toujours rester très ouvrir parce que tu as des surprises et nous des premières surprises qu'on a vu çà a été donc au début on s'était très fixées sur des effets connus et on s'est rendu compte que dès que tu arrives à un certain niveau de musiciens des effets connus et vidéos qu emi group je les ai déjà et moi j'ai mon matos qu'il faut jamais bonne pédale est donc ce qui est ressorti c'est que des trucs que nous on pourrait considérer comme des défauts c'est ça que les intéressés ils ont essayé de tirer au maximum sur tous les trucs qu'ils allaient j'avais entendu en particulier le truc qui marche le mieux c'est tout ce qui est su stein tu sais un peu les niveaux là tu vois ce genre de système donc nous on a des sorts de ibo dynamique équipe est bien loin sur toutes les cordes et que mettra différemment sur chaque note et tout ça et au savon bachand exploiter ça et c'était le truc qui justement est à la marge et qui ne ressemblait à rien et finalement dès qu' on est arrivé à un certain niveau de musiciens c'est ça qui est intéressant c'est que l'on utilisé pour rassurer les gens oui mais c'est pas ça qui va faire tes pro vie vont l'acheter parce que eux ils 
 veulent justement le truc unique qui d'habitude est impossible à faire on ça il ya eu ce côté sue steyn et aussi des côtés très originaux c'est la transformation de l'acoustique de l'instrument donc ça c'est particulier aussi donc on a en charge le temps j'ai pas voulu mettre l'accent là dessus au départ parce que c'est très subtil de booster un peu ton show acoustique transformée et donc voilà j'ai pas mis en avant met donc en effet il ya ces aspects là qui sont d'ailleurs beaucoup plus révélateur de nos technologies que les choses que l'on a mis en avant aujourd'hui qui la technologie importante de plus implicite là dedans dans le contrôle du feedback met donc en effet il ya tous ces aspects là et on s'est dit que pour c'est exactement ce que tu as dit juste pour vous rassurer pour commencer à tirer les gens il fallait déjà il fallait pas les envoyer sur des choses qui étaient complètement étrangère pour qu'ils puissent commencent à s'approprier leur interface pour qu'ils puissent garder leurs références avait dit à tout le temps il faut sincèrement de la guitare augmentés quoi tout en ayant des capacités en plus donc ça c'est la première partie de ce que tu disais donc en effet il ya ça ensuite sur le côté si j'ai bien compris comment plus parler d'eux comment essayer d'avoir quelque chose qui est stable et afar et qui va pouvoir s'en crée au fur et à mesure parce qu'avec ces technologies à ça bouge en permanence et donc il ya toujours le risque enfin en tout cas est ce que dans sa nature même est ce 
 que ces trucs-là reste sur des choses éphémères et qu'il faut toujours renouvelée renouvelé c'est bien ça tu voulais dire 30 attrition un peu est ce que je voulais dire c'était ici j'apprends à ramener sa d'une chose concrète c'est je parlais de la pédagogie dans le sens oui si tu vas un conservatoire tue pas oui tu peux apprendre la guitare on t'explique un certain nombre de techniques de jeu toussaint avec ces nouveaux instruments enfin les instruments électroniques de manière générale la question se pose vraiment franchement c'est comment comment tu enseignes et les trois coups stick dans le cas d'un instrument ou purement numérique vraiment vraiment une poêle dans mon cas il ya un enseignement électro acoustique qui existent mais qui est une école assez particulière dans la semaine il va musique électronique avec ce genre d'instrument que j'ai compris est ce qu'il ya une pédagogie de décevoir parce que votre contrat aujourd'hui avec ce premier produit là tout ce qu'on fait c'est basique c'est juste que des choses qu'ils ont dans leur paie dame qu'ils ont avec leurs leur la rance en bluetooth ils ont dans leurs guitares donc il n'y a pas je cherche on cherche pas dans ce dans cette solitaire on cherche pas à les exploiter les spécificités du monde pour essayer de ensuite les apprendre les transmettent et ou non nous on veut pour eux au lieu d'avoir leur pédalier ils ont l'air plus d'indiquer dans leurs écrans c'est tout ces salles de vie sain ça quoi voilà c'est un travail du son qui font déjà en plus c'est que les guitaristes ont déjà l'habitude de faire par deux dispositifs que nous on intègre dont des guitares acoustiques pour profiter de des questions de profiter des propriétés des caisses des questions de qualité sonore des questions pratiques sortir de la guitare est une question de connectivité pour qu' il puisse échanger entre eux et que quand ils écoutent un cours en ligne au lieu d'avoir sa leçon qui sombra dans l'intérim en soi une guitare avec le maximum de qualité possible question qualité sonore de connectivité et de simplicité 
 du dispositif donc pour ce produit là on n'est pas dans disons un un procès à un produit pour aider à processus créatif que nous on viendrait proposées pour faire des choses vraiment nouvelle et qu on va essayer ensuite d'ap de faire apprendre et voir comment partager ça on n'est pas là dedans on est plutôt sur quelque chose pratique part à tapis sur un bouton et et voilà alors après on peut discuter sur ce que je pense que j'ai fait avant bien sûr des aspects les plus créatifs ou là il ya un an il ya d'autres sujets et d'autres sujets qui arrive je sais pas si tu veux qu'on parle de ça qu'on se concentre sur ce brûlant par le forcément mais peut-être la question que l'algérie a trait à ça c'est dans le design de la smart guitare ces nouveaux apprentissages sont aussi des fois liée à des nouveaux gestes oui et hills à la smala on a reçu à l'elysée revenus oui l'interface est problématique pour ça j'imagine que c'est un choix qui a des raisons bonnes ou mauvaises sans parler qu'elles sont bonnes ou mauvaises mais vous auriez pu j'imagine facilement et acéré des capteurs gyroscopiques il ya des capteurs oui pression des glissières toutes sortes de choses gillett la table complète pour être farci de boutons et ce n'est pas le cas je me l'examen fois c'est vers une forme alter de ce minimalisme exact et qu'est ce qui est important je suis pas m'avoir que c'était ça c'est le premier qui me posent des artistes sont là parce que dans le monde et guitariste eux ils ont une vision très normal de ça mais tu as raison c'est bien normal quand vient plutôt niveau de technologie il y avait au départ on avait plein plein plein de possibilités justement c'est super intéressant et d'ailleurs il y en a certains qui nous pousse à dire j'aurais bien plus si ça et ça et là moi je suis mais pour l'instant tout cas pas dans ce flot de lave alors bye sa revanche ce que je disais avant en fait ces dons une philosophie de réalité augmentée que son ebit a augmenté moi mon idéal c'est le numérique au service de la gousse ticket personne se rend compte de rien en fait c'est ça ça l'idéal que j'ai 
 en tête c'est un numérique intégré en fait dans notre monde physique parce que moi j'ai toujours été très frustré ou critique sur face à toute la complexité du monde physique on a et qu'on a encore du mal à analyser que quand on fait des instruments quand après finalement un appui sur un clavier sur une touche avec ce qu'on a j'ai jamais cru à la sur du long terme comptes andré vers quelque chose avec la qu'on arrivera à la complexité d'un physique moi j'ai toujours vu les choses en laissant partir de déjà de notre monde à nous et je trouve que ça rentre dans une philosophie plus actuelles de développement durable et de foot voilà les gens il se met à apprendre la guitare pendant 10 ans on a on a tout un répertoire on a tourné on a toute une richesse je veux quand ils prennent ça ils se disent qu'ils ont le potentiel d'utiliser toute cette richesse qu'ils ont déjà donc c'est dans cette logique là de s'appuyer vraiment sur l'existant dans une logique de réalité augmentée ce sera autre chose derrière qui va venir compléter cette chose là et qui doit être le moins invasive possible le moins intrusif possible et qui rejoint ce qu'on appelle aujourd'hui les objets connectés l'internet des objets par rapport à ceux visés précédemment sur qu'est-ce que c'est notre époque moi j'essaie de le voir comme ça quelque chose comme ça où finalement les ordinateurs il va en avoir toute façon partout ça va être intégré par tous et qu'il n'ya plus de distinction de ces demandes-là du monde physique et du monde de l'ordinateur et elle ça c'est un démonstrateur de sages le vit comme ça viendrait j'aimerais que ça soit un démonstrateur de ça ou après avoir fait des objets dédiés dont les ordinateurs après avoir commencé à porter ça dans des smartphones et quelques objets ben maintenant on est rentré dans le truc où ça commence à être de plus en plus partout ça commençait de moins en moins intrusif donc on peut avoir un idéal comme ça où on peut s'appuyer sur n'ont pas le savoir faire qui dans le monde numérique mais savoir faire qu'ils ont du monde physique en l'occurrence des instruments acoustiques et ou industrielles sinon des trucs l'industrie troisième dépenses et où j'ai encore un peu le même hymne est lui même idéal un peu comme ça ou ou l'industrie existantes a juste intégré ça et s'arrête complètement intégré tu vois c'est renier un idéal complètement de fusion des choses et tout et partez le symbiose mondiale semblable symbiotique au prince deux titres de très bons les deux jolies de renationaliser sage possibles tels que j'ai pas lu mais voilà c'est ça d'accord je vois qu'ils étaient déjà presque une heure au pied ouais j'aurais deux cas soumis du groupe avait ses salles pas d'acheter malin bon voisinage parmi beaucoup cette année environ seulement des barres noires un petit aussi le gênent pas pour se moquer la première question s'est elle est liée au contexte actuel de production musicale oui dans lequel une très grande un très grand 
 pourcentage de la musique on ne serait pas là précisément mais qui a tendance à grossir encore de la musique qui est produite enregistrés sur cd est enregistrée en studio à l'aide de logiciels de montage qui permettent de tout recomposée et bien souvent d'ailleurs aller programmé à la main c'est à dire que plutôt de faire jouer un batteur on va enregistrer avec la difficulté qu'on va voir après remettre son jeu en place avait ajusté pour les besoins de production les programmer directement avec salut mauvais coups qui permettent d'ajouter du rubato ou que sais-je de tout à tout un pan de la musique produite sur les enregistrements d'accord qui est fait par des méthodes offline démission en fait will play du jeu wii et par rapport à ça c'est dans la fabrication d'un instrument a un côté caen paris quelque part en fait de miser son asthme en dix ans la somme on n'est pas morts on engagement elle en parle il ya un an les gens comment tu vois le la smam guitare ou la position de manière plus générale des instruments dans cette production musicale ou où le numérique a tendance à assister ouais de plus en plus la production musicale que ce soit par des méthodes a rolling ou pas des méthodes voilà c'est un nouveau palais mais je ne connais pas grand chose moi je veux dire juste très simplement comment ça se passe un peu aujourd'hui que l'on parle doucement avec des gens qui sont justement côté studio et quand ils voient seront chez eux ce qu'ils nous disent globalement c'est moi le maximum que je peut choper à la source je choque sur l'instrument parce que tout ce qui se fait en flammes derrière sur les modifications du timbre et tout pour eux et nous disons c'est si on peut éviter on préfère les cons donc ils sont contents quand on leur montre ça pour l'instant on n'a jamais travaillé on n'a jamais été au bout d'un projet comme ça mais en tout cas premier abord pour par d'autres gens plutôt côté studio de ce qu'il nous dit c'est un oui ça m'intéresse parce que moi je son pied d'eau c'est toujours un problème donc si vous pouvez me faire quelque chose programme qui fait que morales ouvrage est pas ce côté pied de ça m'arrange moi j'aime pas trop rajouté les traitements sur une guitare acoustique a posteriori si vous les aviez en amoureux moi j'ai juste à mettre un micro devant avec déjà le truc et dernis et comme il faut très ça m'arrange aussi donc là pour l'instant moi je suis dit tu pas du tout spécialiste ont créé tout ce que j'entends c'est plutôt que par rapport à un instrument de la prise prises de sons disons d'un an suivant donc ça ne couvre pas tout ce que tu as dit avec la pré poulain tous les aspects aussi de jeu instrumental mais disons prises de sons sur par deux timbres de son instrument voilà plus on parle d'une bonne source musée donc si long silence en plus ils sont plus de votre voix la paix on va mener en amont quand même en ce sens avec notamment l'avocat du diable là dessus et je conçois très bien que si on avait rajouté l'effet waouh sur les guitares de jimi hendrix studio n'a pas du tout marché vp ventes à 
 réméré way oui oui mais comme tu dis il ya tout un pan de la musique qui se fait comme ça maintenant et après je peux dire que aussi à tout un retour aussi à la musique live et aux concerts et plus maintenant que la monétisation par le studio est de plus en plus complexes et sinon je sais pas tout cas côté guitare acoustique service à ce que je peux te dire quand même c'est que j'ai découvert un monde de gens qui se retrouvent dans dans des salons affaires guitare acoustique et les associations de gens qui se retrouvent comme s'avance même en soirée pour aller présenter leurs nouvelles compos avec or petit en petit comité comme ça à d'anglais dans les appartements il n'ya plein plein plein plein de choses qui se montent dans le monde comme ça donc par rapport à la guitare je pense que pour ce produit leur particulier quoi ça colle quand même avec c'était pas la grosse industrie je ne sais pas aujourd'hui mais en tout cas ça colle avec une pratique musicale qui est très très très très très courante encore voilà juste qu'on me dit ça reste un truc est vrai que la production automatisée de musique se boit surtout enfin là où elle avait plus flagrante c'est dans la production typiquement de musiques de films ou économiquement de faire venir un orchestre part avoir acheté la wii de son usine à la guitare et de ce point de vue là est un instrument assez populaire que tu peux prendre les cinq îles manière récemment qu'elle est moins sujette à être peut-être un après puisqu'ils sont deux guitares touche france admire pas mal de choses peut-être on est dans un contexte qui leur matérialisation être question à dix mille dollars vas-y on a commencé à perdre ton passé ah oui oui au parcours on va terminer par le l'avenir oui la question elle est très ouverte c'est cunac lankaise qui selon toi faliez as quand même assez sujet de l'origine musique numérique ouais qu'est-ce qui est selon toi et hélas la voie vers laquelle on va où tu voudrais aller ou ok alors la manière dont ça va transformer notre monde et noël aux réalités qui est perceptible alors on déjà si on arrive à faire un truc où on est capable d'avoir une acoustique programmables quelque chose pas simplement sur les instruments ont discuté avec d'autres secteurs à voir son côté design sonore par programmation numérique pourrait le design sonore des objets relation numérique en discute avec l'automobile avec d'autres secteurs qu'ils veulent tu vas tu rentres ans transformer des temples des objets n'est non pas mécaniquement comme ils le font d'habitude en changeant des choses mais par des traits pong bec donc ça bon voila si on est capable 
 d'avoir des choses en particulier ici on arrive à avoir des choses comme je disais que tu n'as pas qu'ils sont moches mais comme on dit une identité sonore qu'on associe aux nouvelles technologies si on arrive à faire savoir plutôt vraiment quelque chose avec les objets qui résonne plus kazan - a plus aiguë plus grave est que tout ça soit programmables ça déjà c'est pas mal donc appliquer design sonore ce qu'on appelle aujourd'hui le design sonore vraiment à la matérialité les autres qui faisaient ça exactement donc ça ça ça m'excite bien déjà et et je pense qu'il ya un enjeu dans nos sociétés où les objets qui sont ni peut avoir des fonctions différentes il va avoir dix diables rouges est donc tout ça c'est bien d'avoir quelque chose qui peut s'adapter pour qu'ils s'adaptent dont la sonorité tennis ça peut servir tout ça c'est un désastre c'est pas exactement ta question sur les instruments numériques alors moi sur sur les vrais instruments au sens large ouais ok on va aussi voiture ça peut être un instrument pour toi ok voilà donc tu as là dessus donner cette flexibilité ses diverses activités averse est liquide aux objets sas à les rendre des instruments état d'une certaine manière parce qu'il est pour ça ça m'intéresse bien sinon après pour la la suite de ça migros ça serait de ne pas se limiter aux sons mais à intégrer les choses plus aller plus vers de l'interprétation ou aller vers par exemple à vacant par les philadephia quand j'étais là la semaine dernière la cité parce que j'aimerais bien dans la boucle ajouté peut-être un accompagnement qui va s'adapter automatiquement à ce que tu fais donc il va aller le chercher dans des bases de données telle musique que tu as envie de jouer donc on va aller te mettre directement des accompagnements couple et ça avec des technologies qui sont pas sonore mais plus de la musique au sens large du symbolique musical si tu peux avoir en même temps tu as partie sur pendant que toi tu es en train de jouer et qu'il reconnaisse que tu fais qui va te faire les effets correspondant on voit la banlieue et tout ça ensemble collectant qu'est ce qu'il faut à côté voilà donc ces genres de choses qui font à voile à ses employés imaginer son hôte chinois qu'est ce que vous aimez collaboration mais à pas comptés ce coq à la bière il ya des gens de ma musique je vois qu'ils étaient là qui était un cube est ici a lancé louise itunes qui étaient là peut-être pourquoi pas discuter avec les gens de guitare pro l'algérie c'était en france étaient des français en fait mais qu'ils font un peu la tow 2006 parfait pour guitaristes où tu as ta partition tablatures les pédales qui correspond avec les tout ça et nous nous intégrer là dedans en disant que ben en fait tout ça ça peut être tout directement ta guitare sur le manche est sûre et voilà donc du point de vue musical intégrer ce qu'on fait nous dans le sont intégrés à des problématiques plus général qui est qui pour l'apprentissage mobiliser et à la pratique 
 la composition et voilà il proposait un outil quasiment alternatif on pourrait dire à des outils peu écran clavier mais qui sont l'instrument lui mais dans tout ça ça passe à doha ce moment c'est pas ça qui fera que tu verras ta partition n'a pas encore prévu que sa démarche est affichée partition mais en tout cas voilà que à voir l'instrument comme interface aussi bien pour l'entrée coup la première sortie de tout au moins toutes altitudes en tant que musicien ça m'intéresserait bien et est sa manière connecter sur tout c'est à dire que l'on est un des enjeux qu'on voit derrière ça ces cubes par exemple l'un des copains qui a great art avec ses propriétés avec tous sont ces données propres que tu peut charger dans une autre guitare donc à changer de timbres avec le même temps que loeb que ta musique à toi tu vas pouvoir la partager plus facilement tu as une sorte de twitter de la musique ou petit à petit pas besoin de rien et est tout de suite la communication a fait très vite entre les gens et de partage se fait facilement ça c'est sur le long terme parce qu'il ya plein plein plein de problèmes techniques ça encore aujourd'hui mais je pense que ses habitudes là elles vont pas partir en quoi le fait que les gens commencent à voir le plus en plus l'habitude d'être en contact en permanence avec du savoir avec deux avec les autres et de grâce au numérique je pense que et ça ça entraîne de plus en plus sûre plus dans la pratique musicale qu'on va vouloir avoir pas instantanément je fais ça je vais pas me prendre la tête et comme avec mon smartphone appuie sur un bouton et j'ai tout de suite le résultat que je dans nos pratiques musicales tout ce qu'on imagine on va vouloir la voir matériellement donc c'est une vision un peu cognitive delà de la pratique sur et de et du numérique dans la pratique un instrument ou le rêve pour moi le truc ultime de sa guitare ou d'autres instruments c'est du sculpteur intrigante est juste un peu simple que ça et quand tu apprends à différents morceaux souvent la musique pop jazz je vais apprendre toi même des choses t'imagines des choses est d'ailleurs la construction mentale elle intéressante en soi mais quand même temps tu puisses avoir matériellement directement j'ai travaillé le solo de jimmy page de services aux jeunes comme tout le monde l'a tous les guitaristes l'on fait un moment tout le monde fait ça avec sa guitare acoustique parce que bon bah tu as a sous la main étaient en train de travailler que tu as fait le début avant qui lui est complètement avec un son clair la chance n'a pas envie de manger voilà que tu puisses avoir un oui tiens là il est passé en dix tours en tête juste à 10 puis voilà quoi c'est ça que j'imagine sur le long terme que tout ce que tu imagines bas juste ça soit détenu puisse la voir matériellement la grande tu vois je vois beaucoup plus inspiré ouais je les joue très très 
 bien vu c'est d'autant mieux que tu as répondu au passage à une question dont je me rends compte que j'ai oublié de te poser que je m'en faisais concernant mais j'ai répondu à temps mais c'est vrai que la surface donnant sur la mésentente mal sur l'importance de la communauté danser dans le monde numérique un oui de manière plus générale dans les instruments comme ça je crois pas moi 7 le groupe aux enjeux comme dans le reste quoi selon lui ce que les gens s'ils sont pas différents quand il quand il ment sur facebook avec leurs copains quand ils font la guitare ça c'est même genre d'intervention de génériques seynod sur notre communauté de plateformes wii voilà peut-être ça à partir mais moi je pense fait je pense que c'est quelque chose de très humain ça s'accélère avec technologie numérique mais toulouse on a 2 bal est de plus en plus on est là dedans et même pour une efficacité dont voit l'ennemi je veux dire mais moi en terme d'efficacité technique vous pouvez presque dire parce que parce que dans ta pratique musicale tu as besoin tu as besoin d'un train dans ton apprentissage dont tu as besoin d'échanger un maximum ici quand tu as pris ton nouveau morceau si ce n'est pas tout de suite le jouer à tes potes et ben tu sais pas si tu peux enfin je sais pas c'est ma vision d'être à moi des choses mais en général je pense que quel que soit le truc quel que soit nouvelles connaissances tu as besoin ensuite de la mettre en oeuvre juste pour avancer quoi même pour apprendre le morceau d'après martin a besoin d'avoir présenté on audition ou avec des pâtes à la maison la trouve beau morceau voir comment c'est reçu va échanger avec quelqu'un qui va jouer une partie du truc le fait de le faire en live ça tu vas le récent tir différemment ça donne une nouvelle idée c'est par cette interaction en permanence même en termes d'efficacité mais courtois pour pour avancer pas besoin d'être connecté en permanence termine en a besoin on travaille bien dit de moi aussi mais très vite il faut que ces choses là se mettent en place et les technologies m'a permis de faire ça bonfol tous les youtube heures et tout ça c'est incroyable comme donc et je pense que nous notre réponse à nous par 
 rapport à ça c'est que la qualité sonore vallée rarement rendez-vous dans tout ça en fait elle n'a pas été elle n'a pas pris de l' la vague du numérique en fait la qualité sonore elle l île élèves voilées youtuber on a des super truc tu as des supers mecs qui font décompose super intéressante sur youtube mais il souvent le son est pourri c'est con mais c'est complètement dépendant de ces technologies de la cua donc ou bien pour avoir un bon sont souvent un matériel très sophistiqué mais c'est pas intégré dans le dans ces nouveaux objets là et nous on aimerait que ce qu'on propose ça soit le truc qui permet de sable parce que si tu es que tu as composé ton morceau avec notre guitare et que tu joues avec le youtuber il joue j'ai bien kiffé sa vidéo après si tu charges sa musique que lui il a écrit tous ont un cover à deux je ne sais pas quel morceau que l'on ait à van houtte charge dans la guitare et après tu l'écoutés matata guitare qui jusque le mec a joué donc en termes de qualité sonore c'est incomparable avec ce que tu as et lui la ligue ou en riant s'il a la même guitare chez lui toi parce que le dire celui là la guitare il s'est enregistré avec ça quand tu charges ce qu'il a fait tu les écoutes dans ta guitare on a une super qualité sonore alors peut-être qu'aujourd'hui c'est un peu con record de mettre en oeuvre donc il va falloir qu'on arrive à faire ça vite et des points techniques vous êtes sur le long terme mais je crois beaucoup plus à ça que de rester complètement avec des matériels qui sont indépendants chez dans la communauté je vois ce que je veux dire si tu as une unité dans l'instrument et déjà la base dans les gens qui partagent les choses ça va simplifier énormément simple d'un côté normatif cieux de type de captation sip de réécoute qui types d'enregistrements plus fait et qui après qu'ils partagent et les autres sont dans les mêmes conditions que toi ça je pense que ça peut accélérer ça comme des outils numériques voilà des plateformes sauf que là la plateforme elle passe par un revers spécifique et une sorte de contre arguments à la modularité qui est caractéristique de tous les voir avec mon nico voilà l'état compliquer la vie en ce moment que ça crève kloten que l'on utilise jeudi à 5 13 5 reste c'est un terme de chien qui parle du fait que l'image et et son fusionnent mais pour les raisons du sinaï qui sont a priori de choses indépendante a les seins espère jouer deux ans à fusionner ok vous nous faire une démo bah oui quand même attente est venu enfin des mots saint chrême causer des crises un crime s y sommes synthèse les ch à la place de thc when it in america hop alors je vais te l'a présenté comme on la présente au blizzard tu valoir alors justement le type de façon de de partager sur la 
 technologie donc voilà huit a aussi dit à fait correct du coup ils s'arrogent ans nous avons tu vois je commence toujours par dire ses guitares normal jeu là c'est vraiment à des tabous même en bill doigt sur ces trucs là ensuite je l'allumé et jeudi mais regardez vous pouvez aller d'un côté sportif dont nous les blue tooth on se connecte à yagg je suis connecté ensuite je l'emmenais souvent toujours le même morceau obtenu l'entendre on est à lyon à chaque fois qu'ils aient que je vais chercher donc je vais chercher ensuite morceaux qui ont enregistré là je leur dis à chaque fois les truites dont +10 comme dans son l'ue continue doubler parce que on n'a pas eu facile seule la guitare ça marche très bien en vie pour les pâquis pour l'instant on oriente très quand même guitariste du tarissement on dit ça on dit les backing tracks donc donc ça on leur présente en disant pour te réécouter par exemple ou bien tu as ton prof qui joue justement est appelé à jouer dans ta guitare mais sinon il ya aussi leur back in tracks donc les accompagnements des guitaristes par exemple ça ça fait beaucoup coûté bizarrement je pensais pas c'est ça qui plaisait plus mais oui c'est ça donc donc ça c'est le premier aspect jeu diffuser du son je te cache pas que pour l'instant si on l'a pas orienté complètement enceintes bluetooth c'est parce que ça pas une super bonne qualité ainsi parce que on profite de la réponse en fréquence d'une guitare donc c'est très bien pour que ta guitare mais sinon 20 ans tant les résonances d'égalité quand tu mets un peu n'importe où et n'importe quoi ici donc là on a travaillé district a croisé qui travaillent sur des halles ou de correction voilà donc la maintenant dans un modèle ans on a ça ça sonne plus comme une enceinte moi j'aime ça de jeu mais d'un autre côté sur l'utilisation ça peut être intéressant aussi donc ça c'est premier aspect deuxième aspect looper quand on entend développer nous pour l'instant c'est juste un en enregistrant rybak et ensuite les effets donc là la petite rivière là où tu peux changer donc en terme d'interface donc la vie si on fait juste plus est moindre dans nos banques qui sont prédéfinis quand on est finie en bluetooth en amont en fait on envoie à l'époque où on décide aujourd'hui depuis un ordi là on développe là mais ça c'est un volume et ça c'est un paramètre de l'effet donc dans la réserve j'aime les tickets je connaîtrai ça c'est un paysage change hélas la vitesse suffisante il traverse un peu de son sang c'est intéressant le faiseur sympathique avec pékin et marche à tous les coups avec les effets c'est quand les enlève je pars à l'acoustique on souhaite augmenter là tu vois quand après qui dit merde ça c'est ce qu'elle touche avec ça je fais ce que tu sais utiliser l'e10 aujourd'hui ça va toujours très vite si en juste l'envié il est remarqué oui l'afrique il a pensé comme 
 un truc et qui fait là ajouté d'arrêt le corps si on peut dire 1 c'est décidé encore une contrôle ibérique de doré normal et j'ai aussi ce que je disais ce qui nous a valu mettre une disto et donc la lys ouvrons ce avait plein de feedback tu vois c'est donc c'est là qu'on s'est rendu compte remettre les cheveux examen rue le système concert l'histoire s'impose finalement on va en ville à l'histoire et et donc voilà que ça chiffre de demain après l'app temps ça tirait un peu toujours les guidances une m mais entre choisir quelqu'un on va zy total de choristes il aimait bien d'être là donc voila tu oses tu fais plus et moins ici et à volume tu peux laisser toute façon je pense ça tu vois chorus et reverb yama ils ont fait france abou seada connaît yama trans acoustique ils ont fait une guitare et m avec un actionneur qui signeront l'habitat ça n'est pas tous nos trucs mais tu peux commencer à faire des petits effets donc t'as pas une fusion comme ça parce que s'arrogent du trou mais tu es tu fonda de la guitare mais tu pourrais comparer regard des cieux j'ai même passé le truc parce que ça c'est aussi un gros sujet en tant qu'entreprise la coordination avec les gongs les compétiteurs les concurrents de fait que j'ai créé la boîte c'est pas cela va commencer à se mettre sur ce marché là ça c'est un moulin d'ascq dont on appelle tous parlé que du côté des moteurs quand même pour aller au bout le type de produit quitte c'est vraiment par rapport à ça aussi l'environnement les concurrents le technologies qui sont utilisés aujourd'hui nous ce qu'on vient apporter il ya un côté très très peu nantis rubin ouais ça c'est très important dans les jungles invite à lever je suis tombé en fait j'ai croisé plus le connais bien je connais sont dans la boîte mais mieux etc lab qui fait là c'est moche oui un super ouais j'ai compris mais je les ai jamais eus mais ouais bah ouais on est sur un truc de quatre chefs waits attaque d'accord sur la longue pour une conférence qu'il était là bas pour présenter une mandoline macally parce qu'il a un projet de recherche ont du talent oui différents instruments traditionnels italiens qui est le nom de l'hymne sorte de cornemuse dont j'ai oublié le nom et je sais pas encore ok a battu voient eux on exactement je pense dans la même vision de cet avenir j'ai connecté guitare en même temps après il a vu la scène suite avec les hommes et essai avec les vidéos et je vais lui la vague l'été pour présenter sa mandoline ok donc j'ai vu qui reprend un peu d'hésiter ils sont un peu plus sur le geste moi j'avais lu ils ont la journée de jérusalem est où les yeux c'est un instrument acoustique un talus en fait commencé c'est ça n'a pas de sens du 6ème juste un manche et à la caisse actions mais passera la casse est découplé quoi le mans millions mais c'est la même après après donc est donc bon c'est vrai qu'à maquiller comme un maquereau ouais gros bonnets ouais je les cherche un peu et auditionné vachement là dessus en piano ils ont fait un truc vachement bien keanu 37b est parce que la transat et klein est aussi par vous même prix que nous assimile 2009 et puis tu as juste envie une petite phrase de la volaille urgence les combats ont commencé 500USD est là mais ça va être un peu plus cher pour la commande c'est sûr énigme pour les premiers mon rêve injuste dans les 750 c'est l'un des pendus des volumes qu'on arrive à faire pour nous c'est très con traire de yamaha qui font et où les traditionnelles quoi moi même qui ne s'est pas grandie terry sera jugé sur mes pattes qui sont venus tarifs pour les heures qui nous achètent une autre scène professionnelle oh en fait ce qu'elle particulier celle de latence la latence au mur elle est hyper fait comparer tout ce qui existe en musique parce que comme on en voit nominé traiter rapidement c'est c'est plus des microsecondes plutôt que des millisecondes a donc ça ça fait une carte électronique qui pas habituelles en bleu pas juste faire du dsp pour nous prouver quoi et cd tout le projet de recherche a dit agréable gallo sa propre carte ouais mélange des choses en mélange des composants traditionnels des choses plus industrielles et des choses plus au nord on est ensemble et c'est ça qui est un peu fait que c'est cher on a vraiment concevoir un truc spécial donc on espère que à terme ça a bien des choses qu standard tranquillement sur leur zone rouge sa jauge énorme bévue ouest qu un peu sur ce cas là vous apporter c'est vraiment ça ce côté non là encore 50 fois plus qu'il ya dix ans ce que nous nous on a vraiment un round 3 les matic jouer vraiment super vite lui ont dit web interactive et oui ils sont multi lots multi connu un bon match car cet acte est là bas c'est intéressant ce qui fait ne pas croiser en drôme et rencontre joliot et

