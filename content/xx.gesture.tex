
\section{Introduction}
	\subsection{L'étude du geste en musique}
	\subsection{Définition du geste}
		\subsubsection{Approche sémiotique}
		\subsubsection{Approche sémiotique}






La musique ne doit [...]pas être considérée comme un ensemble d’objets (sonore ou graphiques) mais comme un faisceau de conduites, qui consistent à la faire ou à l’entendre. (Delalande –Analyser la musique, pourquoi, comment?;p.158) 
 
L’«écoute taxinomique» se caractérise par exemple par une tendance de l’auditeur à former et à lister mentalement des unités d’assez grande taille, sansles qualifier plus qu’il n’est nécessaire pour les distinguer et les mémoriser (p.46-47), tandis que l’«écoute empathique» est attentive à des sensations internes, des ressentis présents «sans chercher à  établir de relations avec les moments antérieurs» ou à «partitionner» l’écoute(p.56-57). La «figurativisation», troisième conduite d’écoute extraite par Delalande, «anime» certains des sons présentés, en opposant le vivant et son contexte(p.74-75).

parce que les attributs qu'on lui confère dépendent en partie de cette qualité sémiotique, qui reste soumise à une interprétation subjective et contextuelle dans un système de valeurs.

La notion de geste sert à désigner un phénomène en apparence simple et pourtant d'une extrêmemt complexité. Une définition courante 

Ses définitions sont multiples selon la perspective que l'on adopte pour l'étudier

pouvant être envisagé sous de multiples aspects: une approche phénoménologique tentera une analyse descriptive du geste (spatiale et temporelle), une approche fonctionnelle apportera une analyse de l'acton effective du geste dans un contexte spécifique (sémiotique, ergodique)

Au niveau physique, le geste implique à la fois des processus conscientisés, ``intentionnels'' et des mouvements répondant à des unités motrices réflexes potentiellement appris par la pratique instrumentale. Ils peuvent être mûs par un but précis (e.g. attraper un objet), par l'entretien d'une liaison tactile continue, être entrainés par leur propre course (sur laquelle des forces telles que la gravité influe).


Le geste n'est pas la cause de la musique : il fait partie de la musique, de l'expérience musicale\footnote{Cette remarque qu'on pourrait appliquer aux musiques acoustiques, se vérifie particulièrement pour les musiques numériques dans lesquelles la relation de causalité entre geste et musique n'est pas nécessaire à leur existence réciproque.}. ``Scientia bene movendi'', l'art de bien se mouvoir disait Saint Augustin (cité dans \cite{delalande_geste_1988}). De ce point de vue, le geste peut être envisagé et analysé de manière étonnamment similaire à la manière dont on analyse la musique dans son aspect sonore. Ainsi, il y a une \textit{poïétique} et une \textit{esthésique} du geste, selon les termes de Molino/Nattiez, \textit{inductive} ou \textit{externe} selon que l'on part du phénomène à étudier vers le récepteur ou l'auteur du phénomène. 

Les DMIs qui se présentent à nous comme des IHM, du fait qu'elles aient été conçue historiquement dans ce contexte technico-industriel, ont souvent été envisagée comme des IHM, avec la notion de tâche à accomplir qui est associée à l'idée d'utilisation de la machine.

Le geste musical n'est pas sémiotique mais pré-sémiotique, créateur de sensation plutôt, que de sens.


Le modèle sémiotique de McNeil, Kendan ne convient pas car le geste musical n'est pas a piori signifiant. Il créé non pas une sens mais une sensation.

\section{Introduction}
	\subsection{L'étude du geste en musique}
		ok
	\subsection{Problèmes de définition}
		\subsubsection{Le geste comme mouvement intentionnel}
			définition
			différent niveaux de conscience
			controle ou non controle
		\subsubsection{Le geste sémiotique}
			McNeil
			Mazzola pré-sémiotique
		\subsubsection{Le geste ergodique}`


Le geste de communication, le geste de manipulation


\section{Aspects fonctionnels du geste}
	\subsection{Le geste sémiotique (signe)}
	\subsection{Le geste ergodique (action)}
	\subsection{Contrôle ou non-contrôle}
	\subsection{Le geste épistémique (perception)}
\section{Aspects qualitatifs du geste}
	\subsection{Phénoménologique}
	\subsection{Emotionnel}


\section{Gestes visibles, gestes audibles}
Différence (surtout dans les DMI) entre le geste vu et la trace (éventuelle) du geste dans le résultat entendu

\section{Geste réalisé, geste perçu}
intention ou non


\section{Le geste dans les HCI}
	le geste n'existe pas en dehors de leur captation  

\section{Le geste instrumental hérité de l'instrument classique}
	continuum énergétique de Cadoz
	gestes ancillaires et de polarisation de Wanderley
	godoy : sound tracing, gestes d'imitation de gestes produisant du son

\section{Aspects qualitatifs du geste}
	\subsection{Phénoménologique}
	\subsection{Emotionnel}

Pelachaud 

\section{Le geste pur : la danse}
Expeertise du geste, mais pas nécessairement dans le but d'un résultat sonore intentionnel, même si le son peut "coller" de manière forte au mouvement


\section{Le geste programmé}


\section{Le geste magique/subversif}

\cite{cadoz_synthese_1981}

\cite{gibet_codage_1987}

\cite{cadoz_instrumental_1988}

\cite{delalande_geste_1988}

UST : \cite{delalande_les_1996}


\subsection{Contrôle ou non-contrôle}

\noindent La plupart de la littérature sur les \glspl{DMI} met l'accent sur la notion de contrôle par le geste (et les interfaces y sont souvent nommées ``contrôleurs gestuels''). La performance musicale n'est toutefois pas exclusivement gouvernée par une relation de contrôle absolu du son. De nombreux exemples dans des styles musicaux très divers viennent démontrer l'absence de contrôle --~voire sa recherche~-- sur certains aspects musicaux : les \textit{couacs} dans le jeu de John Coltrane, le ``laisser-jouer'' de la machine dans la musique noise \footnote{Voir en particulier les analyse de Paul Hegarty et Sarah Benhaim sur le non-contrôle dans la musique noise \cite{hegarty_noise_2007, benhaim_aux_2018}}, les partitions impossibles à interpréter de Brian Ferneyhough, etc. La performance musicale est le théâtre d'un corps mis dans l'instabilité propre au medium sonore et l'instrumentiste peut s'y trouver dans différentes postures face à ce qui jaillit de son instrument (parfois à son insu), pilote à la maîtrise totale, chimiste opérant des mélanges incertains de fluides sonores, chirurgien-ne disséquant le son, funambule en équilibre sur le fil de l'audible, dompteur de cirque face à un instrument sauvage, explorateur-rice dans une jungle inconnue, cascadeur-se se jettant dans le vide, chamane faisant naitre la transe, ... autant de postures différents qui partagent toute une attention extrême à leur environnement


Lyotard 1984 - post-modernisme ``Signalons que le postmodernisme ne fait plus correspondre la valeur esthétique à la beauté. Ce que Lyotard suggère au contraire, c’est une esthétique du sublime qui « présente l’imprésentable ».'' (in Fdili Alaoui)



\subsection{Qualité du geste}
Pelachaud \cite{pelachaud_studies_2009}, d'après Gallagher \cite{gallaher_individual_1992}
\vspace{-1em}
\begin{itemize}[noitemsep]
	\item \textbf{l'extension spatiale} quantity of space taken by a body part (how extended are the arms, how raised are the brows) (see Figure 1). This dimension is related to the dimen- sion ‘expansiveness/spatial extension’ defined by Wallbott and to the ‘dimension expansiveness’ by Gallaher.
	\item \textbf{l'extension temporelle} velocity of execution of a movement (how fast or how slow an arm moves, a head turns). It is related to the ‘animation’ factor of Gallaher.
	\item \textbf{la fluidité} level of continuity of successive movements(jerkyvssmoothmovements). It is similar to the ‘coordination’ dimension defined by Gallaher.
	\item \textbf{la puissance} movementdynamism(weakvsstrong). It is related to the degree of acceleration of body parts. It corresponds to the dimension ‘movement dynamics/energy/power’ defined by Wallbott.
	\item \textbf{l'activation globale} overall quantity of movement on a given channel for the whole animation: a lot of head movement vs no head movement; many hand gestures vs none, etc. This dimension embodies similar information as the ‘expressiveness’ dimension defined by Gallaher.
	\item \textbf{la répétition} repetition of the stroke of a movement. We have added this dimension from the studies presented above to encompass the notion introduced by Isabella Poggi of stroke expansion [44]. For example the stroke of a gesture can be rhyth- mically repeated to mark an emphasis.
\end{itemize}



à mettre dans ephemeral instruments ?
De même que l'histoire a connu une importance prépondérante des musiques traditionnelles\footnote{Les musiques trad existent toujours, et d'une certaine manière, un standard de jazz est une musique traditionnelle}, à une époque où l'absence d'écriture et d'enregistrement rendait cette tradition nécessaire à la répétition, la période allant du Moyen-Age à l'époque contemporaine a été marqué par la prépondérance d'instruments traditionnels, à défaut de pouvoir agencer ces instruments de manière plus souple, comme l'a permis l'écriture pour la composition musicale. La possibilité de ``noter'' ces instruments sous forme de grammes, ainsi que les gestes ne rend plus nécessaire l'existence de tradition instrumentale, ou plutôt, rend plus que jamais possible l'invention instrumentale.

De même que l'on peut travailler une partition et devenir expert dans son interprétation, on peut travailler un instrument pour en devenir expert et travailler les différentes compositions pour cet instrument. On peut aujourd'hui travailler le geste (et l'écoute!) et en devenir expert pour jouer les différents instruments qui s'offrent à ces gestes.