La musique ne doit [...]pas être considérée comme un ensemble d’objets (sonore ou graphiques) mais comme un faisceau de conduites, qui consistent à la faire ou à l’entendre. (Delalande –Analyser la musique, pourquoi, comment?;p.158) 
 
L’«écoute taxinomique» se caractérise par exemple par une tendance de l’auditeur à former et à lister mentalement des unités d’assez grande taille, sansles qualifier plus qu’il n’est nécessaire pour les distinguer et les mémoriser (p.46-47), tandis que l’«écoute empathique» est attentive à des sensations internes, des ressentisprésents «sans chercher à  établir de relations avec les moments antérieurs» ou à «partitionner» l’écoute(p.56-57). La «figurativisation», troisième conduite d’écoute extraite par Delalande, «anime» certains des sons présentés, en opposant le vivant et son contexte(p.74-75).



Le geste dans les HCI

Le geste instrumental hérité de l'instrument classique
	continuum énergétique de Cadoz
	gestes ancillaires et de polarisation de Wanderley
	godoy : sound tracing, gestes d'imitation de gestes produisant du son


Le geste de la danse
	Expeertise du geste, mais pas nécessairement dans le but d'un résultat sonore intentionnel, même si le son peut "coller" de manière forte au mouvement


Le geste enregistré


Le geste magicien

\cite{cadoz_synthese_1981}

\cite{gibet_codage_1987}

\cite{cadoz_instrumental_1988}

\cite{delalande_geste_1988}

UST : \cite{delalande_les_1996}