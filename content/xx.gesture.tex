
\section{Introduction}
	\subsection{L'étude du geste en musique}
	\subsection{Définition du geste}
		\subsubsection{Approche sémiotique}
		\subsubsection{Approche sémiotique}



Proprioception et schema corporels (cf. Miranda unconventional computing)



La musique ne doit [...]pas être considérée comme un ensemble d’objets (sonore ou graphiques) mais comme un faisceau de conduites, qui consistent à la faire ou à l’entendre. (Delalande –Analyser la musique, pourquoi, comment?;p.158) 
 
L’«écoute taxinomique» se caractérise par exemple par une tendance de l’auditeur à former et à lister mentalement des unités d’assez grande taille, sansles qualifier plus qu’il n’est nécessaire pour les distinguer et les mémoriser (p.46-47), tandis que l’«écoute empathique» est attentive à des sensations internes, des ressentis présents «sans chercher à  établir de relations avec les moments antérieurs» ou à «partitionner» l’écoute(p.56-57). La «figurativisation», troisième conduite d’écoute extraite par Delalande, «anime» certains des sons présentés, en opposant le vivant et son contexte(p.74-75).

parce que les attributs qu'on lui confère dépendent en partie de cette qualité sémiotique, qui reste soumise à une interprétation subjective et contextuelle dans un système de valeurs.

La notion de geste sert à désigner un phénomène en apparence simple et pourtant d'une extrêmemt complexité. Une définition courante 

Ses définitions sont multiples selon la perspective que l'on adopte pour l'étudier

pouvant être envisagé sous de multiples aspects: une approche phénoménologique tentera une analyse descriptive du geste (spatiale et temporelle), une approche fonctionnelle apportera une analyse de l'acton effective du geste dans un contexte spécifique (sémiotique, ergodique)

Au niveau physique, le geste implique à la fois des processus conscientisés, ``intentionnels'' et des mouvements répondant à des unités motrices réflexes potentiellement appris par la pratique instrumentale. Ils peuvent être mûs par un but précis (e.g. attraper un objet), par l'entretien d'une liaison tactile continue, être entraînés par leur propre course (sur laquelle des forces telles que la gravité influe).


Le geste n'est pas la cause de la musique : il fait partie de la musique, de l'expérience musicale\footnote{Cette remarque qu'on pourrait appliquer aux musiques acoustiques, se vérifie particulièrement pour les musiques numériques dans lesquelles la relation de causalité entre geste et musique n'est pas nécessaire à leur existence réciproque.}. ``Scientia bene movendi'', l'art de bien se mouvoir disait Saint Augustin (cité dans \cite{delalande_geste_1988}). De ce point de vue, le geste peut être envisagé et analysé de manière étonnamment similaire à la manière dont on analyse la musique dans son aspect sonore. Ainsi, il y a une \textit{poïétique} et une \textit{esthésique} du geste, selon les termes de Molino/Nattiez, \textit{inductive} ou \textit{externe} selon que l'on part du phénomène à étudier vers le récepteur ou l'auteur du phénomène. 

Les DMIs qui se présentent à nous comme des IHM, du fait qu'elles aient été conçue historiquement dans ce contexte technico-industriel, ont souvent été envisagée comme des IHM, avec la notion de tâche à accomplir qui est associée à l'idée d'utilisation de la machine.

Le geste musical n'est pas sémiotique mais pré-sémiotique, créateur de sensation plutôt, que de sens.


Le modèle sémiotique de McNeil, Kendan ne convient pas car le geste musical n'est pas a piori signifiant. Il créé non pas une sens mais une sensation, et la sensation est un espace beaucoup plus vaste et plus ouvert (car continu) que l'espace sémiotique.

\section{Introduction}
	\subsection{L'étude du geste en musique}
		ok
	\subsection{ }
		\subsubsection{Le geste comme mouvement intentionnel}
			définition
			différent niveaux de conscience
			controle ou non controle
		\subsubsection{Le geste sémiotique}
			McNeil
			Mazzola pré-sémiotique
		\subsubsection{Le geste ergodique}`


Le geste de communication, le geste de manipulation


\section{Aspects fonctionnels du geste}
	\subsection{Le geste sémiotique (signe)}
	\subsection{Le geste ergodique (action)}
	\subsection{Contrôle ou non-contrôle}
	\subsection{Le geste épistémique (perception)}
\section{Aspects qualitatifs du geste}
	\subsection{Phénoménologique}
	\subsection{Emotionnel}


\section{Gestes visibles, gestes audibles}
Différence (surtout dans les DMI) entre le geste vu et la trace (éventuelle) du geste dans le résultat entendu

\section{Geste réalisé, geste perçu}
intention ou non


\section{Le geste dans les HCI}
	le geste n'existe pas en dehors de leur captation  

\section{Le geste instrumental hérité de l'instrument classique}
	continuum énergétique de Cadoz
	gestes ancillaires et de polarisation de Wanderley
	godoy : sound tracing, gestes d'imitation de gestes produisant du son

\section{Aspects qualitatifs du geste}
	\subsection{Phénoménologique}
	\subsection{Emotionnel}

Pelachaud 

\section{Le geste pur : la danse}
Expeertise du geste, mais pas nécessairement dans le but d'un résultat sonore intentionnel, même si le son peut "coller" de manière forte au mouvement


\section{Le geste programmé}


\section{Le geste magique/subversif}

\cite{cadoz_synthese_1981}

\cite{gibet_codage_1987}

\cite{cadoz_instrumental_1988}

\cite{delalande_geste_1988}

UST : \cite{delalande_les_1996}





Lyotard 1984 - post-modernisme ``Signalons que le postmodernisme ne fait plus correspondre la valeur esthétique à la beauté. Ce que Lyotard suggère au contraire, c’est une esthétique du sublime qui « présente l’imprésentable ».'' (in Fdili Alaoui)

Guerino Mazzola : \iquote{Gestures are dialogical, live in presence, are circular, elastic, are presemiotic, and are as such already differentiated (being gestural can be ramified into different types).} \cite{mazzola_topos_2018}, p. 59 (852)

Le geste + intention ? => le geste ne passe pas nécessairement pas un dessein connu d'avance, il se produit aussi en réaction à la vivacité de la musique produite dans une circularité qui ne laisse pas de place à la mentalisation de l'instant (flow).



\subsection{Qualité du geste}
Pelachaud \cite{pelachaud_studies_2009}, d'après Gallagher \cite{gallaher_individual_1992}
\vspace{-1em}
\begin{itemize}[noitemsep]
	\item \textbf{l'extension spatiale} quantity of space taken by a body part (how extended are the arms, how raised are the brows) (see Figure 1). This dimension is related to the dimen- sion ‘expansiveness/spatial extension’ defined by Wallbott and to the ‘dimension expansiveness’ by Gallaher.
	\item \textbf{l'extension temporelle} velocity of execution of a movement (how fast or how slow an arm moves, a head turns). It is related to the ‘animation’ factor of Gallaher.
	\item \textbf{la fluidité} level of continuity of successive movements(jerkyvssmoothmovements). It is similar to the ‘coordination’ dimension defined by Gallaher.
	\item \textbf{la puissance} movementdynamism(weakvsstrong). It is related to the degree of acceleration of body parts. It corresponds to the dimension ‘movement dynamics/energy/power’ defined by Wallbott.
	\item \textbf{l'activation globale} overall quantity of movement on a given channel for the whole animation: a lot of head movement vs no head movement; many hand gestures vs none, etc. This dimension embodies similar information as the ‘expressiveness’ dimension defined by Gallaher.
	\item \textbf{la répétition} repetition of the stroke of a movement. We have added this dimension from the studies presented above to encompass the notion introduced by Isabella Poggi of stroke expansion [44]. For example the stroke of a gesture can be rhyth- mically repeated to mark an emphasis.
\end{itemize}




%-------------------------------------------
à mettre dans ephemeral instruments ?

De même que l'histoire a connu une importance prépondérante des musiques traditionnelles\footnote{Les musiques trad existent toujours, et d'une certaine manière, un standard de jazz est une musique traditionnelle}, à une époque où l'absence d'écriture et d'enregistrement rendait cette tradition nécessaire à la répétition, la période allant du Moyen-Age à l'époque contemporaine a été marqué par la prépondérance d'instruments traditionnels, à défaut de pouvoir agencer ces instruments de manière plus souple, comme l'a permis l'écriture pour la composition musicale. La possibilité de ``noter'' ces instruments sous forme de grammes, ainsi que les gestes ne rend plus nécessaire l'existence de tradition instrumentale, ou plutôt, rend plus que jamais possible l'invention instrumentale.

De même que l'on peut travailler une partition et devenir expert dans son interprétation, on peut travailler un instrument pour en devenir expert et travailler les différentes compositions pour cet instrument. On peut aujourd'hui travailler le geste (et l'écoute!) et en devenir expert pour jouer les différents instruments qui s'offrent à ces gestes.

e.g. Jordan Rudess, ``claviériste expert'' jouant du soundplane, du linnstrument



 Cette part incalculable dont parle Stiegler fait écho à la propension du geste à dépasser ses propres limites, non pas seulement sur le plan physique mais également sur le plan cognitif. (cf. example des musiques de Steve Reich, d'abord créées avec l'aide de machine, puis de manière humaine => piano phase solo)
%-------------------------------------------



 
\textbf{Extra notes}\\

Mouvement des machines (cf. Patrick Saint Denis machines mobiles)\\
Mouvement dans l'image (cf. Serge de Laubier musique visuelle => festival, \iquote{amplifier l'écoute par l'image})\\
\iquote{Musique, Incarnation, physique, informatique, voilà quatre fusions de l'alliage dur-doux. (...) Nous nous étonnons devant ces quatre miracles, ces quatre fusions, parce que nous manquons d'une philosophie du mélange} Michel Serres, Musique


Nous avons vu dans les sections précédentes différents aspects concernant la fonctionnalité du geste, dans la perspective du contrôle de ce qui est produit par l'instrument. L'autre extrêmité de la prise en compte du geste est la perspective de sa perception par le spectateur. 

A l'autre extremité, la réception de la performance par le spectacteur/auditeur engage une projection active pour reconstruire du sens : déterminer la cause des sons, anticiper les trajectoires

Celle-ci est prise dans une perception globale impliquant le geste du musicien, la réaction de l'instrument ainsi que les projection active de causalité de la part du spectateur sur la scène à laquelle il assiste. Cette projection se construit sur la base de notre expérience et constitue l'objet central de la recherche en psychologie de la percpetion ou psychologie cognitive.

L'étude de la perception des \glspl{DMI} est arrivée plus tardivement que l'étude de leur contrôle mais a pris une importance croissante ces vingts dernières années. (Auslander, Croft, Bin, Fyans, Benford)


La perception du geste implique plusieurs composantes 

La couture du geste

Ruptre : Kurtag's forte/piano, vivadi in hidustani music

le chercheur créé des catégories dans l'uniformité et des liens entre les choses disparates.

couture entre les différents modes tonaux

uniformité de timbre de l'instrument

John Croft ``These on liveness''  : procedural liveness vs aesthetic liveness (Croft, J. (2007). Theses on liveness. Organised Sound, 12(01), 59. doi:10.1017/
S1355771807001604)

\iquote{Le geste doit être défini comme un mouvement intentionnel plus ou moins complexe, orienté vers un but déterminé qui lui donne un sens individuel, social ou historique.} \cite{imberty_mouvement_2013}


\indent La notion de geste est également utilisée pour décrire des formes indirectement liées au mouvement physique, telles que le ``geste de composition'', en transposant par analogie le mouvement et l'intention le caractérisant. On voit donc que la question de l'intention soulève des questions esthétiques voire politiques : l'inscription de la performance musicale dans la vie sociale et culturelle pose la question de la place du geste dans un système de valeurs spécifique. En particulier, l'époque postmoderne est caractérisée par un art qui attribue davantage d'importance à \textit{l'intérêt} du geste pris dans un contexte \textit{expérimental}, qu'à sa \textit{beauté}, prise dans un contexte \textit{normatif} classique.


subversion lié à au fonctionnement de l'instrument (mapping inversé,etc)
	citation Jodlowsky, absence de cohérence énergétique

subversion lié au geste performé vs geste capté vs geste produit
	(mapping inversé,etc)


subversion lié au métamorphisme des processus instrumentaux
	enchainement de modèle et suture gestuelle
	renvoyer au chapitre 5


\indent Au-delà du timbre et des modes de jeu que le luthier ``programme'' dans l'élaboration d'un instrument acoustique, les \glspl{DMI} permettent d'élaborer une programmation qui à la particularité de pouvoir se déployer dans le temps avec une certaine autonomie, d'y dessiner un trajet\footnote{On peut mettre en regard la notion de trajet, qui suit un chemin, avec celle de plan   \iquote{la notion de référence s'est développée avec la masse grandissante des faits enregistrés, mais les écrits sont chacun une suite compacte, rythmée par des sigles et des notes marginales, dans laquelle le lecteur s'oriente à la manière du chasseur primitif, le long d'un trajet plutôt que sur un plan.} dans \cite{leroi-gourhan_geste_1964}, p 69}, voire d'y construire une narration. 



\noindent Si on ``musique'' dans toutes ces situations, il est cependant important de noter ici que ces différents contextes ont des règles propres, qu'il existe une culture de la pédagogie, comme une culture de la performance (p. ex. ).


Plus fondamentalement, l'usage métonymique de l'adjectif ``expressif'' pour qualifier des interfaces (qui ``permettent l'expressivité''), au lieu de qualifier le jeu lui-même, semble avoir glissé de sa signification orginelle. Il désignait la possibilité de contrôler l'ordinateur via des interfaces de type analogiques plutôt que symboliques et permettant plusieurs actions en parallèle

\footnote{Les deux premières occurences de cette expression dans le contexte de l'\gls{IHM} semble dater de 1991 dans un article de Cort Lippe ou il évoque un contrôle de l'ordinateur via un signal audio \cite{lippe_real-time_1991}, et dans un autre de Kurtenbach et Buxton où la notion d'interface expressive est associée à celle d'\textit{interface de manipulation directe} \cite{kurtenbach_issues_1991}.}


Ces deux catégories de geste d'action et de gestes perçu sont emprunt de la théorie de l'information proposée par Shannon \cite{shannon_mathematical_1948} qui envisage la communication comme un système émetteur-message-récepteur unidirectionnel. 
L'inconvenient d'envisager le geste comme simple émetteur d'un signal (qu'il soit travail ou signe) est qu'il empêche de considérer le geste dans la rétroaction dans laquelle il s'inscrit avec l'instrument. En particulier dans la performance musicale, la rétroaction multimodale (par l'ouïe, la vue, le toucher) entre les geste d'un instrumentiste et son instrument, le son et le public est essentielle.

Il était tentant de l'appliquer à la situation musicale et de voir dans le phénomène sonore un message circulant du musicien vers l'auditeur, et les analyses du geste instrumental s'appuyant sur cette solide base théorique ont permis de développer un certain nombre de concept encore utile pour l'analyse du geste musical.

Cependant, la situation de performance musicale est loin d'être aussi fonctionnelle que celle qui consiste à vouloir transmettre un flux de données. Notamment, la théorie de l'information s'applique à des machines qui ignore totalement le contenu sémantique de ces données et les aspects cognitifs ou les références culturelles des émetteurs et récepteurs.


Certains auteurs ont critiqué cette approche \cite{fyans_where_2009} en remarquant que l'idée selon laquelle une connaissance et une compréhension préalable de l'instrument et de l'idiome était nécessaire pour évaluer une performance musicale, ne pouvait être généralisée aux \glspl{DMI}, à cause de l'émergence rapide de technologies, d'instruments et de pratiques de performance dans ce domaine. 

Cependant, cette critique reste ancrée sur une approche qui considère que le spectateur évalue le \textit{succès} d'une performance selon sa compréhension des intentions de l'instrumentiste.


Le jeu musical joue en partie sur l’attente de l’audience (récompensée ou non) sur la base de règles d'harmonies, d’idiomes (e.g. cadences, résolutions, cycles rythmiques), de citations (e.g. via le sampling), etc..
Affordance des instruments ne peut être réduite aux objectifs d’affordance des IHM.


Jean Haury Meta-Piano
Applebaum Aphasia

gestes incongruent (Musical gestures, Godoy, p.48)


Charlotte Moorman and Name June Paik performing John Cage’s 26’1.1499” for a String Player (Human Cello section

%%%%%%%%%%%%%%%%%%%%%%%%%%%%


Si le geste est un mouvement accompagné d'intention, il faut prendre en compte cette partie intentionelle et essayer de la qualifier, dans la perspective des conséquences qu'elle porte au design des \glspl{DMI}.



\indent Plusieurs exemples de subversion de ces inférences ont été exposés dans la liste précédente. Nous en donnons quelques exemples illustrant cette notion de soudure perceptive.


Il n'est pas nécessaire de se baser sur des lois ancrées dans l'expérience collective, mais également possible de déjouer des règles qui ont été établies en cours de jeu, en modifiant dynamiquement le fonctionnement de l'instrument. Il devient nécessaire dans ce cas de faire en sorte que la ``couture'' entre les différents réglages qui se succèdent soit invisible, 


Elle peut aussi bien s'établir dans la durée, par un changement continu des paramètres (e.g. interpolation vers de nouveaux réglages), que dans une transition abrupte, en transposant dans le nouvel espace de jeu un certain nombre de paramètres de l'ancien espace de jeu afin qu'ils correspondent\footnote{Un exemple concret est donné dans la vidéo présentant les Modèles Intermédiaires Dynamiques (cf. chapitre \ref{ch:algorithms}) : la transition entre les différents modèles présentés se fait en conservant une cohérence visuelle de la représentation (e.g. la position, l'orientation des formes) --~ alors que tout le comportement à changé \url{https://vimeo.com/25740547}.)}. 

Par exemple, dans la performance audiovisuelle \textit{FIB\_R}, une délégation progressive de la production de notes à l'ordinateur, d'abord jouées explicitement en termes de déclenchement et de choix de hauteur (installation d'une  ), puis déclenchée avec une prise en charge de la hauteur par l'ordinateur (selon une échelle pré-établie), puis déclenchée automatiquement par l'activation d'un flux de note plutôt que d'une note isolée par la même touche qui servait préalablement à déclencher une note seule.

\subsubsection{Gestes feints}

\noindent Tout d'abord, se présente le cas où les gestes ne sont pas cachés, mais où subsiste un décalage entre les gestes perçus par le public et les gestes effectivement captés par l'instrument. Ce cas de figure peut se présenter sur des instruments acoustiques (et \textit{a fortiori} sur des \glspl{DMI} expressifs). Par exemple, dans certaines pièces pour piano de György Kurtág, le pianiste doit effectuer un geste d'approche énergique vers le piano, laissant supposer qu'il va jouer \textit{forte}, mais le retenir au dernier moment pour jouer \textit{pianissimo}\footnote{Ce mode de jeu a été observé en concert dans une interprétation des \textit{Játékok} de Kurtág. Après vérification, l'indication n'est pas explicite dans la partition, mais fait partie des instructions données par György Kurtág en vue de l'interprétation de certaines parties, d'après son fils György Kurtág Jr. On y trouve par contre l'indication inverse, consistant à jouer \textit{fortissimo} \iquote{en touchant les touches sans faire de son}}. L'oreille se prépare à entendre un son intense et s'en protège par un réflexe stapédien\footnote{Le réflexe stapédien est la contraction involontaire des deux muscles de l'oreille moyenne, le muscle stapédien et le muscle du marteau. En rendant plus rigide la chaîne des osselets, il atténue le niveau des sons transmis à l'oreille interne.}, renforçant ainsi l'intensité du \textit{pianissimo}.\\
\indent Au-delà de cet exemple particulier, il existe de nombreux moyens pour l'instrumentiste de ``tromper'' l'auditeur/spectateur pour lui faire entendre ce qu'il/elle veut lui faire entendre, plutôt que ce qui est acoustiquement présent, des doigtés alternatifs sur les vents sacrifiant la justesse pour obtenir la fluidité d'enchainement, au gestes accompagnateurs qui en étendent la perception dans l'imaginaire\footnote{Nattiez, Music and discourse p44 : certain pianistes ont l'impression de donner de la ``profondeur'' à un accord en permettant aux doigts de glisser vers l'intérieur du piano après avoir enfoncé les touches. (...) inversement, Braendel : le son de notes soutenues sur le piano peut être modifié... à l'aide de certains mouvements qui rendent la \textit{conception du cantabile} du pianiste visible pour le public. (Delalande, ``vers une psycho-musicologie'' in L'enfant du sonore au musical):166, (Brendel, A. 1976. Musical Thoughts and Afterthoughts. Princeton: Princeton University Press. p.31)}.

\subsubsection{Gestes du playback}

\noindent Nous avons également utilisé des gestes feints dans la performance \textit{FIB\_R} réalisée avec Gladys Brégeon, pour jouer une partie dont la ryhmique très précise et machinique, dont la rigueur métronomique requise n'était atteignable qu'en la confiant à l'ordinateur. Nous effectuons des gestes en rythme avec cette séquence autonome, ce qui laisse au spectateur l'impression que nous la jouons nous-même et contribue au l'esthétique générale de cette performance dans laquelle le rôle occupé par la machine et par les performeurs demeure ambigüe.

\subsubsection{Capteurs dissimulés}

\noindent Les auteurs mentionnent deux exemples utilisant des contrôleurs invisibles pour le public (des pédales de type \textit{foot-switch} supposées peu visibles d'une part, et l'utilisation de capteurs dissimulés dans le corps d'un instrument traditionnel, dont l'action est masquée par les gestes habituels du musicien.)

\subsubsection{Mapping non-conventionnel}

\noindent Dans ce cas de figure, les gestes ne sont pas cachés à proprement parler, mais le résultat de leur action défie certaines règles faisant partie de l'expérience du spectateur, par exemple si l'on fait correspondre un son très doux à un geste violent et inversement.

Le compositeur Philippe Manoury est explicite sur l'intérêt de cette non-transparence gestuelle-sonore dans son travail de composition : \iquote{(...) pour un violoniste, la manière dont il lève le bras et dont il va attaquer le son, la rapidité avec laquelle il prépare son coup d’archet nous renseignent un petit peu, mais pas complètement – parce que l’on ne sait pas quelle hauteur il va jouer – sur certaines catégories du son, comme le fait que le son sera agressif, fort, ou délicat et très doux. (...) Dans la musique instrumentale, cette causalité est très importante car cela participe de la façon dont nous la percevons et l’intérêt, avec la musique électronique, c’est que l’on peut remettre en question cette causalité-là : un tout petit geste peut provoquer une tempête. Entre le geste du pianiste qui va appuyer sur une touche du piano et le son qui va sortir, il y a une machine que j’appelle une boîte noire, qui peut inverser les polarités, c’est-à-dire que je peux très bien programmer la machine de manière à ce que plus le son qui va être joué va être minime, pianissimo, plus le son électronique qui va sortir va être au contraire démesuré : dans ce cas-là, le geste ne correspondra pas du tout au son.} \cite{manoury_philippe_2016}


\iquote{Dans le domaine du geste, les outils technologiques peuvent bien sûr jouer un rôle complice, démultipliant les perspectives, inversant les conséquences attendues, décelant l'infime ou captant par méthode statistique tel ou tel paramètre du jeu musical.} 
\iquote{(...) s’approprier à la manière d’un mime les gestualités sonores qui, malgré les indications de la partition, ne peuvent être réellement considérées et donc interprétées que via le prisme de l’écoute.}
P. Jodlowsky \cite{jodlowski_geste_2006}

\subsubsection{Métamorphisme du mapping}

\noindent Il n'est pas nécessaire de se baser sur des lois ancrées dans l'expérience collective, mais également possible de déjouer des règles qui ont été établies en cours de jeu, en modifiant dynamiquement le fonctionnement de l'instrument pendant le jeu. Il devient nécessaire dans ce cas de faire en sorte que la ``couture'' entre les différents réglages qui se succèdent soit invisible, soit en procédant par un changement continu (e.g. interpolation vers de nouveaux réglages), soit en transposant dans le nouvel espace de jeu un certain nombre de paramètres de l'ancien espace de jeu afin qu'ils correspondent\footnote{Un exemple concret est donné dans la vidéo présentant les Modèles Intermédiaires Dynamiques (cf. chapitre \ref{ch:algorithms}) : la transition entre les différents modèles présentés se fait en conservant une cohérence visuelle de la représentation (e.g. la position, l'orientation des formes) --~ alors que tout le comportement à changé \url{https://vimeo.com/25740547}.)}. 


(Auslander, Croft, Bin, Fyans, Benford)

Au-delà du timbre et des modes de jeu que le luthier ``programme'' dans l'élaboration d'un instrument acoustique, les \glspl{DMI} permettent d'élaborer une programmation qui à la particularité de pouvoir se déployer dans le temps avec une certaine autonomie, d'y dessiner un trajet\footnote{On peut mettre en regard la notion de trajet, qui suit un chemin, avec celle de plan   \iquote{la notion de référence s'est développée avec la masse grandissante des faits enregistrés, mais les écrits sont chacun une suite compacte, rythmée par des sigles et des notes marginales, dans laquelle le lecteur s'oriente à la manière du chasseur primitif, le long d'un trajet plutôt que sur un plan.} dans \cite{leroi-gourhan_geste_1964}, p 69}, voire d'y construire une narration. 

\iquote{Les propos des instruments qui nous entourent ne sont pas obligatoirement les nôtres. Ils appartiennent à ceux qui ont fait produire les instruments. Les détourner, c’est se libérer. Les instruments récents sont fascinants parce que, plus que tout autre, ils abritent des virtualités ignorées et parce qu’ils permettent des actions libératrices.} Villem Flusser, in ``les gestes''

Perception d'erreur du point de vue du spectateur \cite{fyans_ecological_2012}

\cite{emerson_gesture-sound_2018}

L'instrument cryptique qui d'un coup devient transparent (e.g. un solo de percussion chaotique qui retombe sur le temps), entraîne la relecture a posterori de tout ce qui a précédé pour lui attribuer une logique.


La cohérence\footnote{J'appelle ici ``cohérence'' la causalité \textbf{apparente} entre le geste et le son, pour la distinguer de la causalité \textbf{effective}, avec laquelle elle est mise en regard. La littérature en psychologie cognitive utilise cependant davantage le terme de perception de causalité.} entre le geste et le son se base sur un ensemble de valeurs qui ont été analysée dans de nombreux articles de sciences cognitive\footnote{Voir par exemple \cite{michotte_perception_2017} qui compile un grand nombre d'effets amenant à une perception congruente}, en particulier celles de la théorie de la perception, telles que :
\vspace{-1em}
\begin{itemize}[noitemsep]
	\item la congruence temporelle : synchronisme, destin commun, si j'entend le son en même temps que le geste, alors c'est ce geste qui a produit ce son.
	\item la congruence spatiale : si j'entend le son venir d'un endroit précis et que l'instrument se trouve à ce même endroit, alors le son vient de l'instrument.
	\item médiation par un objet intermédiaire dont on pense connaître le fonctionnement
\end{itemize}


\vspace{-1em}
\begin{itemize}[noitemsep]
	\item \textbf{lisible} : la perception du geste et du son apparait cohérente par rapport au système. Par exemple : on voit et on entend une personne parler.
	\item \textbf{illisible} : on voit une personne parler et on entend une autre voix
	\item \textbf{troublante} : c'est le cas du ventriloque : il est bien responsable du son que l'on entend (causalité), mais l'absence de mouvements de lèvre rend la perception visuelle incohérente avec la parception sonore (dissonance cognitive entre vue et ouïe).
	\item \textbf{subversive} : c'est le cas du playback
\end{itemize}

Risset fait remarquer l'importance de l'histoire du son d'origine mécanique dans la perception des sons \cite{risset_son_1992}: \iquote{Il semble à première vue que l'acoustique numérique puisse s'affranchir de la mécanique. Cependant notre ouïe a évolué dans un environnements d'objets vibrants: aussi la prise en considération des contraintes et des particularités des vibrations mécaniques est-elle importante pour comprendre les idiosyncrasies de la perception auditive et pour en tirer parti.
Les limites de l'acoustique numérique dépendent des capacités différentielles de perception davantage que des contraintes mécaniques. Pourtant, notre perception auditive est orientée par un monde de sons produits mécaniquement, et la mécanique ne doit pas être écartée de façon cavalière, comme l'ont suggéré les travaux de Gibson et Cadoz : les spécificités des vibrations mécaniques mettent en lumière l'organisation perceptuelle dans le processus auditif.\footnote{The limitations of digital acoustics depend upon the differential capacities of perception rather than upon the constraints of mechanics. Yet our auditory perception is geared to a world of mechanically-produced sounds, and mechanics should not be given a cavalier dismissal, as the work of Gibson and Cadoz has suggested : the specifics of mechanical vibrations shed light on the the perceptual organization in the hearing process.} TODO : traduire l'anglais, plus riche.}

