\chapter{Interview : Serge De Laubier}
\label{appendix:delaubier}

\section*{Biographie}


\section*{Transcript}
Serge De Laubier, interview du 17/07/2017, dans les studios de Puce Muse, Wissous.


VG Pour commencer peux tu présenter le méta-instrument en rappelant 
SdL l'histoire ? 
VG Peut-être pas toute l'histoire, mais comment ça a commencé et notamment les motivations initiales 
SdL c'est assez simple… il y a convergence de deux idées… quand on a commencé à faire des concerts électroacoustiques, j'ai commencé à développer des instruments, à fabriquer des trucs… plusieurs...et au fûr et à mesure des concerts les instruments s'accumulaient un peu… en gros c'était un instrument de musique… et donc au bout d'un moment je suis dit bon… 
VG des instruments joués en live ? 
SdL oui… il y avait la fameuse règle à poser le papier peint qui jouait sur des larsens à l'intérieur d'une règle métallique… bon, bref…  
VG avec des objets concrets… 
SdL oui oui… c'était l'idée de mettre en scène les musiques électroacoustique et de les jouer en direct… donc il y a eu pas mal d'essais et comme le chantier s'accumulait on s'est dit qu'il faudrait faire un instrument-instrument, ce serait pratique… c'est à dire un instrument sur lequel on puisse modéliser toutes les idées qu'on a, quoi.. Donc ça, ça a été une entrée, et parallèlement à ça on travaillait déjà sur la spatialisation du son, sur l'idée de simuler les déplacement du son en 3D et ce qui ressortait très nettement de cette histoire là, c'est que ça n'était pas intéressant si le son n'était pas pensé en même temps dans l'espace et dans sa nature, du coup on a fabriqué là aussi des dispositifs, au départ on a fabriqué un truc pour spatialiser les sons avec des commandes en tension, au début... on récupérait des tensions des synthés pour piloter en x,y,z les déplacements des… 
VG piloter une table de mixage ? 
SdL non, c'était le PSO, le processeur spatial octophonique… ça c'est 1986… c'est vieux…et donc tout de suite avec le PSO on s'est rendu compte qu'il fallait cogiter sur la manière de piloter des synthés à commande en tension, parce que tourner un bouton c'est un geste un peu limité musicalement… en tout cas par rapport au raffinement d'un violon ou d'un piano, on sent bien qu'il y a un gouffre… donc l'idée ça a été de faire converger les deux, c'est à dire de réfléchir à un instrument à fabriquer des instruments qui intègre la pensée de l'espace et donc les déplacements en x,y,z des sons fabriqués… et du coup en posant le truc comme ça, on avait déjà fabriqué des joysticks coulissants, des trucs comme ça pour les synthés à commande en tension et je me suis dit mais on peut aller plus loing, on peut faire un truc un peu plus sophistiqué… qui reprenne les doigts, qui reprenne… voilà… et donc le premier PSO en concert c'est 1986 et quasiment dans la foulée on démarre le chantier Méta-Instrument... qui débouche en concert en 1989… donc c'est toujours long de fabriquer un méta… (rires) mais l'idée était là, clairement… 
VG donc si je remonte un peu dans le passé, c'est parti du fait que tu faisais déjà de la musique électroacoustique … sur bande ? … ça a commencé avec quoi ? 
SdL oui, Puce Muse c'est fin 1982...moi je sors du conservatoire en 1982, fin 1982 on dépose l'association 
VG en électroacoustique au conservatoire ? 
SdL oui, sorti de chez Schaeffer, Rebel en 1982… et donc dans la foulée on monte l'association et dans la foulée on se dit quand même c'est con de faire des concerts… si on entend mieux chez soi, c'est pas la peine de faire des concerts… donc il faut qu'au concert, il y ait une expérience unique qui vaille le coup de se déplacer… 
VG d'où la spatialisation… 
SdL d'où réfléchir à un système de spatialisation… alors… j'avais un peu sous-estimé le boulot… de loin je me disais  me disais c'est pas très compliqué de faire un truc en électronique, en fait c'est un peu plus compliqué que ce que j'avais imaginé… et donc la machine est opérationnelle fin 1985, les premiers essais et 1986 dépôt de brevet… et vente du brevet qui finance en partie les créations puce muse et les nouvelles recherches…  
VG et la synthèse au début c'était quoi ? 
SdL synthèse modulaire analogique… cobbol, système 100 Roland, synthi… ces trucs là… 
VG et donc tu disais 1989 première version du MI… qu'est ce qui a guidé le choix du design, parce que ça ne ressemble pas aux outils de l'époque… qui étaient surtout basés sur des claviers, ou des boutons… 
SdL si un peu… ce qui ressemble le plus au MI c'est le joystick… 
VG l'idée d'un super joystick avec plus de boutons… ? 
SdL le cahier des charges du MI n'a jamais bougé, c'est manipuler simultanément et indépendamment le plus de données possibles… ce qui est assez rigolo c'est que j'entendais Jacques Rémy qui disait que manipuler plus de trois données, c'est impossible… il doit avoir un peu raison, mais moi aussi… (rires)… et donc là on faisait le compte tout à l'heure, on arrive à 92 données théoriquement manipulables presque indépendamment les unes des autres sur le prochain MI… en tout cas pour moi, ça a du sens…  
VG oui, mais après sur le design tu aurais pu aussi concevoir un ensemble de capteurs sur une table… à l'époque il y avait les tables de mixages qui étaient un peu l'instrument de musique électroacoustique… enfin un des instruments… 
SdL oui, mais non… enfin je ne crois pas moi… d'ailleurs ça ne s'appelle pas instrument, une table de mixage… c'est pas par hasard qu'on ne l'appelle pas instrument… c'est pas une histoire de prix parce qu'il y a des concerts qui coûtent très chers donc … alors... en fait, il me semble que dans la notion d'instrument, il y une notion de réactivité… une console c'est un outil de réglage et… et donc on pose, on déplace,  on écoute, on bouge un, deux boutons en même temps… et s'il y a plus, on fait une mémoire et on rappelle la mémoire… et c'est avant tout une posture de réglage et donc une temporalité de jeu qui n'est pas du tout la même, je pense, qu'un instrument dit acoustique où là, il faut pomper tout le temps pour produire quelque chose… donc si c'est compliqué à définir, c'est à mon avis parce qu'il y a des registres différents… enfin si ce type d'activité, instrument, pas instrument, méta-instrument, contrôleur, enfin bref… il y a un vocabulaire un peu étendu… c'est parce que ça recouvre à mon avis des pratiques différentes et des temporalités différentes… il y a un musicien qui avait fait une pièce pour la MM, Anthony Hécquet, je sais pas si tu te souviens, et Anthony parlait des butineurs… et c'est vrai qu'il y a un côté comme ça… c'est vrai que les instruments acoustiques sont des butineurs et… par exemple c'est très compliqué d'avoir un geste lent et continu (faisant un lent mouvement de déplacement du bras gauche) pendant qu'il y en a un autre qui fait ça (faisant des gestes rapides du bout des doigts de la main droite)… on sent bien que… alors bon c'est jouable, mais … et le geste de réglage c'est encore autre chose… c'est encore plus lent que ça … c'est je déplace, je pose, j'écoute, je déplace, j'écoute… alors… pourquoi tu disais ça ? ah oui… alors, donc du coup, dans le geste, il y a quand même, quand on fabrique un instrument il y a une contrainte, c'est … le corps … et donc forcément, il faut que les instruments, en tout cas ceux qui fonctionnent bien, sont quand même relativement bien adapté au corps… relativement parce que … même les instruments acoustiques, les instrumentistes se détruisent pas mal mais… quand même malgré tout, il arrivent à les pratiquer pendant des années, plusieurs heures par jour, et en général ils tiennent… donc la contrainte du corps est importante et la première contrainte c'est celle des mains, avant la contrainte du corps… parce que c'est le plus agile je pense, le plus agile, rapide, réactif, enfin bref… donc c'est la plus grosse différence sur le MI, sur le MI1, il y a trois touches qui sont en plus des touches raides avec des jauges contraintes, et le MI4 on est à 40 touches… je parle par côté, hein... 40 touches par côté et qui sont des touches molles, et beaucoup plus précises et réactives… et notamment parce que c'est facile de jouer plusieurs touches en même temps avec un seul doigt… ça existe sur les instruments acoustiques aussi… 
VG qui sont des touches de pression… enfin l'ergonomie est … 
SdL pression et attaque 
VG l'ergonomie de l'instrument qui a mené par exemple au design du MI, tu dis qu'elle est guidée par l'ergonomie du corps et en particulier des mains, et ça a guidé j'imagine le choix des capteurs, tu as essentiellement des capteurs de pressions, alors que sur une table on est plus sur des faders linéaires qui sont des réglages qu'on peut mettre à une certaine position et ils y restent sagement, alors que les capteurs de pression ne sont jamais stable en dehors de leur position zéro...  
SdL sur les capteurs de pression il y a plein d'algo qui permettent de transformer les capteurs de pression en capteurs stables...par contre c'est compliqué dans l'autre sens… 
VG c'est à dire en utilisant des algo qui incrémentent ou décrémentent une variable par exemple… 
SdL oui… mais il y a énormément de solutions… là récemment je retravaille sur les nouvelles touches du MI, pour voir un petit peu, je monte en pression dans le nouveau MI… et en fait au fur et à mesure je me perds dans les champs possibles…  mais donc de bien intérioriser ce qu'on veut faire, c'est parfois compliqué… avec huit touches on a un champ de possibles qui est vertigineux… 
VG pour transformer les capteurs bruts de pression en un contrôle destiné à une variable qui ne suit pas forcément la course du capteur… 
SdL voilà… et qu'est ce qu'on fait de toutes ces données… 
VG Est-ce que tu as gardé, du coup… parce que sur le MI3, il y a un fader sur chaque main… 
SdL oui… on a empiré… la il y a des joysticks sous chaque pouce… des joysticks sans ressort… 
VG qui restent en position du coup … Comme des faders multi-axes ... 
SdL oui… sur lesquels on a mis une toute petite friction… pour que quand on tourne la poignée ils ne bougent pas, mais qu'en même temps ils soient très doux quand on les déplace… et ça marche bien… et pareil, on les échantillonne, je pense qu'on va travailler à 1kHz et en 14 bits… donc ça c'est ce qu'on fait déjà sur les grands axes… on est à 800Hz pour le moment et je pense qu'on va accélérer encore un peu… de manière à avoir des dérivées, des accélérations qui soient très propres… et 14 bits sans bruit… on pose l'objet, on le lâche et on une valeur stable en 14 bits… et alors on peut se dire c'est délirant 14bits mais en fait pas du tout, dès qu'on dérive on perd des bits… donc 14 bits sans bruit… et ça a vraiment du sens… donc on a ramé un peu… le dire, ça va vite, mais le faire c'est une autre histoire… on a essayé de faire des versions simples, ça ne marchait pas… bon bref… on a erré pas mal… mais donc tout ça a du sens, avoir une belle mesure, précise, avec une légère friction pour savoir où on est, pour pouvoir déplacer un axe sans bouger l'autre… c'est tout con mais c'est capital… qu'il n'y ait pas de diaphonie… donc si je bouge horizontalement, il faut que je sache que je bouge horizontalement sans bouger verticalement, et si je bouge verticalement, il faut que je puisse bouger que verticalement, et qu'après je puisse bouger les deux sans que le frottement gêne...donc il y a un frottement-guide qui est très fin, mais qui est là… qui est très fin… c'est la même chose sur la poignée… donc là on n'a gardé que deux axes sur la poignée, parce qu'après ça faisait un montage mécanique assez lourd sur trois axes… donc là on a ça, ça et ça (faisant les rotations sur deux axes du poignet) sur le MI et on n'a pas celui là (mimant la rotation manquante)… donc là aussi on a des ressorts qu'on peut régler en rappel, pour avoir juste le soulagement du geste mais sentir l'objet, quoi… là on l'a déjà un petit peu (montrant le MI3)  comme ça… une oscillation amortie… mais c'est plus précis sur le nouveau 
VG et alors sentir qu'on exerce une force… (fin première vidéo) 