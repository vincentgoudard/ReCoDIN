\chapter{Interview : Serge De Laubier}
\label{appendix:delaubier}
\section*{Biographie}

\noindent Compositeur, chercheur et musicien, Serge de Laubier (né en 1958) fonde PUCE MUSE avec lequel il co-invente le Processeur Spatial Octophonique (brevet n°8600153). Il est aussi concepteur du Méta-Instrument et l’auteur des logiciels MIDI Formers (© INA-GRM) qui ont reçu le premier prix au Concours International de logiciels musicaux de Bourges 1996.\\
\indent Il a obtenu plusieurs récompenses notamment le FAUST d’OR 1992, le Grand Prix du Festival Vidéo Art de Locarno 1994, le 1er prix de composition du concours international de Bourges 1995, le prix spécial du jury au SATIS 2001.\\
\indent Ses dernières créations mêlent étroitement musique, images et nouvelles technologies. Elles ont souvent un caractère événementiel et monumental.

\noindent Site web : \url{http://www.pucemuse.com}

\section*{Transcript}
\noindent Serge De Laubier, entretien du 17/07/2017, dans les studios de Puce Muse, Wissous.

\noindent [Les entretiens ne sont pas disponibles publiquement.]


% VG — Pour commencer, pourrais-tu présenter le Méta-Instrument en rappelant ...

% SdL — ... l'histoire? 

% VG — ...peut-être pas toute l'histoire, mais comment ça a commencé et notamment les motivations initiales 

% SdL — c'est assez simple... il y a convergence de deux idées... quand on a commencé à faire des concerts électroacoustiques, j'ai commencé à développer des instruments, à fabriquer des trucs... plusieurs...et au fûr et à mesure des concerts les instruments s'accumulaient un peu... en gros c'était un instrument de musique... et donc au bout d'un moment je suis dit bon... 

% VG — des instruments joués en live? 

% SdL — oui... il y avait la fameuse règle à poser le papier peint qui jouait sur des larsens à l'intérieur d'une règle métallique... bon, bref... 

% VG — avec des objets concrets... 

% SdL — oui oui... c'était l'idée de mettre en scène les musiques électroacoustique et de les jouer en direct... donc il y a eu pas mal d'essais et comme le chantier s'accumulait on s'est dit qu'il faudrait faire un instrument-instrument, ce serait pratique... c'est-à-dire un instrument sur lequel on puisse modéliser toutes les idées qu'on a, quoi.. Donc ça, ça a été une entrée, et parallèlement à ça on travaillait déjà sur la spatialisation du son, sur l'idée de simuler les déplacement du son en 3D et ce qui ressortait très nettement de cette histoire là, c'est que ça n'était pas intéressant si le son n'était pas pensé en même temps dans l'espace et dans sa nature, du coup on a fabriqué là aussi des dispositifs, au départ on a fabriqué un truc pour spatialiser les sons avec des commandes en tension, au début... on récupérait des tensions des synthés pour piloter en x, y, z les déplacements des... 

% VG — piloter une table de mixage? 

% SdL — non, c'était le PSO, le processeur spatial octophonique... ça c'est 1986... c'est vieux...et donc tout de suite avec le PSO on s'est rendu compte qu'il fallait cogiter sur la manière de piloter des synthés à commande en tension, parce que tourner un bouton c'est un geste un peu limité musicalement... en tout cas par rapport au raffinement d'un violon ou d'un piano, on sent bien qu'il y a un gouffre... donc l'idée ça a été de faire converger les deux, c'est-à-dire de réfléchir à un instrument à fabriquer des instruments qui intègre la pensée de l'espace et donc les déplacements en x,y,z des sons fabriqués... et du coup en posant le truc comme ça, on avait déjà fabriqué des joysticks coulissants, des trucs comme ça pour les synthés à commande en tension et je me suis dit mais on peut aller plus loing, on peut faire un truc un peu plus sophistiqué... qui reprenne les doigts, qui reprenne... voilà... et donc le premier PSO en concert c'est 1986 et quasiment dans la foulée on démarre le chantier Méta-Instrument... qui débouche en concert en 1989... donc c'est toujours long de fabriquer un méta... (rires) mais l'idée était là, clairement... 

% VG — donc si je remonte un peu dans le passé, c'est parti du fait que tu faisais déjà de la musique électroacoustique ... sur bande? ... ça a commencé avec quoi? 

% SdL — oui, Puce Muse c'est fin 1982...moi je sors du conservatoire en 1982, fin 1982 on dépose l'association 

% VG — en électroacoustique au conservatoire? 

% SdL — oui, sorti de chez Schaeffer, Rebel en 1982... et donc dans la foulée on monte l'association et dans la foulée on se dit quand même c'est con de faire des concerts... si on entend mieux chez soi, c'est pas la peine de faire des concerts... donc il faut qu'au concert, il y ait une expérience unique qui vaille le coup de se déplacer... 

% VG — d'où la spatialisation... 

% SdL — d'où réfléchir à un système de spatialisation... alors... j'avais un peu sous-estimé le boulot... de loin je me disais me disais c'est pas très compliqué de faire un truc en électronique, en fait c'est un peu plus compliqué que ce que j'avais imaginé... et donc la machine est opérationnelle fin 1985, les premiers essais et 1986 dépôt de brevet... et vente du brevet qui finance en partie les créations puce muse et les nouvelles recherches... 

% VG — et la synthèse au début c'était quoi? 

% SdL — synthèse modulaire analogique... Cobol, System 100 Roland, Synthi... ces trucs là... 

% VG — et donc tu disais 1989 première version du Méta-Instrument ... qu'est ce qui a guidé le choix du design, parce que ça ne ressemble pas aux outils de l'époque... qui étaient surtout basés sur des claviers, ou des boutons... 

% SdL — si un peu... ce qui ressemble le plus au Méta-Instrument, c'est le joystick... 

% VG — l'idée d'un super-joystick avec plus de boutons...? 

% SdL — le cahier des charges du Méta-Instrument n'a jamais bougé, c'est manipuler simultanément et indépendamment le plus de données possibles... ce qui est assez rigolo c'est que j'entendais Jacques Rémus qui disait que manipuler plus de trois données, c'est impossible... il doit avoir un peu raison, mais moi aussi... (rires)... et donc là on faisait le compte tout à l'heure, on arrive à 92 données théoriquement manipulables presque indépendamment les unes des autres sur le prochain Méta-Instrument ... en tout cas pour moi, ça a du sens... 

% VG — oui, mais après sur le design tu aurais pu aussi concevoir un ensemble de capteurs sur une table... à l'époque il y avait les tables de mixages qui étaient un peu l'instrument de musique électroacoustique... enfin un des instruments... 

% SdL — oui, mais non... enfin je ne crois pas moi... d'ailleurs ça ne s'appelle pas instrument, une table de mixage... c'est pas par hasard qu'on ne l'appelle pas instrument... c'est pas une histoire de prix parce qu'il y a des concerts qui coûtent très chers donc ... alors... en fait, il me semble que dans la notion d'instrument, il y une notion de réactivité... une console c'est un outil de réglage et... et donc on pose, on déplace, on écoute, on bouge un, deux boutons en même temps... et s'il y a plus, on fait une mémoire et on rappelle la mémoire... et c'est avant tout une posture de réglage et donc une temporalité de jeu qui n'est pas du tout la même, je pense, qu'un instrument dit ``acoustique'' où là, il faut pomper tout le temps pour produire quelque chose... donc si c'est compliqué à définir, c'est à mon avis parce qu'il y a des registres différents... enfin si ce type d'activité, instrument, pas instrument, méta-instrument, contrôleur, enfin bref... il y a un vocabulaire un peu étendu... c'est parce que ça recouvre à mon avis des pratiques différentes et des temporalités différentes... il y a un musicien qui avait fait une pièce pour la Méta-Mallette, Anthony Hécquet, je sais pas si tu te souviens, et Anthony parlait des butineurs... et c'est vrai qu'il y a un côté comme ça... c'est vrai que les instruments acoustiques sont des butineurs et... par exemple c'est très compliqué d'avoir un geste lent et continu (faisant un lent mouvement de déplacement du bras gauche) pendant qu'il y en a un autre qui fait ça (faisant des gestes rapides du bout des doigts de la main droite)... on sent bien que... alors bon c'est jouable, mais ... et le geste de réglage c'est encore autre chose... c'est encore plus lent que ça ... c'est je déplace, je pose, j'écoute, je déplace, j'écoute... alors... pourquoi tu disais ça? ah oui... alors, donc du coup, dans le geste, il y a quand même, quand on fabrique un instrument il y a une contrainte, c'est ... le corps ... et donc forcément, il faut que les instruments, en tout cas ceux qui fonctionnent bien, sont quand même relativement bien adaptés au corps... relativement parce que ... même les instruments acoustiques, les instrumentistes se détruisent pas mal mais... quand même malgré tout, il arrivent à les pratiquer pendant des années, plusieurs heures par jour, et en général ils tiennent... donc la contrainte du corps est importante et la première contrainte c'est celle des mains, avant la contrainte du corps... parce que c'est le plus agile je pense, le plus agile, rapide, réactif, enfin bref... donc c'est la plus grosse différence sur le Méta-Instrument , sur le Méta-Instrument 1, il y a trois touches qui sont en plus des touches raides avec des jauges contraintes, et le Méta-Instrument 4 on est à 40 touches... je parle par côté, hein... 40 touches par côté et qui sont des touches molles, et beaucoup plus précises et réactives... et notamment parce que c'est facile de jouer plusieurs touches en même temps avec un seul doigt... ça existe sur les instruments acoustiques aussi... 

% VG — qui sont des touches de pression... enfin l'ergonomie est ... 

% SdL — pression et attaque 

% VG — l'ergonomie de l'instrument qui a mené par exemple au design du Méta-Instrument , tu dis qu'elle est guidée par l'ergonomie du corps et en particulier des mains, et ça a guidé j'imagine le choix des capteurs, tu as essentiellement des capteurs de pressions, alors que sur une table on est plus sur des faders linéaires qui sont des réglages qu'on peut mettre à une certaine position et ils y restent sagement, alors que les capteurs de pression ne sont jamais stable en dehors de leur position zéro... 

% SdL — sur les capteurs de pression il y a plein d'algo qui permettent de transformer les capteurs de pression en capteurs stables...par contre c'est compliqué dans l'autre sens... 

% VG — c'est-à-dire en utilisant des algo qui incrémentent ou décrémentent une variable par exemple... 

% SdL — oui... mais il y a énormément de solutions... là récemment je retravaille sur les nouvelles touches du Méta-Instrument, pour voir un petit peu, je monte en pression dans le nouveau Méta-Instrument ... et en fait au fur et à mesure je me perds dans les champs possibles... mais donc de bien intérioriser ce qu'on veut faire, c'est parfois compliqué... avec huit touches on a un champ de possibles qui est vertigineux... 

% VG — pour transformer les capteurs bruts de pression en un contrôle destiné à une variable qui ne suit pas forcément la course du capteur... 

% SdL — voilà... et qu'est ce qu'on fait de toutes ces données... 

% VG — Est-ce que tu as gardé, du coup... parce que sur le Méta-Instrument 3, il y a un fader sur chaque main... 

% SdL — oui... on a empiré... la il y a des joysticks sous chaque pouce... des joysticks sans ressort... 

% VG — qui restent en position du coup... Comme des faders multi-axes ... 

% SdL — oui... sur lesquels on a mis une toute petite friction... pour que quand on tourne la poignée ils ne bougent pas, mais qu'en même temps ils soient très doux quand on les déplace... et ça marche bien... et pareil, on les échantillonne, je pense qu'on va travailler à 1kHz et en 14 bits... donc ça c'est ce qu'on fait déjà sur les grands axes... on est à 800Hz pour le moment et je pense qu'on va accélérer encore un peu... de manière à avoir des dérivées, des accélérations qui soient très propres... et 14 bits sans bruit... on pose l'objet, on le lâche et on une valeur stable en 14 bits... et alors on peut se dire c'est délirant 14bits mais en fait pas du tout, dès qu'on dérive on perd des bits... donc 14 bits sans bruit... et ça a vraiment du sens... donc on a ramé un peu... le dire, ça va vite, mais le faire c'est une autre histoire... on a essayé de faire des versions simples, ça ne marchait pas... bon bref... on a erré pas mal... mais donc tout ça a du sens, avoir une belle mesure, précise, avec une légère friction pour savoir où on est, pour pouvoir déplacer un axe sans bouger l'autre... c'est tout con mais c'est capital... qu'il n'y ait pas de diaphonie... donc si je bouge horizontalement, il faut que je sache que je bouge horizontalement sans bouger verticalement, et si je bouge verticalement, il faut que je puisse bouger que verticalement, et qu'après je puisse bouger les deux sans que le frottement gêne...donc il y a un frottement-guide qui est très fin, mais qui est là... qui est très fin... c'est la même chose sur la poignée... donc là on n'a gardé que deux axes sur la poignée, parce qu'après ça faisait un montage mécanique assez lourd sur trois axes... donc là on a ça, ça et ça (faisant les rotations sur deux axes du poignet) sur le Méta-Instrument et on n'a pas celui là (mimant la rotation manquante)... donc là aussi on a des ressorts qu'on peut régler en rappel, pour avoir juste le soulagement du geste mais sentir l'objet, quoi... là on l'a déjà un petit peu (montrant le Méta-Instrument 3) comme ça... une oscillation amortie... mais c'est plus précis sur le nouveau 

% VG — et alors sentir qu'on exerce une force dans son poignet?... 

% SdL — ou quand la poignée pivote, elle pivote autour de l'avant bras ici... et du coup elle est déséquilibrée c'est-à-dire que l'axe n'est pas au centre de la main, donc comme il y a tout un clavier a priori s'il n'y a pas de ressort, elle tombe comme ça (mimant la poignée qui tombe) et donc il faut la maintenir, ce qui est fatiguant et très désagréable, donc il y a un petit ressort qui est placé et qui la ramène qui fait qu'il n'y a pas d'effort... 

% VG — à la position de repos... 

% SdL — voilà... 

% VG — et tout ce développement des joysticks ... vous faites tout en interne? 

% SdL — non non, là les joysticks c'est des bons joysticks de jeux vidéo, par contre on les adapte en interne 

% VG — vous customisez des choses qui sont vendues déjà faites... 

% SdL — Oui ... 

% VG — par contre les touches? 

% SdL — les touches... non ...enfin... il y a un bout des touches qui est fait par Marc (Sirguy, NdE) et sa boite Eowave et il y a un bout qui est fait par une boite de mouliste qui est au Mans mais ils sous-traitent en Chine et Marc sous-traite en Chine aussi ... c'est-à-dire qu'il y a un bout de la conception qui est faite ici... sur les touches, il y a une grosse partie qui a été dégrossie ici, jusqu'à arriver à une forme très économe... et Marc a adapté l'électronique, les circuits etc. et donc on a contacter Phil Bore (???) avec la pièce définie en 3D et lui donnant le truc pour qu'il la tire ... on en a tiré une centaine. 

% VG — et vous faites l'assemblage final 

% SdL — et on fait l'assemblage, oui... et qu'est ce qu'il y a d'autre dessus?... pour le moment sur la numérisation des potars c'est fait ici, on va peut-être réaliser une carte, faire réaliser une carte pour diminuer l'encombrement et le coût... voilà et les boitiers, donc on a une imprimante 3D qui usine beaucoup là en ce moment... et donc toutes les formes sont imprimées en 3D, les boitiers etc. 

% VG — Oui.... Et ça me fait penser qu'entre 1987 et 2017, il y une petite différence d'accessibilité des technologies qui doit être sensible, à la fois sur l'électronique et sur la possibilité de faire... les imprimantes 3D n'existaient pas en 1987, ça a dû... on mesure le bénéfie 

% SdL — oui, là on a une souplesse de travail... on a eu la réunion ce matin, actuellement on est à une réunion par semaine avec Catherine (Catherine Lhospitel, NdE) qui est plutôt sur le design, Dominique (Dominique Brégeard, NdE) qui est sur la mécanique, moi et Jérémie qui sommes sur l'électronique... donc jamais j'ai eu ça dans les chantiers précédents donc je pense qu'on devrait arriver à quelque chose de bien... là ce qu'on a rajouté qui est nouveau c'est ... donc on travaille le prix... pour que ce soit pas très cher, facile à réaliser et surtout que ce soit modulaire... on voulait faire ça là et puis on a pas réussi...donc moi j'ai des grandes mains, des gens qui ont des petites mains... lundi György (Kurtag Jr, NdE) essayait avec les touches, et il me disait «pour moi c'est pas pratique , les petites touches au fond j'ai du mal» même en le réglant au mieux... mais en fait on peut très bien ne prendre que deux touches... deux méta-touches... une méta-touche c'est huit touches, donc ...et en les plaçant, comme c'est très souple c'est du velcro (le système d'attache des méta-touches, NdE), on place où on veut et si on prend deux touches on peut jouer encore quatre touches par doigts ce qui fait 16 touches à la fin ce qui est déjà pas mal, et là vraiment les tomber à l'endroit où tombent les doigts ... 

% VG — oui, donc la modularité ergonomique aussi... 

% SdL — modularité des touches, en fait il y a 13... oui c'est ça 13 interfaces MIDI sur le nouveau Méta-Instrument ... et donc, après tu peux prendre une interface o en prendre deux, cinq...les placer comme tu veux, prendre une poignée, ne pas prendre de poignée, prendre tout le Méta-Instrument .... Donc ça c'est une idée importante et comme tout sort en MIDI tu peux paramétrer tes interfaces, la réactivité, la sensibilité, la précision, etc. les codes... et après tu peux programmer tout derrière ou tu mettre directement sur Live (Ableton, NdE) ou un séquenceur du commerce, ou un synthé du commerce et jouer directement avec quoi... donc ça c'est vraiment des choses... je dirais le plus gros pas c'est celui là... Quand Marc avait vu le Méta-Instrument 3 il y a... 6 , 7 ans il avait dit «ah mais, tu devrais embarquer de l'intelligence dans les capteurs» et je pense que là, on n'est qu'au début de ça... c'est incroyable maintenant comme les capteurs sont puissants, on peut aller très loin dans l'intégration de l'intelligence des capteurs, intelligence entre guillemets, enfin bon bref 

% VG — qu'est ce que tu entends par intelligence des capteurs? 

% SdL — Eh bien les premiers bouts, ce serait de faire des dérivées de mouvement par exemple, mais après on voit très bien ce qui arrive derrrière... donc, travailler du geste discontinu, du geste continu, travailler du lissage, des résonances, et puis et puis et puis.. bref, tout le chantier du mapping, je pense qu'il y a une bonne partie du chantier mapping qu'on peut imaginer rapatrier dans le capteur parce que... l'air de rien la plupart des logiciels du commerce ne sont pas terrible en mapping, donc c'est assez raide, ils prennent la donnée et puis on joue comme ça avec la donnée brute quoi... 

% VG — dans les capteurs MIDI tu veux dire? Dans les interfaces commerciales? 

% SdL — non non, dans les logiciels ... même si tu prends Live, c'est pas facile de faire, euh, ou alors tu prends un Max for Live... 

% VG — pour changer les courbes de réponse... 

% SdL — voilà c'est ça, mais plus que les courbes de réponse... tu vois bien tout ce qui est possible en mapping qui est quasiment infini... donc je pense qu'il y a une partie du mapping qui va progressivement se ramener dans les capteurs eux-mêmes 

% VG — qui en font des capteurs plus dédiés 

% SdL — oui et que tu paramétreras... donc des capteurs bidirectionnels que tu pourras paramétrer même dynamiquement pour changer les comportements en fonction de ce que tu veux jouer, quoi... 

% VG — pour revenir à la question de tout à l'heure, qui n'était pas forcément tant quand est-ce que commence l'instrument quand finit l'outil technique, qu'est ce qui selon toi oriente le design... quand on parle d'instrument pour toi, c'est l'idée de jouer en direct, en live, pour un concert... du coup les critères qui orientent le design, ce qui définit l'instrument pour toi, est ce que c'est la réactivité et la vitesse à laquelle on peut attraper les différents paramètres de contrôle ou est-ce que ça passe par d'autres choses que ça? 

% SdL — c'est la souplesse pour s'adapter à toutes les idées qu'on peut avoir...plus on peut jouer des idées qu'on a en tête, plus l'objet est riche, en gros 

% VG — et par rapport à la modularité de l'instrument dont tu parlais... 

% SdL — on a l'impression quand même quand on utilises... enfin j'ai souvent eu cette impression de vertige en utilisant le Méta-Instrument en me disant où sont les limites de ce truc là quoi... et les limites, c'est la combinatoire entre tous les capteurs, donc on peut compter le nombre de possibilités, parce que tout est numérique, c'est quand même fini... et donc pendant longtemps j'avais fait l'erreur de dire, c'était sur le Méta-Instrument 2, donc il y avait 32 capteurs qui étaient échantillonnés en 16 bits... donc c'était 2 puissance...et je disais 7 plus 32... et donc un jour j'ai été repris, j'étais toujours fier de mettre un peu d'équations, ça fait savant, j'ai été repris par un gars au fond de la salle qui a dit«je crois que vous faites une petite erreur de calcul c'est pas 2 puissance 7+32, c'est 2 puissance 7x32» ... ce qui fait 2 puissance 224, ce qui fait à peu près 10 puissance 63, ce qui fait pas mal... et donc ça c'était le Méta-Instrument 2, et donc aujourd'hui je compte plus... j'ai arrêté 

% VG — tu parles en termes de combinatoire, de nombre de bits, les valeurs sont échantillonnés sur 14 bits... 

% SdL — à ce moment là c'était sept... 

% VG — ça me fait penser à une vidéo\footnote{\textit{``Will We Ever Run Out of New Music?''} sur la chaine YouTube Vsauce: \url{https://youtu.be/DAcjV60RnRw} , NdE} où quelqu'un tentait de trouver si on aura, un jour, épuisé toutes les musiques qui sont jouables et pour sa base de calcul il prenait le format CD et il disait sur un CD il y a tant de bits, donc ça fait un nombre fini de possibilités... sauf qu'il y a quand même quelque chose qui rentre en compte, c'est la perception derrière et que toutes les musiques ne sont pas différentiables ... donc après plutôt que de compter le nombre de bits, il prenait la combinatoire des notes d'une gamme, qui sont déjà des éléments musicaux de plus haut niveau, car une même note peut prendre un grand nombre de formes différentes et ça m'amène à cette question qui est comment dans ce champ des possibles dont on sent qu'il ne fait que s'élargir, à la fois parce que les technologies ramènent davantage de bits, qu'on peut mettre davantage de capteurs et qu'en plus de ça les choses sont modulaires, comment trouver les chemins qui se dessinent? Où sont les éléments pérennes, les points de pivots... et par exemple en terme de programmation, tu parlais de fonctions qu'on peut intégrer dans les capteurs, comment selon toi se dessinent ces choses là, cette sorte peut-être de vocabulaire de fonctions qui reviennent... est-ce que c'est quelque chose qui se trouvent empiriquement ou... 

% SdL — je ne sais pas très bien... il y a un peu deux pôles... un pôle du plus, du toujours plus... et donc je suis parti pour celui là pour le meilleur et pour le pire... sur le Méta-Instrument 2 j'étais déjà pris de vertige, donc maintenant... donc se repérer... c'est pour ça que je ne sais pas très bien répondre à ça simplement... je vois que nos capacités motrices dépassent largement ce que je fais aujourd'hui avec un Méta-Instrument ... donc allons y quoi... mais dans l'autre sens, même sur une touche, quand tu vois le travail de Jean (Haury, NdT), Jean arrive avec une touche à dire déjà énormément donc... voilà... (rires)... j'ai pas beaucoup de réponse, je pense qu'il y a une partie qui est l'affinité de ne pas supporter les limites... alors que c'est quand même idiot, on sait qu'il suffit d'ouvrir la porte et ça s'agrandit, alors ouvrons la porte... on verra bien après... c'est un peu ça le mécanisme... mais ça ne répond qu'à moitié... après il y a un autre guide, qui est resté et qui reste encore, qui est... je trouve que des instruments comme ça sont intéressants à partir du moment où ils apportent un son nouveau... et ... j'ai pas dit des sons mais un son... quelque part ils permettent d'explorer des territoires qui étaient inexporables avant, quoi... sinon c'est pas la peine... il vaut mieux continuer à travailler avec les instruments qui existent et en fait qu'aujourd'hui, je pense qu'avec les instruments qui existent il y a plein de trucs à faire... mais après c'est plutôt une zone de sensibilité... moi j'étais plus à l'aide avec les techno et je me suis dit bon, j'y vais... j'ai répondu qu'à moitié à ta question mais bon... 

% VG — en même temps, tu as utilisé un terme assez révélateur, tu parlais de «guide», de territoire et ma question serait peut-être à reformuler en terme de cartographie des possibles... quelque part ta réponse consiste à dire, reprend moi si je me trompe, il y a une envie d'aller voir ce qui se passe après les bords de la cartes qu'on connaît et quelque part les deux pôles seraient, il y a cette partie encore inexplorée dans laquelle on avant un peu en explorateur avec son coupe coupe et on essaie de faire sa route là dedans, mais il y a quand même, malgré la jeunesse de ces musiques un vaste territoires qui n'est pas forcément très bien cartographié, et sur lequel il y a déjà beaucoup de chemins qui ont été pris par les uns et les autres ... et c'est peut-être ça l'autre pôle, parce que pour pouvoir aller au-delà des bords de la carte il faut déjà pouvoir naviguer assez rapidement dans les parties déjà un peu connues... 

% SdL — oui... je sais pas... il me semble qu'il y a un guide qui est l'oreille... en fait ce qui est compliqué c'est cette histoire de son nouveau... un de mes grands plaisirs en musique c'est quand même de m'arrêter sur un son en me disant «qu'est ce que c'est.... qu'est ce que c'est que ce truc... j'entends... j'entends pas... qu'est ce qui se passe...» et je trouve qu'à ce moment là, il y a une place pour l'imaginaire qui est fabuleuse... ça part très vite... en fait tous les sons qu'on entend sont toujours nouveaux... donc pourquoi ce truc là n'arrive qu'à certains moments ... donc il y faut qu'il y ait quand même une dimension énigmatique dans ce qu'on entend... «j'ai jamais entendu ça» ...donc je pense que ce sont des outils qui potentiellement prêtent à ça, prêtent à aller voir ... un endroit où on espère entendre et réaliser des choses qu'on n'a jamais entendues... mais qui arrêtent... je me souviens, quand j'étais à Louis Lumière (Ecole nationale supérieure Louis-Lumiere, dédiée au cinéma, à la photographie et au son, NdE) en 1976, je crois, 1977 peut-être, j'étais en stage à l'INA et à un moment on écoute un enregistrement, et c'était Urban Sax (groupe de huit saxophonistes créé par le compositeur Gilbert Artman, NdE) et la première fois, je m'en souviens encore, le son d'Urban Sax, et j'en ai parlé avec Artman que j'ai vu il y a deux ans, je me souviens du jour, c'était en été, j'étais en stage et ooohh... il y en a plein des expériences comme ça... mais je dirais sur Urban c'est vraiment, plus que la composition, c'est le son, cette espèce de masse de saxophones où on ne comprend plus ce que c'est, c'est tellement improbable, c'est juste faux comme il faut... il y en a plein évidemment, et en électroacoustique, il y en a pas mal... on a passé dans, comment ça s'appelle, le logiciel suisse qui permet de séparer les timbres pour transposer ... 

% VG — Mélodyne? 

% SdL — oui c'est ça... on a passé De Natura (De Natura Sonorum, pièce de Bernard Parmegiani) là dedans, et tout d'un coup on voit des bouts, on entend un peu les stratégies de Parmé pour reconstruire des timbres qui fusionnent et on est incapables de séparer les aliments, les matières qui l'ont construit ... et quand même De Natura, on peut l'écouter, le ré-écouter régulièrement en se disantoh mais merde, comment il a fait ça, comme il a pensé ça pour arriver là... et il en parle, en même temps ça paraît simple mais c'est pas simple du tout... alors pourquoi on s'arrête quand De Natura arrive, je sais pas dire, mais je sais que je m'arrête... c'est un peu ça les guides... 

% VG — l'oreille 

% SdL — plus que l'oreille, ça passe par l'oreille, mais ça met en jeu tout, ça met en jeu ton histoire, ta culture... 

% VG — l'oreille et ses souvenirs... 

% SdL — c'est ça qui est riche... 

% VG — et tu disais que le Méta-Instrument devenait, même dans son côté hardware, modulaire, et derrière quand toutes ces données sont envoyés sur une machine, c'est dans des logiciels comme Max qui sont éminemment modulaires aussi... une chose que je trouve intéressante, en tout cas spécifique, aux instruments numériques, même si l'analogique peut-être aussi modulaire avec l'eurorack, etc., mais il y a quelque chose de singulier avec le numérique, qui est que cette modularité est ré-arrangeable en moins de temps qu'un claquement de doigt, en un tour de CPU, à des fréquences qui sont en-deça des capacités de perception de changement... et sur la durée d'un concert, si tu joues une heure, tu n'as pas toujours le même instrument sous les doigts... 

% SdL — ta formulation est rigolote (rires)... vas y continue... 

% VG — tu as toujours la même interface, mais son ergonomie et les sons de synthèse ne sont plus les même et n'ont rien à voir... comment est-ce que tu organises cet espace là, comment il se construit pour toi... il y a des stratégies différentes selon les personnes, je pense par exemple, j'ai vu récemment CHDH (Cyrille Henry et Nicolas Mongermont, NdE) qui jouait en concert d'un instrument qui était un seul gros algorithme avec plein de paramètres, et leur stratégie c'était justement de garder plus ou moins le même instrument sur tout le concert et de passer de presets à d'autres... comment toi tu te représentes ça? 

% SdL — non, souvent c'est plutôt des territoires que j'ai envie d'explorer, des bouts d'idées ou des trucs... donc je connecte.. je mets en doigts tout ça et je joue... donc après j'ai une collection de trucs comme ça et je sélectionne ma collection.. 

% VG — il y a l'idée de fonctionner par tableaux, par scènes... 

% SdL — on pourrait dire par mouvements...par moments.. 

% VG — tout à l'heure quand tu parlais des débuts du Méta-Instrument , tu parlais du fait que tu construisais un instrument par pièce.. 

% SdL — oui oui 

% VG — et quelque part il y a cette logique qui est restée de produire un instrument virtuel par pièce et de passer d'un instrument à l'autre 

% SdL — Oui ... 

% VG — et qu'est ce qui fait que tu restes sur ce fonctionnement là, alors que la technologie te permet potentiellement de ... 

% SdL — il y en a qui bougent un peu... mais j'ai eu des périodes différentes... c'est-à-dire que j'espérais que les gens jouent du Méta-Instrument ... pas que moi... et depuis longtemps j'espérais ça... et depuis très longtemps c'est compliqué... et du coup en 2003, j'ai dit bon on se calme, c'était la sortie du Méta-Instrument 3, il y en a 10 qui ont été faits, voilà, et ceux qui ont été faits, ils sont partis que dans des labos, ils sont même pas partis... alors il y a des musiciens dans les labos, mais ils ne sont pas du tout partis pour des musiciens non-labos... (rires)... donc il y avait un demi-échec quand même, qui était, c'est trop compliqué ou pas adapté, il faut programmer, ça prend du temps, enfin bref... donc il faut travailler autrement, et c'est là que l'idée de la Méta-Mallette est arrivé en disant on va d'abord travailler avec des interfaces du commerce, des gamepads, des joysticks, des tablettes, des trucs comme ça... et puis du coup, ça va donner envie aux gens, une fois qu'ils seront très forts sur leurs machins, de prendre une interface de haut-niveau et de travailler sur du geste expert ou raffiné... en tout cas avec plus de nuances possibles, quoi... et donc si on refait le Méta-Instrument aujourd'hui c'est un peu boucler la boucle, on est revenu en 2003 mais après avoir fait le parcours de la Méta-Mallette et ... en disant bon la Méta-Mallette, maintenant elle est en ligne, on va publier une vingtaine de projets, même plus cette année, que les gens pourront télécharger, essayer, etc. On fabrique des mono-manettes maintenant, qui vont être diffusées à 10000 exemplaires à la rentrée, donc ça c'est jamais arrivé... c'est un signe, ça ne va pas révolutionner le monde, mais c'est un signe d'évolution... je pense que la notion de geste est rentrée dans la problématique musicale, ce qui n'était pas le cas, beaucoup moins le cas avant... les gens n'y croyaient pas, etc. donc l'idée de revenir là c'est de dire qu'il faut arriver à porter le répertoire qui existe et le même travail qui a été fait sur la Méta-Mallette, le faire sur le Méta-Instrument , c'est-à-dire petit à petit publier des instruments, que les gens puissent récupérer, prendre, jouer, développer, etc. Il reste encore un petit peu de travail... 

% VG — et du coup, si je reprend le début de ma question, tu organises tes instruments virtuels par mouvement, mais tu pourrais utiliser un même instrument pour jouer plusieurs mouvements différents, ce qui n'est pas tellement le cas... 

% SdL — si... donc effectivement, quand j'ai développé pour la Méta-Mallette, j'ai moins développé pour le Méta-Instrument , et donc aujourd'hui je recommence une pièce pour Méta-Instrument , sans rien d'autre, qui je pense sera prête dans six mois, et je fais des concerts mixtes Méta-Mallette/Méta-Instrument... donc il y a une forme qui est en développement actuellement que je joue dans les parcs et jardins et qui me fait travailler ça... Je pioche dans mon répertoire de Méta-Instrument et j'ajoute autre chose, c'est des instruments en mue un petit peu... et c'est l'avantage de ces instruments, on peut les faire évoluer, les faire vivre... une fois que ce sera porté, je vais essayer beaucoup d'instruments, d'une part d'en faire des nouveaux et sur ceux qui existent d'apporter de la souplesse, et que ça puisse jouer facilement sur différentes musiques... 

% VG — tu parlais des touches pour la nouvelle Méta-Mallette, les touches que vous avez développés pour le Méta-Instrument 4, c'est les mêmes touches que vous distribuez dans les collèges? 

% SdL — non, dans les collèges en fait ils filent des tablettes avec clavier, dans le Val de Marne... en fait c'est des PC avec écran tactile... mono-tactile, mais bon voilà c'est déjà ça... en fait on utilise toutes les touches du clavier pour jouer, il doit y avoir une soixantaine de touches quand même, ça laisse pas mal de fonctions possibles, plus l'écran tactile, donc ça fait potentiellement des instruments intéressants... là on a fait une version un peu scratch, transformations, où les gens peuvent ramener leurs sons... on essaie de faire quelques banques de sons pour alimenter la pompe au moins au démarrage.... Et du coup de développer des orchestres avec ça... c'est assez rigolo... potentiellement, il y a moyen de jouer et de faire des trucs sympas et les gamins ont ça en fait, ils se promènent, ils vont à l'école avec ça, ils ont ça à la maison... donc là, c'est un peu par hasard que les trucs se sont fait, mais il y avait un prof tonique de collège, qui a dit ah mais on pourrait, votre projet Méta-Vox — qui est un peu une usine à gaz, on pourrait peut-être le mettre sur le... donc là je me suis dit il raconte n'importe quoi, et puis ... quand même c'est pas idiot ce qu'il dit et en réfléchissant, je me suis dit on peut peut-être, faut que je regarde ces tablettes, et le département sympa a dit on vous en prête une et du coup j'ai ramené des bouts de trucs audio-visuels et ça tournait, c'était fluide... je me suis dit merde quand même c'est puissant les tablettes aujourd'hui, on peut faire des sacrés trucs avec ça, donc bon allons y, on développe un petit truc et du coup on y a passé, comme d'hab, un peu plus de temps que prévu mais... on l'a testé avec des gamins qui du coup s'amusent quoi... donc ça c'est quand même le premier critère... on commence par écouter les sons, enregistrer leurs voix, ils sont morts de rire quand ils entendent la voix des copains, scratcher, accélérer, enfin tous les trucs classiques quoi... mais tout de suite il y a le geste hip-hop, banlieue, voilà... c'est drôle mais on voit les gamins quand même émus, pas tous, mais on en voit quand même qui, en sortant de la séance d'expérimentation, ont dit on en veut sur notre portable... donc on a dit on va publier, ce sera gratuit ... bientôt... donc il y a une envie... et je pense qu'une des dimensions qui est importante c'est quand même la dimension d'instrument, c'est-à-dire de ne pas trop réfléchir quand on joue... c'est les gestes qui pilotent, enfin les gestes, l'oreille qui se connectent... donc quand on connaît son instrument, on fait pas mal de trucs... et là l'idée c'est de faire un orchestre, des orchestres... bon, je ne suis pas sûr qu'on y arrive... 

% VG — et par rapport à ce que tu disais tout à l'heure sur la volonté au moment où la Méta-Mallette est arrivée à Puce Muse de mettre ça dans des écoles de musique, pour que des amateurs puissent pratiquer avec un instrument qu'on pourrait appeler instrument d'étude un peu, bon marché et avec des capacités limitées, mais qui permet de pratiquer quand même, pour après passer sur un instrument comme le Méta-Instrument qui serait plus un instrument professionnel... sauf qu'il y a quelque chose de singulier par rapport aux instruments acoustiques quand même c'est que l'ergonomie de l'objet n'est pas du tout la même, il y a des grosses différences de prise en main... 

% SdL — oui 

% VG — qui rejoignent un peu ces questions de méta-morphisme des instruments numériques, où ce qu'on a sous les doigts selon ce qu'on branche sur la machine peut changer du tout au tout... 

% SdL — oui... et l'inverse... on peut changer les accès, ce qui est derrière joue aussi... donc si on fait de la modulation de fréquence avec un clavier de Méta-Instrument , un launchpad ou je ne sais pas quoi, il y a quand même la modulation de fréquence qui est derrière... et c'est un petit peu cette idée là qui était dans la Méta-Mallette, c'est-à-dire il faut démystifier un certain nombre de données, une réverb c'est une réverb, alors il y a différents algo etc, mais une réverb c'est une réverb et une fois qu'on a compris les usages musicaux de ça, on peut très bien du gamepad, passer au launchpad ou passer à je ne sais quel truc... et aujourd'hui on continue, on fait un peu le forcing dans les conservatoires, on va peut-être monter en 2019 un festival avec 7 ou 8 écoles de musique et une école d'art qui se joindrait...pour le moment ils ont dit oui, il faut encore arriver à trouver un peu de sous, mais en tout cas je pense que ça a du sens et puis c'est aussi développer une sensibilité par rapport à ça, c'est-à-dire que, l'écoute... aujourd'hui le solfège est quand même très mal fait... on peut le dire... ça s'appelle formation musicale, ça partait d'une bonne intention je pense... il y a une partie énorme de la musique qui est inexplorée et cette sensibilité au son et cette culture du son est totalement inexplorée dans les écoles de musique ce qui est ... c'est incroyable quoi... c'est presque renversant de voir la lenteur et l'inertie de ce truc là... et donc aujourd'hui on essaie de trouver des passerelles pour que les différents acteurs des conservatoires se disent moi aussi je peux m'y mettre ... c'est-à-dire je pense que si c'est si peu utilisé par les écoles de musique c'est parce qu'il y a aussi une inquiétude, une inquiétude de profs qui diraient oui mais alors moi j'ai travaillé vingt ans pour apprendre mon instrument et puis alors maintenant on va me dire que c'est pas la peine, donc je sais rien, je repars à zéro et puis l'informatique ça me fait chier... bon... d'accord... mais il y a aussi des réjouissances possibles pour un prof d'instrument de découvrir un certain nombre d'usages et donc le premier outil qui finalement se développe pour nous, pas mal là, c'est un outil de connaissance du son, c'est-à-dire à partir de VuSon (un instrument de visualisation du son dans la Méta-Mallette, NdE) et ses extensions... ne serait-ce que ça, c'est-à-dire je branche VuSon, différentes visualisations du son, et je joue... et je me rends compte ce que je n'ai jamais compris, donc tout à coup je découvre ce que c'est des harmoniques, je découvre que quand je joue fort en général j'ai plus d'harmoniques que quand je joue doucement, que je joue souvent pas très juste, que le son n'arrête pas de bouger, que ... voilà, plein de choses comme ça, et ensuite ce qui est assez étonnant, c'est que c'est un usage souvent très désinhibant, c'est-à-dire que les gens, les gamins, quand on met un micro et la voix dans le truc ou l'instrument, tout à coup ils se lâchent pour explorer toutes les formes qu'ils peuvent faire et c'est étonnant... je l'ai vu avec des malentendants et avec des entendants, il y a un truc... et après il y a pas mal de profs qui m'ont fait ce retour là en disant mais il y a des gamins qu'on entendait pas qui tout d'un coup renversaient la table, juste par l'arrivée de la vue et comprendre mieux ce que je fais avec ma voix et avec mon instrument... Donc ça c'est la passerelle la plus simple et l'année prochaine on travaille pas mal sur greffer des capteurs sur des instruments pour appendre à transformer son instrument... 

% VG — Sur instruments acoustiques? 

% SdL — instruments acoustiques, pardon, oui... instruments du conservatoire pour jouer avec ça et comprendre que c'est important de maîtriser aussi le son transformée et de jouer avec et un musicien c'est aussi quelqu'un qui peut proposer des sons, proposer des modes de jeu sur son instrument... encore plus sur des musiciens professionnels, là on continue à travailler avec Versailles (le conservatoire, NdE) mais on va avoir plus d'heures, plus d'élèves, pour des futures professionnels... pour que petit à petit ils arrivent à mettre leurs capteurs, à transformer pour jouer avec et intégrer ces gestes... c'est tout bête mais... intégrer le geste de transformation, on doit l'entendre, quand on joue on doit l'entendre... et très souvent c'est long... je ne m'en rendais pas compte, mais maîtriser un travail sur une pédale c'est long... J'attrape, j'enlève le son, ou j'attrape le son je n'attrape plus le son mais le son est en boucle... rien que ça, je prends ça, je soulève, je prends là, donc je sais que j'ai pris ce son là qui continue à tourner et pendant ce temps là je fais autre chose... voilà, ça c'est des gestes qui doivent être intégrés... si c'est pas intégré ça ne joue pas... c'est tout raide... si c'est un ingé-son qui est à côté qui fait le truc, ça ne marche pas non plus, faut que ce soit les instrumentistes qui fassent... et là, presque rien n'est fait donc il y a beaucoup à inventer quand même 

% VG — Il y a quelque chose qui, malgré la modularité des interfaces et des accès qui ne sont jamais pareils, qui selon toi se retrouve dans le fait de connaître la cuisine de tout ça... 

% SdL — oui oui... 

% VG — qu'indépendamment de l'interface qu'on utilise, que cela soit une pédale, un gamepad, le fait de savoir comment fonctionne une réverb ou un harmoniseur... 

% SdL — et il y a l'oreille... ça c'est vraiment important... des modes de jeu possibles... et plus si affinité, on peut imaginer que tout le monde ne jouent pas du Méta-Instrument mais qu'il y en ait quand même plus que dix qui soient fabriqués et qui aillent dans des labos... 

% VG — c'est penser le modulaire de l'instrument... 

% SdL — oui oui... j'étais un peu psycho-rigide sur l'interface, en dehors du Méta-Instrument , point de salut... je caricature mais qu'à moitié... et donc aujourd'hui je ne pense )pas ça, effectivement il y a plein de choses qui jouent sur la définition de l'instrument et c'est intéressant d'avoir... il y a des universaux... la modulation de fréquence, voilà, ça sonne comme ça et si tu modules un sinus dans un autre en fréquence, tu vas avoir un certain nombre de propriétés mais tu le reconnais à l'oreille, enfin, une fois que tu l'as pratiquée tu l'as à l'oreille, donc en fonction de ce que tu as envie de faire tu vas prendre tel ou tel territoire, si tu veux des touches, comment tu les veux et tu vas assembler tout ça quoi...Là où ça se complique un peu, c'est cette idée de répertoire... voilà... je sens bien qu'il y a quelque chose qui est important là... mais je pense que je ne le formule pas bien... c'est-à-dire je pense qu'il y a quelque chose, en tout cas il faut qu'il y ait du partage et ... de l'objet artistique partagé... il faut que ça existe ça, sinon il y a un truc qui ne va pas... et donc qu'est ce que c'est un objet artistique avec ça, qu'est ce qu'on transmet, qu'est ce qui fait sens, œuvre, là-dedans... c'est une réponse compliqué mais bon on y va on verra bien... ou on verra pas... 

% VG — cette modularité des accès permet aussi de faire rentrer dans un instrument des choses qui sont complètement extérieur à son propre geste aussi, que cela soit dans le jeu collectif, on voit des instruments paraître où on est plusieurs voire très nombreux à jouer en même temps sur le même objet, et en extrapolant on voit bien qu'on peut rentrer n'importe quel type de donnée dans n'importe quelle variable avec des logiciels comme Max et qu'on pourrait mettre la météo ou le cours de la bourse qui va contrôler un paramètre... il y a quelque chose de singulier dans le fait que les branchements d'un instrument sont ouvertes sur l'extérieur... 

% SdL — Il y a toujours des passerelles entre les frontières, beaucoup plus qu'avant... et du coup il y a des difficultés à tracer des paramètres, parce que... c'est ce que tu dis là... 

% VG — tu as fait des concerts participatifs où les gens intervenaient avec des micros... 

% SdL — des micros, des manettes, des lumières... Est-ce que c'est un piano à 30 mains? 

% VG — tu penses que c'est quelque chose de cet ordre là? Comment tu vois ça? 

% SdL — oui, en tout cas il y a ça.. et ce n'est absolument pas dans l'air du temps... c'est assez étonnant mais peut-être que cela va revenir mais... par exemple, on l'a fait avec les manettes où on mettait un haut-parleur quasiment par joueur pour que chacun puisse s'entendre et déjà c'était compliqué, les gens avaient du mal à reconnaître ce qu'ils faisaient donc après je me suis dit, je suis con, il suffit de mettre un micro par personne et après, sa voix c'est sa voix quoi... mais en fait pas du tout, c'était un peu débile, mais même sa voix, quand il y a plein de gens qui chantent autour on l'entend vaguement... c'est difficile de s'entendre dans un chœur par exemple, c'est pour ça qu'on chante comme ça (couvrant son oreille d'une main, NdE), il faut énormément intérioriser sa voix pour que quand on chante quelque chose, sans l'entendre, savoir que cela va sonner comme ça. Et en même temps, un beau chœur c'est un chœur qui fusionne... donc il y a un moment, on entend ça dans la musique, des fusions qui sont hallucinantes et qui fait l'un des graal qu'on poursuit, être ensemble, complètement ensemble, n'être qu'un... donc un piano à quarante mains ... ce qui est difficile mais quelque part il y a un peu une dépersonnalisation ce qui est évidemment inquiétant (rires), et on voit très bien les écueils de ça... politique en tout cas, très vite, l'histoire est chargée là-dessus... en même temps cette envie de fusion et en même temps une méfiance par rapport à la fusion qui est un hyper-individualisme, et actuellement on n'est pas du tout dans la fusion mais dans l'hyper-individualisme... on a son téléphone, son machin... on partage pas 

% VG — c'est politiquement l'inverse d'une culture de la rock-star 

% SdL — il me semble que la star marche assez bien c'est le soliste finalement, c'est le super soliste qu'on va aduler, alors que la fusion c'est faire ensemble un truc 

% VG — oui, je disais que c'est contradictoire avec l'air du temps qui est plutôt à glorifier le soliste... 

% SdL — oui, absolument...le soliste c'est le soi, c'est The Voice où tout le monde peut devenir le meilleur chanteur du monde, etc. Je n'ose pas trop en parler, car tout le monde m'en a parlé, mais je n'ai pas vu le truc mais on sent qu'il y a une mécanique comme ça chez les gamins quand on pratique un peu avec eux, ils ont l'impression qu'ils vont direct être les meilleurs du monde... alors oui, c'est bien que tu aies cet objectif là, mais il faut travailler un petit peu... «ah non regarde, j'y arrive»... après on va rentrer un peu loin... en même temps la politique et la musique c'est assez proche, bref... c'est plus politique... 

% VG — une autre chose... par rapport aux instruments acoustiques pour lesquels le matériau a une complexité qui nous échappe, alors que quand on programme un instrument, il y a une intention très intellectuelle et je me demandais comment tu abordes ce désir, s'il existe, de ré-introduire de l'aléatoire, de la surprise, dans les instruments numériques, qui ont cette phase de conception éminemment intellectuelle et relativement contrôlée 

% SdL — mm....ouais... ça veut dire que tu programmes bien! (rires) Je trouve que des surprises, il y en a tout le temps et souvent l'aléatoire ne sonne pas.... J'ai des instruments sur lesquels on peut doser la quantité d'aléa, tout ça... et je suis souvent très déçu... 

% VG — pas forcément de l'aléatoire, mais de l'imprévisible...qui peut passer par des choses instables... 

% SdL — il y en a beaucoup... là tu vois, en travaillant sur les touches, ne serait-ce que ça, je me dis merde quand même, je connais ça, depuis le temps... et puis je pars sur une idée et je joue et c'est nul... alors qu'est ce que j'avais en tête? Et d'arriver à revenir à l'idée, le point de départ... l'objet qui me semblait faire sens au début... donc là, ces derniers jours, ça m'est arrivé encore un paquet de fois... je trouve que c'est vachement compliqué de trouver le bon geste pour qu'à la fin ça paraisse naturel... facile... facile et dans lequel tu es bien partout, tu es bien piano, tu es bien forte, tu est bien quelle que soit la position de ton instrument, de pression des autres doigts etc. que ça claque ... ça c'est dur... pfff.... C'est très dur... 

% VG — c'est peut-être cette ergonomie qui intègre cette dose d'instabilité confortable... 

% SdL — c'est plus que de l'egonomie...c'est vraiment, pour moi, c'est de la musique... 

% VG — je partage assez ton avis que lorsqu'on met du jitter, du bruit, généralement ça ne marche pas bien 

% SdL — oui c'est un cache misère un peu... c'est comme quand tu appuies sur la pédale du piano tout le temps... 

% VG — et à l'inverse il y a des choses comme tous les mécanismes de feedback qui ont tendance à générer de l'instabilité, souvent fonctionne bien dans le résultat sonore mais sont difficile à ... 

% SdL — à jouer ... 

% VG — à jouer, à maîtriser... 

% SdL — à la fin il faut un instrument riche qui propose beaucoup, qui envoie généreusement, et que tu puisses reprendre quand tu veux comme tu veux... et ça c'est difficile... 

% VG — dans le fait de reprendre les choses, il y a à la fois le fait de pouvoir reprendre quelque chose d'instable mais il y a aussi une mécanique à l'œuvre dans les instruments numériques, une délégation de toute une partie de la production sonore, que cela soit au niveau microscopique du son lui même ou à un niveau plus haut de patterns, et il y a un endroit crucial et délicat qui est cette passation de pouvoir, si on peut dire, entre son propre geste et le geste de la machine, le geste enregistré ou algorithmique... quelque part, il y a presque un geste de ça... par exemple tu parlais tout à l'heure d'enregistrer une boucle et que la boucle est maintenue par la machine et pouvoir la reprendre... tu disais que c'est un geste à apprendre, de savoir faire ça... 

% SdL — En fait, oui, c'est un geste mais c'est surtout apprendre à entendre... c'est assez étonnant.. là en travaillant avec ces étudiants, souvent c'est des jeunes professionnels avec qui je bosse... c'est une pratique qui n'est tellement pas habituelle pour eux que c'est difficile à penser... alors effectivement du coup on peut faire un petit canon, en enregistrant une cellule, en jouant par dessus, tout ça... mais si on ne l'entend pas on ne peut pas y arriver... alors souvent les instruments sont monophoniques, pas tous mais beaucoup quand même, ça veut dire qu'il faut entendre en polyphonie du coup... et puis c'est des nouveaux gestes, sur des gens qui ont pas mal travaillé leur geste, c'est des nouveaux gestes... donc c'est compliqué... nouveaux gestes et nouvelle culture d'oreille, nouvelle manière d'entendre... ça ça prend du temps, c'est pas instantané 

% VG — et j'ai l'impression qu'il y a quelque chose qui est de l'ordre de la résonance... pour attraper le geste au bon moment, il faut savoir faire le geste mais aussi et surtout peut-être le faire au bon moment et ce moment, j'ai l'impression que ce n'est pas forcément tant un moment mesuré ou compté qu'un moment qui est senti, senti à travers l'oreille, on sent arriver un climax sonore ou quelque chose... 

% SdL — un phrasé... 

% VG — et une chose particulière aux instruments numériques par rapport aux instruments acoustiques c'est qu'on ressent assez peu la vibration acoustique via le corps de l'instrument... 

% SdL — ... oui... je pense que le premier truc c'est de savoir ce qui sort de son instrument, c'est-à-dire de pouvoir chanter ce qu'on joue, et au bon volume... c'est tout con, mais c'est compliqué parce que tout bouge dans un instrument, t'arrives dans une salle, tu as le son qui n'est pas réglé au même niveau donc il faut calibrer vite pour être sûr que lorsque tu joues ça sort comme ça au moment où tu joues... 

% VG — ...comme si tu le chantais... 

% SdL — oui, chanter parce que en général ta voix tu sais à peu près qu'elle va sortir comme ça, qu'elle va dire ça avec cette intonation... et souvent sur les instruments électroniques il y a plein de choses qui bougent sans arrêt, c'est difficile de retrouver, par exemple si je joue du Méta-Instrument , savoir le la, parce que tous les instruments sont accordés différemment, il y en a qui n'ont pas de la, il y en a qui en ont, pas sur les mêmes tonalités, bref... par contre si je fais (mimant le geste de frapper un diapason et de le porte à son oreille, NdE) je pense que c'est bon parce que j'ai l'impression de sentir le diapason, je refais le geste et le geste de cogner le diapason et de l'écouter c'est toujours le même... donc au bout d'un moment si on l'a fait suffisamment il suffit de refaire le geste et automatiquement on entend le son... ce qui est difficile sur les instruments électroniques parce que le même geste va déclencher des sons différents, donc il y a vraiment des espaces mentaux à déplacer, c'est vraiment une grosse difficulté, ça veut dire qu'il faut intégrer plein de manières de jouer, plein de comportements... 

% VG — et recoller les bouts entre ces différents territoires déconnectés... 

% SdL — recoller et puis basculer instantanément de l'un à l'autre qui est redoutable... c'est sûr que c'est une nouvelle forme de virtuosité... Je me souviens pendant un concert en 1992 ou 1993 avec le premier Méta-Instrument , on avait quelques pièces sur un concert avec Rémy Dury (co-fondateur de Puce Muse et auteur du Karlax, NdE) et sur un mouvement j'ai vu Rémy ailleurs, et je me suis dit mais qu'est ce qu'il fait... je joue et je me dis mais qu'est ce qu'il fait, qu'est ce qu'il fait... et il a eu un blanc d'un mouvement et effectivement la mécanique est toujours la même, on bascule instantanément et on a toujours le même objet dans le mains, mais qui fait complètement autre chose que ce qu'il faisait il y a 5 secondes... et en fait au mouvement suivant c'est reparti... et on a finit le concert très bien... mais il a eu un blanc d'un mouvement, il ne savait plus... Alors ça sur un piano, non... aïgu, grave, Do... ça ne change pas... 

% VG — pour finir cette discussion, aujourd'hui avec les évolutions en cours, les nouvelles technologies, qu'est ce qui pour toi manque, quels sont les endroits où tu te sens encore frustré par rapport à des choses que tu aimerais mais qui ne sont pas encore là... 

% SdL — la vitesse... 

% VG — la vitesse? 

% SdL — j'aimerais bien travailler dix fois plus vite que je ne travaille... 

% VG — le workflow comme on dit...? 

% SdL — Oui, il y a quand même beaucoup d'endroits... alors on apprend à donner du sens à ces gestes lents, mais je pense que potentiellement... le but serait qu'à tout moment on soit dans l'acte musical... quand on fait, quand on programme on devrait être, on rêve en tout cas d'être dans l'acte musical, et donc quand on s'en éloigne, ça c'est vraiment... On se rend compte souvent que ça ressemble à des fausses notes ou à des actes manqués parce qu'on pense de travers et qu'on n'a pas envie de penser droit probablement (rires)... mais quand même je pense qu'il y a un environnement de travail qui peut encore progresser beaucoup pour être dans cette espèce de pensée, au sens très large, de vie avec les sons... et de ne pas perdre ce fil là 

% VG — ne pas découpler une activité de lutherie et de ... 

% SdL — ...oui, d'écriture au sens très large du mot... lutherie et écriture c'est quasiment la même chose, c'est très proche... que l'écriture ait du sens autant que du direct... 

% VG — c'est très proche dans les instruments numériques... dans la facture acoustique, on forge le son mais c'est vrai que dans les instruments numériques qui intègrent des fragments composés, des patterns... 

% SdL — ... de manière de jouer les sons... oui, c'est sûr que c'est très proche 

% VG — ce serait d'avoir un logiciel plus adapté que Max qui est l'outil actuel... 

% SdL — oui, qui est quand même formidable... qui moi, m'a énormément ouvert la vie musicale... mais je pensais à ce geste de l'écriture, qui une pratique... ce qui est assez étonnant, quand t'as un problème un peu compliqué, tu prends un papier, un crayon, et ce qui se passe à ce moment là, tu as un geste d'écriture qui te permet de résoudre... quand tu commences à écrire, tu as un rythme d'écriture qui est un peu le rythme de la pensée, des fois il faut que tu écrives n'importe quoi, tu mets ça, te le mets à côté de ça... tu as un geste d'écriture, de lutherie, de composition, bref, ce geste très flexible d'élaboration il existe et donc je pense qu'il y aurait un truc à inventer par là pour être plus au cœur, le papier et le crayon c'est quand même fabuleux, c'est une grande grande invention... simplicité, économie, durabilité... c'est redoutable... et pour le moment pour moi c'est trop compliqué donc c'est pour plus tard... mais j'aimerais bien arriver à rester fluide tout le temps... je joue en concert, je me lève le matin, j'ai eu trois idées cette nuit que je rédige et qu'il n'y ait pas de discontinuité entre les deux... 

% VG — un casque à électrodes qui lit dans tes pensées et qui fabrique les instruments à ta place... 

% SdL — Peut-être ... il paraît qu'il y a des logiciels à résoudre les équations qui marchent à peu près bien pour la plupart des équations ... quand il y a des solutions...et effectivement je pense que le chantier qui est intéressant c'est plutôt la question, la bonne question... une fois bien posée, s'emmerder à y répondre, non... c'est où la réponse... c'est vrai qu'on peut être content d'avoir trouvé une réponse, résolu le truc mais je trouve que c'est surtout le porblème bien posé, bien analysé, bien senti... ah merde, faut le programmer!... C'est pas très fluide encore... on essaie, mais quand je vois ces problèmes de touche, j'ai l'impression que c'est simple, mais quand on me demande de dire, je ne sais pas... c'est ça qui est rigolo c'est que de temps en temps, c'est pas si simple... 

% VG — il y a une boite qui a sorti un logiciel, une petite box, à laquelle tu donnes un son acoustique, tu enregistres un son et cela va automatiquement, via des algorithmes génératifs, chercher les bons paramètres d'une synthèse, là c'était sur un DX7, sur une synthèse FM, ça va aller chercher les paramètres qui épousent au mieux le son que tu as enregistré... je ne suis pas sûr pour autant que cela soit la réponse à tout, parce que la compréhension de la mécanique interne de l'algorithme que tu manipules a ses petits points d'accroche qui sont intéressants... 

% SdL — pour moi elle est essentielle... là dessus, j'ai beaucoup de mal à accepter l'IA... je n'ai aucun plaisir dans le truc... quel est le modèle et quel est le sens des choses, notamment en musique je pense qu'il y a un paquet de musique qui sont fait autour de la gravitation, du souffle... après on peut connecter bêtement, on envoie 20000 itérations et l'ordinateur trouve... et puis après il fait... mais on ne comprend pourquoi il le fait (rires)... pour moi ça n'a aucun intérêt... 

% VG — il faut pouvoir aller assez vite dans l'implémentation des algorithmes mais à la fois le chemin que tu prends n'est pas dénué d'intérêt... 

% SdL — ... c'est ça... absolument... 

% VG — il y a une balance entre les deux... 

% SdL — complètement... et du coup c'est souvent l'échec qui est le plus intéressant... 

% VG — ça me semble une bonne conclusion... 

% SdL — ça n'était pas l'objectif de conclure! (rires) 

