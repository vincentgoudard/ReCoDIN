% !TEX root = ../thesis-example.tex
%
\chapter*{Remerciements}
\label{sec:acknowledgement}
\vspace*{-10mm}
\pdfbookmark[0]{Remerciements}{Remerciements}


%J'aimerais remercier chaleureusement toutes les personnes qui m'ont permis de réaliser ce travail; elles sont malheureusement bien trop nombreuses pour toutes être citées ici...\\
\noindent Je tiens tout d'abord à remercier Jean-Dominique Polack et Pierre Couprie, mes directeurs de thèse, et tout particulièrement Hugues Genevois, mon ``co-encadrant'' --~bien étrange titre pour quelqu'un qui se plaît tant à faire sortir la pensée hors des cadres établis. Merci Hugues de m'avoir permis de poursuivre ces recherches ces dernières années, et à tous les trois de m'avoir laissé errer dans des directions diverses, avec une méthodologie souvent plus proche de l'improvisation que de la partition, tout en me faisant confiance dans ce cheminement sinueux!\\
\indent Je tiens à exprimer ma profonde gratitude aux membres du jury, d'avoir accepté d'examiner cette thèse: Brigitte d'Andréa-Novel, Myriam Desainte-Catherine, Laurent Pottier et Marcelo Wanderley, ainsi qu'à Catherine Pélachaud pour sa bienveillance lors du comité de suivi. Vos conseils avisés sont précieux pour la suite de ce travail.\\
\indent Je remercie chaleureusement le Collegium Musicæ, en particulier Cécile Davy-Rigaux, Benoît Fabre et plus particulièrement Agnès Puissilieux, pour sa dynamique et agréable compagnie dans notre bureau commun au LAM. Au-delà du soutien financier que m'a apporté le Collegium pour ces trois années de recherche, les séminaires rassemblant des chercheu.rs.ses de ses différentes composantes ont été, chaque fois, des moments de rencontre très enrichissants dans cette perspective d'inter-disciplinarité, qui sous-tend le présent travail. C'est bien souvent au détour d'une conversation, si anecdotique parût-elle, avec des personnes extérieures à son domaine de recherche, que des connexions s'opèrent que l'on cherchait vainement ailleurs, ou que des théories que l'on croyait solides montrent leurs failles. C'est un plaisir d'avoir pu travailler dans un contexte où l'inter-disciplinarité est ainsi encouragée.\\
\indent Mes remerciements vont également à mes collègues de ONE, pour le plaisir à jouer avec eux autant que pour ce qu'ils ont apporté aux réflexions et à l'élaboration de certains outils : Pierre et Hugues qui, en plus de superviser ce travail de recherche, ont été des partenaires de jeu, Laurence Bouckaert, Jean Haury, György Kurtág Jr. dont la rencontre il y a maintenant plus de quinze ans a largement contribué à la poursuite de certaines intuitions et Serge de Laubier, pour son inspirante habileté à emmener les lutheries numériques, les musiques expérimentales et les questions théoriques qu'elles posent sur les terrains très concrets du spectacle de rue, ou à confronter les aspects techniques de la latence informatique à l'impatience d'un groupe d'enfants.\\
\indent Je remercie également ceux qui m'ont généreusement offert leur temps et leurs idées durant les entretiens : Nicolas Bernier, Nicolas Collins, Serge de Laubier, François Dumeaux, Adrien Mamou-Mani, Luca Turchet, Bruno Zamborlin, Patrick Saint-Denis et José-Miguel Fernandez. Merci également à Pascale Criton, Muriel Colagrande, Maurin Donneaud, Guillaume Evrard, Thor Magnusson, Rémy Müller, Bernard Sève, Paul Stapleton (et quelques autres!) pour les discussions informelles sur les lutheries numériques qui ont également nourri cette recherche, ainsi qu'à mes collègues du LAM (en particulier le groupe ``nouvelles lutheries''): Christophe d'Alessandro, Michèle Castellengo, Louise Condi, Boris Doval, Claudia Fritz, Jean-Marc Fontaine, Manuel Gaulhiac, Jean-Théo Jiolat, Brian Katz, Jean-Loïc Le Carrou, Grégoire Locqueville, Thomas Lucas, Marie-France Mifune, David Poirier-Quinot, Laurent Quartier, Benoît Navarret, David Théry, Xiao Xiao, avec des remerciements particuliers à Gabriela Patiño-Lakatos, pour son art de (me) poser de bonnes questions. Merci à Simona Otarasanu, Sandrine Bandeira, Pierre-Yves Lagrée et Manuel Mayer pour leur soutien lorsqu'il s'est agi de trouver en urgence une salle où jouer une performance audiovisuelle la veille de ma soutenance, en pleine période de grève.\\
\indent Merci à Dušan Barok et Alexandra Elbakyan pour la diffusion des ressources nécessaires lors d'un travail de recherche, auxquelles il est parfois difficile d'accéder dans un monde académique scindé en spécialités. Enfin, c'est une évidence qui mérite d'être explicitée, je suis redevable à toutes les personnes citées dans ce manuscrit, même lorsque leurs idées y sont critiquées, d'avoir nourri ma propre réflexion sur ce sujet. Je salue donc ici la leur, à laquelle j'espère humblement participer.\\
\indent Un immense merci aux ami.e.s qui m'ont accueilli durant mes nombreux aller-retours à Paris: Ingrid, Maï, Guillaume, Karolina, Hugues, Marylise, Guillaume, Hilla, Gaëlle, Ignazio, Fabrizia, Jérome, Agathe, Agnès, Charlotte, Fred; merci à Bertrand Gibert, pour le volume de Leroi-Gourhan qu'il m'a offert au début de ce doctorat, je l'ai bien usé! Merci à Agnès Gallet, Juliette Danjon et François Goudard pour leur relectures et regards extérieurs. À toutes celles et ceux que j'ai oubliées dans cette liste incomplète, qui ont soutenu ce travail, et enfin, un remerciement chaleureux aux collègues et ami.e.s qui, à ma grande surprise, ont fait de moi l'heureux possesseur d'un anthologique synthé analogique à la fin de ma soutenance!\\
\indent À ma famille, pour leurs soutien et encouragements; à Olga et Anatole, pour leur patience autant que pour leur impatience et leur curiosité stimulante; à Gladys Brégeon, pour tout ce qu'elle sait.
\begin{flushright}
Vincent Goudard, 20 février 2020.
\end{flushright}