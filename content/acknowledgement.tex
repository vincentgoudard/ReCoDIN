% !TEX root = ../thesis-example.tex
%
\pdfbookmark[0]{Acknowledgement}{Acknowledgement}
\chapter*{Remerciements}
\label{sec:acknowledgement}
\vspace*{-10mm}


% Je ne pourrai malheureusement pas remercier ici toutes les personnes qui m'ont aidé à réaliser ce travail, dont les origines remontent trop loin pour qu'elles soient vraiment 


% Je tiens à remercier chaleureusement toutes les personnes qui m'ont permis de réaliser ce travail. Au risque de ne pas pouvoir toutes les citer ici, je remercie en particulier mes directeurs de thèse Jean-Dominique Polack, Pierre Couprie et tout particulièrement mon ``co-encadrant'' Hugues Genevois, un bien étrange titre pour quelqu'un qui se plait tant à faire sortir la pensée hors des cadres établis. 

% Merci Hugues, ta confiance tout autant que ton ouverture d'esprit ont été des soutiens 

Également, je remercie le Collegium Musicæ, notamment Cécile Davy-Rigaux, Benoit Fabre et plus particulièrement Agnès Puissilieux, dont j'ai partagé le bureau durant mes présences au LAM. Au delà du soutien financier que m'a apporté le Collegium en m'accordant sa confiance pour ce travail, l'implication aux séminaires rassemblant des chercheu.rs.ses de ses différentes composantes ont été chaque fois des moments très enrichissants dans cette \ul{poursuite} d'inter-disciplinarité. 

Cette inter-disciplinarité n'est pas qu'un terme à la mode, mais se construit à travers des rencontres que le Collegium Musicæ suscite.

% Catherine Pélachaud pour sa bienveillance lors du suivi de cette thèse.

% Il me faut également remercier tous mes collègues de ONE pour le plaisir à jouer avec eux autant que pour ce qu'il ont pu apporter aux réflexions et à l'élaboration de certains outils comme ``John'' : en dehors de Pierre et Hugues qui outre le fait d'encadrer ce travail de recherche ont été des partenaires de jeu, Laurence Bouckaert, Jean Haury et tout particulièrement, György Kurtág Jr. dont la rencontre il y a maintenant plus de quinze ans a été décisive dans le fait de poursuivre certaines intuitions et Serge de Laubier, auprès de qui les quelques années passées à Puce Muse m'ont permi d'apprécier son talent à transposer les lutheries numériques, les musiques expérimentales et les questions théoriques qu'elles posent, sur des terrains très concrèts, d'un spectacle de rue un soir de décembre sous la pluie, ou d'une groupe d'enfants dont l'impatience tourne à fréquence plus élevée que le CPU des ordinateurs high-tech.

% Je remercie également ceux qui ont accepté d'être interviewés Nicolas Bernier, Nicolas Collins, Serge De Laubier, François Dumeaux, Adrien Mamou-Mani, Luca Turchet, Bruno Zamborlin, Patrick Saint Denis, José-Miguel Fernandez, ainsi que celles et ceux avec qui les discussions informmelles ont richement nourri ce travail, même si ces discussions n'y figure pas explicitement : György Kurtág Jr., Marcelo Wanderley, Caroline Traube, Bernard Sève.

% Je remercie également les chercheurs et chercheuses du LAM, mon laboratoire d'accueil,  qui m'a accueilli durant mes venues occasionelles, 

Pascale Criton, Cécile Babiole

% Je tiens à remercier chaleureusement les ami.e.s qui m'ont accueillis durant mes nombreux déplacements parisiens : Ingrid, Guillaume et Karolina, Hugues, Guillaume et Hilla, Gaëlle, Ignazio, Jérome et Agathe.

% Gibert père

% Hugues Genevois, Pierre Couprie, Jean-Dominique Polack, Catherine Pélachaud, LAM

% Agnès Puissilieux et CM

% Interviewés (Nic Collins, Adrien Mamou-Mani, Bruno Zamborlin, François Dumeaux, Serge De Laubier, György Kurtág Jr., Luca Turchet, Nicolas Bernier, Patrick Saint Denis, Marcello Wanderley, Caroline Traube,  Bernard Sève, José-Miguel Fernandez, Rémy Muller, Thor Magnusson, ...)

% Tous ceux rencontrés et collaborations (sans citer explicitement).

% Les ami.e.s qui m'ont hébergés durant mes passages à Paris durant toutes ces dernières années, en particulier Guillaume avec qui les discussions ont nourri

% G, A, O.

% et pour l'accès à toutes les ressources auxquelles il est parfois difficile d'accéder dans un monde académique scindés en spécialités : Dušan Barok, Alexandra Elbakyan.