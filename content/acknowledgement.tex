% !TEX root = ../thesis-example.tex
%
\chapter*{Remerciements}
\label{sec:acknowledgement}
\vspace*{-10mm}
\pdfbookmark[0]{Remerciements}{Remerciements}


J'aimerais remercier chaleureusement toutes les personnes qui m'ont permis de réaliser ce travail; elles sont malheureusement bien trop nombreuses pour toutes être citées ici et je me restreindrai aussi à celles et ceux qui ont plus directement soutenu cette recherche.\\
\indent Je tiens tout d'abord à remercier Jean-Dominique Polack et Pierre Couprie, mes directeurs de thèse, et tout particulièrement Hugues Genevois, mon ``co-encadrant'' --~bien étrange titre pour quelqu'un qui se plaît tant à faire sortir la pensée hors des cadres établis. Merci Hugues de m'avoir permis de poursuivre ces recherches ces dernières années, et à tous les trois de m'avoir laissé errer dans des directions diverses, avec une méthodologie souvent plus proche de l'improvisation que de la partition, tout en me faisant confiance dans ce cheminement sinueux!\\
\indent J'adresse également mes remerciements aux membres du jury, qui ont accepté d'examiner ce travail: Brigitte d'Andréa-Novel, Myriam Desainte-Catherine, Laurent Pottier et Marcelo Wanderley, ainsi qu'à Catherine Pélachaud pour sa bienveillance lors du suivi de thèse. Vos conseils avisés sont précieux pour la poursuite de ce travail!\\
\indent Je remercie chaleureusement le Collegium Musicæ, en particulier Cécile Davy-Rigaux, Benoît Fabre et plus particulièrement Agnès Puissilieux, avec qui j'ai eu plaisir à partager notre bureau durant mes présences au LAM. Au-delà du soutien financier que m'a apporté le Collegium pour ces trois années de recherche, l'implication aux séminaires rassemblant des chercheu.rs.ses de ses différentes composantes ont été, chaque fois, des moments de rencontre très enrichissants dans cette perspective d'inter-disciplinarité, qui sous-tend le présent travail tout comme, de manière plus générale, le travail du Collegium. C'est bien souvent au détour d'une conversation, si anecdotique parût-elle, avec des personnes extérieures à sa propre recherche, que des connexions s'opèrent que l'on cherchait vainement ailleurs, ou que des théories que l'on croyait solides montrent leurs failles. C'est un plaisir d'avoir pu travailler dans un contexte où l'inter-disciplinarité est ainsi encouragée.\\
\indent Mes remerciements vont également à mes collègues de ONE, pour le plaisir à jouer avec eux autant que pour ce qu'ils ont apporté aux réflexions et à l'élaboration de certains outils : Pierre et Hugues qui, en plus de superviser ce travail de recherche, ont été des partenaires de jeu, Laurence Bouckaert, Jean Haury, György Kurtág Jr. dont la rencontre il y a maintenant plus de quinze ans a largement contribué à la poursuite de certaines intuitions et Serge de Laubier, auprès de qui les quelques années passées à Puce Muse m'ont permis d'apprécier son talent à emmener les lutheries numériques, les musiques expérimentales et les questions théoriques qu'elles posent sur les terrains très concrets du spectacle de rue, ou de confronter les aspects techniques de la latence informatique à l'impatience d'un groupe d'enfants.\\
\indent Je remercie également chaleureusement ceux qui m'ont généreusement offert leur temps et leurs idées durant les entretiens : Nicolas Bernier, Nicolas Collins, Serge de Laubier, François Dumeaux, Adrien Mamou-Mani, Luca Turchet, Bruno Zamborlin, Patrick Saint-Denis et José-Miguel Fernandez. Merci également à Pascale Criton, Guillaume Evrard, Rémy Müller, Bernard Sève, Paul Stapleton, Ignazio Trama (et quelques autres!) pour les discussions informelles sur les lutheries numériques qui ont également nourri ce travail, ainsi qu'à mes collègues du LAM (en particulier le groupe ``nouvelles lutheries''): Christophe d'Alessandro, Michèle Castellengo, Louise Condi, Boris Doval, Claudia Fritz, Manuel Gaulhiac, Jean-Théo Jiolat, Jean-Loïc Le Carrou, Grégoire Locqueville, Thomas Lucas, Benoît Navarret, Xiao Xiao, avec des remerciements particuliers à Gabriela Patiño-Lakatos, pour son talent de psychologue à (me) poser de bonnes questions. Merci à Simona Otarasanu, Sandrine Bandeira, Pierre-Yves Lagrée, Manuel Mayer, Cécile Babiole, Emmanuel Ferrand pour leur soutien lorsqu'il s'est agi de trouver en urgence une salle où jouer une performance audiovisuelle la veille de ma soutenance, en pleine période de grève.\\
\indent Pour l'accès à des ressources nécessaires lors d'un travail de recherche, auxquelles il est parfois difficile d'accéder dans un monde académique scindé en spécialités : merci à Dušan Barok et Alexandra Elbakyan.\\
\indent Un immense merci aux ami.e.s qui m'ont accueilli durant mes nombreux aller-retours à Paris : Ingrid, Guillaume, Karolina, Hugues, Marylise, Guillaume, Hilla, Gaëlle, Ignazio, Fabrizia, Jérome, Agathe, Agnès, Charlotte, Fred; merci à Bertrand Gibert, pour le volume de Leroi-Gourhan qu'il m'a offert au début de ce travail, je l'ai bien usé! Merci à Agnès Gallet, Juliette Danjon et François Goudard pour leur relectures et regards extérieurs. À toutes celles et ceux que j'ai oublié.e.s dans cette liste incomplète et qui ont participé à ce travail.\\
\indent À ma famille, pour leurs soutien et encouragements; à Olga et Anatole, pour leur patience autant que pour leur impatience et leur curiosité stimulante; à Gladys Brégeon, pour tout ce qu'elle sait.
\begin{flushright}
Vincent Goudard, 17 février 2020.
\end{flushright}