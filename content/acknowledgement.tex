% !TEX root = ../thesis-example.tex
%
\chapter*{Remerciements}
\label{sec:acknowledgement}
\vspace*{-10mm}
\pdfbookmark[0]{Remerciements}{Remerciements}


J'aimerais remercier chaleureusement toutes les personnes qui m'ont permis de réaliser ce travail, mais elles sont malheureusement bien trop nombreuses pour être toutes citées ici, et je me restreindrai aussi à celles et ceux qui ont plus directement soutenu cette recherche.\\
\indent Je remercie en premier lieu mes directeurs de thèse Jean-Dominique Polack, Pierre Couprie et tout particulièrement mon ``co-encadrant'' Hugues Genevois --~bien étrange titre pour quelqu'un qui se plait tant à faire sortir la pensée hors des cadres établis. Merci à tous les trois de m'avoir laissé errer dans des directions diverses, avec une méthodologie parfois plus proche de l'improvisation que de la partition, tout en restant bienveillant à mon égard.\\
\noindent J'adresse également mes remerciements à aux membres du jury, qui ont accepté d'examiner ce travail, pour leurs conseils aussi avisés que bienveillants : Brigitte d'Andréa-Novel, Myriam Desainte-Catherine, Laurent Pottier et Marcelo Wanderley, ainsi qu'à Catherine Pélachaud pour sa bienveillance lors du suivi de thèse.\\
\indent Également, je remercie chaleureusement le Collegium Musicæ, notamment Cécile Davy-Rigaux, Benoit Fabre et plus particulièrement Agnès Puissilieux, dont j'ai eu le plaisir de partager le bureau durant mes présences au LAM. Au-delà du soutien financier que m'a apporté le Collegium en m'accordant sa confiance pour ce travail, l'implication aux séminaires rassemblant des chercheu.rs.ses de ses différentes composantes ont été, chaque fois, des moments de rencontre très enrichissants dans cette \ul{poursuite} d'inter-disciplinarité. C'est bien souvent au détour d'une conversation, si anecdotique parût-elle, que des connexions s'opèrent que l'on cherchait vainement ailleurs, ou que des théories que l'on croyait solides montrent leur failles. C'est un plaisir, en tant que chercheur, d'avoir pu travailler dans un tel contexte où l'inter-disciplinarité est promue et encouragée.\\
\indent Mes remerciements vont également à mes collègues de ONE, pour le plaisir à jouer avec eux autant que pour ce qu'ils ont pu apporter aux réflexions et à l'élaboration de certains outils : Pierre et Hugues qui outre le fait de superviser ce travail de recherche ont été des partenaires de jeu, Laurence Bouckaert, Jean Haury, György Kurtág Jr. dont la rencontre il y a maintenant plus de quinze ans a été décisive dans le fait de poursuivre certaines intuitions, et Serge de Laubier auprès de qui les quelques années passées à Puce Muse m'ont permis d'apprécier son talent à emmener les lutheries numériques, les musiques expérimentales et les questions théoriques qu'elles posent sur les terrains très concrets d'un spectacle de rue un soir de décembre sous la pluie, ou de confronter la latence informatique à l'impatience d'un groupe d'enfants.\\
\indent Je remercie également chaleureusement ceux qui m'ont généreusement offert leur temps et leurs idées durant les entretiens : Nicolas Bernier, Nicolas Collins, Serge De Laubier, François Dumeaux, Adrien Mamou-Mani, Luca Turchet, Bruno Zamborlin, Patrick Saint-Denis et José-Miguel Fernandez. Merci également à celles et ceux avec qui les discussions informelles ont nourri ce travail, même si ces discussions n'y figurent pas explicitement, notamment : Marcelo Wanderley, Caroline Traube, Bernard Sève, Pascale Criton, Rémy Müller, Thor Magnusson, Paul Stapleton, Enrique Tomás, ainsi qu'aux membres du LAM, en particulier le groupe ``nouvelles lutheries'': Christophe d'Alessandro, Michèle Castellengo, Louise Condi, Boris Doval, Claudia Fritz, Jean-Loïc Le Carrou, Grégoire Locqueville, Benoît Navarret, Xiao Xiao, avec des remerciements particuliers à Gabriela Patiño-Lakatos, pour son art de psychologue à (me) poser de bonnes questions. Merci à Simona Otarasanu, Sandrine Bandeira, Pierre-Yves Lagrée, Manuel Mayer, Cécile Babiole, Emmanuel Ferrand pour leurs conseils lorsqu'il s'est agi de trouver en urgence une salle où jouer la performance audiovisuelle avant la soutenance, en pleine période de grève.\\
\indent Pour l'accès à des ressources nécessaires lors d'un travail de recherche et auxquelles il est parfois difficile d'accéder dans un monde académique scindés en spécialités : Dušan Barok, Alexandra Elbakyan.
\indent Immense merci aux ami.e.s qui m'ont accueilli durant mes nombreux aller-retours à Paris : Ingrid, Guillaume, Karolina, Hugues, Marylise, Guillaume, Hilla, Gaëlle, Ignazio, Fabrizia, Jérome, Agathe, Agnès, Charlotte, Fred.

\indent Merci à Bertrand Gibert, pour le volume de Leroi-Gourhan qu'il m'a offert au début de ce travail, je l'ai bien usé!

\indent Merci à Agnès Gallet, Juliette Danjon et François Goudard pour leur relectures et retours d'yeux extérieurs.

\indent Merci à toutes celles et ceux que j'ai oublié dans cette liste incomplète et qui ont participé à ce travail.

\indent A ma famille, pour leurs soutien et encouragements.\\
À Olga et Anatole, pour leur patience autant que pour leur impatience et leur curiosité stimulante.\\
A Gladys Brégeon, qui sait tout ce que je lui dois.
