% !TEX root = ../thesis-example.tex
%
\chapter*{Remerciements}
\label{sec:acknowledgement}
\vspace*{-10mm}
\pdfbookmark[0]{Remerciements}{Remerciements}


% Je ne pourrai malheureusement pas remercier ici toutes les personnes qui m'ont aidé à réaliser ce travail, dont les origines remontent trop loin pour qu'elles soient vraiment 


Je tiens à remercier chaleureusement toutes les personnes qui m'ont permis de réaliser ce travail. Elles sont malheureusement bien trop nombreuses pour être toutes citées ici, et je me restreindrai aussi à celles et ceux qui ont plus directement soutenu cette recherche.

Je remercie en premier lieu mes directeurs de thèse Jean-Dominique Polack, Pierre Couprie et tout particulièrement mon ``co-encadrant'' Hugues Genevois --~bien étrange titre pour quelqu'un qui se plait tant à faire sortir la pensée hors des cadres établis. Merci à tous les trois de m'avoir laissé errer dans des directions diverses, avec une méthodologie parfois plus proche de l'improvisation que de la partition, tout en restant bienveillant à mon égard.

% Merci Hugues, ta confiance tout autant que ton ouverture d'esprit ont été des soutiens 

Également, je remercie chaleureusement le Collegium Musicæ, notamment Cécile Davy-Rigaux, Benoit Fabre et plus particulièrement Agnès Puissilieux, dont j'ai eu le plaisir de partager le bureau durant mes présences au LAM. Au delà du soutien financier que m'a apporté le Collegium en m'accordant sa confiance pour ce travail, l'implication aux séminaires rassemblant des chercheu.rs.ses de ses différentes composantes ont été, chaque fois, des moments de rencontre très enrichissants dans cette \ul{poursuite} d'inter-disciplinarité. 

C'est bien souvent au détour d'une conversation, si anecdotique parût-elle, que des connexions s'opèrent que l'on cherchait vainement ailleurs, ou que des théories apparemment bien ancrées trouvent leur failles et laissent passer un peu de lumière. C'est un plaisir, en tant que chercheur, d'avoir pu travailler dans un tel contexte où l'inter-disciplinarité est promue et encouragée.

Cette inter-disciplinarité n'est pas qu'un terme à la mode, mais se construit à travers des rencontres que le Collegium Musicæ suscite.

% Catherine Pélachaud pour sa bienveillance lors du suivi de cette thèse.

Mes remerciements vont également à mes collègues de ONE, pour le plaisir à jouer avec eux autant que pour ce qu'ils ont pu apporter aux réflexions et à l'élaboration de certains outils comme ``John'' : Pierre et Hugues qui outre le fait d'encadrer ce travail de recherche ont été des partenaires de jeu, Laurence Bouckaert, Jean Haury, György Kurtág Jr. dont la rencontre il y a maintenant plus de quinze ans a été décisive dans le fait de poursuivre certaines intuitions, et Serge de Laubier auprès de qui les quelques années passées à Puce Muse m'ont permis

 d'apprécier son talent à transposer les lutheries numériques, les musiques expérimentales et les questions théoriques qu'elles posent, sur des terrains très concrèts, d'un spectacle de rue un soir de décembre sous la pluie, ou d'une groupe d'enfants dont l'impatience tourne à fréquence plus élevée que le CPU des ordinateurs high-tech.

Je remercie également chaleureusement ceux qui m'ont généreusement offert leur temps et leurs idées durant les entretiens : Nicolas Bernier, Nicolas Collins, Serge De Laubier, François Dumeaux, Adrien Mamou-Mani, Luca Turchet, Bruno Zamborlin, Patrick Saint-Denis et José-Miguel Fernandez.
Merci également à celles et ceux avec qui les discussions informelles ont nourri ce travail, même si ces discussions n'y figurent pas explicitement : Marcelo Wanderley, Caroline Traube, Bernard Sève, Pascale Criton, ainsi qu'aux membres du LAM, en particulier le groupe nouvelles lutheries: Gabriela Patiño-Lakatos, Christophe d'Alessandro, Boris Doval, Michèle Castellengo, Claudia Fritz, Jean-Loïc Le Carrou, Louise Condi, Xiao Xiao, Grégoire Locqueville.

les membres du groupe nouvelles-lutheries au LAM : 

Pour l'accès à toutes les ressources auxquelles il est parfois difficile d'accéder dans un monde académique scindés en spécialités : Dušan Barok, Alexandra Elbakyan.


% Je remercie également les chercheurs et chercheuses du LAM, mon laboratoire d'accueil,  qui m'a accueilli durant mes venues occasionelles, 
, Cécile Babiole

Je tiens à remercier chaleureusement les ami.e.s qui m'ont accueillis durant mes nombreux aller-retours à Paris : Ingrid, Guillaume, Karolina, Hugues, Guillaume, Hilla, Gaëlle, Ignazio, Fabrizia, Jérome, Agathe. 

Bertrand Gibert, pour le volume de Leroi-Gourhan qu'il m'a offert et que j'ai bien usé!

% Hugues Genevois, Pierre Couprie, Jean-Dominique Polack, Catherine Pélachaud, LAM

% Agnès Puissilieux et CM

% Interviewés (Nic Collins, Adrien Mamou-Mani, Bruno Zamborlin, François Dumeaux, Serge De Laubier, György Kurtág Jr., Luca Turchet, Nicolas Bernier, Patrick Saint Denis, Marcello Wanderley, Caroline Traube,  Bernard Sève, José-Miguel Fernandez, Rémy Muller, Thor Magnusson, ...)

% Tous ceux rencontrés et collaborations (sans citer explicitement).

% Les ami.e.s qui m'ont hébergés durant mes passages à Paris durant toutes ces dernières années, en particulier Guillaume avec qui les discussions ont nourri

% G, A, O.

% et pour l'accès à toutes les ressources auxquelles il est parfois difficile d'accéder dans un monde académique scindés en spécialités : Dušan Barok, Alexandra Elbakyan.

A mes parents, pour leur amour et soutien inconditionnel depuis le premier jour.
À Olga et Anatole, pour leur patience autant que pour leur impatience et leur curiosité stimulante.
A Gladys Brégeon.
