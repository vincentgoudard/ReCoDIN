MATERIAUX

Aspects structurels et ergonomie
	Bruts
		bois (contreplaqué, CNC), métal, plexi, verre, peaux, mousse, textile, laine, cuir)
		=> propriétés acoustiques, mécanique (résistance, élasticité), électriques, électromagnétiques, transparence de ces matériaux, état de surface (poreux, lisse), matériaux ``travaillables'' (e.g. collable, perçable, vissable)
		=> disponibilité et forme sous laquelle ils sont vendus (e.g. plaques de bois, textile au mètre, etc.)
	Les objets
		tambourins, instruments augmentés, les cordes
		la récup

	Le corps comme interface
		Biosignals

	Agencement
		Portabilité, temps de montage, 

Aspects acoustiques



Aspects électroniques


Interfaces d'acquisition
- ouverts et programmables
	- arduino, bela, teensy, etc.


Capteurs intelligents (IHM)
- fermés et détournables

