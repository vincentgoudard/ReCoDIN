MATERIAUX

\textit{Comme nous l'avons vu au chapitre précédent, les DMI se présentent souvent comme des agencements hétérogènes et éphémères.}

1. Introduction : différentes faces de l'interface
	1.1 Interface gestuelle ou interface sensible
	1.2 Un support pour musiquer
	1.3 Instruments acoustico-électrico-électronico-numériques
	
2. Héritage comme facteur de formes
	1.1 Héritage instrumental
	1.2 Héritage du corps
	1.3 Héritage technique
	1.4 Héritage de l'objet
	1.5 Héritage poétique

3. Matériaux
	polymatériau
	3.1 Matériaux bruts
	3.2 Objets détournés

4.Aspects acoustiques
	

3. Aspects électroniques
	Capteurs
	Interfaces d'acquisition
		- ouverts et programmables
		- arduino, bela, teensy, etc.
	Capteurs intelligents (IHM)
		- fermés et détournables


4. Ergonomie, Ergodynamisme
	Ergonomie
		Portabilité, temps de montage
	Ergodynamisme
		Se repérer au toucher
			Présence de l'espace virtuel dans l'objet réel
		Le corps comme interface
			Proprioception, Biosensors, Merleau Ponty

	Toucher avec les mains, 	
	Toucher avec un stylet, un médiator, des Objets
	Modèle intermédaire : exemple des plumes de Serge, pantographe dans Frelia d'Ali Momeni


%%%%%%%%%%%%%%%%%%%%%%%

	Bruts
		bois (contreplaqué, CNC), métal, plexi, verre, peaux, mousse, textile, laine, cuir)
		=> propriétés acoustiques, mécanique (résistance, élasticité), électriques, électromagnétiques, transparence de ces matériaux, état de surface (poreux, lisse), matériaux ``travaillables'' (e.g. collable, perçable, vissable)
		=> disponibilité et forme sous laquelle ils sont vendus (e.g. plaques de bois, textile au mètre, etc.)
	Les objets
		tambourins, instruments augmentés, les cordes
		la récup

	Le corps comme interface
		Biosignals

	Agencement
		Portabilité, temps de montage, 