% !TEX root = ../thesis-example.tex
%
\chapter{Interfaces}
\label{ch:interfaces}

\cleanchapterquote{Users do not care about what is inside the box, as long as the box does what they need done.}{Jef Raskin}{about Human Computer Interfaces}

\section{Généalogie d’une interface de DMI}
\label{sec:interfaces:sec1}

La conception d’une nouvelle interface pour la performance musicale est une tâche complexe, nécessitant de nombreux aller-retours entre conception, fabrication et pratique musicale. Le filigramophone est une interface qui a connu plusieurs versions, suffisamment différentes pour avoir envie de leur donner un nouveau nom à chaque fois et suffisamment similaire pour y voir la continuité d’un seul et même instrument.

 Origine, la tablette graphique
La tablette graphique (nommément un modèle Sapphire de Wacom) a été l’interface originelle qui a servi de base au filigramophone. J’ai commencé à l’utiliser suite à son utilisation dans le cadre de la Méta-Mallette\footnote{Logiciel pour la pratique collective de musique par ordinateur développé par l’association Puce Muse.}. La raison de ce choix est que la tablette graphique offre une interface relativement bon marché (donc déployable en nombre) qui permet un contrôle assez fin de la synthèse sonore.
Un certain nombre de musiciens, compositeurs et concepteurs de NIME l’ont adopté pour leur projet \cite{Zby07}, et Nicolas d’Alessandro a consacré une partie de son travail de thèse \cite{Ale09} à ce sujet.

La tablette graphique permet de bénéficier de l’expertise du geste d’écriture et de dessin.

\begin{quote}
"Dufourt suggests that contemporary music highlights what was rejected in the Greek world : it rather captures the evanescent, the ephemeral, the ambivalent, the Erebus, it favors the endless metamorphosis of qualities and forms; as Nietzsche proclaimed, western music tends toward the liberation of the dyonisiac dimension and the acceptance of the inacceptable part of myths. » Risset, Sound and Music Computing Meets Philosophy, ICMC proc. 2014
\end{quote}

\section{Conclusion}
\label{sec:interfaces:conclusion}
