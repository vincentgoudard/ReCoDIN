1.Introduction
	1.1 L'étude du geste en musique
	1.2 Geste instrumental ou geste musical
	1.3 Contexte du geste dans les DMIs

2. Aspects fonctionnels du modèle instrumental acoustique
	2.1 Cadoz, Wanderley
	2.2 Apports
	2.2 Application aux DMIs

3. Les limites d'une analyse en termes fonctionnels
	3.1 Geste produit, capté, perçu
		Astrid Bin
	3.2 Les musiciens ne sont pas utilisateurs d'instruments
	3.3 La scène et le laboratoire
	3.4 Gestes pré-sémiotiques

4. Geste programmé, re-sonné
	4.1 L'outil comme externalisation chez L-G
	4.2 Le geste programmé (Stiegler et la grammatisation, Deleuze et le diagramme)
	4.3 Le geste de re-sonance
		reenactment

5. Playing with instruments that run
	5.1 Démarrer et arrêter
	5.2 Métaphore du tourneur d'assiette
	5.3 Résonance entre le geste re-sonnant et le son

6. Subversion sonore, subversion gestuelle
	6.1 Sons paradoxaux, gestes paradoxaux
		Risset, Kurtag Jr., Kurtag Père
	6.2 Le geste et le son comme métaphores
		5.2.1 Bayle
		5.2.2 UST
	6.3 Continuité artificielles (morphodynamisme des DMI)
		La musique comme jeu de la (dis)continuité


7. Quelles conséquences pour le design des DMI ?



métaphore gestuelle
- geste de tourneur d'assiette

- surfing (on sinewaves)
=> feedback

target selection



Instrument déterminé par :
- la qualité du geste (typologie, morphologie, etc)
- la qualité du son (typologie, morphologie, etc)
- l'intention de la relation
	- gestes de résonance
	- geste paradoxaux




\iquote{La configuration sensible d'un objet ou d'un geste, que la critique de l'hypothèse de constance fait paraître sous notre regard, ne se saisit pas dans une coïncidence ineffable, elle se ``comprend'' par une sorte d'appropriation dont nous avons tous l'expérience quand nous disons que nous avons ``trouvé'' le lapin dans le feuillage d'une devinette, ou que nous avons ``attrapé'' un mouvement.} Merleau-Ponty, Phénoménologie de la perception.


Comme il le rappelait durant la journée d'hommage à Risset, Cadoz : ``reconstruire par le calcul non seulement le phénomène sonore mais ce qui l'engendre, ce qui le produit '' (...) il faut retourner aux causes (3:26)


Le fait que l'intervention humaine ne soit plus nécessaire pour que la musique soit produite pose précisément la question de ce que l'humain vient faire lors d'une performance musicale. Davantage encore que celle de ses gesticulations, c'est la question de sa présence qui importe le plus.

\iquote{Réduisant l'instrument de musique à sa seule expression de contrôle gestuel, les interfaces mettent au jour la question du geste, et cette question, comme enfouie, refoulée, apparaît alors comme essentielle et transversale à toute l'histoire de la musique.} Bricout BRICOUT, Romain, « Les interfaces musicales : la question des “instruments aphones” », Methodos, no 11, « L’instrument de musique » , op.. cit.. En ligne : <http://methodos.revues.org/2493> (vu le 01/10/2017). 



%%%%%%%%%%%%%%%%%%%%%%%%%%%%%%%%%%%%%%%%%%%%%%%%%%%%%%%%%%%%%%%%%%%%
\section{Le geste dans le temps}

Pour ce qui est du geste en général, le Littré, le Larousse ou le dictionnaire de l'Académie Française s'accordent à le définir comme ``un mouvement du corps, principalement de la main, des bras, de la tête, porteur de sens ou non'' (Larousse).

En tant que phénomène qui s'inscrit dans la durée, il peut être intéressant de considérer les manières dont la définition du geste varie en fonction de son rapport au temps.

\subsection{Le geste et son histoire : tradition gestuelle}
cf. François Dumeaux interview

\subsection{Le geste a priori : l'intention}


Dans le cadre d'une activité spécifique telle que la pratique musicale, impliquant la participation active de l'individu, la plupart des études (todo:mettre plusieurs ref) sur le geste s'accordent à le définir comme l'association d'un mouvement et d'une intention.

Le geste dans les pratiques instrumentales numériques a d'abord commencé par être un geste technique et hors-temps de la performance musicale, celui de la programmation sur ordinateur de son que les machines de l'époques mettait plusieurs heures à calculer. Le geste était donc ici très intentionnel, la programmation informatique laissant peu de place pour crééer sans passer par une étape mentale de fabrication du son.


\subsection{Le geste a posteriori : la signification}


\subsection{Le geste dans l'instant : flow et agentivité}

\subsection{Le geste hors du temps : programme}



Le développement et l'analyse des premiers DMI nourris par la culture des IHM et le modèle de l'instrument acoustique.